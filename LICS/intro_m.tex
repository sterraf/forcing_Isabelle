%%%%%%%%%%%%%%%%%%%%%%%%%%%%%%%%%%%%%%%%%%%%%%%%%%%%%%%%%%%%%%%%%%%%%%          
\section{Introduction}
% no \IEEEPARstart
Zermelo-Fraenkel Set Theory ($\ZF$) has a prominent place among formal
theories; in particular, it provides a foundation for mathematics and
most of the toolkit used everyday by the computer scientist has also
Set Theory at its base (although much of it can be based in other
formalisms, such as Category Theory) \fbox{citation needed,
  Aczel?}. In this time of mechanization of
mathematics~\cite{avigad2018mechanization}, it seems natural to ask
for a mechanization of the most salient results of Set Theory.
\fbox{Y entra de lleno en la intersección de lógica  y cs}

After G\"odel's Incompleteness Theorems, we cannot expect to have a
formal proof of the consistency of Set Theory in $ZF$. Besides its own
consistency, there are other results which are undecided by $ZF$: the
undecidability of Continuum Hypothesis lead to the development of
techniques for independence proofs. First G\"odel introduced the
theory of \emph{inner models}, which gives rise to his model $L$ of
the \emph{Axiom of Constructibility} \cite{godel-L} and proved the
relative consistency of the Axiom of Choice and the Generalized
Continuum Hypothesis with $\ZF$. Thirty years later Paul
J. Cohen~\cite{Cohen-CH-PNAS} devised the technique of \emph{forcing},
which is the only known way of \emph{extending} models of $\ZF$; in
particular, it can be used to prove the relative consistency of the
negation of the Continuum Hypothesis. 

In this work we address a substantial part of formalizing the proof
that given a model $M$ of $\ZF$, any \emph{generic extension} $M[G]$
given by forcing also satisfies $\ZF$. As remarked by
\citet[][p.250]{kunen2011set} \enquote{[...] in verifying that $M[G]$
  is a model for set theory, the hardest axiom to verify is
  Comprehension.}  The most important achievement of this paper is the
mechanization in Isabelle of this result; en route to this, we also
formalized the satisfaction by $M[G]$ of Extensionality, Foundation,
Union, Infinity.

Our development benefited from the extraordinary work done by Lawrence
Paulson \cite{paulson_2003} on the formalization of G\"odel's
constructible universe in the proof assistant \emph{Isabelle}. In the
spirit of the \emph{formal abstract} project~\cite{hales-fabstracts},
we explicit the technical assumptions of forcing that are involved in
the proof of Separation for $M[G]$. It seems that this work is
\emph{Es un lindo test a la capacidad de implementación formal de
  matemática con la tecnología actual}.

\fbox{Acá viene una oración introduciendo lo que sigue.}

In a gross simplification, there are two aspects to a formalization
project like this one: thematic and programmatic. The first concerns
the handling of all the theoretical concepts and results in the
subject, while the second involves the practical issues of the
implementation and design. In the case of forcing, the main intricacy
lies in the first aspect. In this sense, following a sensible
presentation of the material is key.  The authoritative reference 
on the subject during the last 30 years has been Kunen's classical
\cite{kunen1980}. In our
formalizaton we have followed a recent rewrite \cite{kunen2011set}
of that  textbook, which presents the material in the same sharp 
style but offering a lot of details. In some sense this project
wouldn't exist without this book. As alternative, introductory
resources, the  interested reader can check
\cite{chow-beginner-forcing}; also, the book \cite{weaver2014forcing}
contains a thorough treament minimizing the technicalities.

\fbox{El párrafo que sigue lo sacaría}

In this work we address a substantial part the problem of formalizing
the proof that given a model of $\ZF$, its extensions by forcing, 
\emph{generic extensions},
satisfy the same axioms. In particular, the first axiom that
genuinely calls for a use of the forcing machinery is that of
\emph{Separation} (\emph{Aussonderungsaxiom}), otherwise known as
``Axiom of (Restricted) Comprehension''  or ``Specification.'' In
\cite{2018arXiv180705174G} it was shown that  generic extensions of
models of $\ZF$ 
satisfy the Axiom of Pairing, and in this work we show that also
Extensionality, Foundation, Union, Infinity (under extra assumptions),
and Separation go through. The
Axiom of Powerset is not treated in the present paper but it poses no
extra difficulties; we discuss the Axiom of Replacement in the
conclusions. 

The proof that these axioms hold in generic extension is independent
of the \emph{proofs} of the ``fundamental'' theorems of forcing. In
particular, one of the most complex parts of those proofs is the
definition of the \emph{forcing relation}. It turns out that the
particular (``implementation'') details of this  definition  does not
appear in the statement of the fundamental theorems, and our
formalization works inside a \emph{locale} (set of assumptions) that
includes the statement of these theorems. The formalization of the
fundamental theorems of forcing roughly comprises little less than a
half of our full project. 

We briefly describe the contents of each
section. Section~\ref{sec:isabelle} contains the bare minimun
requirements to understand the (meta)logics used in Isabelle. Next, an
overview of the model theory of set theory is presented in
Section~\ref{sec:axioms-models-set-theory}. There is an ``internal''
representation of first-order formulas as sets, implemented by
Paulson; Section~\ref{sec:renaming} discusses syntactical
transformations of the former, mainly permutation of variables. 
In Section~\ref{sec:generic-extensions} the generic extensions are
succintly reviewed and how the treatment of well founded recursion in
Isabelle was enhanced. We take care of the ``easy axioms'' in
Section~\ref{sec:easy-axioms}; these are the ones that
do not depend on the forcing theorems. We describe the latter in
Section~\ref{sec:forcing}. We adapted the  work by Paulson to our
needs, and this is described in
Section~\ref{sec:hack-constructible}. We present the proof
of the Separation Axiom Scheme in Section~\ref{sec:proof-separation},
which follows closely its implementation. A plan for future work and
some immediate conclusions are offered in
Section~\ref{sec:conclusions-future-work}.

%%% Local Variables: 
%%% mode: latex
%%% TeX-master: "Separation_In_MG"
%%% ispell-local-dictionary: "american"
%%% End: 
