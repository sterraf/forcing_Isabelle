\section{Main achievements of the formalization}
\label{sec:main-achievements}

\subsection{32 replacement instances to rule them all}
\label{sec:repl-instances}

We isolated 32 instances of Replacement that are sufficient to force
$\CH$ or $\neg\CH$, which are enumerated below by the name of the
corresponding internalized first order formula. Many of these were already present in
relational form in the \session{ZF-Constructible} library.

The first 13 instances are needed to set up
cardinal arithmetic in $M$, along some other basic constructions.

\begin{itemize}
\item 6 instances for iteration of operations (between parentheses). For each construct,
  Paulson used one replacement instance in order to have absoluteness,
  and a second one to obtain closure.
  \begin{itemize}
  \item
    \isatt{list{\uscore}repl1{\uscore}intf{\uscore}fm} and 
    \isatt{list{\uscore}repl2{\uscore}intf{\uscore}fm}.
    
    For lists on a set $A$ ($X\mapsto \{0 \} \oplus A \times X$, where
    $D\oplus E$ stands for the disjoint union/coproduct $(\{0 \}\times
    D) \cup (\{1 \}\times E)$).
    %
  \item
    \isatt{formula{\uscore}repl1{\uscore}intf{\uscore}fm} and
    \isatt{formula{\uscore}repl1{\uscore}intf{\uscore}fm}.
    
    For (codes of) first order formulas ($X\mapsto (\omega \times
    \omega \oplus \omega \times \omega) \oplus X \times X \oplus X$).
    %
  \item
    \isatt{eclose{\uscore}repl1{\uscore}intf{\uscore}fm} and 
    \isatt{eclose{\uscore}repl2{\uscore}intf{\uscore}fm}.

    For transitive closure ($X\mapsto\union X$).
  \end{itemize}
\item \isatt{tl{\uscore}repl{\uscore}intf{\uscore}fm}.

  Absoluteness of the definition of the
  $n$th element of a list (iteration of the tail operation).
  By transitivity, we do not need an instance for closure.
\end{itemize}
%% The instances so far
%% are needed to interpret locales \locale{M{\uscore}datatypes} and
%% \locale{M{\uscore}eclose}.
%% \locale{M{\uscore}datatypes} is used in the relative definition of
%% \isatt{Vset}.
These formulas, together with the next two, form the set
\isa{instances{\isadigit{1}}{\isacharunderscore}{\kern0pt}fms}.
\begin{itemize}
\item
  4 instances for definitions by well-founded recursion.
  \begin{itemize}
  \item \isatt{wfrec{\uscore}rank{\uscore}fm}.

    For $\in$-rank.
    %
  \item \isatt{trans{\uscore}repl{\uscore}HVFrom{\uscore}fm}.

    For the cumulative hierarchy (rank initial segments).
    %
  \item \isatt{wfrec{\uscore}replacement{\uscore}order{\uscore}pred{\uscore}fm}.

    For ordertypes
    %
  \item
    \isatt{replacement{\uscore}HAleph{\uscore}wfrec{\uscore}repl{\uscore}body{\uscore}fm}.

    For Aleph.
  \end{itemize}
\item
  2 instances for ordertypes.
  \begin{itemize}
  \item
    \isatt{replacement{\uscore}is{\uscore}order{\uscore}eq{\uscore}map{\uscore}fm}.

    Auxiliary instance for the definition of ordertypes, needed for
    the interpretation of \locale{M{\uscore}ordertype}.
    %
  \item
    \isatt{replacement{\uscore}is{\uscore}order{\uscore}body{\uscore}fm}.

    Replacement through $x\mapsto \otype(x)$, for Hartogs' Theorem.
  \end{itemize}
\end{itemize}

We also need a one extra replacement instance $\psi$ on $M$ for each
$\phi$ of the
previous ones to have them in $M[G]$.
\[
  \psi(x,\alpha,y_1,\dots,y_n) \defi \quine{\alpha = \min \bigl\{
    \beta \mid \exists\tau\in V_\beta.\  \mathit{snd}(x) \forces
    \phi\ [\mathit{fst}(x),\tau,y_1,\dots,y_n]\bigr\}}
\]
Here, $\mathit{fst}(\lb a,b\rb) = a$, $\mathit{snd}(\lb a,b\rb) = b$
(with default value $0$ for non pairs).
In our development, the mapping $\phi\mapsto\psi$ defined above is given by the
$\isatt{ground{\uscore}repl{\uscore}fm}$ function, and all ground replacement
instances appear in the locale \locale{M{\uscore}ZF4} and form the set
\isa{instances{\isadigit{4}}{\isacharunderscore}{\kern0pt}fms}. These are expressed using
the \isatt{ground{\uscore}replacement{\uscore}assm(M,\isasymphi)} predicate
obtained by replacing $\phi$ by
$\isatt{ground{\uscore}repl{\uscore}fm}(\phi)$ in Eq.~(\ref{eq:replacement_assm_def}).

That makes 26 instances up to now. For the setup of forcing, we
require the following 3 instances, which form the set \isa{instances{\isacharunderscore}{\kern0pt}ground{\isacharunderscore}{\kern0pt}fms}.
\begin{itemize}
\item 2 instances for definitions by well-founded recursion.
  \begin{itemize}
  \item \isatt{wfrec{\uscore}Hcheck{\uscore}fm}.

    For check-names.
    %
  \item \isatt{wfrec{\uscore}Hfrc{\uscore}at{\uscore}fm}.

    Definition of forcing for atomic formulas.
  \end{itemize}
  %
\item
  \isatt{Lambda{\uscore}in{\uscore}M{\uscore}fm(check{\uscore}fm(2,0,1),1)}.

  Replacement through $x\mapsto \lb x,\check{x}\rb$ (for the
  definition of $\punto{G}$).
  %
\end{itemize}
The next two formulas, which form the set
\isa{instances{\isacharunderscore}{\kern0pt}ground{\isacharunderscore}{\kern0pt}notCH{\isacharunderscore}{\kern0pt}fms},
are needed for the $\Delta$-System Lemma.
\begin{itemize}
\item
  \isatt{replacement{\uscore}is{\uscore}trans{\uscore}apply{\uscore}image{\uscore}fm}.

  Recursive construction of sets using a choice function (as in the
  construction of a wellorder of $X$ given a choice function on $\Pow(X)$).
  %
\item
  \isatt{replacement{\uscore}transrec{\uscore}apply{\uscore}image{\uscore}body{\uscore}fm}.

  Absoluteness of the previous construction.
\end{itemize}
%
The $31$ formulas up to this point are collected into the set
\isa{overhead{\isacharunderscore}{\kern0pt}notCH}, which is enough to
force $\neg\CH$. To force $\CH$, we required one further instance:
%
\begin{itemize}
\item
  \isatt{replacement{\uscore}dcwit{\uscore}repl{\uscore}body{\uscore}fm}.

  Absoluteness of the recursive construction in the proof of the
  Dependent Choices from $\AC$.
\end{itemize}

The particular choice of some of the instances above arose from
Paulson's architecture on which we based our development.
This applies every time
a locale from \session{ZF-Constructible} has to be
interpreted (namely \locale{M{\uscore}datatypes}
\locale{M{\uscore}eclose}, and \locale{M{\uscore}ordertype}).
%% For instance, the first
%% instance required for the definition of relative ordertypes arises
%% from Paulson's \session{ZF-Constructible}.
% https://isabelle.in.tum.de/dist/library/ZF/ZF-Constructible/Rank.html#offset_1123..1139

On the other hand, as explained in
Section~\ref{sec:zfc-axioms-as-locales}, we managed to eliminate the
instance arising from the \locale{M{\uscore}basic} locale. Similarly,
we replaced the original proof of the Schröder-Bernstein Theorem by
Zermelo's one \cite[Exr. x4.27]{moschovakis1994notes}, because the
former required at least one extra instance
% (\isatt{banach{\uscore}iterates{\uscore}fm})
arising from an iteration.

It is to be noted that application of the Forcing Theorems do not
require any extra replacement instances on $M$ (independently of the
formula $\phi$ for which they are invoked). This is not the case for
Separation, at least from our formalization: More instances are needed
as the complexity of $\phi$ grows. One point where this is apparent is
in the proof of Theorem~\ref{th:forcing-thms}(\ref{item:truth-lemma}),
that appears as the \isatt{truth{\uscore}lemma} in our development; it
depends on \isatt{truth{\uscore}lemma'} and
\isatt{truth{\uscore}lemma{\uscore}Neg} which explicitly invoke
\isatt{separation{\uscore}ax}.

%-%-%-%-%-%-%-%-%-%-%-%-%-%-%-%-%-%-%-%-%-%-%-%-%-%-%-%-%-%-%-%-%
\subsection{Models for $\CH$ and its negation}
\label{sec:models-ch-negation}

The statements of the existence of models of $\ZFC + \neg\CH$ and of
$\ZFC + \CH$ negation appear in our formalization as follows:

\begin{isabelle}
\isacommand{corollary}\isamarkupfalse%
\ ctm{\isacharunderscore}{\kern0pt}ZFC{\isacharunderscore}{\kern0pt}imp{\isacharunderscore}{\kern0pt}ctm{\isacharunderscore}{\kern0pt}not{\isacharunderscore}{\kern0pt}CH{\isacharcolon}{\kern0pt}\isanewline
\ \ \isakeyword{assumes}\isanewline
\ \ \ \ {\isachardoublequoteopen}M\ {\isasymapprox}\ {\isasymomega}{\isachardoublequoteclose}\ {\isachardoublequoteopen}Transset{\isacharparenleft}{\kern0pt}M{\isacharparenright}{\kern0pt}{\isachardoublequoteclose}\ {\isachardoublequoteopen}M\ {\isasymTurnstile}\ ZFC{\isachardoublequoteclose}\isanewline
\ \ \isakeyword{shows}\isanewline
\ \ \ \ {\isachardoublequoteopen}{\isasymexists}N{\isachardot}{\kern0pt}\isanewline
\ \ \ \ \ \ M\ {\isasymsubseteq}\ N\ {\isasymand}\ N\ {\isasymapprox}\ {\isasymomega}\ {\isasymand}\ Transset{\isacharparenleft}{\kern0pt}N{\isacharparenright}{\kern0pt}\ {\isasymand}\ N\ {\isasymTurnstile}\ ZFC\ {\isasymunion}\ {\isacharbraceleft}{\kern0pt}{\isasymcdot}{\isasymnot}{\isasymcdot}CH{\isasymcdot}{\isasymcdot}{\isacharbraceright}{\kern0pt}\ {\isasymand}\isanewline
\ \ \ \ \ \ {\isacharparenleft}{\kern0pt}{\isasymforall}{\isasymalpha}{\isachardot}{\kern0pt}\ Ord{\isacharparenleft}{\kern0pt}{\isasymalpha}{\isacharparenright}{\kern0pt}\ {\isasymlongrightarrow}\ {\isacharparenleft}{\kern0pt}{\isasymalpha}\ {\isasymin}\ M\ {\isasymlongleftrightarrow}\ {\isasymalpha}\ {\isasymin}\ N{\isacharparenright}{\kern0pt}{\isacharparenright}{\kern0pt}{\isachardoublequoteclose}
\end{isabelle}

\begin{isabelle}
\isacommand{corollary}\isamarkupfalse%
\ ctm{\isacharunderscore}{\kern0pt}ZFC{\isacharunderscore}{\kern0pt}imp{\isacharunderscore}{\kern0pt}ctm{\isacharunderscore}{\kern0pt}CH{\isacharcolon}{\kern0pt}\isanewline
\ \ \isakeyword{assumes}\isanewline
\ \ \ \ {\isachardoublequoteopen}M\ {\isasymapprox}\ {\isasymomega}{\isachardoublequoteclose}\ {\isachardoublequoteopen}Transset{\isacharparenleft}{\kern0pt}M{\isacharparenright}{\kern0pt}{\isachardoublequoteclose}\ {\isachardoublequoteopen}M\ {\isasymTurnstile}\ ZFC{\isachardoublequoteclose}\isanewline
\ \ \isakeyword{shows}\isanewline
\ \ \ \ {\isachardoublequoteopen}{\isasymexists}N{\isachardot}{\kern0pt}\isanewline
\ \ \ \ \ \ M\ {\isasymsubseteq}\ N\ {\isasymand}\ N\ {\isasymapprox}\ {\isasymomega}\ {\isasymand}\ Transset{\isacharparenleft}{\kern0pt}N{\isacharparenright}{\kern0pt}\ {\isasymand}\ N\ {\isasymTurnstile}\ ZFC\ {\isasymunion}\ {\isacharbraceleft}{\kern0pt}{\isasymcdot}CH{\isasymcdot}{\isacharbraceright}{\kern0pt}\ {\isasymand}\isanewline
\ \ \ \ \ \ {\isacharparenleft}{\kern0pt}{\isasymforall}{\isasymalpha}{\isachardot}{\kern0pt}\ Ord{\isacharparenleft}{\kern0pt}{\isasymalpha}{\isacharparenright}{\kern0pt}\ {\isasymlongrightarrow}\ {\isacharparenleft}{\kern0pt}{\isasymalpha}\ {\isasymin}\ M\ {\isasymlongleftrightarrow}\ {\isasymalpha}\ {\isasymin}\ N{\isacharparenright}{\kern0pt}{\isacharparenright}{\kern0pt}{\isachardoublequoteclose}
\end{isabelle}

As the code indicates, these results are obtained as corollaries from
three theorems in which only a subset of the aforementioned
replacement instances are assumed of the ground model.


\begin{isabelle}
\isacommand{theorem}\isamarkupfalse%
\ extensions{\isacharunderscore}{\kern0pt}of{\isacharunderscore}{\kern0pt}ctms{\isacharcolon}{\kern0pt}\isanewline
\ \ \isakeyword{assumes}\isanewline
\ \ \ \ {\isachardoublequoteopen}M\ {\isasymapprox}\ {\isasymomega}{\isachardoublequoteclose}\ {\isachardoublequoteopen}Transset{\isacharparenleft}{\kern0pt}M{\isacharparenright}{\kern0pt}{\isachardoublequoteclose}\isanewline
\ \ \ \ {\isachardoublequoteopen}M\ {\isasymTurnstile}\ {\isasymcdot}Z{\isasymcdot}\ {\isasymunion}\ {\isacharbraceleft}{\kern0pt}{\isasymcdot}Replacement{\isacharparenleft}{\kern0pt}p{\isacharparenright}{\kern0pt}{\isasymcdot}\ {\isachardot}{\kern0pt}\ p\ {\isasymin}\ overhead{\isacharbraceright}{\kern0pt}{\isachardoublequoteclose}\isanewline
\ \ \ \ {\isachardoublequoteopen}{\isasymPhi}\ {\isasymsubseteq}\ formula{\isachardoublequoteclose}\ {\isachardoublequoteopen}M\ {\isasymTurnstile}\ {\isacharbraceleft}{\kern0pt}\ {\isasymcdot}Replacement{\isacharparenleft}{\kern0pt}ground{\isacharunderscore}{\kern0pt}repl{\isacharunderscore}{\kern0pt}fm{\isacharparenleft}{\kern0pt}{\isasymphi}{\isacharparenright}{\kern0pt}{\isacharparenright}{\kern0pt}{\isasymcdot}\ {\isachardot}{\kern0pt}\ {\isasymphi}\ {\isasymin}\ {\isasymPhi}{\isacharbraceright}{\kern0pt}{\isachardoublequoteclose}\isanewline
\ \ \isakeyword{shows}\isanewline
\ \ \ \ {\isachardoublequoteopen}{\isasymexists}N{\isachardot}{\kern0pt}\isanewline
\ \ \ \ \ \ M\ {\isasymsubseteq}\ N\ {\isasymand}\ N\ {\isasymapprox}\ {\isasymomega}\ {\isasymand}\ Transset{\isacharparenleft}{\kern0pt}N{\isacharparenright}{\kern0pt}\ {\isasymand}\ M{\isasymnoteq}N\ {\isasymand}\isanewline
\ \ \ \ \ \ {\isacharparenleft}{\kern0pt}{\isasymforall}{\isasymalpha}{\isachardot}{\kern0pt}\ Ord{\isacharparenleft}{\kern0pt}{\isasymalpha}{\isacharparenright}{\kern0pt}\ {\isasymlongrightarrow}\ {\isacharparenleft}{\kern0pt}{\isasymalpha}\ {\isasymin}\ M\ {\isasymlongleftrightarrow}\ {\isasymalpha}\ {\isasymin}\ N{\isacharparenright}{\kern0pt}{\isacharparenright}{\kern0pt}\ {\isasymand}\isanewline
\ \ \ \ \ \ {\isacharparenleft}{\kern0pt}{\isacharparenleft}{\kern0pt}M{\isacharcomma}{\kern0pt}\ {\isacharbrackleft}{\kern0pt}{\isacharbrackright}{\kern0pt}{\isasymTurnstile}\ {\isasymcdot}AC{\isasymcdot}{\isacharparenright}{\kern0pt}\ {\isasymlongrightarrow}\ N{\isacharcomma}{\kern0pt}\ {\isacharbrackleft}{\kern0pt}{\isacharbrackright}{\kern0pt}\ {\isasymTurnstile}\ {\isasymcdot}AC{\isasymcdot}{\isacharparenright}{\kern0pt}\ {\isasymand}\ N\ {\isasymTurnstile}\ {\isasymcdot}Z{\isasymcdot}\ {\isasymunion}\ {\isacharbraceleft}{\kern0pt}\ {\isasymcdot}Replacement{\isacharparenleft}{\kern0pt}{\isasymphi}{\isacharparenright}{\kern0pt}{\isasymcdot}\ {\isachardot}{\kern0pt}\ {\isasymphi}\ {\isasymin}\ {\isasymPhi}{\isacharbraceright}{\kern0pt}{\isachardoublequoteclose}
\end{isabelle}

Here, the 14-element \isatt{overhead} is enough to construct a proper
extension. It is  the union of
\isa{instances{\isadigit{1}}{\isacharunderscore}{\kern0pt}fms},
\isa{instances{\isacharunderscore}{\kern0pt}ground{\isacharunderscore}{\kern0pt}fms},
and the two instances
\isatt{replacement{\uscore}HAleph{\uscore}wfrec{\uscore}repl{\uscore}body{\uscore}fm}
and
\isatt{replacement{\uscore}is{\uscore}order{\uscore}eq{\uscore}map{\uscore}fm}.

\begin{isabelle}
\isacommand{theorem}\isamarkupfalse%
\ ctm{\isacharunderscore}{\kern0pt}of{\isacharunderscore}{\kern0pt}not{\isacharunderscore}{\kern0pt}CH{\isacharcolon}{\kern0pt}\isanewline
\ \ \isakeyword{assumes}\isanewline
\ \ \ \ {\isachardoublequoteopen}M\ {\isasymapprox}\ {\isasymomega}{\isachardoublequoteclose}\ {\isachardoublequoteopen}Transset{\isacharparenleft}{\kern0pt}M{\isacharparenright}{\kern0pt}{\isachardoublequoteclose}\ {\isachardoublequoteopen}M\ {\isasymTurnstile}\ ZC\ {\isasymunion}\ {\isacharbraceleft}{\kern0pt}{\isasymcdot}Replacement{\isacharparenleft}{\kern0pt}p{\isacharparenright}{\kern0pt}{\isasymcdot}\ {\isachardot}{\kern0pt}\ p\ {\isasymin}\ overhead{\isacharunderscore}{\kern0pt}notCH{\isacharbraceright}{\kern0pt}{\isachardoublequoteclose}\isanewline
\ \ \ \ {\isachardoublequoteopen}{\isasymPhi}\ {\isasymsubseteq}\ formula{\isachardoublequoteclose}\ {\isachardoublequoteopen}M\ {\isasymTurnstile}\ {\isacharbraceleft}{\kern0pt}\ {\isasymcdot}Replacement{\isacharparenleft}{\kern0pt}ground{\isacharunderscore}{\kern0pt}repl{\isacharunderscore}{\kern0pt}fm{\isacharparenleft}{\kern0pt}{\isasymphi}{\isacharparenright}{\kern0pt}{\isacharparenright}{\kern0pt}{\isasymcdot}\ {\isachardot}{\kern0pt}\ {\isasymphi}\ {\isasymin}\ {\isasymPhi}{\isacharbraceright}{\kern0pt}{\isachardoublequoteclose}\isanewline
\ \ \isakeyword{shows}\isanewline
\ \ \ \ {\isachardoublequoteopen}{\isasymexists}N{\isachardot}{\kern0pt}\isanewline
\ \ \ \ \ \ M\ {\isasymsubseteq}\ N\ {\isasymand}\ N\ {\isasymapprox}\ {\isasymomega}\ {\isasymand}\ Transset{\isacharparenleft}{\kern0pt}N{\isacharparenright}{\kern0pt}\ {\isasymand}\ N\ {\isasymTurnstile}\ ZC\ {\isasymunion}\ {\isacharbraceleft}{\kern0pt}{\isasymcdot}{\isasymnot}{\isasymcdot}CH{\isasymcdot}{\isasymcdot}{\isacharbraceright}{\kern0pt}\ {\isasymunion}\ {\isacharbraceleft}{\kern0pt}\ {\isasymcdot}Replacement{\isacharparenleft}{\kern0pt}{\isasymphi}{\isacharparenright}{\kern0pt}{\isasymcdot}\ {\isachardot}{\kern0pt}\ {\isasymphi}\ {\isasymin}\ {\isasymPhi}{\isacharbraceright}{\kern0pt}\ {\isasymand}\isanewline
\ \ \ \ \ \ {\isacharparenleft}{\kern0pt}{\isasymforall}{\isasymalpha}{\isachardot}{\kern0pt}\ Ord{\isacharparenleft}{\kern0pt}{\isasymalpha}{\isacharparenright}{\kern0pt}\ {\isasymlongrightarrow}\ {\isacharparenleft}{\kern0pt}{\isasymalpha}\ {\isasymin}\ M\ {\isasymlongleftrightarrow}\ {\isasymalpha}\ {\isasymin}\ N{\isacharparenright}{\kern0pt}{\isacharparenright}{\kern0pt}{\isachardoublequoteclose}
\end{isabelle}

\begin{isabelle}
\isacommand{theorem}\isamarkupfalse%
\ ctm{\isacharunderscore}{\kern0pt}of{\isacharunderscore}{\kern0pt}CH{\isacharcolon}{\kern0pt}\isanewline
\ \ \isakeyword{assumes}\isanewline
\ \ \ \ {\isachardoublequoteopen}M\ {\isasymapprox}\ {\isasymomega}{\isachardoublequoteclose}\ {\isachardoublequoteopen}Transset{\isacharparenleft}{\kern0pt}M{\isacharparenright}{\kern0pt}{\isachardoublequoteclose}\isanewline
\ \ \ \ {\isachardoublequoteopen}M\ {\isasymTurnstile}\ ZC\ {\isasymunion}\ {\isacharbraceleft}{\kern0pt}{\isasymcdot}Replacement{\isacharparenleft}{\kern0pt}p{\isacharparenright}{\kern0pt}{\isasymcdot}\ {\isachardot}{\kern0pt}\ p\ {\isasymin}\ overhead{\isacharunderscore}{\kern0pt}CH{\isacharbraceright}{\kern0pt}{\isachardoublequoteclose}\isanewline
\ \ \ \ {\isachardoublequoteopen}{\isasymPhi}\ {\isasymsubseteq}\ formula{\isachardoublequoteclose}\ {\isachardoublequoteopen}M\ {\isasymTurnstile}\ {\isacharbraceleft}{\kern0pt}\ {\isasymcdot}Replacement{\isacharparenleft}{\kern0pt}ground{\isacharunderscore}{\kern0pt}repl{\isacharunderscore}{\kern0pt}fm{\isacharparenleft}{\kern0pt}{\isasymphi}{\isacharparenright}{\kern0pt}{\isacharparenright}{\kern0pt}{\isasymcdot}\ {\isachardot}{\kern0pt}\ {\isasymphi}\ {\isasymin}\ {\isasymPhi}{\isacharbraceright}{\kern0pt}{\isachardoublequoteclose}\isanewline
\ \ \isakeyword{shows}\isanewline
\ \ \ \ {\isachardoublequoteopen}{\isasymexists}N{\isachardot}{\kern0pt}\isanewline
\ \ \ \ \ \ M\ {\isasymsubseteq}\ N\ {\isasymand}\ N\ {\isasymapprox}\ {\isasymomega}\ {\isasymand}\ Transset{\isacharparenleft}{\kern0pt}N{\isacharparenright}{\kern0pt}\ {\isasymand}\ N\ {\isasymTurnstile}\ ZC\ {\isasymunion}\ {\isacharbraceleft}{\kern0pt}{\isasymcdot}CH{\isasymcdot}{\isacharbraceright}{\kern0pt}\ {\isasymunion}\ {\isacharbraceleft}{\kern0pt}\ {\isasymcdot}Replacement{\isacharparenleft}{\kern0pt}{\isasymphi}{\isacharparenright}{\kern0pt}{\isasymcdot}\ {\isachardot}{\kern0pt}\ {\isasymphi}\ {\isasymin}\ {\isasymPhi}{\isacharbraceright}{\kern0pt}\ {\isasymand}\isanewline
\ \ \ \ \ \ {\isacharparenleft}{\kern0pt}{\isasymforall}{\isasymalpha}{\isachardot}{\kern0pt}\ Ord{\isacharparenleft}{\kern0pt}{\isasymalpha}{\isacharparenright}{\kern0pt}\ {\isasymlongrightarrow}\ {\isacharparenleft}{\kern0pt}{\isasymalpha}\ {\isasymin}\ M\ {\isasymlongleftrightarrow}\ {\isasymalpha}\ {\isasymin}\ N{\isacharparenright}{\kern0pt}{\isacharparenright}{\kern0pt}{\isachardoublequoteclose}
\end{isabelle}

Finally, \isatt{overhead{\isacharunderscore}{\kern0pt}CH} is the union
of \isatt{overhead{\isacharunderscore}{\kern0pt}CH} with the $\DC$
instance \isatt{replacement{\isacharunderscore}{\kern0pt}dcwit{\isacharunderscore}{\kern0pt}repl{\isacharunderscore}{\kern0pt}body{\isacharunderscore}{\kern0pt}fm}.
%%% Local Variables: 
%%% mode: latex
%%% TeX-master: "independence_ch_isabelle"
%%% ispell-local-dictionary: "american"
%%% End: 
