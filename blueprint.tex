\documentclass[11pt,english]{article}
\usepackage[bottom=2cm, top=3cm, %% Comment if 
  left=4cm, right=4cm]{geometry}             %% using 'ebook' style 

\input{header-blueprint}
\newcommand{\mt}[1]{$\mathtt{#1}$}
\DeclareUnicodeCharacter{21D2}{\mt\Rightarrow}
\DeclareUnicodeCharacter{2208}{\mt\in}
\DeclareUnicodeCharacter{2200}{\mt\forall}
\DeclareUnicodeCharacter{27F6}{\mt\longrightarrow}
\DeclareUnicodeCharacter{27E6}{\mt\llbracket}
\DeclareUnicodeCharacter{27E7}{\mt\rrbracket}
\DeclareUnicodeCharacter{27F9}{\mt\Longrightarrow}


\renewcommand{\PP}{\mathbb{P}}
\newcommand{\formula}{\ensuremath{\mathtt{formula}}}
\newcommand{\tyi}{\mathtt{i}}
\newcommand{\tyo}{\mathtt{o}}

%%%%%%%%%%%%%%%%%%% Footnotemarks for "Thanks!" %%%%%%%%%%%%%%%%%%
\makeatletter
\renewcommand*{\@fnsymbol}[1]{\ensuremath{\ifcase#1\or *\or \dagger\or \ddagger\or
    \mathsection\or \mathparagraph\or \|\or **\or \dagger\dagger
    \or \ddagger\ddagger \else\@ctrerr\fi}}
\makeatother
%%%%%%%%%%%%%%%%%%%%%%%%%%%%%%%%%%%%%%%%%%%%%%%%%%%%%%%%%%%%%%%%%%

\begin{document}
\title{Forcing for Isabelle}
\author{%
  Pedro Sánchez Terraf%
  \thanks{
    Universidad Nacional de C\'ordoba. 
    Facultad de Matem\'atica, Astronom\'{\i}a,  F\'{\i}sica y
    Computaci\'on.}
  \thanks{
    Centro de Investigaci\'on y Estudios de Matem\'atica (CIEM-FaMAF).
    C\'ordoba. Argentina.}
}
%
\maketitle
%
\begin{abstract}
  Basic definitions of the forcing technique and blueprint for an
  Isabelle/ZF proof of the Forcing Theorems, under the assumption of
  the existence of countable transitive model of $\ZFC$.   
\end{abstract}
%%%%%%%%%%%%%%%%%%%%%%%%%%%%%%%%%%%%%%%%%%%%%%%%%%%%%%%%%%%%%%%%%%


%%%%%%%%%%%%%%%%%%%%%%%%%%%%%%%%%%%%%%%%%%%%%%%%%%%%%%%%%%%%%%%%%%
%
\section{Introduction}
The proofs are mostly based on Chapter IV of
Kunen's~\cite{kunen2011set}. 

\section{Forcing notions}
We will need the following concepts:
\begin{itemize}
\item Forcing poset or \emph{notion}: a triple $\lb \PP, \1, \leq\rb$ where $\leq$ is a
  partial (pre)order on $\PP$ with (a) top element $\1$.
\item Compatible elements. Antichain in $\PP$. 
\item Dense subset $D\sbq\PP$ of a forcing poset.
\item Filter $G\sbq\PP$ in a poset.
\item $\calD$-generic filter for $\PP$, where $\calD$ is a family of
  dense subsets of $\PP$.
\end{itemize}

Most probably, we'll need two or three versions of each of these
definitions: One internalized as an element of \formula;  a relativized
one with a class parameter; and finally one without
relativization. This is done in order to show absolutness and to take
advantage of the Isabelle/ZF ability to operate with terms of type
$\tyi$.

As an example, consider $\PP,R \in M$,  and $R\sbq
\PP\times\PP$. Reflexivity of the relation $R$ is 
stated as follows.
\[
\forall p (p\in \PP \limp \lb p,p\rb \in R).
\]
In Isabelle/ZF, we write
%
\begin{verbatim}
definition 
  reflexivity_abs :: "[i,i] ⇒ o" where
  "reflexivity_abs(P,r) == ∀p . p∈P ⟶ <p,p>∈r"
\end{verbatim}

This is different from what would $M$ see from ``within.'' In
particular, the universal quantifier should range only over elements
of $M$:
\[
\forall p\in M (p\in \PP \limp \lb p,p\rb \in R).
\]
In Isabelle/ZF, we will move to a more general setting, by allowing
that $M$ is a class.
\begin{verbatim}
definition  
  reflexivity_rel :: "[i⇒o,i,i] ⇒ o" where
  "reflexivity_rel(M,P,r) == ∀p[M].  p∈P ⟶ <p,p>∈r"
\end{verbatim}
(Notice: There must be no spaces between \verb|[M]| and the following
period.) This is the relativized version of
\verb|reflexivity_abs|. 

Finally, the internalized version of reflexivity is as follows:
\begin{verbatim}
definition
  reflexivity_fm :: "[i,i]⇒i" where
  "reflexivity_fm(x,y) == Forall(Implies(Member(0,succ(x)),
          Exists(And(pair_fm(1,1,0),Member(0,succ(succ(y)))))))"

lemma reflexivity_type : 
        "⟦x∈nat ; y∈nat⟧ ⟹ reflexivity_fm(x,y)∈formula"
  by (simp add:reflexivity_fm_def)
\end{verbatim}

The last lemma shows that indeed we have defined an element of the set
\formula. The formula \verb|reflexivity_fm| is an internalization of
the Isabelle/ZF formula \verb|reflexivity_abs|, not the relativized
one.




%%%%%%%%%%%%%%%%%%%%%%%%%%%%%%%%%%%%%%%%%%%%%%%%%%%%%%%%%%%%%%%%%%%%%%%%
\bibliographystyle{alpha}
\bibliography{blueprint}

\bigskip

\begin{small}
  \begin{quote}
    \texttt{sterraf@famaf.unc.edu.ar}
    
    CIEM --- Facultad de Matem\'atica, Astronom\'{\i}a y F\'{\i}sica 
    (Fa.M.A.F.) 
    
    Universidad Nacional de C\'ordoba --- Ciudad Universitaria
    
    C\'ordoba 5000. Argentina.
  \end{quote}
\end{small}

\end{document}

%%% Local Variables: 
%%% mode: latex
%%% ispell-local-dictionary: "american"
%%% End: 
