%%%%%%%%%%%%%%%%%%%%%%%%%%%%%%%%%%%%%%%%%%%%%%%%%%%%%%%%%%%%%%%%%%%%%%          
\section{Set models and forcing}
\label{sec:forcing}

\subsection{The $\ZFC$ axioms as locales}
The description of set models of fragments of $\ZFC$ was performed
using Isabelle contexts (\emph{locales}) that fix a variable $M::\tyi$
and pack assumptions stating that $\lb M, \in\rb$ satisfy some
axioms. For example, the locale \isatt{M{\uscore}Z{\uscore}basic}
states that Zermelo set theory holds in $M$. The Union Axiom (\isatt{Union{\uscore}ax}), for
instance, is defined as follows:
\[
\forall x[\isatt{\#\#}M].\ \exists z[\isatt{\#\#}M].\ \isatt{big{\uscore}union}(\isatt{\#\#}M,x,z)
\]
%% % Same a above
%% \begin{isabelle}%
%% {\isasymforall}x{\isacharbrackleft}{\kern0pt}{\isacharhash}{\kern0pt}{\isacharhash}{\kern0pt}M{\isacharbrackright}{\kern0pt}{\isachardot}{\kern0pt}\ {\isasymexists}z{\isacharbrackleft}{\kern0pt}{\isacharhash}{\kern0pt}{\isacharhash}{\kern0pt}M{\isacharbrackright}{\kern0pt}{\isachardot}{\kern0pt}\ big{\isacharunderscore}{\kern0pt}union{\isacharparenleft}{\kern0pt}{\isacharhash}{\kern0pt}{\isacharhash}{\kern0pt}M{\isacharcomma}{\kern0pt}\ x{\isacharcomma}{\kern0pt}\ z{\isacharparenright}{\kern0pt}%
%% \end{isabelle}%
using relativized, relational versions of the axioms of Isabelle/ZF
since the interaction with \session{ZF-Constructible} was smoother
that way and, as it was mentioned in Section~\ref{sec:isabellezf},
this third layer of the formalization has more tools at our
disposal. Later, an equivalent statement “in the codes” (our fourth
layer) is obtained,
\[
  \isatt{Union{\uscore}ax}(\isatt{\#\#}M) \lsii A, [\,]\models \isatt{{\isasymcdot}Union\ Ax{\isasymcdot}}
\]
%% % Same a above
%% \begin{isabelle}
%% Union{\isacharunderscore}{\kern0pt}ax{\isacharparenleft}{\kern0pt}{\isacharhash}{\kern0pt}{\isacharhash}{\kern0pt}A{\isacharparenright}{\kern0pt}\ {\isasymlongleftrightarrow}\ A{\isacharcomma}{\kern0pt}\ {\isacharbrackleft}{\kern0pt}{\isacharbrackright}{\kern0pt}\ {\isasymTurnstile}\ {\isasymcdot}Union\ Ax{\isasymcdot}\isasep
%% \end{isabelle}
where \isatt{{\isasymcdot}Union\ Ax{\isasymcdot}} is the $\formula$ code for the
Union axiom. For the axiom schemes, \session{ZF-Constructible} defines
the expressions
\[
  \isatt{separation}(N,Q)
  \text{ and }
  \isatt{strong{\uscore}replacement}(N,R)
\]
that consume a class $N$ and predicates $Q::\tyi\fun\tyo$ and $R::\tyi\fun\tyi\fun\tyo$, and state that $N$
satisfies the $Q$-instance of Separation and the $R$-instance of
Replacement, respectively. In the statement of the Separation Axiom in
\isatt{M{\uscore}Z{\uscore}basic} and
the many replacement instances, predicates $Q$ and $R$ are actually
defined as the of formulas of the appropriate arity; we can only show
that this form of the axioms hold in generic extensions. 

Further locales gather the assumption of transitivity of $M$ and
particular replacement instances expressed by means of the
\isatt{replacement{\uscore}assm(M,\isasymphi)} predicate ($@$ denotes
list concatenation):
\begin{multline}\label{eq:replacement_assm_def}
\varphi \in \formula  \limp \mathit{env} \in \isatt{list}(M) \limp \isatt{arity}(\varphi) \leq 2+ \isatt{length}(\mathit{env}) \limp \\
 \isatt{strong{\uscore}replacement}(\isatt{\#\#} M, \lambda x\, y.\ (M, [x,y]
\mathbin{@} \mathit{env}  \models \varphi))
\end{multline}
%% % Same as above
%% \begin{isabelle}
%% {\isasymphi}\;{\isasymin}\;formula\ {\isasymlongrightarrow}\ env\;{\isasymin}\;list{\isacharparenleft}{\kern0pt}M{\isacharparenright}{\kern0pt}\ {\isasymlongrightarrow} arity{\isacharparenleft}{\kern0pt}{\isasymphi}{\isacharparenright}{\kern0pt}\;{\isasymle}\;{\isadigit{2}}\ {\isacharplus}{\kern0pt}\isactrlsub {\isasymomega}\ length{\isacharparenleft}{\kern0pt}env{\isacharparenright}{\kern0pt}\ {\isasymlongrightarrow}\isanewline
%% \ \ \ \ strong{\isacharunderscore}{\kern0pt}replacement{\isacharparenleft}{\kern0pt}{\isacharhash}{\kern0pt}{\isacharhash}{\kern0pt}M{\isacharcomma}{\kern0pt}{\isasymlambda}x\ y{\isachardot}{\kern0pt}\ {\isacharparenleft}{\kern0pt}M\ {\isacharcomma}{\kern0pt}\ {\isacharbrackleft}{\kern0pt}x{\isacharcomma}{\kern0pt}y{\isacharbrackright}{\kern0pt}{\isacharat}{\kern0pt}env\ {\isasymTurnstile}\ {\isasymphi}{\isacharparenright}{\kern0pt}{\isacharparenright}
%% \end{isabelle}
Some of those instances, which we included for simplicity of design,
are actually “fake” instances in that they can be obtained from
Powerset. An example of this is the sole instance of the
\isatt{M{\uscore}basic} locale, originally from
\session{ZF-Constructible}; we managed to eliminate it but in
doing so we had to reprove many lemmas from
that session in the weakened context. Our locales
\isatt{M{\uscore}ZF1} through \isatt{M{\uscore}ZF4} list a total of 86
replacement instances, from which there are at least 53 fake ones (see
Section~\ref{sec:33-repl-inst-rule}).

\subsection{The fundamental theorems}
Now let $\lb \PP, {\preceq} ,\1\rb \in M$ be a forcing notion. Given $G\sbq \PP$, we have
$M[G]\defi \{ \val(\PP,G,\punto{a}) : \punto{a}\in M \}$.

The following form of the Forcing Theorems  is the one
that we formalized.
\begin{theorem}\label{th:forcing-thms}
  There exists a function  $\forceisa:: \tyi \fun  \tyi$
  such that for every
  $\phi\in\formula$ and $\punto{a}_0,\dots,\punto{a}_n\in M$,
  \begin{enumerate}
  \item\label{item:definability} (Definability)
    $\forceisa(\phi)\in\formula$, 
  \end{enumerate}
  where the 
  arity of $\forceisa(\phi)$ is at most $\isatt{arity}(\phi) + 4$; and if
  “$p \forces \phi\ [\punto{a}_0,\dots,\punto{a}_n]$”
  denotes
  “$M, [p,\PP,\preceq,\1, \punto{a}_0,\dots,\punto{a}_n]  \models
  \forceisa(\phi)$”, we have:
  \begin{enumerate}
    \setcounter{enumi}{1}
  \item\label{item:truth-lemma} (Truth Lemma) for every $M$-generic $G$,
    \[
      \exists p\in G.\ \; p \forces \phi\ [\punto{a}_0,\dots,\punto{a}_n]
    \]
    is equivalent to 
    \[
      M[G], [\val(\PP,G,\punto{a}_0),\dots,\val(\PP,G,\punto{a}_n)]
      \models \phi.
    \]
  \item \label{item:density-lemma} (Density Lemma) $p \forces \phi\ [\punto{a}_0,\dots,\punto{a}_n]$
    if and only if 
    $\{q\in \PP :  q \forces \phi\ [\punto{a}_0,\dots,\punto{a}_n]\}$
    is dense below $p$.
  \end{enumerate}
\end{theorem}
The statements appear as three separated items in
\theory{Forcing{\uscore}Theorems.thy} in the \session{Independence\_CH} session,
and benefit from the $\isatt{map}$ function that applies a function to
each element of a list. For instance, the Truth Lemma is stated as
follows:
\begin{isabelle}
\isacommand{lemma}\isamarkupfalse%
\ truth{\isacharunderscore}{\kern0pt}lemma{\isacharcolon}{\kern0pt}\isanewline
\ \ \isakeyword{assumes}\isanewline
\ \ \ \ {\isachardoublequoteopen}{\isasymphi}{\isasymin}formula{\isachardoublequoteclose}\ {\isachardoublequoteopen}M{\isacharunderscore}{\kern0pt}generic{\isacharparenleft}{\kern0pt}G{\isacharparenright}{\kern0pt}{\isachardoublequoteclose}\isanewline
\ \ \ \ {\isachardoublequoteopen}env{\isasymin}list{\isacharparenleft}{\kern0pt}M{\isacharparenright}{\kern0pt}{\isachardoublequoteclose}\ {\isachardoublequoteopen}arity{\isacharparenleft}{\kern0pt}{\isasymphi}{\isacharparenright}{\kern0pt}{\isasymle}length{\isacharparenleft}{\kern0pt}env{\isacharparenright}{\kern0pt}{\isachardoublequoteclose}\isanewline
\ \ \isakeyword{shows}\isanewline
\ \ \ \ {\isachardoublequoteopen}{\isacharparenleft}{\kern0pt}{\isasymexists}p{\isasymin}G{\isachardot}{\kern0pt}\ p\ {\isasymtturnstile}\ {\isasymphi}\ env{\isacharparenright}{\kern0pt}\ \ \ {\isasymlongleftrightarrow}\ \ \ M{\isacharbrackleft}{\kern0pt}G{\isacharbrackright}{\kern0pt}{\isacharcomma}{\kern0pt}\ map{\isacharparenleft}{\kern0pt}val{\isacharparenleft}{\kern0pt}P{\isacharcomma}{\kern0pt}G{\isacharparenright}{\kern0pt}{\isacharcomma}{\kern0pt}env{\isacharparenright}{\kern0pt}\ {\isasymTurnstile}\ {\isasymphi}{\isachardoublequoteclose}
\end{isabelle}
where the $\forces$ notation had already been set up in the theory
\theory{Forces{\uscore}Definition} as follows:
\begin{isabelle}
\isacommand{abbreviation}\isamarkupfalse%
\ Forces\ {\isacharcolon}{\kern0pt}{\isacharcolon}{\kern0pt}\ {\isachardoublequoteopen}{\isacharbrackleft}{\kern0pt}i{\isacharcomma}{\kern0pt}\ i{\isacharcomma}{\kern0pt}\ i{\isacharbrackright}{\kern0pt}\ {\isasymRightarrow}\ o{\isachardoublequoteclose}\ \ {\isacharparenleft}{\kern0pt}{\isachardoublequoteopen}{\isacharunderscore}{\kern0pt}\ {\isasymtturnstile}\ {\isacharunderscore}{\kern0pt}\ {\isacharunderscore}{\kern0pt}{\isachardoublequoteclose}\ {\isacharbrackleft}{\kern0pt}{\isadigit{3}}{\isadigit{6}}{\isacharcomma}{\kern0pt}{\isadigit{3}}{\isadigit{6}}{\isacharcomma}{\kern0pt}{\isadigit{3}}{\isadigit{6}}{\isacharbrackright}{\kern0pt}\ {\isadigit{6}}{\isadigit{0}}{\isacharparenright}{\kern0pt}\ \isakeyword{where}\isanewline
\ \ {\isachardoublequoteopen}p\ {\isasymtturnstile}\ {\isasymphi}\ env\ \ \ {\isasymequiv}\ \ \ M{\isacharcomma}{\kern0pt}\ {\isacharparenleft}{\kern0pt}{\isacharbrackleft}{\kern0pt}p{\isacharcomma}{\kern0pt}P{\isacharcomma}{\kern0pt}leq{\isacharcomma}{\kern0pt}{\isasymone}{\isacharbrackright}{\kern0pt}\ {\isacharat}{\kern0pt}\ env{\isacharparenright}{\kern0pt}\ {\isasymTurnstile}\ forces{\isacharparenleft}{\kern0pt}{\isasymphi}{\isacharparenright}{\kern0pt}{\isachardoublequoteclose}\isanewline
\end{isabelle}

We followed Kunen's new \textit{Set Theory} \cite{kunen2011set} to define
$\forceisa$.  Forcing for atomic formulas is described as a mutual
recursion
%% \begin{multline*}
%%   \forceseq (p,t_1,t_2) \defi 
%%   \forall s\in\dom(t_1)\cup\dom(t_2).\ \forall q\pleq p .\\
%%   \forcesmem(q,s,t_1)\lsii \forcesmem(q,s,t_2)
%% \end{multline*}
%% \begin{multline*}
%%   \forcesmem(p,t_1,t_2) \defi  \forall v\pleq p. \ \exists q\pleq v.\\  
%%   \exists s.\ \exists r\in \PP .\ \lb s,r\rb \in  t_2 \land q
%%   \pleq r \land \forceseq(q,t_1,s)
%% \end{multline*}
but then \cite[p.~257]{kunen2011set} it is cast as a single
recursively defined function $\frcat$ over the wellfounded relation
$\isatt{frecR}$ on tuples $\lb \mathit{ft},t_1,t_2,p\rb$ (where
$\mathit{ft}\in\{0,1\}$ indicates the type of the atomic formula being
forced). Forcing for general formulas is defined by recursion on the
datatype $\formula$. Technical details on the implementation and proofs of the
Forcing Theorems have been spelled out in our
\cite{2020arXiv200109715G}.

%%% Local Variables: 
%%% mode: latex
%%% TeX-master: "independence_ch_isabelle"
%%% ispell-local-dictionary: "american"
%%% End: 
