\section{Proof assistants and Isabelle/ZF}
\label{sec:proof-assist-isabelle}

Let us briefly introduce Isabelle \cite{DBLP:books/sp/Paulson94} in the large landscape of proof
assistants (“assistants” for short; also known as “interactive theorem provers”); we refer to the
excellent chapter by \citet{DBLP:series/hhl/HarrisonUW14} for a more
thorough reconstruction of the history of assistants.

It is expected that an assistant aids the human user while mechanizing
some piece of mathematics; the interaction varies from system to
system, but a common interface consists of a display showing the
current goal and assumptions. The user instructs the assistant to
modify them by means of \emph{tactics}; a %% forward rule changes the context
%% with the intention of having in the new context a hypothesis closer to
%% the goal. A backward rule represents a deduction whose conclusion is
%% the goal, thus the new goal corresponds to the premises of the rule. A
proof is completed when the (current) goal is an instance of one of
the assumptions.

In that dialog, the user produces a script of tactics that can be
later reproduced step-by-step by the system (to check, for example,
that an imported theory is correct) or by the user to understand
the proof.
%% \footnote{As we will see there are more declaratives
%% proof languages that aim to have intelligible proof scripts.}

To have any value at all, the system should only allow the application
of sound tactics.
%% ; neither the author nor (any part of)
%% the system should be able to counterfeit proofs.
Edinburgh LCF~\cite{Gordon1979-qm} was an
influential proof assistant in which the critical code (that
constructs proofs in response to user scripts or other modules) was reduced to a small
\emph{kernel}. Hence, by verifying the correctness of the kernel, one
achieves confidence on the whole system.
%% The rest of the system
%% can interact with the kernel via an interface that exposes functions
%% guaranteeing the validity of the proofs.

The metalogic of Isabelle, as well as that of LCF, is based on higher-order logic.
%% In contrast with LCF, Isabelle
%% was designed~\cite{Paulson1990} as a generic theorem for several
%% logics. Isabelle still offers a programmable interface (in ML) that
%% allows the user, for instance, to define a tactic solving a statement
%% about a natural number $n$ by doing cases on whether $n=0$ or
%% $n\neq 0$.  Moreover one can define functions that manipulate terms
%% (in the usual sense) or even formulas.
In contrast, some of the other prominent assistants of today are
based on some (dependent) type theory. Both Coq~\cite{coq} and
Lean~\cite{DBLP:conf/cade/Moura021} are based on the Calculus of
Inductive Constructions
\cite{DBLP:conf/mfps/PfenningP89,DBLP:journals/iandc/CoquandH88},
while Agda~\cite{agda} is an extension of Martin-Löf type
theory~\cite{DBLP:books/daglib/0000395}. Mizar~\cite{mizar} is the
oldest assistant still used today but is far away both in terms of
foundations and architecture from Isabelle; Mizar inspired, though, the
Isar \cite{DBLP:conf/tphol/Wenzel99} dialect used in Isabelle
nowadays, which aims at the production of proof scripts that are
closer to mathematical texts.%
\footnote{%
We recommend the survey \cite{2020arXiv200909541A} by
J.~Avigad for details about the different logical foundations on which
assistants are based.}

% Proof assistants provide diverse aids for the task
% of formalizing a piece of mathematics. They are usually implemented
% using a typed programming language; rigor is enforced by defining a
% type of ``theorems,'' whose members can only be constructed using
% operations stipulated in a small \emph{kernel} which encodes the
% underlying foundation of the assistant. Further developments interact
% with the type of theorems only through the kernel, and thus the latter
% is the only ``trusted'' part of the assistant's code.
%
% Several of the more established assistants (HOL Light, Coq, Isabelle)
% are programmed in some variant of the ML language (which was
% designed for this purpose); the newer Lean, on the other
% hand, was originally conceived as a standalone functional programming
% language with all the features of a standard assistant.

Isabelle also inherited from LCF the possibility for the user to define
tactics to encapsulate common patterns used to solve goals. In
fact, this is the origin of the ML family of languages: a
Meta-Language for programming tactics.
In the case of Isabelle, \emph{Standard} ML is the first of the four layers
on which we worked in this assistant. Both the kernel and the automation
of proofs are coded in ML, sometimes as a substitute for induction on
formulas, as the next section explains.

\subsection{Isabelle metalogic $\Meta$}
\label{sec:isabelle-metalogic-meta}

The second layer of Isabelle is an
intuitionistic fragment of higher-order logic (or simple type theory)
called $\Meta$; its original version was described in \cite{Paulson1989},
and the addition of “sorts” was reported in \cite{Nipkow-LF-91}.

The only predefined type is $\prop$ (“propositions”); new base types
can be postulated when defining objects logics. Types of higher order can be
assembled using the function space constructor $\fun$.

The type of propositions $\prop$ is equipped with a binary operation
${\implies}$ (“meta-implication”) and a universal “meta-quantifier”
$\ALL$, that are used to represent the object
logic rules. As an example, the axiomatization of first-order logic
postulates a type $\tyo$ of booleans, and Modus Ponens
% https://isabelle.in.tum.de/dist/library/FOL/FOL/IFOL.html#IFOL.mp|axiom
is written as
\begin{equation}\label{eq:modus-ponens}
  \ALL P\,Q.\ \ [P\limp Q] \implies ([P] \implies [Q]).
\end{equation}
The square brackets (which are omitted in Isabelle theories, as well
as the outermost universal quantifiers) represent
an injection from $\tyo$ into $\prop$. % ($[P]$ can be read as “$P$
% holds”)
A consequence of this representation is that every formula of
the object logic appears atomic to $\Meta$.

%% Quantification is handled in the meta-level using a functional $\ALL$
%% with polymorphic type $(\alpha \fun \prop) \fun \prop$. 
Types in Isabelle are organized into \emph{classes} and \emph{sorts};
for ease of exposition, we will omit the former.  The axiomatization
of first-order logic postulates a sort $\{\type{term}\}$ (of
“individuals,” or elements of a first-order universe of discourse) and
stipulates that every further type variable $\alpha$ must be of that
sort. In particular, Isabelle/ZF only postulates one new type $\tyi$
(“sets”) of sort $\{\type{term}\}$. Hence, from the type of the universal
quantifier functional $\forall :: (\alpha \fun \tyo) \fun \tyo$, it
follows that it may only be applied to predicates with a variable of
type $\tyi$. This ensures that the object logic is effectively
first-order.

Paulson  \cite{Paulson1989} carried out a proof that the encoding
$\Meta_{\mathrm{IFOL}}$ of
intuitionistic first-order logic IFOL without equality  in the original $\Meta$ is
conservative (there is a correspondence between provable $\phi$ in
IFOL and provable $[\phi]$ in $\Meta_{\mathrm{IFOL}}$) by putting
$\Meta_{\mathrm{IFOL}}$ proofs in \emph{expanded normal form}
\cite{MR0387024}; atomicity as stated after
Equation~(\ref{eq:modus-ponens}) plays a role in this argument. Passing to classical logic does not present
difficulties, but the addition of meta-equality must be taken care of.
Even more so, since the treatment of equality differs between the
original and the present incarnation of $\Meta$; details for the
latter are exhaustively expounded in the recent formalization by
Nipkow and Roßkof \cite{10.1007/978-3-030-79876-5_6}.

The meta-logic $\Meta$ is rather weak; it has no induction/recursion
principles. Types are not inductively presented and, in particular, it
is not possible to prove by induction statements about
object-logic formulas (which are construed as terms of type $\tyi \fun
\dots \fun \tyi \fun \tyo$). Two ways to overcome this limitation are:
\begin{enumerate}
\item
  to construct the
  proof of each instance of the statement by hand or by programming on
  ML; or 
\item
  to encode formulas as sets and prove an internal version statement
  using induction of $\ZF$.
\end{enumerate}

For recursive definitions, only the second option is available, and
that is the way the definition of the forcing relation is implemented
in our formalization.

\subsection{Isabelle/ZF}
\label{sec:isabellezf}

For the most part, the development of set theory in Isabelle is
carried out using its ZF object logic
\cite{DBLP:journals/jar/PaulsonG96}, which is the third logical layer
of the formalization and the most versatile one, since 
Isabelle's native automation is available at this level. Apart from
the type and sort
declarations detailed above, it features a finite axiomatization,
% https://isabelle.in.tum.de/dist/library/FOL/FOL/IFOL.html#ZF_Base.mem|const
with a predicate for membership, constants for the empty set and an
infinite set, and functions $\isa{Pow}::\tyi\fun\tyi$,
$\union::\tyi\fun\tyi$, and $\isa{PrimReplace} :: \tyi \fun (\tyi
\fun \tyi \fun\tyo)\fun \tyi$ (for Replacement). The Axiom of
Replacement
% https://isabelle.in.tum.de/dist/library/FOL/FOL/IFOL.html#ZF_Base.replacement|axiom
has a free predicate variable $P$: % $P ::\tyi\fun\tyi\fun\tyo$:
\begin{multline*}
  (\forall x \in A .\ \forall y\, z.\ P(x, y) \wedge P(x, z)
  \longrightarrow y=z) \implies \\
  b \in \isa{PrimReplace}(A, P)
  \longleftrightarrow(\exists x \in A .\ P(x, b)) 
\end{multline*}
The restrictions on sorts described above ensure that it is not
possible that higher-order quantification gets used in $P$. The
statement of $\AC$ also uses a free higher-order variable to denote
an indexed family of nonempty sets. % but we only
% use $\AC$ for demonstration purposes.

Isabelle/ZF reaches essentially Hessenberg's $|A|\cdot|A| = |A|$. Our decision
(during 2017) to
use this assistant was triggered by its constructibility
library, \session{ZF-Constructible} \citep{paulson_2003},
% https://isabelle.in.tum.de/dist/library/ZF/ZF-Constructible/
which contains the development of $L$, the proof that it satisfies
$\AC$, and a version of the Reflection Principle. The latter was
actually encoded as a series of instructions to Isabelle automatic
proof tools that would prove each particular instance of reflection:
This is an example of what was said at the end of Section~\ref{sec:isabelle-metalogic-meta}.

The development of relativization and absoluteness for classes $C::
\isa{i} \fun \isa{o}$ follows the same pattern. Each particular
concept was manually written in a relational form and relativized.
Here, the contrast between the usual way one regards $\ZF$ as a
first-order theory in the language $\{\in \}$ and the mathematical
practice of freely using defined concepts comes to the
forefront. Assistants have refined mechanisms to cope with defined
concepts and the introduction of new notation (which also make their foundations more complicated
than plain first-order logic), and this is the only way that nontrivial
mathematics can be formalized. But when studying relative interpretations, one
usually assumes a spartan syntax and defines relativization by
induction on formulas of the more succinct language. The approach
taken in \session{ZF-Constructible} is to consider relativizations of
the formulas that define each concept. For instance,
in the case of unions, we find a relativization
%https://isabelle.in.tum.de/dist/library/ZF/ZF-Constructible/Relative.html#Relative.big_union_def_raw|axiom
$\isa{big{\uscore}union}:: (\isa{i} \fun \isa{o}) \fun \isa{i}
\fun \isa{i} \fun \isa{o}$ of the statement
“$\union A = z$”:
\[
 \isa{big{\uscore}union}(M,A,z) \equiv \forall x[M].\ x \in z
 \longleftrightarrow (\exists y[M].\ y \in A \land x \in y)
\]
where $\forall x[M]\dots$ stands for $\forall x.\ M(x)\limp \dots$,
etc. The need to work with \emph{relational} presentations of defined
concepts stems from the fact that the model-theoretic definition of
$L$ requires working with set models and satisfaction, which is
defined for (“codes” of) formulas in the language $\{\in \}$
(viz.\ next Section~\ref{sec:internalized-formulas}).

There is one further point concerning the organization of
\session{ZF-Constructible}. Isabelle provides a very convenient way to
define “contexts,” called \emph{locales}, in which some variables are
fixed and assumptions are made. In the case of the constructibility
library, several locales are defined where the variable $M$ is assumed to
denote a class satisfying certain finite amount of $\ZF$; the weakest
one, \locale{M{\uscore}trans} \cite[Sect.~3]{2020arXiv200109715G}, just
assumes that $M$ is transitive and nonempty. Inside such context, many
absoluteness results are proved. In order to quote those results for a
particular class $C$, one has to \emph{interpret} the locale at
$C$, which amounts to prove that $C$ satisfies the assumptions made by
the context.

\subsection{Internalized formulas}
\label{sec:internalized-formulas}

\session{ZF-Constructible} defines the set $\formula$ of first-order 
formulas in the language $\{ \in \}$, internalized as sets.%
\footnote{These, alongside with lists, are instances of
Isabelle/ZF treatment of inductively defined (internal) datatypes \cite[Sect.~4]{Paulson1995-wz};
induction and recursion theorems for them are proved automatically
(this is in contrast to general well-founded recursion, for which
one has to work with the fundamental recursor $\isa{wfrec}$).}  Its
atomic formulas have the form $\cdot x \in y\cdot$ and $\cdot x =
y\cdot$. (We use dots as a visual aid signaling internalized formulas.)
Variables are represented by de Bruijn indices \cite{MR0321704}, so in
those formulas $x,y \in \omega$; for $\phi\in\formula$ and
$z\in\omega$, $z$ is free in $\phi$ if it occurs under at most $z$
quantifiers. The \isa{arity} function on $\phi$ is one plus the
maximum free index occurring in $\phi$.

The satisfaction predicate
$\isa{sats}::\tyi\fun\tyi\fun\tyi\fun\tyo$ takes as arguments a set
$M$, a list $\mathit{env}\in\isa{list}(M)$ for the assignment of
free indices, and $\phi\in\formula$, and it is written
$M,\mathit{env}\models\phi$ in our formalization.
%% \footnote{Notice that
%%   $M,\mathit{env}\models\phi$ is only meaningful if
%%   $\isa{length}(\mathit{env}) \geqslant
%% \isa{arity}(\phi)$.\textbf{$\leftarrow$ necessary?}}
This completes the
description of the fourth and last formal layer of the development.

Internalized formulas for most (but not all) of the relational
concepts can be obtained by guiding the automatic tactics. But in the
early development of 
\session{ZF-Constructible}, most of the concepts were internalized by
hand; this is the case for union,
\begin{multline*}
  \isa{big{\uscore}union{\uscore}fm}(A,z) \equiv \\
  (({\cdot}\forall{\cdot}\,{\cdot}0 \in \mathit{succ}(z){\cdot} \longleftrightarrow
  ({\cdot}\exists{\cdot}\,{\cdot}0 \in \mathit{succ}(\mathit{succ}(A)){\cdot} \land {\cdot}1 \in
  0{\cdot}\,{\cdot}{\cdot}){\cdot}{\cdot})
\end{multline*}
for which we have the following satisfaction lemma:
% https://isabelle.in.tum.de/dist/library/ZF/ZF-Constructible/L_axioms.html#L_axioms.sats_big_union_fm|fact
\begin{multline}\label{eq:sats_big_union_fm}
  A \in \omega \implies z \in \omega \implies \mathit{env} \in \isa{list}(M)
  \implies \\
  \bigl(M, \mathit{env} \models \isa{big{\uscore}union{\uscore}fm}(A,z)\bigr)%\cdot\union x \isa{ is } y\cdot
  \longleftrightarrow \\
  \isa{big{\uscore}union}(\isa{\#\#} M, \isa{nth}(A,
  \mathit{env}), \isa{nth}(z, \mathit{env}))
\end{multline}
Above, $\isa{nth}(x,\mathit{env})$ is the $x$th element of $\mathit{env}$
and $\isa{\#\#}M::\tyi\fun \tyo$ is the class corresponding to the
set $M::\tyi$.

%%% Local Variables:
%%% mode: latex
%%% TeX-master: "independence_ch_isabelle"
%%% ispell-local-dictionary: "american"
%%% End: 
