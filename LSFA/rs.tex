\subsection{A sequence version of Dependent
  Choices}\label{sec:sequence-version-dc} 
%% In order to prove the Rasiowa-Sikorski lemma in
%% general, a consequence of the Axiom of Choice ($\AC$) must be invoked. This
%% consequence is known as the \emph{Principle of Dependent Choices}:
The Rasiowa-Sikorski lemma follows naturally from a ``pointed''
version of   the \emph{Principle of Dependent Choices ($\DC$)} which,
in turn, is a consequence of the Axiom of Choice ($\AC$). It is
therefore natural to take as a starting point the theory \isatt{AC}
which adds the latter axiom to the toolkit of Isabelle/ZF.

The statement we are interested in is the following:
\begin{quote}
  (Pointed $\DC$) Let $\rr$ be a binary relation on $A$, and $a\in A$. If
  $\forall x\in A.\,  \exists y\in A.\, x\rr y$, then there exists
  $f:\om\to A$ such that $f(0)=a$ and $f(n)\rr f(n+1)$ for all
  $n\in\om$.
\end{quote}

Two different versions of $\DC$ (called $\DC0$ and $\DC(\kappa)$) have
already been formalized by Krzysztof
Grabczewski~\cite{DBLP:journals/jar/PaulsonG96}, as part of a study of
equivalents of $\AC$ (following Rubin and Rubin
\cite{rubin1985equivalents}). Nevertheless, those are not convenient
for our purposes. In fact, the axiom $\DC0$ corresponds essentially to
our Pointed $\DC$ but without the constraint $f(0)=a$; it is a nice
exercise to show that $\DC0$ implies Pointed $\DC$, but a
formalization would have a moderate length. On the other hand,
$\DC(\kappa)$ is rather different in nature and it is tailored to
obtain another proposition equivalent to the axiom of choice
(actually,
$\AC\longleftrightarrow (\forall \kappa .\;
\mathrm{Card}(\kappa)\longrightarrow \DC(\kappa))$).  Finally, the
shortest path from $\AC$ %% as stated in the
%% corresponding Isabelle theory
to $\DC0$ using already formalized
material involves a complicated detour (130+ proof commands spanning
various files of the \isatt{ZF-AC} theory and going through the Well
Ordering Theorem and  $\DC(\omega)$), compared to the mere 11 commands from $\AC$ to
\isatt{AC{\isacharunderscore}func{\isacharunderscore}Pow}. This last
one is the
choice principle that we use in our formalization of Pointed $\DC$, and states 
the existence of
choice functions on $\P(A)\sm \{\mty\}$):
\[
\exists (s: \P(A)\sm \{\mty\}\to A). \, \forall X\sbq A.\ X\neq \mty \limp
s(X) \in X.
\]
Another advantage of taking
\isatt{AC{\isacharunderscore}func{\isacharunderscore}Pow} as a
starting point is that it does not involve proper classes: The version
of  $\AC$ in Isabelle/ZF corresponds to an axiom scheme of first-order
logic and as such is not a standard formulation. 
%% 26 commands to get Pointed $\DC$.

The strategy to prove Pointed $\DC$ (following a proof in Moschovakis
\cite{moschovakis1994notes}) is to define the function $f$ discussed above by
primitive recursion on the naturals, which can be done easily thanks
to the package of Isabelle/ZF
\cite{paulson1995set,paulson2000fixedpoint} for definitions by
recursion on inductively defined sets.\footnote{The package figures
  out the inductive set at hand and checks that the recursive
  definition makes sense; for example, it rejects definitions with a
  missing case.}

\begin{isabelle}\isamarkuptrue%
\isacommand{consts}\isamarkupfalse%
\ dc{\isacharunderscore}witness\ {\isacharcolon}{\isacharcolon}\ {\isachardoublequoteopen}i\ {\isasymRightarrow}\ i\ {\isasymRightarrow}\ i\ {\isasymRightarrow}\ i\ {\isasymRightarrow}\ i\ {\isasymRightarrow}\ i{\isachardoublequoteclose}\isanewline
\isacommand{primrec}\isamarkupfalse%
\isanewline
\ \ wit{\isadigit{0}}\ \ \ {\isacharcolon}\ {\isachardoublequoteopen}dc{\isacharunderscore}witness{\isacharparenleft}{\isadigit{0}}{\isacharcomma}A{\isacharcomma}a{\isacharcomma}s{\isacharcomma}R{\isacharparenright}\ {\isacharequal}\ a{\isachardoublequoteclose}\isanewline
\ \ witrec\ {\isacharcolon}\ {\isachardoublequoteopen}dc{\isacharunderscore}witness{\isacharparenleft}succ{\isacharparenleft}n{\isacharparenright}{\isacharcomma}A{\isacharcomma}a{\isacharcomma}s{\isacharcomma}R{\isacharparenright}\ {\isacharequal}\ s{\isacharbackquote}{\isacharbraceleft}x{\isasymin}A{\isachardot}\ {\isasymlangle}dc{\isacharunderscore}witness{\isacharparenleft}n{\isacharcomma}A{\isacharcomma}a{\isacharcomma}s{\isacharcomma}R{\isacharparenright}{\isacharcomma}x{\isasymrangle}{\isasymin}R\ {\isacharbraceright}{\isachardoublequoteclose}
\end{isabelle}

Besides the natural argument and the parameters $A$, $a$, and $R$, the
function \isatt{dc{\isacharunderscore}witness} has a function $s$ as a
parameter. If this function is a selector for $\P(A)\sm \{\mty\}$, the
function
$f(n)\defi {}$\isatt{dc{\isacharunderscore}witness}$(n,A,a,s,R)$ will
satify $\DC$. Notice that $s$ is a term of type \isatt{i} (a function
construed as a set of pairs) and an expression
\isatt{s{\isacharbackquote}b} is notation for \isatt{apply(s,b)},
where \isatt{apply\ {\isacharcolon}{\isacharcolon}\
  {\isachardoublequoteopen}i\ {\isasymRightarrow}\ i\
  {\isasymRightarrow}\ i{\isachardoublequoteclose}} is the operation
of function application.
% We will not make a distinction in our prose and simply write
% \isatt{s}(\isatt{b}) in this case.

The proof is mostly routine; after a few lemmas (26 proof
commands in total) we obtain the
following theorem:

\begin{isabelle}
\isacommand{theorem}\isamarkupfalse%
\ pointed{\isacharunderscore}DC\ \ {\isacharcolon}\ {\isachardoublequoteopen}{\isacharparenleft}{\isasymforall}x{\isasymin}A{\isachardot}\ {\isasymexists}y{\isasymin}A{\isachardot}\ {\isasymlangle}x{\isacharcomma}y{\isasymrangle}{\isasymin}\ R{\isacharparenright}\ {\isasymLongrightarrow}\isanewline
\ \ \ \ \ \ \ \ \ \ \ \ \ \ \ \ \ \ \ \ \ \ \ {\isasymforall}a{\isasymin}A{\isachardot}\ {\isacharparenleft}{\isasymexists}f\ {\isasymin}\ nat{\isasymrightarrow}A{\isachardot}\ f{\isacharbackquote}{\isadigit{0}}\ {\isacharequal}\ a\ {\isasymand}\ {\isacharparenleft}{\isasymforall}n\ {\isasymin}\ nat{\isachardot}\ {\isasymlangle}f{\isacharbackquote}n{\isacharcomma}f{\isacharbackquote}succ{\isacharparenleft}n{\isacharparenright}{\isasymrangle}{\isasymin}R{\isacharparenright}{\isacharparenright}{\isachardoublequoteclose}
\end{isabelle}

We need a further, ``diagonal'' version of $\DC$  to prove
Rasiowa-Sikorski. That is, if the assumption holds for a sequence of
relations $S_n$,  then $f(n) \mathrel{S_{n+1}} f(n+1)$.

We first obtain a corollary of $\DC$ changing $A$ for
$A\times\mathtt{nat}$:

\begin{isabelle}
\isacommand{corollary}\isamarkupfalse%
\ DC{\isacharunderscore}on{\isacharunderscore}A{\isacharunderscore}x{\isacharunderscore}nat\ {\isacharcolon}\ \isanewline
\ \ {\isachardoublequoteopen}{\isacharparenleft}{\isasymforall}x{\isasymin}A{\isasymtimes}nat{\isachardot}\ {\isasymexists}y{\isasymin}A{\isachardot}\ {\isasymlangle}x{\isacharcomma}{\isasymlangle}y{\isacharcomma}succ{\isacharparenleft}snd{\isacharparenleft}x{\isacharparenright}{\isacharparenright}{\isasymrangle}{\isasymrangle}\ {\isasymin}\ R{\isacharparenright}\ {\isasymLongrightarrow}\isanewline
\ \ \ \ {\isasymforall}a{\isasymin}A{\isachardot}\ {\isacharparenleft}{\isasymexists}f\ {\isasymin}\ nat{\isasymrightarrow}A{\isachardot}\ f{\isacharbackquote}{\isadigit{0}}\ {\isacharequal}\ a\ {\isasymand}\ {\isacharparenleft}{\isasymforall}n\ {\isasymin}\ nat{\isachardot}\ {\isasymlangle}{\isasymlangle}f{\isacharbackquote}n{\isacharcomma}n{\isasymrangle}{\isacharcomma}{\isasymlangle}f{\isacharbackquote}succ{\isacharparenleft}n{\isacharparenright}{\isacharcomma}succ{\isacharparenleft}n{\isacharparenright}{\isasymrangle}{\isasymrangle}{\isasymin}R{\isacharparenright}{\isacharparenright}{\isachardoublequoteclose}
\end{isabelle}

%
The following lemma is then proved automatically:

\begin{isabelle}
\isacommand{lemma}\isamarkupfalse%
\ aux{\isacharunderscore}sequence{\isacharunderscore}DC\ {\isacharcolon}\ {\isachardoublequoteopen}{\isasymforall}x{\isasymin}A{\isachardot}\ {\isasymforall}n{\isasymin}nat{\isachardot}\ {\isasymexists}y{\isasymin}A{\isachardot}\ {\isasymlangle}x{\isacharcomma}y{\isasymrangle}\ {\isasymin}\ S{\isacharbackquote}n\ {\isasymLongrightarrow}\isanewline
 \ \ {\isasymforall}x{\isasymin}A{\isasymtimes}nat{\isachardot}\ {\isasymexists}y{\isasymin}A{\isachardot}\ {\isasymlangle}x{\isacharcomma}{\isasymlangle}y{\isacharcomma}succ{\isacharparenleft}snd{\isacharparenleft}x{\isacharparenright}{\isacharparenright}{\isasymrangle}{\isasymrangle}\ {\isasymin}\ {\isacharbraceleft}{\isasymlangle}{\isasymlangle}w{\isacharcomma}n{\isasymrangle}{\isacharcomma}{\isasymlangle}y{\isacharcomma}m{\isasymrangle}{\isasymrangle}{\isasymin}{\isacharparenleft}A{\isasymtimes}nat{\isacharparenright}{\isasymtimes}{\isacharparenleft}A{\isasymtimes}nat{\isacharparenright}{\isachardot}\ {\isasymlangle}w{\isacharcomma}y{\isasymrangle}{\isasymin}S{\isacharbackquote}m\ {\isacharbraceright}{\isachardoublequoteclose}\isanewline
\ \ %
\isacommand{by}\isamarkupfalse%
\ auto%
\end{isabelle}
%
And finally, we arrive to $\DC$ for a sequence of relations.

\begin{isabelle}
\isacommand{lemma}\isamarkupfalse%
\ sequence{\isacharunderscore}DC{\isacharcolon}\ {\isachardoublequoteopen}{\isasymforall}x{\isasymin}A{\isachardot}\ {\isasymforall}n{\isasymin}nat{\isachardot}\ {\isasymexists}y{\isasymin}A{\isachardot}\ {\isasymlangle}x{\isacharcomma}y{\isasymrangle}\ {\isasymin}\ S{\isacharbackquote}n\ {\isasymLongrightarrow}\isanewline
\ \ \ \ {\isasymforall}a{\isasymin}A{\isachardot}\ {\isacharparenleft}{\isasymexists}f\ {\isasymin}\ nat{\isasymrightarrow}A{\isachardot}\ f{\isacharbackquote}{\isadigit{0}}\ {\isacharequal}\ a\ {\isasymand}\ {\isacharparenleft}{\isasymforall}n\ {\isasymin}\ nat{\isachardot}\ {\isasymlangle}f{\isacharbackquote}n{\isacharcomma}f{\isacharbackquote}succ{\isacharparenleft}n{\isacharparenright}{\isasymrangle}{\isasymin}S{\isacharbackquote}succ{\isacharparenleft}n{\isacharparenright}{\isacharparenright}{\isacharparenright}{\isachardoublequoteclose}
\end{isabelle}

\subsection{The Rasiowa-Sikorski lemma}\label{sec:rasiowa-sikorski-lemma}
In order to state this Lemma, we gather the relevant hypotheses into a locale:

\begin{isabelle}%
\isacommand{locale}\isamarkupfalse%
\ countable{\isacharunderscore}generic\ {\isacharequal}\ forcing{\isacharunderscore}notion\ {\isacharplus}\isanewline
\ \ \isakeyword{fixes}\ {\isasymD}\isanewline
\ \ \isakeyword{assumes}\ countable{\isacharunderscore}subs{\isacharunderscore}of{\isacharunderscore}P{\isacharcolon}\ \ {\isachardoublequoteopen}{\isasymD}\ {\isasymin}\ nat{\isasymrightarrow}Pow{\isacharparenleft}P{\isacharparenright}{\isachardoublequoteclose}\isanewline
\ \ \isakeyword{and}\ \ \ \ \ seq{\isacharunderscore}of{\isacharunderscore}denses{\isacharcolon}\ \ \ \ \ \ \ \ {\isachardoublequoteopen}{\isasymforall}n\ {\isasymin}\ nat{\isachardot}\ dense{\isacharparenleft}{\isasymD}{\isacharbackquote}n{\isacharparenright}{\isachardoublequoteclose}
\end{isabelle}
%
That is, $\calD$ is a sequence of dense subsets of the poset $P$. A
filter is \emph{$\calD$-generic} if it intersects every dense set in
the sequence.

\begin{isabelle}%
\isacommand{definition}\isamarkupfalse%
\ \ D{\isacharunderscore}generic\ {\isacharcolon}{\isacharcolon}\ {\isachardoublequoteopen}i{\isasymRightarrow}o{\isachardoublequoteclose}\ \isakeyword{where}\isanewline
\ \ {\isachardoublequoteopen}D{\isacharunderscore}generic{\isacharparenleft}G{\isacharparenright}\ {\isacharequal}{\isacharequal}\ filter{\isacharparenleft}G{\isacharparenright}\ {\isasymand}\ {\isacharparenleft}{\isasymforall}n{\isasymin}nat{\isachardot}{\isacharparenleft}{\isasymD}{\isacharbackquote}n{\isacharparenright}{\isasyminter}G{\isasymnoteq}{\isadigit{0}}{\isacharparenright}{\isachardoublequoteclose}
\end{isabelle}

We can now state the Rasiowa-Sikorski Lemma.
\begin{isabelle}%
\isacommand{theorem}\isamarkupfalse%
\ rasiowa{\isacharunderscore}sikorski{\isacharcolon}\isanewline
\ \ {\isachardoublequoteopen}p{\isasymin}P\ {\isasymLongrightarrow}\ {\isasymexists}G{\isachardot}\ p{\isasymin}G\ {\isasymand}\ D{\isacharunderscore}generic{\isacharparenleft}G{\isacharparenright}{\isachardoublequoteclose}
\end{isabelle}

The intuitive argument for the result is simple: Once $p_0=p\in P$ is
fixed, we can recursively choose $p_{n+1}$ such that 
$p_n \geq p_{n+1}\in \calD_n$, since $\calD_n$ is dense in $P$. Then
the filter generated by $\{p_n : n\in \om\}$ intersects each set in
the sequence $\{\calD_n\}_n$. This argument appeals to the sequence
version of $\DC$; we have to prove first that the relevant relation
satisfies its hypothesis:

\begin{isabelle}%
\isacommand{lemma}\isamarkupfalse%
\  RS{\isacharunderscore}relation{\isacharcolon}\isanewline
\ \ \isakeyword{assumes}\isanewline
\ \ \ \ \ \ \ \ {\isadigit{1}}{\isacharcolon}\ \ {\isachardoublequoteopen}x{\isasymin}P{\isachardoublequoteclose}\isanewline
\ \ \ \ \ \ \ \ \ \ \ \ \isakeyword{and}\isanewline
\ \ \ \ \ \ \ \ {\isadigit{2}}{\isacharcolon}\ \ {\isachardoublequoteopen}n{\isasymin}nat{\isachardoublequoteclose}\isanewline
\ \ \isakeyword{shows}\isanewline
\ \ \ \ \ \ \ \ \ \ \ \ {\isachardoublequoteopen}{\isasymexists}y{\isasymin}P{\isachardot}\ {\isasymlangle}x{\isacharcomma}y{\isasymrangle}\ {\isasymin}\ {\isacharparenleft}{\isasymlambda}m{\isasymin}nat{\isachardot}\ {\isacharbraceleft}{\isasymlangle}x{\isacharcomma}y{\isasymrangle}{\isasymin}P{\isacharasterisk}P{\isachardot}\ {\isasymlangle}y{\isacharcomma}x{\isasymrangle}{\isasymin}leq\ {\isasymand}\ y{\isasymin}{\isasymD}{\isacharbackquote}{\isacharparenleft}pred{\isacharparenleft}m{\isacharparenright}{\isacharparenright}{\isacharbraceright}{\isacharparenright}{\isacharbackquote}n{\isachardoublequoteclose}
\end{isabelle}
%
As these two proofs are interesting, we have proved them using the
Isar proof language~\cite{DBLP:conf/tphol/Wenzel99}.


%%% Local Variables:
%%% mode: latex
%%% ispell-local-dictionary: "american"
%%% TeX-master: "first_steps_into_forcing"
%%% End:
