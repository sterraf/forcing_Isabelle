\section{Future directions}
\label{sec:conclusion}

One of our main goals was to provide a \emph{proof document}, which
can hopefully be read by mathematicians/set-theorists.

Where did we fail on this?
\begin{enumerate}
\item
  The definition of forces is too codish.
\item
  Proving instances of axiom schemes.
\end{enumerate}

The experience of formalizing math:
\begin{enumerate}
\item It is fun, almost addictive.
\item Feeling of accomplishment after seeing your writings
  validated beyond reasonable doubt (v.g. cofinality).
\item On the grim side: It is easy to forget to put perspective on a
  formalization issue and meditate on paper.
\end{enumerate}

Loose ends:
\begin{enumerate}
\item
  Boolean algebras;
\item
  Gödel for Separation.
\item
  Refine Constructible even more? Building with Lambda replacements
\item
  Connect to \session{ZFC\_in\_HOL} \cite{2022arXiv220503159P}.
\item
  Can we move more of the forcing infraestructure to class locales?
  For instance, names.
\end{enumerate}

\section*{Acknowledgments}
\label{sec:acknowledgments}
We thank Z.~Vidnyánszky for some discussions where the idea of
building the summary in \theory{Definitions\_Main} took shape.
Mikhail Mandrykin helped us with technical assistance regarding the
rendering of formulas in the Isabelle/jEdit interface, which improved
our presentation of this material during the Latin American Congress
of Mathematicians (CLAM 2021); we make this extensive to the rest of
the Isabelle community, for its kind support through the Users mailing
list. We also want to warmly thank Larry Paulson for his work and
encouragement.  Last, and definitely not least, we are deeply grateful
to Ken Kunen for his
inspiring expositions. This formalization is a tiny homage to his
memory.


%%% Local Variables: 
%%% mode: latex
%%% TeX-master: "independence_ch_isabelle"
%%% ispell-local-dictionary: "american"
%%% End: 
