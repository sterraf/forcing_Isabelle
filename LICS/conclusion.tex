%%%%%%%%%%%%%%%%%%%%%%%%%%%%%%%%%%%%%%%%%%%%%%%%%%%%%%%%%%%%%%%%%%%%%%          
\section{Conclusions and future work}
\label{sec:conclusions-future-work}

In a gross simplification, there are two aspects to a formalization
project like this one: thematic and programmatic. The first concerns
the handling of all the theoretical concepts and results in the
subject, while the second involves the practical issues of the
implementation and design. In the case of forcing, the main intricacy
lies in the first aspect. In this sense, following a sensible
presentation of the material is key.  The authoritative reference 
on the subject during the last 30 years has been Kunen's classical
\cite{kunen1980}, and  we have followed a recent rewrite
\cite{kunen2011set} 
of that  textbook, which presents the material in the same sharp 
style but offering a lot of details. In some sense this project
wouldn't exist without this book.

The proof that the axioms of $\ZF$ hold in generic extensions is 
the first important landmark in the path of formalization of the
theory of forcing. The task of proving the Fundamental Theorems can be
considered a part of that landmark, but from a workflow point of view
this can be done independently, and one of our design decisions was to
postpone that task.  Next, various forcing notions must be developed
in order to obtain extensions satisfying diverse properties (v.g., the
Continuum Hypothesis). This yields relative consistency results.

We first discuss the axioms we are now missing. The proof of the
Powerset Axiom does not present new difficulties but one  has to
take care of absoluteness issues; it is one of the
first goals in the immediate future. This is also the case with the
Axiom of Infinity; it can be proved in a direct but cumbersome way. We
preferred to wait to have a full-fledged interface between ctms and the
locales providing recursive constructions from Paulson's
\isatt{Constructibility} session. Then the current proof will hold
with no extra assumptions. The same goes for the results $M\sbq M[G]$
and $G\in M[G]$.

The Replacement Axiom, however, requires some more work to be
done. Both in Kunen and in other presentations \cite{neeman-course}
it requires a relativized version (i.e., showing that it holds for $M$) of
the \emph{Reflection Principle}. In order to state this meta-theoretic
result by Montague, recall that
an equivalent formulation of the Foundation Axiom states that the 
universe of sets can be decomposed in a transfinite hierarchy of
sets: 
\begin{theorem}
  Let $V_{\al}\defi\union\{\P(V_\be) : \be<\al\}$ for each ordinal
  $\al$. Then each $V_\al$ is a set and 
  $\forall x. \exists\al .\ \Ord(\al) \y x\in V_\al$.  
\end{theorem}
\begin{theorem}[Reflection Principle]\label{th:reflection-principle}
  For every finite $\Phi\sbq\ZF$, $\ZF$ proves: ``There exists
  unboundedly many $\al$ such that $V_\al\models \Phi$.''
\end{theorem}
%% This result has also other applications; in particular, it dispenses
%% the need of ctms of the whole of $\ZF$ for consistency proofs (by
%% repeating  the proof of Lemma~\ref{lem:wf-model-implies-ctm}, now
%% starting with a  $V_\al$ satisfying $\Phi$). 

This statement is straightforwardly equivalent to that with $\Phi$
consisting of a single formula. The schematic nature of this result
hints at a proof by induction on formulas, and hence it must be shown
internally. 

This  is an appropriate spot to insist that the internal/external
dichotomy has been a powerful agent in the shaping of our
project. It traces back to the definition of the formula-transformer
$\forceisa$, which must be done by recursion, and this requirement
spills indirectly to the proof of the Separation Axiom (despite the
latter
is not by induction). Now the need of using internalized formulas
reappears with the need of having a general version of the Reflection
Principle (because the $\Phi$ there depends on the instance of Replacement
being proved).

A secondary goal of this project is to assess which assumptions on the
original ctm $M$ are needed to develop the forcing machinery. Up to this
point we have some anecdotal data; for example, to show that a
2-variable instance of Separation holds in $M[G]$, one needs to use a
6-variable instance in $M$. In the future, we expect to condense in a
new locale a finite list of (instances of) axioms on $M$ enough to
perform forcing constructions; this list will likely include all the
instances of Separation and Replacement in $M$ that are needed to
satisfy the requirements of the locales in the
\isatt{ZF-Constructible} session.  

A few words are in order in relation to \isatt{ZF-Constructible}.
We believe that this  session  would benefit from some changes in
the way some results are organized accross theory files; for
intance, by cataloging all or most of the internalized formulas in one
file. Another, most basic example would be to start out with an even
more locale that only assumes $M$ to be a nonempty transitive class,
as many absoluteness results follow from these hypotheses.
As Paulson comments in the sources, it would have been better to
minimize the use of the Powerset Axiom in locales and proofs. There
are  useful natural models that satisfy a fraction of $\ZF$ not including
this particular axiom, and to ensure a broader applicability, it would
be convenient to have  absoluteness results not assuming it.
It is our desire to advocate a future work to a thorough
revision of the development of constructibility to address these needs
and to maximize modularity.


\section*{Acknowledgment}


%%% Local Variables: 
%%% mode: latex
%%% TeX-master: "Separation_In_MG"
%%% ispell-local-dictionary: "american"
%%% End: 
