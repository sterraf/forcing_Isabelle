%%%%%%%%%%%%%%%%%%%%%%%%%%%%%%%%%%%%%%%%%%%%%%%%%%%%%%%%%%%%%%%%%%%%%%          
\section{Isabelle}
Isabelle provides a meta-language called \emph{Pure} that consists of
a fragment of higher order logic. The meta-Boolean type is called
\isatt{prop}. Meta connectives
\isatt{\isasymLongrightarrow} and \isatt{\&\&\&} fulfill the role of
implication and conjunction, and the meta-binder \isatt{\isasymAnd}
corresponds to universal quantification. 

On top of \emph{Pure}, theories/object logics can be defined, with
their own types, connectives and rules. Rules can be written  using
meta-implication: ``$P$, $Q$, and $R$ yield $S$'' can be written
\[
P \ \isatt{\isasymLongrightarrow}\ Q\ \isatt{\isasymLongrightarrow}\ R\ \isatt{\isasymLongrightarrow}\ S
\]
(as usual,  \isatt{\isasymLongrightarrow} asociates to the right), and
syntactic sugar is provided to curry the previous as follows:
\[
\isasymlbrakk P; Q; R \isasymrbrakk \ \isatt{\isasymLongrightarrow}\ S.
\]

We work in the object theory \emph{Isabelle/ZF}. Two types are defined
in this theory: \tyo, the object-Booleans, and \tyi,
sets. It must be insisted that the types are defined axiomatically, not
recursively. That is, although there are constants and functions that
generate elements of both types, neither of them are 
\emph{initial}, in the sense that they are not the least types
obtained by combining the given constants and operations. This will
have concrete consequences in our strategy to approach the development.


%%% Local Variables: 
%%% mode: latex
%%% TeX-master: "Separation_In_MG"
%%% ispell-local-dictionary: "american"
%%% End: 
