% This is samplepaper.tex, a sample chapter demonstrating the
% LLNCS macro package for Springer Computer Science proceedings;
% Version 2.20 of 2017/10/04
%
\documentclass[runningheads]{llncs}
%
\usepackage[utf8]{inputenc}
\usepackage{microtype}
\usepackage{isabelle_indepCH,isabellesym_indepCH}
\input{header-indepCH}
\usepackage{graphicx}
% Used for displaying a sample figure. If possible, figure files should
% be included in EPS format.
%
\hypersetup{
  colorlinks,
  urlcolor={blue},
  linkcolor={blue!50!black},
  citecolor={blue!50!black},
}
% If you use the hyperref package, please uncomment the following line
% to display URLs in blue roman font according to Springer's eBook style:
% \renewcommand\UrlFont{\color{blue}\rmfamily}

\begin{document}
%
\title{The formal verification of the ctm approach to forcing%Some lessons after the formalization of the ctm approach to forcing%
  \thanks{Supported by Secyt-UNC project 33620180100465CB and Conicet.}%
}
%
\titlerunning{Formalization of ctm forcing}%Lessons after formalizing ctm forcing}
% If the paper title is too long for the running head, you can set
% an abbreviated paper title here
%
\author{Emmanuel Gunther\inst{1} \and
Miguel Pagano\inst{1} \and \\
Pedro Sánchez Terraf\inst{1,2}%\orcidID{0000-0003-3928-6942}
\and
Matías Steinberg\inst{1}
}
%
\authorrunning{E.~Gunther, M.~Pagano, P.~Sánchez Terraf, M.~Steinberg}
% First names are abbreviated in the running head.
% If there are more than two authors, 'et al.' is used.
%
\institute{Universidad Nacional de C\'ordoba. 
  \\  Facultad de Matem\'atica, Astronom\'{\i}a,  F\'{\i}sica y
  Computaci\'on.
  \and
    Centro de Investigaci\'on y Estudios de Matem\'atica (CIEM-FaMAF),
    Conicet. C\'ordoba. Argentina. \\
    \email{\{gunther,pagano,sterraf\}@famaf.unc.edu.ar\\
        matias.steinberg@mi.unc.edu.ar}
}
%
\maketitle              % typeset the header of the contribution
%
\begin{abstract}
  We discuss some highlights of our computer-verified
  proof of the construction, given a countable transitive set-model $M$
  of $\ZFC$, of generic extensions  satisfying $\ZFC+\neg\CH$ and $\ZFC+\CH$.
  Moreover, let $\calR$ be the set of instances of the Axiom of
  Replacement. We isolated a 21-element subset $\Omega\sbq\calR$ and
  defined $\calF:\calR\to\calR$
  such that for every finite $\Phi\sbq\calR$
  we have
  $M\models \ZC \cup \calF\dimg\Phi \cup \Omega$ implies
  $M[G]\models \ZC \cup \Phi \cup \{ \neg \CH \}$ for any $M$-generic $G$, where $\ZC$ is
  Zermelo set theory with Choice.
  We also obtained the formulas yielded by the Forcing Definability Theorem
  explicitly.

  To achieve this, we worked in the proof assistant \emph{Isabelle},
  basing our development on the theory Isabelle/ZF by L.~Paulson and
  others.

  %% The vantage point of the talk will be that of a mathematician but
  %% elements from the computer science perspective will be
  %% present. Perhaps some myths regarding what can effectively be done
  %% using  proof assistants/checkers will
  %% be dispelled.

  We also compare our formalization with the recent one by J.~M.~Han and F.~van Doorn in the proof assistant \emph{Lean}.

  \keywords{forcing \and Isabelle/ZF \and countable transitive models
    \and absoluteness \and generic extension.}
\end{abstract}
%
%
%
\section{Introduction}
\label{sec:introduction}

This paper is the culmination of our project on the computerized
formalization of the undecidability of the Continuum Hypothesis
($\CH$) from Zermelo-Fraenkel set theory with Choice ($\ZFC$), under the
assumption of the existence of a countable transitive model (ctm) of
$\ZFC$. In contrast to our reports of the previous steps towards this
goal
\cite{2018arXiv180705174G,2019arXiv190103313G,2020arXiv200109715G}, we
intend here to present our development to the mathematical logic
community. For this reason, we start with a general discussion around
the formalization of mathematics.

\subsection{Formalized mathematics}
The use of computers to assist the creation and verification of
mathematics has seen a steady grow. But the general awareness on the
matter still seems to be a bit scant (even among mathematicians
involved in foundations), and the venues devoted to the communication
of formalized mathematics are, mainly, computer science journals and
conferences: JAR, ITP, IJCAR, CPP, CICM, and others.

Nevertheless, the discussion about the subject in central mathematical
circles is increasing; there were some hints on the ICM2018 panel on
“machine-assisted” proofs
\cite{https://doi.org/10.48550/arxiv.1809.08062} and a lively
promotion by Kevin Buzzard, during his ICM2022 special plenary lecture
\cite{2021arXiv211211598B}.

%% These assistants provide several dialects, among which we single out:
%% \begin{enumerate}
%% \item Procedural: Useful for exploration/research.
%% \item Declarative: Only one that can be read by humans!
%% \end{enumerate}

Before we start an in-depth discussion, a point should be made clear:
A formalized proof is not the same as an \emph{automatic proof}. The
reader surely understands that, aside from trivial results, no current
technology allows us to state a reasonably complex (and correct) theorem
statement and expect to obtain a proof after hitting ``Enter'', at
least not after a humanly feasible wait. On the other hand, it is
quite possible that the same reader has some mental image that
formalizing a proof requires making each application of Modus Ponens
explicit.

The fact is that \emph{proof assistants} are designed for the human prover to
be able to decompose a statement to be proved into smaller subgoals
which can actually be fed into some automatic tool. The balance between
what these tools are able to handle is not  easily appreciated by
intuition: Sometimes, ``trivial'' steps are not solved by them, which
can result in obvious frustration; but they would quickly solve some
goals that do not look like a ``mere computation.''

To appreciate the extent of mathematics formalizable, it is convenient to recall
some major successful projects, such as the Four Color Theorem
\cite{MR2463991}, the Odd Order Theorem
\cite{10.1007/978-3-642-39634-2_14}, and the proof the Kepler's
Conjecture \cite{MR3659768}. There is a vast mathematical corpus at
the Archive of Formal Proofs (AFP) based on Isabelle; and formalizations of
brand new and complex objects like the definition of perfectoid spaces \cite{10.1145/3372885.3373830}
and the Liquid Tensor Experiment \cite{LTE2020,LTE2021} have been
achieved using Lean.

We'll continue our description of proof assistants in
Section~\ref{sec:proof-assist-isabelle}. We kindly invite the reader
to enrich the previous exposition by reading the apt summary by
A.~Koutsoukou-Argyraki \cite{angeliki} and the interviews
therein; some of the experts consulted have also discussed
in \cite{2022arXiv220704779B} the status of formalized versus standard
proof in mathematics.

\subsection{Our achievements}
We formalized a model-theoretic rendition of forcing (Sect.~\ref{sec:forcing}), showing how to
construct proper extensions of ctms of $\ZF$ (respectively, with
$\AC$), and we formalized the basic forcing notions required to obtain
ctms of $\ZFC + \neg\CH$ and of $\ZFC + \CH$ (Sect.~\ref{sec:models-ch-negation}). No metatheoretic issues
(consistency, FOL calculi, etc) were formalized, so we were mainly
concerned with the mathematics of forcing. Nevertheless, by inspecting
the foundations underlying our proof assistant Isabelle
(Section~\ref{sec:isabelle-metalogic-meta}) it can be stated that our
formalization is a bona fide proof in $\ZF$ of the previous
constructions.

In order to reach our goals, we provided basic results that were
missing from Isabelle's $\ZF$ library, starting from ones
involving cardinal successors, countable sets, etc.
(Section~\ref{sec:extension-isabellezf}). We also extended the treatment of relativization of
set-theoretical concepts (Section~\ref{sec:tools-relativization}).
%% We redesigned Isabelle/ZF results on non-absolute concepts to work
%% relative to a class.

One added value that is obtained from the present formalization is
that we identified a handful of instances of Replacement which are
sufficient to set the forcing machinery up
(Section~\ref{sec:repl-instances}), on the basis of Zermelo set theory.
The eagerness to obtain this level of detail might be a consequence of
“an unnatural tendency to investigate, for the most part, trivial
minutiae of the formalism” on our part, as it was put by Cohen
\cite{zbMATH02012060}, but we'd rather say that we were driven by
curiosity.

The latest version of our formalization can be accessed at the
development site of the AFP, via the following link:
\begin{center}
  \url{https://devel.isa-afp.org/entries/Independence_CH.html}
\end{center}
%%% Local Variables: 
%%% mode: latex
%%% TeX-master: "independence_ch_isabelle"
%%% ispell-local-dictionary: "american"
%%% End: 

 
\section{Proof assistants and Isabelle/ZF}
\label{sec:proof-assist-isabelle}

Proof assistants provide diverse aids for the task
of formalizing a piece of mathematics. They are usually implemented
using a typed programming language; rigor is enforced by defining a
type of ``theorems'', whose members can only be constructed using
operations stipulated in a small \emph{kernel} which encodes the
underlying foundation of the assistant. Further developments interact
with the type of theorems only through the kernel, and thus the latter
is the only ``trusted'' part of the assistant's code.

Several of the more established assistants (HOL Light, Coq, Isabelle)
are programmed in some variant of the ML language (which was
designed for this purpose); the newer Lean, on the other
hand, was originally conceived as a standalone functional programming
language with all the features of a standard assistant.

\subsection{Four layers}

In the case of Isabelle, Standard ML is the deeper layer of the
assistant, in which the kernel is written.

The logical foundation of Isabelle is an intuitionistic fragment of
higher-order logic (or simple type theory) called $\Meta$;
original version was described in \cite{Paulson1989}, and the addition
of  sorts appears in \cite{Nipkow-LF-91}.

The only predefined type is $\prop$ (“propositions”); new base types
can be postulated when defining objects logics. Types of higher order can be
assembled using the function space constructor $\fun$.

The type of propositions $\prop$ equipped with a binary operation
$\implies$ (“meta-implication”) that is used to represent the object
logic rules. As an example, the axiomatization of first-order logic
postulates a type $\tyo$ of booleans, and Modus Ponens
% https://isabelle.in.tum.de/dist/library/FOL/FOL/IFOL.html#IFOL.mp|axiom
is written as
\[
  [P\limp Q] \implies ([P] \implies [Q]).
\]
The square brackets represent an injection from $\tyo$ into
$\prop$. % ($[P]$ can be read as “$P$ holds”)
A consequence of this representation is that every
formula of the object logic appears as atomic for $\Meta$.

%% Quantification is handled in the meta-level using a functional $\ALL$
%% with polymorphic type $(\alpha \fun \prop) \fun \prop$. 
Types in Isabelle are organized into \emph{classes} and \emph{sorts};
for ease of exposition, we will omit the former.  The axiomatization
of first-order logic postulates a sort $\type{term}$ (of
“individuals”, or elements of a first-order universe of discourse) and
stipulates that every further type variable $\alpha$ must be of that
sort. In particular, Isabelle/ZF only postulates one new type $\tyi$
(“sets”) of sort $\type{term}$. Hence, from the type of the universal
quantifier functional $\forall :: (\alpha \fun \tyo) \fun \tyo$ it
follows that it may only be applied to predicates with a variable of
type $\tyi$. This ensures that the object logic is effectively
first-order.

Paulson  \cite{Paulson1989} carried out a proof that the encoding
$\Meta_{\mathrm{IFOL}}$ of
intuitionistic first-order logic without equality in the original $\Meta$ is
conservative (there is a correspondence between provable $\phi$ in
IFOL and provable $[\phi]$ in $\Meta_{\mathrm{IFOL}}$) by putting
$\Meta_{\mathrm{IFOL}}$ proofs in \emph{expanded normal form}
\cite{MR0387024}. Passing to classical logic does not presents
difficulties, but the treatment of meta-equality differs between the
original and the present incarnation of $\Meta$; details are
exhaustively expounded in the recent formalization by Nipkow and
Roßkof \cite{10.1007/978-3-030-79876-5_6}.


%% Isabelle \cite{Isabelle,DBLP:books/sp/Paulson94} is a general proof
%% assistant based on fragment of higher-order logic called
%% \emph{Pure}. 
%% The results presented in this work are theorems of a
%% version of $\ZF$ set theory (without the Axiom of Choice, $\AC$) 
%% called \emph{Isabelle/ZF}, which is one of the
%% ``object logics'' that can be defined on top of Pure (which is then
%% used as a language to define rules). Isabelle/ZF defines types
%% \isatt{i} and \isatt{o} for sets and Booleans, resp., and the $\ZF$
%% axioms are written down as terms of type \isatt{o}.
%% 
%% It should be noted that Pure is a very weak framework and has no
%% induction/recursion capabilities. So the only way to define functions
%% by recursion is inside the object logic. (This works the same for
%% Isabelle/HOL.) For this reason, to define the relation of forcing, we
%% needed to resort to \emph{internalized} first-order formulas: they
%% form a recursively defined set \isatt{formula}. For example, the
%% predicate of satisfaction
%% \isatt{sats::i{\isasymRightarrow}i{\isasymRightarrow}i{\isasymRightarrow}o}
%% (written $M,\mathit{ms}\models\phi$ for a set $M$,
%% $\mathit{ms}\in\isatt{list}(M)$ and $\phi\in\formula$)
%% had already been defined by recursion in Paulson~\cite{paulson_2003}.

%%% Local Variables: 
%%% mode: latex
%%% TeX-master: "independence_ch_isabelle"
%%% ispell-local-dictionary: "american"
%%% End: 

 
\section{Relative versions of non-absolute concepts}
\label{sec:relat-vers-non-absol}

The treatment of relativization/internalization described in the
previous section was enough for the Paulson's treatment of
constructibility. This is the case because essentially all the
concepts in the way of proving the consistency of $\AC$ are
absolute, and treatment of relational versions and relativized notions
could be minimized after proving the relevant absoluteness results:
For example, the lemma \isatt{Union{\uscore}abs},
\[
  M(A) \implies M(z) \implies \isatt{big{\uscore}union}(M, A, z) \longleftrightarrow z = \union
  A
\]
proved under the assumption that $M$ is transitive nonempty. Working
in this relational 
way with powersets, cardinalities, and the like would be
unfeasible. As such, cardinal arithmetic was not put in relative form
in \session{ZF-Constructible}.

\subsection{Tools for relativization}
\label{sec:tools-relativization}
In order to cope with this, we added the missing step from the
literature consisting of relative versions of the various non-absolute
functions and programmed in ML limited automatic facilities that
define the needed concepts, and  state and
prove the requisite lemmas. For instance, the 
$\isatt{cardinal}::\tyi \fun \tyi$ function is defined in
\session{Isabelle/ZF}, and the commands
\begin{isabelle}
  \isacommand{relativize}\isamarkupfalse%
  \ \isakeyword{functional}\ {\isachardoublequoteopen}cardinal{\isachardoublequoteclose}\ {\isachardoublequoteopen}cardinal{\isacharunderscore}{\kern0pt}rel{\isachardoublequoteclose}\ \isakeyword{external}\isanewline
  \isacommand{relationalize}\isamarkupfalse%
  \ {\isachardoublequoteopen}cardinal{\isacharunderscore}{\kern0pt}rel{\isachardoublequoteclose}\ {\isachardoublequoteopen}is{\isacharunderscore}{\kern0pt}cardinal{\isachardoublequoteclose}\isanewline
  \isacommand{synthesize}\isamarkupfalse%
  \ {\isachardoublequoteopen}is{\isacharunderscore}{\kern0pt}cardinal{\isachardoublequoteclose}\ \isakeyword{from{\isacharunderscore}{\kern0pt}definition}\ \isakeyword{assuming}\ {\isachardoublequoteopen}nonempty{\isachardoublequoteclose}%
\end{isabelle}
define the relative cardinal function
$\isatt{cardinal{\uscore}rel}::(\tyi \fun \tyo) \fun \tyi \fun\tyi$
(denoted  $|\cdot|^M$, as expected),
its relational version $\isatt{is{\uscore}cardinal}$, the
internalized formula \isatt{is{\uscore}cardinal{\uscore}fm} whose
satisfaction by a set is equivalent to the relational version, and
prove the previous statement (analogous to (\ref{eq:sats_big_union_fm})).
The proof that $\isatt{is{\uscore}cardinal}(M,x,z)$  encodes the
statement $|x|^M = z$ must still be done by hand, since the definition
of $\isatt{cardinal{\uscore}rel}$ already involves some tacit
absoluteness result (“\textit{the least $\ka \in \Ord$ such that \dots}” instead
of “\textit{the least $\ka \in \Ord^M$ such that \dots}”, and the like).

\subsection{Extension of Isabelle/ZF}
\label{sec:extension-isabellezf}
We extended \cite{Delta_System_Lemma-AFP} the material formalized in
Isabelle, from basic results involving function spaces and the
definition of cardinal exponentiation, to a treatment of cofinality
and the Delta System Lemma for $\omega_1$-families. We also included a
concise treatment of the axiom of Dependent Choices $\DC$ and the
general version of Rasiowa-Sikorski Lemma \cite{2018arXiv180705174G}
and a choiceless one for countable preorders.

This material was subsequently put in relative form in our formal
development on transitive class models \cite{Transitive_Models-AFP}
using as an aid the tools from
Section~\ref{sec:tools-relativization}. We also relativized many
original theories appearing in \session{ZF}, including the
fundamentals of cardinal arithmetic, the cumulative hierarchy, and the
definition of the $\ale{}$ function.


%%% Local Variables:
%%% mode: latex
%%% TeX-master: "independence_ch_isabelle"
%%% ispell-local-dictionary: "american"
%%% End: 


%%%%%%%%%%%%%%%%%%%%%%%%%%%%%%%%%%%%%%%%%%%%%%%%%%%%%%%%%%%%%%%%%%%%%%          
\section{Forcing}
\label{sec:forcing}
Let $\lb \PP, {\preceq} ,\1\rb \in M$ be a forcing notion. Given $G\sbq \PP$, we have
$M[G]\defi \{ \val(\PP,G,\punto{a}) : \punto{a}\in M \}$.

The following form of the Forcing Theorems  is the one
that we formalized.
\begin{theorem}
  There exists a function  $\forceisa:: \tyi \fun  \tyi$
  such that for every
  $\phi\in\formula$ and $\punto{a}_0,\dots,\punto{a}_n\in M$,
  \begin{enumerate}
  \item (Definability) $\forceisa(\phi)\in\formula$;
  \item (Truth Lemma) for every $M$-generic $G$,
    \[
      M[G], [\val(\PP,G,\punto{a}_0),\dots,\val(\PP,G,\punto{a}_n)]
      \models \phi
    \]
    is equivalent to 
    \[
      \exists p\in G.\ \; M, [p,\PP,\preceq,\1, \punto{a}_0,\dots,\punto{a}_n]  \models
      \forceisa(\phi).\]

    We use the notation $p \forces
    \phi\ [\punto{a}_0,\dots,\punto{a}_n]$ for this last assertion.
  \item (Density Lemma) $p \forces \phi\ [\punto{a}_0,\dots,\punto{a}_n]$
    if and only if 
    $\{q\in \PP :  q \forces \phi\ [\punto{a}_0,\dots,\punto{a}_n]\}$
    is dense below $p$.
  \end{enumerate}
\end{theorem}

We followed the new Kunen's book to define
$\forceisa$.  Forcing for atomic formulas is described as a mutual
recursion
%% \begin{multline*}
%%   \forceseq (p,t_1,t_2) \defi 
%%   \forall s\in\dom(t_1)\cup\dom(t_2).\ \forall q\pleq p .\\
%%   \forcesmem(q,s,t_1)\lsii \forcesmem(q,s,t_2)
%% \end{multline*}
%% \begin{multline*}
%%   \forcesmem(p,t_1,t_2) \defi  \forall v\pleq p. \ \exists q\pleq v.\\  
%%   \exists s.\ \exists r\in \PP .\ \lb s,r\rb \in  t_2 \land q
%%   \pleq r \land \forceseq(q,t_1,s)
%% \end{multline*}
but then \cite[p.~257]{kunen2011set} it is cast as a single
recursively defined function $\frcat$ over the wellfounded relation
$\isatt{frecR}$ on tuples $\lb \mathit{ft},t_1,t_2,p\rb$ (where
$\mathit{ft}\in\{0,1\}$ indicates the type of the atomic formula being
forced). Forcing for general formulas is defined by recursion on the
datatype $\formula$. Details on the implementation and proofs of the
Forcing Theorems have been spelled out in our
\cite{2020arXiv200109715G}.


It is to be noted that application of the Forcing theorems do not
require any extra Replacement instances on $M$.

%%% Local Variables: 
%%% mode: latex
%%% TeX-master: "independence_ch_isabelle"
%%% ispell-local-dictionary: "american"
%%% End: 


\section{A sample formal proof}
\label{sec:sample-formal-proof}

We present a fragment of the formal version of the proof that the
Powerset Axiom holds in a generic extension, which also serves to
illustrate the Isar dialect of Isabelle.

We quote the relevant
paragraph of Kunen's \cite[Thm.~IV.2.27]{kunen2011set}:
\begin{quote}
  For Power Set (similarly to Union above), it is sufficient to prove
  that whenever $a \in M[G]$, there is a $b \in M[G]$ such that
  $\mathcal{P}(a) \cap M[G] \subseteq b$. Fix $\tau \in
  M^{\mathbb{P}}$ such that $\tau_{G}=a$. Let
  $Q=(\mathcal{P}(\operatorname{dom}(\tau) \times
  \mathbb{P}))^{M}$. This is the set of all names $\vartheta \in
  M^{\mathbb{P}}$ such that $\operatorname{dom}(\vartheta) \subseteq
  \operatorname{dom}(\tau)$. Let $\pi=Q \times\{\1\}$ and let
  $b=\pi_{G}=$ $\left\{\vartheta_{G}: \vartheta \in Q\right\}$. Now,
  consider any $c \in \mathcal{P}(a) \cap M[G]$; we need to show that
  $c \in b$. Fix $\chi \in M^{\mathbb{P}}$ such that
  $\chi_{G}=c$, and let $\vartheta=\{\langle\sigma, p\rangle:
  \sigma \in \operatorname{dom}(\tau) \wedge p \Vdash \sigma \in
  \chi\}$; $\vartheta \in M$ by the Definability Lemma. Since
  $\vartheta \in Q$, we are done if we can show that
  $\vartheta_{G}=c$.
\end{quote}
The assumption $a\in M[G]$ appears in the lemma statement, and the
goal involving $b$ in the first sentence will appear below (signaled
by “{\small (**)}”); formalized
material necessarily tends to be much more linear than usual prose. In
what follows, we
will intersperse the relevant passages of the proof.
\begin{isabelle}
\isacommand{lemma}\isamarkupfalse%
\ Pow{\isacharunderscore}{\kern0pt}inter{\isacharunderscore}{\kern0pt}MG{\isacharcolon}{\kern0pt}\isanewline
\ \ \isakeyword{assumes}\ {\isachardoublequoteopen}a{\isasymin}M{\isacharbrackleft}{\kern0pt}G{\isacharbrackright}{\kern0pt}{\isachardoublequoteclose}\isanewline
\ \ \isakeyword{shows}\ {\isachardoublequoteopen}Pow{\isacharparenleft}{\kern0pt}a{\isacharparenright}{\kern0pt}\ {\isasyminter}\ M{\isacharbrackleft}{\kern0pt}G{\isacharbrackright}{\kern0pt}\ {\isasymin}\ M{\isacharbrackleft}{\kern0pt}G{\isacharbrackright}{\kern0pt}{\isachardoublequoteclose}\isanewline
%
\isacommand{proof}\isamarkupfalse%
\ {\isacharminus}{\kern0pt}
\end{isabelle}
\textit{Fix $\tau \in  M^{\mathbb{P}}$ such that $\tau_{G}=a$.}
\begin{isabelle}
\ \ \isacommand{from}\isamarkupfalse%
\ assms\isanewline
\ \ \isacommand{obtain}\isamarkupfalse%
\ {\isasymtau}\ \isakeyword{where}\ {\isachardoublequoteopen}{\isasymtau}\ {\isasymin}\ M{\isachardoublequoteclose}\ {\isachardoublequoteopen}val{\isacharparenleft}{\kern0pt}G{\isacharcomma}{\kern0pt}\ {\isasymtau}{\isacharparenright}{\kern0pt}\ {\isacharequal}{\kern0pt}\ a{\isachardoublequoteclose}\isanewline
\ \ \ \ \isacommand{using}\isamarkupfalse%
\ GenExtD\ \isacommand{by}\isamarkupfalse%
\ auto
\end{isabelle}
\textit{Let
  $Q=(\mathcal{P}(\operatorname{dom}(\tau) \times
  \mathbb{P}))^{M}$. This is the set of all names $\vartheta \in
  M^{\mathbb{P}}$} [\dots]%% ---it is pretty laborious to show that things
%% are in $M$; we omit 17 lines of code to that effect,
%% that also apply previously proved lemmas.
\begin{isabelle}
\ \ \isacommand{let}\isamarkupfalse%
\ {\isacharquery}{\kern0pt}Q{\isacharequal}{\kern0pt}{\isachardoublequoteopen}Pow\isactrlbsup M\isactrlesup {\isacharparenleft}{\kern0pt}domain{\isacharparenleft}{\kern0pt}{\isasymtau}{\isacharparenright}{\kern0pt}{\isasymtimes}{\isasymbbbP}{\isacharparenright}{\kern0pt}{\isachardoublequoteclose}
\end{isabelle}
\textit{ Let $\pi=Q \times\{\1\}$ and let
  $b=\pi_{G}=$ $\left\{\vartheta_{G}: \vartheta \in Q\right\}$.}
\begin{isabelle}
\ \ \isacommand{let}\isamarkupfalse%
\ {\isacharquery}{\kern0pt}{\isasympi}{\isacharequal}{\kern0pt}{\isachardoublequoteopen}{\isacharquery}{\kern0pt}Q{\isasymtimes}{\isacharbraceleft}{\kern0pt}{\isasymone}{\isacharbraceright}{\kern0pt}{\isachardoublequoteclose}\isanewline
\ \ \isacommand{let}\isamarkupfalse%
\ {\isacharquery}{\kern0pt}b{\isacharequal}{\kern0pt}{\isachardoublequoteopen}val{\isacharparenleft}{\kern0pt}G{\isacharcomma}{\kern0pt}{\isacharquery}{\kern0pt}{\isasympi}{\isacharparenright}{\kern0pt}{\isachardoublequoteclose}
\end{isabelle}
(Recall: \textit{\dots there is a $b\in M[G]$ such that\dots})
\begin{isabelle}
\ \ \isacommand{from}\isamarkupfalse%
\ {\isacartoucheopen}{\isasymtau}{\isasymin}M{\isacartoucheclose}\isanewline
\ \ \isacommand{have}\isamarkupfalse%
\ {\isachardoublequoteopen}domain{\isacharparenleft}{\kern0pt}{\isasymtau}{\isacharparenright}{\kern0pt}{\isasymtimes}{\isasymbbbP}\ {\isasymin}\ M{\isachardoublequoteclose}\ {\isachardoublequoteopen}domain{\isacharparenleft}{\kern0pt}{\isasymtau}{\isacharparenright}{\kern0pt}\ {\isasymin}\ M{\isachardoublequoteclose}\isanewline
\ \ \ \ \isacommand{by}\isamarkupfalse%
\ simp{\isacharunderscore}{\kern0pt}all\isanewline
\ \ \isacommand{then}\isamarkupfalse%
\isanewline
\ \ \isacommand{have}\isamarkupfalse%
\ {\isachardoublequoteopen}{\isacharquery}{\kern0pt}b\ {\isasymin}\ M{\isacharbrackleft}{\kern0pt}G{\isacharbrackright}{\kern0pt}{\isachardoublequoteclose}\isanewline
\ \ \ \ \isacommand{by}\isamarkupfalse%
\ {\isacharparenleft}{\kern0pt}auto\ intro{\isacharbang}{\kern0pt}{\isacharcolon}{\kern0pt}GenExtI{\isacharparenright}{\kern0pt}
\end{isabelle}
\textit{Now,
  consider any $c \in \mathcal{P}(a) \cap M[G]$; we need to show that
  $c \in b$.}
\begin{isabelle}
  \label{goal-on-b}
\ \ \isacommand{have}\isamarkupfalse%
\ {\isachardoublequoteopen}Pow{\isacharparenleft}{\kern0pt}a{\isacharparenright}{\kern0pt}\ {\isasyminter}\ M{\isacharbrackleft}{\kern0pt}G{\isacharbrackright}{\kern0pt}\ {\isasymsubseteq}\ {\isacharquery}{\kern0pt}b{\isachardoublequoteclose}\hfill
\mbox{\rm\small(**)}\isanewline
\ \ \isacommand{proof}\isamarkupfalse%
\isanewline
\ \ \ \ \isacommand{fix}\isamarkupfalse%
\ c\isanewline
\ \ \ \ \isacommand{assume}\isamarkupfalse%
\ {\isachardoublequoteopen}c\ {\isasymin}\ Pow{\isacharparenleft}{\kern0pt}a{\isacharparenright}{\kern0pt}\ {\isasyminter}\ M{\isacharbrackleft}{\kern0pt}G{\isacharbrackright}{\kern0pt}{\isachardoublequoteclose}
\end{isabelle}
\textit{Fix $\chi \in M^{\mathbb{P}}$ such that
  $\chi_{G}=c$,}
\begin{isabelle}
\ \ \ \ \isacommand{then}\isamarkupfalse%
\isanewline
\ \ \ \ \isacommand{obtain}\isamarkupfalse%
\ {\isasymchi}\ \isakeyword{where}\ {\isachardoublequoteopen}c{\isasymin}M{\isacharbrackleft}{\kern0pt}G{\isacharbrackright}{\kern0pt}{\isachardoublequoteclose}\ {\isachardoublequoteopen}{\isasymchi}\ {\isasymin}\ M{\isachardoublequoteclose}\ {\isachardoublequoteopen}val{\isacharparenleft}{\kern0pt}G{\isacharcomma}{\kern0pt}{\isasymchi}{\isacharparenright}{\kern0pt}\ {\isacharequal}{\kern0pt}\ c{\isachardoublequoteclose}\isanewline
\ \ \ \ \ \ \isacommand{using}\isamarkupfalse%
\ GenExt{\isacharunderscore}{\kern0pt}iff\ \isacommand{by}\isamarkupfalse%
\ auto
\end{isabelle}
\textit{and let $\vartheta=\{\langle\sigma, p\rangle:
  \sigma \in \operatorname{dom}(\tau) \wedge p \Vdash \sigma \in
  \chi\}$;}
\begin{isabelle}
\ \ \ \ \isacommand{let}\isamarkupfalse%
\ {\isacharquery}{\kern0pt}{\isasymtheta}{\isacharequal}{\kern0pt}{\isachardoublequoteopen}{\isacharbraceleft}{\kern0pt}{\isasymlangle}{\isasymsigma}{\isacharcomma}{\kern0pt}p{\isasymrangle}\ {\isasymin}domain{\isacharparenleft}{\kern0pt}{\isasymtau}{\isacharparenright}{\kern0pt}{\isasymtimes}\isasymbbbP\ {\isachardot}{\kern0pt}\ p\ {\isasymtturnstile}\ {\isasymcdot}{\isadigit{0}}\ {\isasymin}\ {\isadigit{1}}{\isasymcdot}\ {\isacharbrackleft}{\kern0pt}{\isasymsigma}{\isacharcomma}{\kern0pt}{\isasymchi}{\isacharbrackright}{\kern0pt}\ {\isacharbraceright}{\kern0pt}{\isachardoublequoteclose}
\end{isabelle}
\textit{$\vartheta \in M$ by the Definability Lemma.}
\begin{isabelle}
\ \ \ \ \isacommand{have}\isamarkupfalse%
\ {\isachardoublequoteopen}arity{\isacharparenleft}{\kern0pt}forces{\isacharparenleft}{\kern0pt}\ {\isasymcdot}{\isadigit{0}}\ {\isasymin}\ {\isadigit{1}}{\isasymcdot}\ {\isacharparenright}{\kern0pt}{\isacharparenright}{\kern0pt}\ {\isacharequal}{\kern0pt}\ {\isadigit{6}}{\isachardoublequoteclose}\isanewline
\ \ \ \ \ \ \isacommand{using}\isamarkupfalse%
\ arity{\isacharunderscore}{\kern0pt}forces{\isacharunderscore}{\kern0pt}at\ \isacommand{by}\isamarkupfalse%
\ auto\isanewline
\ \ \ \ \isacommand{with}\isamarkupfalse%
\ {\isacartoucheopen}domain{\isacharparenleft}{\kern0pt}{\isasymtau}{\isacharparenright}{\kern0pt}\ {\isasymin}\ M{\isacartoucheclose}\ {\isacartoucheopen}{\isasymchi}\ {\isasymin}\ M{\isacartoucheclose}\isanewline
\ \ \ \ \isacommand{have}\isamarkupfalse%
\ {\isachardoublequoteopen}{\isacharquery}{\kern0pt}{\isasymtheta}\ {\isasymin}\ M{\isachardoublequoteclose}\isanewline
\ \ \ \ \ \ \isacommand{using}\isamarkupfalse%
\ sats{\isacharunderscore}{\kern0pt}fst{\isacharunderscore}{\kern0pt}snd{\isacharunderscore}{\kern0pt}in{\isacharunderscore}{\kern0pt}M\isanewline
\ \ \ \ \ \ \isacommand{by}\isamarkupfalse%
\ simp\end{isabelle}
\textit{Since
  $\vartheta \in Q$,}
\begin{isabelle}
\ \ \ \ \isacommand{with}\isamarkupfalse%
\ {\isacartoucheopen}domain{\isacharparenleft}{\kern0pt}{\isasymtau}{\isacharparenright}{\kern0pt}{\isasymtimes}{\isasymbbbP}\ {\isasymin}\ M{\isacartoucheclose}\isanewline
\ \ \ \ \isacommand{have}\isamarkupfalse%
\ {\isachardoublequoteopen}{\isacharquery}{\kern0pt}{\isasymtheta}\ {\isasymin}\ {\isacharquery}{\kern0pt}Q{\isachardoublequoteclose}\isanewline
\ \ \ \ \ \ \isacommand{using}\isamarkupfalse%
\ Pow{\isacharunderscore}{\kern0pt}rel{\isacharunderscore}{\kern0pt}char\ \isacommand{by}\isamarkupfalse%
\ auto
\end{isabelle}
\textit{we are done if we can show that
  $\vartheta_{G}=c$.}
\begin{isabelle}
\ \ \ \ \isacommand{have}\isamarkupfalse%
\ {\isachardoublequoteopen}val{\isacharparenleft}{\kern0pt}G{\isacharcomma}{\kern0pt}{\isacharquery}{\kern0pt}{\isasymtheta}{\isacharparenright}{\kern0pt}\ {\isacharequal}{\kern0pt}\ c{\isachardoublequoteclose}\isanewline
%
\ \ \ \ \isacommand{proof} \ \mbox{$[\dots]$}
\end{isabelle}

This cherry-picked example shows that the formalization can be close
to the mathematical exposition and might be useful to reconstruct the
proof from the book; nonetheless, it also has significantly more
details than the mathematical prose, even with some indications to
direct the automatic tools.

There has been some progress on assistants where one writes statements
and proofs in natural language; recently P.~Koepke and his team
achieved magnificent results by using Isabelle/Naproche
\cite{10.1007/978-3-030-81097-9_2} to formalize proofs of several
results (particularly, the proof of König's Theorem). The input
language of Isabelle/Naproche is a \emph{controlled} natural language
that presents the result being formalized as a deduction in
first-order logic, where every assumption and the ``whole logical
scenario'' are explicitly given. From the input language,
Isabelle/Naproche builds ``proof tasks'' that are handled to automatic
theorem provers. As far as we can tell, Isabelle/Naproche is
promising but still unsuitable for a project of the magnitude of ours.

% Undoubtedly, even for this cherry-picked example, the formalization
% looks a bit codish. It is therefore inevitable to compare this to the
% magnificent results obtained by P.~Koepke and his team by using
% Isabelle/Naproche \cite{10.1007/978-3-030-81097-9_2} (particularly,
% the proof of König's Theorem). The trick there consists in presenting the result
% being formalized as a restricted first-order problem, and then each
% proof step can be handled by an automatic theorem prover.

%%% Local Variables: 
%%% mode: latex
%%% TeX-master: "independence_ch_isabelle"
%%% ispell-local-dictionary: "american"
%%% End: 


\section{Main achievements of the formalization}
\label{sec:main-achievements}

\subsection{A sufficient set of replacement instances}
\label{sec:repl-instances}

We isolated 22 instances of Replacement that are sufficient to force
$\CH$ or $\neg\CH$, which are enumerated below by the name of the
corresponding internalized first order formula. Many of these were already present in
relational form in the \session{ZF-Constructible} library.

The first 4 instances, collected in the subset
\isa{instances1{\uscore}fms} of \formula, consist of basic
constructions:

\begin{itemize}
\item 2 instances for transitive closure: one to prove closure under
  iteration of $X\mapsto\union X$ and an auxiliary one used to show absoluteness.
\item 1 instance to define $\in$-rank.
  %
\item 1 instance to construct the cumulative hierarchy (rank initial segments).
\end{itemize}

The next 4 instances (gathered in \isa{instances2{\uscore}fms})
are needed to set up
cardinal arithmetic in $M$:
\begin{itemize}
\item 2 instances for the definition of
  ordertypes: The relevant well-founded recursion and a technical
  auxiliary instance.
\item 2 instances for Aleph: Replacement through the ordertype function (for Hartogs' Theorem) and the well-founded recursion
  using it.
\end{itemize}

We also need a one extra replacement instance $\psi$ on $M$ for each
$\phi$ of the
previous ones to have them in $M[G]$:
\[
  \psi(x,\alpha,y_1,\dots,y_n) \defi \quine{\alpha = \min \bigl\{
    \beta \mid \exists\tau\in V_\beta.\  \mathit{snd}(x) \forces
    \phi\ [\mathit{fst}(x),\tau,y_1,\dots,y_n]\bigr\}}
\]
Here, $\mathit{fst}(\lb a,b\rb) = a$ and $\mathit{snd}(\lb a,b\rb) = b$.
% (with default value $0$ for non pairs).
The map $\phi\mapsto\psi$ is
the function $\calF$ referred to in the abstract.
All such “ground” replacement
instances appear in the locale \locale{M{\uscore}ZF3} and form the set
\isa{instances3{\uscore}fms}.

That makes 16 instances up to now. For the setup of forcing, we
require the following 3 instances, which form the set
\isa{instances{\uscore}ground{\uscore}fms}:
%
\begin{itemize}
\item Well-founded recursion to define check-names.
  %
\item Well-founded recursion for the definition of forcing for atomic formulas.
  %
\item Replacement through $x\mapsto \lb x,\check{x}\rb$ (for the
  definition of $\punto{G}$).
  %
\end{itemize}
The proof of the $\Delta$-System Lemma requires 2 instances which form the set
\isa{instances{\uscore}ground{\uscore}notCH{\uscore}fms}, that are
used for the recursive construction of sets using a choice function (as in the
construction of a wellorder of $X$ given a choice function on
$\Pow(X)$), and to show its absoluteness.

The $21$ formulas up to this point are collected into the set
\isa{overhead{\uscore}notCH} (called $\Omega$ in the abstract), which is enough to
force $\neg\CH$. To force $\CH$, we required one further instance for
the absoluteness of the recursive construction in the proof of
Dependent Choices from $\AC$. A listing with the names of all the formulas
can be found in Appendix~\ref{sec:repl-instances-appendix}.
  
The particular choice of some of the instances above arose from
Paulson's architecture on which we based our development.
This applies every time
a locale from \session{ZF-Constructible} has to be
interpreted (\locale{M{\uscore}eclose} and
\locale{M{\uscore}ordertype}, respectively, for the “auxiliary” instances).
%% For instance, the first
%% instance required for the definition of relative ordertypes arises
%% from Paulson's \session{ZF-Constructible}.
% https://isabelle.in.tum.de/dist/library/ZF/ZF-Constructible/Rank.html#offset_1123..1139

On the other hand, we replaced the original proof of the
Schröder-Bernstein Theorem by Zermelo's one
\cite[Exr. x4.27]{moschovakis1994notes}, because the former required
at least one extra instance
% (\isa{banach{\uscore}iterates{\uscore}fm})
arising from an iteration. We also managed to avoid 12 further
replacements by restructuring some of original theories in
\session{ZF-Constructible}, so these modifications are included as
part of our project.

It is to be noted that the proofs of the Forcing Theorems do not
require any extra replacement; actually, they only need the 7
instances appearing in \isa{instances1{\uscore}fms} and
\isa{instances{\uscore}ground{\uscore}fms}.  But this seems not be
the case for Separation, at least by inspecting our formalization:
More instances holding in $M$ are needed 
as the complexity of $\phi$ grows. One point where this is apparent is
in the proof of Theorem~\ref{th:forcing-thms}(\ref{item:truth-lemma}),
that appears as the \isa{truth{\uscore}lemma} in our development; it
depends on \isa{truth{\uscore}lemma'} and
\isa{truth{\uscore}lemma{\uscore}Neg}, which explicitly invoke
\isa{separation{\uscore}ax}. In any case, our intended grounds
(v.g., the transitive collapse of countable elementary submodels of a
rank initial segment $V_\alpha$ or an $H(\kappa)$) all satisfy full
Separation.


%-%-%-%-%-%-%-%-%-%-%-%-%-%-%-%-%-%-%-%-%-%-%-%-%-%-%-%-%-%-%-%-%
\subsection{Models for $\CH$ and its negation}
\label{sec:models-ch-negation}

The statements of the existence of models of $\ZFC + \neg\CH$ and of
$\ZFC + \CH$  appear in our formalization as follows:

\begin{isabelle}
\isacommand{corollary}\isamarkupfalse%
\ ctm{\isacharunderscore}{\kern0pt}ZFC{\isacharunderscore}{\kern0pt}imp{\isacharunderscore}{\kern0pt}ctm{\isacharunderscore}{\kern0pt}not{\isacharunderscore}{\kern0pt}CH{\isacharcolon}{\kern0pt}\isanewline
\ \ \isakeyword{assumes}\isanewline
\ \ \ \ {\isachardoublequoteopen}M\ {\isasymapprox}\ {\isasymomega}{\isachardoublequoteclose}\ {\isachardoublequoteopen}Transset{\isacharparenleft}{\kern0pt}M{\isacharparenright}{\kern0pt}{\isachardoublequoteclose}\ {\isachardoublequoteopen}M\ {\isasymTurnstile}\ ZFC{\isachardoublequoteclose}\isanewline
\ \ \isakeyword{shows}\isanewline
\ \ \ \ {\isachardoublequoteopen}{\isasymexists}N{\isachardot}{\kern0pt}\isanewline
\ \ \ \ \ \ M\ {\isasymsubseteq}\ N\ {\isasymand}\ N\ {\isasymapprox}\ {\isasymomega}\ {\isasymand}\ Transset{\isacharparenleft}{\kern0pt}N{\isacharparenright}{\kern0pt}\ {\isasymand}\ N\ {\isasymTurnstile}\ ZFC\ {\isasymunion}\ {\isacharbraceleft}{\kern0pt}{\isasymcdot}{\isasymnot}{\isasymcdot}CH{\isasymcdot}{\isasymcdot}{\isacharbraceright}{\kern0pt}\ {\isasymand}\isanewline
\ \ \ \ \ \ {\isacharparenleft}{\kern0pt}{\isasymforall}{\isasymalpha}{\isachardot}{\kern0pt}\ Ord{\isacharparenleft}{\kern0pt}{\isasymalpha}{\isacharparenright}{\kern0pt}\ {\isasymlongrightarrow}\ {\isacharparenleft}{\kern0pt}{\isasymalpha}\ {\isasymin}\ M\ {\isasymlongleftrightarrow}\ {\isasymalpha}\ {\isasymin}\ N{\isacharparenright}{\kern0pt}{\isacharparenright}{\kern0pt}{\isachardoublequoteclose}
\end{isabelle}

\begin{isabelle}
\isacommand{corollary}\isamarkupfalse%
\ ctm{\isacharunderscore}{\kern0pt}ZFC{\isacharunderscore}{\kern0pt}imp{\isacharunderscore}{\kern0pt}ctm{\isacharunderscore}{\kern0pt}CH{\isacharcolon}{\kern0pt}\isanewline
\ \ \isakeyword{assumes}\isanewline
\ \ \ \ {\isachardoublequoteopen}M\ {\isasymapprox}\ {\isasymomega}{\isachardoublequoteclose}\ {\isachardoublequoteopen}Transset{\isacharparenleft}{\kern0pt}M{\isacharparenright}{\kern0pt}{\isachardoublequoteclose}\ {\isachardoublequoteopen}M\ {\isasymTurnstile}\ ZFC{\isachardoublequoteclose}\isanewline
\ \ \isakeyword{shows}\isanewline
\ \ \ \ {\isachardoublequoteopen}{\isasymexists}N{\isachardot}{\kern0pt}\isanewline
\ \ \ \ \ \ M\ {\isasymsubseteq}\ N\ {\isasymand}\ N\ {\isasymapprox}\ {\isasymomega}\ {\isasymand}\ Transset{\isacharparenleft}{\kern0pt}N{\isacharparenright}{\kern0pt}\ {\isasymand}\ N\ {\isasymTurnstile}\ ZFC\ {\isasymunion}\ {\isacharbraceleft}{\kern0pt}{\isasymcdot}CH{\isasymcdot}{\isacharbraceright}{\kern0pt}\ {\isasymand}\isanewline
\ \ \ \ \ \ {\isacharparenleft}{\kern0pt}{\isasymforall}{\isasymalpha}{\isachardot}{\kern0pt}\ Ord{\isacharparenleft}{\kern0pt}{\isasymalpha}{\isacharparenright}{\kern0pt}\ {\isasymlongrightarrow}\ {\isacharparenleft}{\kern0pt}{\isasymalpha}\ {\isasymin}\ M\ {\isasymlongleftrightarrow}\ {\isasymalpha}\ {\isasymin}\ N{\isacharparenright}{\kern0pt}{\isacharparenright}{\kern0pt}{\isachardoublequoteclose}
\end{isabelle}
where $\approx$ is equipotency, and the predicate \isa{Transset}
holds for
transitive sets. Both results are proved without using Choice.

As the excerpts indicate, these results are obtained as corollaries of
two theorems in which only a subset of the aforementioned
replacement instances are assumed of the ground model. We begin the
discussion of these stronger results by
considering extensions of ctms of fragments of $\ZF$.
\begin{isabelle}
\isacommand{theorem}\isamarkupfalse%
\ extensions{\isacharunderscore}{\kern0pt}of{\isacharunderscore}{\kern0pt}ctms{\isacharcolon}{\kern0pt}\isanewline
\ \ \isakeyword{assumes}\isanewline
\ \ \ \ {\isachardoublequoteopen}M\ {\isasymapprox}\ {\isasymomega}{\isachardoublequoteclose}\ {\isachardoublequoteopen}Transset{\isacharparenleft}{\kern0pt}M{\isacharparenright}{\kern0pt}{\isachardoublequoteclose}\isanewline
\ \ \ \ {\isachardoublequoteopen}M\ {\isasymTurnstile}\ {\isasymcdot}Z{\isasymcdot}\ {\isasymunion}\ {\isacharbraceleft}{\kern0pt}{\isasymcdot}Replacement{\isacharparenleft}{\kern0pt}p{\isacharparenright}{\kern0pt}{\isasymcdot}\ {\isachardot}{\kern0pt}\ p\ {\isasymin}\ overhead{\isacharbraceright}{\kern0pt}{\isachardoublequoteclose}\isanewline
\ \ \ \ {\isachardoublequoteopen}{\isasymPhi}\ {\isasymsubseteq}\ formula{\isachardoublequoteclose}\isanewline%
\ \ \ \ {\isachardoublequoteopen}M\ {\isasymTurnstile}\ {\isacharbraceleft}{\kern0pt}\ {\isasymcdot}Replacement{\isacharparenleft}{\kern0pt}ground{\isacharunderscore}{\kern0pt}repl{\isacharunderscore}{\kern0pt}fm{\isacharparenleft}{\kern0pt}{\isasymphi}{\isacharparenright}{\kern0pt}{\isacharparenright}{\kern0pt}{\isasymcdot}\ {\isachardot}{\kern0pt}\ {\isasymphi}\ {\isasymin}\ {\isasymPhi}{\isacharbraceright}{\kern0pt}{\isachardoublequoteclose}\isanewline
\ \ \isakeyword{shows}\isanewline
\ \ \ \ {\isachardoublequoteopen}{\isasymexists}N{\isachardot}{\kern0pt}\isanewline
\ \ \ \ \ \ M\ {\isasymsubseteq}\ N\ {\isasymand}\ N\ {\isasymapprox}\ {\isasymomega}\ {\isasymand}\ Transset{\isacharparenleft}{\kern0pt}N{\isacharparenright}{\kern0pt}\ {\isasymand}\ M{\isasymnoteq}N\ {\isasymand}\isanewline
\ \ \ \ \ \ {\isacharparenleft}{\kern0pt}{\isasymforall}{\isasymalpha}{\isachardot}{\kern0pt}\ Ord{\isacharparenleft}{\kern0pt}{\isasymalpha}{\isacharparenright}{\kern0pt}\ {\isasymlongrightarrow}\ {\isacharparenleft}{\kern0pt}{\isasymalpha}\ {\isasymin}\ M\ {\isasymlongleftrightarrow}\ {\isasymalpha}\ {\isasymin}\ N{\isacharparenright}{\kern0pt}{\isacharparenright}{\kern0pt}\ {\isasymand}\isanewline
\ \ \ \ \ \ {\isacharparenleft}{\kern0pt}{\isacharparenleft}{\kern0pt}M{\isacharcomma}{\kern0pt}\ {\isacharbrackleft}{\kern0pt}{\isacharbrackright}{\kern0pt}{\isasymTurnstile}\ {\isasymcdot}AC{\isasymcdot}{\isacharparenright}{\kern0pt}\ {\isasymlongrightarrow}\ N{\isacharcomma}{\kern0pt}\ {\isacharbrackleft}{\kern0pt}{\isacharbrackright}{\kern0pt}\ {\isasymTurnstile}\ {\isasymcdot}AC{\isasymcdot}{\isacharparenright}{\kern0pt}\ {\isasymand}\isanewline
\ \ \ \ \ \ N\ {\isasymTurnstile}\ {\isasymcdot}Z{\isasymcdot}\ {\isasymunion}\ {\isacharbraceleft}{\kern0pt}\ {\isasymcdot}Replacement{\isacharparenleft}{\kern0pt}{\isasymphi}{\isacharparenright}{\kern0pt}{\isasymcdot}\ {\isachardot}{\kern0pt}\ {\isasymphi}\ {\isasymin}\ {\isasymPhi}{\isacharbraceright}{\kern0pt}{\isachardoublequoteclose}
\end{isabelle}

Here, the 7-element set \isa{overhead} is enough to construct a proper
extension. It is  the union of
\isa{instances{\isadigit{1}}{\isacharunderscore}{\kern0pt}fms} and
\isa{instances{\isacharunderscore}{\kern0pt}ground{\isacharunderscore}{\kern0pt}fms}.
Also,
\isa{{\isasymcdot}Z{\isasymcdot}} denotes Zermelo set theory and one
can use the parameter $\Phi$ to ensure those replacement instances in the extension.

In the
next theorem, the relevant set of formulas is
\isa{overhead{\isacharunderscore}{\kern0pt}notCH}, defined above in
Section~\ref{sec:repl-instances}, and \isa{ZC} denotes Zermelo set
theory plus Choice:

\begin{isabelle}
\isacommand{theorem}\isamarkupfalse%
\ ctm{\isacharunderscore}{\kern0pt}of{\isacharunderscore}{\kern0pt}not{\isacharunderscore}{\kern0pt}CH{\isacharcolon}{\kern0pt}\isanewline
\ \ \isakeyword{assumes}\isanewline
\ \ \ \ {\isachardoublequoteopen}M\ {\isasymapprox}\ {\isasymomega}{\isachardoublequoteclose}\ {\isachardoublequoteopen}Transset{\isacharparenleft}{\kern0pt}M{\isacharparenright}{\kern0pt}{\isachardoublequoteclose}\isanewline
\ \ \ \ {\isachardoublequoteopen}M\ {\isasymTurnstile}\ ZC\ {\isasymunion}\ {\isacharbraceleft}{\kern0pt}{\isasymcdot}Replacement{\isacharparenleft}{\kern0pt}p{\isacharparenright}{\kern0pt}{\isasymcdot}\ {\isachardot}{\kern0pt}\ p\ {\isasymin}\ overhead{\isacharunderscore}{\kern0pt}notCH{\isacharbraceright}{\kern0pt}{\isachardoublequoteclose}\isanewline
\ \ \ \ {\isachardoublequoteopen}{\isasymPhi}\ {\isasymsubseteq}\ formula{\isachardoublequoteclose}\isanewline
\ \ \ \ {\isachardoublequoteopen}M\ {\isasymTurnstile}\ {\isacharbraceleft}{\kern0pt}\ {\isasymcdot}Replacement{\isacharparenleft}{\kern0pt}ground{\isacharunderscore}{\kern0pt}repl{\isacharunderscore}{\kern0pt}fm{\isacharparenleft}{\kern0pt}{\isasymphi}{\isacharparenright}{\kern0pt}{\isacharparenright}{\kern0pt}{\isasymcdot}\ {\isachardot}{\kern0pt}\ {\isasymphi}\ {\isasymin}\ {\isasymPhi}{\isacharbraceright}{\kern0pt}{\isachardoublequoteclose}\isanewline
\ \ \isakeyword{shows}\isanewline
\ \ \ \ {\isachardoublequoteopen}{\isasymexists}N{\isachardot}{\kern0pt}\isanewline
\ \ \ \ \ \ M\ {\isasymsubseteq}\ N\ {\isasymand}\ N\ {\isasymapprox}\ {\isasymomega}\ {\isasymand}\ Transset{\isacharparenleft}{\kern0pt}N{\isacharparenright}{\kern0pt}\ {\isasymand}\isanewline
\ \ \ \ \ \ N\ {\isasymTurnstile}\ ZC\ {\isasymunion}\ {\isacharbraceleft}{\kern0pt}{\isasymcdot}{\isasymnot}{\isasymcdot}CH{\isasymcdot}{\isasymcdot}{\isacharbraceright}{\kern0pt}\ {\isasymunion}\ {\isacharbraceleft}{\kern0pt}\ {\isasymcdot}Replacement{\isacharparenleft}{\kern0pt}{\isasymphi}{\isacharparenright}{\kern0pt}{\isasymcdot}\ {\isachardot}{\kern0pt}\ {\isasymphi}\ {\isasymin}\ {\isasymPhi}{\isacharbraceright}{\kern0pt}\ {\isasymand}\isanewline
\ \ \ \ \ \ {\isacharparenleft}{\kern0pt}{\isasymforall}{\isasymalpha}{\isachardot}{\kern0pt}\ Ord{\isacharparenleft}{\kern0pt}{\isasymalpha}{\isacharparenright}{\kern0pt}\ {\isasymlongrightarrow}\ {\isacharparenleft}{\kern0pt}{\isasymalpha}\ {\isasymin}\ M\ {\isasymlongleftrightarrow}\ {\isasymalpha}\ {\isasymin}\ N{\isacharparenright}{\kern0pt}{\isacharparenright}{\kern0pt}{\isachardoublequoteclose}
\end{isabelle}

Finally, \isa{overhead{\isacharunderscore}{\kern0pt}CH} is the union
of \isa{overhead{\isacharunderscore}{\kern0pt}notCH} with the $\DC$
instance \isa{dc{\uscore}abs{\uscore}fm}:
\begin{isabelle}
\isacommand{theorem}\isamarkupfalse%
\ ctm{\isacharunderscore}{\kern0pt}of{\isacharunderscore}{\kern0pt}CH{\isacharcolon}{\kern0pt}\isanewline
\ \ \isakeyword{assumes}\isanewline
\ \ \ \ {\isachardoublequoteopen}M\ {\isasymapprox}\ {\isasymomega}{\isachardoublequoteclose}\ {\isachardoublequoteopen}Transset{\isacharparenleft}{\kern0pt}M{\isacharparenright}{\kern0pt}{\isachardoublequoteclose}\isanewline
\ \ \ \ {\isachardoublequoteopen}M\ {\isasymTurnstile}\ ZC\ {\isasymunion}\ {\isacharbraceleft}{\kern0pt}{\isasymcdot}Replacement{\isacharparenleft}{\kern0pt}p{\isacharparenright}{\kern0pt}{\isasymcdot}\ {\isachardot}{\kern0pt}\ p\ {\isasymin}\ overhead{\isacharunderscore}{\kern0pt}CH{\isacharbraceright}{\kern0pt}{\isachardoublequoteclose}\isanewline
\ \ \ \ {\isachardoublequoteopen}{\isasymPhi}\ {\isasymsubseteq}\ formula{\isachardoublequoteclose}\isanewline
\ \ \ \ {\isachardoublequoteopen}M\ {\isasymTurnstile}\ {\isacharbraceleft}{\kern0pt}\ {\isasymcdot}Replacement{\isacharparenleft}{\kern0pt}ground{\isacharunderscore}{\kern0pt}repl{\isacharunderscore}{\kern0pt}fm{\isacharparenleft}{\kern0pt}{\isasymphi}{\isacharparenright}{\kern0pt}{\isacharparenright}{\kern0pt}{\isasymcdot}\ {\isachardot}{\kern0pt}\ {\isasymphi}\ {\isasymin}\ {\isasymPhi}{\isacharbraceright}{\kern0pt}{\isachardoublequoteclose}\isanewline
\ \ \isakeyword{shows}\isanewline
\ \ \ \ {\isachardoublequoteopen}{\isasymexists}N{\isachardot}{\kern0pt}\isanewline
\ \ \ \ \ \ M\ {\isasymsubseteq}\ N\ {\isasymand}\ N\ {\isasymapprox}\ {\isasymomega}\ {\isasymand}\ Transset{\isacharparenleft}{\kern0pt}N{\isacharparenright}{\kern0pt}\ {\isasymand}\isanewline
\ \ \ \ \ \ N\ {\isasymTurnstile}\ ZC\ {\isasymunion}\ {\isacharbraceleft}{\kern0pt}{\isasymcdot}CH{\isasymcdot}{\isacharbraceright}{\kern0pt}\ {\isasymunion}\ {\isacharbraceleft}{\kern0pt}\ {\isasymcdot}Replacement{\isacharparenleft}{\kern0pt}{\isasymphi}{\isacharparenright}{\kern0pt}{\isasymcdot}\ {\isachardot}{\kern0pt}\ {\isasymphi}\ {\isasymin}\ {\isasymPhi}{\isacharbraceright}{\kern0pt}\ {\isasymand}\isanewline
\ \ \ \ \ \ {\isacharparenleft}{\kern0pt}{\isasymforall}{\isasymalpha}{\isachardot}{\kern0pt}\ Ord{\isacharparenleft}{\kern0pt}{\isasymalpha}{\isacharparenright}{\kern0pt}\ {\isasymlongrightarrow}\ {\isacharparenleft}{\kern0pt}{\isasymalpha}\ {\isasymin}\ M\ {\isasymlongleftrightarrow}\ {\isasymalpha}\ {\isasymin}\ N{\isacharparenright}{\kern0pt}{\isacharparenright}{\kern0pt}{\isachardoublequoteclose}
\end{isabelle}

%%% Local Variables: 
%%% mode: latex
%%% TeX-master: "independence_ch_isabelle"
%%% ispell-local-dictionary: "american"
%%% End: 


\section{Related work}
\label{sec:related-work}

%% \textbf{Reviewer's comments}
%% {\it
%%   \begin{itemize}
%%   \item There, it would be appropriate to contrast what was done in
%%     Paulson's work on constructibility with the current work on forcing.
%%   \item More to the point, the recent work by Han and van Doorn on
%%     forcing in Lean deserves more discussion.  They have gone further
%%     than the current authors, having proved the independence of the
%%     continuum hypothesis.  They prefer Boolean-valued models as being
%%     more direct in use than the authors' countable transitive models.
%%     \begin{itemize}
%%     \item Readers will want to know whether the type-theoretic approach
%%       is better/worse/just different than using Isabelle/ZF, and
%%     \item are there any benefits to the ctm approach?
%%     \item Is the type-theory encoding of ZF really accurate?
%%     \item How about comparing proofs of equivalent statements in the two
%%       approaches for length and readability?
%%     \end{itemize}
%%   \end{itemize}
%% }

There are various formalizations of Zermelo-Fraenkel set theory in
proof assistants (v.g.\ Mizar, Metamath, and recently Lean
\cite{DBLP:conf/cade/MouraKADR15}) that proceed to different levels of
sophistication. Isabelle/ZF can be regarded as a notational variant of
NGB set theory \cite[Sect.~II.10]{kunen2011set}, because the schemes
of Replacement and Separation feature higher order (free) variables
playing the role of formula variables. It cannot be proved that the
axioms thus written correspond to first order sentences. For this
reason, our relativized versions only apply to set models, where
we restrict those variables to predicates that actually come
from first order formulas. In that sense, the axioms of the locale
\isatt{M{\isacharunderscore}ZF} correspond more faithfully to the
$\ZF$ axioms.

Traditional expositions of the method of forcing
\cite{kunen2011set,Jech_Millennium} are preceded by a study of
relativization and absoluteness. For this reason, it was a natural
choice at the beginning of this project to build on top of Paulson's
formalization of constructibility on Isabelle/ZF, and that was one
of the main early reasons to work on that logic instead of, e.g., HOL
---below we discuss other reasons. In any case, our
development of forcing does not depend on constructibility
itself (in contrast to Cohen's
original presentation, in which ground models are initial segments of the
constructible hierarchy).

A natural question is whether Isabelle/HOL (with a far more solid
framework to work with given its infrastructure and automation) would
have been a better choice than Isabelle/ZF. In fact,
there are two developments of Zermelo-Fraenkel set theory available on
it: \isatt{HOLZF} by Obua \cite{DBLP:conf/ictac/Obua06} and
\isatt{ZFC{\isacharunderscore}in{\isacharunderscore}HOL} by Paulson
\cite{ZFC_in_HOL-AFP}. But these (logically equivalent) frameworks are
higher in consistency strength than Isabelle/ZF. To elaborate on this,
both ZF and HOL are axiomatized on top of Isabelle's metalogic
\emph{Pure}, which is a version of ``intuitionistic higher order
logic.'' In  \cite{Paulson1989} Paulson proves that \emph{Pure}
is sound for intuitionistic first order logic, thus it does not add
any strength to it. On top of this, the axiomatization of Isabelle/ZF
results in a system equiconsistent with $\ZFC$. On the other hand,
showing the consistency of \isatt{HOLZF} (and thus
\isatt{ZFC{\isacharunderscore}in{\isacharunderscore}HOL}) requires
assuming the consistency of $\ZFC$ plus the existence of an
inaccessible cardinal \cite[Sect.~3]{DBLP:conf/ictac/Obua06}. We note,
in contrast, that our extra running assumption of the existence of a
countable transitive model is considerably weaker (directly and
consistency-wise) than the existence of an inaccessible cardinal.

Concerning the formalization of the method of forcing, to the best of
our knowledge there is only one other that deals with forcing for
set theory: the recent \emph{Flypitch} project by Han and van Doorn
\cite{han_et_al:LIPIcs:2019:11074,DBLP:conf/cpp/HanD20}, which
includes a formalization of the independence of CH using the Lean proof
assistant. The Flypitch formalization is largely orthogonal to ours
(it is based on Boolean-valued models, which are interpreted into
type theory through a variant of the Aczel encoding of set theory),
and this precludes a direct comparison of code. But we can highlight
some conceptual differences between our development and the
corresponding fraction of Flypitch.


A first observation concerns consistency strength. The consistency of
Lean requires infinitely many inaccessibles. More precisely, let
Lean$_n$ be the theory of CiC foundations of Lean restricted to $n$
type universes.  Carneiro proved in his MSc thesis~\cite{carneiro-ms-thesis} the consistency of Lean$_n$ from $\ZFC$ plus
the existence of $n$ inaccessible cardinals. It is also reported in
Carneiro's thesis that Werner's results in
\cite{10.5555/645869.668660} can be adapted to show that Lean$_{n+2}$
proves the consistency of the latter theory.  In that sense, although
Flypitch includes proofs of unprovability results in first order
logic, the meta-theoretic machinery used to obtain them is far heavier
than the one we use to operate model-theoretically.

In second place, a formalization of forcing with general partial
orders, generic filters and  ctms has ---in our opinion--- the added value
that this approach is used in an important (perhaps the greatest)
fraction of the literature, both in exposition and in research
articles and monographs. In verifying a piece of mature mathematics as the
present one, representing the actual practice seems paramount to us.
 
Finally, as a matter of taste, one of the main benefits of using transitive
models is that many fundamental notions are absolute and thus most of
the concepts and statements can be interpreted transparently, as we
have noted before. It
also provides a very concrete way to understand generic objects: as
sets that (in the non trivial case) are provably not in the original
model; this dispels any mystical feel around this concept (contrary
to the case when the ground model is the universe of all sets). In
addition, two-valued semantics is closer to our intuition.

%%% Local Variables: 
%%% mode: latex
%%% TeX-master: "forcing_in_isabelle_zf"
%%% ispell-local-dictionary: "american"
%%% End: 


\section{Lessons}\label{sec:lessons}

\subsection{Plan before formalizing}
\begin{enumerate}
\item Ask about your project
  \url{https://mathoverflow.net/q/265435/66044}

  Disclaimer: It may be useless.
\item Antecedents: Precise enumeration of what \session{ZF-Constructible} had.

\item We should had better used predicates for the forcing posets'
  order relations (the way DSL is written), and go for class forcing
  (with $\PP$ a definable subset of $M$). The latter change seems to be
  easy, but the former doesn't.
  
  “Sometimes it felt like sculpting on marble: The fear of having
  carved too deep and hence needing to change the whole stone!”

  Note: This is in direct contradiction with “code fever”
  (\ref{sec:beware-code-fever} below).
\item Read the fine print: Foundations of your proof assistant.
\end{enumerate}

\subsection{Control your bureaucracy. Automate early}
\begin{enumerate}
\item Bureaucracy vs ML programming.
\item The “math” was already formalized on 22 November 2020.
  We finished the last goal on 22 August 2021.
  (Update: 20 November 2021 \& 28 November 2021, for CH)
\item Missing: automation of closure of models under operations.
\item Missing: basic arithmetic for dealing with arities.
\end{enumerate}

\subsection{Beware of scale factors}
\begin{enumerate}
\item It is extremely misleading when automatic tools (\isatt{simp}, \isatt{auto}, etc)
  stop working just because of the sheer size of the goal. Oftentimes,
  in math, we disregard scale issues but they must always be taken
  into account in CS.
\item Example: $\forceisa(0\in 1)$ is expandable,
  $\forceisa(\neg\neg  0\in 1)$ is not.
\item Example: Synthesis of $\forceisa$; could have been fully synthesized,
  but that was dirty “strategy”.
\item The know-how of computer scientists on this kind of engineering is
  very important
\end{enumerate}

\subsection{You might have formalized it, and still be wrong}
\begin{enumerate}
\item Example: restriction of relations.
\item Pollack, “Pollack consistency” by Wiedijk. Opacity of automated
  proofs.
\item Plot twist: You can be right without knowing. Intuition may drive proofs
  even if we are not working on what we believe we are.
\end{enumerate}

\subsection{Beware of the “Code fever”}\label{sec:beware-code-fever}
\begin{enumerate}
\item “We know that doing math is fun---formalization is like DRUGS”
\item Feeling of accomplishment after seeing your writings
  validated beyond reasonable doubt (v.g. cofinality).

\item One easily forgets about the “Power of the Board.”
\end{enumerate}

\subsection{The Devil's on the shortcuts}
\begin{enumerate}
\item
  Our proofs of the “definition of forces” (and many
  consequences) and of the lemma for “forcing a value” of function
  depend on the countability of the ground model. 
\item
  Density arguments (look for “TODO”, “general versions”).
\end{enumerate}

\subsection{Document your project}
\begin{enumerate}
\item \theory{Definitions\_Main}, thanks to Vidnyánszky.
\end{enumerate}

%%% Local Variables: 
%%% mode: latex
%%% TeX-master: "independence_ch_isabelle"
%%% ispell-local-dictionary: "american"
%%% End: 


\section{Conclusion}

TODO:
\begin{enumerate}
\item Implement \emph{Basic Set Theory (BST)} by Kunen in
  Constructible: the use of alternatively Replacement or Powerset to
  prove basic absoluteness and closure resuls.
\item Enhance the automatization of formulas
\item Develop the forcing notions to obtain the independence of $\CH$,
  along with the prerrequisite combinatorial results (v.g.\ the
  $\Delta$-system lemma).
\end{enumerate}

%%% Local Variables: 
%%% mode: latex
%%% TeX-master: "forcing_in_isabelle_zf"
%%% ispell-local-dictionary: "american"
%%% End: 

%
% ---- Bibliography ----
%
% BibTeX users should specify bibliography style 'splncs04'.
% References will then be sorted and formatted in the correct style.
%
%\bibliographystyle{splncs04}
\bibliographystyle{mi-estilo-else}
\bibliography{independence_ch_isabelle}

\appendix

\section{22 replacement instances to rule them all}
\label{sec:repl-instances-appendix}

In \isa{instances1{\uscore}fms}:

\begin{itemize}
\item Transitive closure:
  \begin{itemize}
  \item
    \sout{\isa{eclose{\uscore}repl1{\uscore}intf{\uscore}fm}.} \isa{eclose{\uscore}closed{\uscore}fm}.

    To prove closure under iteration of $X\mapsto\union X$.
  \item
    \sout{\isa{eclose{\uscore}repl2{\uscore}intf{\uscore}fm}} \isa{eclose{\uscore}abs{\uscore}fm}.

    Auxiliary instance used to show absoluteness.
  \end{itemize}
%% The instances so far
%% are needed to interpret locale
%% \locale{M{\uscore}eclose}.
\item \isa{wfrec{\uscore}rank{\uscore}fm}.
  
  For $\in$-rank.
  %
\item \sout{\isa{trans{\uscore}repl{\uscore}HVFrom{\uscore}fm}.} \isa{transrec{\uscore}VFrom{\uscore}fm}.

  For the cumulative hierarchy (rank initial segments).
\end{itemize}

%% The last two and next pair have the same syntactic structure, because
%% they are definitions by well-founded recursion.
In \isa{instances2{\uscore}fms} (for cardinal arithmetic):
\begin{itemize}
\item Ordertypes:
  \begin{itemize}
  \item
    \sout{\isa{wfrec{\uscore}replacement{\uscore}order{\uscore}pred{\uscore}fm}.}
    \isa{wfrec{\uscore}ordertype{\uscore}fm}.

    Well-founded recursion for the construction of ordertypes.
    %
  \item
    \sout{\isa{replacement{\uscore}is{\uscore}order{\uscore}eq{\uscore}map{\uscore}fm}.}
    \isa{omap{\uscore}replacement{\uscore}fm}.
    
    Auxiliary instance for the definition of ordertypes.
  \end{itemize}
\item Aleph:
  \begin{itemize}
    %
  \item
    \sout{\isa{replacement{\uscore}is{\uscore}order{\uscore}body{\uscore}fm}.}
    \isa{ordtype{\uscore}replacement{\uscore}fm}.

    Replacement through the ordertype function, for Hartogs' Theorem.
    %
  \item
    \sout{\isa{replacement{\uscore}HAleph{\uscore}wfrec{\uscore}repl{\uscore}body{\uscore}fm}.}
    \isa{wfrec{\uscore}Aleph{\uscore}fm}.

    The well-founded recursion to define Aleph.
  \end{itemize}
\end{itemize}

In \isa{instances{\uscore}ground{\uscore}fms}:

\begin{itemize}
\item \isa{wfrec{\uscore}Hcheck{\uscore}fm}.
  
  Well-founded recursion to define check-names.
  %
\item \isa{wfrec{\uscore}Hfrc{\uscore}at{\uscore}fm}.

  Well-founded recursion for the definition of forcing for atomic formulas.
  %
\item
  \sout{\isa{Lambda{\uscore}in{\uscore}M{\uscore}fm(check{\uscore}fm(2,0,1),1)}.}
  \isa{lam{\uscore}replacement{\uscore}check{\uscore}fm}.

  Replacement through $x\mapsto \lb x,\check{x}\rb$ (for the
  definition of $\punto{G}$).
  %
\end{itemize}

In 
\isa{instances{\uscore}ground{\uscore}notCH{\uscore}fms}:
\begin{itemize}
\item
  \sout{\isa{replacement{\uscore}is{\uscore}trans{\uscore}apply{\uscore}image{\uscore}fm}.}
  \isa{recursive{\uscore}construction{\uscore}fm}.

  Recursive construction of sets using a choice function.
  %
\item
  \sout{\isa{replacement{\uscore}transrec{\uscore}apply{\uscore}image{\uscore}body{\uscore}fm}.}
  \isa{recursive{\uscore}construction{\uscore}abs{\uscore}fm}.
  
  Absoluteness of the previous construction.
\end{itemize}
%
\begin{itemize}
\item
  \sout{\isa{replacement{\uscore}dcwit{\uscore}repl{\uscore}body{\uscore}fm}.}
  \isa{dc{\uscore}abs{\uscore}fm}.
  
  Absoluteness of the recursive construction in the proof of the
  Dependent Choices from $\AC$.
\end{itemize}

%%% Local Variables:
%%% mode: latex
%%% TeX-master: "independence_ch_isabelle"
%%% ispell-local-dictionary: "american"
%%% End: 


\section{Lambda replacements}\label{sec:lambda-replacements}

The development of the locale structure of the project was a dynamical
process. As further properties of closure of the ground $M$ were
required, we gather the relevant instances of Separation and
Replacement into a new locale (always assuming a class model, for
added generality), and proceded to apply them to those closure proofs.

This procedure lead to a steady grow in the number of interpretation
obligations and therefore, of formula synthesis (since the two axiom
schemes were postulated using codes for formulas). That number would
easily surpass the hundred, and the automatic tools at our disposal
for that task were rudimentary (as discussed in
Section~\ref{sec:bureaucracy-scale-factors}).

Faced with this situation, we decided that we needed some sort of
\emph{compositionality} in order to obtain instances new instances
from the ones already proved: Having Replacement for class functions
$F$ and $G$ does not entail immediately replacement under $F\circ G$
(unless you use one further instance of Separation, and the net gain
is zero). The solution was to postulate for the relevant $F$s, instead
of replacement through $x\mapsto F(x)$, a \emph{lambda replacement}
through $x\mapsto \lb x,F(x)\rb$. The name “lambda” corresponds to the
fact that this type of replacement is equivalent to closure under
$(\lambda x\in A.\ F(x)) \defi \{ \lb x,F(x)\rb : x\in A \}$ for every
$A\in M$.

Now, a fixed set of six replacements and one separation (apart from
those in \locale{M{\uscore}basic}, which also assumes the Powerset
Axiom for the class $M$) is sufficient to obtain the lambda
replacement under $x\mapsto \lb x,F(G(x))\rb$ given those for $F$ and
$G$. To obtain compositions with binary class functions $H$, it is
enough to assume the lambda replacement
$x\mapsto \lb x,H(\mathit{fst}(x),\mathit{snd}(x)))\rb$.


%%% Local Variables:
%%% mode: latex
%%% TeX-master: "independence_ch_isabelle"
%%% ispell-local-dictionary: "american"
%%% End:


\section{Discipline for relativization}
\label{sec:discipline-relativization}

As we said in Sec.~\ref{sec:relat-vers-non-absol}, in order to force
$\CH$ and its negation we depended on having relativized versions of
cardinals, Alephs, etc. It was clear for us that our efforts would be
more efficient if we set up a discipline for relativizing sets (terms
of type $\tyi$) and predicates/relations (terms of type $\tyo$).

Paulson only had, for each set, the relational version. It seemed
clearer to us to have a functional version of the relativized concept.
Going back to our example in \ref{sec:tools-relativization}, for the
concept $\isatt{cardinal}::\tyi \fun \tyi$ we want its relative
version
$\isatt{cardinal{\uscore}rel}::(\tyi \fun \tyo) \fun \tyi \fun\tyi$
and the relational version of the latter
$\isatt{is{\uscore}cardinal}::(\tyi \fun \tyo) \fun \tyi \fun \tyi
\fun \tyo$.

Our first attempt of defining a discipline was inspired by
mathematical considerations: if we might prove that
$\isatt{is{\uscore}cardinal}$ is functional and also prove the
existence of a witness $\isatt{c}$ such that $\isatt{M(c)}$ and
$\isatt{is{\uscore}cardinal(M,x,c)}$ then
$\isatt{cardinal{\uscore}rel(M,x)}$ can be obtained by the operator of
definite descriptions.

Soon we realized that resorting to definite descriptions was needed
only for the most primitive concepts. In fact, once we have a
relativized concept, we can use it to define other relativizations.
For instance, $\isatt{cardinal{\uscore}rel}$ depends on having
relative versions of $\isatt{bij}$. Instead of relationalizing
$\isatt{bij}$ to get $\isatt{is{\uscore}bij}$ and then prove
uniqueness and existence of a witness, we define
$\isatt{bij{\uscore}rel}$ using $\isatt{inj{\uscore}rel}$ and
$\isatt{surj{\uscore}rel}$.

%%% Local Variables: 
%%% mode: latex
%%% TeX-master: "independence_ch_isabelle"
%%% ispell-local-dictionary: "american"
%%% End: 


\section{Recursions in cofinality}\label{sec:recursions-cofinality}

As we mentioned near the end of
Section~\ref{sec:aims-formalization-planning}, we decided to minimize
the requirements being formalized in order to achieve our immediate
goal. In particular, the treatment of cofinality in the companion
project \cite{Delta_System_Lemma-AFP} was left behind.

We already observed that well-founded, and in particular transfinite,
recursion is not easily dealt with in Isabelle/ZF. Nevertheless, and
mainly as a curiosity, we found out that only one recursive
construction is needed for the development of the basic theory of
cofinality (as in \cite[Sect.~I.13]{kunen2011set}), which is used in
the proof of the following “factorization” lemma:

\begin{lemma}
  Let $\del,\ga\in\Ord$ and assume $f:\del\to\ga$ is cofinal.  There exists
  a strictly increasing $g:\cf(\ga)\to \del$ such that $f\circ g$ is
  strictly increasing and cofinal in $\ga$. Moreover, if $f$ is
  strictly increasing, then $g$ must also be cofinal.
\end{lemma}

It turns out that the rest of the basic results on cofinality (namely,
idempotence of $\cf$, that regular ordinals are cardinals, the
cofinality of Alephs, König's Theorem) follow easily from the previous
Lemma by “algebraic” reasoning only.

We therefore expect that the relativization of these
results be straightforward, when time permits.

%%% Local Variables:
%%% mode: latex
%%% TeX-master: "independence_ch_isabelle"
%%% ispell-local-dictionary: "american"
%%% End:


\section{Main definitions of the formalization}\label{sec:definitions_main}

This section, which appears almost verbatim as
the theory \theory{Definitions\_Main} in \cite{Independence_CH-AFP},
might be considered as the bare minimum reading requisite to
trust that our development indeed formalizes the theory of
forcing.

The reader trusting
all the libraries on which our development is based, might jump
directly to Section~\ref{sec:relative-arith}, which treats relative
cardinal arithmetic as implemented in
\isa{T{\kern0pt}r{\kern0pt}a{\kern0pt}n{\kern0pt}s{\kern0pt}i{\kern0pt}t{\kern0pt}i{\kern0pt}v{\kern0pt}e{\kern0pt}{\char`\_}{\kern0pt}M{\kern0pt}o{\kern0pt}d{\kern0pt}e{\kern0pt}l{\kern0pt}s{\kern0pt}}. But in case one wants to dive deeper, the
following sections treat some basic concepts of the ZF logic
(Section~\ref{sec:def-main-ZF}) and in the
\session{ZF-Constructible} library (Section~\ref{sec:def-main-relative})
on which our definitions are built.

\subsection{ZF\label{sec:def-main-ZF}%
}
For the basic logic ZF we restrict ourselves to just a few
concepts (for its axioms, consult Appendix~\ref{appendix:axioms}).
%
\begin{isabelle}%
bij{\isacharparenleft}{\kern0pt}A{\isacharcomma}{\kern0pt}\ B{\isacharparenright}{\kern0pt}\ {\isasymequiv}\isanewline
{\isacharbraceleft}{\kern0pt}f\ {\isasymin}\ A\ {\isasymrightarrow}\ B\ {\isachardot}{\kern0pt}\ {\isasymforall}w{\isasymin}A{\isachardot}{\kern0pt}\ {\isasymforall}x{\isasymin}A{\isachardot}{\kern0pt}\ f\ {\isacharbackquote}{\kern0pt}\ w\ {\isacharequal}{\kern0pt}\ f\ {\isacharbackquote}{\kern0pt}\ x\ {\isasymlongrightarrow}\ w\ {\isacharequal}{\kern0pt}\ x{\isacharbraceright}{\kern0pt}\ {\isasyminter}\isanewline
{\isacharbraceleft}{\kern0pt}f\ {\isasymin}\ A\ {\isasymrightarrow}\ B\ {\isachardot}{\kern0pt}\ {\isasymforall}y{\isasymin}B{\isachardot}{\kern0pt}\ {\isasymexists}x{\isasymin}A{\isachardot}{\kern0pt}\ f\ {\isacharbackquote}{\kern0pt}\ x\ {\isacharequal}{\kern0pt}\ y{\isacharbraceright}{\kern0pt}%
\end{isabelle}%
\begin{isabelle}%
A\ {\isasymapprox}\ B\ {\isasymequiv}\ {\isasymexists}f{\isachardot}{\kern0pt}\ f\ {\isasymin}\ bij{\isacharparenleft}{\kern0pt}A{\isacharcomma}{\kern0pt}\ B{\isacharparenright}{\kern0pt}%
\end{isabelle}%
\begin{isabelle}%
Transset{\isacharparenleft}{\kern0pt}i{\isacharparenright}{\kern0pt}\ {\isasymequiv}\ {\isasymforall}x{\isasymin}i{\isachardot}{\kern0pt}\ x\ {\isasymsubseteq}\ i%
\end{isabelle}%
\begin{isabelle}%
Ord{\isacharparenleft}{\kern0pt}i{\isacharparenright}{\kern0pt}\ {\isasymequiv}\ Transset{\isacharparenleft}{\kern0pt}i{\isacharparenright}{\kern0pt}\ {\isasymand}\ {\isacharparenleft}{\kern0pt}{\isasymforall}x{\isasymin}i{\isachardot}{\kern0pt}\ Transset{\isacharparenleft}{\kern0pt}x{\isacharparenright}{\kern0pt}{\isacharparenright}{\kern0pt}%
\end{isabelle}%
\begin{isabelle}%
i\ {\isacharless}{\kern0pt}\ j\ {\isasymequiv}\ i\ {\isasymin}\ j\ {\isasymand}\ Ord{\isacharparenleft}{\kern0pt}j{\isacharparenright}{\kern0pt}\isasep\isanewline%
i\ {\isasymle}\ j\ {\isasymlongleftrightarrow}\ i\ {\isacharless}{\kern0pt}\ j\ {\isasymor}\ {\isacharparenleft}i\ {\isacharequal}{\kern0pt}\ j\ {\isasymand}\ Ord{\isacharparenleft}{\kern0pt}j{\isacharparenright}{\isacharparenright}{\kern0pt}%
\end{isabelle}%
With the concepts of empty set and successor in place,%

\begin{isabelle}
\isacommand{lemma}
\ empty{\uscore}{\kern0pt}def{\isacharprime}{\kern0pt}{\isacharcolon}{\kern0pt}\ {\isachardoublequoteopen}{\isasymforall}x{\isachardot}{\kern0pt}\ x\ {\isasymnotin}\ {\isadigit{0}}{\isachardoublequoteclose}%
\isanewline
\isacommand{lemma}
\ succ{\uscore}{\kern0pt}def{\isacharprime}{\kern0pt}{\isacharcolon}{\kern0pt}\ {\isachardoublequoteopen}succ{\isacharparenleft}{\kern0pt}i{\isacharparenright}{\kern0pt}\ {\isacharequal}{\kern0pt}\ i\ {\isasymunion}\ {\isacharbraceleft}{\kern0pt}i{\isacharbraceright}{\kern0pt}{\isachardoublequoteclose}%
\end{isabelle}
%
we can define the set of natural numbers \isa{{\isasymomega}}. In the
sources, it is  defined as a fixpoint, but here we just write
its characterization as the first limit ordinal.%
\begin{isabelle}%
Ord{\isacharparenleft}{\kern0pt}{\isasymomega}{\isacharparenright}{\kern0pt}\ {\isasymand}\ {\isadigit{0}}\ {\isacharless}{\kern0pt}\ {\isasymomega}\ {\isasymand}\ {\isacharparenleft}{\kern0pt}{\isasymforall}y{\isachardot}{\kern0pt}\ y\ {\isacharless}{\kern0pt}\ {\isasymomega}\ {\isasymlongrightarrow}\ succ{\isacharparenleft}{\kern0pt}y{\isacharparenright}{\kern0pt}\ {\isacharless}{\kern0pt}\ {\isasymomega}{\isacharparenright}{\kern0pt}\isasep\isanewline%
Ord{\isacharparenleft}{\kern0pt}i{\isacharparenright}{\kern0pt}\ {\isasymand}\ {\isadigit{0}}\ {\isacharless}{\kern0pt}\ i\ {\isasymand}\ {\isacharparenleft}{\kern0pt}{\isasymforall}y{\isachardot}{\kern0pt}\ y\ {\isacharless}{\kern0pt}\ i\ {\isasymlongrightarrow}\ succ{\isacharparenleft}{\kern0pt}y{\isacharparenright}{\kern0pt}\ {\isacharless}{\kern0pt}\ i{\isacharparenright}{\kern0pt}\ {\isasymLongrightarrow}\ {\isasymomega}\ {\isasymle}\ i%
\end{isabelle}%
Then, addition and predecessor on \isa{{\isasymomega}} are inductively
characterized as follows:%
\begin{isabelle}%
m\ {\isacharplus}{\kern0pt}\isactrlsub {\isasymomega}\ succ{\isacharparenleft}{\kern0pt}n{\isacharparenright}{\kern0pt}\ {\isacharequal}{\kern0pt}\ succ{\isacharparenleft}{\kern0pt}m\ {\isacharplus}{\kern0pt}\isactrlsub {\isasymomega}\ n{\isacharparenright}{\kern0pt}\isasep\isanewline%
m\ {\isasymin}\ {\isasymomega}\ {\isasymLongrightarrow}\ m\ {\isacharplus}{\kern0pt}\isactrlsub {\isasymomega}\ {\isadigit{0}}\ {\isacharequal}{\kern0pt}\ m\isasep\isanewline\isanewline%
pred{\isacharparenleft}{\kern0pt}{\isadigit{0}}{\isacharparenright}{\kern0pt}\ {\isacharequal}{\kern0pt}\ {\isadigit{0}}\isasep\isanewline%
pred{\isacharparenleft}{\kern0pt}succ{\isacharparenleft}{\kern0pt}y{\isacharparenright}{\kern0pt}{\isacharparenright}{\kern0pt}\ {\isacharequal}{\kern0pt}\ y%
\end{isabelle}%
Lists on a set \isa{A} can be characterized by being
recursively generated from the empty list \isa{{\isacharbrackleft}{\kern0pt}{\isacharbrackright}{\kern0pt}} and the
operation \isa{Cons} that adds a new element to the left end;
the induction theorem for them shows that the characterization is
“complete”. (Mind the
\isa{\isasymlbrakk P; Q\isasymrbrakk\ \isasymLongrightarrow\ R}
abbreviation for
\isa{P\ \isasymLongrightarrow\ Q\ \isasymLongrightarrow\ R}.)

\begin{isabelle}%
{\isacharbrackleft}{\kern0pt}{\isacharbrackright}{\kern0pt}\ {\isasymin}\ list{\isacharparenleft}{\kern0pt}A{\isacharparenright}{\kern0pt}\isasep\isanewline%
{\isasymlbrakk}a\ {\isasymin}\ A{\isacharsemicolon}{\kern0pt}\ l\ {\isasymin}\ list{\isacharparenleft}{\kern0pt}A{\isacharparenright}{\kern0pt}{\isasymrbrakk}\ {\isasymLongrightarrow}\ Cons{\isacharparenleft}{\kern0pt}a{\isacharcomma}{\kern0pt}\ l{\isacharparenright}{\kern0pt}\ {\isasymin}\ list{\isacharparenleft}{\kern0pt}A{\isacharparenright}{\kern0pt}\isasep\isanewline\isanewline%
{\isasymlbrakk}x\ {\isasymin}\ list{\isacharparenleft}{\kern0pt}A{\isacharparenright}{\kern0pt}{\isacharsemicolon}{\kern0pt}\ P{\isacharparenleft}{\kern0pt}{\isacharbrackleft}{\kern0pt}{\isacharbrackright}{\kern0pt}{\isacharparenright}{\kern0pt}{\isacharsemicolon}{\kern0pt}\ {\isasymAnd}a\ l{\isachardot}{\kern0pt}\ {\isasymlbrakk}a\ {\isasymin}\ A{\isacharsemicolon}{\kern0pt}\ l\ {\isasymin}\ list{\isacharparenleft}{\kern0pt}A{\isacharparenright}{\kern0pt}{\isacharsemicolon}{\kern0pt}\ P{\isacharparenleft}{\kern0pt}l{\isacharparenright}{\kern0pt}{\isasymrbrakk}\ {\isasymLongrightarrow}\isanewline
\ \ P{\isacharparenleft}{\kern0pt}Cons{\isacharparenleft}{\kern0pt}a{\isacharcomma}{\kern0pt}\ l{\isacharparenright}{\kern0pt}{\isacharparenright}{\kern0pt}{\isasymrbrakk}
{\isasymLongrightarrow}\ P{\isacharparenleft}{\kern0pt}x{\isacharparenright}{\kern0pt}%
\end{isabelle}%
Length, concatenation, and \isa{n}th element of lists are
recursively characterized as follows.%
\begin{isabelle}%
length{\isacharparenleft}{\kern0pt}{\isacharbrackleft}{\kern0pt}{\isacharbrackright}{\kern0pt}{\isacharparenright}{\kern0pt}\ {\isacharequal}{\kern0pt}\ {\isadigit{0}}\isasep\isanewline%
length{\isacharparenleft}{\kern0pt}Cons{\isacharparenleft}{\kern0pt}a{\isacharcomma}{\kern0pt}\ l{\isacharparenright}{\kern0pt}{\isacharparenright}{\kern0pt}\ {\isacharequal}{\kern0pt}\ succ{\isacharparenleft}{\kern0pt}length{\isacharparenleft}{\kern0pt}l{\isacharparenright}{\kern0pt}{\isacharparenright}{\kern0pt}\isasep\isanewline\isanewline%
{\isacharbrackleft}{\kern0pt}{\isacharbrackright}{\kern0pt}\ {\isacharat}{\kern0pt}\ ys\ {\isacharequal}{\kern0pt}\ ys\isasep\isanewline%
Cons{\isacharparenleft}{\kern0pt}a{\isacharcomma}{\kern0pt}\ l{\isacharparenright}{\kern0pt}\ {\isacharat}{\kern0pt}\ ys\ {\isacharequal}{\kern0pt}\ Cons{\isacharparenleft}{\kern0pt}a{\isacharcomma}{\kern0pt}\ l\ {\isacharat}{\kern0pt}\ ys{\isacharparenright}{\kern0pt}\isasep\isanewline\isanewline%
nth{\isacharparenleft}{\kern0pt}{\isadigit{0}}{\isacharcomma}{\kern0pt}\ Cons{\isacharparenleft}{\kern0pt}a{\isacharcomma}{\kern0pt}\ l{\isacharparenright}{\kern0pt}{\isacharparenright}{\kern0pt}\ {\isacharequal}{\kern0pt}\ a\isasep\isanewline%
n\ {\isasymin}\ {\isasymomega}\ {\isasymLongrightarrow}\ nth{\isacharparenleft}{\kern0pt}succ{\isacharparenleft}{\kern0pt}n{\isacharparenright}{\kern0pt}{\isacharcomma}{\kern0pt}\ Cons{\isacharparenleft}{\kern0pt}a{\isacharcomma}{\kern0pt}\ l{\isacharparenright}{\kern0pt}{\isacharparenright}{\kern0pt}\ {\isacharequal}{\kern0pt}\ nth{\isacharparenleft}{\kern0pt}n{\isacharcomma}{\kern0pt}\ l{\isacharparenright}{\kern0pt}%
\end{isabelle}%
We have the usual Haskell-like notation for iterated applications
of \isa{Cons}:%
\begin{isabelle}
\isacommand{lemma}\isamarkupfalse%
\ Cons{\isacharunderscore}{\kern0pt}app{\isacharcolon}{\kern0pt}\ {\isachardoublequoteopen}{\isacharbrackleft}{\kern0pt}a{\isacharcomma}{\kern0pt}b{\isacharcomma}{\kern0pt}c{\isacharbrackright}{\kern0pt}\ {\isacharequal}{\kern0pt}\ Cons{\isacharparenleft}{\kern0pt}a{\isacharcomma}{\kern0pt}Cons{\isacharparenleft}{\kern0pt}b{\isacharcomma}{\kern0pt}Cons{\isacharparenleft}{\kern0pt}c{\isacharcomma}{\kern0pt}{\isacharbrackleft}{\kern0pt}{\isacharbrackright}{\kern0pt}{\isacharparenright}{\kern0pt}{\isacharparenright}{\kern0pt}{\isacharparenright}{\kern0pt}{\isachardoublequoteclose}%
\end{isabelle}

%
%
Relative quantifiers restrict the range of the bound variable to a
class \isa{M} of type \isa{i\ {\isasymRightarrow}\ o}; that is, a truth-valued function with
set arguments.%
\begin{isabelle}
\isacommand{lemma}\isamarkupfalse%
\ {\isachardoublequoteopen}{\isasymforall}x{\isacharbrackleft}{\kern0pt}M{\isacharbrackright}{\kern0pt}{\isachardot}{\kern0pt}\ P{\isacharparenleft}{\kern0pt}x{\isacharparenright}{\kern0pt}\ {\isasymequiv}\ {\isasymforall}x{\isachardot}{\kern0pt}\ M{\isacharparenleft}{\kern0pt}x{\isacharparenright}{\kern0pt}\ {\isasymlongrightarrow}\ P{\isacharparenleft}{\kern0pt}x{\isacharparenright}{\kern0pt}{\isachardoublequoteclose}\isanewline
\ \ \ \ \ \ {\isachardoublequoteopen}{\isasymexists}x{\isacharbrackleft}{\kern0pt}M{\isacharbrackright}{\kern0pt}{\isachardot}{\kern0pt}\ P{\isacharparenleft}{\kern0pt}x{\isacharparenright}{\kern0pt}\ {\isasymequiv}\ {\isasymexists}x{\isachardot}{\kern0pt}\ M{\isacharparenleft}{\kern0pt}x{\isacharparenright}{\kern0pt}\ {\isasymand}\ P{\isacharparenleft}{\kern0pt}x{\isacharparenright}{\kern0pt}{\isachardoublequoteclose}
\end{isabelle}
%
%
Finally, a set can be viewed (“cast”) as a class using the
following function of type \isa{i\ {\isasymRightarrow}\ i\ {\isasymRightarrow}\ o}.%
\begin{isabelle}%
{\isacharparenleft}{\kern0pt}{\isacharhash}{\kern0pt}{\isacharhash}{\kern0pt}A{\isacharparenright}{\kern0pt}{\isacharparenleft}{\kern0pt}x{\isacharparenright}{\kern0pt}\ {\isasymlongleftrightarrow}\ x\ {\isasymin}\ A%
\end{isabelle}%
\subsection{Relative concepts\label{sec:def-main-relative}%
}
A list of relative concepts (mostly from the \session{ZF-Constructible}
    library) follows next.%
\begin{isabelle}%
big{\isacharunderscore}{\kern0pt}union{\isacharparenleft}{\kern0pt}M{\isacharcomma}{\kern0pt}\ A{\isacharcomma}{\kern0pt}\ z{\isacharparenright}{\kern0pt}\ {\isasymequiv}\ {\isasymforall}x{\isacharbrackleft}{\kern0pt}M{\isacharbrackright}{\kern0pt}{\isachardot}{\kern0pt}\ x\ {\isasymin}\ z\ {\isasymlongleftrightarrow}\ {\isacharparenleft}{\kern0pt}{\isasymexists}y{\isacharbrackleft}{\kern0pt}M{\isacharbrackright}{\kern0pt}{\isachardot}{\kern0pt}\ y\ {\isasymin}\ A\ {\isasymand}\ x\ {\isasymin}\ y{\isacharparenright}{\kern0pt}%
\end{isabelle}%
\begin{isabelle}%
upair{\isacharparenleft}{\kern0pt}M{\isacharcomma}{\kern0pt}\ a{\isacharcomma}{\kern0pt}\ b{\isacharcomma}{\kern0pt}\ z{\isacharparenright}{\kern0pt}\ {\isasymequiv}\ a\ {\isasymin}\ z\ {\isasymand}\ b\ {\isasymin}\ z\ {\isasymand}\ {\isacharparenleft}{\kern0pt}{\isasymforall}x{\isacharbrackleft}{\kern0pt}M{\isacharbrackright}{\kern0pt}{\isachardot}{\kern0pt}\ x\ {\isasymin}\ z\ {\isasymlongrightarrow}\ x\ {\isacharequal}{\kern0pt}\ a\ {\isasymor}\ x\ {\isacharequal}{\kern0pt}\ b{\isacharparenright}{\kern0pt}%
\end{isabelle}%
\begin{isabelle}%
pair{\isacharparenleft}{\kern0pt}M{\isacharcomma}{\kern0pt}\ a{\isacharcomma}{\kern0pt}\ b{\isacharcomma}{\kern0pt}\ z{\isacharparenright}{\kern0pt}\ {\isasymequiv}\ 
{\isasymexists}x{\isacharbrackleft}{\kern0pt}M{\isacharbrackright}{\kern0pt}{\isachardot}{\kern0pt}\ upair{\isacharparenleft}{\kern0pt}M{\isacharcomma}{\kern0pt}\ a{\isacharcomma}{\kern0pt}\ a{\isacharcomma}{\kern0pt}\ x{\isacharparenright}{\kern0pt}\ {\isasymand}\isanewline
\ \ \ \ \ \ \ \ \ \ \ \ \ \ \ \ \ \ \ \ \ \ {\isacharparenleft}{\kern0pt}{\isasymexists}y{\isacharbrackleft}{\kern0pt}M{\isacharbrackright}{\kern0pt}{\isachardot}{\kern0pt}\ upair{\isacharparenleft}{\kern0pt}M{\isacharcomma}{\kern0pt}\ a{\isacharcomma}{\kern0pt}\ b{\isacharcomma}{\kern0pt}\ y{\isacharparenright}{\kern0pt}\ {\isasymand}\ upair{\isacharparenleft}{\kern0pt}M{\isacharcomma}{\kern0pt}\ x{\isacharcomma}{\kern0pt}\ y{\isacharcomma}{\kern0pt}\ z{\isacharparenright}{\kern0pt}{\isacharparenright}{\kern0pt}%
\end{isabelle}%
\begin{isabelle}%
successor{\isacharparenleft}{\kern0pt}M{\isacharcomma}{\kern0pt}\ a{\isacharcomma}{\kern0pt}\ z{\isacharparenright}{\kern0pt}\ {\isasymequiv}\isanewline
{\isasymexists}x{\isacharbrackleft}{\kern0pt}M{\isacharbrackright}{\kern0pt}{\isachardot}{\kern0pt}\ upair{\isacharparenleft}{\kern0pt}M{\isacharcomma}{\kern0pt}\ a{\isacharcomma}{\kern0pt}\ a{\isacharcomma}{\kern0pt}\ x{\isacharparenright}{\kern0pt}\ {\isasymand}\ {\isacharparenleft}{\kern0pt}{\isasymforall}xa{\isacharbrackleft}{\kern0pt}M{\isacharbrackright}{\kern0pt}{\isachardot}{\kern0pt}\ xa\ {\isasymin}\ z\ {\isasymlongleftrightarrow}\ xa\ {\isasymin}\ x\ {\isasymor}\ xa\ {\isasymin}\ a{\isacharparenright}{\kern0pt}%
\end{isabelle}%
\begin{isabelle}%
empty{\isacharparenleft}{\kern0pt}M{\isacharcomma}{\kern0pt}\ z{\isacharparenright}{\kern0pt}\ {\isasymequiv}\ {\isasymforall}x{\isacharbrackleft}{\kern0pt}M{\isacharbrackright}{\kern0pt}{\isachardot}{\kern0pt}\ x\ {\isasymnotin}\ z%
\end{isabelle}%
\begin{isabelle}%
transitive{\isacharunderscore}{\kern0pt}set{\isacharparenleft}{\kern0pt}M{\isacharcomma}{\kern0pt}\ a{\isacharparenright}{\kern0pt}\ {\isasymequiv}\ {\isasymforall}x{\isacharbrackleft}{\kern0pt}M{\isacharbrackright}{\kern0pt}{\isachardot}{\kern0pt}\ x\ {\isasymin}\ a\ {\isasymlongrightarrow}\ {\isacharparenleft}{\kern0pt}{\isasymforall}xa{\isacharbrackleft}{\kern0pt}M{\isacharbrackright}{\kern0pt}{\isachardot}{\kern0pt}\ xa\ {\isasymin}\ x\ {\isasymlongrightarrow}\ xa\ {\isasymin}\ a{\isacharparenright}{\kern0pt}%
\end{isabelle}%
\begin{isabelle}%
ordinal{\isacharparenleft}{\kern0pt}M{\isacharcomma}{\kern0pt}\ a{\isacharparenright}{\kern0pt}\ {\isasymequiv}\isanewline
transitive{\isacharunderscore}{\kern0pt}set{\isacharparenleft}{\kern0pt}M{\isacharcomma}{\kern0pt}\ a{\isacharparenright}{\kern0pt}\ {\isasymand}\ {\isacharparenleft}{\kern0pt}{\isasymforall}x{\isacharbrackleft}{\kern0pt}M{\isacharbrackright}{\kern0pt}{\isachardot}{\kern0pt}\ x\ {\isasymin}\ a\ {\isasymlongrightarrow}\ transitive{\isacharunderscore}{\kern0pt}set{\isacharparenleft}{\kern0pt}M{\isacharcomma}{\kern0pt}\ x{\isacharparenright}{\kern0pt}{\isacharparenright}{\kern0pt}%
\end{isabelle}%
\begin{isabelle}%
image{\isacharparenleft}{\kern0pt}M{\isacharcomma}{\kern0pt}\ r{\isacharcomma}{\kern0pt}\ A{\isacharcomma}{\kern0pt}\ z{\isacharparenright}{\kern0pt}\ {\isasymequiv}\isanewline
{\isasymforall}y{\isacharbrackleft}{\kern0pt}M{\isacharbrackright}{\kern0pt}{\isachardot}{\kern0pt}\ y\ {\isasymin}\ z\ {\isasymlongleftrightarrow}\ {\isacharparenleft}{\kern0pt}{\isasymexists}w{\isacharbrackleft}{\kern0pt}M{\isacharbrackright}{\kern0pt}{\isachardot}{\kern0pt}\ w\ {\isasymin}\ r\ {\isasymand}\ {\isacharparenleft}{\kern0pt}{\isasymexists}x{\isacharbrackleft}{\kern0pt}M{\isacharbrackright}{\kern0pt}{\isachardot}{\kern0pt}\ x\ {\isasymin}\ A\ {\isasymand}\ pair{\isacharparenleft}{\kern0pt}M{\isacharcomma}{\kern0pt}\ x{\isacharcomma}{\kern0pt}\ y{\isacharcomma}{\kern0pt}\ w{\isacharparenright}{\kern0pt}{\isacharparenright}{\kern0pt}{\isacharparenright}{\kern0pt}%
\end{isabelle}%
\begin{isabelle}%
is{\isacharunderscore}{\kern0pt}apply{\isacharparenleft}{\kern0pt}M{\isacharcomma}{\kern0pt}\ f{\isacharcomma}{\kern0pt}\ x{\isacharcomma}{\kern0pt}\ y{\isacharparenright}{\kern0pt}\ {\isasymequiv}\isanewline
{\isasymexists}xs{\isacharbrackleft}{\kern0pt}M{\isacharbrackright}{\kern0pt}{\isachardot}{\kern0pt}\isanewline
\isaindent{\ \ \ }{\isasymexists}fxs{\isacharbrackleft}{\kern0pt}M{\isacharbrackright}{\kern0pt}{\isachardot}{\kern0pt}\ upair{\isacharparenleft}{\kern0pt}M{\isacharcomma}{\kern0pt}\ x{\isacharcomma}{\kern0pt}\ x{\isacharcomma}{\kern0pt}\ xs{\isacharparenright}{\kern0pt}\ {\isasymand}\ image{\isacharparenleft}{\kern0pt}M{\isacharcomma}{\kern0pt}\ f{\isacharcomma}{\kern0pt}\ xs{\isacharcomma}{\kern0pt}\ fxs{\isacharparenright}{\kern0pt}\ {\isasymand}\isanewline
\isaindent{\ \ \ \ \ }big{\isacharunderscore}{\kern0pt}union{\isacharparenleft}{\kern0pt}M{\isacharcomma}{\kern0pt}\ fxs{\isacharcomma}{\kern0pt}\ y{\isacharparenright}{\kern0pt}%
\end{isabelle}%
\begin{isabelle}%
is{\isacharunderscore}{\kern0pt}function{\isacharparenleft}{\kern0pt}M{\isacharcomma}{\kern0pt}\ r{\isacharparenright}{\kern0pt}\ {\isasymequiv}\isanewline
{\isasymforall}x{\isacharbrackleft}{\kern0pt}M{\isacharbrackright}{\kern0pt}{\isachardot}{\kern0pt}\isanewline
\isaindent{\ \ \ }{\isasymforall}y{\isacharbrackleft}{\kern0pt}M{\isacharbrackright}{\kern0pt}{\isachardot}{\kern0pt}\isanewline
\isaindent{\ \ \ \ \ \ }{\isasymforall}y{\isacharprime}{\kern0pt}{\isacharbrackleft}{\kern0pt}M{\isacharbrackright}{\kern0pt}{\isachardot}{\kern0pt}\isanewline
\isaindent{\ \ \ \ \ \ \ \ \ }{\isasymforall}p{\isacharbrackleft}{\kern0pt}M{\isacharbrackright}{\kern0pt}{\isachardot}{\kern0pt}\isanewline
\isaindent{\ \ \ \ \ \ \ \ \ \ \ \ }{\isasymforall}p{\isacharprime}{\kern0pt}{\isacharbrackleft}{\kern0pt}M{\isacharbrackright}{\kern0pt}{\isachardot}{\kern0pt}\isanewline
\isaindent{\ \ \ \ \ \ \ \ \ \ \ \ \ \ \ }pair{\isacharparenleft}{\kern0pt}M{\isacharcomma}{\kern0pt}\ x{\isacharcomma}{\kern0pt}\ y{\isacharcomma}{\kern0pt}\ p{\isacharparenright}{\kern0pt}\ {\isasymlongrightarrow}\isanewline
\isaindent{\ \ \ \ \ \ \ \ \ \ \ \ \ \ \ }pair{\isacharparenleft}{\kern0pt}M{\isacharcomma}{\kern0pt}\ x{\isacharcomma}{\kern0pt}\ y{\isacharprime}{\kern0pt}{\isacharcomma}{\kern0pt}\ p{\isacharprime}{\kern0pt}{\isacharparenright}{\kern0pt}\ {\isasymlongrightarrow}\ p\ {\isasymin}\ r\ {\isasymlongrightarrow}\ p{\isacharprime}{\kern0pt}\ {\isasymin}\ r\ {\isasymlongrightarrow}\ y\ {\isacharequal}{\kern0pt}\ y{\isacharprime}{\kern0pt}%
\end{isabelle}%
\begin{isabelle}%
is{\isacharunderscore}{\kern0pt}relation{\isacharparenleft}{\kern0pt}M{\isacharcomma}{\kern0pt}\ r{\isacharparenright}{\kern0pt}\ {\isasymequiv}\ {\isasymforall}z{\isacharbrackleft}{\kern0pt}M{\isacharbrackright}{\kern0pt}{\isachardot}{\kern0pt}\ z\ {\isasymin}\ r\ {\isasymlongrightarrow}\ {\isacharparenleft}{\kern0pt}{\isasymexists}x{\isacharbrackleft}{\kern0pt}M{\isacharbrackright}{\kern0pt}{\isachardot}{\kern0pt}\ {\isasymexists}y{\isacharbrackleft}{\kern0pt}M{\isacharbrackright}{\kern0pt}{\isachardot}{\kern0pt}\ pair{\isacharparenleft}{\kern0pt}M{\isacharcomma}{\kern0pt}\ x{\isacharcomma}{\kern0pt}\ y{\isacharcomma}{\kern0pt}\ z{\isacharparenright}{\kern0pt}{\isacharparenright}{\kern0pt}%
\end{isabelle}%
\begin{isabelle}%
is{\isacharunderscore}{\kern0pt}domain{\isacharparenleft}{\kern0pt}M{\isacharcomma}{\kern0pt}\ r{\isacharcomma}{\kern0pt}\ z{\isacharparenright}{\kern0pt}\ {\isasymequiv}\isanewline
{\isasymforall}x{\isacharbrackleft}{\kern0pt}M{\isacharbrackright}{\kern0pt}{\isachardot}{\kern0pt}\ x\ {\isasymin}\ z\ {\isasymlongleftrightarrow}\ {\isacharparenleft}{\kern0pt}{\isasymexists}w{\isacharbrackleft}{\kern0pt}M{\isacharbrackright}{\kern0pt}{\isachardot}{\kern0pt}\ w\ {\isasymin}\ r\ {\isasymand}\ {\isacharparenleft}{\kern0pt}{\isasymexists}y{\isacharbrackleft}{\kern0pt}M{\isacharbrackright}{\kern0pt}{\isachardot}{\kern0pt}\ pair{\isacharparenleft}{\kern0pt}M{\isacharcomma}{\kern0pt}\ x{\isacharcomma}{\kern0pt}\ y{\isacharcomma}{\kern0pt}\ w{\isacharparenright}{\kern0pt}{\isacharparenright}{\kern0pt}{\isacharparenright}{\kern0pt}%
\end{isabelle}%
\begin{isabelle}%
typed{\isacharunderscore}{\kern0pt}function{\isacharparenleft}{\kern0pt}M{\isacharcomma}{\kern0pt}\ A{\isacharcomma}{\kern0pt}\ B{\isacharcomma}{\kern0pt}\ r{\isacharparenright}{\kern0pt}\ {\isasymequiv}\isanewline
is{\isacharunderscore}{\kern0pt}function{\isacharparenleft}{\kern0pt}M{\isacharcomma}{\kern0pt}\ r{\isacharparenright}{\kern0pt}\ {\isasymand}\isanewline
is{\isacharunderscore}{\kern0pt}relation{\isacharparenleft}{\kern0pt}M{\isacharcomma}{\kern0pt}\ r{\isacharparenright}{\kern0pt}\ {\isasymand}\isanewline
is{\isacharunderscore}{\kern0pt}domain{\isacharparenleft}{\kern0pt}M{\isacharcomma}{\kern0pt}\ r{\isacharcomma}{\kern0pt}\ A{\isacharparenright}{\kern0pt}\ {\isasymand}\isanewline
{\isacharparenleft}{\kern0pt}{\isasymforall}u{\isacharbrackleft}{\kern0pt}M{\isacharbrackright}{\kern0pt}{\isachardot}{\kern0pt}\ u\ {\isasymin}\ r\ {\isasymlongrightarrow}\ {\isacharparenleft}{\kern0pt}{\isasymforall}x{\isacharbrackleft}{\kern0pt}M{\isacharbrackright}{\kern0pt}{\isachardot}{\kern0pt}\ {\isasymforall}y{\isacharbrackleft}{\kern0pt}M{\isacharbrackright}{\kern0pt}{\isachardot}{\kern0pt}\ pair{\isacharparenleft}{\kern0pt}M{\isacharcomma}{\kern0pt}\ x{\isacharcomma}{\kern0pt}\ y{\isacharcomma}{\kern0pt}\ u{\isacharparenright}{\kern0pt}\ {\isasymlongrightarrow}\ y\ {\isasymin}\ B{\isacharparenright}{\kern0pt}{\isacharparenright}{\kern0pt}%
\end{isabelle}%
\begin{isabelle}%
is{\isacharunderscore}{\kern0pt}function{\isacharunderscore}{\kern0pt}space{\isacharparenleft}{\kern0pt}M{\isacharcomma}{\kern0pt}\ A{\isacharcomma}{\kern0pt}\ B{\isacharcomma}{\kern0pt}\ fs{\isacharparenright}{\kern0pt}\ {\isasymequiv}\isanewline
M{\isacharparenleft}{\kern0pt}fs{\isacharparenright}{\kern0pt}\ {\isasymand}\ {\isacharparenleft}{\kern0pt}{\isasymforall}f{\isacharbrackleft}{\kern0pt}M{\isacharbrackright}{\kern0pt}{\isachardot}{\kern0pt}\ f\ {\isasymin}\ fs\ {\isasymlongleftrightarrow}\ typed{\isacharunderscore}{\kern0pt}function{\isacharparenleft}{\kern0pt}M{\isacharcomma}{\kern0pt}\ A{\isacharcomma}{\kern0pt}\ B{\isacharcomma}{\kern0pt}\ f{\isacharparenright}{\kern0pt}{\isacharparenright}{\kern0pt}\isasep\isanewline\isanewline%
A\ {\isasymrightarrow}\isactrlbsup M\isactrlesup \ B\ {\isasymequiv}\ THE\ d{\isachardot}{\kern0pt}\ is{\isacharunderscore}{\kern0pt}function{\isacharunderscore}{\kern0pt}space{\isacharparenleft}{\kern0pt}M{\isacharcomma}{\kern0pt}\ A{\isacharcomma}{\kern0pt}\ B{\isacharcomma}{\kern0pt}\ d{\isacharparenright}{\kern0pt}\isasep\isanewline\isanewline%
surjection{\isacharparenleft}{\kern0pt}M{\isacharcomma}{\kern0pt}\ A{\isacharcomma}{\kern0pt}\ B{\isacharcomma}{\kern0pt}\ f{\isacharparenright}{\kern0pt}\ {\isasymequiv}\isanewline
typed{\isacharunderscore}{\kern0pt}function{\isacharparenleft}{\kern0pt}M{\isacharcomma}{\kern0pt}\ A{\isacharcomma}{\kern0pt}\ B{\isacharcomma}{\kern0pt}\ f{\isacharparenright}{\kern0pt}\ {\isasymand}\isanewline
{\isacharparenleft}{\kern0pt}{\isasymforall}y{\isacharbrackleft}{\kern0pt}M{\isacharbrackright}{\kern0pt}{\isachardot}{\kern0pt}\ y\ {\isasymin}\ B\ {\isasymlongrightarrow}\ {\isacharparenleft}{\kern0pt}{\isasymexists}x{\isacharbrackleft}{\kern0pt}M{\isacharbrackright}{\kern0pt}{\isachardot}{\kern0pt}\ x\ {\isasymin}\ A\ {\isasymand}\ is{\isacharunderscore}{\kern0pt}apply{\isacharparenleft}{\kern0pt}M{\isacharcomma}{\kern0pt}\ f{\isacharcomma}{\kern0pt}\ x{\isacharcomma}{\kern0pt}\ y{\isacharparenright}{\kern0pt}{\isacharparenright}{\kern0pt}{\isacharparenright}{\kern0pt}%
\end{isabelle}%

\subsubsection*{Relative version of the $\ZFC$ axioms}
\begin{isabelle}%
extensionality{\isacharparenleft}{\kern0pt}M{\isacharparenright}{\kern0pt}\ {\isasymequiv}\ {\isasymforall}x{\isacharbrackleft}{\kern0pt}M{\isacharbrackright}{\kern0pt}{\isachardot}{\kern0pt}\ {\isasymforall}y{\isacharbrackleft}{\kern0pt}M{\isacharbrackright}{\kern0pt}{\isachardot}{\kern0pt}\ {\isacharparenleft}{\kern0pt}{\isasymforall}z{\isacharbrackleft}{\kern0pt}M{\isacharbrackright}{\kern0pt}{\isachardot}{\kern0pt}\ z\ {\isasymin}\ x\ {\isasymlongleftrightarrow}\ z\ {\isasymin}\ y{\isacharparenright}{\kern0pt}\ {\isasymlongrightarrow}\ x\ {\isacharequal}{\kern0pt}\ y%
\end{isabelle}%
\begin{isabelle}%
foundation{\isacharunderscore}{\kern0pt}ax{\isacharparenleft}{\kern0pt}M{\isacharparenright}{\kern0pt}\ {\isasymequiv}\isanewline
{\isasymforall}x{\isacharbrackleft}{\kern0pt}M{\isacharbrackright}{\kern0pt}{\isachardot}{\kern0pt}\ {\isacharparenleft}{\kern0pt}{\isasymexists}y{\isacharbrackleft}{\kern0pt}M{\isacharbrackright}{\kern0pt}{\isachardot}{\kern0pt}\ y\ {\isasymin}\ x{\isacharparenright}{\kern0pt}\ {\isasymlongrightarrow}\ {\isacharparenleft}{\kern0pt}{\isasymexists}y{\isacharbrackleft}{\kern0pt}M{\isacharbrackright}{\kern0pt}{\isachardot}{\kern0pt}\ y\ {\isasymin}\ x\ {\isasymand}\ {\isasymnot}\ {\isacharparenleft}{\kern0pt}{\isasymexists}z{\isacharbrackleft}{\kern0pt}M{\isacharbrackright}{\kern0pt}{\isachardot}{\kern0pt}\ z\ {\isasymin}\ x\ {\isasymand}\ z\ {\isasymin}\ y{\isacharparenright}{\kern0pt}{\isacharparenright}{\kern0pt}%
\end{isabelle}%
\begin{isabelle}%
upair{\isacharunderscore}{\kern0pt}ax{\isacharparenleft}{\kern0pt}M{\isacharparenright}{\kern0pt}\ {\isasymequiv}\ {\isasymforall}x{\isacharbrackleft}{\kern0pt}M{\isacharbrackright}{\kern0pt}{\isachardot}{\kern0pt}\ {\isasymforall}y{\isacharbrackleft}{\kern0pt}M{\isacharbrackright}{\kern0pt}{\isachardot}{\kern0pt}\ {\isasymexists}z{\isacharbrackleft}{\kern0pt}M{\isacharbrackright}{\kern0pt}{\isachardot}{\kern0pt}\ upair{\isacharparenleft}{\kern0pt}M{\isacharcomma}{\kern0pt}\ x{\isacharcomma}{\kern0pt}\ y{\isacharcomma}{\kern0pt}\ z{\isacharparenright}{\kern0pt}%
\end{isabelle}%
\begin{isabelle}%
Union{\isacharunderscore}{\kern0pt}ax{\isacharparenleft}{\kern0pt}M{\isacharparenright}{\kern0pt}\ {\isasymequiv}\ {\isasymforall}x{\isacharbrackleft}{\kern0pt}M{\isacharbrackright}{\kern0pt}{\isachardot}{\kern0pt}\ {\isasymexists}z{\isacharbrackleft}{\kern0pt}M{\isacharbrackright}{\kern0pt}{\isachardot}{\kern0pt}\ big{\isacharunderscore}{\kern0pt}union{\isacharparenleft}{\kern0pt}M{\isacharcomma}{\kern0pt}\ x{\isacharcomma}{\kern0pt}\ z{\isacharparenright}{\kern0pt}%
\end{isabelle}%
\begin{isabelle}%
power{\isacharunderscore}{\kern0pt}ax{\isacharparenleft}{\kern0pt}M{\isacharparenright}{\kern0pt}\ {\isasymequiv}\ {\isasymforall}x{\isacharbrackleft}{\kern0pt}M{\isacharbrackright}{\kern0pt}{\isachardot}{\kern0pt}\ {\isasymexists}z{\isacharbrackleft}{\kern0pt}M{\isacharbrackright}{\kern0pt}{\isachardot}{\kern0pt}\ {\isasymforall}xa{\isacharbrackleft}{\kern0pt}M{\isacharbrackright}{\kern0pt}{\isachardot}{\kern0pt}\ xa\ {\isasymin}\ z\ {\isasymlongleftrightarrow}\isanewline
\ \ \ \ \ \ \ \ \ \ \ \ \ \ \ \ \ {\isacharparenleft}{\kern0pt}{\isasymforall}xb{\isacharbrackleft}{\kern0pt}M{\isacharbrackright}{\kern0pt}{\isachardot}{\kern0pt}\ xb\ {\isasymin}\ xa\ {\isasymlongrightarrow}\ xb\ {\isasymin}\ x{\isacharparenright}{\kern0pt}%
\end{isabelle}%
\begin{isabelle}%
infinity{\isacharunderscore}{\kern0pt}ax{\isacharparenleft}{\kern0pt}M{\isacharparenright}{\kern0pt}\ {\isasymequiv}\isanewline
{\isasymexists}I{\isacharbrackleft}{\kern0pt}M{\isacharbrackright}{\kern0pt}{\isachardot}{\kern0pt}\isanewline
\isaindent{\ \ \ }{\isacharparenleft}{\kern0pt}{\isasymexists}z{\isacharbrackleft}{\kern0pt}M{\isacharbrackright}{\kern0pt}{\isachardot}{\kern0pt}\ empty{\isacharparenleft}{\kern0pt}M{\isacharcomma}{\kern0pt}\ z{\isacharparenright}{\kern0pt}\ {\isasymand}\ z\ {\isasymin}\ I{\isacharparenright}{\kern0pt}\ {\isasymand}\isanewline
\isaindent{\ \ \ }{\isacharparenleft}{\kern0pt}{\isasymforall}y{\isacharbrackleft}{\kern0pt}M{\isacharbrackright}{\kern0pt}{\isachardot}{\kern0pt}\ y\ {\isasymin}\ I\ {\isasymlongrightarrow}\ {\isacharparenleft}{\kern0pt}{\isasymexists}sy{\isacharbrackleft}{\kern0pt}M{\isacharbrackright}{\kern0pt}{\isachardot}{\kern0pt}\ successor{\isacharparenleft}{\kern0pt}M{\isacharcomma}{\kern0pt}\ y{\isacharcomma}{\kern0pt}\ sy{\isacharparenright}{\kern0pt}\ {\isasymand}\ sy\ {\isasymin}\ I{\isacharparenright}{\kern0pt}{\isacharparenright}{\kern0pt}%
\end{isabelle}%
\begin{isabelle}%
choice{\isacharunderscore}{\kern0pt}ax{\isacharparenleft}{\kern0pt}M{\isacharparenright}{\kern0pt}\ {\isasymequiv}\ {\isasymforall}x{\isacharbrackleft}{\kern0pt}M{\isacharbrackright}{\kern0pt}{\isachardot}{\kern0pt}\ {\isasymexists}a{\isacharbrackleft}{\kern0pt}M{\isacharbrackright}{\kern0pt}{\isachardot}{\kern0pt}\ {\isasymexists}f{\isacharbrackleft}{\kern0pt}M{\isacharbrackright}{\kern0pt}{\isachardot}{\kern0pt}\ ordinal{\isacharparenleft}{\kern0pt}M{\isacharcomma}{\kern0pt}\ a{\isacharparenright}{\kern0pt}\ {\isasymand}\isanewline
\ \ \ \ \ \ \ \ \ \ \ \ \ \ \ \ \ \ surjection{\isacharparenleft}{\kern0pt}M{\isacharcomma}{\kern0pt}\ a{\isacharcomma}{\kern0pt}\ x{\isacharcomma}{\kern0pt}\ f{\isacharparenright}{\kern0pt}%
\end{isabelle}%
\begin{isabelle}%
separation{\isacharparenleft}{\kern0pt}M{\isacharcomma}{\kern0pt}\ P{\isacharparenright}{\kern0pt}\ {\isasymequiv}\ {\isasymforall}z{\isacharbrackleft}{\kern0pt}M{\isacharbrackright}{\kern0pt}{\isachardot}{\kern0pt}\ {\isasymexists}y{\isacharbrackleft}{\kern0pt}M{\isacharbrackright}{\kern0pt}{\isachardot}{\kern0pt}\ {\isasymforall}x{\isacharbrackleft}{\kern0pt}M{\isacharbrackright}{\kern0pt}{\isachardot}{\kern0pt}\ x\ {\isasymin}\ y\ {\isasymlongleftrightarrow}\ x\ {\isasymin}\ z\ {\isasymand}\ P{\isacharparenleft}{\kern0pt}x{\isacharparenright}{\kern0pt}%
\end{isabelle}%
\begin{isabelle}%
univalent{\isacharparenleft}{\kern0pt}M{\isacharcomma}{\kern0pt}\ A{\isacharcomma}{\kern0pt}\ P{\isacharparenright}{\kern0pt}\ {\isasymequiv}\isanewline
{\isasymforall}x{\isacharbrackleft}{\kern0pt}M{\isacharbrackright}{\kern0pt}{\isachardot}{\kern0pt}\ x\ {\isasymin}\ A\ {\isasymlongrightarrow}\ {\isacharparenleft}{\kern0pt}{\isasymforall}y{\isacharbrackleft}{\kern0pt}M{\isacharbrackright}{\kern0pt}{\isachardot}{\kern0pt}\ {\isasymforall}z{\isacharbrackleft}{\kern0pt}M{\isacharbrackright}{\kern0pt}{\isachardot}{\kern0pt}\ P{\isacharparenleft}{\kern0pt}x{\isacharcomma}{\kern0pt}\ y{\isacharparenright}{\kern0pt}\ {\isasymand}\ P{\isacharparenleft}{\kern0pt}x{\isacharcomma}{\kern0pt}\ z{\isacharparenright}{\kern0pt}\ {\isasymlongrightarrow}\ y\ {\isacharequal}{\kern0pt}\ z{\isacharparenright}{\kern0pt}%
\end{isabelle}%
\begin{isabelle}%
strong{\isacharunderscore}{\kern0pt}replacement{\isacharparenleft}{\kern0pt}M{\isacharcomma}{\kern0pt}\ P{\isacharparenright}{\kern0pt}\ {\isasymequiv}\isanewline
{\isasymforall}A{\isacharbrackleft}{\kern0pt}M{\isacharbrackright}{\kern0pt}{\isachardot}{\kern0pt}\isanewline
\isaindent{\ \ \ }univalent{\isacharparenleft}{\kern0pt}M{\isacharcomma}{\kern0pt}\ A{\isacharcomma}{\kern0pt}\ P{\isacharparenright}{\kern0pt}\ {\isasymlongrightarrow}\ {\isacharparenleft}{\kern0pt}{\isasymexists}Y{\isacharbrackleft}{\kern0pt}M{\isacharbrackright}{\kern0pt}{\isachardot}{\kern0pt}\ {\isasymforall}b{\isacharbrackleft}{\kern0pt}M{\isacharbrackright}{\kern0pt}{\isachardot}{\kern0pt}\ b\ {\isasymin}\ Y\ {\isasymlongleftrightarrow}\isanewline
\ \ \ \ \ \ {\isacharparenleft}{\kern0pt}{\isasymexists}x{\isacharbrackleft}{\kern0pt}M{\isacharbrackright}{\kern0pt}{\isachardot}{\kern0pt}\ x\ {\isasymin}\ A\ {\isasymand}
P{\isacharparenleft}{\kern0pt}x{\isacharcomma}{\kern0pt}\ b{\isacharparenright}{\kern0pt}{\isacharparenright}{\kern0pt}{\isacharparenright}{\kern0pt}%
\end{isabelle}%
\subsubsection*{Internalized formulas}
“Codes” for formulas (as sets) are constructed from natural
numbers using \isa{Member}, \isa{Equal}, \isa{Nand},
and \isa{Forall}.%
\begin{isabelle}%
{\isasymlbrakk}x\ {\isasymin}\ {\isasymomega}{\isacharsemicolon}{\kern0pt}\ y\ {\isasymin}\ {\isasymomega}{\isasymrbrakk}\ {\isasymLongrightarrow}\ {\isasymcdot}x\ {\isasymin}\ y{\isasymcdot}\ {\isasymin}\ formula\isasep\isanewline%
{\isasymlbrakk}x\ {\isasymin}\ {\isasymomega}{\isacharsemicolon}{\kern0pt}\ y\ {\isasymin}\ {\isasymomega}{\isasymrbrakk}\ {\isasymLongrightarrow}\ {\isasymcdot}x\ {\isacharequal}{\kern0pt}\ y{\isasymcdot}\ {\isasymin}\ formula\isasep\isanewline%
{\isasymlbrakk}p\ {\isasymin}\ formula{\isacharsemicolon}{\kern0pt}\ q\ {\isasymin}\ formula{\isasymrbrakk}\ {\isasymLongrightarrow}\ {\isasymcdot}{\isasymnot}{\isacharparenleft}{\kern0pt}p\ {\isasymand}\ q{\isacharparenright}{\kern0pt}{\isasymcdot}\ {\isasymin}\ formula\isasep\isanewline%
p\ {\isasymin}\ formula\ {\isasymLongrightarrow}\ {\isacharparenleft}{\kern0pt}{\isasymcdot}{\isasymforall}p{\isasymcdot}{\isacharparenright}{\kern0pt}\ {\isasymin}\ formula\isasep\isanewline\isanewline%
{\isasymlbrakk}x\ {\isasymin}\ formula{\isacharsemicolon}{\kern0pt}\ {\isasymAnd}x\ y{\isachardot}{\kern0pt}\ {\isasymlbrakk}x\ {\isasymin}\ {\isasymomega}{\isacharsemicolon}{\kern0pt}\ y\ {\isasymin}\ {\isasymomega}{\isasymrbrakk}\ {\isasymLongrightarrow}\ P{\isacharparenleft}{\kern0pt}{\isasymcdot}x\ {\isasymin}\ y{\isasymcdot}{\isacharparenright}{\kern0pt}{\isacharsemicolon}{\kern0pt}\isanewline
\isaindent{\ }{\isasymAnd}x\ y{\isachardot}{\kern0pt}\ {\isasymlbrakk}x\ {\isasymin}\ {\isasymomega}{\isacharsemicolon}{\kern0pt}\ y\ {\isasymin}\ {\isasymomega}{\isasymrbrakk}\ {\isasymLongrightarrow}\ P{\isacharparenleft}{\kern0pt}{\isasymcdot}x\ {\isacharequal}{\kern0pt}\ y{\isasymcdot}{\isacharparenright}{\kern0pt}{\isacharsemicolon}{\kern0pt}\isanewline
\isaindent{\ }{\isasymAnd}p\ q{\isachardot}{\kern0pt}\ {\isasymlbrakk}p\ {\isasymin}\ formula{\isacharsemicolon}{\kern0pt}\ P{\isacharparenleft}{\kern0pt}p{\isacharparenright}{\kern0pt}{\isacharsemicolon}{\kern0pt}\ q\ {\isasymin}\ formula{\isacharsemicolon}{\kern0pt}\ P{\isacharparenleft}{\kern0pt}q{\isacharparenright}{\kern0pt}{\isasymrbrakk}\ {\isasymLongrightarrow}\ P{\isacharparenleft}{\kern0pt}{\isasymcdot}{\isasymnot}{\isacharparenleft}{\kern0pt}p\ {\isasymand}\ q{\isacharparenright}{\kern0pt}{\isasymcdot}{\isacharparenright}{\kern0pt}{\isacharsemicolon}{\kern0pt}\isanewline
\isaindent{\ }{\isasymAnd}p{\isachardot}{\kern0pt}\ {\isasymlbrakk}p\ {\isasymin}\ formula{\isacharsemicolon}{\kern0pt}\ P{\isacharparenleft}{\kern0pt}p{\isacharparenright}{\kern0pt}{\isasymrbrakk}\ {\isasymLongrightarrow}\ P{\isacharparenleft}{\kern0pt}{\isacharparenleft}{\kern0pt}{\isasymcdot}{\isasymforall}p{\isasymcdot}{\isacharparenright}{\kern0pt}{\isacharparenright}{\kern0pt}{\isasymrbrakk}\isanewline
{\isasymLongrightarrow}\ P{\isacharparenleft}{\kern0pt}x{\isacharparenright}{\kern0pt}%
\end{isabelle}%
Definitions for the other connectives and the internal existential
quantifier are also provided. For instance, negation:%
\begin{isabelle}%
{\isasymcdot}{\isasymnot}p{\isasymcdot}\ {\isasymequiv}\ {\isasymcdot}{\isasymnot}{\isacharparenleft}{\kern0pt}p\ {\isasymand}\ p{\isacharparenright}{\kern0pt}{\isasymcdot}%
\end{isabelle}%
The \isa{arity} function strictly bounding the free de Bruijn
indices of a formula is defined below:
\begin{isabelle}%
arity{\isacharparenleft}{\kern0pt}{\isasymcdot}x\ {\isasymin}\ y{\isasymcdot}{\isacharparenright}{\kern0pt}\ {\isacharequal}{\kern0pt}\ succ{\isacharparenleft}{\kern0pt}x{\isacharparenright}{\kern0pt}\ {\isasymunion}\ succ{\isacharparenleft}{\kern0pt}y{\isacharparenright}{\kern0pt}\isasep\isanewline%
arity{\isacharparenleft}{\kern0pt}{\isasymcdot}x\ {\isacharequal}{\kern0pt}\ y{\isasymcdot}{\isacharparenright}{\kern0pt}\ {\isacharequal}{\kern0pt}\ succ{\isacharparenleft}{\kern0pt}x{\isacharparenright}{\kern0pt}\ {\isasymunion}\ succ{\isacharparenleft}{\kern0pt}y{\isacharparenright}{\kern0pt}\isasep\isanewline%
arity{\isacharparenleft}{\kern0pt}{\isasymcdot}{\isasymnot}{\isacharparenleft}{\kern0pt}p\ {\isasymand}\ q{\isacharparenright}{\kern0pt}{\isasymcdot}{\isacharparenright}{\kern0pt}\ {\isacharequal}{\kern0pt}\ arity{\isacharparenleft}{\kern0pt}p{\isacharparenright}{\kern0pt}\ {\isasymunion}\ arity{\isacharparenleft}{\kern0pt}q{\isacharparenright}{\kern0pt}\isasep\isanewline%
arity{\isacharparenleft}{\kern0pt}{\isacharparenleft}{\kern0pt}{\isasymcdot}{\isasymforall}p{\isasymcdot}{\isacharparenright}{\kern0pt}{\isacharparenright}{\kern0pt}\ {\isacharequal}{\kern0pt}\ pred{\isacharparenleft}{\kern0pt}arity{\isacharparenleft}{\kern0pt}p{\isacharparenright}{\kern0pt}{\isacharparenright}{\kern0pt}%
\end{isabelle}%
We have the satisfaction relation between $\in$-models and
    first order formulas (given a “environment” list representing
    the assignment of free variables),%
\begin{isabelle}%
{\isasymlbrakk}nth{\isacharparenleft}{\kern0pt}i{\isacharcomma}{\kern0pt}\ env{\isacharparenright}{\kern0pt}\ {\isacharequal}{\kern0pt}\ x{\isacharsemicolon}{\kern0pt}\ nth{\isacharparenleft}{\kern0pt}j{\isacharcomma}{\kern0pt}\ env{\isacharparenright}{\kern0pt}\ {\isacharequal}{\kern0pt}\ y{\isacharsemicolon}{\kern0pt}\ env\ {\isasymin}\ list{\isacharparenleft}{\kern0pt}A{\isacharparenright}{\kern0pt}{\isasymrbrakk}\isanewline
{\isasymLongrightarrow}\ x\ {\isasymin}\ y\ {\isasymlongleftrightarrow}\ A{\isacharcomma}{\kern0pt}\ env\ {\isasymTurnstile}\ {\isasymcdot}i\ {\isasymin}\ j{\isasymcdot}\isasep\isanewline\isanewline%
{\isasymlbrakk}nth{\isacharparenleft}{\kern0pt}i{\isacharcomma}{\kern0pt}\ env{\isacharparenright}{\kern0pt}\ {\isacharequal}{\kern0pt}\ x{\isacharsemicolon}{\kern0pt}\ nth{\isacharparenleft}{\kern0pt}j{\isacharcomma}{\kern0pt}\ env{\isacharparenright}{\kern0pt}\ {\isacharequal}{\kern0pt}\ y{\isacharsemicolon}{\kern0pt}\ env\ {\isasymin}\ list{\isacharparenleft}{\kern0pt}A{\isacharparenright}{\kern0pt}{\isasymrbrakk}\isanewline
{\isasymLongrightarrow}\ x\ {\isacharequal}{\kern0pt}\ y\ {\isasymlongleftrightarrow}\ A{\isacharcomma}{\kern0pt}\ env\ {\isasymTurnstile}\ {\isasymcdot}i\ {\isacharequal}{\kern0pt}\ j{\isasymcdot}\isasep\isanewline\isanewline%
env\ {\isasymin}\ list{\isacharparenleft}{\kern0pt}A{\isacharparenright}{\kern0pt}\ {\isasymLongrightarrow}\ {\isacharparenleft}{\kern0pt}A{\isacharcomma}{\kern0pt}\ env\ {\isasymTurnstile}\ {\isasymcdot}{\isasymnot}{\isacharparenleft}{\kern0pt}p\ {\isasymand}\ q{\isacharparenright}{\kern0pt}{\isasymcdot}{\isacharparenright}{\kern0pt}\ {\isasymlongleftrightarrow}\ {\isasymnot}\ {\isacharparenleft}{\kern0pt}{\isacharparenleft}{\kern0pt}A{\isacharcomma}{\kern0pt}\ env\ {\isasymTurnstile}\ p{\isacharparenright}{\kern0pt}\ {\isasymand}\isanewline%
\ \ {\isacharparenleft}{\kern0pt}A{\isacharcomma}{\kern0pt}\ env\ {\isasymTurnstile}\ q{\isacharparenright}{\kern0pt}{\isacharparenright}{\kern0pt}\isasep\isanewline\isanewline%
env\ {\isasymin}\ list{\isacharparenleft}{\kern0pt}A{\isacharparenright}{\kern0pt}\ {\isasymLongrightarrow}\ {\isacharparenleft}{\kern0pt}A{\isacharcomma}{\kern0pt}\ env\ {\isasymTurnstile}\ {\isacharparenleft}{\kern0pt}{\isasymcdot}{\isasymforall}p{\isasymcdot}{\isacharparenright}{\kern0pt}{\isacharparenright}{\kern0pt}\ {\isasymlongleftrightarrow}\ {\isacharparenleft}{\kern0pt}{\isasymforall}x{\isasymin}A{\isachardot}{\kern0pt}\ A{\isacharcomma}{\kern0pt}\ Cons{\isacharparenleft}{\kern0pt}x{\isacharcomma}{\kern0pt}\ env{\isacharparenright}{\kern0pt}\ {\isasymTurnstile}\ p{\isacharparenright}{\kern0pt}%
\end{isabelle}%
as well as the satisfaction of an arbitrary set of sentences.%
\begin{isabelle}%
A\ {\isasymTurnstile}\ {\isasymPhi}\ {\isasymequiv}\ {\isasymforall}{\isasymphi}{\isasymin}{\isasymPhi}{\isachardot}{\kern0pt}\ A{\isacharcomma}{\kern0pt}\ {\isacharbrackleft}{\kern0pt}{\isacharbrackright}{\kern0pt}\ {\isasymTurnstile}\ {\isasymphi}%
\end{isabelle}%
The internalized (viz. as elements of the set \isa{formula})
versions of the axioms are checked next against the relative statements.%
\begin{isabelle}%
Union{\isacharunderscore}{\kern0pt}ax{\isacharparenleft}{\kern0pt}{\isacharhash}{\kern0pt}{\isacharhash}{\kern0pt}A{\isacharparenright}{\kern0pt}\ {\isasymlongleftrightarrow}\ A{\isacharcomma}{\kern0pt}\ {\isacharbrackleft}{\kern0pt}{\isacharbrackright}{\kern0pt}\ {\isasymTurnstile}\ {\isasymcdot}Union\ Ax{\isasymcdot}\isasep\isanewline%
power{\isacharunderscore}{\kern0pt}ax{\isacharparenleft}{\kern0pt}{\isacharhash}{\kern0pt}{\isacharhash}{\kern0pt}A{\isacharparenright}{\kern0pt}\ {\isasymlongleftrightarrow}\ A{\isacharcomma}{\kern0pt}\ {\isacharbrackleft}{\kern0pt}{\isacharbrackright}{\kern0pt}\ {\isasymTurnstile}\ {\isasymcdot}Powerset\ Ax{\isasymcdot}\isasep\isanewline%
upair{\isacharunderscore}{\kern0pt}ax{\isacharparenleft}{\kern0pt}{\isacharhash}{\kern0pt}{\isacharhash}{\kern0pt}A{\isacharparenright}{\kern0pt}\ {\isasymlongleftrightarrow}\ A{\isacharcomma}{\kern0pt}\ {\isacharbrackleft}{\kern0pt}{\isacharbrackright}{\kern0pt}\ {\isasymTurnstile}\ {\isasymcdot}Pairing{\isasymcdot}\isasep\isanewline%
foundation{\isacharunderscore}{\kern0pt}ax{\isacharparenleft}{\kern0pt}{\isacharhash}{\kern0pt}{\isacharhash}{\kern0pt}A{\isacharparenright}{\kern0pt}\ {\isasymlongleftrightarrow}\ A{\isacharcomma}{\kern0pt}\ {\isacharbrackleft}{\kern0pt}{\isacharbrackright}{\kern0pt}\ {\isasymTurnstile}\ {\isasymcdot}Foundation{\isasymcdot}\isasep\isanewline%
extensionality{\isacharparenleft}{\kern0pt}{\isacharhash}{\kern0pt}{\isacharhash}{\kern0pt}A{\isacharparenright}{\kern0pt}\ {\isasymlongleftrightarrow}\ A{\isacharcomma}{\kern0pt}\ {\isacharbrackleft}{\kern0pt}{\isacharbrackright}{\kern0pt}\ {\isasymTurnstile}\ {\isasymcdot}Extensionality{\isasymcdot}\isasep\isanewline%
infinity{\isacharunderscore}{\kern0pt}ax{\isacharparenleft}{\kern0pt}{\isacharhash}{\kern0pt}{\isacharhash}{\kern0pt}A{\isacharparenright}{\kern0pt}\ {\isasymlongleftrightarrow}\ A{\isacharcomma}{\kern0pt}\ {\isacharbrackleft}{\kern0pt}{\isacharbrackright}{\kern0pt}\ {\isasymTurnstile}\ {\isasymcdot}Infinity{\isasymcdot}\isasep\isanewline\isanewline%
{\isasymphi}\ {\isasymin}\ formula\ {\isasymLongrightarrow}\isanewline
{\isacharparenleft}{\kern0pt}M{\isacharcomma}{\kern0pt}\ {\isacharbrackleft}{\kern0pt}{\isacharbrackright}{\kern0pt}\ {\isasymTurnstile}\ {\isasymcdot}Separation{\isacharparenleft}{\kern0pt}{\isasymphi}{\isacharparenright}{\kern0pt}{\isasymcdot}{\isacharparenright}{\kern0pt}\ {\isasymlongleftrightarrow}\isanewline
{\isacharparenleft}{\kern0pt}{\isasymforall}env{\isasymin}list{\isacharparenleft}{\kern0pt}M{\isacharparenright}{\kern0pt}{\isachardot}{\kern0pt}\isanewline
\isaindent{{\isacharparenleft}{\kern0pt}\ \ \ }arity{\isacharparenleft}{\kern0pt}{\isasymphi}{\isacharparenright}{\kern0pt}\ {\isasymle}\ {\isadigit{1}}\ {\isacharplus}{\kern0pt}\isactrlsub {\isasymomega}\ length{\isacharparenleft}{\kern0pt}env{\isacharparenright}{\kern0pt}\ {\isasymlongrightarrow}\isanewline
\ \ \ \ separation{\isacharparenleft}{\kern0pt}{\isacharhash}{\kern0pt}{\isacharhash}{\kern0pt}M{\isacharcomma}{\kern0pt}\ {\isasymlambda}x{\isachardot}{\kern0pt}\ M{\isacharcomma}{\kern0pt}\ {\isacharbrackleft}{\kern0pt}x{\isacharbrackright}{\kern0pt}\ {\isacharat}{\kern0pt}\ env\ {\isasymTurnstile}\ {\isasymphi}{\isacharparenright}{\kern0pt}{\isacharparenright}{\kern0pt}\isasep\isanewline\isanewline%
{\isasymphi}\ {\isasymin}\ formula\ {\isasymLongrightarrow}\isanewline
{\isacharparenleft}{\kern0pt}M{\isacharcomma}{\kern0pt}\ {\isacharbrackleft}{\kern0pt}{\isacharbrackright}{\kern0pt}\ {\isasymTurnstile}\ {\isasymcdot}Replacement{\isacharparenleft}{\kern0pt}{\isasymphi}{\isacharparenright}{\kern0pt}{\isasymcdot}{\isacharparenright}{\kern0pt}\ {\isasymlongleftrightarrow}\ {\isacharparenleft}{\kern0pt}{\isasymforall}env{\isachardot}{\kern0pt}\ replacement{\isacharunderscore}{\kern0pt}assm{\isacharparenleft}{\kern0pt}M{\isacharcomma}{\kern0pt}\ env{\isacharcomma}{\kern0pt}\ {\isasymphi}{\isacharparenright}{\kern0pt}{\isacharparenright}\isanewline\isanewline%
choice{\isacharunderscore}{\kern0pt}ax{\isacharparenleft}{\kern0pt}{\isacharhash}{\kern0pt}{\isacharhash}{\kern0pt}A{\isacharparenright}{\kern0pt}\ {\isasymlongleftrightarrow}\ A{\isacharcomma}{\kern0pt}\ {\isacharbrackleft}{\kern0pt}{\isacharbrackright}{\kern0pt}\ {\isasymTurnstile}\ {\isasymcdot}AC{\isasymcdot}%
\end{isabelle}%

Finally, the axiom sets are defined as follows.

\begin{isabelle}%
ZF{\isacharunderscore}{\kern0pt}fin\ {\isasymequiv}\isanewline
{\isacharbraceleft}{\kern0pt}{\isasymcdot}Extensionality{\isasymcdot}{\isacharcomma}{\kern0pt}\ {\isasymcdot}Foundation{\isasymcdot}{\isacharcomma}{\kern0pt}\ {\isasymcdot}Pairing{\isasymcdot}{\isacharcomma}{\kern0pt}\ {\isasymcdot}Union\ Ax{\isasymcdot}{\isacharcomma}{\kern0pt}\ {\isasymcdot}Infinity{\isasymcdot}{\isacharcomma}{\kern0pt}\isanewline
\isaindent{{\isacharbraceleft}{\kern0pt}}{\isasymcdot}Powerset\ Ax{\isasymcdot}{\isacharbraceright}{\kern0pt}\isasep\isanewline\isanewline%
ZF{\isacharunderscore}{\kern0pt}schemes\ {\isasymequiv}\isanewline
{\isacharbraceleft}{\kern0pt}{\isasymcdot}Separation{\isacharparenleft}{\kern0pt}p{\isacharparenright}{\kern0pt}{\isasymcdot}\ {\isachardot}{\kern0pt}\ p\ {\isasymin}\ formula{\isacharbraceright}{\kern0pt}\ {\isasymunion}\ {\isacharbraceleft}{\kern0pt}{\isasymcdot}Replacement{\isacharparenleft}{\kern0pt}p{\isacharparenright}{\kern0pt}{\isasymcdot}\ {\isachardot}{\kern0pt}\ p\ {\isasymin}\ formula{\isacharbraceright}{\kern0pt}\isasep\isanewline\isanewline%
{\isasymcdot}Z{\isasymcdot}\ {\isasymequiv}\ ZF{\isacharunderscore}{\kern0pt}fin\ {\isasymunion}\ {\isacharbraceleft}{\kern0pt}{\isasymcdot}Separation{\isacharparenleft}{\kern0pt}p{\isacharparenright}{\kern0pt}{\isasymcdot}\ {\isachardot}{\kern0pt}\ p\ {\isasymin}\ formula{\isacharbraceright}{\kern0pt}\isasep\isanewline%
ZC\ {\isasymequiv}\ {\isasymcdot}Z{\isasymcdot}\ {\isasymunion}\ {\isacharbraceleft}{\kern0pt}{\isasymcdot}AC{\isasymcdot}{\isacharbraceright}{\kern0pt}\isasep\isanewline%
ZF\ {\isasymequiv}\ ZF{\isacharunderscore}{\kern0pt}schemes\ {\isasymunion}\ ZF{\isacharunderscore}{\kern0pt}fin\isasep\isanewline%
ZFC\ {\isasymequiv}\ ZF\ {\isasymunion}\ {\isacharbraceleft}{\kern0pt}{\isasymcdot}AC{\isasymcdot}{\isacharbraceright}{\kern0pt}%
\end{isabelle}%

\subsection{Relativization of infinitary arithmetic\label{sec:relative-arith}%
}
In order to state the defining property of the relative
equipotence relation, we work under the assumptions of the
locale \isa{M{\isacharunderscore}{\kern0pt}cardinals}. They comprise a finite set
of instances of Separation and Replacement to prove
closure properties of the transitive class \isa{M}.%
\begin{isabelle}
\isacommand{lemma}\isamarkupfalse%
\ {\isacharparenleft}{\kern0pt}\isakeyword{in}\ M{\isacharunderscore}{\kern0pt}cardinals{\isacharparenright}{\kern0pt}\ eqpoll{\isacharunderscore}{\kern0pt}def{\isacharprime}{\kern0pt}{\isacharcolon}{\kern0pt}\isanewline
\ \ \isakeyword{assumes}\ {\isachardoublequoteopen}M{\isacharparenleft}{\kern0pt}A{\isacharparenright}{\kern0pt}{\isachardoublequoteclose}\ {\isachardoublequoteopen}M{\isacharparenleft}{\kern0pt}B{\isacharparenright}{\kern0pt}{\isachardoublequoteclose}\ \isakeyword{shows}\ {\isachardoublequoteopen}A\ {\isasymapprox}\isactrlbsup M\isactrlesup \ B\ {\isasymlongleftrightarrow}\ {\isacharparenleft}{\kern0pt}{\isasymexists}f{\isacharbrackleft}{\kern0pt}M{\isacharbrackright}{\kern0pt}{\isachardot}{\kern0pt}\ f\ {\isasymin}\ bij{\isacharparenleft}{\kern0pt}A{\isacharcomma}{\kern0pt}B{\isacharparenright}{\kern0pt}{\isacharparenright}{\kern0pt}{\isachardoublequoteclose}
\end{isabelle}

%
Below, $\mu$ denotes the minimum operator on the ordinals.%
\begin{isabelle}
  \isacommand{lemma}\isamarkupfalse%
\ cardinalities{\isacharunderscore}{\kern0pt}defs{\isacharcolon}{\kern0pt}\isanewline
\ \ \isakeyword{fixes}\ M{\isacharcolon}{\kern0pt}{\isacharcolon}{\kern0pt}{\isachardoublequoteopen}i{\isasymRightarrow}o{\isachardoublequoteclose}\isanewline
\ \ \isakeyword{shows}\isanewline
\ \ \ \ {\isachardoublequoteopen}{\isacharbar}{\kern0pt}A{\isacharbar}{\kern0pt}\isactrlbsup M\isactrlesup \ {\isasymequiv}\ {\isasymmu}\ i{\isachardot}{\kern0pt}\ M{\isacharparenleft}{\kern0pt}i{\isacharparenright}{\kern0pt}\ {\isasymand}\ i\ {\isasymapprox}\isactrlbsup M\isactrlesup \ A{\isachardoublequoteclose}\isanewline
\ \ \ \ {\isachardoublequoteopen}Card\isactrlbsup M\isactrlesup {\isacharparenleft}{\kern0pt}{\isasymalpha}{\isacharparenright}{\kern0pt}\ {\isasymequiv}\ {\isasymalpha}\ {\isacharequal}{\kern0pt}\ {\isacharbar}{\kern0pt}{\isasymalpha}{\isacharbar}{\kern0pt}\isactrlbsup M\isactrlesup {\isachardoublequoteclose}\isanewline
\ \ \ \ {\isachardoublequoteopen}{\isasymkappa}\isactrlbsup {\isasymup}{\isasymnu}{\isacharcomma}{\kern0pt}M\isactrlesup \ {\isasymequiv}\ {\isacharbar}{\kern0pt}{\isasymnu}\ {\isasymrightarrow}\isactrlbsup M\isactrlesup \ {\isasymkappa}{\isacharbar}{\kern0pt}\isactrlbsup M\isactrlesup {\isachardoublequoteclose}\isanewline
\ \ \ \ {\isachardoublequoteopen}{\isacharparenleft}{\kern0pt}{\isasymkappa}\isactrlsup {\isacharplus}{\kern0pt}{\isacharparenright}{\kern0pt}\isactrlbsup M\isactrlesup \ {\isasymequiv}\ {\isasymmu}\ x{\isachardot}{\kern0pt}\ M{\isacharparenleft}{\kern0pt}x{\isacharparenright}{\kern0pt}\ {\isasymand}\ Card\isactrlbsup M\isactrlesup {\isacharparenleft}{\kern0pt}x{\isacharparenright}{\kern0pt}\ {\isasymand}\ {\isasymkappa}\ {\isacharless}{\kern0pt}\ x{\isachardoublequoteclose}
\end{isabelle}
Analogous to the previous Lemma
\isa{eqpoll{\isacharunderscore}{\kern0pt}def{\isacharprime}{\kern0pt}},
the next lemma holds under
the assumptions of the locale \isa{M{\isacharunderscore}{\kern0pt}aleph}. The axiom instances
included are sufficient to state and prove the defining
properties of the relativized \isa{Aleph} function
(in particular, the required ability to perform transfinite recursions).%
\begin{isabelle}%
\isacommand{context}\isamarkupfalse%
\ M{\isacharunderscore}{\kern0pt}aleph\isanewline
\isakeyword{begin}%
\isanewline
\isanewline
{\isasymaleph}\isactrlbsub {\isadigit{0}}\isactrlesub \isactrlbsup M\isactrlesup \ {\isacharequal}{\kern0pt}\ {\isasymomega}\isasep\isanewline%
{\isasymlbrakk}Ord{\isacharparenleft}{\kern0pt}{\isasymalpha}{\isacharparenright}{\kern0pt}{\isacharsemicolon}{\kern0pt}\ M{\isacharparenleft}{\kern0pt}{\isasymalpha}{\isacharparenright}{\kern0pt}{\isasymrbrakk}\ {\isasymLongrightarrow}\ {\isasymaleph}\isactrlbsub succ{\isacharparenleft}{\kern0pt}{\isasymalpha}{\isacharparenright}{\kern0pt}\isactrlesub \isactrlbsup M\isactrlesup \ {\isacharequal}{\kern0pt}\ {\isacharparenleft}{\kern0pt}{\isasymaleph}\isactrlbsub {\isasymalpha}\isactrlesub \isactrlbsup M\isactrlesup \isactrlsup {\isacharplus}{\kern0pt}{\isacharparenright}{\kern0pt}\isactrlbsup M\isactrlesup \isasep\isanewline%
{\isasymlbrakk}Limit{\isacharparenleft}{\kern0pt}{\isasymalpha}{\isacharparenright}{\kern0pt}{\isacharsemicolon}{\kern0pt}\ M{\isacharparenleft}{\kern0pt}{\isasymalpha}{\isacharparenright}{\kern0pt}{\isasymrbrakk}\ {\isasymLongrightarrow}\ {\isasymaleph}\isactrlbsub {\isasymalpha}\isactrlesub \isactrlbsup M\isactrlesup \ {\isacharequal}{\kern0pt}\ {\isacharparenleft}{\kern0pt}{\isasymUnion}j{\isasymin}{\isasymalpha}{\isachardot}{\kern0pt}\ {\isasymaleph}\isactrlbsub j\isactrlesub \isactrlbsup M\isactrlesup {\isacharparenright}{\kern0pt}%
\end{isabelle}%
\isacommand{end}\isamarkupfalse%
\ %
\isamarkupcmt{\isa{M{\isacharunderscore}{\kern0pt}aleph}%
}
\begin{isabelle}
\isacommand{lemma}\isamarkupfalse%
\ ContHyp{\isacharunderscore}{\kern0pt}rel{\isacharunderscore}{\kern0pt}def{\isacharprime}{\kern0pt}{\isacharcolon}{\kern0pt}\isanewline
\ \ \isakeyword{fixes}\ N{\isacharcolon}{\kern0pt}{\isacharcolon}{\kern0pt}{\isachardoublequoteopen}i{\isasymRightarrow}o{\isachardoublequoteclose}\isanewline
\ \ \isakeyword{shows}\isanewline
\ \ \ \ {\isachardoublequoteopen}CH\isactrlbsup N\isactrlesup \ {\isasymequiv}\ {\isasymaleph}\isactrlbsub {\isadigit{1}}\isactrlesub \isactrlbsup N\isactrlesup \ {\isacharequal}{\kern0pt}\ {\isadigit{2}}\isactrlbsup {\isasymup}{\isasymaleph}\isactrlbsub {\isadigit{0}}\isactrlesub \isactrlbsup N\isactrlesup {\isacharcomma}{\kern0pt}N\isactrlesup {\isachardoublequoteclose}
\end{isabelle}

%
%
Under appropriate hypotheses (this time, from the locale \isa{M{\isacharunderscore}{\kern0pt}ZF{\isacharunderscore}{\kern0pt}library}),
   \isa{CH\isactrlbsup M\isactrlesup } is equivalent to its fully relational version \isa{is{\isacharunderscore}{\kern0pt}ContHyp}.
    As a sanity check, we see that if the transitive class is indeed \isa{{\isasymV}},
    we recover the original $\CH$.%
\begin{isabelle}%
M{\isacharunderscore}{\kern0pt}ZF{\isacharunderscore}{\kern0pt}library{\isacharparenleft}{\kern0pt}M{\isacharparenright}{\kern0pt}\ {\isasymLongrightarrow}\ is{\isacharunderscore}{\kern0pt}ContHyp{\isacharparenleft}{\kern0pt}M{\isacharparenright}{\kern0pt}\ {\isasymlongleftrightarrow}\ CH\isactrlbsup M\isactrlesup \isasep\isanewline%
is{\isacharunderscore}{\kern0pt}ContHyp{\isacharparenleft}{\kern0pt}{\isasymV}{\isacharparenright}{\kern0pt}\ {\isasymlongleftrightarrow}\ {\isasymaleph}\isactrlbsub {\isadigit{1}}\isactrlesub \ {\isacharequal}{\kern0pt}\ {\isadigit{2}}\isactrlbsup {\isasymup}{\isasymaleph}\isactrlbsub {\isadigit{0}}\isactrlesub \isactrlesup %
\end{isabelle}%
In turn, the fully relational version evaluated on a nonempty
transitive \isa{A} is equivalent to the satisfaction of the
first-order formula \isa{{\isasymcdot}CH{\isasymcdot}} (since it
actually is a sentence, it does not depend on \isa{env}, which
appears only because the definition of $\models$ requires that argument).%
\begin{isabelle}%
{\isasymlbrakk}env\ {\isasymin}\ list{\isacharparenleft}{\kern0pt}A{\isacharparenright}{\kern0pt}{\isacharsemicolon}{\kern0pt}\ {\isadigit{0}}\ {\isasymin}\ A{\isasymrbrakk}\ {\isasymLongrightarrow}\ is{\isacharunderscore}{\kern0pt}ContHyp{\isacharparenleft}{\kern0pt}{\isacharhash}{\kern0pt}{\isacharhash}{\kern0pt}A{\isacharparenright}{\kern0pt}\ {\isasymlongleftrightarrow}\ A{\isacharcomma}{\kern0pt}\ env\ {\isasymTurnstile}\ {\isasymcdot}CH{\isasymcdot}%
\end{isabelle}%
%% \subsection{Forcing \label{sec:def-main-forcing}%
%% }
%% Our first milestone was to obtain a proper extension using forcing.
%% Its original proof didn't required the previous developments involving
%% below.%
%% \begin{isabelle}%
%% {\isasymlbrakk}M\ {\isasymapprox}\ {\isasymomega}{\isacharsemicolon}{\kern0pt}\ Transset{\isacharparenleft}{\kern0pt}M{\isacharparenright}{\kern0pt}{\isacharsemicolon}{\kern0pt}\ M\ {\isasymTurnstile}\ ZF{\isasymrbrakk}\isanewline
%% {\isasymLongrightarrow}\ {\isasymexists}N{\isachardot}{\kern0pt}\ M\ {\isasymsubseteq}\ N\ {\isasymand}\isanewline
%% \isaindent{{\isasymLongrightarrow}\ {\isasymexists}N{\isachardot}{\kern0pt}\ }N\ {\isasymapprox}\ {\isasymomega}\ {\isasymand}\isanewline
%% \isaindent{{\isasymLongrightarrow}\ {\isasymexists}N{\isachardot}{\kern0pt}\ }Transset{\isacharparenleft}{\kern0pt}N{\isacharparenright}{\kern0pt}\ {\isasymand}\isanewline
%% \isaindent{{\isasymLongrightarrow}\ {\isasymexists}N{\isachardot}{\kern0pt}\ }N\ {\isasymTurnstile}\ ZF\ {\isasymand}\isanewline
%% \isaindent{{\isasymLongrightarrow}\ {\isasymexists}N{\isachardot}{\kern0pt}\ }M\ {\isasymnoteq}\ N\ {\isasymand}\ {\isacharparenleft}{\kern0pt}{\isasymforall}{\isasymalpha}{\isachardot}{\kern0pt}\ Ord{\isacharparenleft}{\kern0pt}{\isasymalpha}{\isacharparenright}{\kern0pt}\ {\isasymlongrightarrow}\ {\isasymalpha}\ {\isasymin}\ M\ {\isasymlongleftrightarrow}\ {\isasymalpha}\ {\isasymin}\ N{\isacharparenright}{\kern0pt}\ {\isasymand}\ {\isacharparenleft}{\kern0pt}{\isacharparenleft}{\kern0pt}M{\isacharcomma}{\kern0pt}\ {\isacharbrackleft}{\kern0pt}{\isacharbrackright}{\kern0pt}\ {\isasymTurnstile}\ {\isasymcdot}AC{\isasymcdot}{\isacharparenright}{\kern0pt}\ {\isasymlongrightarrow}\ N\ {\isasymTurnstile}\ ZFC{\isacharparenright}{\kern0pt}%
%% \end{isabelle}%
%% We can finally state our main results, namely, the existence of models
%% for $\ZFC + \CH$ and $\ZFC + \neg\CH$ under the assumption of a ctm of $\ZFC$.%
%% \begin{isabelle}%
%% {\isasymlbrakk}M\ {\isasymapprox}\ {\isasymomega}{\isacharsemicolon}{\kern0pt}\ Transset{\isacharparenleft}{\kern0pt}M{\isacharparenright}{\kern0pt}{\isacharsemicolon}{\kern0pt}\ M\ {\isasymTurnstile}\ ZFC{\isasymrbrakk}\isanewline
%% {\isasymLongrightarrow}\ {\isasymexists}N{\isachardot}{\kern0pt}\ M\ {\isasymsubseteq}\ N\ {\isasymand}\isanewline
%% \isaindent{{\isasymLongrightarrow}\ {\isasymexists}N{\isachardot}{\kern0pt}\ }N\ {\isasymapprox}\ {\isasymomega}\ {\isasymand}\isanewline
%% \isaindent{{\isasymLongrightarrow}\ {\isasymexists}N{\isachardot}{\kern0pt}\ }Transset{\isacharparenleft}{\kern0pt}N{\isacharparenright}{\kern0pt}\ {\isasymand}\ N\ {\isasymTurnstile}\ ZFC\ {\isasymunion}\ {\isacharbraceleft}{\kern0pt}{\isasymcdot}{\isasymnot}{\isasymcdot}CH{\isasymcdot}{\isasymcdot}{\isacharbraceright}{\kern0pt}\ {\isasymand}\ {\isacharparenleft}{\kern0pt}{\isasymforall}{\isasymalpha}{\isachardot}{\kern0pt}\ Ord{\isacharparenleft}{\kern0pt}{\isasymalpha}{\isacharparenright}{\kern0pt}\ {\isasymlongrightarrow}\ {\isasymalpha}\ {\isasymin}\ M\ {\isasymlongleftrightarrow}\ {\isasymalpha}\ {\isasymin}\ N{\isacharparenright}{\kern0pt}%
%% \end{isabelle}%
%% \begin{isabelle}%
%% {\isasymlbrakk}M\ {\isasymapprox}\ {\isasymomega}{\isacharsemicolon}{\kern0pt}\ Transset{\isacharparenleft}{\kern0pt}M{\isacharparenright}{\kern0pt}{\isacharsemicolon}{\kern0pt}\ M\ {\isasymTurnstile}\ ZFC{\isasymrbrakk}\isanewline
%% {\isasymLongrightarrow}\ {\isasymexists}N{\isachardot}{\kern0pt}\ M\ {\isasymsubseteq}\ N\ {\isasymand}\isanewline
%% \isaindent{{\isasymLongrightarrow}\ {\isasymexists}N{\isachardot}{\kern0pt}\ }N\ {\isasymapprox}\ {\isasymomega}\ {\isasymand}\isanewline
%% \isaindent{{\isasymLongrightarrow}\ {\isasymexists}N{\isachardot}{\kern0pt}\ }Transset{\isacharparenleft}{\kern0pt}N{\isacharparenright}{\kern0pt}\ {\isasymand}\ N\ {\isasymTurnstile}\ ZFC\ {\isasymunion}\ {\isacharbraceleft}{\kern0pt}{\isasymcdot}CH{\isasymcdot}{\isacharbraceright}{\kern0pt}\ {\isasymand}\ {\isacharparenleft}{\kern0pt}{\isasymforall}{\isasymalpha}{\isachardot}{\kern0pt}\ Ord{\isacharparenleft}{\kern0pt}{\isasymalpha}{\isacharparenright}{\kern0pt}\ {\isasymlongrightarrow}\ {\isasymalpha}\ {\isasymin}\ M\ {\isasymlongleftrightarrow}\ {\isasymalpha}\ {\isasymin}\ N{\isacharparenright}{\kern0pt}%
%% \end{isabelle}%
%% In the above three statements, the function \isa{ground{\isacharunderscore}{\kern0pt}repl{\isacharunderscore}{\kern0pt}fm}
%% takes an element \isa{{\isasymphi}} of \isa{formula} and returns the
%% replacement instance in the ground model that produces the
%% \isa{{\isasymphi}}-replacement instance in the generic extension. The next
%% result is stated in the context \isa{G{\isacharunderscore}{\kern0pt}generic{\isadigit{1}}}, which assumes
%% the existence of a generic filter.
%% %
%% \begin{isabelle}%
%% \isacommand{context}\isamarkupfalse%
%% \ G{\isacharunderscore}{\kern0pt}generic{\isadigit{1}}\isanewline
%% \isakeyword{begin}\isanewline
%% \isanewline
%% {\isasymlbrakk}{\isasymphi}\ {\isasymin}\ formula{\isacharsemicolon}{\kern0pt}\ M{\isacharcomma}{\kern0pt}\ {\isacharbrackleft}{\kern0pt}{\isacharbrackright}{\kern0pt}\ {\isasymTurnstile}\ {\isasymcdot}Replacement{\isacharparenleft}{\kern0pt}ground{\isacharunderscore}{\kern0pt}repl{\isacharunderscore}{\kern0pt}fm{\isacharparenleft}{\kern0pt}{\isasymphi}{\isacharparenright}{\kern0pt}{\isacharparenright}{\kern0pt}{\isasymcdot}{\isasymrbrakk}\isanewline
%% {\isasymLongrightarrow}\ M{\isacharbrackleft}{\kern0pt}G{\isacharbrackright}{\kern0pt}{\isacharcomma}{\kern0pt}\ {\isacharbrackleft}{\kern0pt}{\isacharbrackright}{\kern0pt}\ {\isasymTurnstile}\ {\isasymcdot}Replacement{\isacharparenleft}{\kern0pt}{\isasymphi}{\isacharparenright}{\kern0pt}{\isasymcdot}%
%% \end{isabelle}%
%% \isacommand{end}\isamarkupfalse%
%% \ %
%% \isamarkupcmt{\isa{G{\isacharunderscore}{\kern0pt}generic{\isadigit{1}}}%
%% }

%%% Local Variables:
%%% mode: latex
%%% TeX-master: "independence_ch_isabelle"
%%% ispell-local-dictionary: "american"
%%% End:


\end{document}

%%% Local Variables: 
%%% mode: latex
%%% ispell-local-dictionary: "american"
%%% End: 
