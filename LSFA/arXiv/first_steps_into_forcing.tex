\documentclass{article}
\usepackage{isabelle,isabellesym}
\usepackage{hyperref}
\usepackage[numbers]{natbib}

\usepackage{amsmath}
%\usepackage{amsthm}
\usepackage{amsfonts}
\usepackage{amssymb}
%\usepackage{bbm}  % Para el \bb{1}
%\usepackage[numbers]{natbib}
\usepackage{enumitem}
\usepackage{babel}
%\usepackage{babelbib}
\usepackage{multidef}
\usepackage{verbatim}
\usepackage{stmaryrd} %% para \llbracket
%%
%% \usepackage[bottom=2cm, top=2cm, left=2cm, right=2cm]{geometry}
%% \usepackage{titling}
%% \setlength{\droptitle}{-10ex} 
%%
\renewcommand{\o}{\vee}
\renewcommand{\O}{\bigvee}
\newcommand{\y}{\wedge}
\newcommand{\Y}{\bigwedge}
\newcommand{\limp}{\rightarrow}
\newcommand{\lsii}{\leftrightarrow}
%%

\DeclareMathOperator{\cf}{cf}
\DeclareMathOperator{\dom}{dom}
\DeclareMathOperator{\im}{img}
\DeclareMathOperator{\Fn}{Fn}
\DeclareMathOperator{\rk}{rk}
\DeclareMathOperator{\mos}{mos}
\DeclareMathOperator{\trcl}{trcl}
\DeclareMathOperator*{\diag}{\bigtriangleup}
\DeclareMathOperator{\Con}{Con}
\DeclareMathOperator{\Club}{Club}


\newcommand{\modelo}[1]{\mathbf{#1}}
\newcommand{\axiomas}[1]{\mathit{#1}}
\newcommand{\clase}[1]{\mathsf{#1}}
\newcommand{\poset}[1]{\mathbb{#1}}
\newcommand{\operador}[1]{\mathbf{#1}}

%% \newcommand{\Lim}{\clase{Lim}}
%% \newcommand{\Reg}{\clase{Reg}}
%% \newcommand{\Card}{\clase{Card}}
%% \newcommand{\On}{\clase{On}}
%% \newcommand{\WF}{\clase{WF}}
%% \newcommand{\HF}{\clase{HF}}
%% \newcommand{\HC}{\clase{HC}}
%%
%% El siguiente comando reemplaza todos los anteriores:
%%
\multidef{\clase{#1}}{Card,HC,HF,Lim,On->Ord,Reg,WF,Ord}
\newcommand{\ON}{\On}

%% En lugar de usar todo el paquete bbm:
\DeclareMathAlphabet{\mathbbm}{U}{bbm}{m}{n} 
\newcommand{\1}{\mathbbm{1}}

%%
%% \newcommand{\calD}{\mathcal{D}}
%% \newcommand{\calS}{\mathcal{S}}
%% \newcommand{\calU}{\mathcal{U}}
%% \newcommand{\calB}{\mathcal{B}}
%% \newcommand{\calL}{\mathcal{L}}
%% \newcommand{\calF}{\mathcal{F}}
%% \newcommand{\calT}{\mathcal{T}}
%% \newcommand{\calW}{\mathcal{W}}
%% \newcommand{\calA}{\mathcal{A}}
%%
%% El siguiente comando reemplaza todos los anteriores:
%%
\multidef[prefix=cal]{\mathcal{#1}}{A-Z}
%%
%% \newcommand{\A}{\modelo{A}}
%% \newcommand{\BB}{\modelo{B}}
%% \newcommand{\ZZ}{\modelo{Z}}
%% \newcommand{\PP}{\modelo{P}}
%% \newcommand{\QQ}{\modelo{Q}}
%% \newcommand{\RR}{\modelo{R}}
%%
%% El siguiente comando reemplaza todos los anteriores:
%%
\multidef{\modelo{#1}}{A,BB->B,CC->C,NN->N,PP->P,QQ->Q,RR->R,ZZ->Z}

\multidef[prefix=p]{\mathbb{#1}}{A-Z}
%% \newcommand{\B}{\modelo{B}}
%% \newcommand{\C}{\modelo{C}}
%% \newcommand{\F}{\modelo{F}}
%% \newcommand{\D}{\modelo{D}}

\newcommand{\Th}{\mb{Th}}
\newcommand{\Mod}{\mb{Mod}}

\newcommand{\Se}{\operador{S^\prec}}
\newcommand{\Pu}{\operador{P_u}}
\renewcommand{\Pr}{\operador{P_R}}
\renewcommand{\H}{\operador{H}}
\renewcommand{\S}{\operador{S}}
\newcommand{\I}{\operador{I}}
\newcommand{\E}{\operador{E}}

\newcommand{\se}{\preccurlyeq}
\newcommand{\ee}{\succ}
\newcommand{\id}{\approx}
\newcommand{\subm}{\subseteq}
\newcommand{\ext}{\supseteq}
\newcommand{\iso}{\cong}
%%
\renewcommand{\emptyset}{\varnothing}
\newcommand{\rel}{\mathcal{R}}
\newcommand{\Pow}{\mathop{\mathcal{P}}}
\renewcommand{\P}{\Pow}
\newcommand{\BP}{\mathrm{BP}}
\newcommand{\func}{\rightarrow}
\newcommand{\ord}{\mathrm{Ord}}
\newcommand{\R}{\mathbb{R}}
\newcommand{\N}{\mathbb{N}}
\newcommand{\Z}{\mathbb{Z}}
\renewcommand{\I}{\mathbb{I}}
\newcommand{\Q}{\mathbb{Q}}
\newcommand{\B}{\mathbf{B}}
\newcommand{\<}{\langle}
\renewcommand{\>}{\rangle}
\newcommand{\lb}{\langle}
\newcommand{\rb}{\rangle}
\newcommand{\impl}{\rightarrow}
\newcommand{\ent}{\Rightarrow}
\newcommand{\tne}{\Leftarrow}
\newcommand{\sii}{\Leftrightarrow}
\renewcommand{\phi}{\varphi}
\newcommand{\phis}{{\varphi^*}}
\renewcommand{\th}{\theta}
\newcommand{\Lda}{\Lambda}
\newcommand{\La}{\Lambda}
\newcommand{\lda}{\lambda}
\newcommand{\ka}{\kappa}
\newcommand{\del}{\delta}
\newcommand{\de}{\delta}
\newcommand{\ze}{\zeta}
%\newcommand{\ }{\ }
\newcommand{\la}{\lambda}
\newcommand{\al}{\alpha}
\newcommand{\be}{\beta}
\newcommand{\ga}{\gamma}
\newcommand{\Ga}{\Gamma}
\newcommand{\ep}{\varepsilon}
\newcommand{\De}{\Delta}
\newcommand{\defi}{\mathrel{\mathop:}=}
\newcommand{\forces}{\Vdash}
%\newcommand{\ap}{\mathbin{\wideparen{\ }}}
\newcommand{\Tree}{{\mathrm{Tr}_\N}}
\newcommand{\PTree}{{\mathrm{PTr}_\N}}
\newcommand{\NWO}{\mathit{NWO}}
\newcommand{\Suc}{{\N^{<\N}}}%
\newcommand{\init}{\mathsf{i}}
\newcommand{\ap}{\mathord{^\smallfrown}}
\newcommand{\Cantor}{\mathcal{C}}
%\newcommand{\C}{\Cantor}
\newcommand{\Baire}{\mathcal{N}}
\newcommand{\sig}{\ensuremath{\sigma}}
\newcommand{\fsig}{\ensuremath{F_\sigma}}
\newcommand{\gdel}{\ensuremath{G_\delta}}
\newcommand{\Sig}{\ensuremath{\boldsymbol{\Sigma}}}
\newcommand{\bPi}{\ensuremath{\boldsymbol{\Pi}}}
\newcommand{\Del}{\ensuremath{\boldsymbol\Delta}}
%\renewcommand{\F}{\operador{F}}
\newcommand{\ths}{{\theta^*}}
\newcommand{\om}{\ensuremath{\omega}}
%\renewcommand{\c}{\complement}
\newcommand{\comp}{\mathsf{c}}
\newcommand{\co}[1]{\left(#1\right)^\comp}
\newcommand{\len}[1]{\left|#1\right|}
\DeclareMathOperator{\tlim}{\overline{\mathrm{TLim}}}
\newcommand{\card}[1]{{\left|#1\right|}}
\newcommand{\bigcard}[1]{{\bigl|#1\bigr|}}
%
% Cardinality
%
\newcommand{\lec}{\leqslant_c}
\newcommand{\gec}{\geqslant_c}
\newcommand{\lc}{<_c}
\newcommand{\gc}{>_c}
\newcommand{\eqc}{=_c}
\newcommand{\biy}{\approx}
\newcommand*{\ale}[1]{\aleph_{#1}}
%
\newcommand{\Zerm}{\axiomas{Z}}
\newcommand{\ZC}{\axiomas{ZC}}
\newcommand{\AC}{\axiomas{AC}}
\newcommand{\DC}{\axiomas{DC}}
\newcommand{\MA}{\axiomas{MA}}
\newcommand{\CH}{\axiomas{CH}}
\newcommand{\ZFC}{\axiomas{ZFC}}
\newcommand{\ZF}{\axiomas{ZF}}
\newcommand{\Inf}{\axiomas{Inf}}
%
% Cardinal characteristics
%
\newcommand{\cont}{\mathfrak{c}}
\newcommand{\spl}{\mathfrak{s}}
\newcommand{\bound}{\mathfrak{b}}
\newcommand{\mad}{\mathfrak{a}}
\newcommand{\tower}{\mathfrak{t}}
%
\renewcommand{\hom}[2]{{}^{#1}\hskip-0.116ex{#2}}
\newcommand{\pred}[1][{}]{\mathop{\mathrm{pred}_{#1}}}
%% Postfix operator with supressable space:
%% \newcommand*{\iseg}{\relax\ifnum\lastnodetype>0 \mskip\medmuskip\fi{\downarrow}} %
\newcommand*{\iseg}{{\downarrow}}
\newcommand{\rr}{\mathrel{R}}
\newcommand{\restr}{\upharpoonright}
%\newcommand{\type}{\mathtt{}}
\newcommand{\app}{\mathop{\mathrm{Aprox}}}
\newcommand{\hess}{\triangleleft}
\newcommand{\bx}{\bar{x}}
\newcommand{\by}{\bar{y}}
\newcommand{\bz}{\bar{z}}
\newcommand{\union}{\mathop{\textstyle\bigcup}}
\newcommand{\sm}{\setminus}
\newcommand{\sbq}{\subseteq}
\newcommand{\nsbq}{\subseteq}
\newcommand{\mty}{\emptyset}
\newcommand{\dimg}{\text{\textup{``}}} % direct image
\newcommand{\quine}[1]{\ulcorner{\!#1\!}\urcorner}
%\newcommand{\ntrm}[1]{\textsl{\textbf{#1}}}
\newcommand{\Null}{\calN\!\mathit{ull}}
\DeclareMathOperator{\club}{Club}
\DeclareMathOperator{\otp}{otp}

%%%%%%%%%%%%%%%%%%%%%%%%%
% Variant aleph, beth, etc
% From http://tex.stackexchange.com/q/170476/69595
\makeatletter
\@ifpackageloaded{txfonts}\@tempswafalse\@tempswatrue
\if@tempswa
  \DeclareFontFamily{U}{txsymbols}{}
  \DeclareFontFamily{U}{txAMSb}{}
  \DeclareSymbolFont{txsymbols}{OMS}{txsy}{m}{n}
  \SetSymbolFont{txsymbols}{bold}{OMS}{txsy}{bx}{n}
  \DeclareFontSubstitution{OMS}{txsy}{m}{n}
  \DeclareSymbolFont{txAMSb}{U}{txsyb}{m}{n}
  \SetSymbolFont{txAMSb}{bold}{U}{txsyb}{bx}{n}
  \DeclareFontSubstitution{U}{txsyb}{m}{n}
  \DeclareMathSymbol{\aleph}{\mathord}{txsymbols}{64}
  \DeclareMathSymbol{\beth}{\mathord}{txAMSb}{105}
  \DeclareMathSymbol{\gimel}{\mathord}{txAMSb}{106}
  \DeclareMathSymbol{\daleth}{\mathord}{txAMSb}{107}
\fi
\makeatother

%%%%%%%%%%%%%%%%%%%%%%%%%%%%%%%%%%%%%%%%%%%%%%%%%%%%%%%%%%%%
%%
%% Theorem Environments
%%
%% \newtheorem{theorem}{Theorem}
%% \newtheorem{lemma}[theorem]{Lemma}
%% \newtheorem{prop}[theorem]{Proposition}
%% \newtheorem{corollary}[theorem]{Corollary}
%% \newtheorem{claim}{Claim}
%% \newtheorem*{claim*}{Claim}
%% \theoremstyle{definition}
%% \newtheorem{definition}[theorem]{Definition}
%% \newtheorem{remark}[theorem]{Remark}
%% \newtheorem{example}[theorem]{Example}
%% \theoremstyle{remark}
%% \newtheorem*{remark*}{Remark}
%%
%%%%%%%%%%%%%%%%%%%%%%%%%%%%%%%%%%%%%%%%%%%%%%%%%%%%%%%%%%%%%%%%%%%%%%

%% \newenvironment{inducc}{\begin{list}{}{\itemindent=2.5em \labelwidth=4em}}{\end{list}}
%% \newcommand{\caso}[1]{\item[\fbox{#1}]}
\newenvironment{proofofclaim}{\begin{proof}[Proof of Claim]}{\end{proof}}


%%% Local Variables: 
%%% mode: latex
%%% TeX-master: "first_steps_into_forcing"
%%% End: 


\begin{document}
\title{First steps towards a formalization of Forcing}
\author{Emmanuel Gunther
  \and 
  Miguel Pagano
  \and 
  Pedro S\'anchez Terraf}
\maketitle

\begin{abstract} 
  We lay the ground for an Isabelle/ZF formalization of Cohen's technique of
  \emph{forcing}. We formalize the definition of forcing notions as
  preorders with top, dense subsets, and generic filters. We formalize
  a version of the principle of Dependent Choices and using it
  we prove the Rasiowa-Sikorski lemma on the existence of generic filters.
  
  Given a transitive set $M$, we define its generic extension $M[G]$,
  the canonical names for elements of $M$, and finally show that if $M$
  satisfies the axiom of pairing, then $M[G]$ also does. We also prove
  $M[G]$ is transitive.
\end{abstract}
\keywords{
Isabelle/ZF, forcing, preorder, Rasiowa-Sikorski lemma, names, generic extension.
}

%%%%%%%%%%%%%%%%%%%%%%%%%%%%%%%%%%%%%%%%%%%%%%%%%%%%%%%%%%%%%%%%%%%%%%%%%%%%%%%%
\section{Introduction}
\label{sec:introduction}

This paper is the culmination of our project on the computerized
formalization of the undecidability of the Continuum Hypothesis
($\CH$) from Zermelo-Fraenkel set theory with Choice ($\ZFC$), under the
assumption of the existence of a countable transitive model (ctm) of
$\ZFC$. In contrast to our reports of the previous steps towards this
goal
\cite{2018arXiv180705174G,2019arXiv190103313G,2020arXiv200109715G}, we
intend here to present our development to the mathematical logic
community. For this reason, we start with a general discussion around
the formalization of mathematics.

\subsection{Formalized mathematics}
The use of computers to assist the creation and verification of
mathematics has seen a steady grow. But the general awareness on the
matter still seems to be a bit scant (even among mathematicians
involved in foundations), and the venues devoted to the communication
of formalized mathematics are, mainly, computer science journals and
conferences: JAR, ITP, IJCAR, CPP, CICM, and others.

Nevertheless, the discussion about the subject in central mathematical
circles is increasing; there were some hints on the ICM2018 panel on
“machine-assisted” proofs
\cite{https://doi.org/10.48550/arxiv.1809.08062} and a lively
promotion by Kevin Buzzard, during his ICM2022 special plenary lecture
\cite{2021arXiv211211598B}.

%% These assistants provide several dialects, among which we single out:
%% \begin{enumerate}
%% \item Procedural: Useful for exploration/research.
%% \item Declarative: Only one that can be read by humans!
%% \end{enumerate}

Before we start an in-depth discussion, a point should be made clear:
A formalized proof is not the same as an \emph{automatic proof}. The
reader surely understands that, aside from results of a very specific sort, no current
technology allows us to write a reasonably complex (and correct)
theorem statement in a computer and expect to obtain a proof after hitting “Enter”, at
least not after a humanly feasible wait. On the other hand, it is
quite possible that the same reader has some mental image that
formalizing a proof requires making each application of Modus Ponens
explicit.

The fact is that \emph{proof assistants} are designed for the human prover to
be able to decompose a statement to be proved into smaller subgoals
which can actually be fed into some automatic tool. The balance between
what these tools are able to handle is not  easily appreciated by
intuition: Sometimes, ``trivial'' steps are not solved by them, which
can result in obvious frustration; but they would quickly solve some
goals that do not look like a ``mere computation.''

To appreciate the extent of mathematics formalizable, it is convenient to recall
some major successful projects, such as the Four Color Theorem
\cite{MR2463991}, the Odd Order Theorem
\cite{10.1007/978-3-642-39634-2_14}, and the proof the Kepler's
Conjecture \cite{MR3659768}. There is a vast mathematical corpus at
the Archive of Formal Proofs (AFP) based on Isabelle; and formalizations of
brand new mathematics like the Liquid Tensor Experiment
\cite{LTE2020,LTE2021} and the definition of perfectoid spaces \cite{10.1145/3372885.3373830}
have been achieved using Lean.

We will continue our description of proof assistants in
Section~\ref{sec:proof-assist-isabelle}. We kindly invite the reader
to enrich the previous exposition by reading the apt summary by
A.~Koutsoukou-Argyraki \cite{angeliki} and the interviews
therein; some of the experts consulted have also discussed
in \cite{2022arXiv220704779B} the status of formalized versus standard
proof in mathematics.

\subsection{Our achievements}
We formalized a model-theoretic rendition of forcing (Sect.~\ref{sec:forcing}), showing how to
construct proper extensions of ctms of $\ZF$ (respectively, with
$\AC$), and we formalized the basic forcing notions required to obtain
ctms of $\ZFC + \neg\CH$ and of $\ZFC + \CH$ (Sect.~\ref{sec:models-ch-negation}). No metatheoretic issues
(consistency, FOL calculi, etc) were formalized, so we were mainly
concerned with the mathematics of forcing. Nevertheless, by inspecting
the foundations underlying our proof assistant Isabelle
(Section~\ref{sec:isabelle-metalogic-meta}) it can be stated that our
formalization is a bona fide proof in $\ZF$ of the previous
constructions.

In order to reach our goals, we provided basic results that were
missing from Isabelle's $\ZF$ library, starting from ones
involving cardinal successors, countable sets, etc.
(Section~\ref{sec:extension-isabellezf}). We also extended the treatment of relativization of
set-theoretical concepts (Section~\ref{sec:tools-relativization}).
%% We redesigned Isabelle/ZF results on non-absolute concepts to work
%% relative to a class.

One added value that is obtained from the present formalization is
that we identified a handful of instances of Replacement which are
sufficient to set the forcing machinery up
(Section~\ref{sec:repl-instances}), on the basis of Zermelo set theory.
The eagerness to obtain this level of detail might be a consequence of
“an unnatural tendency to investigate, for the most part, trivial
minutiae of the formalism” on our part, as it was put by Cohen
\cite{zbMATH02012060}, but we would rather say that we were driven by
curiosity.

The code of our formalization can be accessed at the
AFP site, via the following link:
\begin{center}
  \url{https://www.isa-afp.org/entries/Independence_CH.html}
\end{center}

%%% Local Variables: 
%%% mode: latex
%%% TeX-master: "independence_ch_isabelle"
%%% ispell-local-dictionary: "american"
%%% End: 


% \input{why-formalize}

\section{Related work}
\label{sec:related-work}

\textbf{Reviewer's comments}
{\it
  \begin{itemize}
  \item There, it would be appropriate to contrast what was done in
    Paulson's work on constructibility with the current work on forcing.
  \item More to the point, the recent work by Han and van Doorn on
    forcing in Lean deserves more discussion.  They have gone further
    than the current authors, having proved the independence of the
    continuum hypothesis.  They prefer Boolean-valued models as being
    more direct in use than the authors' countable transitive models.
    \begin{itemize}
    \item Readers will want to know whether the type-theoretic approach
      is better/worse/just different than using Isabelle/ZF, and
    \item are there any benefits to the ctm approach?
    \item Is the type-theory encoding of ZF really accurate?
    \item How about comparing proofs of equivalent statements in the two
      approaches for length and readability?
    \end{itemize}
  \end{itemize}
}

To the best of our knowledge, all of the previous works in
formalization of the method 
of forcing have been done in different variants of type theory, and
none of them uses the ctm approach. The
most important is the recent one by 
Han and van Doorn
\cite{han_et_al:LIPIcs:2019:11074,DBLP:conf/cpp/HanD20}, which includes
a formalization of the independence of $\CH$ by means
the Boolean-valued approach to forcing, using the Lean
proof assistant \cite{DBLP:conf/cade/MouraKADR15}.


\begin{itemize}
\item The consistency strength of Lean requires infinitely
  many inaccessibles. More precisely, let Lean$_n$ be the theory of
  CiC foundations of Lean restricted to $n$ type universes.  Carneiro
  \cite{carneiro-ms-thesis}, proved the consistency of Lean$_n$ from
  $\ZFC$ plus the existence of $n$ inaccessible
  cardinals. It is also reported in \cite{carneiro-ms-thesis} that
  Werner's results in \cite{10.5555/645869.668660} can be adapted to
  show that Lean$_{n+2}$ proves the consistency of the latter theory. 

  On the other hand, Isabelle's \emph{Pure} is based on
  ``intuitionistic higher order logic.'' In Paulson
  \cite{Paulson1989} it is proved that \emph{Pure} is sound for
  intuitionistic first order logic, thus it does not add any strength
  to it. On top of this, the axiomatization of Isabelle/ZF results in
  a system equiconsistent with $\ZFC$. Our running assumption, that of
  the existence of a countable transitive model, is considerably
  weaker (directly and consistency-wise) than the existence of a
  single inaccesible cardinal. In that sense, directly obtain
  unprovability results in first order logic, the meta theoretic
  machinery used to obtain them is far heavier than the one we use to
  operate model-theoretically.
  %
\item We may discuss in finer detail the shape of the axioms of
  Isabelle/ZF. It is perhaps more correct to say it is an
  notational variant of NBG set theory, because the schemes of
  Replacement and Separation feature higher order (free) variables
  playing the role of formula variables. It can't be proved that the
  axioms thus written correspond to first order sentences. This is the
  reason that our relativized versions only apply to set models, where
  we can restrict the formula variables to predicates that actually
  come from first order variables. In that sense, the axioms of the
  locale \isatt{M{\isacharunderscore}ZF} correspond faithfully to the
  $ZF$ axioms.
\item \fbox{\bf take care of repetitions} In our opinion, one of the
  main benefits of using transitive models is that many fundamental
  notions are absolute and thus the many statements can be interpreted
  transparently. It also provides a very concrete way to understand
  generic objects: as sets that (in the non trivial case) are provably
  not in the original model; this dispells any mystical feel around
  this concept (contrary to the case when the ground model is the
  universe of all sets). In addition, two-valued semantics is
  closer to our intuition ($\leftarrow$ revise).
\end{itemize}
%%% Local Variables: 
%%% mode: latex
%%% TeX-master: "forcing_in_isabelle_zf"
%%% ispell-local-dictionary: "american"
%%% End: 


%-%-%-%-%-%-%-%-%-%-%-%-%-%-%-%-%-%-%-%-%-%-%-%-%-%-%-%-%-%-%-%-%-%-%-%-%-%-%-%
\section{Isabelle/ZF}\label{sec:isabellezf}

Let us introduce briefly Paulson's formalization of ZF
\cite{paulson2017isabelle} in Isabelle and the main aspects of his
formal proof for the relative consistency of the Axiom of Choice
\cite{paulson_2003}; we will only focus on those aspects that are
essential to keep this paper self-contained, and refer the interested
reader to Paulson's articles.
%% Isabelle/ZF is a theory built upon the
%% core library FOL of classical first order logic.
Isabelle/ZF includes a development of classical first order logic,
FOL. Both of them are  built upon the core library \emph{Pure}. 

In Isabelle/ZF sets are \emph{individuals}, i.e.\ terms of type
\isatt{i} and formulas have type \isatt{o} (akin to a \emph{Bool}
type, but at the object level).  The axiomatization of $\ZFC$ in
Isabelle/ZF proceeds by postulating a binary predicate
\isatt{\ensuremath{\in}} and several set constructors (terms and
functions with values in \isatt{i}) corresponding to the empty set (the
constant \isatt{\isadigit{0}}), powersets, and one further constant
\isatt{inf} for an infinite set. The axioms, being formulas, are terms
of type \isatt{o}; the foundation axiom, for example, is formalized as
(the universal closure of) \isa{{\isachardoublequoteopen}A\
  {\isacharequal}\ {\isadigit{0}}\ {\isasymor}%
  \ {\isacharparenleft}{\isasymexists}x{\isasymin}A{\isachardot}\ %
  {\isasymforall}y{\isasymin}x{\isachardot}\ %
  y{\isasymnotin}A{\isacharparenright}%
  {\isachardoublequoteclose}}. %
Besides the axioms, Isabelle/ZF also introduces several definitions
(for example, pairs and sets defined by comprehension using
separation) and syntactic abbreviations to keep the formalization
close to the customary manner of doing mathematics.  Working with the
library and extending it is quite straightforward.  As an example, we
introduce a new term-former (which is a combination of instances of
replacement and separation) denoting the image of a function over a
set defined by comprehension, namely
$\{b(x): x\in A\text{ and }Q(x)\}$:
%% Since we find a need to combine the replacement and
%% separation schemes to speak of the image of a function over some set
%% defined by comprehension, we introduced a new term-former:
\begin{isabelle}
  \isacommand{definition}\isamarkupfalse%
  \ SepReplace\ {\isacharcolon}{\isacharcolon}\
  {\isachardoublequoteopen}{\isacharbrackleft}i{\isacharcomma}\
  i{\isasymRightarrow}i{\isacharcomma}\ i{\isasymRightarrow}\
  o{\isacharbrackright}\
  {\isasymRightarrow}i{\isachardoublequoteclose}\
  \isakeyword{where}\isanewline \ \
  {\isachardoublequoteopen}SepReplace{\isacharparenleft}A{\isacharcomma}b{\isacharcomma}Q{\isacharparenright}\
  {\isacharequal}{\isacharequal}\ {\isacharbraceleft}y\ {\isachardot}\
  x{\isasymin}A{\isacharcomma}\
  y{\isacharequal}b{\isacharparenleft}x{\isacharparenright}\
  {\isasymand}\
  Q{\isacharparenleft}x{\isacharparenright}{\isacharbraceright}{\isachardoublequoteclose}
\end{isabelle}
\noindent %% It is more convenient to add an abbreviation for a simpler
%% writing and a nicer reading, so we use
We are then able to add the abbreviation \isa{{\isacharbraceleft}b\
  {\isachardot}{\isachardot}\ x{\isasymin}A{\isacharcomma}\
  Q{\isacharbraceright}} as a notation for
\isa{SepReplace{\isacharparenleft}A{\isacharcomma}b{\isacharcomma}Q{\isacharparenright}}. The
characterization of our new constructor is given by
\begin{isabelle}
\isacommand{lemma}\isamarkupfalse%
\ Sep{\isacharunderscore}and{\isacharunderscore}Replace{\isacharcolon}\ {\isachardoublequoteopen}{\isacharbraceleft}b{\isacharparenleft}x{\isacharparenright}\ {\isachardot}{\isachardot}\ x{\isasymin}A{\isacharcomma}\ Q{\isacharparenleft}x{\isacharparenright}\ {\isacharbraceright}\ {\isacharequal}\ {\isacharbraceleft}b{\isacharparenleft}x{\isacharparenright}\ {\isachardot}\ x{\isasymin}{\isacharbraceleft}y{\isasymin}A{\isachardot}\ Q{\isacharparenleft}y{\isacharparenright}{\isacharbraceright}{\isacharbraceright}{\isachardoublequoteclose}
\end{isabelle}

%% We are not discussing how, if possible at all, to express that some
%% set satisfies some axiom. For example, we might well be interested in
%% proving $\emptyset \models \mathit{foundation}$, but there is no
%% sensible way to state this, without encoding formulas and axioms in
%% ZFC.\footnote{Indeed Paulson has encoded formulas using a very general
%%   method to encode recursive definitions. In our development, although
%%   not reported in this paper, we use this internalization.}
%% 
%% \fbox{ You need this in order to formalize $\models$ as a predicate of
%%   type $i\ent i \ent o$. (Don't know how to put that here.)
%% }

%% In
%% Isabelle/ZF there are, however, relativized versions of the
%% axioms. This alternative statement of the axioms are relativized with
%% respect to classes, which in Isabelle/ZF corresponds to predicates over
%% sets
We now discuss relativization in Isabelle/ZF. Relativized versions of the
 axioms can be found in the formalization of constructibility \cite{paulson_2003}. For
 example, the relativized Axiom of Foundation is
\begin{isabelle}
\isacommand{definition}\isamarkupfalse%
\ foundation{\isacharunderscore}ax\ {\isacharcolon}{\isacharcolon}\ {\isachardoublequoteopen}{\isacharparenleft}i{\isacharequal}{\isachargreater}o{\isacharparenright}\ {\isacharequal}{\isachargreater}\ o{\isachardoublequoteclose}\ \isakeyword{where}\isanewline
\ \ \ \ {\isachardoublequoteopen}foundation{\isacharunderscore}ax{\isacharparenleft}M{\isacharparenright}\ {\isacharequal}{\isacharequal}\isanewline
\ \ \ \ \ \ \ \ {\isasymforall}x{\isacharbrackleft}M{\isacharbrackright}{\isachardot}\ {\isacharparenleft}{\isasymexists}y{\isacharbrackleft}M{\isacharbrackright}{\isachardot}\ y{\isasymin}x{\isacharparenright}\ {\isasymlongrightarrow}\ {\isacharparenleft}{\isasymexists}y{\isacharbrackleft}M{\isacharbrackright}{\isachardot}\ y{\isasymin}x\ {\isacharampersand}\ {\isachartilde}{\isacharparenleft}{\isasymexists}z{\isacharbrackleft}M{\isacharbrackright}{\isachardot}\ z{\isasymin}x\ {\isacharampersand}\ z\ {\isasymin}\ y{\isacharparenright}{\isacharparenright}{\isachardoublequoteclose}
\end{isabelle}

% ∀x[c]
\noindent The relativized quantifier
\isa{{\isasymforall}x{\isacharbrackleft}M{\isacharbrackright}{\isachardot}\
  P(x)} is a shorthand for \isa{{\isasymforall}x{\isachardot}\ M(x)
  {\isasymlongrightarrow} P(x)}. In order to express that a (set) model
satisfies this axiom we use  the ``coercion''
\isa{{\isacharhash}{\isacharhash} :: i ={\isachargreater} (i
  ={\isachargreater} o)} (that maps a set $A$ to the predicate
$\lambda x . (x\in A)$) provided by Isabelle/ZF. As a trivial example we
can show that the empty set satisfies Foundation:
\begin{isabelle}
\isacommand{lemma}\isamarkupfalse%
\ emp{\isacharunderscore}foundation\ {\isacharcolon}\ {\isachardoublequoteopen}foundation{\isacharunderscore}ax{\isacharparenleft}{\isacharhash}{\isacharhash}{\isadigit{0}}{\isacharparenright}{\isachardoublequoteclose}
\end{isabelle}


% Let us ponder if one can state, let alone prove, that some set \isa{M
%   :: i} satisfies a statement \isa{\ensuremath{\phi} :: o}. Since
% \isa{o} does not have an inductive definition, we cannot define the
% satisfaction of \isa{\ensuremath{\phi}}. 

% As we have outlined above, relativization is a key concept when
% proving that axioms satisfied by some model $M$ are also satisfied in
% another model constructed from $M$. In this paper, we show the complete
% proof that if a set \isa{M :: i} satisfies the 

Mathematical texts usually start by fixing a context that defines
parameters and assumptions needed to develop theorems
and results. In Isabelle the way of defining contexts is through
\emph{locales}~\cite{ballarin2010tutorial}.
Locales can be combined and extended adding more parameters and assuming
more facts, leading to a new locale. For example a context describing
lattices can be extended to distributive lattices.
The way to instantiate a locale is by \emph{interpreting} it, which consists
of giving concrete values to parameters and proving the assumptions.
In our work, we use locales to organize the formalization and to make
explicit the assumptions of the most important results.

We make a final comment about the implementation tools provided by the
Isabelle framework.

An IDE called \texttt{jEdit} is bundled with the standard Isabelle
distribution and it offers the user a fair amount of tools in order to
manage theory files, searching for theorems and concepts spread
through the source files, and includes tracing utilities for the
automatic tools. A main feature is a window showing the \emph{proof
  state}, where the active (sub)goals are shown, along with the already
obtained results and possibly errors. 

Isabelle proofs can be written in two dialects. 
The most basic one and older (in which most of the original
work by Paulson is written) follows a procedural approach, where one
applies several tactics in order to decompose the goal into simpler
ones and then solving them (with the aid of automation). This approach
has the drawback that a proof akin to mathematical writing can't be
read out from the code, since the intermediate steps are only shown in
the proof state. For this reason, the proof language
\emph{Isar} was developed, starting with Wenzel's
work~\cite{DBLP:conf/tphol/Wenzel99}. Isar is mostly declarative, and
its main purpose is to construct \emph{proof documents} that (in
principle) can be read and understood without the need of running the
code. 

We started this development by using the procedural approach, but soon
we realized that for our purposes the Isar language was far more
appropriate.

%%% Local Variables:
%%% mode: latex
%%% ispell-local-dictionary: "american"
%%% TeX-master: "first_steps_into_forcing"
%%% End:


\section{Forcing notions}\label{sec:forcing-notions}
%\textit{Para no confundir, directamente usamos este nombre en lugar de
% ``forcing posets'' para los preórdenes con un máximo distinguido}

In this section we present a proof of the Rasiowa-Sikorski lemma which
uses the principle of dependent choices. We start by introducing
the necessary definitions about preorders; then, we explain and prove
the principle of dependent choice most suitable for our purpose.

It is to be noted that the order of presentation of the material
deviates a bit from the dependency of the source  files. The
file containing the most basic results and definitions that follow
imports that containing the results of
Subsection~\ref{sec:sequence-version-dc}.


\begin{definition}
  A preorder on a set $P$ is a binary relation ${\leqslant}$ which is
  reflexive and transitive.
\end{definition}

The preorder relation will be represented as a set of pairs, and hence
it is a term of type
\isatt{i}.
%
%
%
%
%
%
%
%
%
\begin{definition}
  Given a preorder $(P,\leqslant)$ we say that two elements $p,q$ are
  \emph{compatible} if they have a lower bound in $P$. Notice that
  the elements of $P$ are also sets, therefore they have type
  \isatt{i}.
  \begin{isabelle}%
  \isacommand{definition}\isamarkupfalse%
\ compat{\isacharunderscore}in\ {\isacharcolon}{\isacharcolon}\ {\isachardoublequoteopen}i{\isasymRightarrow}i{\isasymRightarrow}i{\isasymRightarrow}i{\isasymRightarrow}o{\isachardoublequoteclose}\ \isakeyword{where}\isanewline
\ \ {\isachardoublequoteopen}compat{\isacharunderscore}in{\isacharparenleft}P{\isacharcomma}leq{\isacharcomma}p{\isacharcomma}q{\isacharparenright}\ {\isacharequal}{\isacharequal}\ {\isasymexists}d{\isasymin}P\ {\isachardot}\ {\isasymlangle}d{\isacharcomma}p{\isasymrangle}{\isasymin}leq\ {\isasymand}\ {\isasymlangle}d{\isacharcomma}q{\isasymrangle}{\isasymin}leq{\isachardoublequoteclose}
\end{isabelle}
\end{definition}

%
%
%
%
%
%
%
%
%
%
%
%
%
%
%
%
%
%

\begin{definition}
  A \emph{forcing notion} is a preorder $(P,\leqslant)$ with a maximal element $\mathbbm{1} \in P$.
  \begin{isabelle}
\isacommand{locale}\isamarkupfalse%
\ forcing{\isacharunderscore}notion\ {\isacharequal}\isanewline
\ \ \isakeyword{fixes}\ P\ leq\ one\isanewline
\ \ \isakeyword{assumes}\ one{\isacharunderscore}in{\isacharunderscore}P{\isacharcolon}  \ {\isachardoublequoteopen}one\ {\isasymin}\ P{\isachardoublequoteclose}\isanewline
 \ \ \isakeyword{and}\ leq{\isacharunderscore}preord{\isacharcolon} \ \ \ {\isachardoublequoteopen}preorder{\isacharunderscore}on{\isacharparenleft}P{\isacharcomma}leq{\isacharparenright}{\isachardoublequoteclose}\isanewline
 \ \ \isakeyword{and}\ one{\isacharunderscore}max{\isacharcolon}  \ \ {\isachardoublequoteopen}{\isasymforall}p{\isasymin}P{\isachardot}\ {\isasymlangle}p{\isacharcomma}one{\isasymrangle}{\isasymin}leq{\isachardoublequoteclose}
\end{isabelle}
\end{definition}
%
\noindent The locale \isatt{forcing{\isacharunderscore}notion}  introduces a mathematical
context where we work assuming the forcing notion
$(P,\leqslant, \mathbbm{1})$. 
%
%
%
%
%
In the following definitions we are in
the locale \isatt{forcing{\isacharunderscore}notion}.

A set $D$ is \emph{dense} if every element $p\in P$ has a lower bound
in $D$ and there is also a weaker definition which asks for a lower
bound in $D$ only for the elements below some fixed element $q$. 
\begin{isabelle}
  \isacommand{definition}\isamarkupfalse%
\ dense\ {\isacharcolon}{\isacharcolon}\ {\isachardoublequoteopen}i{\isasymRightarrow}o{\isachardoublequoteclose}\ \isakeyword{where}\isanewline
\ \ {\isachardoublequoteopen}dense{\isacharparenleft}D{\isacharparenright}\ {\isacharequal}{\isacharequal}\ {\isasymforall}p{\isasymin}P{\isachardot}\ {\isasymexists}d{\isasymin}D\ {\isachardot}\ {\isasymlangle}d{\isacharcomma}p{\isasymrangle}{\isasymin}leq{\isachardoublequoteclose}\isanewline
\isanewline
\isacommand{definition}\isamarkupfalse%
\ dense{\isacharunderscore}below\ {\isacharcolon}{\isacharcolon}\ {\isachardoublequoteopen}i{\isasymRightarrow}i{\isasymRightarrow}o{\isachardoublequoteclose}\ \isakeyword{where}\isanewline
\ \ {\isachardoublequoteopen}dense{\isacharunderscore}below{\isacharparenleft}D{\isacharcomma}q{\isacharparenright}\ {\isacharequal}{\isacharequal}\ {\isasymforall}p{\isasymin}P{\isachardot}\ {\isasymlangle}p{\isacharcomma}q{\isasymrangle}{\isasymin}leq\ {\isasymlongrightarrow}\ {\isacharparenleft}{\isasymexists}d{\isasymin}D\ {\isachardot}\ {\isasymlangle}d{\isacharcomma}p{\isasymrangle}{\isasymin}leq{\isacharparenright}{\isachardoublequoteclose}
\end{isabelle}
Since
the relation $\leqslant$ is reflexive, it is obvious that $P$ is
dense. Actually, this follows automatically once the appropriate definitions are
unfolded:
\begin{isabelle}
\isacommand{lemma}\isamarkupfalse%
\ P{\isacharunderscore}dense{\isacharcolon}\ {\isachardoublequoteopen}dense{\isacharparenleft}P{\isacharparenright}{\isachardoublequoteclose}\isanewline
%
%\isadelimproof
\ \ %
%\endisadelimproof
%
%\isatagproof
\isacommand{using}\isamarkupfalse%
\ leq{\isacharunderscore}preord\isanewline
\ \ \isacommand{unfolding}\isamarkupfalse%
\ preorder{\isacharunderscore}on{\isacharunderscore}def\ refl{\isacharunderscore}def\ dense{\isacharunderscore}def\isanewline
\ \ \isacommand{by}\isamarkupfalse%
\ blast%
%\endisatagproof
\end{isabelle}
Here, the automatic tactic \isa{blast} solves the goal. In the
procedural approach, goals are refined with the command
\textbf{apply}~\emph{tactic}, and proofs are finished using \textbf{done}. 
Then \textbf{by $\dots$} is an idiom for 
\textbf{apply $\dots$ done}.
 
We say that $F\subseteq P$ is increasing (or upward closed) if every
extension of any element in $F$ is also in $F$.
\begin{isabelle}
\isacommand{definition}\isamarkupfalse%
\ increasing\ {\isacharcolon}{\isacharcolon}\ {\isachardoublequoteopen}i{\isasymRightarrow}o{\isachardoublequoteclose}\ \isakeyword{where}\isanewline
\ \ {\isachardoublequoteopen}increasing{\isacharparenleft}F{\isacharparenright}\ {\isacharequal}{\isacharequal}\ {\isasymforall}x{\isasymin}F{\isachardot}\ {\isasymforall}\ p{\isasymin}P\ {\isachardot}\ {\isasymlangle}x{\isacharcomma}p{\isasymrangle}{\isasymin}leq\ {\isasymlongrightarrow}\ p{\isasymin}F{\isachardoublequoteclose}
\end{isabelle}
A filter is an increasing set $G$ with all its elements being compatible in $G$.
\begin{isabelle}
\isacommand{definition}\isamarkupfalse%
\ filter\ {\isacharcolon}{\isacharcolon}\ {\isachardoublequoteopen}i{\isasymRightarrow}o{\isachardoublequoteclose}\ \isakeyword{where}\isanewline
\ \ {\isachardoublequoteopen}filter{\isacharparenleft}G{\isacharparenright}\ {\isacharequal}{\isacharequal}\ G{\isasymsubseteq}P\ {\isasymand}\ increasing{\isacharparenleft}G{\isacharparenright}\ {\isasymand}\isanewline \ \ {\isacharparenleft}{\isasymforall}p{\isasymin}G{\isachardot}\ {\isasymforall}q{\isasymin}G{\isachardot}\ compat{\isacharunderscore}in{\isacharparenleft}G{\isacharcomma}leq{\isacharcomma}p{\isacharcomma}q{\isacharparenright}{\isacharparenright}{\isachardoublequoteclose}
\end{isabelle}

We finally introduce the upward closure of a set
and prove that the closure of $A$ is a filter if its elements are
compatible in $A$.
\begin{isabelle}
\isacommand{definition}\isamarkupfalse%
\ upclosure\ {\isacharcolon}{\isacharcolon}\ {\isachardoublequoteopen}i{\isasymRightarrow}i{\isachardoublequoteclose}\ \isakeyword{where}\isanewline
\ \ {\isachardoublequoteopen}upclosure{\isacharparenleft}A{\isacharparenright}\ {\isacharequal}{\isacharequal}\ {\isacharbraceleft}p{\isasymin}P{\isachardot}{\isasymexists}a{\isasymin}A{\isachardot}{\isasymlangle}a{\isacharcomma}p{\isasymrangle}{\isasymin}leq{\isacharbraceright}{\isachardoublequoteclose}\isanewline
\isacommand{lemma}\isamarkupfalse%
\ \ closure{\isacharunderscore}compat{\isacharunderscore}filter{\isacharcolon}
\ \ {\isachardoublequoteopen}A{\isasymsubseteq}P\ {\isasymLongrightarrow}\isanewline\ \ {\isacharparenleft}{\isasymforall}p{\isasymin}A{\isachardot}{\isasymforall}q{\isasymin}A{\isachardot}\ compat{\isacharunderscore}in{\isacharparenleft}A{\isacharcomma}leq{\isacharcomma}p{\isacharcomma}q{\isacharparenright}{\isacharparenright}\ {\isasymLongrightarrow}\ filter{\isacharparenleft}upclosure{\isacharparenleft}A{\isacharparenright}{\isacharparenright}{\isachardoublequoteclose}
\end{isabelle}
As usual
with procedural proofs, the refinement process goes ``backwards,''
from the main goal to simpler ones. The proof of this last lemma takes
21 lines and 34 proof commands and is one of the longest procedural
proofs in the development.  It was  at
the moment of its implementation that we realized that a declarative
approach was best because, apart from being more readable, the
reasoning flows mostly in a forward fashion.

%%% Local Variables:
%%% mode: latex
%%% ispell-local-dictionary: "american"
%%% TeX-master: "first_steps_into_forcing"
%%% End:




\subsection{A sequence version of $\DC$}\label{sec:sequence-version-dc}
In order to prove the Rasiowa-Sikorski lemma in
general, a version of the Axiom of Choice ($\AC$) must be proved. This
version is known as the \emph{Principle of Dependent choices}:
\begin{quote}
  ($\DC$) Let $\rr$ be a binary relation on $A$, and $a\in A$. If
  $\forall x\in A\,  \exists y\in A\, x\rr y$, then there exists
  $f:\om\to A$ such that $f(0)=a$ and $f(n)\rr f(n+1)$ for all
  $n\in\om$.
\end{quote}

A different version of $\DC$ (without the constraint $f(0)=a$, but
allowing a parameter that allows to prove an equivalence with $\AC$) was
formalized by Krzysztof Grabczewski as part of the Isabelle/ZF
library, along with many 
statements that follow from or are equivalent to $\AC$. Since the
proofs there required a greater generality, it would have been very
contrived to derived this ``pointed'' version of $\DC$ from that
development. Instead, we preferred to give a direct proof based in the
one that appears in Moschovakis \cite{moschovakis1994notes}. Another
advantage is that this proof depends on the following version of $\AC$
(existence of choice functions on $\P(A)\sm \{\mty\}$):
\[
\exists s: \P(A)\sm \{\mty\}\to A \, \forall X\sbq A.\ X\neq \mty \limp
s(X) \in X.
\]
This statement is more standard than the one deemed ``$\AC$'' in the
Isabelle/ZF development, that involves a silent use of Replacement
since it is written as an axiom scheme. 

The strategy is to define $f$ above by primitive recursion. 

\begin{isabelle}\isamarkuptrue%
\isacommand{consts}\isamarkupfalse%
\ dc{\isacharunderscore}witness\ {\isacharcolon}{\isacharcolon}\ {\isachardoublequoteopen}i\ {\isasymRightarrow}\ i\ {\isasymRightarrow}\ i\ {\isasymRightarrow}\ i\ {\isasymRightarrow}\ i\ {\isasymRightarrow}\ i{\isachardoublequoteclose}\isanewline
\isacommand{primrec}\isamarkupfalse%
\isanewline
\ \ wit{\isadigit{0}}\ \ \ {\isacharcolon}\ {\isachardoublequoteopen}dc{\isacharunderscore}witness{\isacharparenleft}{\isadigit{0}}{\isacharcomma}A{\isacharcomma}a{\isacharcomma}s{\isacharcomma}R{\isacharparenright}\ {\isacharequal}\ a{\isachardoublequoteclose}\isanewline
\ \ witrec\ {\isacharcolon}{\isachardoublequoteopen}dc{\isacharunderscore}witness{\isacharparenleft}succ{\isacharparenleft}n{\isacharparenright}{\isacharcomma}A{\isacharcomma}a{\isacharcomma}s{\isacharcomma}R{\isacharparenright}\ {\isacharequal}\ s{\isacharbackquote}{\isacharbraceleft}x{\isasymin}A{\isachardot}\ {\isasymlangle}dc{\isacharunderscore}witness{\isacharparenleft}n{\isacharcomma}A{\isacharcomma}a{\isacharcomma}s{\isacharcomma}R{\isacharparenright}{\isacharcomma}x{\isasymrangle}{\isasymin}R\ {\isacharbraceright}{\isachardoublequoteclose}
\end{isabelle}

The function \isatt{dc{\isacharunderscore}witness} has, apart from $A$, $a$ and $R$, a
function $s$ as a parameter. If this function is a selector for
$\P(A)\sm \{\mty\}$, the function $f\defi {}$\isatt{dc{\isacharunderscore}witness}$(A,a,s,R)$
will satify $\DC$. Notice that $s$ is a term of type \isatt{i} (a
function construed as a set of pairs) and an expression
\isatt{s{\isacharbackquote}b} is notation for  \isatt{apply(s,b)},
where
\isatt{apply\ {\isacharcolon}{\isacharcolon}\ {\isachardoublequoteopen}i\ {\isasymRightarrow}\ i\ {\isasymRightarrow}\ i{\isachardoublequoteclose}}
is the operation of function application. We will not make a
distinction in our prose and simply write \isatt{s}(\isatt{b}) in this case.

The proof is mostly routine; after a few lemmas we obtain the
following theorem:

\begin{isabelle}
\isacommand{theorem}\isamarkupfalse%
\ pointed{\isacharunderscore}DC\ \ {\isacharcolon}\ {\isachardoublequoteopen}{\isacharparenleft}{\isasymforall}x{\isasymin}A{\isachardot}\ {\isasymexists}y{\isasymin}A{\isachardot}\ {\isasymlangle}x{\isacharcomma}y{\isasymrangle}{\isasymin}\ R{\isacharparenright}\ {\isasymLongrightarrow}\isanewline
\ \ \ \ \ \ \ \ \ \ \ \ \ \ \ \ \ \ \ \ \ \ \ {\isasymforall}a{\isasymin}A{\isachardot}\ {\isacharparenleft}{\isasymexists}f\ {\isasymin}\ nat{\isasymrightarrow}A{\isachardot}\ f{\isacharbackquote}{\isadigit{0}}\ {\isacharequal}\ a\ {\isasymand}\ {\isacharparenleft}{\isasymforall}n\ {\isasymin}\ nat{\isachardot}\ {\isasymlangle}f{\isacharbackquote}n{\isacharcomma}f{\isacharbackquote}succ{\isacharparenleft}n{\isacharparenright}{\isasymrangle}{\isasymin}R{\isacharparenright}{\isacharparenright}{\isachardoublequoteclose}
\end{isabelle}

We need a further, ``diagonal'' version of $\DC$  to prove
Rasiowa-Sikorski. That is, if the assumption holds for a sequence of
relations $S_n$,  then $f(n) \mathrel{S_{n+1}} f(n+1)$.

We first obtain a corollary of $\DC$ changing $A$ for
$A\times\mathtt{nat}$:

\begin{isabelle}
\isacommand{lemma}\isamarkupfalse%
\ aux{\isacharunderscore}DC{\isacharunderscore}on{\isacharunderscore}AxNat{\isadigit{2}}\ {\isacharcolon}\ {\isachardoublequoteopen}{\isasymforall}x{\isasymin}A{\isasymtimes}nat{\isachardot}\ {\isasymexists}y{\isasymin}A{\isachardot}\ {\isasymlangle}x{\isacharcomma}{\isasymlangle}y{\isacharcomma}succ{\isacharparenleft}snd{\isacharparenleft}x{\isacharparenright}{\isacharparenright}{\isasymrangle}{\isasymrangle}\ {\isasymin}\ R\ {\isasymLongrightarrow}\isanewline
\ \ \ \ \ \ \ \ \ \ \ \ \ \ \ \ \ \ {\isasymforall}x{\isasymin}A{\isasymtimes}nat{\isachardot}\ {\isasymexists}y{\isasymin}A{\isasymtimes}nat{\isachardot}\ {\isasymlangle}x{\isacharcomma}y{\isasymrangle}\ {\isasymin}\ {\isacharbraceleft}{\isasymlangle}a{\isacharcomma}b{\isasymrangle}{\isasymin}R{\isachardot}\ snd{\isacharparenleft}b{\isacharparenright}\ {\isacharequal}\ succ{\isacharparenleft}snd{\isacharparenleft}a{\isacharparenright}{\isacharparenright}{\isacharbraceright}{\isachardoublequoteclose}
\end{isabelle}
%
The following lemma is then proved automatically:

\begin{isabelle}
\isacommand{lemma}\isamarkupfalse%
\ aux{\isacharunderscore}sequence{\isacharunderscore}DC{\isadigit{2}}\ {\isacharcolon}\ {\isachardoublequoteopen}{\isasymforall}x{\isasymin}A{\isachardot}\ {\isasymforall}n{\isasymin}nat{\isachardot}\ {\isasymexists}y{\isasymin}A{\isachardot}\ {\isasymlangle}x{\isacharcomma}y{\isasymrangle}\ {\isasymin}\ S{\isacharbackquote}n\ {\isasymLongrightarrow}\isanewline
\ \ \ \ \ \ \ \ {\isasymforall}x{\isasymin}A{\isasymtimes}nat{\isachardot}\ {\isasymexists}y{\isasymin}A{\isachardot}\ {\isasymlangle}x{\isacharcomma}{\isasymlangle}y{\isacharcomma}succ{\isacharparenleft}snd{\isacharparenleft}x{\isacharparenright}{\isacharparenright}{\isasymrangle}{\isasymrangle}\ {\isasymin}\ {\isacharbraceleft}{\isasymlangle}{\isasymlangle}x{\isacharcomma}n{\isasymrangle}{\isacharcomma}{\isasymlangle}y{\isacharcomma}m{\isasymrangle}{\isasymrangle}{\isasymin}{\isacharparenleft}A{\isasymtimes}nat{\isacharparenright}{\isasymtimes}{\isacharparenleft}A{\isasymtimes}nat{\isacharparenright}{\isachardot}\ {\isasymlangle}x{\isacharcomma}y{\isasymrangle}{\isasymin}S{\isacharbackquote}m\ {\isacharbraceright}{\isachardoublequoteclose}\isanewline
\ \ %
\isacommand{by}\isamarkupfalse%
\ auto%
\end{isabelle}
%
And finally, we arrive to $\DC$ for a sequence of relations.

\begin{isabelle}
\isacommand{lemma}\isamarkupfalse%
\ sequence{\isacharunderscore}DC{\isacharcolon}\ {\isachardoublequoteopen}{\isasymforall}x{\isasymin}A{\isachardot}\ {\isasymforall}n{\isasymin}nat{\isachardot}\ {\isasymexists}y{\isasymin}A{\isachardot}\ {\isasymlangle}x{\isacharcomma}y{\isasymrangle}\ {\isasymin}\ S{\isacharbackquote}n\ {\isasymLongrightarrow}\isanewline
\ \ \ \ {\isasymforall}a{\isasymin}A{\isachardot}\ {\isacharparenleft}{\isasymexists}f\ {\isasymin}\ nat{\isasymrightarrow}A{\isachardot}\ f{\isacharbackquote}{\isadigit{0}}\ {\isacharequal}\ a\ {\isasymand}\ {\isacharparenleft}{\isasymforall}n\ {\isasymin}\ nat{\isachardot}\ {\isasymlangle}f{\isacharbackquote}n{\isacharcomma}f{\isacharbackquote}succ{\isacharparenleft}n{\isacharparenright}{\isasymrangle}{\isasymin}S{\isacharbackquote}succ{\isacharparenleft}n{\isacharparenright}{\isacharparenright}{\isacharparenright}{\isachardoublequoteclose}
\end{isabelle}

\subsection{The Rasiowa-Sikorski lemma}\label{sec:rasiowa-sikorski-lemma}
In order to state this Lemma, we gather the relevant hypotheses into a locale:

\begin{isabelle}%
\isacommand{locale}\isamarkupfalse%
\ countable{\isacharunderscore}generic\ {\isacharequal}\ forcing{\isacharunderscore}notion\ {\isacharplus}\isanewline
\ \ \isakeyword{fixes}\ {\isasymD}\isanewline
\ \ \isakeyword{assumes}\ countable{\isacharunderscore}subs{\isacharunderscore}of{\isacharunderscore}P{\isacharcolon}\ \ {\isachardoublequoteopen}{\isasymD}\ {\isasymin}\ nat{\isasymrightarrow}Pow{\isacharparenleft}P{\isacharparenright}{\isachardoublequoteclose}\isanewline
\ \ \isakeyword{and}\ \ \ \ \ seq{\isacharunderscore}of{\isacharunderscore}denses{\isacharcolon}\ \ \ \ \ \ \ \ {\isachardoublequoteopen}{\isasymforall}n\ {\isasymin}\ nat{\isachardot}\ dense{\isacharparenleft}{\isasymD}{\isacharbackquote}n{\isacharparenright}{\isachardoublequoteclose}
\end{isabelle}
%
That is, $\calD$ is a sequence of dense subsets of the poset $P$. A
filter is \emph{$\calD$-generic} if it intersects every dense set in
the sequence.

\begin{isabelle}%
\isacommand{definition}\isamarkupfalse%
\ {\isacharparenleft}\isakeyword{in}\ countable{\isacharunderscore}generic{\isacharparenright}\isanewline
\ \ D{\isacharunderscore}generic\ {\isacharcolon}{\isacharcolon}\ {\isachardoublequoteopen}i{\isasymRightarrow}o{\isachardoublequoteclose}\ \isakeyword{where}\isanewline
\ \ {\isachardoublequoteopen}D{\isacharunderscore}generic{\isacharparenleft}G{\isacharparenright}\ {\isacharequal}{\isacharequal}\ filter{\isacharparenleft}G{\isacharparenright}\ {\isasymand}\ {\isacharparenleft}{\isasymforall}n{\isasymin}nat{\isachardot}{\isacharparenleft}{\isasymD}{\isacharbackquote}n{\isacharparenright}{\isasyminter}G{\isasymnoteq}{\isadigit{0}}{\isacharparenright}{\isachardoublequoteclose}
\end{isabelle}

We can now state the Rasiowa-Sikorski Lemma.
\begin{isabelle}%
\isacommand{theorem}\isamarkupfalse%
\ {\isacharparenleft}\isakeyword{in}\ countable{\isacharunderscore}generic{\isacharparenright}\ rasiowa{\isacharunderscore}sikorski{\isacharcolon}\isanewline
\ \ {\isachardoublequoteopen}p{\isasymin}P\ {\isasymLongrightarrow}\ {\isasymexists}G{\isachardot}\ p{\isasymin}G\ {\isasymand}\ D{\isacharunderscore}generic{\isacharparenleft}G{\isacharparenright}{\isachardoublequoteclose}
\end{isabelle}

The intuitive argument for the result is simple: Once $p_0=p\in P$ is
fixed, we can recursively choose $p_{n+1}$ such that 
$p_n \geq p_{n+1}\in \calD_n$, since $\calD_n$ is dense in $P$. Then
the filter generated by $\{p_n : n\in \om\}$ intersects each set in
the sequence $\{\calD_n\}_n$. This argument appeals to the sequence
version of $\DC$; we have to prove first that the relevant relation
satisfies its hypothesis:

\begin{isabelle}%
\isacommand{lemma}\isamarkupfalse%
\ {\isacharparenleft}\isakeyword{in}\ countable{\isacharunderscore}generic{\isacharparenright}\ RS{\isacharunderscore}relation{\isacharcolon}\isanewline
\ \ \isakeyword{assumes}\isanewline
\ \ \ \ \ \ \ \ {\isadigit{1}}{\isacharcolon}\ \ {\isachardoublequoteopen}x{\isasymin}P{\isachardoublequoteclose}\isanewline
\ \ \ \ \ \ \ \ \ \ \ \ \isakeyword{and}\isanewline
\ \ \ \ \ \ \ \ {\isadigit{2}}{\isacharcolon}\ \ {\isachardoublequoteopen}n{\isasymin}nat{\isachardoublequoteclose}\isanewline
\ \ \isakeyword{shows}\isanewline
\ \ \ \ \ \ \ \ \ \ \ \ {\isachardoublequoteopen}{\isasymexists}y{\isasymin}P{\isachardot}\ {\isasymlangle}x{\isacharcomma}y{\isasymrangle}\ {\isasymin}\ {\isacharparenleft}{\isasymlambda}m{\isasymin}nat{\isachardot}\ {\isacharbraceleft}{\isasymlangle}x{\isacharcomma}y{\isasymrangle}{\isasymin}P{\isacharasterisk}P{\isachardot}\ {\isasymlangle}y{\isacharcomma}x{\isasymrangle}{\isasymin}leq\ {\isasymand}\ y{\isasymin}{\isasymD}{\isacharbackquote}{\isacharparenleft}pred{\isacharparenleft}m{\isacharparenright}{\isacharparenright}{\isacharbraceright}{\isacharparenright}{\isacharbackquote}n{\isachardoublequoteclose}
\end{isabelle}
%
Both of the proofs written in the Isar language.



%%% Local Variables:
%%% mode: latex
%%% ispell-local-dictionary: "american"
%%% TeX-master: "first_steps_into_forcing"
%%% End:



%%%%%%%%%%%%%%%%%%%%%%%%%%%%%%%%%%%%%%%%%%%%%%%%%%%%%%%%%%%%%%%%%%%%%%%%%%%%%%%%
\section{The generic extension}

Cohen's technique of forcing consists of constructing new models of
$\ZFC$ by adding a \emph{generic} subset $G$ of the forcing notion $P$
(a preorder with top). Given a model $M$ of $\ZFC$, the extension with
the generic subset $G$ is called \emph{the generic extension} of $M$,
denoted $M[G]$.  In this section we introduce all the necessary
concepts and results for defining $M[G]$; namely, we show, using
Rasiowa-Sikorski, that every preorder in a ctm admits a generic filter
and also develop the machinery of names. As an application of the
latter, we prove some basic
results about the generic extension.

\subsection{The generic filter}
\label{sec:generic-filter}
The following locale gathers the data needed to ensure the 
existence of an $M$-generic filter for a poset \isa{P}. 

\begin{isabelle}
\isacommand{locale}\isamarkupfalse%
\ forcing{\isacharunderscore}data\ {\isacharequal}\ forcing{\isacharunderscore}notion\ {\isacharplus}\isanewline
\ \ \isakeyword{fixes}\ M\ enum\isanewline
\ \ \isakeyword{assumes}\ M{\isacharunderscore}countable{\isacharcolon}\ \ \ \ \ \ {\isachardoublequoteopen}enum{\isasymin}bij{\isacharparenleft}nat{\isacharcomma}M{\isacharparenright}{\isachardoublequoteclose}\isanewline
\ \ \ \ \ \ \isakeyword{and}\ P{\isacharunderscore}in{\isacharunderscore}M{\isacharcolon}\ \ \ \ \ \ \ \ \ \ \ {\isachardoublequoteopen}P\ {\isasymin}\ M{\isachardoublequoteclose}\isanewline
\ \ \ \ \ \ \isakeyword{and}\ leq{\isacharunderscore}in{\isacharunderscore}M{\isacharcolon}\ \ \ \ \ \ \ \ \ {\isachardoublequoteopen}leq\ {\isasymin}\ M{\isachardoublequoteclose}\isanewline
\ \ \ \ \ \ \isakeyword{and}\ trans{\isacharunderscore}M{\isacharcolon}\ \ \ \ \ \ \ \ \ \ {\isachardoublequoteopen}Transset{\isacharparenleft}M{\isacharparenright}{\isachardoublequoteclose}
\end{isabelle}

An immediate consequence of the Rasiowa-Sikorski Lemma is the
existence of an $M$-generic filter for a poset \isa{P}.

\begin{isabelle}

\isacommand{lemma}\isamarkupfalse%
\ generic{\isacharunderscore}filter{\isacharunderscore}existence{\isacharcolon}\ \isanewline
\ \ {\isachardoublequoteopen}p{\isasymin}P\ {\isasymLongrightarrow}\ {\isasymexists}G{\isachardot}\ p{\isasymin}G\ {\isasymand}\ M{\isacharunderscore}generic{\isacharparenleft}G{\isacharparenright}{\isachardoublequoteclose}
\end{isabelle}

\noindent By defining an appropriate countable sequence of dense subsets of \isa{P},
\begin{isabelle}

\ \ \isacommand{let}\isamarkupfalse%
\isanewline
\ \ \ \ \ \ \ \ \ \ \ \ \ \ {\isacharquery}D{\isacharequal}{\isachardoublequoteopen}{\isasymlambda}n{\isasymin}nat{\isachardot}\ {\isacharparenleft}if\ {\isacharparenleft}enum{\isacharbackquote}n{\isasymsubseteq}P\ {\isasymand}\ dense{\isacharparenleft}enum{\isacharbackquote}n{\isacharparenright}{\isacharparenright}\ \ then\ enum{\isacharbackquote}n\ else\ P{\isacharparenright}{\isachardoublequoteclose}
\end{isabelle}
\noindent we can instantiate the locale \isatt{countable{\isacharunderscore}generic}

\begin{isabelle}

\ \ \isacommand{have}\isamarkupfalse%
\ \isanewline
\ \ \ \ \ \ \ \ \ Eq{\isadigit{2}}{\isacharcolon}\ {\isachardoublequoteopen}{\isasymforall}n{\isasymin}nat{\isachardot}\ {\isacharquery}D{\isacharbackquote}n\ {\isasymin}\ Pow{\isacharparenleft}P{\isacharparenright}{\isachardoublequoteclose}\isanewline
\ \ \ \ \isacommand{by}\isamarkupfalse%
\ auto\isanewline
\ \ \isacommand{then}\isamarkupfalse%
\ \isacommand{have}\isamarkupfalse%
\isanewline
\ \ \ \ \ \ \ \ \ Eq{\isadigit{3}}{\isacharcolon}\ {\isachardoublequoteopen}{\isacharquery}D{\isacharcolon}nat{\isasymrightarrow}Pow{\isacharparenleft}P{\isacharparenright}{\isachardoublequoteclose}\isanewline
\ \ \ \ \isacommand{by}\isamarkupfalse%
\ {\isacharparenleft}rule\ lam{\isacharunderscore}codomain{\isacharparenright}\isanewline
\ \ \isacommand{have}\isamarkupfalse%
\isanewline
\ \ \ \ \ \ \ \ \ Eq{\isadigit{4}}{\isacharcolon}\ {\isachardoublequoteopen}{\isasymforall}n{\isasymin}nat{\isachardot}\ dense{\isacharparenleft}{\isacharquery}D{\isacharbackquote}n{\isacharparenright}{\isachardoublequoteclose}
\end{isabelle}
\dots
\begin{isabelle}
\ \ \isacommand{from}\isamarkupfalse%
\ Eq{\isadigit{3}}\ \isakeyword{and}\ Eq{\isadigit{4}}\ \isacommand{interpret}\isamarkupfalse%
\ \isanewline
\ \ \ \ \ \ \ \ \ \ cg{\isacharcolon}\ countable{\isacharunderscore}generic\ P\ leq\ one\ {\isacharquery}D\ \isanewline
\ \ \ \ \isacommand{by}\isamarkupfalse%
\ {\isacharparenleft}unfold{\isacharunderscore}locales{\isacharcomma}\ auto{\isacharparenright}
\end{isabelle}
%
and then a  $\calD$-generic filter given by Rasiowa-Sikorski will be $M$-generic by construction. 

\begin{isabelle}

\ \ \isacommand{from}\isamarkupfalse%
\ cg{\isachardot}rasiowa{\isacharunderscore}sikorski\ \isakeyword{and}\ Eq{\isadigit{1}}\ \isacommand{obtain}\isamarkupfalse%
\ G\ \isakeyword{where}\ \isanewline
\ \ \ \ \ \ \ \ \ Eq{\isadigit{6}}{\isacharcolon}\ {\isachardoublequoteopen}p{\isasymin}G\ {\isasymand}\ filter{\isacharparenleft}G{\isacharparenright}\ {\isasymand}\ {\isacharparenleft}{\isasymforall}n{\isasymin}nat{\isachardot}{\isacharparenleft}{\isacharquery}D{\isacharbackquote}n{\isacharparenright}{\isasyminter}G{\isasymnoteq}{\isadigit{0}}{\isacharparenright}{\isachardoublequoteclose}\isanewline
\ \ \ \ \isacommand{unfolding}\isamarkupfalse%
\ cg{\isachardot}D{\isacharunderscore}generic{\isacharunderscore}def\ \isacommand{by}\isamarkupfalse%
\ blast\isanewline
\ \ \isacommand{then}\isamarkupfalse%
\ \isacommand{have}\isamarkupfalse%
\isanewline
\ \ \ \ \ \ \ \ \ Eq{\isadigit{7}}{\isacharcolon}\ {\isachardoublequoteopen}{\isacharparenleft}{\isasymforall}D{\isasymin}M{\isachardot}\ D{\isasymsubseteq}P\ {\isasymand}\ dense{\isacharparenleft}D{\isacharparenright}{\isasymlongrightarrow}D{\isasyminter}G{\isasymnoteq}{\isadigit{0}}{\isacharparenright}{\isachardoublequoteclose}
\end{isabelle}

\noindent We omit the rest of this Isar proof.

%%% Local Variables:
%%% mode: latex
%%% ispell-local-dictionary: "american"
%%% TeX-master: "first_steps_into_forcing"
%%% End:



%-%-%-%-%-%-%-%-%-%-%-%-%-%-%-%-%-%-%-%-%-%-%-%-%-%-%-%-%-%--%-%-%-%-%-%-%-%
\subsection{Names}
\label{sec:names}
In this section we formalize the function $\val$ that allows to
construct the elements of the generic extension $M[G]$ from elements
of the ctm $M$ and the generic filter $G$. The definition of $\val$
can be written succintly as
%
\begin{equation}\label{eq:def-val}
\val(G,\tau)\defi \{\val(G,\sigma) : \exists p\in \PP.\,
(\lb\sigma,p\rb\in\tau \y p\in G)\}.
\end{equation}
%
As with any recursive definition, to justify that the definition makes
sense (i.e., it defines anything at all) one has to explicit the
well-founded relation which is used. In this case, we use the relation
$\mathit{ed}$:
\[
x \mathrel{\mathit{ed}} y \iff \exists p . \lb x,p\rb\in y.
\]
Recall that in $\ZFC$, an ordered pair $\lb x,y \rb$ is the set
$\{\{x\},\{x,y\}\}$. It is trivial to deduce the well-foundedness of
$\mathit{ed}$ from the fact that $\in$ is well-founded, which follows
from the Foundation Axiom.

The formalization of $\val$ in Isabelle/ZF involves the function
\isa{wfrec},\footnote{Notice that this form of recursive
  definitions is more general than the one used in the previous
  section to define \isa{dc{\isacharunderscore}witness}.} which takes as an argument a
well-founded relation of type \isa{i}, i.e.\ a set of pairs. Therefore
we restrict $\mathit{ed}$ above to a set.
%
\begin{isabelle}
\isacommand{definition}\isamarkupfalse%
\isanewline
\ \ edrel\ {\isacharcolon}{\isacharcolon}\ {\isachardoublequoteopen}i\ {\isasymRightarrow}\ i{\isachardoublequoteclose}\ \isakeyword{where}\isanewline
\ \ {\isachardoublequoteopen}edrel{\isacharparenleft}A{\isacharparenright}\ {\isacharequal}{\isacharequal}\ {\isacharbraceleft}{\isacharless}x{\isacharcomma}y{\isachargreater}\ {\isasymin}\ A{\isacharasterisk}A\ {\isachardot}\ x\ {\isasymin}\ domain{\isacharparenleft}y{\isacharparenright}{\isacharbraceright}{\isachardoublequoteclose}
\end{isabelle}
%
Since \isa{edrel(A)} is a subset of a  well-founded relation (the
transitive closure of the membership relation restricted to \isa{A}),
then it is well-founded as well.

\begin{isabelle}
\isacommand{lemma}\isamarkupfalse%
\ edrel{\isacharunderscore}sub{\isacharunderscore}memrel{\isacharcolon}\ {\isachardoublequoteopen}edrel{\isacharparenleft}A{\isacharparenright}\ {\isasymsubseteq}\ trancl{\isacharparenleft}Memrel{\isacharparenleft}eclose{\isacharparenleft}A{\isacharparenright}{\isacharparenright}{\isacharparenright}{\isachardoublequoteclose}
\end{isabelle}
\dots
\begin{isabelle}
\isacommand{lemma}\isamarkupfalse%
\ wf{\isacharunderscore}edrel\ {\isacharcolon}\ {\isachardoublequoteopen}wf{\isacharparenleft}edrel{\isacharparenleft}A{\isacharparenright}{\isacharparenright}{\isachardoublequoteclose}\isanewline
\ \ \isacommand{apply}\isamarkupfalse%
\ {\isacharparenleft}rule\ wf{\isacharunderscore}subset\ {\isacharbrackleft}of\ {\isachardoublequoteopen}trancl{\isacharparenleft}Memrel{\isacharparenleft}eclose{\isacharparenleft}A{\isacharparenright}{\isacharparenright}{\isacharparenright}{\isachardoublequoteclose}{\isacharbrackright}{\isacharparenright}\isanewline
\ \ \isacommand{apply}\isamarkupfalse%
\ {\isacharparenleft}auto\ simp\ add{\isacharcolon}edrel{\isacharunderscore}sub{\isacharunderscore}memrel\ wf{\isacharunderscore}trancl\ wf{\isacharunderscore}Memrel{\isacharparenright}\isanewline
\ \ \isacommand{done}\isamarkupfalse%
%
\end{isabelle}

\noindent We now turn to the definition of $\val$. In order to use
\isa{wfrec} we need to define a functional \isa{Hv}.

\begin{isabelle}
\isacommand{definition}\isamarkupfalse%
\isanewline
\ \ Hv\ {\isacharcolon}{\isacharcolon}\ {\isachardoublequoteopen}i{\isasymRightarrow}i{\isasymRightarrow}i{\isasymRightarrow}i{\isachardoublequoteclose}\ \isakeyword{where}\isanewline
\ \ {\isachardoublequoteopen}Hv{\isacharparenleft}G{\isacharcomma}x{\isacharcomma}f{\isacharparenright}\ {\isacharequal}{\isacharequal}\ {\isacharbraceleft}\ f{\isacharbackquote}y\ {\isachardot}{\isachardot}\ y{\isasymin}\ domain{\isacharparenleft}x{\isacharparenright}{\isacharcomma}\ {\isasymexists}p{\isasymin}P{\isachardot}\ {\isacharless}y{\isacharcomma}p{\isachargreater}\ {\isasymin}\ x\ {\isasymand}\ p\ {\isasymin}\ G\ {\isacharbraceright}{\isachardoublequoteclose}\isanewline
\isanewline
\isacommand{definition}\isamarkupfalse%
\isanewline
\ \ val\ {\isacharcolon}{\isacharcolon}\ {\isachardoublequoteopen}i{\isasymRightarrow}i{\isasymRightarrow}i{\isachardoublequoteclose}\ \isakeyword{where}\isanewline
\ \ {\isachardoublequoteopen}val{\isacharparenleft}G{\isacharcomma}{\isasymtau}{\isacharparenright}\ {\isacharequal}{\isacharequal}\ wfrec{\isacharparenleft}edrel{\isacharparenleft}eclose{\isacharparenleft}M{\isacharparenright}{\isacharparenright}{\isacharcomma}\ {\isasymtau}{\isacharcomma}\ Hv{\isacharparenleft}G{\isacharparenright}{\isacharparenright}{\isachardoublequoteclose}
\end{isabelle}
Then we can recover the recursive expression~(\ref{eq:def-val}) thanks to the
following lemma:
%
\begin{isabelle}
\isacommand{lemma}\isamarkupfalse%
\ def{\isacharunderscore}val{\isacharcolon}
\isanewline
   {\isachardoublequoteopen}x{\isasymin}M\ {\isasymLongrightarrow}\ val{\isacharparenleft}G{\isacharcomma}x{\isacharparenright}\ {\isacharequal}\ {\isacharbraceleft}val{\isacharparenleft}G{\isacharcomma}t{\isacharparenright}\ {\isachardot}{\isachardot}\ t{\isasymin}domain{\isacharparenleft}x{\isacharparenright}{\isacharcomma}\ {\isasymexists}p{\isasymin}P\ {\isachardot}\ {\isasymlangle}t{\isacharcomma}\ p{\isasymrangle}{\isasymin}x\ {\isasymand}\ p{\isasymin}G{\isacharbraceright}{\isachardoublequoteclose}
\end{isabelle}

We can finally define the generic extension of $M$ by $G$, also
setting up the notation $M[G]$ for it:
\begin{isabelle}
\isacommand{definition}\isamarkupfalse%
\isanewline
\ \ GenExt\ {\isacharcolon}{\isacharcolon}\ {\isachardoublequoteopen}i{\isasymRightarrow}i{\isachardoublequoteclose} \ {\isacharparenleft}{\isachardoublequoteopen}M{\isacharbrackleft}{\isacharunderscore}{\isacharbrackright}{\isachardoublequoteclose}\ {\isadigit{9}}{\isadigit{0}}{\isacharparenright}\ \isakeyword{where} \isanewline
\ \ {\isachardoublequoteopen}GenExt{\isacharparenleft}G{\isacharparenright}{\isacharequal}{\isacharequal}\ {\isacharbraceleft}val{\isacharparenleft}G{\isacharcomma}{\isasymtau}{\isacharparenright}{\isachardot}\ {\isasymtau}\ {\isasymin}\ M{\isacharbraceright}{\isachardoublequoteclose}
\end{isabelle}

%% It is to be noted that this results holds under the assumption that
%% the name \isa{t} is in $M$.

We now provide names for elements in $M$. That is, for each $x\in M$,
we define $\chk(x)$ (usually denoted by $\check{x}$ in the literature)
such that $\val(G,\chk(x))=x$. This will show that $M\sbq M[G]$, with
a caveat we make explicit in the end of this section. As explained in
the introduction, the fact that $M[G]$ extends $M$ is crucial to show
that $\ZFC$ holds in the former.
%
The definition of $\chk(x)$ is a straightforward $\in$-recursion:
\begin{equation}
  \label{eq:def-check}
  \chk(x)\defi\{\lb\chk(y),\1\rb : y\in x\}
\end{equation}
Now the set-relation argument for \isa{wfrec} is the membership
relation restricted to a set \isa{A}, \isa{Memrel(A)}.

\begin{isabelle}
\isacommand{definition}\isamarkupfalse%
\ \isanewline
\ \ Hcheck\ {\isacharcolon}{\isacharcolon}\ {\isachardoublequoteopen}{\isacharbrackleft}i{\isacharcomma}i{\isacharbrackright}\ {\isasymRightarrow}\ i{\isachardoublequoteclose}\ \isakeyword{where}\isanewline
\ \ {\isachardoublequoteopen}Hcheck{\isacharparenleft}z{\isacharcomma}f{\isacharparenright}\ \ {\isacharequal}{\isacharequal}\ {\isacharbraceleft}\ {\isacharless}f{\isacharbackquote}y{\isacharcomma}one{\isachargreater}\ {\isachardot}\ y\ {\isasymin}\ z{\isacharbraceright}{\isachardoublequoteclose}\isanewline
\isanewline
\isacommand{definition}\isamarkupfalse%
\isanewline
\ \ check\ {\isacharcolon}{\isacharcolon}\ {\isachardoublequoteopen}i\ {\isasymRightarrow}\ i{\isachardoublequoteclose}\ \isakeyword{where}\isanewline
\ \ {\isachardoublequoteopen}check{\isacharparenleft}x{\isacharparenright}\ {\isacharequal}{\isacharequal}\ wfrec{\isacharparenleft}Memrel{\isacharparenleft}eclose{\isacharparenleft}{\isacharbraceleft}x{\isacharbraceright}{\isacharparenright}{\isacharparenright}{\isacharcomma}\ x\ {\isacharcomma}\ Hcheck{\isacharparenright}{\isachardoublequoteclose}
\end{isabelle}
Here, \isa{eclose} returns the (downward) $\in$-closure of its
argument. The main result is stated as follows:

\begin{isabelle}
\isacommand{lemma}\isamarkupfalse%
\ valcheck\ {\isacharcolon}\ {\isachardoublequoteopen}y\ {\isasymin}\ M\ {\isasymLongrightarrow}\ one\ {\isasymin}\ G\ {\isasymLongrightarrow}\ {\isasymforall}x{\isasymin}M{\isachardot}\ check{\isacharparenleft}x{\isacharparenright}\ {\isasymin}\ M\ {\isasymLongrightarrow}\isanewline \ \ \ \  \ \  \ \  \ \  val{\isacharparenleft}G{\isacharcomma}check{\isacharparenleft}y{\isacharparenright}{\isacharparenright}\ \ {\isacharequal}\ y{\isachardoublequoteclose}
\end{isabelle}

On the one hand, the only requirement on the set \isa{G} is that it
contains the top element of the poset \isa{P}. On the other hand, it
is necessary that all the relevant names are indeed in $M$ (i.e.,
$\chk(x) \in M$ if $x\in M$). It requires a serious development to
fulfill this assumption. One of the hardest parts of Paulson's
formalization of constructibility involves showing that models are
closed under recursive construction. We will eventually formalize that
if $M\models\ZFC$ and the arguments of \isa{wfrec} are in $M$, then
its value also is. This will require to adapt to ctm models several
locales defined in \cite{paulson_2003} that were intended to be used
for the class of constructible sets.




%%% Local Variables:
%%% mode: latex
%%% ispell-local-dictionary: "american"
%%% TeX-master: "first_steps_into_forcing"
%%% End:



%%%%%%%%%%%%%%%%%%%%%%%%%%%%%%%%%%%%%%%%%%%%%%%%%%%%%%%%%%%%%%%%%%%%%%%%%%%%%%%%
\section{Pairing in the generic extension}
\label{sec:pairing-generic-extension}
Given a predicate
\isa{C\ {\isacharcolon}{\isacharcolon}\ {\isachardoublequoteopen}i{\isasymRightarrow}o{\isachardoublequoteclose}},
we can consider the \emph{class} of all sets \isa{x} such that
\isa{C(x)}. In \cite{MR2507047} a large corpus of set-theoretical
concepts are expressed in relative form.

In the case of the Pairing Axiom the following definition in 
Paulson's formalization captures
what a class \isa{C} believes about a set being the unordered pair of
other two:
\begin{isabelle}
\isacommand{definition}\isamarkupfalse%
\isanewline
\ \ upair\ {\isacharcolon}{\isacharcolon}\ {\isachardoublequoteopen}{\isacharbrackleft}i{\isasymRightarrow}o{\isacharcomma}i{\isacharcomma}i{\isacharcomma}i{\isacharbrackright}\ {\isasymRightarrow}\ o{\isachardoublequoteclose}\ \isakeyword{where}\isanewline
\ \ \ \ {\isachardoublequoteopen}upair{\isacharparenleft}C{\isacharcomma}a{\isacharcomma}b{\isacharcomma}z{\isacharparenright}\ {\isacharequal}{\isacharequal}\ a\ {\isasymin}\ z\ {\isasymand}\ b\ {\isasymin}\ z\ {\isasymand}\ {\isacharparenleft}{\isasymforall}x{\isacharbrackleft}C{\isacharbrackright}{\isachardot}\ x{\isasymin}z\ {\isasymlongrightarrow}\ x\ {\isacharequal}\ a\ {\isasymor}\ x\ {\isacharequal}\ b{\isacharparenright}{\isachardoublequoteclose}
\end{isabelle}
%
Then the fact that Pairing axiom holds in \isa{C} is stated as follows.
%
\begin{isabelle}
\isacommand{definition}\isamarkupfalse%
\isanewline
\ \ upair{\isacharunderscore}ax\ {\isacharcolon}{\isacharcolon}\ {\isachardoublequoteopen}{\isacharparenleft}i{\isasymRightarrow}o{\isacharparenright}\ {\isasymRightarrow}\ o{\isachardoublequoteclose}\ \isakeyword{where}\isanewline
\ \ \ \ {\isachardoublequoteopen}upair{\isacharunderscore}ax{\isacharparenleft}C{\isacharparenright}\ {\isacharequal}{\isacharequal}\ {\isasymforall}x{\isacharbrackleft}C{\isacharbrackright}{\isachardot}\ {\isasymforall}y{\isacharbrackleft}C{\isacharbrackright}{\isachardot}\ {\isasymexists}z{\isacharbrackleft}C{\isacharbrackright}{\isachardot}\ upair{\isacharparenleft}C{\isacharcomma}x{\isacharcomma}y{\isacharcomma}z{\isacharparenright}{\isachardoublequoteclose}
\end{isabelle}

In order use the \isa{upair{\isacharunderscore}ax} for a set model, we
use the \isatt{\#\#}  operator.  We state the main result of this
Section in the context \isatt{forcing{\isacharunderscore}data}.  
%
\begin{isabelle}
\isacommand{lemma}\isamarkupfalse%
\ pair{\isacharunderscore}preserv\ {\isacharcolon}\ \isanewline
\ \ {\isachardoublequoteopen}one\ {\isasymin}\ G\ {\isasymLongrightarrow}\ upair{\isacharunderscore}ax{\isacharparenleft}{\isacharhash}{\isacharhash}M{\isacharparenright}\ {\isasymLongrightarrow}\ upair{\isacharunderscore}ax{\isacharparenleft}{\isacharhash}{\isacharhash}M{\isacharbrackleft}G{\isacharbrackright}{\isacharparenright}{\isachardoublequoteclose}
\end{isabelle}

Let $x$ and $y$ be elements in $M[G]$. By definition of the generic extension, there exist
elements $\tau$ and $\rho$ in $M$ such that $x = \val(G,\tau)$ and
$y = \val(G,\rho)$.  We need to find an element in $M[G]$ that contains exactly
these elements; for that we should construct a name $\sigma\in M$ such that
$\val(G,\sigma) = \{ \val(G,\tau) , \val(G,\rho) \}$. 

The candidate, motivated by the definition of $\chk$,  is
$\sigma = \{\langle \tau , \mathrm{one} \rangle , \langle \rho , \mathrm{one} \rangle \}$. 
Our remaining tasks are to show 
\begin{enumerate}
  \item \label{item:1}$\sigma \in M$, and
  \item \label{item:2} $\val(G,\sigma) = \{ \val(G,\tau) , \val(G,\rho) \}$
\end{enumerate}

By the implementation of pairs  in $\ZFC$, showing (\ref{item:1})
involves using that the
pairing axiom holds in $M$ and the absoluteness of pairing
thanks to $M$ being transitive. 

\begin{isabelle}
\isacommand{lemma}\isamarkupfalse%
\ pairs{\isacharunderscore}in{\isacharunderscore}M\ {\isacharcolon}\ \isanewline
\ \ {\isachardoublequoteopen}\ {\isasymlbrakk}\ upair{\isacharunderscore}ax{\isacharparenleft}{\isacharhash}{\isacharhash}M{\isacharparenright}\ {\isacharsemicolon}\ a\ {\isasymin}\ M\ {\isacharsemicolon}\ b\ {\isasymin}\ M\ {\isacharsemicolon}\ c\ {\isasymin}\ M\ {\isacharsemicolon}\ d\ {\isasymin}\ M\ {\isasymrbrakk}\ {\isasymLongrightarrow}\ {\isacharbraceleft}{\isasymlangle}a{\isacharcomma}c{\isasymrangle}{\isacharcomma}{\isasymlangle}b{\isacharcomma}d{\isasymrangle}{\isacharbraceright}\ {\isasymin}\ M{\isachardoublequoteclose}
\end{isabelle}

Item (\ref{item:1}) then follows because \isa{\isasymtau}, \isa{\isasymrho} and
\isa{one} belong to \isa{M} (the last fact holds because \isa{one\isasymin P}, \isa{P\isasymin M} and
\isa{M} is transitive).

\begin{isabelle}
\isacommand{lemma}\isamarkupfalse%
\ sigma{\isacharunderscore}in{\isacharunderscore}M\ {\isacharcolon}\isanewline
\ \ {\isachardoublequoteopen}upair{\isacharunderscore}ax{\isacharparenleft}{\isacharhash}{\isacharhash}M{\isacharparenright}\ {\isasymLongrightarrow}\ one\ {\isasymin}\ G\ {\isasymLongrightarrow}\ {\isasymtau}\ {\isasymin}\ M\ {\isasymLongrightarrow}\ {\isasymrho}\ {\isasymin}\ M\ {\isasymLongrightarrow}\ {\isacharbraceleft}{\isasymlangle}{\isasymtau}{\isacharcomma}one{\isasymrangle}{\isacharcomma}{\isasymlangle}{\isasymrho}{\isacharcomma}one{\isasymrangle}{\isacharbraceright}\ {\isasymin}\ M{\isachardoublequoteclose}

\isacommand{by}\isamarkupfalse%
\ {\isacharparenleft}rule\ pairs{\isacharunderscore}in{\isacharunderscore}M{\isacharcomma}simp{\isacharunderscore}all\ add{\isacharcolon}\ upair{\isacharunderscore}ax{\isacharunderscore}def\ one{\isacharunderscore}in{\isacharunderscore}M{\isacharparenright}%
\end{isabelle}

Under the assumption that \isa{one} belongs to the set \isa{G},
(\ref{item:2}) follows from \isa{def\_val} almost automatically:

\begin{isabelle}
\isacommand{lemma}\isamarkupfalse%
\ valsigma\ {\isacharcolon}\isanewline
\ \ {\isachardoublequoteopen}one\ {\isasymin}\ G\ {\isasymLongrightarrow}\ {\isacharbraceleft}{\isasymlangle}{\isasymtau}{\isacharcomma}one{\isasymrangle}{\isacharcomma}{\isasymlangle}{\isasymrho}{\isacharcomma}one{\isasymrangle}{\isacharbraceright}\ {\isasymin}\ M\ {\isasymLongrightarrow}\isanewline
\ \ \ val{\isacharparenleft}G{\isacharcomma}{\isacharbraceleft}{\isasymlangle}{\isasymtau}{\isacharcomma}one{\isasymrangle}{\isacharcomma}{\isasymlangle}{\isasymrho}{\isacharcomma}one{\isasymrangle}{\isacharbraceright}{\isacharparenright}\ {\isacharequal}\ {\isacharbraceleft}val{\isacharparenleft}G{\isacharcomma}{\isasymtau}{\isacharparenright}{\isacharcomma}val{\isacharparenleft}G{\isacharcomma}{\isasymrho}{\isacharparenright}{\isacharbraceright}{\isachardoublequoteclose}
\end{isabelle}

This ends the proof of the Pairing Axiom in $M[G]$.



%%%%%%%%%%%%%%%%%%%%%%%%%%%%%%%%%%%%%%%%%%%%%%%%%%%%%%%%%%%%%%%%%%%%%%%%%%%%%%%%
\section{Conclusions and future work}
There are several technical milestones that have to be reached in the
course of a formalization of the theory of forcing. The first one, and most
obvious, is the bulk of set- and meta-theoretical concepts needed to work
with. This pushed us, in a sense,  into building on top of Isabelle/ZF,
since we know of no other development in set theory of such
depth (and breadth). In this paper we worked on setting the stage for the work with
generic extensions; in particular, this involves some purely mathematical
results, as the Rasiowa-Sikorski lemma. 

Other milestones in this formalization project
involve 
\begin{enumerate}
\item the definition
  of the forcing relation, 
\item proving the Fundamental Theorem of forcing
  (that relates truth in $M$ to that in $M[G]$), and 
\item using it to show
  that $M[G]\models \ZFC$. 
\end{enumerate}
The theory is very modular and this is
witnessed by the fact 
that the last goal does not depend on the proof of the Fundamental
Theorem nor on the definition of the forcing relation. Our next task
will be to obtain the last goal in that enumeration. 

To this end, we will develop an interface between Paulson's
relativization results and countable models of $\ZFC$. This will show
that every ctm $M$ is closed under well-founded recursion and, in
particular, that contains names for each of its
elements. Consequently, the proof of  $M\sbq M[G]$ will be
complete. A landmark will be to prove the Axiom Scheme
of Separation (the first that needs to use the machinery of forcing
nontrivially). As a part of the new formalization, we will provide
Isar versions of the longer applicative proofs presented in this work.

\ack{We'd like to thank the anonymous referees for reading the paper
  carefully and for their detailed and constructive criticism.}
%%% Local Variables:
%%% mode: latex
%%% ispell-local-dictionary: "american"
%%% TeX-master: "first_steps_into_forcing"
%%% End:


%\bibliographystyle{mi-estilo-else}
%\bibliography{../citados}
\documentclass[9pt]{entcs} \usepackage{entcsmacro}
\usepackage{graphicx}
\sloppy

\usepackage{amsmath}
%\usepackage{amsthm}
\usepackage{amsfonts}
\usepackage{amssymb}
%\usepackage{bbm}  % Para el \bb{1}
%\usepackage[numbers]{natbib}
\usepackage{enumitem}
\usepackage{babel}
%\usepackage{babelbib}
\usepackage{multidef}
\usepackage{verbatim}
\usepackage{stmaryrd} %% para \llbracket
%%
%% \usepackage[bottom=2cm, top=2cm, left=2cm, right=2cm]{geometry}
%% \usepackage{titling}
%% \setlength{\droptitle}{-10ex} 
%%
\renewcommand{\o}{\vee}
\renewcommand{\O}{\bigvee}
\newcommand{\y}{\wedge}
\newcommand{\Y}{\bigwedge}
\newcommand{\limp}{\rightarrow}
\newcommand{\lsii}{\leftrightarrow}
%%

\DeclareMathOperator{\cf}{cf}
\DeclareMathOperator{\dom}{dom}
\DeclareMathOperator{\im}{img}
\DeclareMathOperator{\Fn}{Fn}
\DeclareMathOperator{\rk}{rk}
\DeclareMathOperator{\mos}{mos}
\DeclareMathOperator{\trcl}{trcl}
\DeclareMathOperator*{\diag}{\bigtriangleup}
\DeclareMathOperator{\Con}{Con}
\DeclareMathOperator{\Club}{Club}


\newcommand{\modelo}[1]{\mathbf{#1}}
\newcommand{\axiomas}[1]{\mathit{#1}}
\newcommand{\clase}[1]{\mathsf{#1}}
\newcommand{\poset}[1]{\mathbb{#1}}
\newcommand{\operador}[1]{\mathbf{#1}}

%% \newcommand{\Lim}{\clase{Lim}}
%% \newcommand{\Reg}{\clase{Reg}}
%% \newcommand{\Card}{\clase{Card}}
%% \newcommand{\On}{\clase{On}}
%% \newcommand{\WF}{\clase{WF}}
%% \newcommand{\HF}{\clase{HF}}
%% \newcommand{\HC}{\clase{HC}}
%%
%% El siguiente comando reemplaza todos los anteriores:
%%
\multidef{\clase{#1}}{Card,HC,HF,Lim,On->Ord,Reg,WF,Ord}
\newcommand{\ON}{\On}

%% En lugar de usar todo el paquete bbm:
\DeclareMathAlphabet{\mathbbm}{U}{bbm}{m}{n} 
\newcommand{\1}{\mathbbm{1}}

%%
%% \newcommand{\calD}{\mathcal{D}}
%% \newcommand{\calS}{\mathcal{S}}
%% \newcommand{\calU}{\mathcal{U}}
%% \newcommand{\calB}{\mathcal{B}}
%% \newcommand{\calL}{\mathcal{L}}
%% \newcommand{\calF}{\mathcal{F}}
%% \newcommand{\calT}{\mathcal{T}}
%% \newcommand{\calW}{\mathcal{W}}
%% \newcommand{\calA}{\mathcal{A}}
%%
%% El siguiente comando reemplaza todos los anteriores:
%%
\multidef[prefix=cal]{\mathcal{#1}}{A-Z}
%%
%% \newcommand{\A}{\modelo{A}}
%% \newcommand{\BB}{\modelo{B}}
%% \newcommand{\ZZ}{\modelo{Z}}
%% \newcommand{\PP}{\modelo{P}}
%% \newcommand{\QQ}{\modelo{Q}}
%% \newcommand{\RR}{\modelo{R}}
%%
%% El siguiente comando reemplaza todos los anteriores:
%%
\multidef{\modelo{#1}}{A,BB->B,CC->C,NN->N,PP->P,QQ->Q,RR->R,ZZ->Z}

\multidef[prefix=p]{\mathbb{#1}}{A-Z}
%% \newcommand{\B}{\modelo{B}}
%% \newcommand{\C}{\modelo{C}}
%% \newcommand{\F}{\modelo{F}}
%% \newcommand{\D}{\modelo{D}}

\newcommand{\Th}{\mb{Th}}
\newcommand{\Mod}{\mb{Mod}}

\newcommand{\Se}{\operador{S^\prec}}
\newcommand{\Pu}{\operador{P_u}}
\renewcommand{\Pr}{\operador{P_R}}
\renewcommand{\H}{\operador{H}}
\renewcommand{\S}{\operador{S}}
\newcommand{\I}{\operador{I}}
\newcommand{\E}{\operador{E}}

\newcommand{\se}{\preccurlyeq}
\newcommand{\ee}{\succ}
\newcommand{\id}{\approx}
\newcommand{\subm}{\subseteq}
\newcommand{\ext}{\supseteq}
\newcommand{\iso}{\cong}
%%
\renewcommand{\emptyset}{\varnothing}
\newcommand{\rel}{\mathcal{R}}
\newcommand{\Pow}{\mathop{\mathcal{P}}}
\renewcommand{\P}{\Pow}
\newcommand{\BP}{\mathrm{BP}}
\newcommand{\func}{\rightarrow}
\newcommand{\ord}{\mathrm{Ord}}
\newcommand{\R}{\mathbb{R}}
\newcommand{\N}{\mathbb{N}}
\newcommand{\Z}{\mathbb{Z}}
\renewcommand{\I}{\mathbb{I}}
\newcommand{\Q}{\mathbb{Q}}
\newcommand{\B}{\mathbf{B}}
\newcommand{\<}{\langle}
\renewcommand{\>}{\rangle}
\newcommand{\lb}{\langle}
\newcommand{\rb}{\rangle}
\newcommand{\impl}{\rightarrow}
\newcommand{\ent}{\Rightarrow}
\newcommand{\tne}{\Leftarrow}
\newcommand{\sii}{\Leftrightarrow}
\renewcommand{\phi}{\varphi}
\newcommand{\phis}{{\varphi^*}}
\renewcommand{\th}{\theta}
\newcommand{\Lda}{\Lambda}
\newcommand{\La}{\Lambda}
\newcommand{\lda}{\lambda}
\newcommand{\ka}{\kappa}
\newcommand{\del}{\delta}
\newcommand{\de}{\delta}
\newcommand{\ze}{\zeta}
%\newcommand{\ }{\ }
\newcommand{\la}{\lambda}
\newcommand{\al}{\alpha}
\newcommand{\be}{\beta}
\newcommand{\ga}{\gamma}
\newcommand{\Ga}{\Gamma}
\newcommand{\ep}{\varepsilon}
\newcommand{\De}{\Delta}
\newcommand{\defi}{\mathrel{\mathop:}=}
\newcommand{\forces}{\Vdash}
%\newcommand{\ap}{\mathbin{\wideparen{\ }}}
\newcommand{\Tree}{{\mathrm{Tr}_\N}}
\newcommand{\PTree}{{\mathrm{PTr}_\N}}
\newcommand{\NWO}{\mathit{NWO}}
\newcommand{\Suc}{{\N^{<\N}}}%
\newcommand{\init}{\mathsf{i}}
\newcommand{\ap}{\mathord{^\smallfrown}}
\newcommand{\Cantor}{\mathcal{C}}
%\newcommand{\C}{\Cantor}
\newcommand{\Baire}{\mathcal{N}}
\newcommand{\sig}{\ensuremath{\sigma}}
\newcommand{\fsig}{\ensuremath{F_\sigma}}
\newcommand{\gdel}{\ensuremath{G_\delta}}
\newcommand{\Sig}{\ensuremath{\boldsymbol{\Sigma}}}
\newcommand{\bPi}{\ensuremath{\boldsymbol{\Pi}}}
\newcommand{\Del}{\ensuremath{\boldsymbol\Delta}}
%\renewcommand{\F}{\operador{F}}
\newcommand{\ths}{{\theta^*}}
\newcommand{\om}{\ensuremath{\omega}}
%\renewcommand{\c}{\complement}
\newcommand{\comp}{\mathsf{c}}
\newcommand{\co}[1]{\left(#1\right)^\comp}
\newcommand{\len}[1]{\left|#1\right|}
\DeclareMathOperator{\tlim}{\overline{\mathrm{TLim}}}
\newcommand{\card}[1]{{\left|#1\right|}}
\newcommand{\bigcard}[1]{{\bigl|#1\bigr|}}
%
% Cardinality
%
\newcommand{\lec}{\leqslant_c}
\newcommand{\gec}{\geqslant_c}
\newcommand{\lc}{<_c}
\newcommand{\gc}{>_c}
\newcommand{\eqc}{=_c}
\newcommand{\biy}{\approx}
\newcommand*{\ale}[1]{\aleph_{#1}}
%
\newcommand{\Zerm}{\axiomas{Z}}
\newcommand{\ZC}{\axiomas{ZC}}
\newcommand{\AC}{\axiomas{AC}}
\newcommand{\DC}{\axiomas{DC}}
\newcommand{\MA}{\axiomas{MA}}
\newcommand{\CH}{\axiomas{CH}}
\newcommand{\ZFC}{\axiomas{ZFC}}
\newcommand{\ZF}{\axiomas{ZF}}
\newcommand{\Inf}{\axiomas{Inf}}
%
% Cardinal characteristics
%
\newcommand{\cont}{\mathfrak{c}}
\newcommand{\spl}{\mathfrak{s}}
\newcommand{\bound}{\mathfrak{b}}
\newcommand{\mad}{\mathfrak{a}}
\newcommand{\tower}{\mathfrak{t}}
%
\renewcommand{\hom}[2]{{}^{#1}\hskip-0.116ex{#2}}
\newcommand{\pred}[1][{}]{\mathop{\mathrm{pred}_{#1}}}
%% Postfix operator with supressable space:
%% \newcommand*{\iseg}{\relax\ifnum\lastnodetype>0 \mskip\medmuskip\fi{\downarrow}} %
\newcommand*{\iseg}{{\downarrow}}
\newcommand{\rr}{\mathrel{R}}
\newcommand{\restr}{\upharpoonright}
%\newcommand{\type}{\mathtt{}}
\newcommand{\app}{\mathop{\mathrm{Aprox}}}
\newcommand{\hess}{\triangleleft}
\newcommand{\bx}{\bar{x}}
\newcommand{\by}{\bar{y}}
\newcommand{\bz}{\bar{z}}
\newcommand{\union}{\mathop{\textstyle\bigcup}}
\newcommand{\sm}{\setminus}
\newcommand{\sbq}{\subseteq}
\newcommand{\nsbq}{\subseteq}
\newcommand{\mty}{\emptyset}
\newcommand{\dimg}{\text{\textup{``}}} % direct image
\newcommand{\quine}[1]{\ulcorner{\!#1\!}\urcorner}
%\newcommand{\ntrm}[1]{\textsl{\textbf{#1}}}
\newcommand{\Null}{\calN\!\mathit{ull}}
\DeclareMathOperator{\club}{Club}
\DeclareMathOperator{\otp}{otp}

%%%%%%%%%%%%%%%%%%%%%%%%%
% Variant aleph, beth, etc
% From http://tex.stackexchange.com/q/170476/69595
\makeatletter
\@ifpackageloaded{txfonts}\@tempswafalse\@tempswatrue
\if@tempswa
  \DeclareFontFamily{U}{txsymbols}{}
  \DeclareFontFamily{U}{txAMSb}{}
  \DeclareSymbolFont{txsymbols}{OMS}{txsy}{m}{n}
  \SetSymbolFont{txsymbols}{bold}{OMS}{txsy}{bx}{n}
  \DeclareFontSubstitution{OMS}{txsy}{m}{n}
  \DeclareSymbolFont{txAMSb}{U}{txsyb}{m}{n}
  \SetSymbolFont{txAMSb}{bold}{U}{txsyb}{bx}{n}
  \DeclareFontSubstitution{U}{txsyb}{m}{n}
  \DeclareMathSymbol{\aleph}{\mathord}{txsymbols}{64}
  \DeclareMathSymbol{\beth}{\mathord}{txAMSb}{105}
  \DeclareMathSymbol{\gimel}{\mathord}{txAMSb}{106}
  \DeclareMathSymbol{\daleth}{\mathord}{txAMSb}{107}
\fi
\makeatother

%%%%%%%%%%%%%%%%%%%%%%%%%%%%%%%%%%%%%%%%%%%%%%%%%%%%%%%%%%%%
%%
%% Theorem Environments
%%
%% \newtheorem{theorem}{Theorem}
%% \newtheorem{lemma}[theorem]{Lemma}
%% \newtheorem{prop}[theorem]{Proposition}
%% \newtheorem{corollary}[theorem]{Corollary}
%% \newtheorem{claim}{Claim}
%% \newtheorem*{claim*}{Claim}
%% \theoremstyle{definition}
%% \newtheorem{definition}[theorem]{Definition}
%% \newtheorem{remark}[theorem]{Remark}
%% \newtheorem{example}[theorem]{Example}
%% \theoremstyle{remark}
%% \newtheorem*{remark*}{Remark}
%%
%%%%%%%%%%%%%%%%%%%%%%%%%%%%%%%%%%%%%%%%%%%%%%%%%%%%%%%%%%%%%%%%%%%%%%

%% \newenvironment{inducc}{\begin{list}{}{\itemindent=2.5em \labelwidth=4em}}{\end{list}}
%% \newcommand{\caso}[1]{\item[\fbox{#1}]}
\newenvironment{proofofclaim}{\begin{proof}[Proof of Claim]}{\end{proof}}


%%% Local Variables: 
%%% mode: latex
%%% TeX-master: "first_steps_into_forcing"
%%% End: 

\def\lastname{Gunther, Pagano, S\'anchez Terraf}
\begin{document}
\begin{frontmatter}
  \title{First steps towards a formalization of Forcing}
  \author{Emmanuel Gunther%\thanksref{ALL}
    \thanksref{myemail}}
  \address{FaMAF\\ Universidad Nacional de C\'ordoba\\
    C\'ordoba, Argentina} \author{Miguel Pagano\thanksref{coemail}}
  \address{FaMAF\\Universidad Nacional de C\'ordoba\\
    C\'ordoba, Argentina}
  \author{Pedro S\'anchez Terraf\thanksref{co2email}}
  \address{CIEM-FaMAF\\Universidad Nacional de C\'ordoba\\
    C\'ordoba, Argentina}
 \thanks[myemail]{Email:
    \href{mailto:gunther@famaf.unc.edu.ar} {\texttt{\normalshape
        gunther@famaf.unc.edu.ar}}} \thanks[coemail]{Email:
    \href{mailto:pagano@famaf.unc.edu.ar} {\texttt{\normalshape
        pagano@famaf.unc.edu.ar}}}  \thanks[co2email]{Email:
    \href{mailto:sterraf@famaf.unc.edu.ar} {\texttt{\normalshape
        sterraf@famaf.unc.edu.ar}}} 
 \thanks[ALL]{Supported by Secyt-UNC project 33620180100465CB.} 
\begin{abstract} 
\end{abstract}
\begin{keyword}
Isabelle/ZF, forcing, preorder, Rasiowa-Sikorski lemma, names, generic extension.
\end{keyword}
\end{frontmatter}

%%%%%%%%%%%%%%%%%%%%%%%%%%%%%%%%%%%%%%%%%%%%%%%%%%%%%%%%%%%%%%%%%%%%%%%%%%%%%%%%
\section{Introduction}\label{sec:introduction}

%-%-%-%-%-%-%-%-%-%-%-%-%-%-%-%-%-%-%-%-%-%-%-%-%-%-%-%-%-%-%-%-%-%-%-%-%-%-%-%
\subsection{Models of ZFC}

%-%-%-%-%-%-%-%-%-%-%-%-%-%-%-%-%-%-%-%-%-%-%-%-%-%-%-%-%-%-%-%-%-%-%-%-%-%-%-%
\subsection{Isabelle/ZF}



%%%%%%%%%%%%%%%%%%%%%%%%%%%%%%%%%%%%%%%%%%%%%%%%%%%%%%%%%%%%%%%%%%%%%%%%%%%%%%%%
\section{Forcing notions}\label{sec:forcing-posets}
\textit{Para no confundir, directamente usamos este nombre en lugar de
  ``forcing posets'' para los preórdenes con un máximo distinguido}

%-%-%-%-%-%-%-%-%-%-%-%-%-%-%-%-%-%-%-%-%-%-%-%-%-%-%-%-%-%-%-%-%-%-%-%-%-%-%-%
\subsection{The Rasiowa-Sikorski lemma}
In order to formalize of the proof of the Rasiowa-Sikorski lemma in
general, a version of the Axiom of Choice ($\AC$) must be proved. This
version is known as the \emph{Principle of Dependent choices}:
\begin{quote}
  ($\DC$) Let $\rr$ be a binary relation on $A$, and $a\in A$. If
  $\forall x\in A\,  \exists y\in A\, x\rr y$, then there exists
  $f:\om\to A$ such that $f(0)=a$ and $f(n)\rr f(n+1)$ for all
  $n\in\om$.
\end{quote}

A different version of $\DC$ (without the constraint $f(0)=a$, but
allowing a parameter that allows to prove an equivalence with $\AC$) was
formalized as part of the Isabelle/ZF library, along with many
statements that follow from or are equivalent to $\AC$. Since the
proofs there required a greater generality, it would have been very
contrived to derived this ``pointed'' version of $\DC$ from that
development. Instead, we preferred to give a direct proof based in the
one that appears in Moschovakis \cite{moschovakis1994notes}. Another
advantage is that this proof depends on the following version of $\AC$
(existence of choice functions on $\P(X)\sm \{\mty\}$):
\[
\exists f:\P(X)\sm \{\mty\}\to X \, \forall A\sbq X.\ A\neq \mty \limp
f(A) \in A.
\]
This statement is more standard than the one deemed ``$\AC$'' in the
Isabelle/ZF development, that involves a silent use of Replacement
since it is written as an axiom scheme.


%%%%%%%%%%%%%%%%%%%%%%%%%%%%%%%%%%%%%%%%%%%%%%%%%%%%%%%%%%%%%%%%%%%%%%%%%%%%%%%%
\section{Names}


%%%%%%%%%%%%%%%%%%%%%%%%%%%%%%%%%%%%%%%%%%%%%%%%%%%%%%%%%%%%%%%%%%%%%%%%%%%%%%%%
\section{The generic extension}
Forcing is a technique to extend countable transitive models of
$\ZFC$. This process is guaranteed to preserve the $\ZFC$
axioms but allows to fine-tune what other first order properties the
extension will have. The first, both historically and in importance,
application of forcing was the proof that $\AC$ and $\CH$ were not
derivable from $\ZFC$; this was done by Cohen in its seminal work
\cite{Cohen-CH-PNAS}.

 

%%%%%%%%%%%%%%%%%%%%%%%%%%%%%%%%%%%%%%%%%%%%%%%%%%%%%%%%%%%%%%%%%%%%%%%%%%%%%%%%
\section{Future work}
\begin{itemize}
\item Develop an interface between Paulson's relativization results
  and countable models $M$ of $\ZFC$. This will show
  that every such $M$ is closed under well-founded recursion and, in
  particular, that contains names for each of its
  elements. Consequently, $M\sbq M[G]$.
\item Prove that the basic axioms of $\ZFC$ hold in the generic
  extension. 
\item After prototyping the forcing relation, prove the Axiom Scheme
  of Separation (the first that needs to use the machinery of forcing
  nontrivially).
\end{itemize}
\bibliographystyle{entcs}
\bibliography{citados,journals&preprints,conferencias}
\end{document}

%%% Local Variables:
%%% mode: latex
%%% ispell-local-dictionary: "american"
%%% End:

\end{document}

%%% Local Variables:
%%% mode: latex
%%% ispell-local-dictionary: "american"
%%% End:
