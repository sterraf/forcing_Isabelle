%%%%%%%%%%%%%%%%%%%%%%%%%%%%%%%%%%%%%%%%%%%%%%%%%%%%%%%%%%%%%%%%%%%%%%          
\section{Set models and forcing}
\label{sec:forcing}

\subsection{The $\ZFC$ axioms as locales}\label{sec:zfc-axioms-as-locales}
The description of set models of fragments of $\ZFC$ was performed
using locales that fix a variable $M::\tyi$ and pack assumptions
stating that $\lb M, \in\rb$ satisfy some axioms; for example, the
locale \locale{M{\uscore}Z{\uscore}basic} states that Zermelo set
theory holds in $M$. It would be natural to state those assumptions
directly as the corresponding satisfactions, as in
\[
  M, [\,]\models \isa{{\isasymcdot}Union\ Ax{\isasymcdot}}
\]
where \isa{{\isasymcdot}Union\ Ax{\isasymcdot}} is the $\formula$ code for the
Union Axiom. We actually decided to express the axioms other than the
infinite schemes in relational form, by using terms already
available in \session{ZF-Constructible} and for which useful lemmas
had already been proved (and, as it was mentioned in Section~\ref{sec:isabellezf},
this third layer of the formalization has more tools at our
disposal); the Union Axiom (\isa{Union{\uscore}ax}), for
instance, is defined as follows:
\[
\forall x[\isa{\#\#}M].\ \exists z[\isa{\#\#}M].\ \isa{big{\uscore}union}(\isa{\#\#}M,x,z)
\]
%% % Same a above
%% \begin{isabelle}%
%% {\isasymforall}x{\isacharbrackleft}{\kern0pt}{\isacharhash}{\kern0pt}{\isacharhash}{\kern0pt}M{\isacharbrackright}{\kern0pt}{\isachardot}{\kern0pt}\ {\isasymexists}z{\isacharbrackleft}{\kern0pt}{\isacharhash}{\kern0pt}{\isacharhash}{\kern0pt}M{\isacharbrackright}{\kern0pt}{\isachardot}{\kern0pt}\ big{\isacharunderscore}{\kern0pt}union{\isacharparenleft}{\kern0pt}{\isacharhash}{\kern0pt}{\isacharhash}{\kern0pt}M{\isacharcomma}{\kern0pt}\ x{\isacharcomma}{\kern0pt}\ z{\isacharparenright}{\kern0pt}%
%% \end{isabelle}%
Both assumptions are then shown to be equivalent:
\[
  \isa{Union{\uscore}ax}(\isa{\#\#}M) \lsii M, [\,]\models \isa{{\isasymcdot}Union\ Ax{\isasymcdot}}
\]
%% % Same a above
%% \begin{isabelle}
%% Union{\isacharunderscore}{\kern0pt}ax{\isacharparenleft}{\kern0pt}{\isacharhash}{\kern0pt}{\isacharhash}{\kern0pt}A{\isacharparenright}{\kern0pt}\ {\isasymlongleftrightarrow}\ A{\isacharcomma}{\kern0pt}\ {\isacharbrackleft}{\kern0pt}{\isacharbrackright}{\kern0pt}\ {\isasymTurnstile}\ {\isasymcdot}Union\ Ax{\isasymcdot}\isasep
%% \end{isabelle}

For stating the axiom schemes, \session{ZF-Constructible} defines
$\isa{separation}(N,Q)$:
\[
  \forall z[N].\ \exists y[N].\ \forall
  x[N].\ x\in y \longleftrightarrow x\in z \land Q(x),
\]
and $\isa{strong{\uscore}replacement}(N,R)$:
\begin{multline*}
  \forall
  A[N].\ \isa{univalent}(N,A,R) \longrightarrow \\
  (\exists Y[N].\ \forall b[N].\ b\in Y\longleftrightarrow (\exists x[N].\ x\in
  A \land R(x,b))),
\end{multline*}
%% \begin{isabelle}%
%% separation{\isacharparenleft}{\kern0pt}N{\isacharcomma}{\kern0pt}\ Q{\isacharparenright}{\kern0pt}\ {\isasymequiv}\ {\isasymforall}z{\isacharbrackleft}{\kern0pt}N{\isacharbrackright}{\kern0pt}{\isachardot}{\kern0pt}\ {\isasymexists}y{\isacharbrackleft}{\kern0pt}N{\isacharbrackright}{\kern0pt}{\isachardot}{\kern0pt}\ {\isasymforall}x{\isacharbrackleft}{\kern0pt}N{\isacharbrackright}{\kern0pt}{\isachardot}{\kern0pt}\ x\ {\isasymin}\ y\ {\isasymlongleftrightarrow}\ x\ {\isasymin}\ z\ {\isasymand}\ Q{\isacharparenleft}{\kern0pt}x{\isacharparenright}{\kern0pt}%
%% \end{isabelle}%
%% \begin{isabelle}%
%% strong{\isacharunderscore}{\kern0pt}replacement{\isacharparenleft}{\kern0pt}N{\isacharcomma}{\kern0pt}\ R{\isacharparenright}{\kern0pt}\ {\isasymequiv}\isanewline
%% {\isasymforall}A{\isacharbrackleft}{\kern0pt}N{\isacharbrackright}{\kern0pt}{\isachardot}{\kern0pt}\isanewline
%% \isaindent{\ \ \ }univalent{\isacharparenleft}{\kern0pt}N{\isacharcomma}{\kern0pt}\ A{\isacharcomma}{\kern0pt}\ R{\isacharparenright}{\kern0pt}\ {\isasymlongrightarrow}\ {\isacharparenleft}{\kern0pt}{\isasymexists}Y{\isacharbrackleft}{\kern0pt}N{\isacharbrackright}{\kern0pt}{\isachardot}{\kern0pt}\ {\isasymforall}b{\isacharbrackleft}{\kern0pt}N{\isacharbrackright}{\kern0pt}{\isachardot}{\kern0pt}\ b\ {\isasymin}\ Y\ {\isasymlongleftrightarrow}\isanewline
%% \ \ \ \ \ \ {\isacharparenleft}{\kern0pt}{\isasymexists}x{\isacharbrackleft}{\kern0pt}N{\isacharbrackright}{\kern0pt}{\isachardot}{\kern0pt}\ x\ {\isasymin}\ A\ {\isasymand}
%% R{\isacharparenleft}{\kern0pt}x{\isacharcomma}{\kern0pt}\ b{\isacharparenright}{\kern0pt}{\isacharparenright}{\kern0pt}{\isacharparenright}{\kern0pt}%
%% \end{isabelle}%
whose first argument $N$ is a class and their second arguments $Q$ and
$R$ are predicates of types $Q :: \tyi \fun \tyo$ and $R ::
\tyi \fun \tyi \fun \tyo$, respectively (the \isatt{univalent}
predicate states that $R$ is functional in its first variable; see Appendix~\ref{sec:def-main-relative}). The Separation Axiom appears in
\locale{M{\uscore}Z{\uscore}basic} as follows, where 
\isa{\isasymphi} is free and \isa{\isacharat} denotes
list concatenation:
\begin{isabelle}
separation{\isacharunderscore}{\kern0pt}ax{\isacharcolon}{\kern0pt}\ {\isachardoublequoteopen}{\isasymphi}\ {\isasymin}\ formula\ {\isasymLongrightarrow}\ env\ {\isasymin}\ list{\isacharparenleft}{\kern0pt}M{\isacharparenright}{\kern0pt}\ {\isasymLongrightarrow}\isanewline
\ \ \ \ \ \ \ \ \ \ \ \ \ \ \ \ \ \ \ \ arity{\isacharparenleft}{\kern0pt}{\isasymphi}{\isacharparenright}{\kern0pt}\ {\isasymle}\ {\isadigit{1}}\ {\isacharplus}{\kern0pt}\isactrlsub {\isasymomega}\ length{\isacharparenleft}{\kern0pt}env{\isacharparenright}{\kern0pt}\ {\isasymLongrightarrow}\isanewline
\ \ \ \ \ \ \ \ \ \ \ \ \ \ \ \ \ \ \ \ separation{\isacharparenleft}{\kern0pt}{\isacharhash}{\kern0pt}{\isacharhash}{\kern0pt}M{\isacharcomma}{\kern0pt}{\isasymlambda}x{\isachardot}{\kern0pt}\ {\isacharparenleft}{\kern0pt}M{\isacharcomma}{\kern0pt}\ {\isacharbrackleft}{\kern0pt}x{\isacharbrackright}{\kern0pt}\ {\isacharat}{\kern0pt}\ env\ {\isasymTurnstile}\ {\isasymphi}{\isacharparenright}{\kern0pt}{\isacharparenright}{\kern0pt}{\isachardoublequoteclose}
\end{isabelle}
%% \sout{we can only show
%% that this form of the axioms hold in generic extensions.}
Note that the predicate $Q$ mentioned above corresponds to the
satisfaction of \isa{\isasymphi}.

In contrast to Separation, we stated each instance of the Replacement
Axiom separately by means of the
$\isa{replacement{\uscore}assm}(M,\mathit{env},\phi)$ predicate:
\begin{isabelle}
{\isasymphi}\ {\isasymin}\ formula\ {\isasymlongrightarrow}\ env\ {\isasymin}\ list{\isacharparenleft}{\kern0pt}M{\isacharparenright}{\kern0pt}\ {\isasymlongrightarrow}\isanewline
\ \ arity{\isacharparenleft}{\kern0pt}{\isasymphi}{\isacharparenright}{\kern0pt}\ {\isasymle}\ {\isadigit{2}}\ {\isacharplus}{\kern0pt}\isactrlsub {\isasymomega}\ length{\isacharparenleft}{\kern0pt}env{\isacharparenright}{\kern0pt}\ {\isasymlongrightarrow}\isanewline
\ \ \ \ strong{\isacharunderscore}{\kern0pt}replacement{\isacharparenleft}{\kern0pt}{\isacharhash}{\kern0pt}{\isacharhash}{\kern0pt}M{\isacharcomma}{\kern0pt}{\isasymlambda}x\ y{\isachardot}{\kern0pt}\ {\isacharparenleft}{\kern0pt}M\ {\isacharcomma}{\kern0pt}\ {\isacharbrackleft}{\kern0pt}x{\isacharcomma}{\kern0pt}y{\isacharbrackright}{\kern0pt}{\isacharat}{\kern0pt}env\ {\isasymTurnstile}\ {\isasymphi}{\isacharparenright}{\kern0pt}{\isacharparenright}{\kern0pt}{\isachardoublequoteclose},
\end{isabelle}
%% % Same as above
%% \begin{multline}\label{eq:replacement_assm_def}
%% \varphi \in \formula  \limp \mathit{env} \in \isa{list}(M) \limp \isa{arity}(\varphi) \leq 2+ \isa{length}(\mathit{env}) \limp \\
%%  \isa{strong{\uscore}replacement}(\isa{\#\#} M, \lambda x\, y.\ (M, [x,y]
%% \mathbin{@} \mathit{env}  \models \varphi))
%% \end{multline}
which encodes the instance of replacement for the model $M$
corresponding to the predicate $R(x,y)$ given by
$\phi^M(x,y,x_0,\dots,x_n)$, where $\isatt{env} = [x_0,\dots,x_n]$.

In turn, the \isa{{\isasymcdot}Replacement{\isasymcdot}} function
takes a formula code and returns the corresponding replacement
instance:
\begin{isabelle}
{\isasymphi}\ {\isasymin}\ formula\ {\isasymLongrightarrow}\isanewline
{\isacharparenleft}{\kern0pt}M{\isacharcomma}{\kern0pt}\ {\isacharbrackleft}{\kern0pt}{\isacharbrackright}{\kern0pt}\ {\isasymTurnstile}\ {\isasymcdot}Replacement{\isacharparenleft}{\kern0pt}{\isasymphi}{\isacharparenright}{\kern0pt}{\isasymcdot}{\isacharparenright}{\kern0pt}\ {\isasymlongleftrightarrow}\ {\isacharparenleft}{\kern0pt}{\isasymforall}env{\isachardot}{\kern0pt}\ replacement{\isacharunderscore}{\kern0pt}assm{\isacharparenleft}{\kern0pt}M{\isacharcomma}{\kern0pt}\ env{\isacharcomma}{\kern0pt}\ {\isasymphi}{\isacharparenright}{\kern0pt}{\isacharparenright}
\end{isabelle}
%%%%%%%
%% \sout{In \session{ZF-Constructible} some results are proved assuming some instances of
%% separation or replacement (mejor si ponemos ejemplo concreto). In contrast,
%% our locale \isa{M{\uscore}Z{\uscore}basic} assumes the full scheme of separation in terms
%% of satisfaction of some internalized formula but no instance of replacement. (Mostrar el enunciado de separation\_ax:
%% es distinto a Reemplazo y se nombra después)
%% 
%% In order to have a finer control of the instances of replacement needed,
%% we postulate further locales with the instances needed to interpret at
%% the class\dots
%% (tal vez no sea necesario mostrar la definición de .Replacement.)}
%%%%%%%

Starting from \locale{M{\uscore}Z{\uscore}basic}, stronger locales are
defined by assuming more replacement instances.
These assumptions are then invoked to interpret at the class
$\isa{\#\#} M$ the relevant locales appearing in
\session{ZF-Constructible}, and further ones required for the relative
results from Section~\ref{sec:extension-isabellezf}. See
Section~\ref{sec:repl-instances} for details.

%-%-%-%-%-%-%-%-%-%-%-%-%-%-%-%-%-%-%-%-%-%-%-%-%-%-%-%-%-%-%-%-%

\subsection{The fundamental theorems}\label{sec:fundamental-theorems}
At this point, we work inside the locale \isa{M{\uscore}ctm1} that
assumes $M$ to be
countable and transitive, and satisfies some fragment of
$\ZFC$%
\footnote{%
Namely, Zermelo set theory plus the 7 replacement instances included in
the locales \locale{M{\uscore}ZF1} and \locale{M{\uscore}ZF{\uscore}ground}.
}.
This is further extended by assuming a forcing notion
$\lb \PP, {\preceq} ,\1\rb \in M$. The actual implementation reads:

\begin{isabelle}
\isacommand{locale}\isamarkupfalse%
\ forcing{\isacharunderscore}{\kern0pt}notion\ {\isacharequal}{\kern0pt}\isanewline
\ \ \isakeyword{fixes}\ P\ {\isacharparenleft}{\kern0pt}{\isacartoucheopen}{\isasymbbbP}{\isacartoucheclose}{\isacharparenright}{\kern0pt}\ \isakeyword{and}\ leq\ \isakeyword{and}\ one\ {\isacharparenleft}{\kern0pt}{\isacartoucheopen}{\isasymone}{\isacartoucheclose}{\isacharparenright}{\kern0pt}\isanewline
\ \ \isakeyword{assumes}\ one{\isacharunderscore}{\kern0pt}in{\isacharunderscore}{\kern0pt}P{\isacharcolon}{\kern0pt}\ \ \ \ \ \ \ {\isachardoublequoteopen}{\isasymone}\ {\isasymin}\ {\isasymbbbP}{\isachardoublequoteclose}\isanewline
\ \ \ \ \isakeyword{and}\ leq{\isacharunderscore}{\kern0pt}preord{\isacharcolon}{\kern0pt}\ \ \ \ \ \ \ {\isachardoublequoteopen}preorder{\isacharunderscore}{\kern0pt}on{\isacharparenleft}{\kern0pt}{\isasymbbbP}{\isacharcomma}{\kern0pt}leq{\isacharparenright}{\kern0pt}{\isachardoublequoteclose}\isanewline
\ \ \ \ \isakeyword{and}\ one{\isacharunderscore}{\kern0pt}max{\isacharcolon}{\kern0pt}\ \ \ \ \ \ \ \ \ \ {\isachardoublequoteopen}{\isasymforall}p{\isasymin}{\isasymbbbP}{\isachardot}{\kern0pt}\ {\isasymlangle}p{\isacharcomma}{\kern0pt}{\isasymone}{\isasymrangle}{\isasymin}leq{\isachardoublequoteclose}
\end{isabelle}

\begin{isabelle}
\isacommand{locale}\isamarkupfalse%
\ forcing{\isacharunderscore}{\kern0pt}data{\isadigit{1}}\ {\isacharequal}{\kern0pt}\ forcing{\isacharunderscore}{\kern0pt}notion\ {\isacharplus}{\kern0pt}\ M{\isacharunderscore}{\kern0pt}ctm{\isadigit{1}}\ {\isacharplus}{\kern0pt}\isanewline
\ \ \isakeyword{assumes}\ P{\isacharunderscore}{\kern0pt}in{\isacharunderscore}{\kern0pt}M{\isacharcolon}{\kern0pt}\ \ \ \ \ \ \ \ \ \ \ {\isachardoublequoteopen}{\isasymbbbP}\ {\isasymin}\ M{\isachardoublequoteclose}\isanewline
\ \ \ \ \isakeyword{and}\ leq{\isacharunderscore}{\kern0pt}in{\isacharunderscore}{\kern0pt}M{\isacharcolon}{\kern0pt}\ \ \ \ \ \ \ \ \ {\isachardoublequoteopen}leq\ {\isasymin}\ M{\isachardoublequoteclose}
\end{isabelle}
The version of the Forcing Theorems that we formalized follows the
considerations on the $\forces^*$ relation as discussed in Kunen's new
\emph{Set Theory}
\cite[p.~257ff]{kunen2011set}.
We defined forcing for atomic formulas by recursion on names in an
analogous fashion. But, in contrast to the point made on
p.~260 of this book, the structural recursion used to define the forcing
relation was replaced by one  involving codes for formulas. Thus, the metatheoretic formula
transformer $\phi\mapsto\mathit{Forces}_\phi$ was replaced by a
set-theoretic class function $\forceisa:: \tyi \fun \tyi$, which was defined by using
Isabelle/ZF facilities for primitive recursion.

Next, we state this version of the fundamental theorems in a compact
way. For any $G\subseteq \PP$, our notation for the extension of $M$ by
$G$ is the  customary one: $M[G]\defi
\{ \val(G,\tau) : \tau\in M \}$, where the interpretation
$\val(G,\tau)$ is defined by well-founded recursion on
$\tau$.

\begin{theorem}\label{th:forcing-thms}
  For every
  $\phi\in\formula$ with $\isa{arity}(\phi)\leq n$ and $\tau_1,\dots,\tau_n\in M$,
  \begin{enumerate}
  \item\label{item:definability} (Definability)
    $\forceisa(\phi)\in\formula$, 
  \end{enumerate}
  where the 
  arity of $\forceisa(\phi)$ is at most $\isa{arity}(\phi) + 4$; and if
  “$p \forces \phi\ [\tau_1,\dots,\tau_n]$”
  denotes
  “$M, [p,\PP,\preceq,\1, \tau_1,\dots,\tau_n]  \models
  \forceisa(\phi)$”, we have:
  \begin{enumerate}
    \setcounter{enumi}{1}
  \item\label{item:truth-lemma} (Truth Lemma) for every $M$-generic $G$,
    \[
      \exists p\in G.\ \; p \forces \phi\ [\tau_1,\dots,\tau_n]
    \]
    is equivalent to 
    \[
      M[G], [\val(G,\tau_1),\dots,\val(G,\tau_n)]
      \models \phi.
    \]
  \item \label{item:density-lemma} (Density Lemma) $p \forces \phi\ [\tau_1,\dots,\tau_n]$
    if and only if 
    $\{q\in \PP :  q \forces \phi\ [\tau_1,\dots,\tau_n]\}$
    is dense below $p$.
  \end{enumerate}
\end{theorem}
The items in Theorem~\ref{th:forcing-thms} appear in our
\session{Independence\_CH} session \cite{Independence_CH-AFP} as three
separate lemmas (located in the theory
\theory{Forcing{\uscore}Theorems}).
For instance, the Truth Lemma is stated as
follows:
\begin{isabelle}
\isacommand{lemma}\isamarkupfalse%
\ truth{\isacharunderscore}{\kern0pt}lemma{\isacharcolon}{\kern0pt}\isanewline
\ \ \isakeyword{assumes}\isanewline
\ \ \ \ {\isachardoublequoteopen}{\isasymphi}{\isasymin}formula{\isachardoublequoteclose}\isanewline
\ \ \ \ {\isachardoublequoteopen}env{\isasymin}list{\isacharparenleft}{\kern0pt}M{\isacharparenright}{\kern0pt}{\isachardoublequoteclose}\ {\isachardoublequoteopen}arity{\isacharparenleft}{\kern0pt}{\isasymphi}{\isacharparenright}{\kern0pt}{\isasymle}length{\isacharparenleft}{\kern0pt}env{\isacharparenright}{\kern0pt}{\isachardoublequoteclose}\isanewline
\ \ \isakeyword{shows}\isanewline
\ \ \ \ {\isachardoublequoteopen}{\isacharparenleft}{\kern0pt}{\isasymexists}p{\isasymin}G{\isachardot}{\kern0pt}\ p\ {\isasymtturnstile}\ {\isasymphi}\ env{\isacharparenright}{\kern0pt}\ \ \ {\isasymlongleftrightarrow}\ \ \ M{\isacharbrackleft}{\kern0pt}G{\isacharbrackright}{\kern0pt}{\isacharcomma}{\kern0pt}\ map{\isacharparenleft}{\kern0pt}val{\isacharparenleft}{\kern0pt}G{\isacharparenright}{\kern0pt}{\isacharcomma}{\kern0pt}env{\isacharparenright}{\kern0pt}\ {\isasymTurnstile}\ {\isasymphi}{\isachardoublequoteclose}
\end{isabelle}
%% where the $\isa{map}$ function that applies a function to
%% each element of a list and
where the $\forces$ notation (and its precedence) had already been set up in the
\theory{Forces{\uscore}Definition} theory.

Kunen first describes forcing for atomic formulas using a mutual
recursion
%% \begin{multline*}
%%   \forceseq (p,t_1,t_2) \defi 
%%   \forall s\in\dom(t_1)\cup\dom(t_2).\ \forall q\pleq p .\\
%%   \forcesmem(q,s,t_1)\lsii \forcesmem(q,s,t_2)
%% \end{multline*}
%% \begin{multline*}
%%   \forcesmem(p,t_1,t_2) \defi  \forall v\pleq p. \ \exists q\pleq v.\\  
%%   \exists s.\ \exists r\in \PP .\ \lb s,r\rb \in  t_2 \land q
%%   \pleq r \land \forceseq(q,t_1,s)
%% \end{multline*}
but then \cite[p.~257]{kunen2011set} it is cast as a single
recursively defined function $F$ over a well-founded  relation $R$.
In our formalization, these are called $\frcat$ and 
$\isa{frecR}$, respectively, and are defined on tuples $\lb \mathit{ft},t_1,t_2,p\rb$ (where
$\mathit{ft}\in\{0,1\}$ indicates whether the atomic formula being
forced is an equality or a membership, respectively).
Forcing for general formulas is then defined by recursion on the
datatype $\formula$ as indicated above. Technical details on the
implementation and proofs of the
Forcing Theorems have been spelled out in our
\cite{2020arXiv200109715G}.

%%% Local Variables: 
%%% mode: latex
%%% TeX-master: "independence_ch_isabelle"
%%% ispell-local-dictionary: "american"
%%% End: 
