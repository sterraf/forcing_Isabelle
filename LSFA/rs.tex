\subsection{The Rasiowa-Sikorski lemma}
In order to formalize of the proof of the Rasiowa-Sikorski lemma in
general, a version of the Axiom of Choice ($\AC$) must be proved. This
version is known as the \emph{Principle of Dependent choices}:
\begin{quote}
  ($\DC$) Let $\rr$ be a binary relation on $A$, and $a\in A$. If
  $\forall x\in A\,  \exists y\in A\, x\rr y$, then there exists
  $f:\om\to A$ such that $f(0)=a$ and $f(n)\rr f(n+1)$ for all
  $n\in\om$.
\end{quote}

A different version of $\DC$ (without the constraint $f(0)=a$, but
allowing a parameter that allows to prove an equivalence with $\AC$) was
formalized as part of the Isabelle/ZF library, along with many
statements that follow from or are equivalent to $\AC$. Since the
proofs there required a greater generality, it would have been very
contrived to derived this ``pointed'' version of $\DC$ from that
development. Instead, we preferred to give a direct proof based in the
one that appears in Moschovakis \cite{moschovakis1994notes}. Another
advantage is that this proof depends on the following version of $\AC$
(existence of choice functions on $\P(X)\sm \{\mty\}$):
\[
\exists s: \P(A)\sm \{\mty\}\to A \, \forall X\sbq A.\ X\neq \mty \limp
s(X) \in X.
\]
This statement is more standard than the one deemed ``$\AC$'' in the
Isabelle/ZF development, that involves a silent use of Replacement
since it is written as an axiom scheme. 

The strategy is to define $f$ above by primitive recursion. 

\begin{isabellebody}\isamarkuptrue%
\isanewline
\isacommand{consts}\isamarkupfalse%
\ dc{\isacharunderscore}witness\ {\isacharcolon}{\isacharcolon}\ {\isachardoublequoteopen}i\ {\isasymRightarrow}\ i\ {\isasymRightarrow}\ i\ {\isasymRightarrow}\ i\ {\isasymRightarrow}\ i\ {\isasymRightarrow}\ i{\isachardoublequoteclose}\isanewline
\isacommand{primrec}\isamarkupfalse%
\isanewline
\ \ wit{\isadigit{0}}\ \ \ {\isacharcolon}\ {\isachardoublequoteopen}dc{\isacharunderscore}witness{\isacharparenleft}{\isadigit{0}}{\isacharcomma}A{\isacharcomma}a{\isacharcomma}s{\isacharcomma}R{\isacharparenright}\ {\isacharequal}\ a{\isachardoublequoteclose}\isanewline
\ \ witrec\ {\isacharcolon}{\isachardoublequoteopen}dc{\isacharunderscore}witness{\isacharparenleft}succ{\isacharparenleft}n{\isacharparenright}{\isacharcomma}A{\isacharcomma}a{\isacharcomma}s{\isacharcomma}R{\isacharparenright}\ {\isacharequal}\ s{\isacharbackquote}{\isacharbraceleft}x{\isasymin}A{\isachardot}\ {\isasymlangle}dc{\isacharunderscore}witness{\isacharparenleft}n{\isacharcomma}A{\isacharcomma}a{\isacharcomma}s{\isacharcomma}R{\isacharparenright}{\isacharcomma}x{\isasymrangle}{\isasymin}R\ {\isacharbraceright}{\isachardoublequoteclose}\isanewline
\end{isabellebody}

The function \verb|dc_witness| has, appart from $A$, $a$ and $R$, a
function $s$ as a parameter. If this function is a selector for
$\P(A)\sm \{\mty\}$, the function $f\defi$\verb|dc_witness|$(A,a,s,R)$
will satify $\DC$.

The proof is mostly routine; after a few lemmas we obtain the
following theorem:

\begin{isabellebody}
\isanewline
\isacommand{theorem}\isamarkupfalse%
\ pointed{\isacharunderscore}DC\ \ {\isacharcolon}\ {\isachardoublequoteopen}{\isacharparenleft}{\isasymforall}x{\isasymin}A{\isachardot}\ {\isasymexists}y{\isasymin}A{\isachardot}\ {\isasymlangle}x{\isacharcomma}y{\isasymrangle}{\isasymin}\ R{\isacharparenright}\ {\isasymLongrightarrow}\isanewline
\ \ \ \ \ \ \ \ \ \ \ \ \ \ \ \ \ \ \ \ \ \ \ {\isasymforall}a{\isasymin}A{\isachardot}\ {\isacharparenleft}{\isasymexists}f\ {\isasymin}\ nat{\isasymrightarrow}A{\isachardot}\ f{\isacharbackquote}{\isadigit{0}}\ {\isacharequal}\ a\ {\isasymand}\ {\isacharparenleft}{\isasymforall}n\ {\isasymin}\ nat{\isachardot}\ {\isasymlangle}f{\isacharbackquote}n{\isacharcomma}f{\isacharbackquote}succ{\isacharparenleft}n{\isacharparenright}{\isasymrangle}{\isasymin}R{\isacharparenright}{\isacharparenright}{\isachardoublequoteclose}\isanewline
\end{isabellebody}

We need a further, ``diagonal'' version of $\DC$  to prove
Rasiowa-Sikorski. That is, if the assumption holds for a sequence of
relations $S_n$,  then $f(n) \mathrel{S_{n+1}} f(n+1)$.

We first obtain a corollary of $\DC$ changing $A$ for
$A\times\mathtt{nat}$:

\begin{isabellebody}
\isanewline
\isacommand{lemma}\isamarkupfalse%
\ aux{\isacharunderscore}DC{\isacharunderscore}on{\isacharunderscore}AxNat{\isadigit{2}}\ {\isacharcolon}\ {\isachardoublequoteopen}{\isasymforall}x{\isasymin}A{\isasymtimes}nat{\isachardot}\ {\isasymexists}y{\isasymin}A{\isachardot}\ {\isasymlangle}x{\isacharcomma}{\isasymlangle}y{\isacharcomma}succ{\isacharparenleft}snd{\isacharparenleft}x{\isacharparenright}{\isacharparenright}{\isasymrangle}{\isasymrangle}\ {\isasymin}\ R\ {\isasymLongrightarrow}\isanewline
\ \ \ \ \ \ \ \ \ \ \ \ \ \ \ \ \ \ {\isasymforall}x{\isasymin}A{\isasymtimes}nat{\isachardot}\ {\isasymexists}y{\isasymin}A{\isasymtimes}nat{\isachardot}\ {\isasymlangle}x{\isacharcomma}y{\isasymrangle}\ {\isasymin}\ {\isacharbraceleft}{\isasymlangle}a{\isacharcomma}b{\isasymrangle}{\isasymin}R{\isachardot}\ snd{\isacharparenleft}b{\isacharparenright}\ {\isacharequal}\ succ{\isacharparenleft}snd{\isacharparenleft}a{\isacharparenright}{\isacharparenright}{\isacharbraceright}{\isachardoublequoteclose}\isanewline
\end{isabellebody}
%
The following lemma is then proved automatically:

\begin{isabellebody}
\isanewline
\isacommand{lemma}\isamarkupfalse%
\ aux{\isacharunderscore}sequence{\isacharunderscore}DC{\isadigit{2}}\ {\isacharcolon}\ {\isachardoublequoteopen}{\isasymforall}x{\isasymin}A{\isachardot}\ {\isasymforall}n{\isasymin}nat{\isachardot}\ {\isasymexists}y{\isasymin}A{\isachardot}\ {\isasymlangle}x{\isacharcomma}y{\isasymrangle}\ {\isasymin}\ S{\isacharbackquote}n\ {\isasymLongrightarrow}\isanewline
\ \ \ \ \ \ \ \ {\isasymforall}x{\isasymin}A{\isasymtimes}nat{\isachardot}\ {\isasymexists}y{\isasymin}A{\isachardot}\ {\isasymlangle}x{\isacharcomma}{\isasymlangle}y{\isacharcomma}succ{\isacharparenleft}snd{\isacharparenleft}x{\isacharparenright}{\isacharparenright}{\isasymrangle}{\isasymrangle}\ {\isasymin}\ {\isacharbraceleft}{\isasymlangle}{\isasymlangle}x{\isacharcomma}n{\isasymrangle}{\isacharcomma}{\isasymlangle}y{\isacharcomma}m{\isasymrangle}{\isasymrangle}{\isasymin}{\isacharparenleft}A{\isasymtimes}nat{\isacharparenright}{\isasymtimes}{\isacharparenleft}A{\isasymtimes}nat{\isacharparenright}{\isachardot}\ {\isasymlangle}x{\isacharcomma}y{\isasymrangle}{\isasymin}S{\isacharbackquote}m\ {\isacharbraceright}{\isachardoublequoteclose}\isanewline
\ \ %
\isacommand{by}\isamarkupfalse%
\ auto%
\isanewline
\end{isabellebody}
%
And finally, we arrive to $\DC$ for a sequence of relations.

\begin{isabellebody}
\isanewline
\isacommand{lemma}\isamarkupfalse%
\ sequence{\isacharunderscore}DC{\isacharcolon}\ {\isachardoublequoteopen}{\isasymforall}x{\isasymin}A{\isachardot}\ {\isasymforall}n{\isasymin}nat{\isachardot}\ {\isasymexists}y{\isasymin}A{\isachardot}\ {\isasymlangle}x{\isacharcomma}y{\isasymrangle}\ {\isasymin}\ S{\isacharbackquote}n\ {\isasymLongrightarrow}\isanewline
\ \ \ \ {\isasymforall}a{\isasymin}A{\isachardot}\ {\isacharparenleft}{\isasymexists}f\ {\isasymin}\ nat{\isasymrightarrow}A{\isachardot}\ f{\isacharbackquote}{\isadigit{0}}\ {\isacharequal}\ a\ {\isasymand}\ {\isacharparenleft}{\isasymforall}n\ {\isasymin}\ nat{\isachardot}\ {\isasymlangle}f{\isacharbackquote}n{\isacharcomma}f{\isacharbackquote}succ{\isacharparenleft}n{\isacharparenright}{\isasymrangle}{\isasymin}S{\isacharbackquote}succ{\isacharparenleft}n{\isacharparenright}{\isacharparenright}{\isacharparenright}{\isachardoublequoteclose}\isanewline
\end{isabellebody}


%%% Local Variables:
%%% mode: latex
%%% ispell-local-dictionary: "american"
%%% TeX-master: "first_steps_into_forcing"
%%% End:
