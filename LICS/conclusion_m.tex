%%%%%%%%%%%%%%%%%%%%%%%%%%%%%%%%%%%%%%%%%%%%%%%%%%%%%%%%%%%%%%%%%%%%%%          
\section{Conclusions and future work}
\label{sec:conclusions-future-work}

The ultimate goal of our project is a complete mechanization of
forcing allowing for further developments (formalization of the
relative consistency of $\CH$), with the long-term hope that 
working set-theorists will adopt these formal tools as an aid to their
research. In the current paper we reported a first major
milestone towards that goal; viz. a formal proof in Isabelle/ZF of the
satisfaction by generic extensions of most of the $\ZF$ axioms, except
for the axiom-scheme of Replacement.

We cannot overstate the importance of following the sharp and detailed 
presentation of forcing given by
\citet{kunen2011set}. In fact, it helped us to delineate the
\emph{thematic} aspects of our formalization; i.e.~the handling of all
the theoretical concepts and results in the subject and it informed
the structure of locales organizing our development.

% In a gross simplification, there are two aspects to a formalization
% project like this one: thematic and programmatic. The first concerns
% the handling of all the theoretical concepts and results in the
% subject, while the second involves the practical issues of the
% implementation and design. In the case of forcing, the main intricacy
% lies in the first aspect. In this sense, following a sensible
% presentation of the material is key.  The authoritative reference 
% on the subject during the last 30 years has been Kunen's classical
% \cite{kunen1980}, and  we have followed a recent rewrite
% \cite{kunen2011set} 
% of that  textbook, which presents the material in the same sharp 
% style but offering a lot of details. In some sense this project
% wouldn't exist without this book.


% Repite intro el p\'arrafo  que sigue
%% The proof of Separation depends on the Fundamental Theorems of
%% Forcing; in the spirit of the Formal Abstracts project we have only
%% stated those theorems in our current development, because from a
%% engineering point of view this can be done independently. The task of
%% proving the Fundamental Theorems is another landmark of our project,
%% after completing the formalization that generic extensions satisfy
%% the axiom scheme of Replacement.

% Next, various forcing notions must be developed in order to
% obtain extensions satisfying diverse properties (v.g., the Continuum
% Hypothesis). This yields relative consistency results.

Two axioms have not been treated in full. Infinity was proved under
two extra assumptions on the model; when we develop a full-fledged
interface between ctms of $\ZF$  and the locales 
providing recursive constructions from Paulson's
\isatt{ZF-Constructible} session, the same current proof will hold
with no extra assumptions. The same goes for the results $M\sbq M[G]$
and $G\in M[G]$. 

The Replacement Axiom, however, requires some more work to be
done. Both
in Kunen and in other presentations (e.g. \cite{neeman-course}) it
requires a relativized version (i.e., showing that it holds for $M$)
of the \emph{Reflection Principle}. In order to state this
meta-theoretic result by Montague, recall that an equivalent
formulation of the Foundation Axiom states that the universe of sets
can be decomposed in a transfinite hierarchy of sets:
\begin{theorem}
  Let $V_{\al}\defi\union\{\P(V_\be) : \be<\al\}$ for each ordinal
  $\al$. Then each $V_\al$ is a set and 
  $\forall x. \exists\al .\ \Ord(\al) \y x\in V_\al$.  
\end{theorem}
\begin{theorem}[Reflection Principle]\label{th:reflection-principle}
  For every finite $\Phi\sbq\ZF$, $\ZF$ proves: ``There exists
  unboundedly many $\al$ such that $V_\al\models \Phi$.''
\end{theorem}
%% This result has also other applications; in particular, it dispenses
%% the need of ctms of the whole of $\ZF$ for consistency proofs (by
%% repeating  the proof of Lemma~\ref{lem:wf-model-implies-ctm}, now
%% starting with a  $V_\al$ satisfying $\Phi$). 

% This statement is straightforwardly equivalent to that with $\Phi$
% consisting of a single formula. 
It is obvious that we can take the conjunction of $\Phi$ and state
Theorem~\ref{th:reflection-principle} for a single formula, say $\phi$.
The schematic nature of this result
hints at a proof by induction on formulas, and hence it must be shown
internally. 

This is an appropriate point to insist that the internal/external
dichotomy has been a powerful agent in the shaping of our project.
This tension was also pondered by Paulson in his formalization of
G\"odel's constructible universe \cite{paulson_2003}; after choosing a
shallow embedding of $\ZF$, every argument proved by induction on
formulas (or functions defined by recursion) should be done using
internalized formulas. Working on top of Paulson's library, we
prototyped the formula-transformer $\forceisa$, which is defined for
internalized formulas, and this affects indirectly the proof of the
Separation Axiom (despite the latter is not by induction). The proof
of Replacement also calls for internalized formulas, because one needs
a general version of the Reflection Principle (since the formula
$\phi$ involved depends on the specific instance of Replacement being
proved).

A secondary, more prosaic, outcome of this project is to precisely
assess which assumptions on the ground model $M$ are needed to develop
the forcing machinery. The obvious are transitivity and $M$ being
countable (but keep in mind Lemma~\ref{lem:wf-model-implies-ctm}); the
first because many absoluteness results follows from this, the latter
for the existence of generic filters. A
more anecdotal one is that to show that an instance of Separation
with at most two parameters holds in $M[G]$, one needs to assume a
particular  six-parameter
instance in $M$ (four extra parameters can be directly blamed on
$\forceisa$).  The purpose of identifying those assumptions is to assemble
in a locale the specific (instances of) axioms that should satisfy the
ground model in order to perform forcing constructions; this list will
likely include all the instances of Separation and Replacement that
are needed to satisfy the requirements of the locales in the
\isatt{ZF-Constructible} session.

We have already commented on our hacking of \isatt{ZF-Constructible}
to maximize its modularity and thus the re-usability in other
formalizations. We think it would be desirable to organize it
somewhat differently: a trivial change is to catalog in one file all
the internalized formulas. A more
conceptual modification would be to start out with an even more basic
locale that only assumes $M$ to be a non-empty transitive class, as
many absoluteness results follow from this hypothesis. Furthermore, as
Paulson comments in the 
sources, it would have been better to minimize the use of the Powerset
Axiom in locales and proofs. There are useful natural models that
satisfy a sub-theory of $\ZF$ not including Powerset, and to ensure a
broader applicability, it would be convenient to have absoluteness
results not assuming it. We plan to contribute back to the official
distribution of Isabelle/ZF with a thorough revision of the development of
constructibility.

%% \section*{Acknowledgment}
%% 


%%% Local Variables: 
%%% mode: latex
%%% TeX-master: "Separation_In_MG"
%%% ispell-local-dictionary: "american"
%%% End: 
