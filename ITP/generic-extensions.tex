%%%%%%%%%%%%%%%%%%%%%%%%%%%%%%%%%%%%%%%%%%%%%%%%%%%%%%%%%%%%%%%%%%%%%%          
\section{Generic extensions}
\label{sec:generic-extensions}
We will swiftly review some definitions in order to reach the concept
of \emph{generic extension}. As first preliminary definitions, a \emph{forcing
notion} $\lb\PP,\leq,\1\rb$ is simply a preorder with top, and a \emph{filter}
$G\sbq\PP$ is an increasing subset which is downwards
compatible. Given a ctm $M$ of $\ZF$, a forcing
notion in $M$, and a filter $G$, a new set $M[G]$ is defined. Each
element $a\in M[G]$ is 
determined by its \emph{name} $\dot a$ in $M$. Actually, the structure of
each $\dot a$ is used to construct $a$. They are related by a
map $\val$ that takes $G$ as a parameter:
\[
\val(G,\dot a) = a.
\] 
Then the extension is defined by the image of the map $\val(G,\cdot)$:
\[
M[G] \defi \{\val(G,\tau): \tau\in M\}.
\]
Metatheoretically, it is straightforward to see that $M[G]$ is a
transitive set that satisfies some axioms of $\ZF$ (see
Section~\ref{sec:easy-axioms}) and includes $M\cup\{G\}$. Nevertheless
there is no a priori reason for $M[G]$ to satisfy either Separation, Powerset
or Replacement. The original insight by Cohen was to define the notion
of \emph{genericity} for a filter $G\sbq\PP$ and to prove that
whenever $G$ is generic, $M[G]$ will satisfy $\ZF$. Remember that a
filter is generic if it intersects all the dense sets in $M$; in
\cite{2018arXiv180705174G} we formalized the Rasiowa-Sikorski lemma which
proves the existence of generic filters for ctms.

The Separation Axiom  is the first that requires the notion of
genericity and the use of the forcing machinery, which we review in
the Section~\ref{sec:forcing}.

%%% Local Variables: 
%%% mode: latex
%%% TeX-master: "Separation_In_MG"
%%% ispell-local-dictionary: "american"
%%% End: 
