\documentclass[11pt,a4paper]{article}
\usepackage{isabelle,isabellesym}
\usepackage[numbers]{natbib}
\usepackage{babel}

\usepackage{relsize}
\DeclareRobustCommand{\isactrlbsub}{\emph\bgroup\math{}\sb\bgroup\mbox\bgroup\isaspacing\itshape\smaller}
\DeclareRobustCommand{\isactrlesub}{\egroup\egroup\endmath\egroup}
\DeclareRobustCommand{\isactrlbsup}{\emph\bgroup\math{}\sp\bgroup\mbox\bgroup\isaspacing\itshape\smaller}
\DeclareRobustCommand{\isactrlesup}{\egroup\egroup\endmath\egroup}

% further packages required for unusual symbols (see also
% isabellesym.sty), use only when needed

\usepackage{amssymb}
  %for \<leadsto>, \<box>, \<diamond>, \<sqsupset>, \<mho>, \<Join>,
  %\<lhd>, \<lesssim>, \<greatersim>, \<lessapprox>, \<greaterapprox>,
  %\<triangleq>, \<yen>, \<lozenge>

%\usepackage{eurosym}
  %for \<euro>

%\usepackage[only,bigsqcap]{stmaryrd}
  %for \<Sqinter>

%\usepackage{eufrak}
  %for \<AA> ... \<ZZ>, \<aa> ... \<zz> (also included in amssymb)

%\usepackage{textcomp}
  %for \<onequarter>, \<onehalf>, \<threequarters>, \<degree>, \<cent>,
  %\<currency>

% this should be the last package used
\usepackage{pdfsetup}

% urls in roman style, theory text in math-similar italics
\urlstyle{rm}
\isabellestyle{it}

% for uniform font size
%\renewcommand{\isastyle}{\isastyleminor}
\newcommand{\forces}{\Vdash}
\newcommand{\dom}{\mathsf{dom}}
\renewcommand{\isacharunderscorekeyword}{\mbox{\_}}
\renewcommand{\isacharunderscore}{\mbox{\_}}
\renewcommand{\isasymtturnstile}{\isamath{\Vdash}}
\renewcommand{\isacharminus}{-}
\newcommand{\axiomas}[1]{\mathit{#1}}
\newcommand{\ZFC}{\axiomas{ZFC}}
\newcommand{\ZF}{\axiomas{ZF}}
\newcommand{\AC}{\axiomas{AC}}
\newcommand{\CH}{\axiomas{CH}}

\begin{document}

\title{Transitive Models of Fragments of ZF}
\author{Emmanuel Gunther\thanks{Universidad Nacional de C\'ordoba. 
    Facultad de Matem\'atica, Astronom\'{\i}a,  F\'{\i}sica y
    Computaci\'on.}
  \and
  Miguel Pagano\footnotemark[1]
  \and
  Pedro S\'anchez Terraf\footnotemark[1] \thanks{Centro de Investigaci\'on y Estudios de Matem\'atica
    (CIEM-FaMAF), Conicet. C\'ordoba. Argentina.
    Supported by Secyt-UNC project 33620180100465CB.}
  \and
  Mat\'{\i}as Steinberg\footnotemark[1]
}
\maketitle

\begin{abstract}
  We extend the ZF-Constructibility session, by relativizing several
  theories of Isabelle/ZF to a transitive class. This work is a
  prerrequisite to our formalization of the independence of the
  Continuum Hypothesis.
\end{abstract}


\tableofcontents

% sane default for proof documents
\parindent 0pt\parskip 0.5ex

\section{Introduction}

% generated text of all theories
\input{session}


\end{document}

%%% Local Variables:
%%% mode: latex
%%% TeX-master: t
%%% End:
