\section{Introduction}
\label{sec:introduction}

\subsection{Formalized mathematics}
There is a growing trend in the use of computers to assist in the
\emph{formal verification} of mathematics. Note, for instance, the forthcoming
\href{https://web.archive.org/web/20211220044218/https://icm2022.org/plenary-lectures#Kevin-Buzzard}{ICM2022
  plenary lecture} by Kevin Buzzard.

Several venues devoted to the communication of formalized mathematics;
today, mainly CS conferences: ITP, IJCAR, CPP, CICM, and others.

Many proof assistants, each with a different flavour and underlying
foundations: Naproche (Koepke, mainly FOL + ATP), Mizar,
Isabelle/HOL, Lean, \dots, Coq, Agda,\dots.

These assistants provide several dialects, among which we single out:
\begin{enumerate}
\item Procedural: Useful for exploration/research.
\item Declarative: Only one that can be read by humans!
\end{enumerate}

Before we start an in-depth discussion, a point should be made clear:
A formalized proof is not the same as an \emph{automatic proof}. The
reader surely understands that, aside from trivial results, no current
technology allows us to state a reasonable (and correct) theorem
statement and expect to obtain a proof after hitting ``Enter'', at
least not after a humanly feasible wait. On the other hand, it is
quite possible that the same reader has some mental image that
formalizing a proof requires making each application of Modus Ponens
explicit.

The fact is that proof assistants are designed for the human prover to
be able to decompose a statement to be proved into smaller subgoals
which can actually be fed into some automatic tool. The balance between
what these tools are able to handle is not  easily appreciated by
intuition: Sometimes, ``trivial'' steps are not solved by them, which
can result in obvious frustration; but they would quickly solve some
goals that do not look like a ``mere computation.''

The same happens to the extent of mathematics formalizable (or even,
which has \emph{already} been formalized): The community seems not to
be fully aware of the potential of the technology. A recent discussion
at the FOM mailing list shows this state of affairs within our fellow
logicians. It is thus convenient to recall
some major formalization projects, such as the Four Color Theorem
\cite{MR2463991}, the Odd Order Theorem
\cite{10.1007/978-3-642-39634-2_14}, and the proof the Kepler's
Conjecture \cite{MR3659768}. There is a vast mathematical corpus at
the Archive of Formal Proofs based on Isabelle, and formalizations of
brand new and complex objects like the definition of perfectoid spaces \cite{10.1145/3372885.3373830}
and the Liquid Tensor Experiment \cite{LTE2020,LTE2021} on Lean.

\subsection{Our achievements}
We formalized a model-theoretic rendition of forcing (Section~\ref{sec:forcing}), showing how to
construct proper extension of ctms of $\ZF$ (respectively, with
$\AC$), and we formalized the basic forcing notions required to obtain
ctms of $\ZFC + \neg\CH$ and of $\ZFC + \CH$. No meta-theoretic issues
(consistency, FOL calculi, etc) were formalized, so we were merely
concerned with the mathematics of forcing. Nevertheless, by inspecting
the foundations underlying our proof assistant Isabelle
(Section~\ref{sec:isabelle-metalogic-meta}) it can be stated that our
formalization is closest to a proof in $\ZF$ of the previous
implications.

In order to reach our goals, we provided missing results, from basic ones
involving cardinal successors, countable sets, etc
(Section~\ref{sec:extension-isabellezf}); we extended the treatment of relativization of
set-theoretical concepts (Section~\ref{sec:tools-relativization}).
%% We redesigned Isabelle/ZF results on non-absolute concepts to work
%% relative to a class.

One added value that is obtained from the present formalization is
that we identified a handfull of instances of Replacement which are
sufficient to set the forcing machinery up (Section~\ref{sec:repl-instances}).
%% (Separation is a bit more complicated).
%% A verification that no choice  is needed at the level of the
%%  object logic.
The eagerness to obtain this level of detail might be a consequence of
“an unnatural tendency to investigate, for the most part, trivial
minutiae of the formalism,” as it was put by Cohen
\cite{zbMATH02012060}, but we were also drove by curiosity.

%%% Local Variables: 
%%% mode: latex
%%% TeX-master: "independence_ch_isabelle"
%%% ispell-local-dictionary: "american"
%%% End: 
