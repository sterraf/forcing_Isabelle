\section{Conclusions and future work}
There are several technical milestones that have to be reached in the
course of a formalization of the theory of forcing. The first one, and most
obvious, is the bulk of set- and meta-theoretical concepts needed to work
with. This pushed us, in a sense,  into building on top of Isabelle/ZF,
since we know of no other development in set theory of such
depth (and breadth). In this paper we worked on setting the stage for the work with
generic extensions; in particular, this involves some purely mathematical
results, as the Rasiowa-Sikorski lemma. 

Other milestones in this formalization project
involve 
\begin{enumerate}
\item the definition
  of the forcing relation, 
\item proving the Fundamental Theorem of forcing
  (that relates truth in $M$ to that in $M[G]$), and 
\item using it to show
  that $M[G]\models \ZFC$. 
\end{enumerate}
The theory is very modular and this is
witnessed by the fact 
that the last goal does not depend on the proof of the Fundamental
Theorem nor on the definition of the forcing relation. Our next task
will be to obtain the last goal in that enumeration. 

To this end, we will develop an interface between Paulson's
relativization results and countable models of $\ZFC$. This will show
that every ctm $M$ is closed under well-founded recursion and, in
particular, that contains names for each of its
elements. Consequently, the proof of  $M\sbq M[G]$ will be
complete. A landmark will be to prove the Axiom Scheme
of Separation (the first that needs to use the machinery of forcing
nontrivially). As a part of the new formalization, we will provide
Isar versions of the longer applicative proofs presented in this work.

%%% Local Variables:
%%% mode: latex
%%% ispell-local-dictionary: "american"
%%% TeX-master: "first_steps_into_forcing"
%%% End:
