%%%%%%%%%%%%%%%%%%%%%%%%%%%%%%%%%%%%%%%%%%%%%%%%%%%%%%%%%%%%%%%%%%%%%%          
\section{Forcing}
\label{sec:forcing}
Let $\lb \PP, {\preceq} ,\1\rb \in M$ be a forcing notion. Given $G\sbq \PP$, we have
$M[G]\defi \{ \val(\PP,G,\punto{a}) : \punto{a}\in M \}$.

The following form of the Forcing Theorems  is the one
that we formalized.
\begin{theorem}
  There exists a function  $\forceisa:: \tyi \fun  \tyi$
  such that for every
  $\phi\in\formula$ and $\punto{a}_0,\dots,\punto{a}_n\in M$,
  \begin{enumerate}
  \item (Definability) $\forceisa(\phi)\in\formula$;
  \item (Truth Lemma) for every $M$-generic $G$,
    \[
      M[G], [\val(\PP,G,\punto{a}_0),\dots,\val(\PP,G,\punto{a}_n)]
      \models \phi
    \]
    is equivalent to 
    \[
      \exists p\in G.\ \; M, [p,\PP,\preceq,\1, \punto{a}_0,\dots,\punto{a}_n]  \models
      \forceisa(\phi).\]

    We use the notation $p \forces
    \phi\ [\punto{a}_0,\dots,\punto{a}_n]$ for this last assertion.
  \item (Density Lemma) $p \forces \phi\ [\punto{a}_0,\dots,\punto{a}_n]$
    if and only if 
    $\{q\in \PP :  q \forces \phi\ [\punto{a}_0,\dots,\punto{a}_n]\}$
    is dense below $p$.
  \end{enumerate}
\end{theorem}

We followed the new Kunen's book to define
$\forceisa$.  Forcing for atomic formulas is described as a mutual
recursion
%% \begin{multline*}
%%   \forceseq (p,t_1,t_2) \defi 
%%   \forall s\in\dom(t_1)\cup\dom(t_2).\ \forall q\pleq p .\\
%%   \forcesmem(q,s,t_1)\lsii \forcesmem(q,s,t_2)
%% \end{multline*}
%% \begin{multline*}
%%   \forcesmem(p,t_1,t_2) \defi  \forall v\pleq p. \ \exists q\pleq v.\\  
%%   \exists s.\ \exists r\in \PP .\ \lb s,r\rb \in  t_2 \land q
%%   \pleq r \land \forceseq(q,t_1,s)
%% \end{multline*}
but then \cite[p.~257]{kunen2011set} it is cast as a single
recursively defined function $\frcat$ over the wellfounded relation
$\isatt{frecR}$ on tuples $\lb \mathit{ft},t_1,t_2,p\rb$ (where
$\mathit{ft}\in\{0,1\}$ indicates the type of the atomic formula being
forced). Forcing for general formulas is defined by recursion on the
datatype $\formula$. Details on the implementation and proofs of the
Forcing Theorems have been spelled out in our
\cite{2020arXiv200109715G}.


It is to be noted that application of the Forcing theorems do not
require any extra Replacement instances on $M$.

%%% Local Variables: 
%%% mode: latex
%%% TeX-master: "independence_ch_isabelle"
%%% ispell-local-dictionary: "american"
%%% End: 
