\section{Example of proper extension}
\label{sec:example-proper-extension}

Even when the axioms of $\ZFC$ are proved in the generic extension,
one can't claim that the magic of forcing has taken place unless one
is able to provide some \emph{proper} extension with the \emph{same
ordinals}. After all, one is assuming from starters a model $M$ of $\ZFC$,
and in some trivial cases $M[G]$ might end up to be exactly $M$; this
is where \emph{proper} enters the stage. But, for instance, in the
presence of large cardinals, a model $M'\supsetneq M$ might be an
end-extension of $M$ ---this is were we ask the two models to have the
same ordinals, the same \emph{height}. 

Two short theory files contain the relevant
results. \verb|Ordinals_In_MG.thy| shows, using the closure of $M$
under ranks, that $M$ and $M[G]$ share the same ordinals (actually,
ranks of elements of $M[G]$ are bounded by the ranks of their names in
$M$):
\begin{isabelle}
\isacommand{lemma}\isamarkupfalse%
\ rank{\isacharunderscore}val{\isacharcolon}\ {\isachardoublequoteopen}rank{\isacharparenleft}val{\isacharparenleft}G{\isacharcomma}x{\isacharparenright}{\isacharparenright}\ {\isasymle}\ rank{\isacharparenleft}x{\isacharparenright}{\isachardoublequoteclose}\isanewline
\isacommand{lemma}\isamarkupfalse%
\ Ord{\isacharunderscore}MG{\isacharunderscore}iff{\isacharcolon}\isanewline
\ \ \isakeyword{assumes}\ {\isachardoublequoteopen}Ord{\isacharparenleft}{\isasymalpha}{\isacharparenright}{\isachardoublequoteclose}\ \isanewline
\ \ \isakeyword{shows}\ {\isachardoublequoteopen}{\isasymalpha}\ {\isasymin}\ M\ {\isasymlongleftrightarrow}\ {\isasymalpha}\ {\isasymin}\ M{\isacharbrackleft}G{\isacharbrackright}{\isachardoublequoteclose}
\end{isabelle}

To prove these results, we found it useful to formalize induction over
the relation \isatt{ed}$(x,y) \defi x\in\dom(y)$, which is key
to arguments involving names.
\begin{isabelle}
\isacommand{lemma}\isamarkupfalse%
\ ed{\isacharunderscore}induction{\isacharcolon}\isanewline
\ \ \isakeyword{assumes}\ {\isachardoublequoteopen}{\isasymAnd}x{\isachardot}\ {\isasymlbrakk}{\isasymAnd}y{\isachardot}\ \ ed{\isacharparenleft}y{\isacharcomma}x{\isacharparenright}\ {\isasymLongrightarrow}\ Q{\isacharparenleft}y{\isacharparenright}\ {\isasymrbrakk}\ {\isasymLongrightarrow}\ Q{\isacharparenleft}x{\isacharparenright}{\isachardoublequoteclose}\isanewline
\ \ \isakeyword{shows}\ {\isachardoublequoteopen}Q{\isacharparenleft}a{\isacharparenright}{\isachardoublequoteclose}
\end{isabelle}

\verb|Succession_Poset.thy| contains our first example of a poset
that interprets the locale
\isatt{forcing{\isacharunderscore}notion}, essentially the notion for
adding one Cohen real. It is the set $2^{<\om}$ of all finite binary
sequences partially  ordered by reverse inclusion.
The sufficient condition for a proper extension is that
the forcing poset is \emph{separative}: every element has two
incompatible (\isatt{{\isasymbottom}s}) extensions. Here,
\isatt{seq{\isacharunderscore}upd{\isacharparenleft}f{\isacharcomma}x{\isacharparenright}}
adds \isatt{x} to the end of the sequence \isatt{f}.

\begin{isabelle}
\isacommand{lemma}\isamarkupfalse%
\ seqspace{\isacharunderscore}separative{\isacharcolon}\isanewline
\ \ \isakeyword{assumes}\ {\isachardoublequoteopen}f{\isasymin}{\isadigit{2}}{\isacharcircum}{\isacharless}{\isasymomega}{\isachardoublequoteclose}\isanewline
\ \ \isakeyword{shows}\ {\isachardoublequoteopen}seq{\isacharunderscore}upd{\isacharparenleft}f{\isacharcomma}{\isadigit{0}}{\isacharparenright}\ {\isasymbottom}s\ seq{\isacharunderscore}upd{\isacharparenleft}f{\isacharcomma}{\isadigit{1}}{\isacharparenright}{\isachardoublequoteclose}
\end{isabelle}
 
We prove in the theory file \verb|Proper_Extension.thy| that, in
general, every separative forcing notion gives rise to a proper
extension.

%%% Local Variables: 
%%% mode: latex
%%% TeX-master: "forcing_in_isabelle_zf"
%%% ispell-local-dictionary: "american"
%%% End: 
