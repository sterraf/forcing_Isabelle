\section{Future directions}
\label{sec:conclusion}

There are many possibilities for further work starting from this
formalization. We will mention just a few.

Obvious missing pieces would be proving the standard properties of
general Cohen posets $\Fn_\kappa(I,J)$, and to modify the core
definitions to allow for class forcing. We would also like to try the
Boolean valued approach to compare the (un)ease of formalization using
Isabelle/ZF.

Another desirable goal is to construct transitive set models of $\ZFC$
from a large cardinal. There's some work to be done for that: Even the
definition of inaccessibles, and of the transitive collapse (for that
matter), are still missing.

In line with identifying (almost) minimal set of axioms, we would like
to obtained formalized results concerning which instances of
Separation are needed to use forcing. A necessary ingredient will
certainly be an implementation of Gödel operations
\cite[Thm.~13.4]{Jech_Millennium}.

In our previous landmark \cite{2020arXiv200109715G}, we contributed
with some modifications to \session{ZF-Constructible}; this is now
part of the official Isabelle distribution.  We intend ask Isabelle
maintainers to consider the more modular versions of some of those
theories that we are presenting in this project.

As final words about our journey, we believe that, as in mathematics
in general, the experience of working in a formal environment can be
daunting, but at the same time extremely rewarding: The feeling of
accomplishment after seeing one's writings validated beyond doubt is
in some ways comparable that of finding a proof of an important
lemma. It also allows subtly different ways of reasoning (with their
own merits and pitfalls---it is easy to forget how easy a proof on
paper is once you are fully engaged in directing your assistant). We
hope that at some point these experiencies are shared by our community
at large.


\section*{Acknowledgments}
\label{sec:acknowledgments}
We thank Z.~Vidnyánszky for some discussions where the idea of
building the summary in \theory{Definitions\_Main} took shape.
Mikhail Mandrykin helped us with technical assistance regarding the
rendering of formulas in the Isabelle/jEdit interface, which improved
our presentation of this material during the Latin American Congress
of Mathematicians (CLAM 2021); we make this extensive to the rest of
the Isabelle community, for its kind support through the Users mailing
list. We also want to warmly thank Larry Paulson for his work and
encouragement.  Last, and definitely not least, we are deeply grateful
to Ken Kunen for his
inspiring expositions. This formalization is a tiny homage to his
memory.


%%% Local Variables: 
%%% mode: latex
%%% TeX-master: "independence_ch_isabelle"
%%% ispell-local-dictionary: "american"
%%% End: 
