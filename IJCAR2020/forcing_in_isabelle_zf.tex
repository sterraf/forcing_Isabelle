% This is samplepaper.tex, a sample chapter demonstrating the
% LLNCS macro package for Springer Computer Science proceedings;
% Version 2.20 of 2017/10/04
%
\documentclass[runningheads]{llncs}
%
\usepackage[utf8]{inputenc}
\usepackage{isabelle,isabellesym}
\usepackage{amsmath}
%\usepackage{amsthm}
\usepackage{amsfonts}
\usepackage{amssymb}
%\usepackage{bbm}  % Para el \bb{1}
%\usepackage[numbers]{natbib}
\usepackage{enumitem}
\usepackage{babel}
%\usepackage{babelbib}
\usepackage{multidef}
\usepackage{verbatim}
\usepackage{stmaryrd} %% para \llbracket
%%
%% \usepackage[bottom=2cm, top=2cm, left=2cm, right=2cm]{geometry}
%% \usepackage{titling}
%% \setlength{\droptitle}{-10ex} 
%%
\renewcommand{\o}{\vee}
\renewcommand{\O}{\bigvee}
\newcommand{\y}{\wedge}
\newcommand{\Y}{\bigwedge}
\newcommand{\limp}{\rightarrow}
\newcommand{\lsii}{\leftrightarrow}
%%

\DeclareMathOperator{\cf}{cf}
\DeclareMathOperator{\dom}{dom}
\DeclareMathOperator{\im}{img}
\DeclareMathOperator{\Fn}{Fn}
\DeclareMathOperator{\rk}{rk}
\DeclareMathOperator{\mos}{mos}
\DeclareMathOperator{\trcl}{trcl}
\DeclareMathOperator*{\diag}{\bigtriangleup}
\DeclareMathOperator{\Con}{Con}
\DeclareMathOperator{\Club}{Club}


\newcommand{\modelo}[1]{\mathbf{#1}}
\newcommand{\axiomas}[1]{\mathit{#1}}
\newcommand{\clase}[1]{\mathsf{#1}}
\newcommand{\poset}[1]{\mathbb{#1}}
\newcommand{\operador}[1]{\mathbf{#1}}

%% \newcommand{\Lim}{\clase{Lim}}
%% \newcommand{\Reg}{\clase{Reg}}
%% \newcommand{\Card}{\clase{Card}}
%% \newcommand{\On}{\clase{On}}
%% \newcommand{\WF}{\clase{WF}}
%% \newcommand{\HF}{\clase{HF}}
%% \newcommand{\HC}{\clase{HC}}
%%
%% El siguiente comando reemplaza todos los anteriores:
%%
\multidef{\clase{#1}}{Card,HC,HF,Lim,On->Ord,Reg,WF,Ord}
\newcommand{\ON}{\On}

%% En lugar de usar todo el paquete bbm:
\DeclareMathAlphabet{\mathbbm}{U}{bbm}{m}{n} 
\newcommand{\1}{\mathbbm{1}}

%%
%% \newcommand{\calD}{\mathcal{D}}
%% \newcommand{\calS}{\mathcal{S}}
%% \newcommand{\calU}{\mathcal{U}}
%% \newcommand{\calB}{\mathcal{B}}
%% \newcommand{\calL}{\mathcal{L}}
%% \newcommand{\calF}{\mathcal{F}}
%% \newcommand{\calT}{\mathcal{T}}
%% \newcommand{\calW}{\mathcal{W}}
%% \newcommand{\calA}{\mathcal{A}}
%%
%% El siguiente comando reemplaza todos los anteriores:
%%
\multidef[prefix=cal]{\mathcal{#1}}{A-Z}
%%
%% \newcommand{\A}{\modelo{A}}
%% \newcommand{\BB}{\modelo{B}}
%% \newcommand{\ZZ}{\modelo{Z}}
%% \newcommand{\PP}{\modelo{P}}
%% \newcommand{\QQ}{\modelo{Q}}
%% \newcommand{\RR}{\modelo{R}}
%%
%% El siguiente comando reemplaza todos los anteriores:
%%
\multidef{\modelo{#1}}{A,BB->B,CC->C,NN->N,PP->P,QQ->Q,RR->R,ZZ->Z}

\multidef[prefix=p]{\mathbb{#1}}{A-Z}
%% \newcommand{\B}{\modelo{B}}
%% \newcommand{\C}{\modelo{C}}
%% \newcommand{\F}{\modelo{F}}
%% \newcommand{\D}{\modelo{D}}

\newcommand{\Th}{\mb{Th}}
\newcommand{\Mod}{\mb{Mod}}

\newcommand{\Se}{\operador{S^\prec}}
\newcommand{\Pu}{\operador{P_u}}
\renewcommand{\Pr}{\operador{P_R}}
\renewcommand{\H}{\operador{H}}
\renewcommand{\S}{\operador{S}}
\newcommand{\I}{\operador{I}}
\newcommand{\E}{\operador{E}}

\newcommand{\se}{\preccurlyeq}
\newcommand{\ee}{\succ}
\newcommand{\id}{\approx}
\newcommand{\subm}{\subseteq}
\newcommand{\ext}{\supseteq}
\newcommand{\iso}{\cong}
%%
\renewcommand{\emptyset}{\varnothing}
\newcommand{\rel}{\mathcal{R}}
\newcommand{\Pow}{\mathop{\mathcal{P}}}
\renewcommand{\P}{\Pow}
\newcommand{\BP}{\mathrm{BP}}
\newcommand{\func}{\rightarrow}
\newcommand{\ord}{\mathrm{Ord}}
\newcommand{\R}{\mathbb{R}}
\newcommand{\N}{\mathbb{N}}
\newcommand{\Z}{\mathbb{Z}}
\renewcommand{\I}{\mathbb{I}}
\newcommand{\Q}{\mathbb{Q}}
\newcommand{\B}{\mathbf{B}}
\newcommand{\<}{\langle}
\renewcommand{\>}{\rangle}
\newcommand{\lb}{\langle}
\newcommand{\rb}{\rangle}
\newcommand{\impl}{\rightarrow}
\newcommand{\ent}{\Rightarrow}
\newcommand{\tne}{\Leftarrow}
\newcommand{\sii}{\Leftrightarrow}
\renewcommand{\phi}{\varphi}
\newcommand{\phis}{{\varphi^*}}
\renewcommand{\th}{\theta}
\newcommand{\Lda}{\Lambda}
\newcommand{\La}{\Lambda}
\newcommand{\lda}{\lambda}
\newcommand{\ka}{\kappa}
\newcommand{\del}{\delta}
\newcommand{\de}{\delta}
\newcommand{\ze}{\zeta}
%\newcommand{\ }{\ }
\newcommand{\la}{\lambda}
\newcommand{\al}{\alpha}
\newcommand{\be}{\beta}
\newcommand{\ga}{\gamma}
\newcommand{\Ga}{\Gamma}
\newcommand{\ep}{\varepsilon}
\newcommand{\De}{\Delta}
\newcommand{\defi}{\mathrel{\mathop:}=}
\newcommand{\forces}{\Vdash}
%\newcommand{\ap}{\mathbin{\wideparen{\ }}}
\newcommand{\Tree}{{\mathrm{Tr}_\N}}
\newcommand{\PTree}{{\mathrm{PTr}_\N}}
\newcommand{\NWO}{\mathit{NWO}}
\newcommand{\Suc}{{\N^{<\N}}}%
\newcommand{\init}{\mathsf{i}}
\newcommand{\ap}{\mathord{^\smallfrown}}
\newcommand{\Cantor}{\mathcal{C}}
%\newcommand{\C}{\Cantor}
\newcommand{\Baire}{\mathcal{N}}
\newcommand{\sig}{\ensuremath{\sigma}}
\newcommand{\fsig}{\ensuremath{F_\sigma}}
\newcommand{\gdel}{\ensuremath{G_\delta}}
\newcommand{\Sig}{\ensuremath{\boldsymbol{\Sigma}}}
\newcommand{\bPi}{\ensuremath{\boldsymbol{\Pi}}}
\newcommand{\Del}{\ensuremath{\boldsymbol\Delta}}
%\renewcommand{\F}{\operador{F}}
\newcommand{\ths}{{\theta^*}}
\newcommand{\om}{\ensuremath{\omega}}
%\renewcommand{\c}{\complement}
\newcommand{\comp}{\mathsf{c}}
\newcommand{\co}[1]{\left(#1\right)^\comp}
\newcommand{\len}[1]{\left|#1\right|}
\DeclareMathOperator{\tlim}{\overline{\mathrm{TLim}}}
\newcommand{\card}[1]{{\left|#1\right|}}
\newcommand{\bigcard}[1]{{\bigl|#1\bigr|}}
%
% Cardinality
%
\newcommand{\lec}{\leqslant_c}
\newcommand{\gec}{\geqslant_c}
\newcommand{\lc}{<_c}
\newcommand{\gc}{>_c}
\newcommand{\eqc}{=_c}
\newcommand{\biy}{\approx}
\newcommand*{\ale}[1]{\aleph_{#1}}
%
\newcommand{\Zerm}{\axiomas{Z}}
\newcommand{\ZC}{\axiomas{ZC}}
\newcommand{\AC}{\axiomas{AC}}
\newcommand{\DC}{\axiomas{DC}}
\newcommand{\MA}{\axiomas{MA}}
\newcommand{\CH}{\axiomas{CH}}
\newcommand{\ZFC}{\axiomas{ZFC}}
\newcommand{\ZF}{\axiomas{ZF}}
\newcommand{\Inf}{\axiomas{Inf}}
%
% Cardinal characteristics
%
\newcommand{\cont}{\mathfrak{c}}
\newcommand{\spl}{\mathfrak{s}}
\newcommand{\bound}{\mathfrak{b}}
\newcommand{\mad}{\mathfrak{a}}
\newcommand{\tower}{\mathfrak{t}}
%
\renewcommand{\hom}[2]{{}^{#1}\hskip-0.116ex{#2}}
\newcommand{\pred}[1][{}]{\mathop{\mathrm{pred}_{#1}}}
%% Postfix operator with supressable space:
%% \newcommand*{\iseg}{\relax\ifnum\lastnodetype>0 \mskip\medmuskip\fi{\downarrow}} %
\newcommand*{\iseg}{{\downarrow}}
\newcommand{\rr}{\mathrel{R}}
\newcommand{\restr}{\upharpoonright}
%\newcommand{\type}{\mathtt{}}
\newcommand{\app}{\mathop{\mathrm{Aprox}}}
\newcommand{\hess}{\triangleleft}
\newcommand{\bx}{\bar{x}}
\newcommand{\by}{\bar{y}}
\newcommand{\bz}{\bar{z}}
\newcommand{\union}{\mathop{\textstyle\bigcup}}
\newcommand{\sm}{\setminus}
\newcommand{\sbq}{\subseteq}
\newcommand{\nsbq}{\subseteq}
\newcommand{\mty}{\emptyset}
\newcommand{\dimg}{\text{\textup{``}}} % direct image
\newcommand{\quine}[1]{\ulcorner{\!#1\!}\urcorner}
%\newcommand{\ntrm}[1]{\textsl{\textbf{#1}}}
\newcommand{\Null}{\calN\!\mathit{ull}}
\DeclareMathOperator{\club}{Club}
\DeclareMathOperator{\otp}{otp}

%%%%%%%%%%%%%%%%%%%%%%%%%
% Variant aleph, beth, etc
% From http://tex.stackexchange.com/q/170476/69595
\makeatletter
\@ifpackageloaded{txfonts}\@tempswafalse\@tempswatrue
\if@tempswa
  \DeclareFontFamily{U}{txsymbols}{}
  \DeclareFontFamily{U}{txAMSb}{}
  \DeclareSymbolFont{txsymbols}{OMS}{txsy}{m}{n}
  \SetSymbolFont{txsymbols}{bold}{OMS}{txsy}{bx}{n}
  \DeclareFontSubstitution{OMS}{txsy}{m}{n}
  \DeclareSymbolFont{txAMSb}{U}{txsyb}{m}{n}
  \SetSymbolFont{txAMSb}{bold}{U}{txsyb}{bx}{n}
  \DeclareFontSubstitution{U}{txsyb}{m}{n}
  \DeclareMathSymbol{\aleph}{\mathord}{txsymbols}{64}
  \DeclareMathSymbol{\beth}{\mathord}{txAMSb}{105}
  \DeclareMathSymbol{\gimel}{\mathord}{txAMSb}{106}
  \DeclareMathSymbol{\daleth}{\mathord}{txAMSb}{107}
\fi
\makeatother

%%%%%%%%%%%%%%%%%%%%%%%%%%%%%%%%%%%%%%%%%%%%%%%%%%%%%%%%%%%%
%%
%% Theorem Environments
%%
%% \newtheorem{theorem}{Theorem}
%% \newtheorem{lemma}[theorem]{Lemma}
%% \newtheorem{prop}[theorem]{Proposition}
%% \newtheorem{corollary}[theorem]{Corollary}
%% \newtheorem{claim}{Claim}
%% \newtheorem*{claim*}{Claim}
%% \theoremstyle{definition}
%% \newtheorem{definition}[theorem]{Definition}
%% \newtheorem{remark}[theorem]{Remark}
%% \newtheorem{example}[theorem]{Example}
%% \theoremstyle{remark}
%% \newtheorem*{remark*}{Remark}
%%
%%%%%%%%%%%%%%%%%%%%%%%%%%%%%%%%%%%%%%%%%%%%%%%%%%%%%%%%%%%%%%%%%%%%%%

%% \newenvironment{inducc}{\begin{list}{}{\itemindent=2.5em \labelwidth=4em}}{\end{list}}
%% \newcommand{\caso}[1]{\item[\fbox{#1}]}
\newenvironment{proofofclaim}{\begin{proof}[Proof of Claim]}{\end{proof}}


%%% Local Variables: 
%%% mode: latex
%%% TeX-master: "first_steps_into_forcing"
%%% End: 

\usepackage{graphicx}
% Used for displaying a sample figure. If possible, figure files should
% be included in EPS format.
%
\hypersetup{
  colorlinks,
  urlcolor={blue},
  linkcolor={blue!50!black},
  citecolor={blue!50!black},
}
% If you use the hyperref package, please uncomment the following line
% to display URLs in blue roman font according to Springer's eBook style:
\renewcommand\UrlFont{\color{blue}\rmfamily}

\begin{document}
%
\title{Formalization of Forcing in Isabelle/ZF%
  \thanks{Supported by Secyt-UNC project 33620180100465CB and Conicet.}%
}
%
%\titlerunning{Abbreviated paper title}
% If the paper title is too long for the running head, you can set
% an abbreviated paper title here
%
\author{Emmanuel Gunther\inst{1} \and
Miguel Pagano\inst{1} \and \\
Pedro Sánchez Terraf\inst{1,2}\orcidID{0000-0003-3928-6942}}
%
\authorrunning{E.~Gunther, M.~Pagano, P.~Sánchez Terraf}
%\authorrunning{E. Gunther et al.}
% First names are abbreviated in the running head.
% If there are more than two authors, 'et al.' is used.
%
\institute{Universidad Nacional de C\'ordoba. 
  \\  Facultad de Matem\'atica, Astronom\'{\i}a,  F\'{\i}sica y
    Computaci\'on. \\
\email{\{gunther,pagano,sterraf\}@famaf.unc.edu.ar}
\and
    Centro de Investigaci\'on y Estudios de Matem\'atica (CIEM-FaMAF),
    Conicet. C\'ordoba. Argentina.
}
%
\maketitle              % typeset the header of the contribution
%
\begin{abstract}
We formalize the development of forcing in the set theory framework of
Isabelle/ZF. Under the assumption of the existence of countable
transitive model of $\ZF$, we construct a generic extension and show
that the latter also satisfies $\ZF$.
\keywords{Isabelle/ZF \and forcing \and names \and generic extension \and constructibility.}
\end{abstract}
%
%
%
\section{Introduction}
\label{sec:introduction}

This paper is the culmination of our project on the computerized
formalization of the undecidability of the Continuum Hypothesis
($\CH$) from Zermelo-Fraenkel set theory with Choice ($\ZFC$), under the
assumption of the existence of a countable transitive model (ctm) of
$\ZFC$. In contrast to our reports of the previous steps towards this
goal
\cite{2018arXiv180705174G,2019arXiv190103313G,2020arXiv200109715G}, we
intend here to present our development to the mathematical logic
community. For this reason, we start with a general discussion around
the formalization of mathematics.

\subsection{Formalized mathematics}
The use of computers to assist the creation and verification of
mathematics has seen a steady grow. But the general awareness on the
matter still seems to be a bit scant (even among mathematicians
involved in foundations), and the venues devoted to the communication
of formalized mathematics are, mainly, computer science journals and
conferences: JAR, ITP, IJCAR, CPP, CICM, and others.

Nevertheless, the discussion about the subject in central mathematical
circles is increasing; there were some hints on the ICM2018 panel on
“machine-assisted” proofs
\cite{https://doi.org/10.48550/arxiv.1809.08062} and a lively
promotion by Kevin Buzzard, during his ICM2022 special plenary lecture
\cite{2021arXiv211211598B}.

%% These assistants provide several dialects, among which we single out:
%% \begin{enumerate}
%% \item Procedural: Useful for exploration/research.
%% \item Declarative: Only one that can be read by humans!
%% \end{enumerate}

Before we start an in-depth discussion, a point should be made clear:
A formalized proof is not the same as an \emph{automatic proof}. The
reader surely understands that, aside from results of a very specific sort, no current
technology allows us to write a reasonably complex (and correct)
theorem statement in a computer and expect to obtain a proof after hitting “Enter”, at
least not after a humanly feasible wait. On the other hand, it is
quite possible that the same reader has some mental image that
formalizing a proof requires making each application of Modus Ponens
explicit.

The fact is that \emph{proof assistants} are designed for the human prover to
be able to decompose a statement to be proved into smaller subgoals
which can actually be fed into some automatic tool. The balance between
what these tools are able to handle is not  easily appreciated by
intuition: Sometimes, ``trivial'' steps are not solved by them, which
can result in obvious frustration; but they would quickly solve some
goals that do not look like a ``mere computation.''

To appreciate the extent of mathematics formalizable, it is convenient to recall
some major successful projects, such as the Four Color Theorem
\cite{MR2463991}, the Odd Order Theorem
\cite{10.1007/978-3-642-39634-2_14}, and the proof the Kepler's
Conjecture \cite{MR3659768}. There is a vast mathematical corpus at
the Archive of Formal Proofs (AFP) based on Isabelle; and formalizations of
brand new mathematics like the Liquid Tensor Experiment
\cite{LTE2020,LTE2021} and the definition of perfectoid spaces \cite{10.1145/3372885.3373830}
have been achieved using Lean.

We will continue our description of proof assistants in
Section~\ref{sec:proof-assist-isabelle}. We kindly invite the reader
to enrich the previous exposition by reading the apt summary by
A.~Koutsoukou-Argyraki \cite{angeliki} and the interviews
therein; some of the experts consulted have also discussed
in \cite{2022arXiv220704779B} the status of formalized versus standard
proof in mathematics.

\subsection{Our achievements}
We formalized a model-theoretic rendition of forcing (Sect.~\ref{sec:forcing}), showing how to
construct proper extensions of ctms of $\ZF$ (respectively, with
$\AC$), and we formalized the basic forcing notions required to obtain
ctms of $\ZFC + \neg\CH$ and of $\ZFC + \CH$ (Sect.~\ref{sec:models-ch-negation}). No metatheoretic issues
(consistency, FOL calculi, etc) were formalized, so we were mainly
concerned with the mathematics of forcing. Nevertheless, by inspecting
the foundations underlying our proof assistant Isabelle
(Section~\ref{sec:isabelle-metalogic-meta}) it can be stated that our
formalization is a bona fide proof in $\ZF$ of the previous
constructions.

In order to reach our goals, we provided basic results that were
missing from Isabelle's $\ZF$ library, starting from ones
involving cardinal successors, countable sets, etc.
(Section~\ref{sec:extension-isabellezf}). We also extended the treatment of relativization of
set-theoretical concepts (Section~\ref{sec:tools-relativization}).
%% We redesigned Isabelle/ZF results on non-absolute concepts to work
%% relative to a class.

One added value that is obtained from the present formalization is
that we identified a handful of instances of Replacement which are
sufficient to set the forcing machinery up
(Section~\ref{sec:repl-instances}), on the basis of Zermelo set theory.
The eagerness to obtain this level of detail might be a consequence of
“an unnatural tendency to investigate, for the most part, trivial
minutiae of the formalism” on our part, as it was put by Cohen
\cite{zbMATH02012060}, but we would rather say that we were driven by
curiosity.

The code of our formalization can be accessed at the
AFP site, via the following link:
\begin{center}
  \url{https://www.isa-afp.org/entries/Independence_CH.html}
\end{center}

%%% Local Variables: 
%%% mode: latex
%%% TeX-master: "independence_ch_isabelle"
%%% ispell-local-dictionary: "american"
%%% End: 


\section{Isabelle and (meta)theories}

\begin{enumerate}
\item \emph{Pure} (meta-theory).
\item Isabelle/ZF + $\exists M$ ctm of $\ZF(C)$ (theory). 
\item $\formula$ (inner theory).
\item \textbf{Theorem}: $\exists N$ ctm of $\ZF(C)+\neg\CH$.
\end{enumerate}

in comparison to the ``proof theory approach'':

\begin{enumerate}
\item Primitive recursive arithmetic (meta-theory).
\item \textbf{Theorem}: Con($\ZF(C)$) $\implies$ Con($\ZF(C)+\neg\CH$).
\end{enumerate}

and then to the ``type-theoretic approach'':

\begin{enumerate}
\item CiC (meta-theory).
\item CiC (theory, $\geq \ZFC + \exists \kappa$ inaccessible).
\item \textbf{Theorem}: Con($\ZF(C)+\neg\CH$).
\end{enumerate}

%%% Local Variables: 
%%% mode: latex
%%% TeX-master: "forcing_in_isabelle_zf"
%%% ispell-local-dictionary: "american"
%%% End: 


\section{Relativization,  absoluteness, and the axioms}
\label{sec:relat-absol}

The concepts of relativization and absoluteness (due to Gödel, in his
proof of the relative consistency of $\AC$ \cite{godel-L}) 
are both
prerrequisites and powerful tools in working with transitive
models. A \emph{class} is simply a predicate $C(x)$ with at least one
free variable $x$.
The \emph{relativization} $\phi^C(\bx)$ of a set-theoretic
definition
$\phi$ (of a relation such
as ``$x$ is a subset of $y$'' or of a function like $y=\P(x)$) to
a class $C$ is obtained by restricting all of its quantifiers to $C$.

\[
x \sbq^C y \equiv \forall z.\ C(z) \longrightarrow (z\in x
\longrightarrow z\in y)
\]

The new formula $\phi^C(\bx)$ corresponds to what is obtained by defining
the concept ``inside'' $C$. In fact, for a class corresponding to a
set $c$ (i.e.\ $C(x) \defi x \in c$), the relativization $\phi^C$ of a 
sentence $\phi$ is equivalent to the satisfaction of $\phi$ in the
first-order model $\lb c, \in\rb$.

It turns out that many concepts mean the
same after relativization to a nonempty transitive class $C$; formally
\[
\forall\bx.\ C(\bx) \longrightarrow (\phi^C(\bx) \longleftrightarrow
\phi(\bx))
\]
When this is the case, we say that the relation defined by $\phi$ is
\emph{absolute for transitive models}. (The absoluteness of functions
moreover requires that
the relativized definition also behaves functionally over $C$.) As
examples, the relation of inclusion $\subseteq$ ---and actually, any
relation defined by a formula (equivalent to one) using only bounded
quantifiers 
$(\forall x\in y)$ and $(\exists x\in y)$--- is absolute for
transitive models. On the contrary, this is not the case with the powerset
operation.

A benefit of the work with transitive models is that many 
concepts concepts (pairs, unions, and fundamentally ordinals) are
uniform across the universe \isatt{i}, a ctm (of an adequate fragment of
$\ZF$) $M$ and any of its extensions $M[G]$

A part of this project is to refactorize Paulson's formalization
\cite{paulson_2003} of Gödel's \cite{godel-L}. The main objective is
to maximize applicability of the relativization machinery by adjusting
the hypothesis of a greater part of it early development. Paulson's
architecture had only in mind the consistency of $\ZFC$, but, for
instance, in order to apply it in the development of forcing, too much
is assumed at the beginning; more seriously, some assumptions can't be
regarded as ``first-order'' (v.g. the Replacement Scheme).

The version of \isatt{ZF-Constructible} we present weakens the
assumptions of many absoluteness theorems to that
of a nonempty transitive class; also there are some stronger results
such as the relativization of powersets.

Apart from the axiom schemes, the $\ZFC$ 
axioms are stated as predicates on classes (that is,
of type \isatt{(i{\isasymRightarrow}o){\isasymRightarrow}o}); some
of them were already defined by Paulson, and this formulation allows a
better interaction with \isatt{ZF-Constructible}. 
The axioms of Pairing, Union, Foundation,
Extensionality, and Infinity are relativizations of the respective
traditional 
first-order sentences to the class argument. For the Axiom of Choice
we selected a version best suited for the work with transitive
models: the relativization of a sentence stating that for every $x$
there is surjection from an ordinal onto $x$. Finally, Separation and
Replacement were treated separately to effectively obtain first-order
versions. It is to be noted that predicates in Isabelle/ZF are akin to
second order variables and thus do not correspond to first-order
formulas; in particular, there is no induction principle for functions
of type \isatt{i{\isasymRightarrow}o}. For that reason, Separation and
Replacement predicate \emph{on the satisfaction} of a formula $\phi$.
With respect to our previous \cite{2019arXiv190103313G}, we lifted the
arity restriction for the parameter $\phi$ on these schemes,
streamlining various proofs as an immediate benefit and avoiding the
need for tupling.

Once this class versions of the axioms are set up, it is not hard to
apply our synthesis method to obtain their internal, first-order
counterparts.
\begin{framed}
  Some code for axiom synthesis.
\end{framed}

%%% Local Variables: 
%%% mode: latex
%%% TeX-master: "forcing_in_isabelle_zf"
%%% ispell-local-dictionary: "american"
%%% End: 


\section{The definition of $\forceisa$}
\label{sec:definition-forces}

The core of the development is showing the definability of the
relation of forcing. As we explained in our previous
report~\cite[Sect.~8]{2019arXiv190103313G}, this comprises the
definition of a function $\forceisa$ that maps the set of internal
formulas into itself. It is the very reason of applicability of
forcing that the satisfaction of a first-order formula $\phi$ in all
of the generic extensions of a ctm $M$ can be ``controlled'' in a
definable way from $M$ (viz., by satisfaction of the formula
$\forceisa(\phi)$).

In fact, given a forcing notion $\PP$ (i.e. a preorder with a top element)
in a ctm $M$,
Kunen defines the \emph{forcing relation} model-theoretically 
by considering all extensions $M[G]$ with $G$ generic for $\PP$.
Then two fundamental results are proved, the Truth Lemma and the
Definability Lemma; but the proof of the first one is based on the
formula that witnesses Definability. To make sense of this in our 
formalization, we started with the internalized relation and then
proved that it is equivalent to the semantic version 
(``\isatt{definition{\isacharunderscore}of{\isacharunderscore}forces},'' in
the next section).
For that reason, the usual notation of the forcing relation 
$p \Vdash \phi\ \mathit{env}$ (for $\mathit{env}$ a list of
``names''), abbreviates in our code the
satisfaction by $M$ of $\forceisa(\phi)$:
\begin{isabelle}
\ \ {\isachardoublequoteopen}p\ {\isasymtturnstile}\ {\isasymphi}\ env\ \ \ {\isasymequiv}\ \ \ M{\isacharcomma}\ {\isacharparenleft}{\isacharbrackleft}p{\isacharcomma}P{\isacharcomma}leq{\isacharcomma}one{\isacharbrackright}\ {\isacharat}\ env{\isacharparenright}
    {\isasymTurnstile}\ forces{\isacharparenleft}{\isasymphi}{\isacharparenright}{\isachardoublequoteclose}
\end{isabelle}

The definition of $\forceisa$ proceeds by recursion
over the set $\formula$ and its base case, that is, for
atomic formulas, is (in)famously the most complicated one. Actually,
newcomers can be puzzled by the fact that forcing for atomic
formulas is also defined by (mutual) recursion: to know if $\tau_1\in\tau_2$ is
forced by $p$ (notation: $\forcesmem(p,\tau_1,\tau_2)$), one must check if $\tau_1=\sigma$ is forced for $\sigma$
moving in the transitive closure of $\tau_2$. To disentangle this, one
must realize that this last recursion must be described syntactically:
the definition of $\forceisa(\phi)$ for atomic $\phi$ is then an
internal definition of the alleged recursion on names. 

Our aim was to follow the definition proposed by Kunen
in~\cite[p.~257]{kunen2011set}, where the following mutual recursion
is given:
\begin{multline}\label{eq:def-forcing-equality}
  \forceseq (p,t_1,t_2) \defi 
  \forall s\in\dom(t_1)\cup\dom(t_2).\ \forall q\pleq p .\\
  \forcesmem(q,s,t_1)\lsii 
  \forcesmem(q,s,t_2),
\end{multline}
\begin{multline}\label{eq:def-forcing-membership}
  \forcesmem(p,t_1,t_2) \defi  \forall v\pleq p. \ \exists q\pleq v. \\
  \exists s.\ \exists r\in \PP .\ \lb s,r\rb \in
      t_2 \land q \pleq r \land \forceseq(q,t_1,s)
\end{multline}
Note that the definition of $\forcesmem$ is equivalent to require 
 the set 
\[
\{q\pleq p : \exists \lb s,r\rb\in t_2 . \ q\pleq r \land \forceseq(q,t_1,s)\}
\]
to be dense below $p$.

It was not straightforward to use the recursion machinery of
Isabelle/ZF to define $\forceseq$ and $\forcesmem$. For this, we
defined a relation $\frecR$ on 4-tuples of elements of $M$, proved
that it is well-founded and, more important, we also proved an
induction principle for this relation:
%
\begin{isabelle}
\isacommand{lemma}\isamarkupfalse%
\ forces{\isacharunderscore}induction{\isacharcolon}\isanewline
\ \ \isakeyword{assumes}\isanewline
\ \ \ \ {\isachardoublequoteopen}{\isasymAnd}{\isasymtau}\ {\isasymtheta}{\isachardot}\ {\isasymlbrakk}{\isasymAnd}{\isasymsigma}{\isachardot}\ {\isasymsigma}{\isasymin}domain{\isacharparenleft}{\isasymtheta}{\isacharparenright}\ {\isasymLongrightarrow}\ Q{\isacharparenleft}{\isasymtau}{\isacharcomma}{\isasymsigma}{\isacharparenright}{\isasymrbrakk}\ {\isasymLongrightarrow}\ R{\isacharparenleft}{\isasymtau}{\isacharcomma}{\isasymtheta}{\isacharparenright}{\isachardoublequoteclose}\footnotemark\isanewline
\ \ \ \ {\isachardoublequoteopen}{\isasymAnd}{\isasymtau}\ {\isasymtheta}{\isachardot}\ {\isasymlbrakk}{\isasymAnd}{\isasymsigma}{\isachardot}\ {\isasymsigma}{\isasymin}domain{\isacharparenleft}{\isasymtau}{\isacharparenright}\ {\isasymunion}\ domain{\isacharparenleft}{\isasymtheta}{\isacharparenright}\ {\isasymLongrightarrow}\ R{\isacharparenleft}{\isasymsigma}{\isacharcomma}{\isasymtau}{\isacharparenright}\ {\isasymand}\ R{\isacharparenleft}{\isasymsigma}{\isacharcomma}{\isasymtheta}{\isacharparenright}{\isasymrbrakk}\isanewline
\ \ \ \ \ \  {\isasymLongrightarrow}\ Q{\isacharparenleft}{\isasymtau}{\isacharcomma}{\isasymtheta}{\isacharparenright}{\isachardoublequoteclose}\isanewline
\ \ \isakeyword{shows}\isanewline
\ \ \ \ {\isachardoublequoteopen}Q{\isacharparenleft}{\isasymtau}{\isacharcomma}{\isasymtheta}{\isacharparenright}\ {\isasymand}\ R{\isacharparenleft}{\isasymtau}{\isacharcomma}{\isasymtheta}{\isacharparenright}{\isachardoublequoteclose}
\end{isabelle}
\footnotetext{The logical primitives of \emph{Pure} are
\isatt{\isasymLongrightarrow}, \isatt{\&\&\&}, and \isatt{\isasymAnd}
(implication, conjunction, and universal
quantification, resp.), which operate on the meta-Booleans
\isatt{prop}.}
%
and 
obtained both functions as cases of a another one, 
$\forcesat$, using a single recursion on $\frecR$. Then we obtained 
(\ref{eq:def-forcing-equality}) and (\ref{eq:def-forcing-membership})
as our corollaries \isatt{def{\isacharunderscore}forces{\isacharunderscore}eq} and
\isatt{def{\isacharunderscore}forces{\isacharunderscore}mem}.

Other approaches, like the one in Neeman~\cite{neeman-course} (and
Kunen's older book \cite{kunen1980}), prefer
to have a single, more complicated, definition by simple recursion for
$\forceseq$ and then define $\forcesmem$ explicitly. On hindsight,
this might have been a little simpler to do, but we preferred to be as
faithful to the text as possible concerning this point.

Once $\forcesat$ and its relativized version
$\isatt{is{\isacharunderscore}forces{\isacharunderscore}at}$ were
defined, we proceeded to show absoluteness and provided internal
definitions for the recursion on names using results in
\isatt{ZF-Constructible}. This finished the atomic case of the
formula-transformer $\forceisa$. The characterization of $\forceisa$
for negated and universal quantified formulas is given by the
following lemmas, respectively:
%
\begin{isabelle}
\isacommand{lemma}\isamarkupfalse%
\ sats{\isacharunderscore}forces{\isacharunderscore}Neg{\isacharcolon}\isanewline
\ \ \isakeyword{assumes}\isanewline
\ \ \ \ {\isachardoublequoteopen}p{\isasymin}P{\isachardoublequoteclose}\ {\isachardoublequoteopen}env\ {\isasymin}\ list{\isacharparenleft}M{\isacharparenright}{\isachardoublequoteclose}\ {\isachardoublequoteopen}{\isasymphi}{\isasymin}formula{\isachardoublequoteclose}\isanewline
\ \ \isakeyword{shows}\isanewline
\ \ \ \ {\isachardoublequoteopen}M{\isacharcomma}\ {\isacharbrackleft}p{\isacharcomma}P{\isacharcomma}leq{\isacharcomma}one{\isacharbrackright}\ {\isacharat}\ env\ {\isasymTurnstile}\ forces{\isacharparenleft}Neg{\isacharparenleft}{\isasymphi}{\isacharparenright}{\isacharparenright}\ \ \ {\isasymlongleftrightarrow}\ \isanewline
\ \ \ \ \ {\isasymnot}{\isacharparenleft}{\isasymexists}q{\isasymin}M{\isachardot}\ q{\isasymin}P\ {\isasymand}\ is{\isacharunderscore}leq{\isacharparenleft}{\isacharhash}{\isacharhash}M{\isacharcomma}leq{\isacharcomma}q{\isacharcomma}p{\isacharparenright}\ {\isasymand}\ \isanewline
\ \ \ \ \ \ \ \ \ \ M{\isacharcomma}\ {\isacharbrackleft}q{\isacharcomma}P{\isacharcomma}leq{\isacharcomma}one{\isacharbrackright}{\isacharat}env\ {\isasymTurnstile}\ forces{\isacharparenleft}{\isasymphi}{\isacharparenright}{\isacharparenright}{\isachardoublequoteclose}\isanewline

\isacommand{lemma}\isamarkupfalse%
\ sats{\isacharunderscore}forces{\isacharunderscore}Forall{\isacharcolon}\isanewline
\ \ \isakeyword{assumes}\isanewline
\ \ \ \ {\isachardoublequoteopen}p{\isasymin}P{\isachardoublequoteclose}\ {\isachardoublequoteopen}env\ {\isasymin}\ list{\isacharparenleft}M{\isacharparenright}{\isachardoublequoteclose}\ {\isachardoublequoteopen}{\isasymphi}{\isasymin}formula{\isachardoublequoteclose}\isanewline
\ \ \isakeyword{shows}\isanewline
\ \ \ \ {\isachardoublequoteopen}M{\isacharcomma}{\isacharbrackleft}p{\isacharcomma}P{\isacharcomma}leq{\isacharcomma}one{\isacharbrackright}\ {\isacharat}\ env\ {\isasymTurnstile}\ forces{\isacharparenleft}Forall{\isacharparenleft}{\isasymphi}{\isacharparenright}{\isacharparenright}\ {\isasymlongleftrightarrow}\ \isanewline
\ \ \ \ \ {\isacharparenleft}{\isasymforall}x{\isasymin}M{\isachardot}\ \ \ M{\isacharcomma}\ {\isacharbrackleft}p{\isacharcomma}P{\isacharcomma}leq{\isacharcomma}one{\isacharcomma}x{\isacharbrackright}\ {\isacharat}\ env\ {\isasymTurnstile}\ forces{\isacharparenleft}{\isasymphi}{\isacharparenright}{\isacharparenright}{\isachardoublequoteclose}
\end{isabelle}

Let us note in passing another improvement over our previous report:
we made a couple of new technical results concerning recursive
definitions. Paulson proved absoluteness of functions defined by
well-founded recursion over a transitive relation. Some of our
definitions by recursion (\emph{check} and \emph{forces}) do not fit
in that scheme.  One can replace the relation $R$ for its transitive
closure $R^+$ in the recursive definition because one can prove, in
general, that
$F\!\upharpoonright\!(R^{-1}(x))(y) =
F\!\upharpoonright\!({R^+}^{-1}(x))(y)$ whenever $(x,y) \in R$.


%%% Local Variables: 
%%% mode: latex
%%% TeX-master: "forcing_in_isabelle_zf"
%%% ispell-local-dictionary: "american"
%%% End: 


\section{The axiom of replacement}


%%% Local Variables: 
%%% mode: latex
%%% TeX-master: "forcing_in_isabelle_zf"
%%% ispell-local-dictionary: "american"
%%% End: 


\section{Conclusions and future work}
There are several technical milestones that have to be reached in the
course of a formalization of the theory of forcing. The first one, and most
obvious, is the bulk of set- and meta-theoretical concepts needed to work
with. This pushed us, in a sense,  into building on top of Isabelle/ZF,
since we know of no other development in set theory of such
depth (and breadth). In this paper we worked on setting the stage for the work with
generic extensions; in particular, this involves some purely mathematical
results, as the Rasiowa-Sikorski lemma. 

Other milestones in this formalization project
involve 
\begin{enumerate}
\item the definition
  of the forcing relation, 
\item proving the Fundamental Theorem of forcing
  (that relates truth in $M$ to that in $M[G]$), and 
\item using it to show
  that $M[G]\models \ZFC$. 
\end{enumerate}
The theory is very modular and this is
witnessed by the fact 
that the last goal does not depend on the proof of the Fundamental
Theorem nor on the definition of the forcing relation. Our next task
will be to obtain the last goal in that enumeration. 

To this end, we will develop an interface between Paulson's
relativization results and countable models of $\ZFC$. This will show
that every ctm $M$ is closed under well-founded recursion and, in
particular, that contains names for each of its
elements. Consequently, the proof of  $M\sbq M[G]$ will be
complete. A landmark will be to prove the Axiom Scheme
of Separation (the first that needs to use the machinery of forcing
nontrivially). As a part of the new formalization, we will provide
Isar versions of the longer applicative proofs presented in this work.

\ack{We'd like to thank the anonymous referees for reading the paper
  carefully and for their detailed and constructive criticism.}
%%% Local Variables:
%%% mode: latex
%%% ispell-local-dictionary: "american"
%%% TeX-master: "first_steps_into_forcing"
%%% End:

%
% ---- Bibliography ----
%
% BibTeX users should specify bibliography style 'splncs04'.
% References will then be sorted and formatted in the correct style.
%
\bibliographystyle{splncs04}
\bibliography{forcing_in_isabelle_zf}
\end{document}
