\section{Relative versions of non-absolute concepts}
\label{sec:relat-vers-non-absol}

The treatment of relativization/internalization described in the
previous section was enough for the Paulson's treatment of
constructibility. This is the case because essentially all the
concepts in the way of proving the consistency of $\AC$ are
absolute, and treatment of relational versions and relativized notions
could be minimized after proving the relevant absoluteness results:
For example, the lemma \isatt{Union{\uscore}abs},
\[
  M(A) \implies M(z) \implies \isatt{big{\uscore}union}(M, A, z) \longleftrightarrow z = \union
  A
\]
proved under the assumption that $M$ is transitive nonempty. Working in this
way with powersets, cardinalities, and the like would be
unfeasible.

In order to cope with this, we added the missing step from the
literature consisting of relative versions of the various non-absolute
functions and limited automatic facilities that define, state, and
prove the needed concepts. For instance, the 
$\isatt{cardinal}::\tyi \fun \tyi$ function is defined in
\session{Isabelle/ZF}, and the commands
\begin{isabelle}
  \isacommand{relativize}\isamarkupfalse%
  \ \isakeyword{functional}\ {\isachardoublequoteopen}cardinal{\isachardoublequoteclose}\ {\isachardoublequoteopen}cardinal{\isacharunderscore}{\kern0pt}rel{\isachardoublequoteclose}\ \isakeyword{external}\isanewline
  \isacommand{relationalize}\isamarkupfalse%
  \ {\isachardoublequoteopen}cardinal{\isacharunderscore}{\kern0pt}rel{\isachardoublequoteclose}\ {\isachardoublequoteopen}is{\isacharunderscore}{\kern0pt}cardinal{\isachardoublequoteclose}\isanewline
  \isacommand{synthesize}\isamarkupfalse%
  \ {\isachardoublequoteopen}is{\isacharunderscore}{\kern0pt}cardinal{\isachardoublequoteclose}\ \isakeyword{from{\isacharunderscore}{\kern0pt}definition}\ \isakeyword{assuming}\ {\isachardoublequoteopen}nonempty{\isachardoublequoteclose}%
\end{isabelle}
define the relative cardinal function
$\isatt{cardinal{\uscore}rel}::(\tyi \fun \tyo) \fun \tyi \fun\tyi$,
its relational version $\isatt{is{\uscore}cardinal}$ that encodes the
statement $|x|^M = z$, the
internalized formula \isatt{is{\uscore}cardinal{\uscore}fm} whose
satisfaction by a set is equivalent to the relational version, and the
proof of the previous statements.


%%% Local Variables: 
%%% mode: latex
%%% TeX-master: "independence_ch_isabelle"
%%% ispell-local-dictionary: "american"
%%% End: 
