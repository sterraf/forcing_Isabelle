\section{Related work}
\label{sec:related-work}

There is another formalization of forcing in Lean by Han and van
Doorn, under the name \emph{Flypitch} \cite{han_et_al:LIPIcs:2019:11074,DBLP:conf/cpp/HanD20}. When our project started, we
were unaware of this initiative, and the same as them, we were deeply
influenced by Wiedijk's list of 100 theorems \cite{Formalizing100}.

Many aspects make their formalization different from ours. Their
presentation of the mathematics is somewhat more elegant and cohesive,
since they go for the Boolean valued approach; they also  set up the
calculus of first-order logic, and \emph{en route} to forcing they
formalized the basic model theory of Boolean valued models and Gödel's
Completeness Theorem. They also provided the treatment of the regular
open algebra, and the general version of the Delta System
Lemma. Putting this together they readily obtain a proof that
$\ZFC\nvdash \CH$ \cite{han_et_al:LIPIcs:2019:11074} and after
formalizing collapse forcing they show  $\ZFC\nvdash \neg\CH$
\cite[Sect.~5.6]{DBLP:conf/cpp/HanD20}.

It should be emphasized, however, that the Flypitch project was
carried out assuming a rather strong metatheory.
Carneiro \cite{carneiro-ms-thesis} reports that Werner's results in
\cite{10.5555/645869.668660} can be adapted to show that the base
logic  of Lean (restricted to $n$
type universes) proves the consistency of $\ZFC$ plus $n$
inaccessibles. Han and van Doorn did use universes in their
implementation; for instance, ordinals are “defined as equivalence
classes of (well-ordered) types, [\dots] one
universe level higher than the types used to construct them”
\cite{han_et_al:LIPIcs:2019:11074}. It is not clear to us if they are
able to avoid such strength: At least, $\Con(\ZFC)$ is provable in
their context. %, a requisite to get either unprovability result.
% by Gödel's Second).

On a lesser note, in order to prove $\AC$ in the generic extension,
Flypitch requires choice in the metatheory
\cite[p.~11]{han_et_al:LIPIcs:2019:11074}, while our formalization works
entirely in $\ZF$. %%  and only imports the Isabelle theory postulating
%% $\AC$ for demostration purposes.

We believe that our alternative formalization using the ctm approach
over Isabelle/ZF might be more appealing to set-theorists because of the
type-theoretic machinery used in Flypitch, and % might be a bit alien
since absoluteness grants us extra naturality. This last point may also
be illustrated by the treatment of ordinals; in our formalization, as
it is expected intuitively, the following are equivalent for every $x$
in a ctm $M$:
\begin{itemize}
\item $\isa{Ord}(x)$;
\item $\isa{ordinal}(\isa{\#\#}M,x)$ (the relational
  definition relativized to $M$);
\item
  $M,[x] \models {\cdot}0\isa{ is ordinal}{\cdot}$.
%% \item
%%   $M[G], [x] \models {\cdot}0\isa{ is ordinal}{\cdot}$ for every generic $G$,
\end{itemize}
where ${\cdot}n\isa{ is ordinal}{\cdot}$ is the code for the
appropriate first-order formula  ($0$ is a de
Bruijn index above!). In contrast, Han and van Doorn require an
injection from the ordinals of the corresponding type universe into
their encoding of a model of $\ZF$, and a further necessary injection into the Boolean
valued model using checks---this last step obviously appears in our
presentation, but the $\val$ function used to construct $M[G]$ will turn
check-names into the corresponding argument, as expected.

This faithfulness to set-theoretical practice does not come for
free. Recursive constructions and inductive definitions are far easier
to perform in the Calculus of Inductive Constructions on which Lean
is based, and in Isabelle/ZF are rather cumbersome. Also, a
typed discipline provides aid to write succinctly and many assumptions
are satisfied by mere notation. To be clear, those benefits come from
doing set theory in a non set-theoretical meta-language. On the other
hand, Isar proofs, as the one shown in
Sec.~\ref{sec:sample-formal-proof}, are easier to write and understand
than the language of tactics of Lean.

A sweet spot combining the best of both worlds is to be found on
developments in Isabelle/HOL based on the AFP entry
\session{ZFC\_in\_HOL} by Paulson \cite{ZFC_in_HOL-AFP}. There is a
range of results in combinatorics and other set-theoretical material
that was swiftly formalized in this setting: Nash-Williams theorem and
Larson's $\forall k\ \omega^{\omega}\longrightarrow(\omega^\omega,k)$
\cite{doi:10.1080/10586458.2021.1980464}, Design Theory
\cite{10.1007/978-3-030-81097-9_1}, and Wetzel's problem
\cite{2022arXiv220503159P}; this last paper describes in a brief and
clear way the convenience of the interaction with Isabelle/HOL
(which is also paid in consistency-strength currency
\cite[Sect.~3]{DBLP:conf/ictac/Obua06}).

%%% Local Variables: 
%%% mode: latex
%%% TeX-master: "independence_ch_isabelle"
%%% ispell-local-dictionary: "american"
%%% End: 
