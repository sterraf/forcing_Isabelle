\section{Conclusions and future work}
There are several technical milestones that have to be traversed in the
course of a formalization of the theory of forcing. The first and
obvious one is the bulk of set- and meta-theoretical concepts needed to work
with. This pushed us, in a sense,  into building on top of Isabelle/ZF,
since we know of no other development in set theory of such
depth. In this paper we worked on setting the stage for the work with
generic extensions; in particular, this involves some purely mathematical
results, as the Rasiowa-Sikorski lemma. 

Other milestones in this formalization project involve the definition
of the forcing relation, proving the Fundamental Theorem of forcing
(that relates truth in $M$ to that in $M[G]$), and using it to show
that $M[G]\models \ZFC$. The theory is extremely modular in the sense
that the last goal does not depend on the proof of the Fundamental
Theorem nor on the definition of the forcing relation. In a next work,
we will obtain the last goal.

\begin{itemize}
\item Write Isar versions of most proofs in the development.
\item Develop an interface between Paulson's relativization results
  and countable models $M$ of $\ZFC$. This will show
  that every such $M$ is closed under well-founded recursion and, in
  particular, that contains names for each of its
  elements. Consequently, $M\sbq M[G]$.
\item Prove that the basic axioms of $\ZFC$ hold in the generic
  extension. 
\item After prototyping the forcing relation, prove the Axiom Scheme
  of Separation (the first that needs to use the machinery of forcing
  nontrivially).
\end{itemize}

%%% Local Variables:
%%% mode: latex
%%% ispell-local-dictionary: "american"
%%% TeX-master: "first_steps_into_forcing"
%%% End:
