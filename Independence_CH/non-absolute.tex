\section{Relative versions of non-absolute concepts}
\label{sec:relat-vers-non-absol}

The treatment of relativization/internalization described in the
previous sections was enough for the Paulson's treatment of
constructibility. This is the case because essentially all the
concepts in the way of proving the consistency of $\AC$ are
absolute, and treatment of relational versions and relativized notions
could be minimized after proving the relevant absoluteness results:
For example, the lemma \isatt{Union{\uscore}abs},
\[
  M(A) \implies M(z) \implies \isatt{big{\uscore}union}(M, A, z) \longleftrightarrow z = \union
  A
\]
proved under the assumption that $M$ is transitive nonempty. Working
in this relational 
way with powersets, cardinalities, and the like would be
unfeasible. As such, cardinal arithmetic was not put in relative form
in \session{ZF-Constructible}.

\subsection{Tools for relativization}
\label{sec:tools-relativization}
In order to cope with this, we added the missing step from the
literature consisting of relative versions of the various non-absolute
functions and programmed in ML limited automatic facilities that
define the needed concepts, and  state and
prove the requisite lemmas. For instance, the 
$\isatt{cardinal}::\tyi \fun \tyi$ function is defined in
\session{Isabelle/ZF}, and the commands
\begin{isabelle}
  \isacommand{relativize}\isamarkupfalse%
  \ \isakeyword{functional}\ {\isachardoublequoteopen}cardinal{\isachardoublequoteclose}\ {\isachardoublequoteopen}cardinal{\isacharunderscore}{\kern0pt}rel{\isachardoublequoteclose}\ \isakeyword{external}\isanewline
  \isacommand{relationalize}\isamarkupfalse%
  \ {\isachardoublequoteopen}cardinal{\isacharunderscore}{\kern0pt}rel{\isachardoublequoteclose}\ {\isachardoublequoteopen}is{\isacharunderscore}{\kern0pt}cardinal{\isachardoublequoteclose}\isanewline
  \isacommand{synthesize}\isamarkupfalse%
  \ {\isachardoublequoteopen}is{\isacharunderscore}{\kern0pt}cardinal{\isachardoublequoteclose}\ \isakeyword{from{\isacharunderscore}{\kern0pt}definition}\ \isakeyword{assuming}\ {\isachardoublequoteopen}nonempty{\isachardoublequoteclose}%
\end{isabelle}
define the relative cardinal function
$\isatt{cardinal{\uscore}rel}::(\tyi \fun \tyo) \fun \tyi \fun\tyi$
(denoted  $|\cdot|^M$, as expected),
the relational version of the latter $\isatt{is{\uscore}cardinal}$, the
internalized formula \isatt{is{\uscore}cardinal{\uscore}fm} whose
satisfaction by a set is equivalent to the relational version, and
prove the previous statement (analogous to (\ref{eq:sats_big_union_fm})).
The proof that $\isatt{is{\uscore}cardinal}(M,x,z)$  encodes the
statement $|x|^M = z$ must still be done by hand, since the definition
of $\isatt{cardinal{\uscore}rel}$ already involves some tacit
absoluteness results (“\textit{the least $z \in \Ord$ such that $z
  \approx^M x$}” instead
of “\textit{the least $z \in \Ord^M$ such that $z
  \approx^M x$}”, and the like).

\subsection{Extension of Isabelle/ZF}
\label{sec:extension-isabellezf}
We extended \cite{Delta_System_Lemma-AFP} the material formalized in
Isabelle, from basic results involving function spaces and the
definition of cardinal exponentiation, to a treatment of cofinality
and the Delta System Lemma for $\omega_1$-families. We also included a
concise treatment of the axiom of Dependent Choices $\DC$ and the
general version of Rasiowa-Sikorski Lemma \cite{2018arXiv180705174G}
and a choiceless one for countable preorders.

This material was subsequently put in relative form in our formal
development on transitive class models \cite{Transitive_Models-AFP}
using as an aid the tools from
Section~\ref{sec:tools-relativization}. We also relativized many
original theories appearing in \session{ZF}, including the
fundamentals of cardinal arithmetic, the cumulative hierarchy, and the
definition of the $\ale{}$ function.


%%% Local Variables:
%%% mode: latex
%%% TeX-master: "independence_ch_isabelle"
%%% ispell-local-dictionary: "american"
%%% End: 
