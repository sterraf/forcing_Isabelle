\section{Relative versions of non-absolute concepts}
\label{sec:relat-vers-non-absol}

The treatment of relativization/internalization described in the
previous sections was enough for Paulson's treatment of
constructibility. This is the case because essentially all the
concepts in the way of proving the consistency of $\AC$ are
absolute, and the treatment of relational versions and relativized notions
could be minimized after proving the relevant absoluteness results:
For example, the lemma \isa{Union{\uscore}abs},
\[
  M(A) \implies M(z) \implies \isa{big{\uscore}union}(M, A, z) \longleftrightarrow z = \union
  A
\]
proved under the assumption that $M$ is transitive and nonempty.

Our first attempt to relativize cardinal arithmetic proceeded in the
same way
and we rapidly found out that stating and proving statements like $(||A||
= |A|) ^M$ in a completely relational language was extremely
cumbersome. This observation lead to the discovery of the discipline
expounded in the next subsection.

%% Working
%% in this relational 
%% way with powersets, cardinalities, and the like would be
%% unfeasible. As such, cardinal arithmetic was not put in relative form
%% in \session{ZF-Constructible}.

\subsection{Discipline and tools for relativization}
\label{sec:tools-relativization}
The missing step, that naturally appears in the literature, consists
of having relative \emph{functions} like $\Pow^M$, and the ability to
translate between the different presentations discussed so far.

To achieve this, we provide automatic tools to ease the definitions of
such relative versions, their fully relational counterparts, and the
internalized formulas. For instance, consider the
$\isa{cardinal}::\tyi \fun \tyi$ function defined in
\session{Isabelle/ZF}. Then the commands
\begin{isabelle}
  \isacommand{relativize}\isamarkupfalse%
  \ \isakeyword{functional}\ {\isachardoublequoteopen}cardinal{\isachardoublequoteclose}\ {\isachardoublequoteopen}cardinal{\isacharunderscore}{\kern0pt}rel{\isachardoublequoteclose}\ \isakeyword{external}\isanewline
  \isacommand{relationalize}\isamarkupfalse%
  \ {\isachardoublequoteopen}cardinal{\isacharunderscore}{\kern0pt}rel{\isachardoublequoteclose}\ {\isachardoublequoteopen}is{\isacharunderscore}{\kern0pt}cardinal{\isachardoublequoteclose}\isanewline
  \isacommand{synthesize}\isamarkupfalse%
  \ {\isachardoublequoteopen}is{\isacharunderscore}{\kern0pt}cardinal{\isachardoublequoteclose}\ \isakeyword{from{\isacharunderscore}{\kern0pt}definition}\ \isakeyword{assuming}\ {\isachardoublequoteopen}nonempty{\isachardoublequoteclose}%
\end{isabelle}
define the relative cardinal function
$\isa{cardinal{\uscore}rel}::(\tyi \fun \tyo) \fun \tyi \fun\tyi$
(denoted  $|\cdot|^M$, as expected),
the relational version $\isa{is{\uscore}cardinal}$ of the latter, the
internalized formula \isa{is{\uscore}cardinal{\uscore}fm} whose
satisfaction by a set is equivalent to the relational version, and
prove the previous statement (analogous to (\ref{eq:sats_big_union_fm})).
The proof that $\isa{is{\uscore}cardinal}(M,x,z)$  encodes the
statement $|x|^M = z$ must still be done by hand, since the definition
of $\isa{cardinal{\uscore}rel}$ already involves some tacit
absoluteness results (“\textit{the least $z \in \Ord$ such that $z
  \approx^M x$}” instead
of “\textit{the least $z \in \Ord^M$ such that $z
  \approx^M x$}”, and the like).

\subsection{Extension of Isabelle/ZF}
\label{sec:extension-isabellezf}
We extended \cite{Delta_System_Lemma-AFP} the material formalized in
Isabelle, from basic results involving function spaces and the
definition of cardinal exponentiation, to a treatment of cofinality
and the Delta System Lemma for $\omega_1$-families. We also included a
concise treatment of the axiom of Dependent Choices $\DC$ and the
general version of Rasiowa-Sikorski Lemma \cite{2018arXiv180705174G}
and a choiceless one for countable preorders.

This material was subsequently put in relative form in our formal
development on transitive class models \cite{Transitive_Models-AFP}
using as an aid the tools from
Section~\ref{sec:tools-relativization}. We also relativized many
original theories appearing in Isabelle/ZF, including the
fundamentals of cardinal arithmetic, the cumulative hierarchy, and the
definition of the $\ale{}$ function.


%%% Local Variables:
%%% mode: latex
%%% TeX-master: "independence_ch_isabelle"
%%% ispell-local-dictionary: "american"
%%% End: 
