%-%-%-%-%-%-%-%-%-%-%-%-%-%-%-%-%-%-%-%-%-%-%-%-%-%-%-%-%-%-%-%-%-%-%-%-%-%-%-%
\subsection{Isabelle/ZF}

Let us introduce briefly Paulson's formalization of ZF
\cite{paulson2017isabelle} in Isabelle and aspects of his formal proof
for the relative consistency of the Axiom of Choice
\cite{paulson_2003}; we will only focus on those aspects that are
essential to keep this paper self-contained, and refer the interested
reader to Paulson's articles. Isabelle/ZF is a theory built upon the
core library FOL of classical first order logic.

In Isabelle/ZF sets are \emph{individuals}, that is terms of type
\isatt{i} and formulas have type \isatt{o}. As in usual presentations
of ZF, Isabelle/ZF has only one relation symbol
\isatt{\ensuremath{\epsilon}} (equality and inclusion are defined
using membership); it also features one constant \isatt{\isadigit{0}}
for the empty set and \isatt{inf} for an infinite set and some other
set constructors, again terms of type \isatt{i}, to ease the
developments. The axioms, being formulas, are terms of type \isatt{o};
the foundation axiom, for example, is formalized as:
\isa{{\isachardoublequoteopen}A\ {\isacharequal}\ {\isadigit{0}}\
  {\isasymor}%
  \ {\isacharparenleft}{\isasymexists}x{\isasymin}A{\isachardot}\ %
  {\isasymforall}y{\isasymin}x{\isachardot}\ %
  y{\isasymnotin}A{\isacharparenright}%
  {\isachardoublequoteclose}}. Besides the axioms, Isabelle/ZF also
introduces several definitions (for example, pairs and sets defined by
comprehension using separation) and syntactic abbreviations to keep
the formalization close to the customary manner of doing mathematics.
Working with the library and extendig it is quite
straightforward. Since we find a need to combine the replacement and
separation schemes to speak of the image of a function over some set
defined by comprehension, we introduced a new term-former:
\begin{isabelle}
  \isacommand{definition}\isamarkupfalse%
  \ SepReplace\ {\isacharcolon}{\isacharcolon}\
  {\isachardoublequoteopen}{\isacharbrackleft}i{\isacharcomma}\
  i{\isasymRightarrow}i{\isacharcomma}\ i{\isasymRightarrow}\
  o{\isacharbrackright}\
  {\isasymRightarrow}i{\isachardoublequoteclose}\
  \isakeyword{where}\isanewline \ \
  {\isachardoublequoteopen}SepReplace{\isacharparenleft}A{\isacharcomma}b{\isacharcomma}Q{\isacharparenright}\
  {\isacharequal}{\isacharequal}\ {\isacharbraceleft}y\ {\isachardot}\
  x{\isasymin}A{\isacharcomma}\
  y{\isacharequal}b{\isacharparenleft}x{\isacharparenright}\
  {\isasymand}\
  Q{\isacharparenleft}x{\isacharparenright}{\isacharbraceright}{\isachardoublequoteclose}
\end{isabelle}
\noindent It is more convenient to add an abbreviation for a simpler
writing and a nicer reading, so we use \isa{{\isacharbraceleft}b\
  {\isachardot}{\isachardot}\ x{\isasymin}A{\isacharcomma}\
  Q{\isacharbraceright}} for
\isa{SepReplace{\isacharparenleft}A{\isacharcomma}b{\isacharcomma}Q{\isacharparenright}}. The
characterization of our new constructor is given by
\begin{isabelle}
\isacommand{lemma}\isamarkupfalse%
\ Sep{\isacharunderscore}and{\isacharunderscore}Replace{\isacharcolon}\ {\isachardoublequoteopen}{\isacharbraceleft}b{\isacharparenleft}x{\isacharparenright}\ {\isachardot}{\isachardot}\ x{\isasymin}A{\isacharcomma}\ P{\isacharparenleft}x{\isacharparenright}\ {\isacharbraceright}\ {\isacharequal}\ {\isacharbraceleft}b{\isacharparenleft}x{\isacharparenright}\ {\isachardot}\ x{\isasymin}{\isacharbraceleft}y{\isasymin}A{\isachardot}\ P{\isacharparenleft}y{\isacharparenright}{\isacharbraceright}{\isacharbraceright}{\isachardoublequoteclose}
\end{isabelle}


% ∀x[c]
% locales


%%% Local Variables:
%%% mode: latex
%%% ispell-local-dictionary: "american"
%%% TeX-master: "first_steps_into_forcing"
%%% End:
