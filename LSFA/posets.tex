\section{Forcing notions}\label{sec:forcing-posets}
%\textit{Para no confundir, directamente usamos este nombre en lugar de
% ``forcing posets'' para los preórdenes con un máximo distinguido}

In this section we introduce some elementary definitions that are used
in the rest of the paper. They are useful to explain some aspects of
Isabelle/ZF.

\begin{definition}
  A preorder on a set $P$ is a binary relation ${\leqslant}$ which is
  reflexive and transitive.
\end{definition}

In Isabelle/ZF sets are \emph{individuals}, that is terms of type
\isatt{i}; since relations are also sets, the relation ${\leqslant}$
also has type \isatt{i}. The other type of Isabelle/ZF is \isatt{o},
which corresponds to formulas; for example the property that
${\leqslant}$ is reflexive (on $P$) is a formula.
\begin{isabelle}
\isacommand{definition}\isamarkupfalse%
\ refl{\isacharunderscore}on\ {\isacharcolon}{\isacharcolon}\ {\isachardoublequoteopen}{\isacharbrackleft}i{\isacharcomma}i{\isacharbrackright}\ {\isasymRightarrow}o{\isachardoublequoteclose}\ \isakeyword{where}\isanewline
\ \ {\isachardoublequoteopen}refl{\isacharunderscore}on{\isacharparenleft}P{\isacharcomma}leq{\isacharparenright}\ {\isacharequal}{\isacharequal}\ {\isasymforall}\ x\ {\isasymin}\ P\ {\isachardot}\ {\isasymlangle}x{\isacharcomma}x{\isasymrangle}\ {\isasymin}\ leq{\isachardoublequoteclose}
\end{isabelle}

Let us remark that the relation is ``untyped'', we cannot assert a
priori that ${\leqslant} \subseteq P \times P$, therefore we need to
restrict the notion of ``reflexive'' with respect to the set $P$.
\begin{definition}
  Given a preorder $(P,\leqslant)$ we say that two elements $p,q$ are
  \emph{compatible} if they have a lower-bound in $P$. Notice that
  the elements of $P$ are also sets, therefore they have type
  \isatt{i}.
  \begin{isabelle}%
  \isacommand{definition}\isamarkupfalse%
\ compat{\isacharunderscore}in\ {\isacharcolon}{\isacharcolon}\ {\isachardoublequoteopen}i{\isasymRightarrow}i{\isasymRightarrow}i{\isasymRightarrow}i{\isasymRightarrow}o{\isachardoublequoteclose}\ \isakeyword{where}\isanewline
\ \ {\isachardoublequoteopen}compat{\isacharunderscore}in{\isacharparenleft}P{\isacharcomma}leq{\isacharcomma}p{\isacharcomma}q{\isacharparenright}\ {\isacharequal}{\isacharequal}\ {\isasymexists}d{\isasymin}P\ {\isachardot}\ {\isasymlangle}d{\isacharcomma}p{\isasymrangle}{\isasymin}leq\ {\isasymand}\ {\isasymlangle}d{\isacharcomma}q{\isasymrangle}{\isasymin}leq{\isachardoublequoteclose}
\end{isabelle}
\end{definition}

A subset $A \subseteq P$ is an \emph{antichain} if every pair of
elements in $A$ are incompatible.
\begin{isabelle}
\isacommand{definition}\isamarkupfalse%
\ antichain\ {\isacharcolon}{\isacharcolon}\ {\isachardoublequoteopen}i{\isasymRightarrow}i{\isasymRightarrow}i{\isasymRightarrow}o{\isachardoublequoteclose}\ \isakeyword{where}\isanewline
\ \ {\isachardoublequoteopen}antichain{\isacharparenleft}P{\isacharcomma}leq{\isacharcomma}A{\isacharparenright}\ {\isacharequal}{\isacharequal}\ A{\isasymsubseteq}P\ {\isasymand}\ {\isacharparenleft}{\isasymforall}p{\isasymin}A{\isachardot}{\isasymforall}q{\isasymin}A{\isachardot}{\isacharparenleft}{\isasymnot}\ compat{\isacharunderscore}in{\isacharparenleft}P{\isacharcomma}leq{\isacharcomma}p{\isacharcomma}q{\isacharparenright}{\isacharparenright}{\isacharparenright}{\isachardoublequoteclose}
\end{isabelle}

A poset $(P,\leqslant)$ has the \emph{countable chain condition} (ccc)
if every antichain in $P$ is countable.
\begin{isabelle}
  \isacommand{definition}\isamarkupfalse%
  \ ccc\ {\isacharcolon}{\isacharcolon}\ {\isachardoublequoteopen}i\ {\isasymRightarrow}\ i\ {\isasymRightarrow}\ o{\isachardoublequoteclose}\ \isakeyword{where}\isanewline
\ \ {\isachardoublequoteopen}ccc{\isacharparenleft}P{\isacharcomma}leq{\isacharparenright}\ {\isacharequal}{\isacharequal}\ {\isasymforall}A{\isachardot}\ antichain{\isacharparenleft}P{\isacharcomma}leq{\isacharcomma}A{\isacharparenright}\ {\isasymlongrightarrow}\ {\isacharbar}A{\isacharbar}\ {\isasymle}\ nat{\isachardoublequoteclose}
\end{isabelle}

\begin{definition}
  A \emph{forcing notion} is a preorder $(P,\leqslant)$ with a maximal element $\mathbbm{1} \in P$.
  \begin{isabelle}
\isacommand{locale}\isamarkupfalse%
\ forcing{\isacharunderscore}poset\ {\isacharequal}\isanewline
\ \ \isakeyword{fixes}\ P\ leq\ one\isanewline
\ \ \isakeyword{assumes}\ one{\isacharunderscore}in{\isacharunderscore}P{\isacharcolon}\ \ \ \ \ \ \ \ \ {\isachardoublequoteopen}one\ {\isasymin}\ P{\isachardoublequoteclose}\isanewline
\ \ \ \ \ \ \isakeyword{and}\ leq{\isacharunderscore}preord{\isacharcolon}\ \ \ \ \ \ \ {\isachardoublequoteopen}preorder{\isacharunderscore}on{\isacharparenleft}P{\isacharcomma}leq{\isacharparenright}{\isachardoublequoteclose}\isanewline
\ \ \ \ \ \ \isakeyword{and}\ one{\isacharunderscore}max{\isacharcolon}\ \ \ \ \ \ \ \ \ \ {\isachardoublequoteopen}{\isasymforall}p{\isasymin}P{\isachardot}\ {\isasymlangle}p{\isacharcomma}one{\isasymrangle}{\isasymin}leq{\isachardoublequoteclose}
\end{isabelle}
\end{definition}
%% REVISAR si queremos hablar de "extensión".
\noindent The locale \isatt{forcing{\isacharunderscore}poset} 

\fbox{Global replace \isatt{poset}$\to$\isatt{notion} ?}

introduces a mathematical
context where we work assuming the forcing notion
$(P,\leqslant, \mathbbm{1})$. It is customary to refer to the elements
of $P$ as \emph{forcing conditions} and think the relation $\leqslant$
as a notion of extension, thus $(p,q) \in \leqslant$ is read as ``$p$
extends $q$''; which is justified because often $\leqslant$ is
instantiated to $\supseteq$. In the following definitions we are in
the locale \isatt{forcing{\isacharunderscore}poset}.

A set $D$ is \emph{dense} if every element $p\in P$ has a lower bound
in $D$ and there is also a weaker definition which asks for a lower
bound in $D$ only for the elements below some fixed element $q$. Since
the relation $\leqslant$ is reflexive, it is obvious that $P$ is
dense.
\begin{isabelle}
  \isacommand{definition}\isamarkupfalse%
\ dense\ {\isacharcolon}{\isacharcolon}\ {\isachardoublequoteopen}i{\isasymRightarrow}o{\isachardoublequoteclose}\ \isakeyword{where}\isanewline
\ \ {\isachardoublequoteopen}dense{\isacharparenleft}D{\isacharparenright}\ {\isacharequal}{\isacharequal}\ {\isasymforall}p{\isasymin}P{\isachardot}\ {\isasymexists}d{\isasymin}D\ {\isachardot}\ {\isasymlangle}d{\isacharcomma}p{\isasymrangle}{\isasymin}leq{\isachardoublequoteclose}\isanewline
\isanewline
\isacommand{definition}\isamarkupfalse%
\ dense{\isacharunderscore}below\ {\isacharcolon}{\isacharcolon}\ {\isachardoublequoteopen}i{\isasymRightarrow}i{\isasymRightarrow}o{\isachardoublequoteclose}\ \isakeyword{where}\isanewline
\ \ {\isachardoublequoteopen}dense{\isacharunderscore}below{\isacharparenleft}D{\isacharcomma}q{\isacharparenright}\ {\isacharequal}{\isacharequal}\ {\isasymforall}p{\isasymin}P{\isachardot}\ {\isasymlangle}p{\isacharcomma}q{\isasymrangle}{\isasymin}leq\ {\isasymlongrightarrow}\ {\isacharparenleft}{\isasymexists}d{\isasymin}D\ {\isachardot}\ {\isasymlangle}d{\isacharcomma}p{\isasymrangle}{\isasymin}leq{\isacharparenright}{\isachardoublequoteclose}\isanewline
\isanewline
\isacommand{lemma}\isamarkupfalse%
\ P{\isacharunderscore}dense{\isacharcolon}\ {\isachardoublequoteopen}dense{\isacharparenleft}P{\isacharparenright}{\isachardoublequoteclose}
\end{isabelle}

We say that $F\subseteq P$ is increasing (or up-set or upward closed) if every
extension of any element in $F$ is also in $F$.
\begin{isabelle}
\isacommand{definition}\isamarkupfalse%
\ increasing\ {\isacharcolon}{\isacharcolon}\ {\isachardoublequoteopen}i{\isasymRightarrow}o{\isachardoublequoteclose}\ \isakeyword{where}\isanewline
\ \ {\isachardoublequoteopen}increasing{\isacharparenleft}F{\isacharparenright}\ {\isacharequal}{\isacharequal}\ {\isasymforall}x{\isasymin}F{\isachardot}\ {\isasymforall}\ p\ {\isasymin}\ P\ {\isachardot}\ {\isasymlangle}x{\isacharcomma}p{\isasymrangle}{\isasymin}leq\ {\isasymlongrightarrow}\ p{\isasymin}F{\isachardoublequoteclose}
\end{isabelle}
A filter is an increasing set $G$ with all its elements being compatible in $G$.
\begin{isabelle}
\isacommand{definition}\isamarkupfalse%
\ filter\ {\isacharcolon}{\isacharcolon}\ {\isachardoublequoteopen}i{\isasymRightarrow}o{\isachardoublequoteclose}\ \isakeyword{where}\isanewline
\ \ {\isachardoublequoteopen}filter{\isacharparenleft}G{\isacharparenright}\ {\isacharequal}{\isacharequal}\ G{\isasymsubseteq}P\ {\isasymand}\ increasing{\isacharparenleft}G{\isacharparenright}\ {\isasymand}\ {\isacharparenleft}{\isasymforall}p{\isasymin}G{\isachardot}\ {\isasymforall}q{\isasymin}G{\isachardot}\ compat{\isacharunderscore}in{\isacharparenleft}G{\isacharcomma}leq{\isacharcomma}p{\isacharcomma}q{\isacharparenright}{\isacharparenright}{\isachardoublequoteclose}
\end{isabelle}

We conclude this section by introducing the upward closure of a set
and proving that the closure of $A$ is a filter if its elements are
compatible in $A$.
\begin{isabelle}
\isacommand{definition}\isamarkupfalse%
\ upclosure\ {\isacharcolon}{\isacharcolon}\ {\isachardoublequoteopen}i{\isasymRightarrow}i{\isachardoublequoteclose}\ \isakeyword{where}\isanewline
\ \ {\isachardoublequoteopen}upclosure{\isacharparenleft}A{\isacharparenright}\ {\isacharequal}{\isacharequal}\ {\isacharbraceleft}p{\isasymin}P{\isachardot}{\isasymexists}a{\isasymin}A{\isachardot}{\isasymlangle}a{\isacharcomma}p{\isasymrangle}{\isasymin}leq{\isacharbraceright}{\isachardoublequoteclose}\isanewline
\isacommand{lemma}\isamarkupfalse%
\ \ closure{\isacharunderscore}compat{\isacharunderscore}filter{\isacharcolon}
\ \ {\isachardoublequoteopen}A{\isasymsubseteq}P\ {\isasymLongrightarrow}\ {\isacharparenleft}{\isasymforall}p{\isasymin}A{\isachardot}{\isasymforall}q{\isasymin}A{\isachardot}\ compat{\isacharunderscore}in{\isacharparenleft}A{\isacharcomma}leq{\isacharcomma}p{\isacharcomma}q{\isacharparenright}{\isacharparenright}\ {\isasymLongrightarrow}\ filter{\isacharparenleft}upclosure{\isacharparenleft}A{\isacharparenright}{\isacharparenright}{\isachardoublequoteclose}
\end{isabelle}

%%% Local Variables:
%%% mode: latex
%%% ispell-local-dictionary: "american"
%%% TeX-master: "first_steps_into_forcing"
%%% End:


