\section{The axioms of replacement and choice}
\label{sec:axioms-replacement-choice}

In our report~\cite{2019arXiv190103313G} we proved that any generic
extension preserves the satisfaction of almost all the axioms,
including the separation scheme (we adapted those proofs to the
current statement of the axiom schemes). Our proofs that Replacement
and choice hold in every generic extension depend on further
relativized concepts and closure properties.

\subsection{Replacement}

The proof of the Replacement Axiom scheme in $M[G]$ in Kunen uses the
Reflection Principle relativized to $M$. We took an alternative
pathway, following Neeman \cite{neeman-course}. In his course notes,
he uses the relativization of the cumulative hierarchy of sets. 

The
family of all sets of rank less than $\alpha$ is called
\isatt{Vset}$(\alpha)$ in Isabelle/ZF. We showed, in the theory file
\verb|Relative_Univ.thy|
 the following
relativization and closure results concerning this function, for a
class $M$ satisfying the locale \isatt{M{\isacharunderscore}eclose}
plus the Powerset Axiom and four instances of replacement.
%
\begin{isabelle}
\isacommand{lemma}\isamarkupfalse%
\ Vset{\isacharunderscore}abs{\isacharcolon}\ {\isachardoublequoteopen}{\isasymlbrakk}\ M{\isacharparenleft}i{\isacharparenright}{\isacharsemicolon}\ M{\isacharparenleft}V{\isacharparenright}{\isacharsemicolon}\ Ord{\isacharparenleft}i{\isacharparenright}\ {\isasymrbrakk}\ {\isasymLongrightarrow}\ \isanewline
\ \ \ \ \ \ \ \  \ \  \ \ \ \ \ \ \ \ \ \ is{\isacharunderscore}Vset{\isacharparenleft}M{\isacharcomma}i{\isacharcomma}V{\isacharparenright}\ {\isasymlongleftrightarrow}\ V\ {\isacharequal}\ {\isacharbraceleft}x{\isasymin}Vset{\isacharparenleft}i{\isacharparenright}{\isachardot}\ M{\isacharparenleft}x{\isacharparenright}{\isacharbraceright}{\isachardoublequoteclose}
\end{isabelle}
\begin{isabelle}
\isacommand{lemma}\isamarkupfalse%
\ Vset{\isacharunderscore}closed{\isacharcolon}\ {\isachardoublequoteopen}{\isasymlbrakk}\ M{\isacharparenleft}i{\isacharparenright}{\isacharsemicolon}\ Ord{\isacharparenleft}i{\isacharparenright}\ {\isasymrbrakk}\ {\isasymLongrightarrow}\ M{\isacharparenleft}{\isacharbraceleft}x{\isasymin}Vset{\isacharparenleft}i{\isacharparenright}{\isachardot}\ M{\isacharparenleft}x{\isacharparenright}{\isacharbraceright}{\isacharparenright}{\isachardoublequoteclose}
\end{isabelle}
We also have the basic result
\begin{isabelle}
\isacommand{lemma}\isamarkupfalse%
\ M{\isacharunderscore}into{\isacharunderscore}Vset{\isacharcolon}\isanewline
\ \ \isakeyword{assumes}\ {\isachardoublequoteopen}M{\isacharparenleft}a{\isacharparenright}{\isachardoublequoteclose}\isanewline
\ \ \isakeyword{shows}\ {\isachardoublequoteopen}{\isasymexists}i{\isacharbrackleft}M{\isacharbrackright}{\isachardot}\ {\isasymexists}V{\isacharbrackleft}M{\isacharbrackright}{\isachardot}\ ordinal{\isacharparenleft}M{\isacharcomma}i{\isacharparenright}\ {\isasymand}\ is{\isacharunderscore}Vfrom{\isacharparenleft}M{\isacharcomma}{\isadigit{0}}{\isacharcomma}i{\isacharcomma}V{\isacharparenright}\ {\isasymand}\ a{\isasymin}V{\isachardoublequoteclose}
\end{isabelle}
stating that $M$ is included in 
$\union\{\isatt{Vset}^M(\alpha) : \alpha\in M\}$ (actually they are equal).

For the proof of the Replacement Axiom, we assume that $\phi$ is
functional in its first two variables when interpreted in $M[G]$ and
the first ranges over the domain \isatt{c}${}\in M[G]$. Then we show
that the collection of
all values of the second variable, when the first ranges over
\isatt{c}, belongs to $M[G]$:
%
\begin{isabelle}
\isacommand{lemma}\isamarkupfalse%
\ Replace{\isacharunderscore}sats{\isacharunderscore}in{\isacharunderscore}MG{\isacharcolon}\isanewline
\ \ \isakeyword{assumes}\isanewline
\ \ \ \ {\isachardoublequoteopen}c{\isasymin}M{\isacharbrackleft}G{\isacharbrackright}{\isachardoublequoteclose}\ {\isachardoublequoteopen}env\ {\isasymin}\ list{\isacharparenleft}M{\isacharbrackleft}G{\isacharbrackright}{\isacharparenright}{\isachardoublequoteclose}\isanewline
\ \ \ \ {\isachardoublequoteopen}{\isasymphi}\ {\isasymin}\ formula{\isachardoublequoteclose}\ {\isachardoublequoteopen}arity{\isacharparenleft}{\isasymphi}{\isacharparenright}\ {\isasymle}\ {\isadigit{2}}\ {\isacharhash}{\isacharplus}\ length{\isacharparenleft}env{\isacharparenright}{\isachardoublequoteclose}\isanewline
\ \ \ \ {\isachardoublequoteopen}univalent{\isacharparenleft}{\isacharhash}{\isacharhash}M{\isacharbrackleft}G{\isacharbrackright}{\isacharcomma}\ c{\isacharcomma}\ {\isasymlambda}x\ v{\isachardot}\ {\isacharparenleft}M{\isacharbrackleft}G{\isacharbrackright}{\isacharcomma}\ {\isacharbrackleft}x{\isacharcomma}v{\isacharbrackright}{\isacharat}env\ {\isasymTurnstile}\ {\isasymphi}{\isacharparenright}{\isacharparenright}{\isachardoublequoteclose}\isanewline
\ \ \isakeyword{shows}\isanewline
\ \ \ \ {\isachardoublequoteopen}{\isacharbraceleft}v{\isachardot}\ x{\isasymin}c{\isacharcomma}\ v{\isasymin}M{\isacharbrackleft}G{\isacharbrackright}\ {\isasymand}\ {\isacharparenleft}M{\isacharbrackleft}G{\isacharbrackright}{\isacharcomma}\ {\isacharbrackleft}x{\isacharcomma}v{\isacharbrackright}{\isacharat}env\ {\isasymTurnstile}\ {\isasymphi}{\isacharparenright}{\isacharbraceright}\ {\isasymin}\ M{\isacharbrackleft}G{\isacharbrackright}{\isachardoublequoteclose}
\end{isabelle}
%
From this, the satisfaction of the Replacement Axiom in $M[G]$ follows
very easily.

The proof of the previous lemma, following Neeman, proceeds as usual
by turning an argument concerning elements of $M[G]$ to one involving
names lying in $M$, and connecting both worlds by using the forcing
theorems. In the case at hand, by functionality of $\phi$ we know that
for every $x\in c\cap M[G]$ there exists exactly one $v\in M[G]$ such
that
$M[G], [x,v]\mathbin{@} \mathit{env} \models \phi$. Now,
given a name $\pi'\in M$ for $c$, every name of an element of $c$
belongs to $\pi\defi \dom(\pi')\times \PP$, which is easily seen to be
in $M$. We will use $\pi$ to be the domain in an application of the
Replacement Axiom in $M$. But now, obviously, we have lost
functionality since there are many names $\dot v\in M$ for a fixed $v$
in $M[G]$. To solve this issue, for each $\rho p \defi\lb\rho,p\rb\in
\pi$ we calculate the
minimum rank of some $\tau\in M$ such that 
$p\forces \phi(\rho,\tau,\dots)$ if there is one, or $0$ otherwise. By
Replacement in $M$, we can show that the supremum \isatt{?sup} of these ordinals
belongs to $M$ and we can construct a \isatt{?bigname} $\defi$ 
\isatt{{\isacharbraceleft}x{\isasymin}Vset{\isacharparenleft}{\isacharquery}sup{\isacharparenright}{\isachardot}\ x\ {\isasymin}\
}$M$\isatt{{\isacharbraceright}\ {\isasymtimes}\ {\isacharbraceleft}one{\isacharbraceright}}
whose interpretation by (any generic) $G$ will include all possible elements
as $v$ above.

The previous calculation required some absoluteness and closure
results regarding the minimum ordinal binder, \isatt{Least}$(Q)$, also
denoted $\mu x. Q(x)$, that can be found in the theory file
\verb|Least.thy|.

\subsection{Choice}
A first important observation is that the proof of $\AC$ in $M[G]$
only requires the assumption that $M$ satisfies (a finite fragment of)
$\ZFC$. There is no need to invoke Choice in the metatheory.

Although our previous version of the development used $\AC$, that was
only needed to show the Rasiowa-Sikorski Lemma (RSL) for
arbitrary posets. We have modularized the proof of the latter
and now the version for countable posets that we use to show the
existence of generic filters
does not require Choice (as it is well known). We also bundled the
full RSL along with our implementation of the principle of dependent
choices in an independent branch of the dependency graph, which is the
only place where the theory \texttt{ZF.AC} is invoked.

Our statement of the Axiom of Choice is the one preferred for
arguments involving transitive classes satisfying $\ZF$:
%
\begin{center}
\isatt{{\isasymforall}x{\isacharbrackleft}M{\isacharbrackright}{\isachardot}\ {\isasymexists}a{\isacharbrackleft}M{\isacharbrackright}{\isachardot}\ {\isasymexists}f{\isacharbrackleft}M{\isacharbrackright}{\isachardot}\ ordinal{\isacharparenleft}M{\isacharcomma}a{\isacharparenright}\ {\isasymand}\ surjection{\isacharparenleft}M{\isacharcomma}a{\isacharcomma}x{\isacharcomma}f{\isacharparenright}}
\end{center}
%
The Simplifier is able to show automatically that this
statement is equivalent to the next one, in which the real notions of
ordinal and surjection appear:
%
\begin{center}
\isatt{{\isasymforall}x{\isacharbrackleft}M{\isacharbrackright}{\isachardot}\ {\isasymexists}a{\isacharbrackleft}M{\isacharbrackright}{\isachardot}\ {\isasymexists}f{\isacharbrackleft}M{\isacharbrackright}{\isachardot}\ Ord{\isacharparenleft}a{\isacharparenright}\ {\isasymand}\ f\ {\isasymin}\ surj{\isacharparenleft}a{\isacharcomma}x{\isacharparenright}}
\end{center}

As with the forcing axioms, the proof of $\AC$ in $M[G]$ follows the pattern of Kunen
\cite[IV.2.27]{kunen2011set} and is rather
straightforward; the only complicated technical point being to show
that the relevant name belongs to $M$. We assume that \isatt{a}${}\neq\emptyset$
belongs to $M[G]$ and has a name $\tau\in M$. By $\AC$ in $M$, there
is a surjection \isatt{s} from an ordinal $\alpha\in M$ ($\subseteq M[G]$) onto
$\dom(\tau)$. Now
%
\begin{center}
\isatt{{\isacharbraceleft}opair{\isacharunderscore}name{\isacharparenleft}check{\isacharparenleft}{\isasymbeta}{\isacharparenright}{\isacharcomma}s{\isacharbackquote}{\isasymbeta}{\isacharparenright}{\isachardot}\ {\isasymbeta}{\isasymin}{\isasymalpha}{\isacharbraceright}\ {\isasymtimes}\ {\isacharbraceleft}one{\isacharbraceright}}
\end{center}
%
is a name for a function \isatt{f} with domain $\alpha$ such that \isatt{a}
is included in its range, and where
\isatt{opair{\isacharunderscore}name}$(\sig,\rho)$ is a name for the
ordered pair $\lb\val(G,\sig),\val(G,\rho)\rb$. From this, $\AC$ in
$M[G]$ follows easily.

\subsection{The main theorem}
With all these elements in place, we are able to transcript the main
theorem of our formalization:
\begin{isabelle}
\isacommand{theorem}\isamarkupfalse%
\ extensions{\isacharunderscore}of{\isacharunderscore}ctms{\isacharcolon}\isanewline
\ \ \isakeyword{assumes}\ \isanewline
\ \ \ \ {\isachardoublequoteopen}M\ {\isasymapprox}\ nat{\isachardoublequoteclose}\ {\isachardoublequoteopen}Transset{\isacharparenleft}M{\isacharparenright}{\isachardoublequoteclose}\ {\isachardoublequoteopen}M\ {\isasymTurnstile}\ ZF{\isachardoublequoteclose}\isanewline
\ \ \isakeyword{shows}\ \isanewline
\ \ \ \ {\isachardoublequoteopen}{\isasymexists}N{\isachardot}\ \isanewline
\ \ \ \ \ \ M\ {\isasymsubseteq}\ N\ {\isasymand}\ N\ {\isasymapprox}\ nat\ {\isasymand}\ Transset{\isacharparenleft}N{\isacharparenright}\ {\isasymand}\ N\ {\isasymTurnstile}\ ZF\ {\isasymand}\ M{\isasymnoteq}N\ {\isasymand}\isanewline
\ \ \ \ \ \ {\isacharparenleft}{\isasymforall}{\isasymalpha}{\isachardot}\ Ord{\isacharparenleft}{\isasymalpha}{\isacharparenright}\ {\isasymlongrightarrow}\ {\isacharparenleft}{\isasymalpha}\ {\isasymin}\ M\ {\isasymlongleftrightarrow}\ {\isasymalpha}\ {\isasymin}\ N{\isacharparenright}{\isacharparenright}\ {\isasymand}\isanewline
\ \ \ \ \ \ {\isacharparenleft}M{\isacharcomma}\ {\isacharbrackleft}{\isacharbrackright}{\isasymTurnstile}\ AC\ {\isasymlongrightarrow}\ N\ {\isasymTurnstile}\ ZFC{\isacharparenright}{\isachardoublequoteclose}
\end{isabelle}
Here, \isatt{\isasymapprox} stands for equipotency, \isatt{nat} is the
set of natural numbers, and the predicate 
\isatt{Transset} indicates transitivity; and as usual, \isatt{AC}
denotes the Axiom of Choice, and \isatt{ZF} and \isatt{ZFC} the
corresponding subsets of \isatt{formula}.

%%% Local Variables: 
%%% mode: latex
%%% TeX-master: "forcing_in_isabelle_zf"
%%% ispell-local-dictionary: "american"
%%% End: 
