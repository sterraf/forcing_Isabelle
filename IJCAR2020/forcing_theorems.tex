\section{The forcing theorems}

After the definition of $\forceisa$ is complete, the proof of the
Fundamental Theorems of Forcing is comparatively straightforward, and
we were able to follow Kunen very closely. The more involved points of
this part of the development were those where we needed to proved that
various (dense) subsets of $\PP$ were in $M$; for this, we had to
recourse to several absoluteness ad-hoc lemmas.

The first results concern characterizations of the forcing
relation. Two of them are \isatt{Forces{\isacharunderscore}Member}:
\begin{center}
  \isatt{{\isacharparenleft}p\ {\isasymtturnstile}\ Member{\isacharparenleft}n{\isacharcomma}m{\isacharparenright}\ env{\isacharparenright}\ {\isasymlongleftrightarrow}\ forces{\isacharunderscore}mem{\isacharparenleft}p{\isacharcomma}t{\isadigit{1}}{\isacharcomma}t{\isadigit{2}}{\isacharparenright}},
\end{center}
where \isatt{t{\isadigit{1}}} and \isatt{t{\isadigit{1}}} are the
\isatt{n}th resp.\ \isatt{m}th elements of \isatt{env}, and  \isatt{Forces{\isacharunderscore}Forall}:
\begin{center}
  \isatt{{\isacharparenleft}p\ {\isasymtturnstile}\ Forall{\isacharparenleft}{\isasymphi}{\isacharparenright}\ env{\isacharparenright}\ {\isasymlongleftrightarrow}\ {\isacharparenleft}{\isasymforall}x{\isasymin}M{\isachardot}\ {\isacharparenleft}p\ {\isasymtturnstile}\ {\isasymphi}\ {\isacharparenleft}{\isacharbrackleft}x{\isacharbrackright}\ {\isacharat}\ env{\isacharparenright}{\isacharparenright}{\isacharparenright}}.
\end{center}
These two, along with  \isatt{Forces{\isacharunderscore}Equal} and
\isatt{Forces{\isacharunderscore}Nand}, appear in Kunen as the
inductive definition of the forcing relation \cite[Def.~IV.2.42]{kunen2011set}.

As with the previous section, the proofs of the forcing theorems have two different
flavours: The ones for the atomic formulas proceed by using the
principle of 
\isatt{forces\_induction}, and then an induction on
$\formula$ wraps the former with the remaining cases (\isatt{Nand} and \isatt{Forall}). 

As an example of the first class, we can take a look at our
formalization of \cite[Lem.~IV.2.40(a)]{kunen2011set}:

\begin{isabelle}
  \isacommand{lemma}\isamarkupfalse%
  \ IV{\isadigit{2}}{\isadigit{4}}{\isadigit{0}}a{\isacharcolon}\isanewline
  \ \ \isakeyword{assumes}\isanewline
  \ \ \ \ {\isachardoublequoteopen}M{\isacharunderscore}generic{\isacharparenleft}G{\isacharparenright}{\isachardoublequoteclose}\isanewline
  \ \ \isakeyword{shows}\ \isanewline
  \ \ \ \ {\isachardoublequoteopen}{\isacharparenleft}{\isasymtau}{\isasymin}M{\isasymlongrightarrow}{\isasymtheta}{\isasymin}M{\isasymlongrightarrow}{\isacharparenleft}{\isasymforall}p{\isasymin}G{\isachardot}forces{\isacharunderscore}eq{\isacharparenleft}p{\isacharcomma}{\isasymtau}{\isacharcomma}{\isasymtheta}{\isacharparenright}{\isasymlongrightarrow}val{\isacharparenleft}G{\isacharcomma}{\isasymtau}{\isacharparenright}{\isacharequal}val{\isacharparenleft}G{\isacharcomma}{\isasymtheta}{\isacharparenright}{\isacharparenright}{\isacharparenright}{\isasymand}\isanewline
  \ \ \ \ \ {\isacharparenleft}{\isasymtau}{\isasymin}M{\isasymlongrightarrow}{\isasymtheta}{\isasymin}M{\isasymlongrightarrow}{\isacharparenleft}{\isasymforall}p{\isasymin}G{\isachardot}forces{\isacharunderscore}mem{\isacharparenleft}p{\isacharcomma}{\isasymtau}{\isacharcomma}{\isasymtheta}{\isacharparenright}{\isasymlongrightarrow}val{\isacharparenleft}G{\isacharcomma}{\isasymtau}{\isacharparenright}{\isasymin}val{\isacharparenleft}G{\isacharcomma}{\isasymtheta}{\isacharparenright}{\isacharparenright}{\isacharparenright}{\isachardoublequoteclose}
\end{isabelle}
%
Its proof starts by an introduction of \isatt{forces\_induction};
the  inductive cases for each atomic type were handled before as
separate lemmas (\isatt{IV240a\_mem} and \isatt{IV240a\_eq}). We
illustrate with the statement of the latter.
%
\begin{isabelle}
\isacommand{lemma}\isamarkupfalse%
\ IV{\isadigit{2}}{\isadigit{4}}{\isadigit{0}}a{\isacharunderscore}eq{\isacharcolon}\isanewline
\ \ \isakeyword{assumes}\isanewline
\ \ \ \ {\isachardoublequoteopen}M{\isacharunderscore}generic{\isacharparenleft}G{\isacharparenright}{\isachardoublequoteclose}\ {\isachardoublequoteopen}p{\isasymin}G{\isachardoublequoteclose}\ {\isachardoublequoteopen}forces{\isacharunderscore}eq{\isacharparenleft}p{\isacharcomma}{\isasymtau}{\isacharcomma}{\isasymtheta}{\isacharparenright}{\isachardoublequoteclose}\isanewline
\ \ \ \ \isakeyword{and}\isanewline
\ \ \ \ IH{\isacharcolon}{\isachardoublequoteopen}{\isasymAnd}q\ {\isasymsigma}{\isachardot}\ q{\isasymin}P\ {\isasymLongrightarrow}\ q{\isasymin}G\ {\isasymLongrightarrow}\ {\isasymsigma}{\isasymin}domain{\isacharparenleft}{\isasymtau}{\isacharparenright}\ {\isasymunion}\ domain{\isacharparenleft}{\isasymtheta}{\isacharparenright}\ {\isasymLongrightarrow}\ \isanewline
\ \ \ \ \ \ \ \ {\isacharparenleft}forces{\isacharunderscore}mem{\isacharparenleft}q{\isacharcomma}{\isasymsigma}{\isacharcomma}{\isasymtau}{\isacharparenright}\ {\isasymlongrightarrow}\ val{\isacharparenleft}G{\isacharcomma}{\isasymsigma}{\isacharparenright}\ {\isasymin}\ val{\isacharparenleft}G{\isacharcomma}{\isasymtau}{\isacharparenright}{\isacharparenright}\ {\isasymand}\isanewline
\ \ \ \ \ \ \ \ {\isacharparenleft}forces{\isacharunderscore}mem{\isacharparenleft}q{\isacharcomma}{\isasymsigma}{\isacharcomma}{\isasymtheta}{\isacharparenright}\ {\isasymlongrightarrow}\ val{\isacharparenleft}G{\isacharcomma}{\isasymsigma}{\isacharparenright}\ {\isasymin}\ val{\isacharparenleft}G{\isacharcomma}{\isasymtheta}{\isacharparenright}{\isacharparenright}{\isachardoublequoteclose}\isanewline
\ \ \isakeyword{shows}\isanewline
\ \ \ \ {\isachardoublequoteopen}val{\isacharparenleft}G{\isacharcomma}{\isasymtau}{\isacharparenright}\ {\isacharequal}\ val{\isacharparenleft}G{\isacharcomma}{\isasymtheta}{\isacharparenright}{\isachardoublequoteclose}
\end{isabelle}

As an example of the second kind of induction (on formulas), we choose the
following relatively simple result:

\begin{isabelle}
\isacommand{lemma}\isamarkupfalse%
\ strengthening{\isacharunderscore}lemma{\isacharcolon}\isanewline
\ \ \isakeyword{assumes}\ \isanewline
\ \ \ \ {\isachardoublequoteopen}p{\isasymin}P{\isachardoublequoteclose}\ {\isachardoublequoteopen}{\isasymphi}{\isasymin}formula{\isachardoublequoteclose}\ {\isachardoublequoteopen}r{\isasymin}P{\isachardoublequoteclose}\ {\isachardoublequoteopen}r{\isasympreceq}p{\isachardoublequoteclose}\isanewline
\ \ \isakeyword{shows}\isanewline
\ \ \ \ {\isachardoublequoteopen}{\isasymAnd}env{\isachardot}\ env{\isasymin}list{\isacharparenleft}M{\isacharparenright}\ {\isasymLongrightarrow}\ arity{\isacharparenleft}{\isasymphi}{\isacharparenright}{\isasymle}length{\isacharparenleft}env{\isacharparenright}\ {\isasymLongrightarrow}\ p\ {\isasymtturnstile}\ {\isasymphi}\ env\isanewline 
\ \ \ \ \ {\isasymLongrightarrow}\ r\ {\isasymtturnstile}\ {\isasymphi}\ env{\isachardoublequoteclose}\isanewline
%
%
\isacommand{using}\isamarkupfalse%
\ assms{\isacharparenleft}{\isadigit{2}}{\isacharparenright}
\end{isabelle}
%
The proof is divided in the 4 cases of definition of an element of $\formula$,
%
\begin{isabelle}
\isacommand{proof}\isamarkupfalse%
\ {\isacharparenleft}induct{\isacharparenright}\isanewline
\ \ \isacommand{case}\isamarkupfalse%
\ {\isacharparenleft}Member\ n\ m{\isacharparenright}\isanewline
\ \ \isacommand{then}\isamarkupfalse%
\isanewline
\ \ \dots
\isanewline
\ \ \isacommand{show}\isamarkupfalse%
\ {\isacharquery}case\ \isanewline
\ \ \ \ \isacommand{using}\isamarkupfalse%
\ Forces{\isacharunderscore}Member{\isacharbrackleft}of\ {\isacharunderscore}\ {\isachardoublequoteopen}nth{\isacharparenleft}n{\isacharcomma}env{\isacharparenright}{\isachardoublequoteclose}\ {\isachardoublequoteopen}nth{\isacharparenleft}m{\isacharcomma}env{\isacharparenright}{\isachardoublequoteclose}\ env\ n\ m{\isacharbrackright}\isanewline
\ \ \ \ \ \ strengthening{\isacharunderscore}mem{\isacharbrackleft}of\ p\ r\ {\isachardoublequoteopen}nth{\isacharparenleft}n{\isacharcomma}env{\isacharparenright}{\isachardoublequoteclose}\ {\isachardoublequoteopen}nth{\isacharparenleft}m{\isacharcomma}env{\isacharparenright}{\isachardoublequoteclose}{\isacharbrackright}\ \isacommand{by}\isamarkupfalse%
\ simp
\end{isabelle}
%
where the final step depends on previously proved
\isatt{strengthening{\isacharunderscore}mem} and the characterization of
$\forceisa$ for membership 


The case of equality is entirely analogous, and the \isatt{Nand} and
\isatt{Forall} cases are handled very simply.
%
\begin{isabelle}
\isacommand{next}\isamarkupfalse%
\isanewline
\ \ \isacommand{case}\isamarkupfalse%
\ {\isacharparenleft}Equal\ n\ m{\isacharparenright}\isanewline
\ \ \dots\isanewline
\isacommand{next}\isamarkupfalse%
\isanewline
\ \ \isacommand{case}\isamarkupfalse%
\ {\isacharparenleft}Nand\ {\isasymphi}\ {\isasympsi}{\isacharparenright}\isanewline
\ \ \isacommand{with}\isamarkupfalse%
\ assms\isanewline
\ \ \isacommand{show}\isamarkupfalse%
\ {\isacharquery}case\ \isanewline
\ \ \ \ \isacommand{using}\isamarkupfalse%
\ Forces{\isacharunderscore}Nand\ Transset{\isacharunderscore}intf{\isacharbrackleft}OF\ trans{\isacharunderscore}M\ {\isacharunderscore}\ P{\isacharunderscore}in{\isacharunderscore}M{\isacharbrackright}\ pair{\isacharunderscore}in{\isacharunderscore}M{\isacharunderscore}iff\isanewline
\ \ \ \ \ \ Transset{\isacharunderscore}intf{\isacharbrackleft}OF\ trans{\isacharunderscore}M\ {\isacharunderscore}\ leq{\isacharunderscore}in{\isacharunderscore}M{\isacharbrackright}\ leq{\isacharunderscore}transD\ \isacommand{by}\isamarkupfalse%
\ auto\isanewline
\isacommand{next}\isamarkupfalse%
\isanewline
\ \ \isacommand{case}\isamarkupfalse%
\ {\isacharparenleft}Forall\ {\isasymphi}{\isacharparenright}\isanewline
\ \ \isacommand{with}\isamarkupfalse%
\ assms\isanewline
\ \ \isacommand{have}\isamarkupfalse%
\ {\isachardoublequoteopen}p\ {\isasymtturnstile}\ {\isasymphi}\ {\isacharparenleft}{\isacharbrackleft}x{\isacharbrackright}\ {\isacharat}\ env{\isacharparenright}{\isachardoublequoteclose}\ \isakeyword{if}\ {\isachardoublequoteopen}x{\isasymin}M{\isachardoublequoteclose}\ \isakeyword{for}\ x\isanewline
\ \ \ \ \isacommand{using}\isamarkupfalse%
\ that\ Forces{\isacharunderscore}Forall\ \isacommand{by}\isamarkupfalse%
\ simp\isanewline
\ \ \isacommand{with}\isamarkupfalse%
\ \underline{Forall}\ \isanewline
\ \ \isacommand{have}\isamarkupfalse%
\ {\isachardoublequoteopen}r\ {\isasymtturnstile}\ {\isasymphi}\ {\isacharparenleft}{\isacharbrackleft}x{\isacharbrackright}\ {\isacharat}\ env{\isacharparenright}{\isachardoublequoteclose}\ \isakeyword{if}\ {\isachardoublequoteopen}x{\isasymin}M{\isachardoublequoteclose}\ \isakeyword{for}\ x\isanewline
\ \ \ \ \isacommand{using}\isamarkupfalse%
\ that\ pred{\isacharunderscore}le{\isadigit{2}}\ \isacommand{by}\isamarkupfalse%
\ {\isacharparenleft}simp{\isacharparenright}\isanewline
\ \ \isacommand{with}\isamarkupfalse%
\ assms\ \underline{Forall}\isanewline
\ \ \isacommand{show}\isamarkupfalse%
\ {\isacharquery}case\ \isanewline
\ \ \ \ \isacommand{using}\isamarkupfalse%
\ Forces{\isacharunderscore}Forall\ \isacommand{by}\isamarkupfalse%
\ simp\isanewline
\isacommand{qed}\isamarkupfalse
\end{isabelle}
%
It can be noted that the inductive hypothesis
gets used in the last case (underlined here as
\isatt{\underline{Forall}}), but not in the case for \isatt{Nand}.



%%% Local Variables: 
%%% mode: latex
%%% TeX-master: "forcing_in_isabelle_zf"
%%% ispell-local-dictionary: "american"
%%% End: 
