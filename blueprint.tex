\documentclass[11pt,english]{article}
\usepackage[bottom=2cm, top=3cm, %% Comment if 
  left=4cm, right=4cm]{geometry}             %% using 'ebook' style 
\usepackage{framed}

\usepackage[utf8]{inputenc}
\usepackage{amsmath}
\usepackage{amsthm}
\usepackage{amsfonts}
\usepackage{amssymb}
%\usepackage{bbm}  % Para el \bb{1}
\usepackage[numbers]{natbib}
\usepackage{enumitem}
\usepackage{babel}
%\usepackage{babelbib}
\usepackage{multidef}
\usepackage{verbatim}
\usepackage{stmaryrd} %% para \llbracket
%%
%% \usepackage[bottom=2cm, top=2cm, left=2cm, right=2cm]{geometry}
%% \usepackage{titling}
%% \setlength{\droptitle}{-10ex} 
%%
\renewcommand{\o}{\vee}
\renewcommand{\O}{\bigvee}
\newcommand{\y}{\wedge}
\newcommand{\Y}{\bigwedge}
\newcommand{\limp}{\rightarrow}
\newcommand{\lsii}{\leftrightarrow}
%%

\DeclareMathOperator{\cf}{cf}
\DeclareMathOperator{\dom}{dom}
\DeclareMathOperator{\im}{img}
\DeclareMathOperator{\Fn}{Fn}
\DeclareMathOperator{\rk}{rk}
\DeclareMathOperator{\mos}{mos}
\DeclareMathOperator{\trcl}{trcl}
\DeclareMathOperator*{\diag}{\bigtriangleup}
\DeclareMathOperator{\Con}{Con}
\DeclareMathOperator{\Club}{Club}


\newcommand{\modelo}[1]{\mathbf{#1}}
\newcommand{\axiomas}[1]{\mathit{#1}}
\newcommand{\clase}[1]{\mathsf{#1}}
\newcommand{\poset}[1]{\mathbb{#1}}
\newcommand{\operador}[1]{\mathbf{#1}}

%% \newcommand{\Lim}{\clase{Lim}}
%% \newcommand{\Reg}{\clase{Reg}}
%% \newcommand{\Card}{\clase{Card}}
%% \newcommand{\On}{\clase{On}}
%% \newcommand{\WF}{\clase{WF}}
%% \newcommand{\HF}{\clase{HF}}
%% \newcommand{\HC}{\clase{HC}}
%%
%% El siguiente comando reemplaza todos los anteriores:
%%
\multidef{\clase{#1}}{Card,HC,HF,Lim,On->Ord,Reg,WF,Ord}
\newcommand{\ON}{\On}

%% En lugar de usar todo el paquete bbm:
\DeclareMathAlphabet{\mathbbm}{U}{bbm}{m}{n} 
\newcommand{\1}{\mathbbm{1}}

%%
%% \newcommand{\calD}{\mathcal{D}}
%% \newcommand{\calS}{\mathcal{S}}
%% \newcommand{\calU}{\mathcal{U}}
%% \newcommand{\calB}{\mathcal{B}}
%% \newcommand{\calL}{\mathcal{L}}
%% \newcommand{\calF}{\mathcal{F}}
%% \newcommand{\calT}{\mathcal{T}}
%% \newcommand{\calW}{\mathcal{W}}
%% \newcommand{\calA}{\mathcal{A}}
%%
%% El siguiente comando reemplaza todos los anteriores:
%%
\multidef[prefix=cal]{\mathcal{#1}}{A-Z}
%%
%% \newcommand{\A}{\modelo{A}}
%% \newcommand{\BB}{\modelo{B}}
%% \newcommand{\ZZ}{\modelo{Z}}
%% \newcommand{\PP}{\modelo{P}}
%% \newcommand{\QQ}{\modelo{Q}}
%% \newcommand{\RR}{\modelo{R}}
%%
%% El siguiente comando reemplaza todos los anteriores:
%%
\multidef{\modelo{#1}}{A,BB->B,CC->C,NN->N,PP->P,QQ->Q,RR->R,ZZ->Z}

\multidef[prefix=p]{\mathbb{#1}}{A-Z}
%% \newcommand{\B}{\modelo{B}}
%% \newcommand{\C}{\modelo{C}}
%% \newcommand{\F}{\modelo{F}}
%% \newcommand{\D}{\modelo{D}}

\newcommand{\Th}{\mb{Th}}
\newcommand{\Mod}{\mb{Mod}}

\newcommand{\Se}{\operador{S^\prec}}
\newcommand{\Pu}{\operador{P_u}}
\renewcommand{\Pr}{\operador{P_R}}
\renewcommand{\H}{\operador{H}}
\renewcommand{\S}{\operador{S}}
\newcommand{\I}{\operador{I}}
\newcommand{\E}{\operador{E}}

\newcommand{\se}{\preccurlyeq}
\newcommand{\ee}{\succ}
\newcommand{\id}{\approx}
\newcommand{\subm}{\subseteq}
\newcommand{\ext}{\supseteq}
\newcommand{\iso}{\cong}
%%
\renewcommand{\emptyset}{\varnothing}
\newcommand{\rel}{\mathcal{R}}
\newcommand{\Pow}{\mathop{\mathcal{P}}}
\renewcommand{\P}{\Pow}
\newcommand{\BP}{\mathrm{BP}}
\newcommand{\func}{\rightarrow}
\newcommand{\ord}{\mathrm{Ord}}
\newcommand{\R}{\mathbb{R}}
\newcommand{\N}{\mathbb{N}}
\newcommand{\Z}{\mathbb{Z}}
\renewcommand{\I}{\mathbb{I}}
\newcommand{\Q}{\mathbb{Q}}
\newcommand{\B}{\mathbf{B}}
\newcommand{\<}{\langle}
\renewcommand{\>}{\rangle}
\newcommand{\lb}{\langle}
\newcommand{\rb}{\rangle}
\newcommand{\impl}{\rightarrow}
\newcommand{\ent}{\Rightarrow}
\newcommand{\tne}{\Leftarrow}
\newcommand{\sii}{\Leftrightarrow}
\renewcommand{\phi}{\varphi}
\newcommand{\phis}{{\varphi^*}}
\renewcommand{\th}{\theta}
\newcommand{\Lda}{\Lambda}
\newcommand{\La}{\Lambda}
\newcommand{\lda}{\lambda}
\newcommand{\ka}{\kappa}
\newcommand{\del}{\delta}
\newcommand{\de}{\delta}
\newcommand{\ze}{\zeta}
%\newcommand{\ }{\ }
\newcommand{\la}{\lambda}
\newcommand{\al}{\alpha}
\newcommand{\be}{\beta}
\newcommand{\ga}{\gamma}
\newcommand{\Ga}{\Gamma}
\newcommand{\ep}{\varepsilon}
\newcommand{\De}{\Delta}
\newcommand{\defi}{\mathrel{\mathop:}=}
\newcommand{\forces}{\Vdash}
%\newcommand{\ap}{\mathbin{\wideparen{\ }}}
\newcommand{\Tree}{{\mathrm{Tr}_\N}}
\newcommand{\PTree}{{\mathrm{PTr}_\N}}
\newcommand{\NWO}{\mathit{NWO}}
\newcommand{\Suc}{{\N^{<\N}}}%
\newcommand{\init}{\mathsf{i}}
\newcommand{\ap}{\mathord{^\smallfrown}}
\newcommand{\Cantor}{\mathcal{C}}
%\newcommand{\C}{\Cantor}
\newcommand{\Baire}{\mathcal{N}}
\newcommand{\sig}{\ensuremath{\sigma}}
\newcommand{\fsig}{\ensuremath{F_\sigma}}
\newcommand{\gdel}{\ensuremath{G_\delta}}
\newcommand{\Sig}{\ensuremath{\boldsymbol{\Sigma}}}
\newcommand{\bPi}{\ensuremath{\boldsymbol{\Pi}}}
\newcommand{\Del}{\ensuremath{\boldsymbol\Delta}}
%\renewcommand{\F}{\operador{F}}
\newcommand{\ths}{{\theta^*}}
\newcommand{\om}{\ensuremath{\omega}}
%\renewcommand{\c}{\complement}
\newcommand{\comp}{\mathsf{c}}
\newcommand{\co}[1]{\left(#1\right)^\comp}
\newcommand{\len}[1]{\left|#1\right|}
\DeclareMathOperator{\tlim}{\overline{\mathrm{TLim}}}
\newcommand{\card}[1]{{\left|#1\right|}}
\newcommand{\bigcard}[1]{{\bigl|#1\bigr|}}
%
% Cardinality
%
\newcommand{\lec}{\leqslant_c}
\newcommand{\gec}{\geqslant_c}
\newcommand{\lc}{<_c}
\newcommand{\gc}{>_c}
\newcommand{\eqc}{=_c}
\newcommand{\biy}{\approx}
\newcommand*{\ale}[1]{\aleph_{#1}}
%
\newcommand{\Zerm}{\axiomas{Z}}
\newcommand{\ZC}{\axiomas{ZC}}
\newcommand{\AC}{\axiomas{AC}}
\newcommand{\DC}{\axiomas{DC}}
\newcommand{\MA}{\axiomas{MA}}
\newcommand{\CH}{\axiomas{CH}}
\newcommand{\ZFC}{\axiomas{ZFC}}
\newcommand{\ZF}{\axiomas{ZF}}
\newcommand{\Inf}{\axiomas{Inf}}
%
% Cardinal characteristics
%
\newcommand{\cont}{\mathfrak{c}}
\newcommand{\spl}{\mathfrak{s}}
\newcommand{\bound}{\mathfrak{b}}
\newcommand{\mad}{\mathfrak{a}}
\newcommand{\tower}{\mathfrak{t}}
%
\renewcommand{\hom}[2]{{}^{#1}\hskip-0.116ex{#2}}
\newcommand{\pred}[1][{}]{\mathop{\mathrm{pred}_{#1}}}
%% Postfix operator with supressable space:
%% \newcommand*{\iseg}{\relax\ifnum\lastnodetype>0 \mskip\medmuskip\fi{\downarrow}} %
\newcommand*{\iseg}{{\downarrow}}
\newcommand{\rr}{\mathrel{R}}
\newcommand{\restr}{\upharpoonright}
%\newcommand{\type}{\mathtt{}}
\newcommand{\app}{\mathop{\mathrm{Aprox}}}
\newcommand{\hess}{\triangleleft}
\newcommand{\bx}{\bar{x}}
\newcommand{\by}{\bar{y}}
\newcommand{\bz}{\bar{z}}
\newcommand{\union}{\mathop{\textstyle\bigcup}}
\newcommand{\sm}{\setminus}
\newcommand{\sbq}{\subseteq}
\newcommand{\nsbq}{\subseteq}
\newcommand{\mty}{\emptyset}
\newcommand{\dimg}{\text{\textup{``}}} % direct image
\newcommand{\quine}[1]{\ulcorner{\!#1\!}\urcorner}
%\newcommand{\ntrm}[1]{\textsl{\textbf{#1}}}
\newcommand{\Null}{\calN\!\mathit{ull}}
\DeclareMathOperator{\club}{Club}
\DeclareMathOperator{\otp}{otp}

%%%%%%%%%%%%%%%%%%%%%%%%%
% Variant aleph, beth, etc
% From http://tex.stackexchange.com/q/170476/69595
\makeatletter
\@ifpackageloaded{txfonts}\@tempswafalse\@tempswatrue
\if@tempswa
  \DeclareFontFamily{U}{txsymbols}{}
  \DeclareFontFamily{U}{txAMSb}{}
  \DeclareSymbolFont{txsymbols}{OMS}{txsy}{m}{n}
  \SetSymbolFont{txsymbols}{bold}{OMS}{txsy}{bx}{n}
  \DeclareFontSubstitution{OMS}{txsy}{m}{n}
  \DeclareSymbolFont{txAMSb}{U}{txsyb}{m}{n}
  \SetSymbolFont{txAMSb}{bold}{U}{txsyb}{bx}{n}
  \DeclareFontSubstitution{U}{txsyb}{m}{n}
  \DeclareMathSymbol{\aleph}{\mathord}{txsymbols}{64}
  \DeclareMathSymbol{\beth}{\mathord}{txAMSb}{105}
  \DeclareMathSymbol{\gimel}{\mathord}{txAMSb}{106}
  \DeclareMathSymbol{\daleth}{\mathord}{txAMSb}{107}
\fi
\makeatother

%%%%%%%%%%%%%%%%%%%%%%%%%%%%%%%%%%%%%%%%%%%%%%%%%%%%%%%%%%%%
%%
%% Theorem Environments
%%
\newtheorem{theorem}{Theorem}
\newtheorem{lemma}[theorem]{Lemma}
\newtheorem{prop}[theorem]{Proposition}
\newtheorem{corollary}[theorem]{Corollary}
\newtheorem{claim}{Claim}
\newtheorem*{claim*}{Claim}
\theoremstyle{definition}
\newtheorem{definition}[theorem]{Definition}
\newtheorem{remark}[theorem]{Remark}
\newtheorem{example}[theorem]{Example}
\theoremstyle{remark}
\newtheorem*{remark*}{Remark}
%%
%%%%%%%%%%%%%%%%%%%%%%%%%%%%%%%%%%%%%%%%%%%%%%%%%%%%%%%%%%%%%%%%%%%%%%

%% \newenvironment{inducc}{\begin{list}{}{\itemindent=2.5em \labelwidth=4em}}{\end{list}}
%% \newcommand{\caso}[1]{\item[\fbox{#1}]}
\newenvironment{proofofclaim}{\begin{proof}[Proof of Claim]}{\end{proof}}


%%% Local Variables: 
%%% mode: latex
%%% TeX-master: "blueprint"
%%% End: 

%%%%%%%%%%%%%%%%%%% % %   Unicodings   % % %%%%%%%%%%%%%%%%%%%%%
\newcommand{\mt}[1]{$\mathtt{#1}$}
\DeclareUnicodeCharacter{21D2}{\mt\Rightarrow}
\DeclareUnicodeCharacter{2208}{\mt\in}
\DeclareUnicodeCharacter{2200}{\mt\forall}
\DeclareUnicodeCharacter{2203}{\mt\exists}
\DeclareUnicodeCharacter{2227}{\mt\wedge}
\DeclareUnicodeCharacter{27F6}{\mt\longrightarrow}
\DeclareUnicodeCharacter{27E6}{\mt\llbracket}
\DeclareUnicodeCharacter{27E7}{\mt\rrbracket}
\DeclareUnicodeCharacter{27F9}{\mt\Longrightarrow}
\DeclareUnicodeCharacter{03A6}{\mt\Phi}
\DeclareUnicodeCharacter{03BB}{\mt\lambda}
%\DeclareUnicodeCharacter{}{\mt}


\renewcommand{\PP}{\mathbb{P}}
\renewcommand{\app}{\mathrm{App}}
\newcommand{\formula}{\ensuremath{\mathtt{formula}}}
\newcommand{\tyi}{\mathtt{i}}
\newcommand{\tyo}{\mathtt{o}}
\newcommand{\forceisa}{\mathtt{forces}}
\newcommand{\equ}{\mathbf{e}}
\newcommand{\bel}{\mathbf{b}}
\newcommand{\atr}{\mathit{atr}}
\newcommand{\reee}{\mathrel{\mathtt{eee}}}
\DeclareMathOperator{\val}{val}


%%%%%%%%%%%%%%%%%%% Footnotemarks for "Thanks!" %%%%%%%%%%%%%%%%%%
\makeatletter
\renewcommand*{\@fnsymbol}[1]{\ensuremath{\ifcase#1\or *\or \dagger\or \ddagger\or
    \mathsection\or \mathparagraph\or \|\or **\or \dagger\dagger
    \or \ddagger\ddagger \else\@ctrerr\fi}}
\makeatother
%%%%%%%%%%%%%%%%%%%%%%%%%%%%%%%%%%%%%%%%%%%%%%%%%%%%%%%%%%%%%%%%%%

\begin{document}
\title{Forcing for Isabelle}
\author{%
  Pedro Sánchez Terraf%
  \thanks{
    Universidad Nacional de C\'ordoba. 
    Facultad de Matem\'atica, Astronom\'{\i}a,  F\'{\i}sica y
    Computaci\'on.}
  \thanks{
    Centro de Investigaci\'on y Estudios de Matem\'atica (CIEM-FaMAF).
    C\'ordoba. Argentina.}
}
%
\maketitle
%
\begin{abstract}
  Basic definitions of the forcing technique and blueprint for an
  Isabelle/ZF proof of the Forcing Theorems, under the assumption of
  the existence of a countable transitive model of $\ZFC$.   
\end{abstract}
%%%%%%%%%%%%%%%%%%%%%%%%%%%%%%%%%%%%%%%%%%%%%%%%%%%%%%%%%%%%%%%%%%


%%%%%%%%%%%%%%%%%%%%%%%%%%%%%%%%%%%%%%%%%%%%%%%%%%%%%%%%%%%%%%%%%%
%
\section{Introduction}
The proofs are mostly based on Chapter IV of
Kunen's~\cite{kunen2011set}, and the implementation will make use of
the work by Paulson~\cite{paulson_2003}. 

\section{Forcing notions}
We will need the following concepts:
\begin{itemize}
\item Forcing poset or \emph{notion}: a triple $\lb \PP, \1, \leq\rb$ where $\leq$ is a
  partial (pre)order on $\PP$ with (a distinguished) top element $\1$.
\item Compatible elements. Antichain in $\PP$. 
\item Dense subset $D\sbq\PP$ of a forcing poset.
\item Filter $G\sbq\PP$ in a poset.
\item $\calD$-generic filter for $\PP$, where $\calD$ is a family of
  dense subsets of $\PP$.
\end{itemize}
(Check \verb|ZF/Order.thy| before doing any work.)

Most probably, we'll need two or three versions of each of these
definitions: One internalized as an element of \formula;  a relativized
one with a class parameter; and finally one without
relativization. This is done in order to show absoluteness and to take
advantage of the Isabelle/ZF ability to operate with terms of type
$\tyi$.

As an example, consider $\PP,R \in M$,  and $R\sbq
\PP\times\PP$. Reflexivity of the relation $R$ is 
stated as follows.
\[
\phi(\PP,R)\defi \forall p (p\in \PP \limp \lb p,p\rb \in R).
\]
In Isabelle/ZF, we write
%
\begin{verbatim}
definition 
  reflexivity :: "[i,i] ⇒ o" where
  "reflexivity(P,r) == ∀p . p∈P ⟶ <p,p>∈r"
\end{verbatim}

This is different from what would $M$ see from ``within.'' In
particular, the universal quantifier should range only over elements
of $M$:
\[
\phi^M\defi\forall p\in M (p\in \PP \limp \lb p,p\rb^M \in R).
\]
The use of the relativized term $\lb p,p\rb^M$ implicitly assumes that
there is an element of $M$ satisfying its defining formula. In
Isabelle/ZF, we might move to a more general setting, by allowing  $M$
to be a class. Here, the reference to the ordered pair  is through a
relational definition.
\begin{verbatim}
definition  
  reflexivity_rel :: "[i⇒o,i,i] ⇒ o" where
  "reflexivity_rel(M,P,r) == ∀p[M].  p∈P ⟶ 
                             (∃x[M]. pair(M,p,p,x) ∧ x∈r)"
\end{verbatim}
(Notice: There must be no spaces between \verb|[M]| and the following
period.) This is the relativized version of
\verb|reflexivity|. 

Finally, the internalized version $\quine{\phi}$ of reflexivity is as
follows: 
\begin{verbatim}
definition
  reflexivity_fm :: "[i,i]⇒i" where
  "reflexivity_fm(x,y) == Forall(Implies(Member(0,succ(x)),
          Exists(And(pair_fm(1,1,0),Member(0,succ(succ(y)))))))"

lemma reflexivity_type : 
        "⟦x∈nat ; y∈nat⟧ ⟹ reflexivity_fm(x,y)∈formula"
  by (simp add:reflexivity_fm_def)
\end{verbatim}

The last lemma shows that indeed we have defined an element of the set
\formula. The formula \verb|reflexivity_fm| is an internalization of
the Isabelle/ZF formula \verb|reflexivity|, not the relativized
one.

Under the assumption of transitivity of $M$, the three versions are
equivalent. This is an \emph{absoluteness result}:
\[
\phi(\PP,R) \lsii \phi^M(\PP,R) \lsii M;[\PP,R]\models\quine{\phi},
\]
when $\PP,R\in M$, and the list $[\PP,R]$ is the environment or
interpretation of free variables. 

\begin{definition}
  $G$ is \emph{$M$-generic for $\PP$} if it is $\calD$-generic for
  $\PP$ for the family $\calD$ of all dense subsets of $\PP$ that live
  in $M$.
\end{definition}
This last definition is \emph{not} to be relativized, since for
nontrivial $\PP$, $G\notin M$. Absoluteness results as the one above
always assume that the free variables range in the set or class to
which one is relativizing.
\section{Names}
\begin{framed}
  The following is mostly independent of the previous section, but %
  \begin{bfseries}%
    we assume that $\1\in\PP\in M$, $M$ is transitive and 
    $1\in G\sbq\PP$.
  \end{bfseries}
  (Note that, in general, $G\notin M$.)
\end{framed}

\begin{definition}
  \begin{enumerate}
  \item The relation \verb|eee|$(M)$ is the restriction to $M$ of the
    relational composition 
    ${\in}\circ{\in}\circ{\in}$. 
  \item For $\tau\in M$, the \emph{$G$-value} or \emph{interpretation
    under $G$} of $\tau$ is
    \[
    \val(\tau,\PP,G)\defi \{\val(\sigma,\PP,G) : \exists p\in \PP
    (\lb\sigma,p\rb\in\tau \y p\in G)\}.
    \]
  \end{enumerate}
\end{definition}

The definition of $\val$ proceeds by well-founded recursion on
\verb|eee|$(M)$. Determining  $\val(\tau,\PP,G)$ uses the values
$\val(\sigma,\PP,G)$ for smaller $\sigma$:
\[
\sigma\in\{\sigma\}\in\lb\sigma,p\rb\in\tau.
\]

\begin{remark}
  Slightly generalizing the recursive scheme for $\val$ we have:
  \[
  F(\tau,\bx)\defi \{F(\sigma,\bx) : \phi(\sigma,\tau,\bx) \y
  \sigma\mathrel{R}\tau\}, 
  \]
  where $R$ is some well-founded relation. Using this scheme the
  \emph{Mostowski collapse} of $R$ can be defined:
  \[
  \mos(\tau)\defi \{\mos(\sigma) : \sigma\mathrel{R}\tau\},
  \]
  which is a tool of independent interest.
\end{remark}

\begin{definition}
  $\tau$ is a \emph{$\PP$-name} if its elements are of the form
  $\lb\sigma,p\rb$ where $p\in\PP$ and $\sigma$ is a $\PP$-name.
\end{definition}
The usual way to define a property by well-founded recursion is to
define the ``characteristic function'' of that property. In this case,
define by recursion the function $F:M\to\om$ such that $F(x)=1$  if
$x$ is a $\PP$-name, or $F(x)=0$ otherwise.

\begin{definition}
  The \emph{extension of $M$ by $G$} is
  \[
  \verb|gen_ext|(M,\PP,G) = M[G] \defi \{\val(\tau,\PP,G): \tau\in M\}.
  \]
\end{definition}
The standard definition of $M[G]$ adds the clause that $\tau$ should be
a $\PP$-name, but this is not necessary. When $G$ is $M$-generic for
$\PP$, we say that $M[G]$ is a \emph{generic extension}.

\section{The forcing relation}
In order to prove the independence of $\CH$, we will assume the
existence of countable transitive model (ctm) $M$ of $\ZFC$. The
satisfaction relation $M,\mathit{env}\models\Phi$ (where $\Phi$ is a set of
sentences and $\mathit{env}$ is an interpretation for the free
variables) is coded in Isabelle/ZF by using \verb|satT|:
\begin{verbatim}
definition
  satT :: "[i,i,i] => o" where
  "satT(A,Φ,env) == ∀ p ∈ Φ. sats(A,p,env)"
\end{verbatim}
The function \verb|satT| takes a subset of \formula{} as its second
argument, and it is based on the already defined function \verb|sats|.

\begin{framed}
  It might be necessary to redefine \verb|sats| and the related
  function \verb|satisfies| in order to have an independent argument $B$
  to be the range of values of the environment. We would have, for
  example, 
\begin{verbatim}
consts   satisfies :: "[i,i]=>i"
primrec 
  "satisfies(A,B,Member(x,y)) =
      (λenv ∈ list(B). bool_of_o (nth(x,env) ∈ nth(y,env)))"
\end{verbatim}
  \dots
  
\noindent and then
\begin{verbatim}
abbreviation
  sats :: "[i,i,i] => o" where
  "sats(A,p,env) == satisfies(A,A,p)`env = 1"
\end{verbatim}
\end{framed}

Given a ctm $M$, and an $M$-generic filter $G\sbq\PP$, the Forcing
Theorems relate satisfaction of a formula 
$\phi$ in the generic extension $M[G]$ to the satisfaction of another formula
$\phi'$ in $M$. The map $\phi\mapsto\phi'$ is defined by recursion on
the structure of $\phi$. The fact that this map works (in particular,
in order to show that $M[G]\models\ZFC$), we must use the fact that
$M\models\ZFC$. Since we can only assert the latter using internalized
formulas, the aformentioned map will be defined as a function from the
set \formula{} into itself. 

Up to this point, the main reason for working with internalized
versions is that  it is not possible to do recursion with formulas of
type $\tyo$ (i.e., the formulas in the first-order logic of
Isabelle/ZF).

We will now make more precise the definition of the map
$\phi\mapsto\phi'$ and how it relates satisfaction in $M$ to that in
$M[G]$. Actually, if the formula $\phi$ has $n$ free variables,
$\phi'$ will have $n+4$ free variables, where the first four account
for the forcing notion and a particular condition. If
$\phi=\phi(x_1,\dots,x_n)$, the standard notation for $\phi'$ is
\[
p\forces_{\PP,\leq,\1}^* \phi(t_1,\dots,t_n).
\]
Here, $\PP,\leq,\1,$ and $p\in\PP$ are the extra parameters. 
%% For the
%% time being, we just consider $\tau_1,\dots,\tau_n$ to be ordinary
%% variables. 
Given $\quine\phi\in\formula$, we will write this  in
Isabelle/ZF by using  
\[
\forceisa(\quine\phi,\PP,\leq,\1,p)
\]
Afterwards, the \emph{forcing relation} $\forces$ is defined by
relativizing/interpreting $\forces^*$ in a ctm $M$, for fixed
$\lb\PP,\leq,\1\rb\in M$:
\[
p\forces \phi(t_1,\dots,t_n) \defi 
\bigl(p\forces_{\PP,\leq,\1}^* \phi(t_1,\dots,t_n)\bigr)^M\!.
\]

\subsection{Forcing for atomic formulas}\label{sec:forcing-atomic-formulas}
The base case of the definition of the forcing relation, that is, for
atomic formulas, is (in)famously the most complicated one. 
The definition adopted by Kunen~\cite[p.~257]{kunen2011set} is the
following, with two mutually recursive clauses:
\begin{multline}\label{eq:def-forcing-equality}
  \forceisa({t_1= t_2},\PP,\leq,\1,p) \defi 
  \forall s\in\dom(t_1)\cup\dom(t_2)\;\forall q\leq p :\\
  \forceisa({s\in t_1},\PP,\leq,\1,q)\lsii 
  \forceisa({s\in t_1},\PP,\leq,\1,q),
\end{multline}
\begin{multline}\label{eq:def-forcing-membership}
  \forceisa({t_1\in t_2},\PP,\leq,\1,p) \defi  \\
  \text{``}\{q\leq p : \exists \lb s,r\rb\in t_2 (q\leq r \y
  \forceisa({t_1 =s},\PP,\leq,\1,q))\} \\
  \text{is dense below $p$.''}
\end{multline}
Other approaches, like the one in Neeman~\cite{neeman-course}, prefer
to have a single, more complicated, clause for $t_1=t_2$ using
well-founded recursion on $t_1$
and $t_2$ and then define the case $t_1\in t_2$ explicitly. 

In any case, these definitions assume that $t_i$ are \emph{names}.
In fact, the intuitive definition is by recursion on \emph{sentences}
that are written in the language $\{\in\}$ plus every name as a
constant; thus they
depend on the theorem of well-founded recursion,
so $\forceisa(\phi,\PP,\leq,\1,p)$ for atomic $\phi$
must be obtained from the proof of that theorem. Below, a defined
relation $R$ is \emph{set-like} if $R^{-1}\{a\}$ is a set for every $a$.
%
\begin{theorem}[{\cite[I.9.11]{kunen2011set}}]
  Assume that $R$ is a
  set-like well-founded defined relation on a class $A$, assume that 
  $\forall  x,s\,\exists!y\,\phi(x,s,y)$ and write $G(x,s)$ for the
  only $y$ such that $\phi(x,s,y)$. Then there exists another formula 
   $\psi$ such that $\forall x\,\exists!y\,\psi(x,y)$ and if we write
  $F(x)$ for the only $y$ satisfying $\psi(x,y)$, we have
  \[
  \forall a\in A: F(a) = G(a,F\restr R^{-1}\{a\}).
  \]
\end{theorem}
%
Observe that since $F$ is a definable function, its restriction to the
set $R^{-1}\{a\}$ must also be a set by the Replacement Axiom.

There are several steps to apply this theorem to justify the
definition of forcing for atomic formulas:
\begin{enumerate}
\item As mentioned before, $F$ will be the characteristic function of
  the predicate $\forceisa$.
  %
\item The class $A$ and the relation $R$ must be
  defined.
  %
\item Write down explicitly $\psi$ as an element of
  $\formula$. 
\end{enumerate}
%
Luckily, the formula $\psi$ is spelled out in Kunen's.

The  function $F$ will essentially have 
the same  arguments  of $\forceisa$, but we will deconstruct the
formula and pack everything as a tuple. Namely, we will have
\begin{align}
  F(\lb\equ,t_1,t_2,\PP,\leq,\1,p\rb) =1 &\iff  
  \forceisa(t_1=t_2,\PP,\leq,\1,p) \label{eq:forcing-F-equ}\\
  F(\lb\bel,s,t_1,\PP,\leq,\1,p\rb) =1 &\iff  
  \forceisa(s\in t_1,\PP,\leq,\1,q). \label{eq:forcing-F-bel}
\end{align}
Here, $\equ$ and $\bel$ are absolute constants that encode which type of
formula we are considering (for instance, $0$ and $1$, 
respectively). We can choose our $A$ to be 
$\{\equ,\bel\}\times V^6$ (i.e., all 7-tuples whose first coordinate
are  $\equ$ or $\bel$).

Rewriting the clauses (\ref{eq:def-forcing-equality}) and
(\ref{eq:def-forcing-membership}), we might obtain a first
approximation to $G$. Notice that its domain is $A\times V$.

$G(\lb\atr,t_1,t_2,\PP,\leq,\1,p\rb,f) =1$ if one of the following hold:
\begin{itemize}
\item $\atr=\equ$ and   $\forall s\in\dom(t_1)\cup\dom(t_2)\;\forall q\leq p :
  f(\lb\bel,s,t_1,\PP,\leq,\1,q\rb) =1  \lsii
  f(\lb\bel,s,t_2,\PP,\leq,\1,q\rb) =1$; or
\item $\atr=\bel$ and $\{q\leq p : \exists \lb s,r\rb\in t_2 (q\leq r \y
  f(\lb\equ,t_1,s,\PP,\leq,\1,q\rb) =1)\}$ is dense below $p$.
\end{itemize}
In any other case, $G(\lb\atr,t_1,t_2,\PP,\leq,\1,p\rb,f) =0$. In the
above, $f(v)=1$ should be interpreted as $\lb v,1\rb\in f$ (i.e., $f$
must be defined for the argument at hand). From this scheme an element
of $\formula$ must be produced. (This is the reason why  $\equ,\bel$
were chosen to be definable.)

The definition of the relation $R$ is straightforward and follows the
pattern of \cite[p.~257]{kunen2011set}. For instance, we have that 
\[
\lb\bel,s_1,t_1,\PP,\leq,\1,q\rb\mathrel{R}
\lb\equ,s_2,t_2,\PP,\leq,\1,p\rb
\]
if and only if 
\begin{itemize}
\item $s_1 \reee  s_2$ or $s_1\reee t_2$; and
\item $t_1 = s_2$ or $t_1 = t_2$.
\end{itemize}
%
\begin{framed}
  It seems necessary to prove that the relation $\reee$ is
  well-founded independently of the universe $M$ where one is
  working. We only
  have a proof that ${\reee}(M)$ is well-founded for each $M$.
\end{framed}

With these elements we might reproduce the formula $\psi$ that defines
$F$ as in \cite[p.~49]{kunen2011set}. Let  $\app(d,h)$  be the formula
that states:
\begin{enumerate}
\item \label{item:dom-R-cerrado} $d$ is ``downwards $R$-closed'':
  $\forall y \in d, 
  R^{-1}\{y\}\subseteq d$;
\item \label{item:4} $h$ is a function with $\dom h =d \subseteq A$; and
\item \label{item:3} for all $y\in d$, $h(y) = G(y,h\restr(R^{-1}\{y\}))$.
\end{enumerate}
Then the formula defining $F$ is the following:
\[
(x\notin A \y y=0) \o \bigl(x\in A \y \exists d,h\, (\app(d,h)\y
x\in d \y h(x)=y)\bigr).
\]
The well-foundedness of $R$ is used to show that this formula is indeed
functional, and that it satisfies the required recursive equation
involving $G$. The following intermediate result is needed in the
sequel.
\begin{lemma}[Uniqueness of approximations]\label{lem:uniq-approx}
  Assume $\app(d,h)$ and $\app(d',h')$. Then
  $h(x)=h'(x)$ for all  $x\in d \cap d'$. In particular,
  \[
  h\restr(\{x\}\cup R^{-1}\{x\})=h'\restr(\{x\}\cup R^{-1}\{x\})
  \]
  for  all $x\in d \cap d'$. 
\end{lemma}

  

Once this is all translated to an element of \formula,
we'll be able to write the base case of the recursive definition of
$\forceisa$ using Equations~(\ref{eq:forcing-F-equ})
and~(\ref{eq:forcing-F-bel}).
 
\subsection{Inductive step}
%
The nexts steps in the definition of $\forceisa$ are straightforward
\cite[IV.2.42]{kunen2011set}. We just highlight some of the cases.
\begin{itemize}
\item $\forceisa(\neg\phi,\PP,\leq,\1,p) \defi \neg\exists
  q\leq p \,(\neg \forceisa(\phi,\PP,\leq,\1,q))$.
\item
  $\forceisa(\phi\y\psi,\dots)\defi\forceisa(\phi,\PP,\leq,\1,p)\y\forceisa(\psi,\PP,\leq,\1,p)$.
\item $\forceisa(\forall x\,\phi(x),\PP,\leq,\1,p)\defi \forall x\,\forceisa(\phi(x),\PP,\leq,\1,p)$. 
\item
  $\forceisa(\phi\o\psi,\dots)\defi\text{``}\{q:\forceisa(\phi,\PP,\leq,\1,q)\o\forceisa(\psi,\PP,\leq,\1,q)\}$
  is dense below $p$''. 
\end{itemize}
%
Actually, recursion over $\formula$ uses only one propositional
connective (``nand'') so we need the combination of the first two to
obtain $\forceisa(\mathtt{Nand(\phi,\psi)},\dots)$:
\[
\neg\exists
  q\leq p :\neg\bigl(\forceisa(\phi,\PP,\leq,\1,q)\y\forceisa(\psi,\PP,\leq,\1,q)\bigr).
\]

\section{The fundamental theorems}
These are stated as two lemmas: \emph{Truth} and
\emph{Definability}. The second one ``simply'' says that the forcing
relation is definable. The previous recursive construction amounts to
that. The Truth Lemma states that the forcing relation indeed relates
satisfaction in $M[G]$ to that in $M$. 
\begin{lemma}[Truth Lemma]\label{lem:truth-lemma}
  Let $M$ be a ctm of $\ZFC$, $\lb\PP,\leq,\1\rb$ a forcing notion
  in $M$, $p\in\PP$, and $\phi(x_1,\dots,x_n)$ a formula in the
  language of set 
  theory with all free variables displayed. Then the
  following are equivalent, for all $\tau_1,\dots,\tau_n\in M$: 
  \begin{enumerate}
  \item     $M[G]\models\phi(\val(\tau_1,G),\dots,\val(\tau_n,G))$ for
    all filters $G$, $M$-generic for $\PP$, such that $p\in G$.
  \item $M\models\forceisa(\phi(\tau_1,\dots,\tau_n),\PP,\leq,\1,p)$.
  \end{enumerate}
\end{lemma}
A bit more formally, one should write the parameters to the left of
$\models$:
\begin{gather*}
  \forall G : p\in G \limp M[G],[\val(\tau_1,G),\dots,\val(\tau_n,G)]
  \models\phi(x_1,\dots,x_n) \\
  \text{if and only if}\\
  M,[\tau_1,\dots,\tau_n,\PP,\leq,\1,p] 
  \models\forceisa(\phi(x_1,\dots,x_n),x_{n+1},x_{n+2},x_{n+3},x_{n+4})
\end{gather*}
The proof of the Truth Lemma is by recursion in elements of
$\formula$. The following auxiliary results (adapted from
\cite[IV.2.43]{kunen2011set}) are also proved by recursion in
$\formula$.
\begin{lemma}\label{lem:strengthen} 
  Assume the same hypothesis of Lemma~\ref{lem:truth-lemma}.
  $M\models\forceisa(\phi,\PP,\leq,\1,p)$ and $p_1\leq p$
  implies $M\models\forceisa(\phi,\PP,\leq,\1,p_1)$.
\end{lemma}
\begin{lemma}\label{lem:density}
  Assume the same hypothesis of Lemma~\ref{lem:truth-lemma}. $M\models\forceisa(\phi,\dots)$ if and only if
    $M\models$``$\{p_1\leq p : \forceisa(\phi,\PP,\leq,\1,p_1)\}$ is
    dense below $p$.''
\end{lemma}
\begin{lemma}\label{lem:forcing-negation}
  Under the same hypothesis of Lemma~\ref{lem:truth-lemma}, we have
  $M\models\forceisa(\phi,\dots)$ if and only if 
  $M\models\neg\exists
  q\leq p \,(\neg \forceisa(\phi,\PP,\leq,\1,q))$.
\end{lemma}

\subsection{Truth for atomic formulas}
We start by proving  Lemma~\ref{lem:strengthen}.% and~\ref{lem:density}.
\begin{proof}[Proof of  Lemma~\ref{lem:strengthen} (atomic formulas)]
  Let $M$, $\lb\PP,\leq,\1\rb$, $p\in\PP$ as in the hypothesis of
  Lemma~\ref{lem:truth-lemma}, let $p_1\leq p$ and let $\phi(x_1,x_2)$
  be atomic. We 
  have to show that for every    $\tau_1,\tau_2\in M$, if
  \begin{equation} \label{eq:M-phi-atom}
    M,[\tau_1,\tau_2,\PP,\leq,\1,p] 
    \models\forceisa(\phi(x_1,x_2),x_{3},x_{4},x_{5},x_{6})
  \end{equation}
  then the same holds for the environment
  $[\tau_1,\tau_2,\PP,\leq,\1,p_1]$. For this, we will deconstruct the
  definition of $\forceisa$ in several steps to pin down where the
  hypothesis of $p_1\leq p$ might be used.
  
  So, let $\phi$ be $x_1\in x_2$ and assume (\ref{eq:M-phi-atom}). By
  our definition of $\forceisa$  for atomic formulas, we have
  $F(\bx)=1$, where
  $\bx=\lb\equ,\tau_1,\tau_2,\PP,\leq,\1,p\rb$. Expanding the
  definition of $F$, we obtain that
  \[
  (\bx\notin A \y 1=0) \o \bigl(\bx\in A \y \exists d,h\, (\app(d,h)\y
  \bx\in d \y h(\bx)=1)\bigr)
  \]
  holds in $M$. 
  \begin{framed}
    By using appropriate absoluteness results we might translate this
    statement written as an element of $\formula$ to a sentence
    written in Isabelle/ZF first order logic, making theorems to
    simplify the statement available, as in the arguments
    that follow.
  \end{framed}
  \noindent Simplifying, we obtain 
  \begin{equation}\label{eq:1}
    \exists d,h\, (\app(d,h)\y   \bx\in d \y h(\bx)=1).
  \end{equation}
  Take such $d,h\in M$; these will be added to the environment. We
  will define $d',h'$ that satisfy 
  $\app(d',h')\y   \by\in d' \y h(\by)=1$ for
  $\by=\lb\equ,\tau_1,\tau_2,\PP,\leq,\1,p_1\rb$, thereby proving
  $F(\by)=1$ and hence the thesis.

  By  definition of $\app$ and the fact that $\bx\in d$, we have
  \[
  1= h(\bx) = G(\bx,h\restr(R^{-1}\{\bx\})).
  \]
  Now the  definition of $G$ for $\bx=\lb\equ,\dots\rb$, yields
  \[
  M,[d,h,\tau_1,\tau_2,\PP,\leq,\1,p]\models \forall
  s\in\dom(x_1)\cup\dom(x_2)\;\forall q\leq x_6 : \Phi
  \]
  with $\Phi\defi
  f(\lb\bel,s,x_1,x_{3},x_{4},x_{5},x_{6}\rb) =1  \lsii
  f(\lb\bel,s,x_2,x_{3},x_{4},x_{5},x_{6}\rb) =1$ and 
  $f\defi h\restr(R^{-1}\{\bx\})$.
  By absoluteness we have
  \[
  \forall
  \sigma\in\dom(\tau_1)\cup\dom(\tau_2)\;\forall q'\leq p :
  M,[\sigma,q',d,h,\tau_1,\tau_2,\PP,\leq,\1,p]\models  \Phi.
  \]
  Since $q'\leq p_1$ implies $q'\leq p$, we have
  \[
  \forall
  \sigma\in\dom(\tau_1)\cup\dom(\tau_2)\;\forall q'\leq p_1 :
  M,[\sigma,q',d,h,\tau_1,\tau_2,\PP,\leq,\1,p_1]\models  \Phi.
  \]
  Using absoluteness again we have
  \[
  M,[d,h,\tau_1,\tau_2,\PP,\leq,\1,p_1]\models \forall
  s\in\dom(x_1)\cup\dom(x_2)\;\forall q\leq p_1 : \Phi.
  \]
  Hence we obtain $G(\by,h\restr(R^{-1}\{\bx\}))=1$. 

  Now define   $d'\defi d\cup \{\by\}$  and $h'\defi
  h\cup\{\lb\by,1\rb\}$. Since
  $R^{-1}\{\bx\} =R^{-1}\{\by\}$ and $\by\in A$, we obtain
  $\app(d',h')$. This, together with the fact that $h(\by)=1$, shows
  $F(\by)=1$ and hence (\ref{eq:M-phi-atom}).

  The proof for $\phi= x_1\in x_2$ should be similar.
\end{proof}
Lemma~\ref{lem:density} also follows the same pattern, there is a bit
more of mathematics to it. The same goes for
Lemma~\ref{lem:forcing-negation}.


We restate Lemma~\ref{lem:truth-lemma} in an alternative way, for
atomic formulas.
\begin{lemma}
  Let $M$ be a ctm of $\ZFC$, $\lb\PP,\leq,\1\rb$ a forcing notion
  in $M$, $p\in\PP$, $G\sbq\PP$ $M$-generic for $\PP$ and $\phi(x_1,x_2)$ an
  atomic formula. Then the following hold for all
  $\tau_1,\tau_2\in M$:
  \begin{enumerate}
  \item If $p\in G$ and
    \begin{equation} \label{eq:M-sats-forces-atom}
      M,[\tau_1,\tau_2,\PP,\leq,\1,p] 
      \models\forceisa(\phi(x_1,x_2),x_{3},x_{4},x_{5},x_{6})
    \end{equation}
    then
    \begin{equation} \label{eq:MG-sats-atom}
      M[G],[\val(\tau_1,G),\val(\tau_2,G)]
      \models\phi(x_1,x_2).
    \end{equation}
  \item If (\ref{eq:MG-sats-atom}) holds 
    then there is some $p\in G$ such that
    (\ref{eq:M-sats-forces-atom}) holds for that $p$.
  \end{enumerate}
\end{lemma}
This proof consists of two cases for $\phi$ (namely, $x_1\in x_2$ and
$x_1=x_2$) and it is done by well-founded recursion over $R$ (defined
in Section~\ref{sec:forcing-atomic-formulas}).


%%%%%%%%%%%%%%%%%%%%%%%%%%%%%%%%%%%%%%%%%%%%%%%%%%%%%%%%%%%%%%%%%%%%%%%%
\bibliographystyle{alpha}
\bibliography{blueprint}

\bigskip

\begin{small}
  \begin{quote}
    \texttt{sterraf@famaf.unc.edu.ar}
    
    CIEM --- Facultad de Matem\'atica, Astronom\'{\i}a y F\'{\i}sica 
    (Fa.M.A.F.) 
    
    Universidad Nacional de C\'ordoba --- Ciudad Universitaria
    
    C\'ordoba 5000. Argentina.
  \end{quote}
\end{small}

\end{document}

%%% Local Variables: 
%%% mode: latex
%%% ispell-local-dictionary: "american"
%%% End: 
