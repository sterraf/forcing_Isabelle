%-%-%-%-%-%-%-%-%-%-%-%-%-%-%-%-%-%-%-%-%-%-%-%-%-%-%-%-%-%-%-%-%-%-%-%-%-%-%-%
\section{Isabelle/ZF}\label{sec:isabellezf}

Let us introduce briefly Paulson's formalization of ZF
\cite{paulson2017isabelle} in Isabelle and the main aspects of his
formal proof for the relative consistency of the Axiom of Choice
\cite{paulson_2003}; we will only focus on those aspects that are
essential to keep this paper self-contained, and refer the interested
reader to Paulson's articles.
%% Isabelle/ZF is a theory built upon the
%% core library FOL of classical first-order logic.
Isabelle/ZF includes a development of classical first-order logic,
FOL. Both of them are  built upon the core library \emph{Pure}. 

In Isabelle/ZF sets are \emph{individuals}, i.e.\ terms of type
\isatt{i} and formulas have type \isatt{o} (akin to a \emph{Bool}
type, but at the object level).  The axiomatization of $\ZFC$ in
Isabelle/ZF proceeds by postulating a binary predicate
\isatt{\ensuremath{\in}} and several set constructors (terms and
functions with values in \isatt{i}) corresponding to the empty set (the
constant \isatt{\isadigit{0}}), powersets, and one further constant
\isatt{inf} for an infinite set. The axioms, being formulas, are terms
of type \isatt{o}; the foundation axiom, for example, is formalized as
(the universal closure of) \isa{{\isachardoublequoteopen}A\
  {\isacharequal}\ {\isadigit{0}}\ {\isasymor}%
  \ {\isacharparenleft}{\isasymexists}x{\isasymin}A{\isachardot}\ %
  {\isasymforall}y{\isasymin}x{\isachardot}\ %
  y{\isasymnotin}A{\isacharparenright}%
  {\isachardoublequoteclose}}. %
Besides the axioms, Isabelle/ZF also introduces several definitions
(for example, pairs and sets defined by comprehension using
separation) and syntactic abbreviations to keep the formalization
close to the customary manner of doing mathematics.  Working with the
library and extending it is quite straightforward.  As an example, we
introduce a new term-former (which is a combination of instances of
replacement and separation) denoting the image of a function over a
set defined by comprehension, namely
$\{b(x): x\in A\text{ and }Q(x)\}$:
%
%
%
\begin{isabelle}
  \isacommand{definition}\isamarkupfalse%
  \ SepReplace\ {\isacharcolon}{\isacharcolon}\
  {\isachardoublequoteopen}{\isacharbrackleft}i{\isacharcomma}\
  i{\isasymRightarrow}i{\isacharcomma}\ i{\isasymRightarrow}\
  o{\isacharbrackright}\
  {\isasymRightarrow}i{\isachardoublequoteclose}\
  \isakeyword{where}\isanewline \ \
  {\isachardoublequoteopen}SepReplace{\isacharparenleft}A{\isacharcomma}b{\isacharcomma}Q{\isacharparenright}\
  {\isacharequal}{\isacharequal}\ {\isacharbraceleft}y\ {\isachardot}\
  x{\isasymin}A{\isacharcomma}\
  y{\isacharequal}b{\isacharparenleft}x{\isacharparenright}\
  {\isasymand}\
  Q{\isacharparenleft}x{\isacharparenright}{\isacharbraceright}{\isachardoublequoteclose}
\end{isabelle}
\noindent %
%
We are then able to add the abbreviation \isa{{\isacharbraceleft}b\
  {\isachardot}{\isachardot}\ x{\isasymin}A{\isacharcomma}\
  Q{\isacharbraceright}} as a notation for
\isa{SepReplace{\isacharparenleft}A{\isacharcomma}b{\isacharcomma}Q{\isacharparenright}}. The
characterization of our new constructor is given by
\begin{isabelle}
\isacommand{lemma}\isamarkupfalse%
\ Sep{\isacharunderscore}and{\isacharunderscore}Replace{\isacharcolon}\ {\isachardoublequoteopen}{\isacharbraceleft}b{\isacharparenleft}x{\isacharparenright}\ {\isachardot}{\isachardot}\ x{\isasymin}A{\isacharcomma}\ Q{\isacharparenleft}x{\isacharparenright}\ {\isacharbraceright}\ {\isacharequal}\ {\isacharbraceleft}b{\isacharparenleft}x{\isacharparenright}\ {\isachardot}\ x{\isasymin}{\isacharbraceleft}y{\isasymin}A{\isachardot}\ Q{\isacharparenleft}y{\isacharparenright}{\isacharbraceright}{\isacharbraceright}{\isachardoublequoteclose}
\end{isabelle}

%
%
%
%
%
%
%
%
%
%
%

%
%
%
%
%
We now discuss relativization in Isabelle/ZF. Relativized versions of the
 axioms can be found in the formalization of constructibility \cite{paulson_2003}. For
 example, the relativized Axiom of Foundation is
\begin{isabelle}
\isacommand{definition}\isamarkupfalse%
\ foundation{\isacharunderscore}ax\ {\isacharcolon}{\isacharcolon}\ {\isachardoublequoteopen}{\isacharparenleft}i{\isacharequal}{\isachargreater}o{\isacharparenright}\ {\isacharequal}{\isachargreater}\ o{\isachardoublequoteclose}\ \isakeyword{where}\isanewline
 {\isachardoublequoteopen}foundation{\isacharunderscore}ax{\isacharparenleft}M{\isacharparenright}\ {\isacharequal}{\isacharequal}\isanewline
  {\isasymforall}x{\isacharbrackleft}M{\isacharbrackright}{\isachardot}\ {\isacharparenleft}{\isasymexists}y{\isacharbrackleft}M{\isacharbrackright}{\isachardot}\ y{\isasymin}x{\isacharparenright}\ {\isasymlongrightarrow}\ {\isacharparenleft}{\isasymexists}y{\isacharbrackleft}M{\isacharbrackright}{\isachardot}\ y{\isasymin}x\ {\isacharampersand}\ {\isachartilde}{\isacharparenleft}{\isasymexists}z{\isacharbrackleft}M{\isacharbrackright}{\isachardot}\ z{\isasymin}x\ {\isacharampersand}\ z\ {\isasymin}\ y{\isacharparenright}{\isacharparenright}{\isachardoublequoteclose}
\end{isabelle}

%
\noindent The relativized quantifier
\isa{{\isasymforall}x{\isacharbrackleft}M{\isacharbrackright}{\isachardot}\
  P(x)} is a shorthand for \isa{{\isasymforall}x{\isachardot}\ M(x)
  {\isasymlongrightarrow} P(x)}. In order to express that a (set) model
satisfies this axiom we use  the ``coercion''
\isa{{\isacharhash}{\isacharhash} :: i ={\isachargreater} (i
  ={\isachargreater} o)} (that maps a set $A$ to the predicate
$\lambda x . (x\in A)$) provided by Isabelle/ZF. As a trivial example we
can show that the empty set satisfies Foundation:
\begin{isabelle}
\isacommand{lemma}\isamarkupfalse%
\ emp{\isacharunderscore}foundation\ {\isacharcolon}\ {\isachardoublequoteopen}foundation{\isacharunderscore}ax{\isacharparenleft}{\isacharhash}{\isacharhash}{\isadigit{0}}{\isacharparenright}{\isachardoublequoteclose}
\end{isabelle}


%
%
%
%

%
%
%
%

Mathematical texts usually start by fixing a context that defines
parameters and assumptions needed to develop theorems
and results. In Isabelle the way of defining contexts is through
\emph{locales}~\cite{ballarin2010tutorial}.
Locales can be combined and extended by adding more parameters and assuming
more facts, leading to a new locale. For example a context describing
lattices can be extended to distributive lattices.
The way to instantiate a locale is by \emph{interpreting} it, which consists
of giving concrete values to parameters and proving the assumptions.
In our work, we use locales to organize the formalization and to make
explicit the assumptions of the most important results.

Let us close this section with a brief comment about the facilities
provided by the Isabelle framework. The edition is done in an IDE
called \texttt{jEdit}, which is bundled with the standard Isabelle
distribution; it offers the user a fair amount of tools in order to
manage theory files, searching for theorems and concepts spread
through the source files, and includes tracing utilities for the
automatic tools. A main feature is a window showing the \emph{proof
  state}, where the active (sub)goals are shown, along with the
already obtained results and possibly errors.

Isabelle proofs can be written in two dialects.  The older one, and
also more basic, follows a procedural approach, where one applies
several tactics in order to decompose the goal into simpler ones and
then solving them (with the aid of automation); the original work by
Paulson used this method. Under this approach proofs are constructed
top-down resulting in proof-scripts that conceal the mathematical
reasoning behind the proof, since the intermediate steps are only
shown in the proof state. For this reason, the proof language
\emph{Isar} was developed, starting with Wenzel's
work~\cite{DBLP:conf/tphol/Wenzel99}. Isar is mostly declarative, and
its main purpose is to construct \emph{proof documents} that (in
principle) can be read and understood without the need of running the
code.

We started this development using the procedural approach, but soon
after we realized that for our purposes the Isar language was far more
appropriate.

%%% Local Variables:
%%% mode: latex
%%% ispell-local-dictionary: "american"
%%% TeX-master: "first_steps_into_forcing"
%%% End:
