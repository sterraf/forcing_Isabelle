\documentclass{beamer}
\usetheme{metropolis}           % Use metropolis theme
\usepackage[spanish]{babel}
\usepackage{amsmath,mathtools,array}
\usepackage{amssymb} 
\usepackage{amsthm} 
\usepackage{tikz}
\usepackage{microtype}
% \usepackage{../paper/paper}

\title{First steps towards a formalization of Forcing}
\author{Emmanuel Gunther\and \emph{Miguel Pagano}\and Pedro Sánchez Terraf}
\date{LSFA - Fortaleza, 26 September 2018}
\input{../header-draft}
\usepackage{../isabelle}
\usepackage{../isabellesym}
\begin{document}

\maketitle
%% Outline
% 

\begin{frame}{Zermelo-Fraenkel Set Theory}
  \begin{block}{}
    Set theory arose at the early twentieth century as one
    possible foundation for mathematics.
  \end{block}
  \begin{block}{}
    More formally, set theory is a first order theory ($ZFC$) with no
    constants and one relational symbol $\in$. % It has some axioms
    % (extensionality, pairing, foundation, union, powerset, infinity)
    % and two axiom schemes (separation and replacement). This theory
    % is known by $ZF$; while $ZFC$ is $ZF$ plus the axiom of Choice.
  \end{block}
  \begin{block}{}
    To say that $ZFC$ is a possible foundation for mathematics means
    that any mathematical theorem, say $T$, can be deduced from $ZFC$:
    \[ ZFC \vdash T \]
  \end{block}
\end{frame}
\begin{frame}{Cantor's Question and Gödel's and Cohen's answers}
  \begin{block}{}
    Cantor posed the \emph{Continuum Hypothesis} ($\CH$):
    \begin{quote}
      Every uncountable subset of $\R$ is equipotent with $\R$.
    \end{quote}
  \end{block}
  \begin{block}{}
    Gödel showed that $\CH$ is \emph{relatively consistent} with $ZFC$.
  \end{block}
  \begin{block}{}
    Later, Cohen showed that also the negation of $\CH$ is relatively
    consistent with $ZFC$. For this he invented the technique of
    \emph{forcing}.
  \end{block}
\end{frame}

\begin{frame}{Some preliminaries}
  \begin{block}{}
    To say that $T$ is \emph{relatively consistent} with $ZFC$ can be
    understood in two ways:
    \begin{enumerate}
    \item Construct a model $M'$ of $ZFC+T$ from a model $M$ for
      $ZFC$.
    \item Alternatively, deduce a contradiction in $ZFC$ from a
      contradiction in $ZFC+T$.% (Interestingly both proofs give
      % rise to algorithms for this)
    \end{enumerate}
  \end{block}

  \begin{block}{}
    Cohen's proofs construct a model $ZFC+\neg\CH$ assuming a
    countable and transitive model (ctm).
  \end{block}
  \begin{itemize}
  \item  $M$ is a \emph{transitive} model of $ZFC$ if $x \in M$ implies
    $x\subset M$.
  \item $M$ is \emph{countable} if it is equipotent with $\N$.
  \end{itemize}

  % \begin{block}{}
  %   Notice that it is fine to ask for a ctm because at the end we
  %   want to transforms contradictions 
  % \end{block}
\end{frame}

\begin{frame}{Forcing}
  \begin{block}{}
    \begin{itemize}
    \item Given a ctm $M$ and a \emph{generic} set $G$ \emph{for} $M$,
      one constructs a new ctm $M[G]$.
    \item One defines a poset $(P,\leq)$, where $P \in M$. The
      existence of a generic filter $G$ follows from Rasiowa-Sikorski.
    \item $M[G] = \{\val(G,x)\, |\, x \in M \}$ 
    \item Depending on the definition of $P$, $M[G]$ will satisfy
      $\CH$ or $\neg\CH$.
    \end{itemize}
  \end{block}
\end{frame}

\begin{frame}{Our goal: formalize forcing}
  \begin{block}{Questions?}
    \begin{itemize}
    \item Why forcing? Has not been formalized yet. % (Paulson formalized Gödel's result)
    \item Related work. Paulson formalized Gödel's proof. %is the largest development and
%      we can build on top of his work.
    \item Which proof assistant? Because of the previous answer:
      Isabelle/ZF.
    \end{itemize}
  \end{block}
\end{frame}

\begin{frame}{Isabelle/ZF}
  \begin{block}{Axiomatization of ZF in Isabelle/FOL}
    \begin{itemize}
    \item Sets are terms of type \isatt{i} and formulas have type
      \isatt{o}. % None of them are recursive data-types.
    \item Axioms of $ZFC$ are postulated as axioms of Isabelle:
      \isa{extension: A\ =\ B\ \isasymlongleftrightarrow\ A\ \isasymsubseteq\ B\ \isasymand\ B \isasymsubseteq\ A}
    \item A theorem of $ZFC$ is a \emph{lemma} of Isabelle/ZF:
           \includegraphics[scale=1.3]{eqI.png}
     \item In order to speak of formulas and satisfaction
           (models), Paulson coded formulas (and much more) as sets.
         \item He also defined relativized (to a class) versions of the axioms\\
\hspace{-0.5cm}   \isa{upair\_ax(M) == \isasymforall x[M]. \isasymforall  y[M].\isasymexists z[M]. upair(M,x,y,z)}
     \end{itemize}
\end{block}
\end{frame}

\begin{frame}{First steps towards Forcings}
  \begin{block}{Our strategy}
    \begin{enumerate}
    \item Be as modular as possible, using \emph{locales}.
    \item Define interfaces to deal with models of $ZFC$.
    \item Exploit the modularity of forcing to divide the work.
    \end{enumerate}
  \end{block}
\end{frame}

\begin{frame}{What have we done?}
  \begin{block}{Existence of a generic filter}
  \begin{itemize}
  \item Proved a principle of dependent choice.
  \item Using that principle, proved Rasiowa-Sikorski.
  \item If $M$ is a ctm and $P\in M$ is a poset, then
    there exists a generic filter for $M$.
  \end{itemize}
\end{block}
\begin{block}{Definition of $M[G]$}
  \begin{itemize}
  \item Defined \emph{names} for $x \in M$ and \emph{$\val$}.
  \item Defined $M[G]$, assuming $M$ ctm and $G$ generic for $M$.
  \item Proved $M \subseteq M[G]$.% (using $\check{x}$).
\end{itemize}
\end{block}
\end{frame}

\begin{frame}{$M[G] \models ZFC$}
  \begin{block}{Axiom by axiom}
    \begin{itemize}
    \item One of the hardest part is transfering truth on $M$ to
      truth on $M[G]$.
    % \item The most obvious, say for pairing, would be
    %   \isa{sats(M,upair\_fm,[])} implies \isa{sats(M[G],upair\_fm,[])}.
    \item We proved \isa{upair\_ax(\#\#M)} implies \isa{upair\_ax(\#\#M[G])}
    \item $M[G]\models \mathit{separation}$ (almost).
    \end{itemize}
  \end{block}
\end{frame}

\begin{frame}{Future work: The next steps}
  \begin{block}{Soon}
    \begin{itemize}
    \item $x \in M$ implies $\check{x}\in M$.
    \item Define the poset we are interested in ($Add(\omega,\kappa)$).
    \item Prove that $M[G]$ satisfies union, foundation, infinity.
    \end{itemize}
  \end{block}
  \begin{block}{Not so soon}
    \begin{itemize}
    \item Define the forcing relation ($\forces$).
    \item Prove the fundamental theorem of forcing.
    \item $M[G]\models ZFC$.
    \item If $P$ is ccc, then cardinals are preserved in $M[G]$.
    \item Prove that for any generic $G$ for $Add(\omega,\aleph_2)$,
      then $M[G]\models ZFC+\neg\CH$.
    \end{itemize}
  \end{block}
\end{frame}


\begin{frame}{Questions?}
  \begin{center}
    \begin{block}{}
      \large{Thank you!}
    \end{block}
  \end{center}
\end{frame}

\end{document}
