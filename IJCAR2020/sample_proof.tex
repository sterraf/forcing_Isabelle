\section{A sample proof}

We would like to illustrate the point of the simplicity of proofs
after the definition of $\forceisa$ is ready to use. We consider the following:

\begin{isabelle}
\isacommand{lemma}\isamarkupfalse%
\ strengthening{\isacharunderscore}lemma{\isacharcolon}\isanewline
\ \ \isakeyword{assumes}\ \isanewline
\ \ \ \ {\isachardoublequoteopen}p{\isasymin}P{\isachardoublequoteclose}\ {\isachardoublequoteopen}{\isasymphi}{\isasymin}formula{\isachardoublequoteclose}\ {\isachardoublequoteopen}r{\isasymin}P{\isachardoublequoteclose}\ {\isachardoublequoteopen}r{\isasympreceq}p{\isachardoublequoteclose}\isanewline
\ \ \isakeyword{shows}\isanewline
\ \ \ \ {\isachardoublequoteopen}{\isasymAnd}env{\isachardot}\ env{\isasymin}list{\isacharparenleft}M{\isacharparenright}\ {\isasymLongrightarrow}\ arity{\isacharparenleft}{\isasymphi}{\isacharparenright}{\isasymle}length{\isacharparenleft}env{\isacharparenright}\ {\isasymLongrightarrow}\ p\ {\isasymtturnstile}\ {\isasymphi}\ env\isanewline 
\ \ \ \ \ {\isasymLongrightarrow}\ r\ {\isasymtturnstile}\ {\isasymphi}\ env{\isachardoublequoteclose}\isanewline
%
%
\isacommand{using}\isamarkupfalse%
\ assms{\isacharparenleft}{\isadigit{2}}{\isacharparenright}
\end{isabelle}
%
The proof is divided in the 4 cases of definition of an element of $\formula$,
%
\begin{isabelle}
\isacommand{proof}\isamarkupfalse%
\ {\isacharparenleft}induct{\isacharparenright}\isanewline
\ \ \isacommand{case}\isamarkupfalse%
\ {\isacharparenleft}Member\ n\ m{\isacharparenright}\isanewline
\ \ \isacommand{then}\isamarkupfalse%
\isanewline
\ \ \isacommand{have}\ {\isachardoublequoteopen}n<length(env){\isachardoublequoteclose} {\isachardoublequoteopen}m<length(env){\isachardoublequoteclose}\isanewline
\ \ \ \  using arities{\isacharunderscore}at{\isacharunderscore}aux by simp{\isacharunderscore}all\isanewline
\ \ \isacommand{moreover}\isanewline
\ \ \isacommand{assume} {\isachardoublequoteopen}env{\isasymin}list(M){\isachardoublequoteclose}\isanewline
\ \ \isacommand{moreover}\isanewline
\ \ \isacommand{note} assms Member\isanewline
\ \ \isacommand{show}\isamarkupfalse%
\ {\isacharquery}case\ \isanewline
\ \ \ \ \isacommand{using}\isamarkupfalse%
\ Forces{\isacharunderscore}Member{\isacharbrackleft}of\ {\isacharunderscore}\ {\isachardoublequoteopen}nth{\isacharparenleft}n{\isacharcomma}env{\isacharparenright}{\isachardoublequoteclose}\ {\isachardoublequoteopen}nth{\isacharparenleft}m{\isacharcomma}env{\isacharparenright}{\isachardoublequoteclose}\ env\ n\ m{\isacharbrackright}\isanewline
\ \ \ \ \ \ strengthening{\isacharunderscore}mem{\isacharbrackleft}of\ p\ r\ {\isachardoublequoteopen}nth{\isacharparenleft}n{\isacharcomma}env{\isacharparenright}{\isachardoublequoteclose}\ {\isachardoublequoteopen}nth{\isacharparenleft}m{\isacharcomma}env{\isacharparenright}{\isachardoublequoteclose}{\isacharbrackright}\ \isacommand{by}\isamarkupfalse%
\ simp
\end{isabelle}
%
where the final step depends on the previously
proved lemma
\isatt{strengthening{\isacharunderscore}mem} and the characterization of
$\forceisa$ for membership.


The case of equality is entirely analogous, and the \isatt{Nand} and
\isatt{Forall} cases are handled very simply.
%
\begin{isabelle}
\isacommand{next}\isamarkupfalse%
\isanewline
\ \ \isacommand{case}\isamarkupfalse%
\ {\isacharparenleft}Equal\ n\ m{\isacharparenright}\isanewline
\ \ \dots\isanewline
\isacommand{next}\isamarkupfalse%
\isanewline
\ \ \isacommand{case}\isamarkupfalse%
\ {\isacharparenleft}Nand\ {\isasymphi}\ {\isasympsi}{\isacharparenright}\isanewline
\ \ \isacommand{with}\isamarkupfalse%
\ assms\isanewline
\ \ \isacommand{show}\isamarkupfalse%
\ {\isacharquery}case\ \isanewline
\ \ \ \ \isacommand{using}\isamarkupfalse%
\ Forces{\isacharunderscore}Nand\ Transset{\isacharunderscore}intf{\isacharbrackleft}OF\ trans{\isacharunderscore}M\ {\isacharunderscore}\ P{\isacharunderscore}in{\isacharunderscore}M{\isacharbrackright}\ pair{\isacharunderscore}in{\isacharunderscore}M{\isacharunderscore}iff\isanewline
\ \ \ \ \ \ Transset{\isacharunderscore}intf{\isacharbrackleft}OF\ trans{\isacharunderscore}M\ {\isacharunderscore}\ leq{\isacharunderscore}in{\isacharunderscore}M{\isacharbrackright}\ leq{\isacharunderscore}transD\ \isacommand{by}\isamarkupfalse%
\ auto\isanewline
\isacommand{next}\isamarkupfalse%
\isanewline
\ \ \isacommand{case}\isamarkupfalse%
\ {\isacharparenleft}Forall\ {\isasymphi}{\isacharparenright}\isanewline
\ \ \isacommand{with}\isamarkupfalse%
\ assms\isanewline
\ \ \isacommand{have}\isamarkupfalse%
\ {\isachardoublequoteopen}p\ {\isasymtturnstile}\ {\isasymphi}\ {\isacharparenleft}{\isacharbrackleft}x{\isacharbrackright}\ {\isacharat}\ env{\isacharparenright}{\isachardoublequoteclose}\ \isakeyword{if}\ {\isachardoublequoteopen}x{\isasymin}M{\isachardoublequoteclose}\ \isakeyword{for}\ x\isanewline
\ \ \ \ \isacommand{using}\isamarkupfalse%
\ that\ Forces{\isacharunderscore}Forall\ \isacommand{by}\isamarkupfalse%
\ simp\isanewline
\ \ \isacommand{with}\isamarkupfalse%
\ \underline{Forall}\ \isanewline
\ \ \isacommand{have}\isamarkupfalse%
\ {\isachardoublequoteopen}r\ {\isasymtturnstile}\ {\isasymphi}\ {\isacharparenleft}{\isacharbrackleft}x{\isacharbrackright}\ {\isacharat}\ env{\isacharparenright}{\isachardoublequoteclose}\ \isakeyword{if}\ {\isachardoublequoteopen}x{\isasymin}M{\isachardoublequoteclose}\ \isakeyword{for}\ x\isanewline
\ \ \ \ \isacommand{using}\isamarkupfalse%
\ that\ pred{\isacharunderscore}le{\isadigit{2}}\ \isacommand{by}\isamarkupfalse%
\ {\isacharparenleft}simp{\isacharparenright}\isanewline
\ \ \isacommand{with}\isamarkupfalse%
\ assms\ \underline{Forall}\isanewline
\ \ \isacommand{show}\isamarkupfalse%
\ {\isacharquery}case\ \isanewline
\ \ \ \ \isacommand{using}\isamarkupfalse%
\ Forces{\isacharunderscore}Forall\ \isacommand{by}\isamarkupfalse%
\ simp\isanewline
\isacommand{qed}\isamarkupfalse
\end{isabelle}
%
It can be noted that the inductive hypothesis
gets used in the last case (underlined here as
\isatt{\underline{Forall}}), but not in the case for \isatt{Nand}.

%%% Local Variables: 
%%% mode: latex
%%% TeX-master: "forcing_in_isabelle_zf"
%%% ispell-local-dictionary: "american"
%%% End: 
