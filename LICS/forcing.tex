%%%%%%%%%%%%%%%%%%%%%%%%%%%%%%%%%%%%%%%%%%%%%%%%%%%%%%%%%%%%%%%%%%%%%%          
\section{Forcing}
Given a ctm $M$, and an $M$-generic filter $G\sbq\PP$, the Forcing
Theorems relate satisfaction of a formula 
$\phi$ in the generic extension $M[G]$ to the satisfaction of another
formula $\forceisa(\phi)$ in $M$. The map $\forceisa$ is defined by
recursion on 
the structure of $\phi$. In order to state the properties of this map
in sufficient generality to prove that  $M[G]$ satisfies $\ZFC$, we work with
internalized formulas, because it is not possible to carry inductive
arguments over \tyo. These internalized versions are the elements of
the recursively constructed set \formula{} (of type \tyi).

We will now make more precise the definition of the map
$\forceisa$ and how it relates satisfaction in $M$ to that in
$M[G]$. Actually, if the formula $\phi$ has $n$ free variables,
$\forceisa(\phi)$ will have $n+4$ free variables, where the first four account
for the forcing notion and a particular condition. 

In referring to formulas with free variables and their interpretation,
we use the following conventions. If  $M$ is a set, $a_0,\dots,a_n$
are elements of $M$, and $\phi$ has its free variables
among $x_0,\dots,x_n$, we write
\[
M,[a_0,\dots,a_n] \models \phi(x_0,\dots,x_n)
\]
to denote that $\phi$ is satisfied by $M$ when $x_i$ is interpreted
to be $a_i$ ($i=0,\dots,n$). We call the list $[a_0,\dots,a_n]$ the
\emph{environment}. In the case of a formula of the form
$\forceisa(\phi)$, we'll make an abuse of notation and indicate the
variables inside the argument of $\forceisa$. As an example, take the
formula $\phi\defi x_1\in x_0$. Then
\[
M,[a,b] \models x_1\in x_0
\]
will hold whenever $b\in a$; and instead of writing $\forceisa(\phi)$
we will write $\forceisa(x_5\in x_4)$, as in
\[
M,[\PP,\leq,\1,p,\tau,\rho] \models \forceisa(x_5\in x_4).
\]

If
$\phi=\phi(x_0,\dots,x_n)$, the notation used by Kunen
\cite{kunen2011set,kunen1980} for $\forceisa(\phi)$ is 
\[
p\forces_{\PP,\leq,\1}^* \phi(x_0,\dots,x_n).
\]
Here, the extra parameters are $\PP,\leq,\1,$ and $p\in\PP$, and the
first three are usually omitted. %(since the forcing notion is fixed
% throughout the discussion).  
Afterwards, the \emph{forcing relation}
$\forces$ can be obtained by 
interpreting $\forces^*$ in a ctm $M$, for fixed
$\lb\PP,\leq,\1\rb\in M$: $p\forces \phi(\tau_0,\dots,\tau_n)$ holds
if and only if
\begin{equation}\label{eq:3}
M,[\PP,\leq,\1,p,\tau_0,\dots,\tau_n]\models \forceisa(x_4,\dots,x_{n+4}).
\end{equation}

\subsection{The fundamental theorems}
These are stated in \cite{kunen2011set} as two lemmas: \emph{Truth} and
\emph{Definability}.  Modern treatments of the theory of forcing start
by defining the 
forcing relation semantically and later it is proved  that the
characterization given by (\ref{eq:3}) indeed holds, and hence the
forcing relation is \emph{definable}. In our case, definability merely
stands for a typing condition:
\begin{isabelle}
{\isachardoublequoteopen}{\isasymphi}\ {\isasymin}\ formula\ {\isasymLongrightarrow}\ forces{\isacharparenleft}{\isasymphi}{\isacharparenright}\ {\isasymin}\ formula{\isachardoublequoteclose}
\end{isabelle}

Then the definition of the forcing relation is stated as a
\begin{lemma}[Definition of Forcing]\label{lem:definition-of-forcing}
  Let $M$ be a ctm of $\ZFC$, $\lb\PP,\leq,\1\rb$ a forcing notion
  in $M$, $p\in\PP$, and $\phi(x_0,\dots,x_n)$ a formula in the
  language of set 
  theory with all free variables displayed. Then the
  following are equivalent, for all $\tau_0,\dots,\tau_n\in M$:
  \begin{enumerate}
  \item $M,[\PP,\leq,\1,p,\tau_0,\dots,\tau_n] 
  \models\forceisa(\phi(x_4,\dots,x_{n+4}))$.
  \item For all $M$-generic filters $G$ such that $p\in G$,
    $M[G],[\val(G,\tau_0),\dots,\val(G,\tau_n)] \models\phi(x_0,\dots,x_n)$.
  \end{enumerate}
\end{lemma}
  
The Truth Lemma states that the forcing
relation indeed relates 
satisfaction in $M[G]$ to that in $M$. 
\begin{lemma}[Truth Lemma]\label{lem:truth-lemma}
  Assume the same hypothesis of
  Lemma~\ref{lem:definition-of-forcing}. Then the
  following are equivalent, for all $\tau_0,\dots,\tau_n\in M$, and
  $M$-generic $G$: 
  \begin{enumerate}
  \item $M[G],[\val(G,\tau_0),\dots,\val(G,\tau_n)]
  \models\phi(x_0,\dots,x_n)$.
  \item  There exists $p\in G$ such that $M,[\PP,\leq,\1,p,\tau_0,\dots,\tau_n] 
  \models\forceisa(\phi(x_4,\dots,x_{n+4}))$.
  \end{enumerate}
\end{lemma}
The proof of the Truth Lemma is by recursion in elements of
$\formula$. The following auxiliary results (adapted from
\cite[IV.2.43]{kunen2011set}) are also proved by recursion in
$\formula$.
\begin{lemma}[Strengthening]\label{lem:strengthen} 
  Assume the same hypothesis of Lemma~\ref{lem:definition-of-forcing}.
  $M, [\PP,\leq,\1,p,\dots]\models\forceisa(\phi)$ and $p_1\leq p$
  implies $M, [\PP,\leq,\1,p_1,\dots]\models\forceisa(\phi)$.
\end{lemma}
\begin{lemma}[Density]\label{lem:density}
  Assume the same hypothesis of
  Lemma~\ref{lem:definition-of-forcing}. $M,[\dots]\models\forceisa(\phi)$ 
  if and only if 
  $M,[\dots]\models$``$\{p_1\leq p : \forceisa(\phi,\PP,\leq,\1,p_1)\}$ is
  dense below $p$.''
\end{lemma}

%%% Local Variables: 
%%% mode: latex
%%% TeX-master: "Separation_In_MG"
%%% ispell-local-dictionary: "american"
%%% End: 
