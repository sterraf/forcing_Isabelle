\documentclass[11pt,a4paper]{article}
\usepackage{isabelle,isabellesym}
\usepackage[numbers]{natbib}
\usepackage{babel}

\usepackage{relsize}
\DeclareRobustCommand{\isactrlbsub}{\emph\bgroup\math{}\sb\bgroup\mbox\bgroup\isaspacing\itshape\smaller}
\DeclareRobustCommand{\isactrlesub}{\egroup\egroup\endmath\egroup}
\DeclareRobustCommand{\isactrlbsup}{\emph\bgroup\math{}\sp\bgroup\mbox\bgroup\isaspacing\itshape\smaller}
\DeclareRobustCommand{\isactrlesup}{\egroup\egroup\endmath\egroup}

% further packages required for unusual symbols (see also
% isabellesym.sty), use only when needed

\usepackage{amssymb}
  %for \<leadsto>, \<box>, \<diamond>, \<sqsupset>, \<mho>, \<Join>,
  %\<lhd>, \<lesssim>, \<greatersim>, \<lessapprox>, \<greaterapprox>,
  %\<triangleq>, \<yen>, \<lozenge>

%\usepackage{eurosym}
  %for \<euro>

%\usepackage[only,bigsqcap]{stmaryrd}
  %for \<Sqinter>

%\usepackage{eufrak}
  %for \<AA> ... \<ZZ>, \<aa> ... \<zz> (also included in amssymb)

%\usepackage{textcomp}
  %for \<onequarter>, \<onehalf>, \<threequarters>, \<degree>, \<cent>,
  %\<currency>

% this should be the last package used
\usepackage{pdfsetup}

% urls in roman style, theory text in math-similar italics
\urlstyle{rm}
\isabellestyle{it}

% for uniform font size
%\renewcommand{\isastyle}{\isastyleminor}
\newcommand{\forces}{\Vdash}
\newcommand{\dom}{\mathsf{dom}}
\renewcommand{\isacharunderscorekeyword}{\mbox{\_}}
\renewcommand{\isacharunderscore}{\mbox{\_}}
\renewcommand{\isasymtturnstile}{\isamath{\Vdash}}
\renewcommand{\isacharminus}{-}
\newcommand{\session}[1]{\textit{#1}}
\newcommand{\theory}[1]{\texttt{#1}}
\newcommand{\axiomas}[1]{\mathit{#1}}
\newcommand{\ZFC}{\axiomas{ZFC}}
\newcommand{\ZF}{\axiomas{ZF}}
\newcommand{\AC}{\axiomas{AC}}
\newcommand{\CH}{\axiomas{CH}}

\begin{document}

\title{Transitive Models of Fragments of ZF}
\author{Emmanuel Gunther\thanks{Universidad Nacional de C\'ordoba.
    Facultad de Matem\'atica, Astronom\'{\i}a,  F\'{\i}sica y
    Computaci\'on.}
  \and
  Miguel Pagano\footnotemark[1]
  \and
  Pedro S\'anchez Terraf\footnotemark[1] \thanks{Centro de Investigaci\'on y Estudios de Matem\'atica
    (CIEM-FaMAF), Conicet. C\'ordoba. Argentina.
    Supported by Secyt-UNC project 33620180100465CB.}
  \and
  Mat\'{\i}as Steinberg\footnotemark[1]
}
\maketitle

\begin{abstract}
  We extend the ZF-Constructibility library by relativizing theories
  of the Isabelle/ZF and Delta System Lemma sessions to a transitive
  class. We also relativize Paulson's work on Aleph and our former
  treatment of the Axiom of Dependent Choices. This work is a
  prerequisite to our formalization of the independence of the
  Continuum Hypothesis.
\end{abstract}


\tableofcontents

% sane default for proof documents
\parindent 0pt\parskip 0.5ex

\section{Introduction}

Relativization of concepts is a key tool to obtain results in forcing,
as it is explained in \cite[Sect.~3]{2020arXiv200109715G} and elsewhere.

In this session, we cast some theories in relative form, in a
way that they now refer to a fixed class $M$ as the universe of
discourse. Whenever it was possible, we tried to minimize the changes
to the structure and proof scripts of the absolute concepts. For
this reason, some comments of the original text as well as
outdated \textbf{apply} commands appear profusely in the following
theories.

A repeated pattern that appears is that the relativized result can be
proved \emph{mutatis mutandis}, with remaining proof obligations that
the objects constructed actually belong to the model $M$. Another
aspect was that the management of higher order constructs always posed
some extra problems, already noted by Paulson \cite[Sect.~7.3]{MR2051585}.

In the theory \theory{Lambda\_Replacement} we introduce a new locale assuming
two instances of separation and twelve instances of ``lambda replacements''
(i.e., replacements using definable functions of the form $\lambda x y. y=\langle x, f(x) \rangle$)
that allow for having some form of compositionality of further instances
of separations and replacements.

We proceed to enumerate the theories that were ``ported'' to relative
form, briefly commenting on each of them. Below, we refer to the
original theories as the \emph{source} and, correspondingly, call
\emph{target} the relativized version. We omit the \theory{.thy}
suffixes.

\begin{enumerate}
\item From \session{ZF}:
  \begin{enumerate}
  \item \theory{Univ}. Here we decided to relativize only the term
    \isatt{Vfrom} that constructs the cumulative hierarchy up to some
    ordinal length and starting from an arbitrary set.
  \item \theory{Cardinal}. There are two targets for this source,
    \theory{Least} and \theory{Cardinal\_Relative}. Both require some
    fair amount of preparation, trying to take advantage of absolute
    concepts. It is not straightforward to compare source and targets
    in a line-by-line fashion at this point.
  \item \theory{CardinalArith}. The hardest part was to formalize the
    cardinal successor function. We also disregarded the part treating
    finite cardinals since it is an absolute concept. Apart from that,
    the relative version closely parallels the source.
  \item \theory{Cardinal\_AC}. After some boilerplate, porting was
    rather straightforward, excepting cardinal arithmetic involving
    the higher-order union operator.
  \end{enumerate}
\item From \session{ZF-Constructible}:
  \begin{enumerate}
  \item \theory{Normal}. The target here is \theory{Aleph\_Relative}
    since that is the only concept that we ported. Instead of porting
    all the machinery of normal functions (since it involved
    higher-order variables), we particularized the results for the
    Aleph function. We also used an alternative definition of the
    latter that worked better with our relativization discipline.
  \end{enumerate}
\item From \session{Delta\_System\_Lemma}:
  \begin{enumerate}
  \item \theory{ZF\_Library}. The target includes a big section of
    auxiliary lemmas and commands that aid the relativization. We
    needed to make explicit the witnesses (mainly functions) in some of the
    existential results proved in the source, since only in that way
    we would be able to show that they belonged to the model.
  \item \theory{Cardinal\_Library}. Porting was relatively
    straightforward; most of the extra work laid in adjusting locale
    assumptions to obtain an appropriate context to state and prove
    the theorems.
  \item \theory{Delta\_System}. Same comments as in the case of
    \theory{Cardinal\_Library} apply here.
  \end{enumerate}
\item From \session{Forcing}:
  \begin{enumerate}
  \item \theory{Pointed\_DC}. This case was similar to
    \theory{Cardinal\_AC} above, although a bit of care was needed to
    handle the recursive construction. Also, a fraction of the theory
    \theory{AC} from \session{ZF} was ported here as it was a
    prerequisite. A complete relativization of \theory{AC} would be
    desirable but still missing.
  \end{enumerate}
\end{enumerate}
% generated text of all theories

\input{session}

\bibliographystyle{root}
\bibliography{root}

\end{document}

%%% Local Variables:
%%% mode: latex
%%% TeX-master: t
%%% ispell-local-dictionary: "american"
%%% End:
