%%%%%%%%%%%%%%%%%%%%%%%%%%%%%%%%%%%%%%%%%%%%%%%%%%%%%%%%%%%%%%%%%%%%%%          
\section{Introduction}
% no \IEEEPARstart
Zermelo-Fraenkel Set Theory $\ZF$) has a prominent place among formal
theories. The reason for this is that it formalizes many intuitive
properties of the notion of set. As such, it can be used as a
foundation for mathematics and thus it has been
thoroughly studied. 

One of the main initial concerns was to decide if this foundation
completely axiomatized all of mathematics. As it was shown by G\"odel
in 1931, no  such axiomatization (therefore, including artihmetic) can
be complete. A refinement of this result, is G\"odel's Second
Incompleteness Theorem that says that no consistent formal system that
includes arithmetic can prove its own consistency. As a matter of
fact, many problem within the area of Set Theory, not directly related
to consistency or logic, have been proved to be undecided by the
theory. Two techniques have been developed to do this. One is G\"odel
theory of \emph{inner models}, which started with his model $L$ of
\emph{Axiom of Constructibility} and was used to proved the relative
consistency of the Axiom of Choice and the Generalized Continuum
Hypothesis with $\ZF$.  he has been formalized by Lawrence
Paulson\cite{paulson_2003} in the proof assistant \emph{Isabelle}, 

We follow \cite{kunen2011set}. An introduction to the issues discussed
in this paper can be found in \cite{2018arXiv180705174G}. An
introduction to forcing can be found at \cite{chow-beginner-forcing},
and the book \cite{weaver2014forcing} contains a thorough treament
minimizing the technicalities.

An enumeration of achievements to be presented:
\begin{enumerate}
\item Discussion of the Isabelle view of life.
  \begin{enumerate}
  \item Meta (Pure) vs Object (FOL+ZF) theory.
  \item Object theory is presented axiomatically; in particular,
    first-order ``formulas'' are not such (i.e., the \emph{initial}
    algebra of formulas is \textbf{not} being specified). Thus, we are
    not able to perform recursion on terms of type \tyo.
  \end{enumerate}
\item Discussion of locales.
  \begin{enumerate}
  \item How they help presenting complex results.
  \item Their key role in stating the Fundamental Theorems (without
    defining \isatt{forces}).
  \item \isatt{forces} is defined for elements of \formula.
  \end{enumerate}
\item Enhancements in the recursion machinery.
  \begin{enumerate}
  \item Impact on names.
  \end{enumerate}
\item Hacking of \isatt{ZF-Constructible}.
  \begin{enumerate}
  \item Elimination of the \tyo-schematic Replacement Axiom from
    locales.
  \item Replacing \isatt{M(nat)} by \isatt{M(0)} (merely non
    emptiness) from Relative (and other places?).
  \end{enumerate}
\item Inteface between first-order properties of $M$ and those of
  $\#\#M$.
  \begin{enumerate}
  \item Enumerate properties and statistics.
  \end{enumerate}
\item Details of the proof of Separation.
\end{enumerate}

%%% Local Variables: 
%%% mode: latex
%%% TeX-master: "Separation_In_MG"
%%% ispell-local-dictionary: "american"
%%% End: 
