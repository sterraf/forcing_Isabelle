% This is samplepaper.tex, a sample chapter demonstrating the
% LLNCS macro package for Springer Computer Science proceedings;
% Version 2.20 of 2017/10/04
%
\documentclass[runningheads]{llncs}
%
\usepackage[utf8]{inputenc}
\usepackage{isabelle_indepCH,isabellesym_indepCH}
\input{header-indepCH}
\usepackage{graphicx}
% Used for displaying a sample figure. If possible, figure files should
% be included in EPS format.
%
\hypersetup{
  colorlinks,
  urlcolor={blue},
  linkcolor={blue!50!black},
  citecolor={blue!50!black},
}
% If you use the hyperref package, please uncomment the following line
% to display URLs in blue roman font according to Springer's eBook style:
% \renewcommand\UrlFont{\color{blue}\rmfamily}

\begin{document}
%
\title{Some lessons after the formalization of the ctm approach to forcing%
  \thanks{Supported by Secyt-UNC project 33620180100465CB and Conicet.}%
}
%
\titlerunning{Lessons after formalizing ctm forcing}
% If the paper title is too long for the running head, you can set
% an abbreviated paper title here
%
\author{Emmanuel Gunther\inst{1} \and
Miguel Pagano\inst{1} \and \\
Pedro Sánchez Terraf\inst{1,2}%\orcidID{0000-0003-3928-6942}
\and
Matías Steinberg\inst{1}
}
%
\authorrunning{E.~Gunther, M.~Pagano, P.~Sánchez Terraf, M.~Steinberg}
% First names are abbreviated in the running head.
% If there are more than two authors, 'et al.' is used.
%
\institute{Universidad Nacional de C\'ordoba. 
  \\  Facultad de Matem\'atica, Astronom\'{\i}a,  F\'{\i}sica y
  Computaci\'on.
  \and
    Centro de Investigaci\'on y Estudios de Matem\'atica (CIEM-FaMAF),
    Conicet. C\'ordoba. Argentina. \\
    \email{\{gunther,pagano,sterraf\}@famaf.unc.edu.ar\\
        matias.steinberg@mi.unc.edu.ar}
}
%
\maketitle              % typeset the header of the contribution
%
\begin{abstract}
  We'll discuss some highlights of our computer-verified
  proof of the construction, given a countable transitive set model $M$
  of $\ZFC$, of generic extensions  satisfying $\ZFC+\neg\CH$ and $\ZFC+\CH$.
  In particular,
  we isolated a set $\Delta$ of 39 instances
  of Axiom of Replacement and a function $F$
  such that such that for any finite fragment $\Phi\sbq\ZFC$,
  $F(\Phi)\sbq\ZFC$ is also finite and if
  $M\models F(\Phi) + \Delta$ then $M[G]\models \Phi + \neg \CH$.
  We also obtained the formulas yielded by the Forcing Definability Theorem
  explicitly.

  To achieve this, we worked in the proof assistant \emph{Isabelle},
  basing our development on the theory Isabelle/ZF by L.~Paulson and
  others.

  %% The vantage point of the talk will be that of a mathematician but
  %% elements from the computer science perspective will be
  %% present. Perhaps some myths regarding what can effectively be done
  %% using  proof assistants/checkers will
  %% be dispelled.

  We'll also compare our formalization with the recent one by Jesse
  M.~Han and Floris van Doorn in the proof assistant \emph{Lean}.

  \keywords{forcing \and Isabelle/ZF \and countable transitive models
    \and absoluteness \and generic extension \and constructibility.}
\end{abstract}
%
%
%
\section{Introduction}
\label{sec:introduction}

This paper is the culmination of our project on the computerized
formalization of the undecidability of the Continuum Hypothesis
($\CH$) from Zermelo-Fraenkel set theory with Choice ($\ZFC$), under the
assumption of the existence of a countable transitive model (ctm) of
$\ZFC$. In contrast to our reports of the previous steps towards this
goal
\cite{2018arXiv180705174G,2019arXiv190103313G,2020arXiv200109715G}, we
intend here to present our development to the mathematical logic
community. For this reason, we start with a general discussion around
the formalization of mathematics.

\subsection{Formalized mathematics}
The use of computers to assist the creation and verification of
mathematics has seen a steady grow. But the general awareness on the
matter still seems to be a bit scant (even among mathematicians
involved in foundations), and the venues devoted to the communication
of formalized mathematics are, mainly, computer science journals and
conferences: JAR, ITP, IJCAR, CPP, CICM, and others.

Nevertheless, the discussion about the subject in central mathematical
circles is increasing; there were some hints on the ICM2018 panel on
“machine-assisted” proofs
\cite{https://doi.org/10.48550/arxiv.1809.08062} and a lively
promotion by Kevin Buzzard, during his ICM2022 special plenary lecture
\cite{2021arXiv211211598B}.

%% These assistants provide several dialects, among which we single out:
%% \begin{enumerate}
%% \item Procedural: Useful for exploration/research.
%% \item Declarative: Only one that can be read by humans!
%% \end{enumerate}

Before we start an in-depth discussion, a point should be made clear:
A formalized proof is not the same as an \emph{automatic proof}. The
reader surely understands that, aside from trivial results, no current
technology allows us to state a reasonably complex (and correct) theorem
statement and expect to obtain a proof after hitting ``Enter'', at
least not after a humanly feasible wait. On the other hand, it is
quite possible that the same reader has some mental image that
formalizing a proof requires making each application of Modus Ponens
explicit.

The fact is that \emph{proof assistants} are designed for the human prover to
be able to decompose a statement to be proved into smaller subgoals
which can actually be fed into some automatic tool. The balance between
what these tools are able to handle is not  easily appreciated by
intuition: Sometimes, ``trivial'' steps are not solved by them, which
can result in obvious frustration; but they would quickly solve some
goals that do not look like a ``mere computation.''

To appreciate the extent of mathematics formalizable, it is convenient to recall
some major successful projects, such as the Four Color Theorem
\cite{MR2463991}, the Odd Order Theorem
\cite{10.1007/978-3-642-39634-2_14}, and the proof the Kepler's
Conjecture \cite{MR3659768}. There is a vast mathematical corpus at
the Archive of Formal Proofs (AFP) based on Isabelle; and formalizations of
brand new and complex objects like the definition of perfectoid spaces \cite{10.1145/3372885.3373830}
and the Liquid Tensor Experiment \cite{LTE2020,LTE2021} have been
achieved using Lean.

We'll continue our description of proof assistants in
Section~\ref{sec:proof-assist-isabelle}. We kindly invite the reader
to enrich the previous exposition by reading the apt summary by
A.~Koutsoukou-Argyraki \cite{angeliki} and the interviews
therein; some of the experts consulted have also discussed
in \cite{2022arXiv220704779B} the status of formalized versus standard
proof in mathematics.

\subsection{Our achievements}
We formalized a model-theoretic rendition of forcing (Sect.~\ref{sec:forcing}), showing how to
construct proper extensions of ctms of $\ZF$ (respectively, with
$\AC$), and we formalized the basic forcing notions required to obtain
ctms of $\ZFC + \neg\CH$ and of $\ZFC + \CH$ (Sect.~\ref{sec:models-ch-negation}). No metatheoretic issues
(consistency, FOL calculi, etc) were formalized, so we were mainly
concerned with the mathematics of forcing. Nevertheless, by inspecting
the foundations underlying our proof assistant Isabelle
(Section~\ref{sec:isabelle-metalogic-meta}) it can be stated that our
formalization is a bona fide proof in $\ZF$ of the previous
constructions.

In order to reach our goals, we provided basic results that were
missing from Isabelle's $\ZF$ library, starting from ones
involving cardinal successors, countable sets, etc.
(Section~\ref{sec:extension-isabellezf}). We also extended the treatment of relativization of
set-theoretical concepts (Section~\ref{sec:tools-relativization}).
%% We redesigned Isabelle/ZF results on non-absolute concepts to work
%% relative to a class.

One added value that is obtained from the present formalization is
that we identified a handful of instances of Replacement which are
sufficient to set the forcing machinery up
(Section~\ref{sec:repl-instances}), on the basis of Zermelo set theory.
The eagerness to obtain this level of detail might be a consequence of
“an unnatural tendency to investigate, for the most part, trivial
minutiae of the formalism” on our part, as it was put by Cohen
\cite{zbMATH02012060}, but we'd rather say that we were driven by
curiosity.

The latest version of our formalization can be accessed at the
development site of the AFP, via the following link:
\begin{center}
  \url{https://devel.isa-afp.org/entries/Independence_CH.html}
\end{center}
%%% Local Variables: 
%%% mode: latex
%%% TeX-master: "independence_ch_isabelle"
%%% ispell-local-dictionary: "american"
%%% End: 

%% 
%% \section{Isabelle and (meta)theories}
\label{sec:isabelle-metatheories}

\begin{enumerate}
\item \emph{Pure} (meta-theory).
\item Isabelle/ZF + $\exists M$ ctm of $\ZF(C)$ (theory). 
\item $\formula$ (inner theory).
\item \textbf{Theorem}: $\exists N$ ctm of $\ZF(C)+\neg\CH$.
\end{enumerate}

in comparison to the ``proof theory approach'':

\begin{enumerate}
\item Primitive recursive arithmetic (meta-theory).
\item \textbf{Theorem}: Con($\ZF(C)$) $\implies$ Con($\ZF(C)+\neg\CH$).
\end{enumerate}

and then to the ``type-theoretic approach'':

\begin{enumerate}
\item CiC (meta-theory).
\item CiC (theory, $\geq \ZFC + \exists \kappa$ inaccessible).
\item \textbf{Theorem}: Con($\ZF(C)+\neg\CH$).
\end{enumerate}

%%% Local Variables: 
%%% mode: latex
%%% TeX-master: "forcing_in_isabelle_zf"
%%% ispell-local-dictionary: "american"
%%% End: 

%% 
%% \section{Relativization,  absoluteness, and the axioms}
\label{sec:relat-absol}

The concepts of relativization and absoluteness (due to Gödel, in his
proof of the relative consistency of $\AC$ \cite{godel-L}) 
are both
prerrequisites and powerful tools in working with transitive
models. A \emph{class} is simply a predicate $C(x)$ with at least one
free variable $x$.
The \emph{relativization} $\phi^C(\bx)$ of a set-theoretic
definition
$\phi$ (of a relation such
as ``$x$ is a subset of $y$'' or of a function like $y=\P(x)$) to
a class $C$ is obtained by restricting all of its quantifiers to $C$.

\[
x \sbq^C y \equiv \forall z.\ C(z) \longrightarrow (z\in x
\longrightarrow z\in y)
\]

The new formula $\phi^C(\bx)$ corresponds to what is obtained by defining
the concept ``inside'' $C$. In fact, for a class corresponding to a
set $c$ (i.e.\ $C(x) \defi x \in c$), the relativization $\phi^C$ of a 
sentence $\phi$ is equivalent to the satisfaction of $\phi$ in the
first-order model $\lb c, \in\rb$.

It turns out that many concepts mean the
same after relativization to a nonempty transitive class $C$; formally
\[
\forall\bx.\ C(\bx) \longrightarrow (\phi^C(\bx) \longleftrightarrow
\phi(\bx))
\]
When this is the case, we say that the relation defined by $\phi$ is
\emph{absolute for transitive models}. (The absoluteness of functions
moreover requires that
the relativized definition also behaves functionally over $C$.) As
examples, the relation of inclusion $\subseteq$ ---and actually, any
relation defined by a formula (equivalent to one) using only bounded
quantifiers 
$(\forall x\in y)$ and $(\exists x\in y)$--- is absolute for
transitive models. On the contrary, this is not the case with the powerset
operation.

A benefit of the work with transitive models is that many 
concepts concepts (pairs, unions, and fundamentally ordinals) are
uniform across the universe \isatt{i}, a ctm (of an adequate fragment of
$\ZF$) $M$ and any of its extensions $M[G]$

A part of this project is to refactorize Paulson's formalization
\cite{paulson_2003} of Gödel's \cite{godel-L}. The main objective is
to maximize applicability of the relativization machinery by adjusting
the hypothesis of a greater part of it early development. Paulson's
architecture had only in mind the consistency of $\ZFC$, but, for
instance, in order to apply it in the development of forcing, too much
is assumed at the beginning; more seriously, some assumptions can't be
regarded as ``first-order'' (v.g. the Replacement Scheme).

The version of \isatt{ZF-Constructible} we present weakens the
assumptions of many absoluteness theorems to that
of a nonempty transitive class; also there are some stronger results
such as the relativization of powersets.

Apart from the axiom schemes, the $\ZFC$ 
axioms are stated as predicates on classes (that is,
of type \isatt{(i{\isasymRightarrow}o){\isasymRightarrow}o}); some
of them were already defined by Paulson, and this formulation allows a
better interaction with \isatt{ZF-Constructible}. 
The axioms of Pairing, Union, Foundation,
Extensionality, and Infinity are relativizations of the respective
traditional 
first-order sentences to the class argument. For the Axiom of Choice
we selected a version best suited for the work with transitive
models: the relativization of a sentence stating that for every $x$
there is surjection from an ordinal onto $x$. Finally, Separation and
Replacement were treated separately to effectively obtain first-order
versions. It is to be noted that predicates in Isabelle/ZF are akin to
second order variables and thus do not correspond to first-order
formulas; in particular, there is no induction principle for functions
of type \isatt{i{\isasymRightarrow}o}. For that reason, Separation and
Replacement predicate \emph{on the satisfaction} of a formula $\phi$.
With respect to our previous \cite{2019arXiv190103313G}, we lifted the
arity restriction for the parameter $\phi$ on these schemes,
streamlining various proofs as an immediate benefit and avoiding the
need for tupling.

Once this class versions of the axioms are set up, it is not hard to
apply our synthesis method to obtain their internal, first-order
counterparts.\footnote{It should be noted that our method for
  synthetic definition requires that the schematic goal had a
  particular format where the only unbound schematic variable occurs
  as an (open) formula whose satisfaction is equivalent to some
  relativized predicate.}
\begin{framed}
  Some code for axiom synthesis.
\end{framed}

%%% Local Variables: 
%%% mode: latex
%%% TeX-master: "forcing_in_isabelle_zf"
%%% ispell-local-dictionary: "american"
%%% End: 

%% 
%% \section{The definition of $\forceisa$}
\label{sec:definition-forces}

The core of the development is showing the definability of the
relation of forcing. As explained in
\cite[Sect.~8]{2019arXiv190103313G}, this comprises the 
definition of a function $\forceisa$ that maps the set of internal
formulas into itself. 

The definition of $\forceisa$ proceeds by recursion
over the set $\formula$ and its base case, that is, for
atomic formulas, is (in)famously the most complicated one. Actually,
newcomers can be puzzled by the fact that forcing for atomic
formulas is also defined by (mutual) recursion: to know if $\tau_1\in\tau_2$ is
forced, one must check if $\tau_1=\sigma$ is forced for $\sigma$
moving in the transitive closure of $\tau_2$. To disentangle this, one
must realize that this last recursion must be described syntactically:
the definition of $\forceisa(\phi)$ for atomic $\phi$ is then an
internal definition of the alleged recursion on names. 

Our aim was to follow Kunen~\cite[p.~257]{kunen2011set}, where the
following mutual recursion is given:
\begin{multline}\label{eq:def-forcing-equality}
  \forceseq (p,t_1,t_2) \defi 
  \forall s\in\dom(t_1)\cup\dom(t_2).\ \forall q\pleq p .\\
  \forcesmem(q,s,t_1)\lsii 
  \forcesmem(q,s,t_2),
\end{multline}
\begin{multline}\label{eq:def-forcing-membership}
  \forcesmem(p,t_1,t_2) \defi  \forall v\pleq p. \ \exists q\pleq v. \\
  \exists s.\ \exists r\in \PP .\ \lb s,r\rb \in
      t_2 \land q \pleq r \land \forceseq(q,t_1,s)
\end{multline}
Note that the definition of $\forcesmem$ states that the set 
\[
\{q\pleq p : \exists \lb s,r\rb\in t_2 . \ q\pleq r \land \forceseq(q,t_1,s)\}
\]
is dense below $p$.

It was not straightforward to use the recursion machinery of
Isabelle/ZF to define $\forceseq$ and $\forcesmem$. For this, we
defined a relation $\frecR$ on 4-tuples of elements of $M$, proved
that it is wellfounded ---indeed, we also proved an
induction principle for this relation:
%
\begin{isabelle}
\isacommand{lemma}\isamarkupfalse%
\ forces{\isacharunderscore}induction{\isacharcolon}\isanewline
\ \ \isakeyword{assumes}\isanewline
\ \ \ \ {\isachardoublequoteopen}{\isasymAnd}{\isasymtau}\ {\isasymtheta}{\isachardot}\ {\isasymlbrakk}{\isasymAnd}{\isasymsigma}{\isachardot}\ {\isasymsigma}{\isasymin}domain{\isacharparenleft}{\isasymtheta}{\isacharparenright}\ {\isasymLongrightarrow}\ Q{\isacharparenleft}{\isasymtau}{\isacharcomma}{\isasymsigma}{\isacharparenright}{\isasymrbrakk}\ {\isasymLongrightarrow}\ R{\isacharparenleft}{\isasymtau}{\isacharcomma}{\isasymtheta}{\isacharparenright}{\isachardoublequoteclose}\isanewline
\ \ \ \ {\isachardoublequoteopen}{\isasymAnd}{\isasymtau}\ {\isasymtheta}{\isachardot}\ {\isasymlbrakk}{\isasymAnd}{\isasymsigma}{\isachardot}\ {\isasymsigma}{\isasymin}domain{\isacharparenleft}{\isasymtau}{\isacharparenright}\ {\isasymunion}\ domain{\isacharparenleft}{\isasymtheta}{\isacharparenright}\ {\isasymLongrightarrow}\ R{\isacharparenleft}{\isasymsigma}{\isacharcomma}{\isasymtau}{\isacharparenright}\ {\isasymand}\ R{\isacharparenleft}{\isasymsigma}{\isacharcomma}{\isasymtheta}{\isacharparenright}{\isasymrbrakk}\isanewline
\ \ \ \ \ \  {\isasymLongrightarrow}\ Q{\isacharparenleft}{\isasymtau}{\isacharcomma}{\isasymtheta}{\isacharparenright}{\isachardoublequoteclose}\isanewline
\ \ \isakeyword{shows}\isanewline
\ \ \ \ {\isachardoublequoteopen}Q{\isacharparenleft}{\isasymtau}{\isacharcomma}{\isasymtheta}{\isacharparenright}\ {\isasymand}\ R{\isacharparenleft}{\isasymtau}{\isacharcomma}{\isasymtheta}{\isacharparenright}{\isachardoublequoteclose}
\end{isabelle}
%
and 
obtained both functions as cases of a another one, 
$\forcesat$, using a single recursion on $\frecR$. Then we obtained 
(\ref{eq:def-forcing-equality}) and (\ref{eq:def-forcing-membership})
as our corollaries \isatt{def{\isacharunderscore}forces{\isacharunderscore}eq} and
\isatt{def{\isacharunderscore}forces{\isacharunderscore}mem}.

Other approaches, like the one in Neeman~\cite{neeman-course} (and
Kunen's older \cite{kunen1980}), prefer
to have a single, more complicated, definition by simple recursion for
$\forceseq$ and then define $\forcesmem$ explicitly. On hindsight,
this might have been a little simpler to do, but we preferred to be as
faithful to the text as possible concerning this point.

Once $\forcesat$ and its relativized version $\isatt{is{\isacharunderscore}forces{\isacharunderscore}at}$
were defined, we proceeded to show absoluteness and provided internal
definitions for the recursion on names using results in
\isatt{ZF-Constructible}. This finished the atomic case of the
formula-transformer $\forceisa$. 

\begin{framed}
  In order to prove absoluteness of functions defined by wellfounded
  recursion, we need to show that we can replace the relevant
  wellfounded relations by their transitive closures. (EXPAND HERE OR
  SOMEWHERE ELSE).
\end{framed}


%%% Local Variables: 
%%% mode: latex
%%% TeX-master: "forcing_in_isabelle_zf"
%%% ispell-local-dictionary: "american"
%%% End: 

%% 
%% \section{The forcing theorems}
\label{sec:forcing-theorems}

After the definition of $\forceisa$ is complete, the proof of the
Fundamental Theorems of Forcing is comparatively straightforward, and
we were able to follow Kunen very closely. The more involved points of
this part of the development were those where we needed to proved that
various (dense) subsets of $\PP$ were in $M$; for this, we had to
recourse to several ad-hoc absoluteness lemmas.

The first results concern characterizations of the forcing
relation. Two of them are \isatt{Forces{\isacharunderscore}Member}:
\begin{center}
  \isatt{{\isacharparenleft}p\ {\isasymtturnstile}\ Member{\isacharparenleft}n{\isacharcomma}m{\isacharparenright}\ env{\isacharparenright}\ {\isasymlongleftrightarrow}\ forces{\isacharunderscore}mem{\isacharparenleft}p{\isacharcomma}t{\isadigit{1}}{\isacharcomma}t{\isadigit{2}}{\isacharparenright}},
\end{center}
where \isatt{t{\isadigit{1}}} and \isatt{t{\isadigit{1}}} are the
\isatt{n}th resp.\ \isatt{m}th elements of \isatt{env}, and  \isatt{Forces{\isacharunderscore}Forall}:
\begin{center}
  \isatt{{\isacharparenleft}p\ {\isasymtturnstile}\ Forall{\isacharparenleft}{\isasymphi}{\isacharparenright}\ env{\isacharparenright}\ {\isasymlongleftrightarrow}\ {\isacharparenleft}{\isasymforall}x{\isasymin}M{\isachardot}\ {\isacharparenleft}p\ {\isasymtturnstile}\ {\isasymphi}\ {\isacharparenleft}{\isacharbrackleft}x{\isacharbrackright}\ {\isacharat}\ env{\isacharparenright}{\isacharparenright}{\isacharparenright}}.
\end{center}
Equivalent statements, along with the ones corresponding to \isatt{Forces{\isacharunderscore}Equal} and
\isatt{Forces{\isacharunderscore}Nand}, appear in Kunen as the
inductive definition of the forcing relation \cite[Def.~IV.2.42]{kunen2011set}.

As with the previous section, the proofs of the forcing theorems have two different
flavours: The ones for the atomic formulas proceed by using the
principle of 
\isatt{forces{\isacharunderscore}induction}, and then an induction on
$\formula$ wraps the former with the remaining cases (\isatt{Nand} and \isatt{Forall}). 

As an example of the first class, we can take a look at our
formalization of \cite[Lem.~IV.2.40(a)]{kunen2011set}:

\begin{isabelle}
  \isacommand{lemma}\isamarkupfalse%
  \ IV{\isadigit{2}}{\isadigit{4}}{\isadigit{0}}a{\isacharcolon}\isanewline
  \ \ \isakeyword{assumes}\isanewline
  \ \ \ \ {\isachardoublequoteopen}M{\isacharunderscore}generic{\isacharparenleft}G{\isacharparenright}{\isachardoublequoteclose}\isanewline
  \ \ \isakeyword{shows}\ \isanewline
  \ \ \ \ {\isachardoublequoteopen}{\isacharparenleft}{\isasymtau}{\isasymin}M{\isasymlongrightarrow}{\isasymtheta}{\isasymin}M{\isasymlongrightarrow}{\isacharparenleft}{\isasymforall}p{\isasymin}G{\isachardot}forces{\isacharunderscore}eq{\isacharparenleft}p{\isacharcomma}{\isasymtau}{\isacharcomma}{\isasymtheta}{\isacharparenright}{\isasymlongrightarrow}val{\isacharparenleft}G{\isacharcomma}{\isasymtau}{\isacharparenright}{\isacharequal}val{\isacharparenleft}G{\isacharcomma}{\isasymtheta}{\isacharparenright}{\isacharparenright}{\isacharparenright}{\isasymand}\isanewline
  \ \ \ \ \ {\isacharparenleft}{\isasymtau}{\isasymin}M{\isasymlongrightarrow}{\isasymtheta}{\isasymin}M{\isasymlongrightarrow}{\isacharparenleft}{\isasymforall}p{\isasymin}G{\isachardot}forces{\isacharunderscore}mem{\isacharparenleft}p{\isacharcomma}{\isasymtau}{\isacharcomma}{\isasymtheta}{\isacharparenright}{\isasymlongrightarrow}val{\isacharparenleft}G{\isacharcomma}{\isasymtau}{\isacharparenright}{\isasymin}val{\isacharparenleft}G{\isacharcomma}{\isasymtheta}{\isacharparenright}{\isacharparenright}{\isacharparenright}{\isachardoublequoteclose}
\end{isabelle}
%
Its proof starts by an introduction of \isatt{forces{\isacharunderscore}induction};
the  inductive cases for each atomic type were handled before as
separate lemmas (\isatt{IV240a{\isacharunderscore}mem} and \isatt{IV240a{\isacharunderscore}eq}). We
illustrate with the statement of the latter.
%
\begin{isabelle}
\isacommand{lemma}\isamarkupfalse%
\ IV{\isadigit{2}}{\isadigit{4}}{\isadigit{0}}a{\isacharunderscore}eq{\isacharcolon}\isanewline
\ \ \isakeyword{assumes}\isanewline
\ \ \ \ {\isachardoublequoteopen}M{\isacharunderscore}generic{\isacharparenleft}G{\isacharparenright}{\isachardoublequoteclose}\ {\isachardoublequoteopen}p{\isasymin}G{\isachardoublequoteclose}\ {\isachardoublequoteopen}forces{\isacharunderscore}eq{\isacharparenleft}p{\isacharcomma}{\isasymtau}{\isacharcomma}{\isasymtheta}{\isacharparenright}{\isachardoublequoteclose}\isanewline
\ \ \ \ \isakeyword{and}\isanewline
\ \ \ \ IH{\isacharcolon}{\isachardoublequoteopen}{\isasymAnd}q\ {\isasymsigma}{\isachardot}\ q{\isasymin}P\ {\isasymLongrightarrow}\ q{\isasymin}G\ {\isasymLongrightarrow}\ {\isasymsigma}{\isasymin}domain{\isacharparenleft}{\isasymtau}{\isacharparenright}\ {\isasymunion}\ domain{\isacharparenleft}{\isasymtheta}{\isacharparenright}\ {\isasymLongrightarrow}\ \isanewline
\ \ \ \ \ \ \ \ {\isacharparenleft}forces{\isacharunderscore}mem{\isacharparenleft}q{\isacharcomma}{\isasymsigma}{\isacharcomma}{\isasymtau}{\isacharparenright}\ {\isasymlongrightarrow}\ val{\isacharparenleft}G{\isacharcomma}{\isasymsigma}{\isacharparenright}\ {\isasymin}\ val{\isacharparenleft}G{\isacharcomma}{\isasymtau}{\isacharparenright}{\isacharparenright}\ {\isasymand}\isanewline
\ \ \ \ \ \ \ \ {\isacharparenleft}forces{\isacharunderscore}mem{\isacharparenleft}q{\isacharcomma}{\isasymsigma}{\isacharcomma}{\isasymtheta}{\isacharparenright}\ {\isasymlongrightarrow}\ val{\isacharparenleft}G{\isacharcomma}{\isasymsigma}{\isacharparenright}\ {\isasymin}\ val{\isacharparenleft}G{\isacharcomma}{\isasymtheta}{\isacharparenright}{\isacharparenright}{\isachardoublequoteclose}\isanewline
\ \ \isakeyword{shows}\isanewline
\ \ \ \ {\isachardoublequoteopen}val{\isacharparenleft}G{\isacharcomma}{\isasymtau}{\isacharparenright}\ {\isacharequal}\ val{\isacharparenleft}G{\isacharcomma}{\isasymtheta}{\isacharparenright}{\isachardoublequoteclose}
\end{isabelle}

Examples of proofs  using the second kind of induction include
the basic \isatt{strengthening{\isacharunderscore}lemma} and the main
results in this section, the lemmas of  Density  (actually, its nontrivial
direction
\isatt{dense{\isacharunderscore}below{\isacharunderscore}imp{\isacharunderscore}forces})
and Truth, 
which we reproduce next.
%
\begin{isabelle}
\isacommand{lemma}\isamarkupfalse%
\ density{\isacharunderscore}lemma{\isacharcolon}\isanewline
\ \ \isakeyword{assumes}\isanewline
\ \ \ \ {\isachardoublequoteopen}p{\isasymin}P{\isachardoublequoteclose}\ {\isachardoublequoteopen}{\isasymphi}{\isasymin}formula{\isachardoublequoteclose}\ {\isachardoublequoteopen}env{\isasymin}list{\isacharparenleft}M{\isacharparenright}{\isachardoublequoteclose}\ {\isachardoublequoteopen}arity{\isacharparenleft}{\isasymphi}{\isacharparenright}{\isasymle}length{\isacharparenleft}env{\isacharparenright}{\isachardoublequoteclose}\isanewline
\ \ \isakeyword{shows}\isanewline
\ \ \ \ {\isachardoublequoteopen}{\isacharparenleft}p\ {\isasymtturnstile}\ {\isasymphi}\ env{\isacharparenright}\ {\isasymlongleftrightarrow}\ dense{\isacharunderscore}below{\isacharparenleft}{\isacharbraceleft}q{\isasymin}P{\isachardot}\ {\isacharparenleft}q\ {\isasymtturnstile}\ {\isasymphi}\ env{\isacharparenright}{\isacharbraceright}{\isacharcomma}p{\isacharparenright}{\isachardoublequoteclose}
\end{isabelle}
\begin{isabelle}
\isacommand{lemma}\isamarkupfalse%
\ truth{\isacharunderscore}lemma{\isacharcolon}\isanewline
\ \ \isakeyword{assumes}\ \isanewline
\ \ \ \ {\isachardoublequoteopen}{\isasymphi}{\isasymin}formula{\isachardoublequoteclose}\ {\isachardoublequoteopen}M{\isacharunderscore}generic{\isacharparenleft}G{\isacharparenright}{\isachardoublequoteclose}\isanewline
\ \ \isakeyword{shows}\ \isanewline
\ \ \ \ \ {\isachardoublequoteopen}{\isasymAnd}env{\isachardot}\ env{\isasymin}list{\isacharparenleft}M{\isacharparenright}\ {\isasymLongrightarrow}\ arity{\isacharparenleft}{\isasymphi}{\isacharparenright}{\isasymle}length{\isacharparenleft}env{\isacharparenright}\ {\isasymLongrightarrow}\ \isanewline
\ \ \ \ \ \ {\isacharparenleft}{\isasymexists}p{\isasymin}G{\isachardot}\ {\isacharparenleft}p\ {\isasymtturnstile}\ {\isasymphi}\ env{\isacharparenright}{\isacharparenright}\ {\isasymlongleftrightarrow}\ sats{\isacharparenleft}M{\isacharbrackleft}G{\isacharbrackright}{\isacharcomma}{\isasymphi}{\isacharcomma}map{\isacharparenleft}val{\isacharparenleft}G{\isacharparenright}{\isacharcomma}env{\isacharparenright}{\isacharparenright}{\isachardoublequoteclose}
\end{isabelle}
%
From these results, the semantical characterization of the forcing
relation (the ``definition of $\forces$''
\cite[IV.2.22]{kunen2011set}) follows easily:
\begin{isabelle}
\isacommand{lemma}\isamarkupfalse%
\ definition{\isacharunderscore}of{\isacharunderscore}forces{\isacharcolon}\isanewline
\ \ \isakeyword{assumes}\isanewline
\ \ \ \ {\isachardoublequoteopen}p{\isasymin}P{\isachardoublequoteclose}\ {\isachardoublequoteopen}{\isasymphi}{\isasymin}formula{\isachardoublequoteclose}\ {\isachardoublequoteopen}env{\isasymin}list{\isacharparenleft}M{\isacharparenright}{\isachardoublequoteclose}\ {\isachardoublequoteopen}arity{\isacharparenleft}{\isasymphi}{\isacharparenright}{\isasymle}length{\isacharparenleft}env{\isacharparenright}{\isachardoublequoteclose}\isanewline
\ \ \isakeyword{shows}\isanewline
\ \ \ \ {\isachardoublequoteopen}{\isacharparenleft}p\ {\isasymtturnstile}\ {\isasymphi}\ env{\isacharparenright}\ {\isasymlongleftrightarrow}\isanewline
\ \ \ \ \ {\isacharparenleft}{\isasymforall}G{\isachardot}{\isacharparenleft}M{\isacharunderscore}generic{\isacharparenleft}G{\isacharparenright}{\isasymand}\ p{\isasymin}G{\isacharparenright}{\isasymlongrightarrow}sats{\isacharparenleft}M{\isacharbrackleft}G{\isacharbrackright}{\isacharcomma}{\isasymphi}{\isacharcomma}map{\isacharparenleft}val{\isacharparenleft}G{\isacharparenright}{\isacharcomma}env{\isacharparenright}{\isacharparenright}{\isacharparenright}{\isachardoublequoteclose}
\end{isabelle}

The present statement of the Fundamental Theorems is almost exactly
the same of those of \cite{2019arXiv190103313G}, with the only
modification being the bound on arities and a missing typing
constraint. This implied only minor adjustments in the proofs of the
satisfaction of axioms.

%%% Local Variables: 
%%% mode: latex
%%% TeX-master: "forcing_in_isabelle_zf"
%%% ispell-local-dictionary: "american"
%%% End: 

%% 
%% \section{Example of proper extension}

Even when the axioms of $\ZFC$ are proved in the generic extension,
one can't claim that the magic of forcing has taken place unless one
is able to provide some \emph{proper} extension with the \emph{same
ordinals}. After all, one is assuming from starters a model $M$ of $\ZFC$,
and in some trivial cases $M[G]$ might end up to be exactly $M$; this
is where \emph{proper} enters the stage. But, for instance, in the
presence of large cardinals, a model $M'\supsetneq M$ might be an
end-extension of $M$ ---this is were we ask the two models to have the
same ordinals, the same \emph{height}. 

Two short theory files contains the relevant
results. \texttt{Ordinals\_In\_MG.thy} shows, using the closure of $M$
under ranks, that $M$ and $M[G]$ share the same ordinals (actually,
ranks of elements of $M[G]$ are bounded by the ranks of their names in
$M$):
\begin{isabelle}
\isacommand{lemma}\isamarkupfalse%
\ rank{\isacharunderscore}val{\isacharcolon}\ {\isachardoublequoteopen}rank{\isacharparenleft}val{\isacharparenleft}G{\isacharcomma}x{\isacharparenright}{\isacharparenright}\ {\isasymle}\ rank{\isacharparenleft}x{\isacharparenright}{\isachardoublequoteclose}\isanewline
\isacommand{lemma}\isamarkupfalse%
\ Ord{\isacharunderscore}MG{\isacharunderscore}iff{\isacharcolon}\isanewline
\ \ \isakeyword{assumes}\ {\isachardoublequoteopen}Ord{\isacharparenleft}{\isasymalpha}{\isacharparenright}{\isachardoublequoteclose}\ \isanewline
\ \ \isakeyword{shows}\ {\isachardoublequoteopen}{\isasymalpha}\ {\isasymin}\ M\ {\isasymlongleftrightarrow}\ {\isasymalpha}\ {\isasymin}\ M{\isacharbrackleft}G{\isacharbrackright}{\isachardoublequoteclose}
\end{isabelle}

\texttt{Non\_Constructible.thy} contains the first example of a poset
that interprets the locale
\isatt{forcing{\isacharunderscore}notion}. It is the set of all binary
lists with the partial order given by  ``\isatt{xs} is an extension of
\isatt{ys}.''


%%% Local Variables: 
%%% mode: latex
%%% TeX-master: "forcing_in_isabelle_zf"
%%% ispell-local-dictionary: "american"
%%% End: 

%% 
%% \section{The axioms of replacement and choice}
\label{sec:axioms-replacement-choice}

Apart from the obvious modifications arising from the change in the
statement of the axiom schemes, the proofs that all axioms apart from
those in the title hold in $M[G]$ are obtained exactly as in
\cite{2019arXiv190103313G}. 

\subsection{Replacement}

The proof of the Replacement Axiom scheme in $M[G]$ in Kunen uses the
Reflection Principle relativized to $M$. We took an alternative
pathway, following Neeman \cite{neeman-course}. In his course notes,
he uses the relativization of the cumulative hierarchy of sets. 

The
family of all sets of rank less than $\alpha$ is called
\isatt{Vset}$(\alpha)$ in Isabelle/ZF. We showed, in the theory file
\verb|Relative_Univ.thy|
 the following
relativization and closure results concerning this function, for a
class $M$ satisfying the locale \isatt{M{\isacharunderscore}eclose}
plus the Powerset Axiom and four instances of replacement.
%
\begin{isabelle}
\isacommand{lemma}\isamarkupfalse%
\ Vset{\isacharunderscore}abs{\isacharcolon}\ {\isachardoublequoteopen}{\isasymlbrakk}\ M{\isacharparenleft}i{\isacharparenright}{\isacharsemicolon}\ M{\isacharparenleft}V{\isacharparenright}{\isacharsemicolon}\ Ord{\isacharparenleft}i{\isacharparenright}\ {\isasymrbrakk}\ {\isasymLongrightarrow}\ is{\isacharunderscore}Vset{\isacharparenleft}M{\isacharcomma}i{\isacharcomma}V{\isacharparenright}\ {\isasymlongleftrightarrow}\ V\ {\isacharequal}\ {\isacharbraceleft}x{\isasymin}Vset{\isacharparenleft}i{\isacharparenright}{\isachardot}\ M{\isacharparenleft}x{\isacharparenright}{\isacharbraceright}{\isachardoublequoteclose}
\end{isabelle}
\begin{isabelle}
\isacommand{lemma}\isamarkupfalse%
\ Vset{\isacharunderscore}closed{\isacharcolon}\ {\isachardoublequoteopen}{\isasymlbrakk}\ M{\isacharparenleft}i{\isacharparenright}{\isacharsemicolon}\ Ord{\isacharparenleft}i{\isacharparenright}\ {\isasymrbrakk}\ {\isasymLongrightarrow}\ M{\isacharparenleft}{\isacharbraceleft}x{\isasymin}Vset{\isacharparenleft}i{\isacharparenright}{\isachardot}\ M{\isacharparenleft}x{\isacharparenright}{\isacharbraceright}{\isacharparenright}{\isachardoublequoteclose}
\end{isabelle}
We also have the basic result
\begin{isabelle}
\isacommand{lemma}\isamarkupfalse%
\ M{\isacharunderscore}into{\isacharunderscore}Vset{\isacharcolon}\isanewline
\ \ \isakeyword{assumes}\ {\isachardoublequoteopen}M{\isacharparenleft}a{\isacharparenright}{\isachardoublequoteclose}\isanewline
\ \ \isakeyword{shows}\ {\isachardoublequoteopen}{\isasymexists}i{\isacharbrackleft}M{\isacharbrackright}{\isachardot}\ {\isasymexists}V{\isacharbrackleft}M{\isacharbrackright}{\isachardot}\ ordinal{\isacharparenleft}M{\isacharcomma}i{\isacharparenright}\ {\isasymand}\ is{\isacharunderscore}Vfrom{\isacharparenleft}M{\isacharcomma}{\isadigit{0}}{\isacharcomma}i{\isacharcomma}V{\isacharparenright}\ {\isasymand}\ a{\isasymin}V{\isachardoublequoteclose}
\end{isabelle}
stating that $M$ is included in 
$\union\{\isatt{Vset}(\alpha) : \alpha\in M\}$ (it's actually equal).

For the proof of the Replacement Axiom, we assume that $\phi$ is
functional in its first two variables when interpreted in $M[G]$ and
the first ranges over the domain \isatt{c}${}\in M[G]$. Then we show
that the collection of
all values of the second variable, when the first ranges over
\isatt{c}, belongs to $M[G]$:
%
\begin{isabelle}
\isacommand{lemma}\isamarkupfalse%
\ Replace{\isacharunderscore}sats{\isacharunderscore}in{\isacharunderscore}MG{\isacharcolon}\isanewline
\ \ \isakeyword{assumes}\isanewline
\ \ \ \ {\isachardoublequoteopen}c{\isasymin}M{\isacharbrackleft}G{\isacharbrackright}{\isachardoublequoteclose}\ {\isachardoublequoteopen}env\ {\isasymin}\ list{\isacharparenleft}M{\isacharbrackleft}G{\isacharbrackright}{\isacharparenright}{\isachardoublequoteclose}\isanewline
\ \ \ \ {\isachardoublequoteopen}{\isasymphi}\ {\isasymin}\ formula{\isachardoublequoteclose}\ {\isachardoublequoteopen}arity{\isacharparenleft}{\isasymphi}{\isacharparenright}\ {\isasymle}\ {\isadigit{2}}\ {\isacharhash}{\isacharplus}\ length{\isacharparenleft}env{\isacharparenright}{\isachardoublequoteclose}\isanewline
\ \ \ \ {\isachardoublequoteopen}univalent{\isacharparenleft}{\isacharhash}{\isacharhash}M{\isacharbrackleft}G{\isacharbrackright}{\isacharcomma}\ c{\isacharcomma}\ {\isasymlambda}x\ v{\isachardot}\ {\isacharparenleft}M{\isacharbrackleft}G{\isacharbrackright}\ {\isacharcomma}\ {\isacharbrackleft}x{\isacharcomma}v{\isacharbrackright}{\isacharat}env\ {\isasymTurnstile}\ {\isasymphi}{\isacharparenright}{\isacharparenright}{\isachardoublequoteclose}\isanewline
\ \ \isakeyword{shows}\isanewline
\ \ \ \ {\isachardoublequoteopen}{\isacharbraceleft}v{\isachardot}\ x{\isasymin}c{\isacharcomma}\ v{\isasymin}M{\isacharbrackleft}G{\isacharbrackright}\ {\isasymand}\ {\isacharparenleft}M{\isacharbrackleft}G{\isacharbrackright}\ {\isacharcomma}\ {\isacharbrackleft}x{\isacharcomma}v{\isacharbrackright}{\isacharat}env\ {\isasymTurnstile}\ {\isasymphi}{\isacharparenright}{\isacharbraceright}\ {\isasymin}\ M{\isacharbrackleft}G{\isacharbrackright}{\isachardoublequoteclose}
\end{isabelle}
%
From this, the satisfaction of the Replacement Axiom in $M[G]$ follows
very easily.

The proof of the previous lemma, following Neeman, proceeds as usual
by turning an argument concerning elements of $M[G]$ to one involving
names lying in $M$, and connecting both worlds by using the forcing
theorems. In the case at hand, by functionality of $\phi$ we know that
for every $x\in c\cap M[G]$ there exists exactly one $v\in M[G]$ such
that
\isatt{sats{\isacharparenleft}M{\isacharbrackleft}G{\isacharbrackright}{\isacharcomma}{\isasymphi}{\isacharcomma}{\isacharbrackleft}x{\isacharcomma}v{\isacharbrackright}{\isacharat}env{\isacharparenright}}. Now,
given a name $\pi'\in M$ for $c$, every name of an element of $c$
belongs to $\pi\defi \dom(\pi')\times \PP$, which is easily seen to be
in $M$. We'll use $\pi$ to be the domain in an application of the
Replacement Axiom in $M$. But now, obviously, we have lost
functionality since there are many names $\dot v\in M$ for a fixed $v$
in $M[G]$. To solve this issue, for each $\rho p \defi\lb\rho,p\rb\in
\pi$ we calculate the
minimum rank of some $\tau\in M$ such that 
$p\forces \phi(\rho,\tau,\dots)$ if there is one, or $0$ otherwise. By
Replacement in $M$, we can show that the supremum \isatt{?sup} of these ordinals
belongs to $M$ and we can construct a \isatt{?bigname} $\defi$ 
\isatt{{\isacharbraceleft}x{\isasymin}Vset{\isacharparenleft}{\isacharquery}sup{\isacharparenright}{\isachardot}\ x\ {\isasymin}\ M{\isacharbraceright}\ {\isasymtimes}\ {\isacharbraceleft}one{\isacharbraceright}}
whose interpretation by (any generic) $G$ will include all possible elements
as $v$ above.

The previous calculation required some absoluteness and closure
results regarding the minimum ordinal binder, \isatt{Least}$(Q)$, also
denoted $\mu x. Q(x)$, that can be found in the theory file
\verb|Least.thy|.

\subsection{Choice}
A first important observation is that the proof of $\AC$ in $M[G]$
only requires the assumption that $M$ satisfies (a finite fragment of)
$\ZFC$. There is no need to invoke Choice in the metatheory.

Although our previous version of the developement used $\AC$, that was
only needed to show the Rasiowa-Sikorski Lemma (RSL) for
arbitrary posets. We have modularized the proof of the latter
and now the version for countable posets that we use to show the
existence of generic filters
does not require Choice (as it is well known). We also bundled the
full RSL along with our implementation of the principle of dependent
choices in an independent branch of the dependency graph, which is the
only place where the theory \texttt{ZF.AC} is invoked.

Our statement of the Axiom of Choice is the one preferred for
arguments involving transitive classes satisfying $\ZF$:
%
\begin{center}
\isatt{{\isasymforall}x{\isacharbrackleft}M{\isacharbrackright}{\isachardot}\ {\isasymexists}a{\isacharbrackleft}M{\isacharbrackright}{\isachardot}\ {\isasymexists}f{\isacharbrackleft}M{\isacharbrackright}{\isachardot}\ ordinal{\isacharparenleft}M{\isacharcomma}a{\isacharparenright}\ {\isasymand}\ surjection{\isacharparenleft}M{\isacharcomma}a{\isacharcomma}x{\isacharcomma}f{\isacharparenright}}
\end{center}
%
The Simplifier is able to show automatically that this
statement is equivalent to the next one, in which the real notions of
ordinal and surjection appear:
%
\begin{center}
\isatt{{\isasymforall}x{\isacharbrackleft}M{\isacharbrackright}{\isachardot}\ {\isasymexists}a{\isacharbrackleft}M{\isacharbrackright}{\isachardot}\ {\isasymexists}f{\isacharbrackleft}M{\isacharbrackright}{\isachardot}\ Ord{\isacharparenleft}a{\isacharparenright}\ {\isasymand}\ f\ {\isasymin}\ surj{\isacharparenleft}a{\isacharcomma}x{\isacharparenright}}
\end{center}

As with the forcing axioms, the proof of $\AC$ in $M[G]$ follows the pattern of Kunen
\cite[IV.2.27]{kunen2011set} and is rather
straightforward; the only complicated technical point being to show
that the relevant name belongs to $M$. We assume that \isatt{a}${}\neq\emptyset$
belongs to $M[G]$ and has a name $\tau\in M$. By $\AC$ in $M$, there
is a surjection \isatt{s} from an ordinal $\alpha\in M$ ($\subseteq M[G]$) onto
$\dom(\tau)$. Now
%
\begin{center}
\isatt{{\isacharbraceleft}opair{\isacharunderscore}name{\isacharparenleft}check{\isacharparenleft}{\isasymbeta}{\isacharparenright}{\isacharcomma}s{\isacharbackquote}{\isasymbeta}{\isacharparenright}{\isachardot}\ {\isasymbeta}{\isasymin}{\isasymalpha}{\isacharbraceright}\ {\isasymtimes}\ {\isacharbraceleft}one{\isacharbraceright}}
\end{center}
%
is a name for a function \isatt{f} with domain $\alpha$ such that \isatt{a}
is included in its range, and where
\isatt{opair{\isacharunderscore}name}$(\sig,\rho)$ is a name for the
ordered pair $\lb\val(G,\sig),\val(G,\rho)\rb$. From this, $\AC$ in
$M[G]$ follows easily.

\subsection{The main theorem}
With all these elements in place, we are able to transcript the main
theorem of our formalization:
\begin{isabelle}
\isacommand{theorem}\isamarkupfalse%
\ extensions{\isacharunderscore}of{\isacharunderscore}ctms{\isacharcolon}\isanewline
\ \ \isakeyword{assumes}\ \isanewline
\ \ \ \ {\isachardoublequoteopen}M\ {\isasymapprox}\ nat{\isachardoublequoteclose}\ {\isachardoublequoteopen}Transset{\isacharparenleft}M{\isacharparenright}{\isachardoublequoteclose}\ {\isachardoublequoteopen}M\ {\isasymTurnstile}\ ZF{\isachardoublequoteclose}\isanewline
\ \ \isakeyword{shows}\ \isanewline
\ \ \ \ {\isachardoublequoteopen}{\isasymexists}N{\isachardot}\ \isanewline
\ \ \ \ \ \ M\ {\isasymsubseteq}\ N\ {\isasymand}\ N\ {\isasymapprox}\ nat\ {\isasymand}\ Transset{\isacharparenleft}N{\isacharparenright}\ {\isasymand}\ N\ {\isasymTurnstile}\ ZF\ {\isasymand}\ M{\isasymnoteq}N\ {\isasymand}\isanewline
\ \ \ \ \ \ {\isacharparenleft}{\isasymforall}{\isasymalpha}{\isachardot}\ Ord{\isacharparenleft}{\isasymalpha}{\isacharparenright}\ {\isasymlongrightarrow}\ {\isacharparenleft}{\isasymalpha}\ {\isasymin}\ M\ {\isasymlongleftrightarrow}\ {\isasymalpha}\ {\isasymin}\ N{\isacharparenright}{\isacharparenright}\ {\isasymand}\isanewline
\ \ \ \ \ \ {\isacharparenleft}M{\isacharcomma}\ {\isacharbrackleft}{\isacharbrackright}{\isasymTurnstile}\ AC\ {\isasymlongrightarrow}\ N\ {\isasymTurnstile}\ ZFC{\isacharparenright}{\isachardoublequoteclose}
\end{isabelle}
Here, \isatt{\isasymapprox} stands for equipotence, \isatt{nat} is the
set of natural numbers, and the predicate 
\isatt{Transset} indicates transitivity.

%%% Local Variables: 
%%% mode: latex
%%% TeX-master: "forcing_in_isabelle_zf"
%%% ispell-local-dictionary: "american"
%%% End: 

%% 
\section{Random comments}

\begin{enumerate}
\item The concept of absoluteness (\verb|_abs|) splitted into two components
  during the formalization: \verb|is_foo_iff| and \verb|foo_rel_abs| (this last one
  being the concept of absoluteness from  “mathematical practice”) and
  the first one involves a purely relational presentation, introduced
  by Paulson.
\item  \textbf{Pros and cons of an untyped environment}: Closer to the
  set-theoretic point of view/prone to the most stupid errors.
\item \textbf{“Pros” and cons of lack of automation}: we had to be extremely
  detailed, which provides explicit information / the formalization
  work was pain in the neck and much lower.
\item In order to extract information concerning the proof, the need for a
  proof assistant with a more computational flavour (Agda, Coq)
  arises! As a second thought, obviously.
\end{enumerate}

%%% Local Variables: 
%%% mode: latex
%%% TeX-master: "independence_ch_isabelle"
%%% ispell-local-dictionary: "american"
%%% End: 


\section{Related work}
\label{sec:related-work}

%% \textbf{Reviewer's comments}
%% {\it
%%   \begin{itemize}
%%   \item There, it would be appropriate to contrast what was done in
%%     Paulson's work on constructibility with the current work on forcing.
%%   \item More to the point, the recent work by Han and van Doorn on
%%     forcing in Lean deserves more discussion.  They have gone further
%%     than the current authors, having proved the independence of the
%%     continuum hypothesis.  They prefer Boolean-valued models as being
%%     more direct in use than the authors' countable transitive models.
%%     \begin{itemize}
%%     \item Readers will want to know whether the type-theoretic approach
%%       is better/worse/just different than using Isabelle/ZF, and
%%     \item are there any benefits to the ctm approach?
%%     \item Is the type-theory encoding of ZF really accurate?
%%     \item How about comparing proofs of equivalent statements in the two
%%       approaches for length and readability?
%%     \end{itemize}
%%   \end{itemize}
%% }

There are various formalizations of Zermelo-Fraenkel set theory in
proof assistants (v.g.\ Mizar, Metamath, and recently Lean
\cite{DBLP:conf/cade/MouraKADR15}) that proceed to different levels of
sophistication. Isabelle/ZF can be regarded as a notational variant of
NGB set theory \cite[Sect.~II.10]{kunen2011set}, because the schemes
of Replacement and Separation feature higher order (free) variables
playing the role of formula variables. It cannot be proved that the
axioms thus written correspond to first order sentences. For this
reason, our relativized versions only apply to set models, where
we restrict those variables to predicates that actually come
from first order formulas. In that sense, the axioms of the locale
\isatt{M{\isacharunderscore}ZF} correspond more faithfully to the
$\ZF$ axioms.

Traditional expositions of the method of forcing
\cite{kunen2011set,Jech_Millennium} are preceded by a study of
relativization and absoluteness. For this reason, it was a natural
choice at the beginning of this project to build on top of Paulson's
formalization of constructibility on Isabelle/ZF, and that was one
of the main early reasons to work on that logic instead of, e.g., HOL
---below we discuss other reasons. In any case, our
development of forcing does not depend on constructibility
itself (in contrast to Cohen's
original presentation, in which ground models are initial segments of the
constructible hierarchy).

A natural question is whether Isabelle/HOL (with a far more solid
framework to work with given its infrastructure and automation) would
have been a better choice than Isabelle/ZF. In fact,
there are two developments of Zermelo-Fraenkel set theory available on
it: \isatt{HOLZF} by Obua \cite{DBLP:conf/ictac/Obua06} and
\isatt{ZFC{\isacharunderscore}in{\isacharunderscore}HOL} by Paulson
\cite{ZFC_in_HOL-AFP}. But these (logically equivalent) frameworks are
higher in consistency strength than Isabelle/ZF. To elaborate on this,
both ZF and HOL are axiomatized on top of Isabelle's metalogic
\emph{Pure}, which is a version of ``intuitionistic higher order
logic.'' In  \cite{Paulson1989} Paulson proves that \emph{Pure}
is sound for intuitionistic first order logic, thus it does not add
any strength to it. On top of this, the axiomatization of Isabelle/ZF
results in a system equiconsistent with $\ZFC$. On the other hand,
showing the consistency of \isatt{HOLZF} (and thus
\isatt{ZFC{\isacharunderscore}in{\isacharunderscore}HOL}) requires
assuming the consistency of $\ZFC$ plus the existence of an
inaccessible cardinal \cite[Sect.~3]{DBLP:conf/ictac/Obua06}. We note,
in contrast, that our extra running assumption of the existence of a
countable transitive model is considerably weaker (directly and
consistency-wise) than the existence of an inaccessible cardinal.

Concerning the formalization of the method of forcing, to the best of
our knowledge there is only one other that deals with forcing for
set theory: the recent \emph{Flypitch} project by Han and van Doorn
\cite{han_et_al:LIPIcs:2019:11074,DBLP:conf/cpp/HanD20}, which
includes a formalization of the independence of CH using the Lean proof
assistant. The Flypitch formalization is largely orthogonal to ours
(it is based on Boolean-valued models, which are interpreted into
type theory through a variant of the Aczel encoding of set theory),
and this precludes a direct comparison of code. But we can highlight
some conceptual differences between our development and the
corresponding fraction of Flypitch.


A first observation concerns consistency strength. The consistency of
Lean requires infinitely many inaccessibles. More precisely, let
Lean$_n$ be the theory of CiC foundations of Lean restricted to $n$
type universes.  Carneiro proved in his MSc thesis~\cite{carneiro-ms-thesis} the consistency of Lean$_n$ from $\ZFC$ plus
the existence of $n$ inaccessible cardinals. It is also reported in
Carneiro's thesis that Werner's results in
\cite{10.5555/645869.668660} can be adapted to show that Lean$_{n+2}$
proves the consistency of the latter theory.  In that sense, although
Flypitch includes proofs of unprovability results in first order
logic, the meta-theoretic machinery used to obtain them is far heavier
than the one we use to operate model-theoretically.

In second place, a formalization of forcing with general partial
orders, generic filters and  ctms has ---in our opinion--- the added value
that this approach is used in an important (perhaps the greatest)
fraction of the literature, both in exposition and in research
articles and monographs. In verifying a piece of mature mathematics as the
present one, representing the actual practice seems paramount to us.
 
Finally, as a matter of taste, one of the main benefits of using transitive
models is that many fundamental notions are absolute and thus most of
the concepts and statements can be interpreted transparently, as we
have noted before. It
also provides a very concrete way to understand generic objects: as
sets that (in the non trivial case) are provably not in the original
model; this dispels any mystical feel around this concept (contrary
to the case when the ground model is the universe of all sets). In
addition, two-valued semantics is closer to our intuition.

%%% Local Variables: 
%%% mode: latex
%%% TeX-master: "forcing_in_isabelle_zf"
%%% ispell-local-dictionary: "american"
%%% End: 


\section{Lessons}\label{sec:lessons}

\subsection{Plan before formalizing}
\begin{enumerate}
\item Ask about your project
  \url{https://mathoverflow.net/q/265435/66044}

  Disclaimer: It may be useless.
\item Antecedents: Precise enumeration of what \session{ZF-Constructible} had.

\item We should had better used predicates for the forcing posets'
  order relations (the way DSL is written), and go for class forcing
  (with $\PP$ a definable subset of $M$). The latter change seems to be
  easy, but the former doesn't.
  
  “Sometimes it felt like sculpting on marble: The fear of having
  carved too deep and hence needing to change the whole stone!”

  Note: This is in direct contradiction with “code fever”
  (\ref{sec:beware-code-fever} below).
\item Read the fine print: Foundations of your proof assistant.
\end{enumerate}

\subsection{Control your bureaucracy. Automate early}
\begin{enumerate}
\item Bureaucracy vs ML programming.
\item The “math” was already formalized on 22 November 2020.
  We finished the last goal on 22 August 2021.
  (Update: 20 November 2021 \& 28 November 2021, for CH)
\item Missing: automation of closure of models under operations.
\item Missing: basic arithmetic for dealing with arities.
\end{enumerate}

\subsection{Beware of scale factors}
\begin{enumerate}
\item It is extremely misleading when automatic tools (\isatt{simp}, \isatt{auto}, etc)
  stop working just because of the sheer size of the goal. Oftentimes,
  in math, we disregard scale issues but they must always be taken
  into account in CS.
\item Example: $\forceisa(0\in 1)$ is expandable,
  $\forceisa(\neg\neg  0\in 1)$ is not.
\item Example: Synthesis of $\forceisa$; could have been fully synthesized,
  but that was dirty “strategy”.
\item The know-how of computer scientists on this kind of engineering is
  very important
\end{enumerate}

\subsection{You might have formalized it, and still be wrong}
\begin{enumerate}
\item Example: restriction of relations.
\item Pollack, “Pollack consistency” by Wiedijk. Opacity of automated
  proofs.
\item Plot twist: You can be right without knowing. Intuition may drive proofs
  even if we are not working on what we believe we are.
\end{enumerate}

\subsection{Beware of the “Code fever”}\label{sec:beware-code-fever}
\begin{enumerate}
\item “We know that doing math is fun---formalization is like DRUGS”
\item Feeling of accomplishment after seeing your writings
  validated beyond reasonable doubt (v.g. cofinality).

\item One easily forgets about the “Power of the Board.”
\end{enumerate}

\subsection{The Devil's on the shortcuts}
\begin{enumerate}
\item
  Our proofs of the “definition of forces” (and many
  consequences) and of the lemma for “forcing a value” of function
  depend on the countability of the ground model. 
\item
  Density arguments (look for “TODO”, “general versions”).
\end{enumerate}

\subsection{Document your project}
\begin{enumerate}
\item \theory{Definitions\_Main}, thanks to Vidnyánszky.
\end{enumerate}

%%% Local Variables: 
%%% mode: latex
%%% TeX-master: "independence_ch_isabelle"
%%% ispell-local-dictionary: "american"
%%% End: 

%%
%% \section{Conclusion}

TODO:
\begin{enumerate}
\item Implement \emph{Basic Set Theory (BST)} by Kunen in
  Constructible: the use of alternatively Replacement or Powerset to
  prove basic absoluteness and closure resuls.
\item Enhance the automatization of formulas
\item Develop the forcing notions to obtain the independence of $\CH$,
  along with the prerrequisite combinatorial results (v.g.\ the
  $\Delta$-system lemma).
\end{enumerate}

%%% Local Variables: 
%%% mode: latex
%%% TeX-master: "forcing_in_isabelle_zf"
%%% ispell-local-dictionary: "american"
%%% End: 

%
% ---- Bibliography ----
%
% BibTeX users should specify bibliography style 'splncs04'.
% References will then be sorted and formatted in the correct style.
%
%\bibliographystyle{splncs04}
\bibliographystyle{mi-estilo-else}
\bibliography{independence_ch_isabelle,clam2021}

\end{document}

%%% Local Variables: 
%%% mode: latex
%%% ispell-local-dictionary: "american"
%%% End: 
