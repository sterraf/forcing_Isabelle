\subsection{A sequence version of $\DC$}\label{sec:sequence-version-dc}
In order to prove the Rasiowa-Sikorski lemma in
general, a version of the Axiom of Choice ($\AC$) must be proved. This
version is known as the \emph{Principle of Dependent Choices}:
\begin{quote}
  ($\DC$) Let $\rr$ be a binary relation on $A$, and $a\in A$. If
  $\forall x\in A\,  \exists y\in A\, x\rr y$, then there exists
  $f:\om\to A$ such that $f(0)=a$ and $f(n)\rr f(n+1)$ for all
  $n\in\om$.
\end{quote}

A different version of $\DC$ (without the constraint $f(0)=a$, but
allowing a parameter that allows to prove an equivalence with $\AC$) was
formalized by Krzysztof Grabczewski as part of the Isabelle/ZF
library, along with many 
statements that follow from or are equivalent to $\AC$. Since the
proofs there required a greater generality, it would have been very
contrived to derive this ``pointed'' version of $\DC$ from that
development. Instead, we preferred to give a direct proof based in the
one that appears in Moschovakis \cite{moschovakis1994notes}. Another
advantage is that this proof depends on the following version of $\AC$
(existence of choice functions on $\P(A)\sm \{\mty\}$):
\[
\exists (s: \P(A)\sm \{\mty\}\to A). \, \forall X\sbq A.\ X\neq \mty \limp
s(X) \in X.
\]
This statement is more standard than ``$\AC$'' as stated in the
Isabelle/ZF development, which involves a silent use of Replacement
since it is written as an axiom scheme. 

The strategy is to define $f$ above by primitive recursion on the
natural, which can be done easily thanks to the package for
definitions by recursion provided by Isabelle/ZF
\cite{paulson1995set,paulson2000fixedpoint}.\footnote{The package
  checks that the recursive definition makes sense; for example, it
  rejects definitions with a missing case.}

\begin{isabelle}\isamarkuptrue%
\isacommand{consts}\isamarkupfalse%
\ dc{\isacharunderscore}witness\ {\isacharcolon}{\isacharcolon}\ {\isachardoublequoteopen}i\ {\isasymRightarrow}\ i\ {\isasymRightarrow}\ i\ {\isasymRightarrow}\ i\ {\isasymRightarrow}\ i\ {\isasymRightarrow}\ i{\isachardoublequoteclose}\isanewline
\isacommand{primrec}\isamarkupfalse%
\isanewline
\ \ wit{\isadigit{0}}\ \ \ {\isacharcolon}\ {\isachardoublequoteopen}dc{\isacharunderscore}witness{\isacharparenleft}{\isadigit{0}}{\isacharcomma}A{\isacharcomma}a{\isacharcomma}s{\isacharcomma}R{\isacharparenright}\ {\isacharequal}\ a{\isachardoublequoteclose}\isanewline
\ \ witrec\ {\isacharcolon}\ {\isachardoublequoteopen}dc{\isacharunderscore}witness{\isacharparenleft}succ{\isacharparenleft}n{\isacharparenright}{\isacharcomma}A{\isacharcomma}a{\isacharcomma}s{\isacharcomma}R{\isacharparenright}\ {\isacharequal}\ s{\isacharbackquote}{\isacharbraceleft}x{\isasymin}A{\isachardot}\ {\isasymlangle}dc{\isacharunderscore}witness{\isacharparenleft}n{\isacharcomma}A{\isacharcomma}a{\isacharcomma}s{\isacharcomma}R{\isacharparenright}{\isacharcomma}x{\isasymrangle}{\isasymin}R\ {\isacharbraceright}{\isachardoublequoteclose}
\end{isabelle}

The function \isatt{dc{\isacharunderscore}witness} has, apart from $A$, $a$ and $R$, a
function $s$ as a parameter. If this function is a selector for
$\P(A)\sm \{\mty\}$, the function $f(n)\defi {}$\isatt{dc{\isacharunderscore}witness}$(n,A,a,s,R)$
will satify $\DC$. Notice that $s$ is a term of type \isatt{i} (a
function construed as a set of pairs) and an expression
\isatt{s{\isacharbackquote}b} is notation for  \isatt{apply(s,b)},
where
\isatt{apply\ {\isacharcolon}{\isacharcolon}\ {\isachardoublequoteopen}i\ {\isasymRightarrow}\ i\ {\isasymRightarrow}\ i{\isachardoublequoteclose}}
is the operation of function application. We will not make a
distinction in our prose and simply write \isatt{s}(\isatt{b}) in this case.

The proof is mostly routine; after a few lemmas we obtain the
following theorem:

\begin{isabelle}
\isacommand{theorem}\isamarkupfalse%
\ pointed{\isacharunderscore}DC\ \ {\isacharcolon}\ {\isachardoublequoteopen}{\isacharparenleft}{\isasymforall}x{\isasymin}A{\isachardot}\ {\isasymexists}y{\isasymin}A{\isachardot}\ {\isasymlangle}x{\isacharcomma}y{\isasymrangle}{\isasymin}\ R{\isacharparenright}\ {\isasymLongrightarrow}\isanewline
\ \ \ \ \ \ \ \ \ \ \ \ \ \ \ \ \ \ \ \ \ \ \ {\isasymforall}a{\isasymin}A{\isachardot}\ {\isacharparenleft}{\isasymexists}f\ {\isasymin}\ nat{\isasymrightarrow}A{\isachardot}\ f{\isacharbackquote}{\isadigit{0}}\ {\isacharequal}\ a\ {\isasymand}\ {\isacharparenleft}{\isasymforall}n\ {\isasymin}\ nat{\isachardot}\ {\isasymlangle}f{\isacharbackquote}n{\isacharcomma}f{\isacharbackquote}succ{\isacharparenleft}n{\isacharparenright}{\isasymrangle}{\isasymin}R{\isacharparenright}{\isacharparenright}{\isachardoublequoteclose}
\end{isabelle}

We need a further, ``diagonal'' version of $\DC$  to prove
Rasiowa-Sikorski. That is, if the assumption holds for a sequence of
relations $S_n$,  then $f(n) \mathrel{S_{n+1}} f(n+1)$.

We first obtain a corollary of $\DC$ changing $A$ for
$A\times\mathtt{nat}$:

\begin{isabelle}
\isacommand{lemma}\isamarkupfalse%
\ aux{\isacharunderscore}DC{\isacharunderscore}on{\isacharunderscore}AxNat{\isadigit{2}}\ {\isacharcolon}\ {\isachardoublequoteopen}{\isasymforall}x{\isasymin}A{\isasymtimes}nat{\isachardot}\ {\isasymexists}y{\isasymin}A{\isachardot}\ {\isasymlangle}x{\isacharcomma}{\isasymlangle}y{\isacharcomma}succ{\isacharparenleft}snd{\isacharparenleft}x{\isacharparenright}{\isacharparenright}{\isasymrangle}{\isasymrangle}\ {\isasymin}\ R\ {\isasymLongrightarrow}\isanewline
\ \ \ \ \ \ \ \ \ \ \ \ \ \ \ \ \ \ {\isasymforall}x{\isasymin}A{\isasymtimes}nat{\isachardot}\ {\isasymexists}y{\isasymin}A{\isasymtimes}nat{\isachardot}\ {\isasymlangle}x{\isacharcomma}y{\isasymrangle}\ {\isasymin}\ {\isacharbraceleft}{\isasymlangle}a{\isacharcomma}b{\isasymrangle}{\isasymin}R{\isachardot}\ snd{\isacharparenleft}b{\isacharparenright}\ {\isacharequal}\ succ{\isacharparenleft}snd{\isacharparenleft}a{\isacharparenright}{\isacharparenright}{\isacharbraceright}{\isachardoublequoteclose}
\end{isabelle}
%
The following lemma is then proved automatically:

\begin{isabelle}
\isacommand{lemma}\isamarkupfalse%
\ aux{\isacharunderscore}sequence{\isacharunderscore}DC{\isadigit{2}}\ {\isacharcolon}\ {\isachardoublequoteopen}{\isasymforall}x{\isasymin}A{\isachardot}\ {\isasymforall}n{\isasymin}nat{\isachardot}\ {\isasymexists}y{\isasymin}A{\isachardot}\ {\isasymlangle}x{\isacharcomma}y{\isasymrangle}\ {\isasymin}\ S{\isacharbackquote}n\ {\isasymLongrightarrow}\isanewline
\ \ \ \ \ \ \ \ {\isasymforall}x{\isasymin}A{\isasymtimes}nat{\isachardot}\ {\isasymexists}y{\isasymin}A{\isachardot}\ {\isasymlangle}x{\isacharcomma}{\isasymlangle}y{\isacharcomma}succ{\isacharparenleft}snd{\isacharparenleft}x{\isacharparenright}{\isacharparenright}{\isasymrangle}{\isasymrangle}\ {\isasymin}\ {\isacharbraceleft}{\isasymlangle}{\isasymlangle}x{\isacharcomma}n{\isasymrangle}{\isacharcomma}{\isasymlangle}y{\isacharcomma}m{\isasymrangle}{\isasymrangle}{\isasymin}{\isacharparenleft}A{\isasymtimes}nat{\isacharparenright}{\isasymtimes}{\isacharparenleft}A{\isasymtimes}nat{\isacharparenright}{\isachardot}\ {\isasymlangle}x{\isacharcomma}y{\isasymrangle}{\isasymin}S{\isacharbackquote}m\ {\isacharbraceright}{\isachardoublequoteclose}\isanewline
\ \ %
\isacommand{by}\isamarkupfalse%
\ auto%
\end{isabelle}
%
And finally, we arrive to $\DC$ for a sequence of relations.

\begin{isabelle}
\isacommand{lemma}\isamarkupfalse%
\ sequence{\isacharunderscore}DC{\isacharcolon}\ {\isachardoublequoteopen}{\isasymforall}x{\isasymin}A{\isachardot}\ {\isasymforall}n{\isasymin}nat{\isachardot}\ {\isasymexists}y{\isasymin}A{\isachardot}\ {\isasymlangle}x{\isacharcomma}y{\isasymrangle}\ {\isasymin}\ S{\isacharbackquote}n\ {\isasymLongrightarrow}\isanewline
\ \ \ \ {\isasymforall}a{\isasymin}A{\isachardot}\ {\isacharparenleft}{\isasymexists}f\ {\isasymin}\ nat{\isasymrightarrow}A{\isachardot}\ f{\isacharbackquote}{\isadigit{0}}\ {\isacharequal}\ a\ {\isasymand}\ {\isacharparenleft}{\isasymforall}n\ {\isasymin}\ nat{\isachardot}\ {\isasymlangle}f{\isacharbackquote}n{\isacharcomma}f{\isacharbackquote}succ{\isacharparenleft}n{\isacharparenright}{\isasymrangle}{\isasymin}S{\isacharbackquote}succ{\isacharparenleft}n{\isacharparenright}{\isacharparenright}{\isacharparenright}{\isachardoublequoteclose}
\end{isabelle}

\subsection{The Rasiowa-Sikorski lemma}\label{sec:rasiowa-sikorski-lemma}
In order to state this Lemma, we gather the relevant hypotheses into a locale:

\begin{isabelle}%
\isacommand{locale}\isamarkupfalse%
\ countable{\isacharunderscore}generic\ {\isacharequal}\ forcing{\isacharunderscore}notion\ {\isacharplus}\isanewline
\ \ \isakeyword{fixes}\ {\isasymD}\isanewline
\ \ \isakeyword{assumes}\ countable{\isacharunderscore}subs{\isacharunderscore}of{\isacharunderscore}P{\isacharcolon}\ \ {\isachardoublequoteopen}{\isasymD}\ {\isasymin}\ nat{\isasymrightarrow}Pow{\isacharparenleft}P{\isacharparenright}{\isachardoublequoteclose}\isanewline
\ \ \isakeyword{and}\ \ \ \ \ seq{\isacharunderscore}of{\isacharunderscore}denses{\isacharcolon}\ \ \ \ \ \ \ \ {\isachardoublequoteopen}{\isasymforall}n\ {\isasymin}\ nat{\isachardot}\ dense{\isacharparenleft}{\isasymD}{\isacharbackquote}n{\isacharparenright}{\isachardoublequoteclose}
\end{isabelle}
%
That is, $\calD$ is a sequence of dense subsets of the poset $P$. A
filter is \emph{$\calD$-generic} if it intersects every dense set in
the sequence.

\begin{isabelle}%
\isacommand{definition}\isamarkupfalse%

\ \ D{\isacharunderscore}generic\ {\isacharcolon}{\isacharcolon}\ {\isachardoublequoteopen}i{\isasymRightarrow}o{\isachardoublequoteclose}\ \isakeyword{where}\isanewline
\ \ {\isachardoublequoteopen}D{\isacharunderscore}generic{\isacharparenleft}G{\isacharparenright}\ {\isacharequal}{\isacharequal}\ filter{\isacharparenleft}G{\isacharparenright}\ {\isasymand}\ {\isacharparenleft}{\isasymforall}n{\isasymin}nat{\isachardot}{\isacharparenleft}{\isasymD}{\isacharbackquote}n{\isacharparenright}{\isasyminter}G{\isasymnoteq}{\isadigit{0}}{\isacharparenright}{\isachardoublequoteclose}
\end{isabelle}

We can now state the Rasiowa-Sikorski Lemma.
\begin{isabelle}%
\isacommand{theorem}\isamarkupfalse%
\ rasiowa{\isacharunderscore}sikorski{\isacharcolon}\isanewline
\ \ {\isachardoublequoteopen}p{\isasymin}P\ {\isasymLongrightarrow}\ {\isasymexists}G{\isachardot}\ p{\isasymin}G\ {\isasymand}\ D{\isacharunderscore}generic{\isacharparenleft}G{\isacharparenright}{\isachardoublequoteclose}
\end{isabelle}

The intuitive argument for the result is simple: Once $p_0=p\in P$ is
fixed, we can recursively choose $p_{n+1}$ such that 
$p_n \geq p_{n+1}\in \calD_n$, since $\calD_n$ is dense in $P$. Then
the filter generated by $\{p_n : n\in \om\}$ intersects each set in
the sequence $\{\calD_n\}_n$. This argument appeals to the sequence
version of $\DC$; we have to prove first that the relevant relation
satisfies its hypothesis:

\begin{isabelle}%
\isacommand{lemma}\isamarkupfalse%
\  RS{\isacharunderscore}relation{\isacharcolon}\isanewline
\ \ \isakeyword{assumes}\isanewline
\ \ \ \ \ \ \ \ {\isadigit{1}}{\isacharcolon}\ \ {\isachardoublequoteopen}x{\isasymin}P{\isachardoublequoteclose}\isanewline
\ \ \ \ \ \ \ \ \ \ \ \ \isakeyword{and}\isanewline
\ \ \ \ \ \ \ \ {\isadigit{2}}{\isacharcolon}\ \ {\isachardoublequoteopen}n{\isasymin}nat{\isachardoublequoteclose}\isanewline
\ \ \isakeyword{shows}\isanewline
\ \ \ \ \ \ \ \ \ \ \ \ {\isachardoublequoteopen}{\isasymexists}y{\isasymin}P{\isachardot}\ {\isasymlangle}x{\isacharcomma}y{\isasymrangle}\ {\isasymin}\ {\isacharparenleft}{\isasymlambda}m{\isasymin}nat{\isachardot}\ {\isacharbraceleft}{\isasymlangle}x{\isacharcomma}y{\isasymrangle}{\isasymin}P{\isacharasterisk}P{\isachardot}\ {\isasymlangle}y{\isacharcomma}x{\isasymrangle}{\isasymin}leq\ {\isasymand}\ y{\isasymin}{\isasymD}{\isacharbackquote}{\isacharparenleft}pred{\isacharparenleft}m{\isacharparenright}{\isacharparenright}{\isacharbraceright}{\isacharparenright}{\isacharbackquote}n{\isachardoublequoteclose}
\end{isabelle}
%
Both of the proofs written in the Isar language.



%%% Local Variables:
%%% mode: latex
%%% ispell-local-dictionary: "american"
%%% TeX-master: "first_steps_into_forcing"
%%% End:
