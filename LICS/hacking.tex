%%%%%%%%%%%%%%%%%%%%%%%%%%%%%%%%%%%%%%%%%%%%%%%%%%%%%%%%%%%%%%%%%%%%%%          
\section{Hacking of \isatt{ZF-Constructible}}
\label{sec:hack-constructible}
In \cite{paulson_2003}, Paulson presented his formalization of the
relative consistency of the Axiom of Choice. This development is
included inside the Isabelle distribution under the session 
\isatt{ZF-Constructible}. The main technical devices, invented by
G\"odel for this purpose, are \emph{relativization} and
\emph{absoluteness}. In a nutshell, to relativize a formula $\phi$ to
a class $C$, it is enough to restrict its quantifiers to $C$. The
example of \isatt{upair\_ax} in
Section~\ref{sec:axioms-models-set-theory}, the relativized version of
the Pairing Axiom, is extracted from \texttt{Relative}, one of the
core theories of \isatt{ZF-Constructible}. On the other hand, $\phi$
is \emph{absolute} for $C$ if it is equivalent to its relativization,
meaning that the statement made by $\phi$ coincides with what $C$
``believes'' $\phi$ is saying. Paulson shows that under certain
hypotheses  on a class $M$ (condensed in the locale \isatt{M\_trivial}), a plethora of
absoluteness and closure results can be proved about $M$.

The development of forcing, and the study of ctms in general, takes
absoluteness as a starting point. We were not able to work with
\isatt{ZF-Constructible} right out-of-the-box. The main reason is that
we can't expect to state the ``class version'' of Replacement for a
\emph{set} $M$ by
using first-order formulas, since predicates
\isatt{P::"i\isasymRightarrow o"} can't
be proved to be only the definable ones. Therefore, we had to make
some modifications in several
locales to make the results available as tools for the present and
future developments.

%% There are several lemmas that were declared, in later developments, as
%% introduction/simplification rules (notably, the rule
%% \isatt{equalityI}). They raised a warning, and we have 
%% eliminated some of them, but in some cases we had to keep them because
%% the proof works because it is insisted that the rule is \emph{safe}.

The most notable changes, located in the theories \texttt{Relative}
and \isatt{WF\_absolute}, are
the following:
\begin{enumerate}
\item\label{item:1} The locale \isatt{M\_trivial}
  does not assume that the underlying class $M$ satisfies the relative
  Axiom of replacement.  As a consequence, the lemma
  \isatt{strong\_replacementI} is no longer valid and was commented
  out.
\item\label{item:2}  Originally the Powerset Axiom was assumed by the
  locale \isatt{M\_trivial},   we moved this requirement to \isatt{M\_basic}. 
\item\label{item:3} We replaced the need that the set of natural
  numbers is in $M$ by the   milder hypothesis that $M(0)$. Actually,
  most results should follow 
  by only assuming that $M$ is non-empty.
\item We moved the requirement $M(\mathtt{nat})$ to the locale
  \isatt{M\_trancl}, where it is needed for the first time. Some results,
  for instance \isatt{rtran\_closure\_mem\_iff} and 
  \isatt{iterates\_imp\_wfrec\_replacement} had to be moved inside that
  locale. 
\end{enumerate}
Because of these changes, some theory files from the
\isatt{ZF-Constructible} session have been included among ours.

The proof, for instance, that the constructible universe $L$ satisfies
the modified locale \isatt{M\_trivial} holds with minor
modifications. Nevertheless, in order to have a neater presentation,
we have stripped off several sections concerning $L$ from the theories
\isatt{L\_axioms} and \isatt{Internalize}, and we merged them to form
the new file  \isatt{Internalizations}. 

%%% Local Variables: 
%%% mode: latex
%%% TeX-master: "Separation_In_MG"
%%% ispell-local-dictionary: "american"
%%% End: 
