\section{Recursions in cofinality and the Delta System Lemma}\label{sec:recursions-cofinality}

As we mentioned near the end of
Section~\ref{sec:aims-formalization-planning}, we decided to minimize
the requirements being formalized in order to achieve our immediate
goal. In particular, the treatment of cofinality in the companion
project \cite{Delta_System_Lemma-AFP} was left behind.

We already observed that well-founded, and in particular transfinite,
recursion is not easily dealt with in Isabelle/ZF. Nevertheless, and
mainly as a curiosity, we found out that only one recursive
construction is needed for the development of the basic theory of
cofinality (as in \cite[Sect.~I.13]{kunen2011set}), which is used in
the proof of the following “factorization” lemma:

\begin{lemma}
  Let $\del,\ga\in\Ord$ and assume $f:\del\to\ga$ is cofinal.  There exists
  a strictly increasing $g:\cf(\ga)\to \del$ such that $f\circ g$ is
  strictly increasing and cofinal in $\ga$. Moreover, if $f$ is
  strictly increasing, then $g$ must also be cofinal.
\end{lemma}

It turns out that the rest of the basic results on cofinality (namely,
idempotence of $\cf$, that regular ordinals are cardinals, the
cofinality of Alephs, König's Theorem) follow easily from the previous
Lemma by “algebraic” reasoning only.
We expect that the relativization of these
results be straightforward, when time permits.

The AFP entry \cite{Delta_System_Lemma-AFP} also includes the
formalization of the (absolute) Delta System Lemma (DSL). Formalizing its
proof was rather straightforward, once the many prerequisites were
taken care of. Some of those were really basic, for instance:
\begin{itemize}
\item $\omega$ injects into every infinite set;
\item surjective images of countable sets are countable;
\item the union over a countable index set $J$ of a family $X :: \tyi
  \fun \tyi$ of countable sets is countable.
\end{itemize}
It was also convenient to isolate the relevant recursive construction
principle (\isatt{bounded{\uscore}cardinal{\uscore}selection}) that
appears in the proof of DSL, which was also useful for showing the
first item:
\begin{lemma}
  Assume $F$ is nonempty,
  $\gamma$ is a cardinal, and $Q$ is a binary predicate over $F$ satisfying 
  \[
    \forall X \sbq F.\ |X| < \gamma \implies \exists a\in F.\ \forall
    x\in X.\ Q(s,a).
  \]
  Then there exists  $S:\gamma \to F$ such that
  $Q(S(\alpha),S(\beta))$ for all $\alpha<\beta<\gamma$.
\end{lemma}

Concerning the relativization of the proof of DSL for its use in
forcing, it required inserting proofs that all the relevant objects
lied in the model $M$; this was only tedious. But the point where the
last item above was used required some expansion due to an application
of Replacement (the relevant $X$ was a class function).


%%% Local Variables:
%%% mode: latex
%%% TeX-master: "independence_ch_isabelle"
%%% ispell-local-dictionary: "american"
%%% End:
