\section{Related work}
\label{sec:related-work}

\paragraph*{Question} How do they overcome Tarski's undefinability of truth?

In some sense, their presentation is mathematically more "mature",
but the type-theoretic machinery might be a bit alien to most
set-theorists.

Quoting Flypitch-CPP,
\begin{quote}
  The types of cardinals and ordinals in mathlib, which
  are defined as equivalence classes of (well-ordered) types, live one
  universe level higher than the types used to construct them
\end{quote}

\paragraph*{Questions} How do they handle recursions? Are these all
\emph{external}? Does this has any impact on their (non?)ability of
identifying replacement instances used by their approach?

They seem to require $\AC$ in their meta-theory.

\subsection{Random comments}

\begin{enumerate}
\item  \textbf{Pros and cons of an untyped environment}: Closer to the
  set-theoretic point of view/prone to the most stupid errors.
\item \textbf{“Pros” and cons of lack of automation}: we had to be extremely
  detailed, which provides explicit information / the formalization
  work was pain in the neck and much lower.
\item In order to extract information concerning the proof, the need for a
  proof assistant with a more computational flavour (Agda, Coq)
  arises! As a second thought, obviously.
\end{enumerate}

%%% Local Variables: 
%%% mode: latex
%%% TeX-master: "independence_ch_isabelle"
%%% ispell-local-dictionary: "american"
%%% End: 
