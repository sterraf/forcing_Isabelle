%%%%%%%%%%%%%%%%%%%%%%%%%%%%%%%%%%%%%%%%%%%%%%%%%%%%%%%%%%%%%%%%%%%%%%          
\section{Generic extensions}

We will swiftly review some definitions in order to reach the concept
of \emph{generic extension}. Given a ctm $M$ of $\ZFC$, a forcing
notion $\lb\PP,\leq,\1\rb$, and 
$G\in\P(\PP)\sm M$, a new set $M[G]$ is defined. Each element $a\in M[G]$ is
determined by an element $\dot a$ of $M$. Actually, the structure of
each $\dot a$ is used to construct $a$. They are related by a
map $\val$ that takes $G$ a parameter:
\[
\val(G,\dot a) = a.
\] 
Then the extension is defined by the image of the map $\val(G,\cdot)$:
\[
M[G] \defi \{\val(G,\tau): \tau\in M\}.
\]
Metatheoretically, it is straightforward to see that $M[G]$ is a
transitive set that satisfies the axioms of Pairing and Union and
includes $M\cup\{G\}$. Nevertheless there are no a priori reasons for
$M[G]$ to satisfy the rest of the $\ZFC$ 
axioms. The original insight by Cohen was to define the notion of
\emph{genericity} for a filter $G\sbq\PP$ and to prove that whenever
$G$ is generic, $M[G]$ will satisfy $\ZFC$.

The Separation Axiom scheme is the first that requires the notion of
genericity and the use of the forcing machinery, which we review in
the next section.

%%% Local Variables: 
%%% mode: latex
%%% TeX-master: "Separation_In_MG"
%%% ispell-local-dictionary: "american"
%%% End: 
