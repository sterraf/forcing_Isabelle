\section{Related work}
\label{sec:related-work}

%% \textbf{Reviewer's comments}
%% {\it
%%   \begin{itemize}
%%   \item There, it would be appropriate to contrast what was done in
%%     Paulson's work on constructibility with the current work on forcing.
%%   \item More to the point, the recent work by Han and van Doorn on
%%     forcing in Lean deserves more discussion.  They have gone further
%%     than the current authors, having proved the independence of the
%%     continuum hypothesis.  They prefer Boolean-valued models as being
%%     more direct in use than the authors' countable transitive models.
%%     \begin{itemize}
%%     \item Readers will want to know whether the type-theoretic approach
%%       is better/worse/just different than using Isabelle/ZF, and
%%     \item are there any benefits to the ctm approach?
%%     \item Is the type-theory encoding of ZF really accurate?
%%     \item How about comparing proofs of equivalent statements in the two
%%       approaches for length and readability?
%%     \end{itemize}
%%   \end{itemize}
%% }

There are various formalizations of Zermelo-Fraenkel set theory in
proof assistants (v.g.\ Mizar, Metamath, and recently Lean
\cite{DBLP:conf/cade/MouraKADR15}) that proceed to different levels of
sophistication. Isabelle/ZF can be regarded as a notational variant of
NGB set theory \cite[Sect.~II.10]{kunen2011set}, because the schemes
of Replacement and Separation feature higher order (free) variables
playing the role of formula variables. It can't be proved that the
axioms thus written correspond to first order sentences. This is the
reason that our relativized versions only apply to set models, where
we restrict those variables to predicates that actually come
from first order formulas. In that sense, the axioms of the locale
\isatt{M{\isacharunderscore}ZF} correspond more faithfully to the
$\ZF$ axioms.

Traditional expositions of the method of forcing
\cite{kunen2011set,Jech_Millennium} are preceded by a study of
relativization and absoluteness. For this reason, it was a natural
choice at the beginning of this project to build on top of Paulson's
formalization of constructibility on Isabelle/ZF, and that was a one
of the main early reasons to work on that logic instead of, e.g., HOL
---there are other reasons, that are discussed below. In any case, our
development of forcing does not depend on constructibility
itself (in contrast to Cohen's
original presentation, in which ground models are initial segments of the
constructible hierarchy).

A natural question is whether Isabelle/HOL (with a far more solid
framework to work with given its infrastructure and automation) would
have been a better choice than Isabelle/ZF. In fact,
there are two developments of Zermelo-Fraenkel set theory available on
it: \isatt{HOLZF} by Obua \cite{DBLP:conf/ictac/Obua06} and
\isatt{ZFC{\isacharunderscore}in{\isacharunderscore}HOL} by Paulson
\cite{ZFC_in_HOL-AFP}. But these (logically equivalent) frameworks are
higher in consistency strength than Isabelle/ZF. To elaborate on this,
both ZF and HOL are axiomatized on top of Isabelle's metalogic
\emph{Pure}, which is a version of ``intuitionistic higher order
logic.'' In  \cite{Paulson1989} Paulson proves that \emph{Pure}
is sound for intuitionistic first order logic, thus it does not add
any strength to it. On top of this, the axiomatization of Isabelle/ZF
results in a system equiconsistent with $\ZFC$. On the other hand,
showing the consistency of \isatt{HOLZF} (and thus
\isatt{ZFC{\isacharunderscore}in{\isacharunderscore}HOL}) requires
assuming the consistency of $\ZFC$ plus the existence of an
inaccesible cardinal \cite[Sect.~3]{DBLP:conf/ictac/Obua06}. We note,
in contrast, that our extra running assumption of the existence of a
countable transitive model is considerably weaker (directly and
consistency-wise) than the existence of an inaccessible cardinal.

Concerning the formalization of the method 
of forcing, all of the previous works (to the best of our knowledge)
have been done interpreting set theory into type theory, and
none of them uses the ctm approach. The
most important is the recent \emph{Flypitch} project  by 
Han and van Doorn
\cite{han_et_al:LIPIcs:2019:11074,DBLP:conf/cpp/HanD20}, which includes
a formalization of the independence of $\CH$ by means
the Boolean-valued approach to forcing, using the Lean
proof assistant. The Flypitch
formalization is largely orthogonal to ours, and this
precludes a direct comparison of code. But we can
highlight some conceptual differences between our development and the
corresponding fraction of Flypitch.

A first observation concerns consistency strength. The consistency of
Lean requires infinitely many inaccessibles. More precisely, let
Lean$_n$ be the theory of CiC foundations of Lean restricted to $n$
type universes.  Carneiro \cite{carneiro-ms-thesis}, proved the
consistency of Lean$_n$ from $\ZFC$ plus the existence of $n$
inaccessible cardinals. It is also reported in
\cite{carneiro-ms-thesis} that Werner's results in
\cite{10.5555/645869.668660} can be adapted to show that Lean$_{n+2}$
proves the consistency of the latter theory.  In that sense, although
Flypitch includes proofs of unprovability results in first order
logic, the meta-theoretic machinery used to obtain them is far heavier
than the one we use to operate model-theoretically.

In second place, a formalization of forcing with general partial
orders, generic filters and  ctms has ---in our opinion--- the added value
that this approach is used in an important (perhaps the greatest)
fraction of the literature, both in exposition and in research
articles and monographs. In verifying a piece of mature mathematics as the
present one, representing the actual practice seems paramount to us.
 
Finally, as a matter of taste, one of the main benefits of using transitive
models is that many fundamental notions are absolute and thus most of
the concepts and statements can be interpreted transparently, as we
have noted before. It
also provides a very concrete way to understand generic objects: as
sets that (in the non trivial case) are provably not in the original
model; this dispells any mystical feel around this concept (contrary
to the case when the ground model is the universe of all sets). In
addition, two-valued semantics is closer to our intuition.

%%% Local Variables: 
%%% mode: latex
%%% TeX-master: "forcing_in_isabelle_zf"
%%% ispell-local-dictionary: "american"
%%% End: 
