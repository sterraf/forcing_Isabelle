\section{22 replacement instances to rule them all}
\label{sec:repl-instances-appendix}

We isolated 22 instances of Replacement that are sufficient to force
$\CH$ or $\neg\CH$, which are enumerated below by the name of the
corresponding internalized first order formula. Many of these were already present in
relational form in the \session{ZF-Constructible} library.

The first 4 instances, collected in the subset
\isatt{instances1{\uscore}fms} of \formula, consist of basic
constructions.

\begin{itemize}
\item 2 instances for transitive closure.
  \begin{itemize}
  \item
    \sout{\isatt{eclose{\uscore}repl1{\uscore}intf{\uscore}fm}.} \isatt{eclose{\uscore}closed{\uscore}fm}.

    To prove closure under iteration of $X\mapsto\union X$.
  \item
    \sout{\isatt{eclose{\uscore}repl2{\uscore}intf{\uscore}fm}} \isatt{eclose{\uscore}abs{\uscore}fm}.

    Auxiliary instance used to show absoluteness.
  \end{itemize}
%% The instances so far
%% are needed to interpret locale
%% \locale{M{\uscore}eclose}.
\item \isatt{wfrec{\uscore}rank{\uscore}fm}.
  
  For $\in$-rank.
  %
\item \sout{\isatt{trans{\uscore}repl{\uscore}HVFrom{\uscore}fm}.} \isatt{transrec{\uscore}VFrom{\uscore}fm}.

  For the cumulative hierarchy (rank initial segments).
\end{itemize}

%% The last two and next pair have the same syntactic structure, because
%% they are definitions by well-founded recursion.
The next 4 instances (gathered in \isatt{instances2{\uscore}fms})
are needed to set up
cardinal arithmetic in $M$.
\begin{itemize}
\item 2 instances for ordertypes:
  \begin{itemize}
  \item
    \sout{\isatt{wfrec{\uscore}replacement{\uscore}order{\uscore}pred{\uscore}fm}.}
    \isatt{wfrec{\uscore}ordertype{\uscore}fm}.

    Well-founded recursion for the construction of ordertypes.
    %
  \item
    \sout{\isatt{replacement{\uscore}is{\uscore}order{\uscore}eq{\uscore}map{\uscore}fm}.}
    \isatt{omap{\uscore}replacement{\uscore}fm}.
    
    Auxiliary instance for the definition of ordertypes.
  \end{itemize}
\item 2 instances for Aleph:
  \begin{itemize}
    %
  \item
    \sout{\isatt{replacement{\uscore}is{\uscore}order{\uscore}body{\uscore}fm}.}
    \isatt{otype{\uscore}replacement{\uscore}fm}.

    Replacement through $x\mapsto \otype(x)$, for Hartogs' Theorem.
    %
  \item
    \sout{\isatt{replacement{\uscore}HAleph{\uscore}wfrec{\uscore}repl{\uscore}body{\uscore}fm}.}
    \isatt{wfrec{\uscore}Aleph{\uscore}fm}.

    The well-founded recursion to define Aleph.
  \end{itemize}
\end{itemize}

We also need a one extra replacement instance $\psi$ on $M$ for each
$\phi$ of the
previous ones to have them in $M[G]$:
\[
  \psi(x,\alpha,y_1,\dots,y_n) \defi \quine{\alpha = \min \bigl\{
    \beta \mid \exists\tau\in V_\beta.\  \mathit{snd}(x) \forces
    \phi\ [\mathit{fst}(x),\tau,y_1,\dots,y_n]\bigr\}}
\]
Here, $\mathit{fst}(\lb a,b\rb) = a$, $\mathit{snd}(\lb a,b\rb) = b$
(with default value $0$ for non pairs).
In our development, the mapping $\phi\mapsto\psi$ defined above is given by the
$\isatt{ground{\uscore}repl{\uscore}fm}$ function, and all ground replacement
instances appear in the locale \locale{M{\uscore}ZF3} and form the set
\isatt{instances3{\uscore}fms}. These are expressed using
the $\isatt{ground{\uscore}replacement{\uscore}assm}(M,\mathit{env},\phi)$ predicate
obtained by replacing $\phi$ by
$\isatt{ground{\uscore}repl{\uscore}fm}(\phi)$ in Eq.~(\ref{eq:replacement_assm_def}).

That makes 16 instances up to now. For the setup of forcing, we
require the following 3 instances, which form the set
\isatt{instances{\uscore}ground{\uscore}fms}.

\begin{itemize}
\item \isatt{wfrec{\uscore}Hcheck{\uscore}fm}.
  
  Well-founded recursion to define check-names.
  %
\item \isatt{wfrec{\uscore}Hfrc{\uscore}at{\uscore}fm}.

  Well-founded recursion for the definition of forcing for atomic formulas.
  %
\item
  \sout{\isatt{Lambda{\uscore}in{\uscore}M{\uscore}fm(check{\uscore}fm(2,0,1),1)}.}
  \isatt{lam{\uscore}replacement{\uscore}check{\uscore}fm}.

  Replacement through $x\mapsto \lb x,\check{x}\rb$ (for the
  definition of $\punto{G}$).
  %
\end{itemize}
The next two formulas, which form the set
\isatt{instances{\uscore}ground{\uscore}notCH{\uscore}fms},
are needed for the $\Delta$-System Lemma.
\begin{itemize}
\item
  \sout{\isatt{replacement{\uscore}is{\uscore}trans{\uscore}apply{\uscore}image{\uscore}fm}.}
  \isatt{recursive{\uscore}construction{\uscore}fm}.

  Recursive construction of sets using a choice function (as in the
  construction of a wellorder of $X$ given a choice function on $\Pow(X)$).
  %
\item
  \sout{\isatt{replacement{\uscore}transrec{\uscore}apply{\uscore}image{\uscore}body{\uscore}fm}.}
  \isatt{recursive{\uscore}construction{\uscore}abs{\uscore}fm}.
  
  Absoluteness of the previous construction.
\end{itemize}
%
The $21$ formulas up to this point are collected into the set
\isatt{overhead{\uscore}notCH}, which is enough to
force $\neg\CH$. To force $\CH$, we required one further instance:
%
\begin{itemize}
\item
  \sout{\isatt{replacement{\uscore}dcwit{\uscore}repl{\uscore}body{\uscore}fm}.}
  \isatt{dc{\uscore}abs{\uscore}fm}.
  
  Absoluteness of the recursive construction in the proof of the
  Dependent Choices from $\AC$.
\end{itemize}

The particular choice of some of the instances above arose from
Paulson's architecture on which we based our development.
This applies every time
a locale from \session{ZF-Constructible} has to be
interpreted (\locale{M{\uscore}eclose} and
\locale{M{\uscore}ordertype}, respectively, for the “auxiliary” instances).
%% For instance, the first
%% instance required for the definition of relative ordertypes arises
%% from Paulson's \session{ZF-Constructible}.
% https://isabelle.in.tum.de/dist/library/ZF/ZF-Constructible/Rank.html#offset_1123..1139

On the other hand, we replaced the original proof of the
Schröder-Bernstein Theorem by Zermelo's one
\cite[Exr. x4.27]{moschovakis1994notes}, because the former required
at least one extra instance
% (\isatt{banach{\uscore}iterates{\uscore}fm})
arising from an iteration. We also managed to avoid 12 further
replacements by restructuring some of original theories in
\session{ZF-Constructible}, so these modifications are included as
part of our project.

It is to be noted that the proofs of the Forcing Theorems do not
require any extra replacement; actually, they only need the 7
instances appearing in \isatt{instances1{\uscore}fms} and
\isatt{instances{\uscore}ground{\uscore}fms}.  But this seems not be
the case for Separation, at least by inspecting our formalization:
More instances holding in $M$ are needed 
as the complexity of $\phi$ grows. One point where this is apparent is
in the proof of Theorem~\ref{th:forcing-thms}(\ref{item:truth-lemma}),
that appears as the \isatt{truth{\uscore}lemma} in our development; it
depends on \isatt{truth{\uscore}lemma'} and
\isatt{truth{\uscore}lemma{\uscore}Neg}, which explicitly invoke
\isatt{separation{\uscore}ax}. In any case, our intended grounds
(v.g., the transitive collapse of countable elementary submodels of a
rank initial segment $V_\alpha$ or an $H(\kappa)$) all satisfy full
Separation.

