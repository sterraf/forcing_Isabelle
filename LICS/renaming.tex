%%%%%%%%%%%%%%%%%%%%%%%%%%%%%%%%%%%%%%%%%%%%%%%%%%%%%%%%%%%%%%%%%%%%%%          
\section{Renaming}

In the course of our work we need to reason about renaming of formulas
and its effect on their satisfiability. Since internalized formulas
are implemented using de Bruijn indices, thus the arity of a formula
$\phi$ gives the least set containing all the free variables in
$\phi$. Following \citet{fiore-abssyn}, one can understand the arity
of a formula as the context of the free variables and renamings are,
consequently, mappings changing the context.


%%% Local Variables: 
%%% mode: latex
%%% TeX-master: "Separation_In_MG"
%%% ispell-local-dictionary: "american"
%%% End: 
