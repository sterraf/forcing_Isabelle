%%%%%%%%%%%%%%%%%%%%%%%%%%%%%%%%%%%%%%%%%%%%%%%%%%%%%%%%%%%%%%%%%%%%%%          
\section{Conclusions and future work}
\label{sec:conclusions-future-work}

% Recordamos el objetivo del proyecto a "medio-término"
% (auto-ironía). Podemos luego mencionar la distinción entre thematic y
% programmatic (describiendo esta también). Luego mostramos ejemplos de
% decisiones en ambos aspectos: Thematic: renaming(?), locales para
% forcing (algo más?); Programmatic: trabajar en o, interfaz entre
% modelo (satisfacción de fórmulas internalizadas) y meta-satisfacción
% de axiomas relativizados en o. Indicando por qué esas decisiones nos
% parecieron adecuadas.

% Luego señalar los logros del paper poniéndolos en perspectiva del
% proyecto, destacando que sólo nos queda un axioma para tener M[G]|=
% ZF. (Alguien quizás se podría preguntar por qué no está C, si es así
% señalarlo). Explayarse sobre la prueba de replacement.

The ultimate goal of our project is a complete mechanization of
forcing allowing for further developments (formalization of the
relative consistency of CH) and even as a research tool for the
working set-theorist. In the current paper we reported a first major
milestone towards that goal; viz. a formal proof in Isabelle/ZF of the
satisfaction by generic extensions of most of the ZFC axioms, except
for the axiom-scheme of Replacement.

We cannot overstate the importance of following the detailed
presentation of forcing, even in its sharp style, given by
\citet{kunen2011set}. In fact, it helped us to delineate the
\emph{thematic} aspects of our formalization; i.e.~the handling of all
the theoretical concepts and results in the subject and it informed
the structure of locales organizing our development.

% In a gross simplification, there are two aspects to a formalization
% project like this one: thematic and programmatic. The first concerns
% the handling of all the theoretical concepts and results in the
% subject, while the second involves the practical issues of the
% implementation and design. In the case of forcing, the main intricacy
% lies in the first aspect. In this sense, following a sensible
% presentation of the material is key.  The authoritative reference 
% on the subject during the last 30 years has been Kunen's classical
% \cite{kunen1980}, and  we have followed a recent rewrite
% \cite{kunen2011set} 
% of that  textbook, which presents the material in the same sharp 
% style but offering a lot of details. In some sense this project
% wouldn't exist without this book.

The proof of Separation depends on the Fundamental Theorems of
forcing; in the spirit of the Formal Abstract project we have only
stated those theorems in our current development, because from a
engineering point of view this can be done independently. The task of
proving the Fundamental Theorems is another landmark of our project,
after completing the formalization that generic extensions satisfy
the axiom scheme of Replacement.

% Next, various forcing notions must be developed in order to
% obtain extensions satisfying diverse properties (v.g., the Continuum
% Hypothesis). This yields relative consistency results.

% We first discuss the axioms we are now missing. We preferred to wait
% to have a full-fledged interface between ctms and the locales
% providing recursive constructions from Paulson's
% \isatt{Constructibility} session. Then the current proof will hold
% with no extra assumptions. The same goes for the results $M\sbq M[G]$
% and $G\in M[G]$.

Let us discuss in more detail the proof of the Replacement Axiom. Both
in Kunen and in other presentations (e.g. \cite{neeman-course}) it
requires a relativized version (i.e., showing that it holds for $M$)
of the \emph{Reflection Principle}. In order to state this
meta-theoretic result by Montague, recall that an equivalent
formulation of the Foundation Axiom states that the universe of sets
can be decomposed in a transfinite hierarchy of sets:
\begin{theorem}
  Let $V_{\al}\defi\union\{\P(V_\be) : \be<\al\}$ for each ordinal
  $\al$. Then each $V_\al$ is a set and 
  $\forall x. \exists\al .\ \Ord(\al) \y x\in V_\al$.  
\end{theorem}
\begin{theorem}[Reflection Principle]\label{th:reflection-principle}
  For every finite $\Phi\sbq\ZF$, $\ZF$ proves: ``There exists
  unboundedly many $\al$ such that $V_\al\models \Phi$.''
\end{theorem}
%% This result has also other applications; in particular, it dispenses
%% the need of ctms of the whole of $\ZF$ for consistency proofs (by
%% repeating  the proof of Lemma~\ref{lem:wf-model-implies-ctm}, now
%% starting with a  $V_\al$ satisfying $\Phi$). 

This statement is straightforwardly equivalent to that with $\Phi$
consisting of a single formula. The schematic nature of this result
hints at a proof by induction on formulas, and hence it must be shown
internally. 

This is an appropriate spot to insist that the internal/external
dichotomy has been a powerful agent in the shaping of our project. It
traces back to the definition of the formula-transformer $\forceisa$,
which must be done by recursion, and this requirement spills
indirectly to the proof of the Separation Axiom (despite the latter is
not by induction). Now the need of using internalized formulas
reappears with the need of having a general version of the Reflection
Principle (because the $\Phi$ there depends on the instance of
Replacement being proved).

A secondary outcome of this project is to assess which assumptions on
the ground model $M$ are needed to develop the forcing machinery. Up
to this point we have some anecdotal data; for example, to show that a
2-variable instance of Separation holds in $M[G]$, one needs to use a
6-variable instance in $M$. In the future, we expect to condense in a
new locale a finite list of (instances of) axioms on $M$ enough to
perform forcing constructions; this list will likely include all the
instances of Separation and Replacement in $M$ that are needed to
satisfy the requirements of the locales in the
\isatt{ZF-Constructible} session.

A few words are in order in relation to \isatt{ZF-Constructible}.
We believe that this  session  would benefit from some changes in
the way some results are organized accross theory files; for
intance, by cataloging all or most of the internalized formulas in one
file. Another, most basic example would be to start out with an even
more locale that only assumes $M$ to be a nonempty transitive class,
as many absoluteness results follow from these hypotheses.
As Paulson comments in the sources, it would have been better to
minimize the use of the Powerset Axiom in locales and proofs. There
are  useful natural models that satisfy a fraction of $\ZF$ not including
this particular axiom, and to ensure a broader applicability, it would
be convenient to have  absoluteness results not assuming it.
It is our desire to advocate a future work to a thorough
revision of the development of constructibility to address these needs
and to maximize modularity.


\section*{Acknowledgment}


%%% Local Variables: 
%%% mode: latex
%%% TeX-master: "Separation_In_MG"
%%% ispell-local-dictionary: "american"
%%% End: 
