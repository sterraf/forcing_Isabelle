\section{Pairing in the generic extension}
\label{sec:pairing-generic-extension}

In this section we show that the generic extension satisfies the
pairing axiom; the purpose of this section is to show how to prove
that $M[G]$ models one of the axioms of $\ZFC$, assuming that $M$
satisfies $\ZFC$.\footnote{The proof that $M[G]$ satisfies pairing
  only needs that $M$ satisfies pairing.} In the locale
\isa{M{\isacharunderscore}extra{\isacharunderscore}assms} we stated
the assumption \isa{sats{\isacharunderscore}upair{\isacharunderscore}ax} which captures that
$M$ satisfies pairing. We use \emph{relativized} versions of the axioms
in order to express satisfaction.

%% MIGUEL: versión vieja
% In the previous section we defined the generic extension $M[G]$; we
% have yet to prove that $M[G]$ is a model of $\ZFC$. This is equivalent
% to show that each axiom relativized with respect to $M[G]$ holds. In
% this section we show that the generic extension satisfies the pairing
% axiom, after explaining relativization in Isabelle/ZF.

As we have already mentioned, in Paulson's library, the relativized
versions of the $\ZFC$ axioms are defined for classes (which are
defined as predicates over sets). The definition
\isa{upair{\isacharunderscore}ax} corresponds to the Pairing Axiom:

\begin{isabelle}
\isacommand{definition}\isamarkupfalse%
\isanewline
\ \ upair\ {\isacharcolon}{\isacharcolon}\ {\isachardoublequoteopen}{\isacharbrackleft}i{\isasymRightarrow}o{\isacharcomma}i{\isacharcomma}i{\isacharcomma}i{\isacharbrackright}\ {\isasymRightarrow}\ o{\isachardoublequoteclose}\ \isakeyword{where}\isanewline
\ \ \ \ {\isachardoublequoteopen}upair{\isacharparenleft}C{\isacharcomma}a{\isacharcomma}b{\isacharcomma}z{\isacharparenright}\ {\isacharequal}{\isacharequal}\ a\ {\isasymin}\ z\ {\isasymand}\ b\ {\isasymin}\ z\ {\isasymand}\ {\isacharparenleft}{\isasymforall}x{\isacharbrackleft}C{\isacharbrackright}{\isachardot}\ x{\isasymin}z\ {\isasymlongrightarrow}\ x\ {\isacharequal}\ a\ {\isasymor}\ x\ {\isacharequal}\ b{\isacharparenright}{\isachardoublequoteclose}
\end{isabelle}
%
%% Then the fact that Pairing axiom holds in \isa{C} is stated as follows.
%
\begin{isabelle}
\isacommand{definition}\isamarkupfalse%
\isanewline
\ \ upair{\isacharunderscore}ax\ {\isacharcolon}{\isacharcolon}\ {\isachardoublequoteopen}{\isacharparenleft}i{\isasymRightarrow}o{\isacharparenright}\ {\isasymRightarrow}\ o{\isachardoublequoteclose}\ \isakeyword{where}\isanewline
\ \ \ \ {\isachardoublequoteopen}upair{\isacharunderscore}ax{\isacharparenleft}C{\isacharparenright}\ {\isacharequal}{\isacharequal}\ {\isasymforall}x{\isacharbrackleft}C{\isacharbrackright}{\isachardot}\ {\isasymforall}y{\isacharbrackleft}C{\isacharbrackright}{\isachardot}\ {\isasymexists}z{\isacharbrackleft}C{\isacharbrackright}{\isachardot}\ upair{\isacharparenleft}C{\isacharcomma}x{\isacharcomma}y{\isacharcomma}z{\isacharparenright}{\isachardoublequoteclose}
\end{isabelle}

% Given a predicate
% \isa{C\ {\isacharcolon}{\isacharcolon}\ {\isachardoublequoteopen}i{\isasymRightarrow}o{\isachardoublequoteclose}},
% we can consider the \emph{class} of all sets \isa{x} such that
% \isa{C(x)}. In \cite{paulson_2003} a large corpus of set-theoretical
% concepts are expressed in relative form.

% In the case of the Pairing Axiom the following definition in 
% Paulson's formalization captures
% what a class \isa{C} believes about a set being the unordered pair of
% other two:

We state the main result of this section in the context
\isa{M{\isacharunderscore}extra{\isacharunderscore}assms}.
%
\begin{isabelle}
\isacommand{lemma}\isamarkupfalse%
\ pairing{\isacharunderscore}axiom\ {\isacharcolon}\ \isanewline
\ \ {\isachardoublequoteopen}one\ {\isasymin}\ G\ {\isasymLongrightarrow}\ upair{\isacharunderscore}ax{\isacharparenleft}{\isacharhash}{\isacharhash}M{\isacharbrackleft}G{\isacharbrackright}{\isacharparenright}{\isachardoublequoteclose}
\end{isabelle}

Let $x$ and $y$ be elements in $M[G]$. By definition of the generic extension, there exist
elements $\tau$ and $\rho$ in $M$ such that $x = \val(G,\tau)$ and
$y = \val(G,\rho)$.  We need to find an element in $M[G]$ that contains exactly
these elements; for that we should construct a name $\sigma\in M$ such that
$\val(G,\sigma) = \{ \val(G,\tau) , \val(G,\rho) \}$. 

The candidate, motivated by the definition of $\chk$,  is
$\sigma = \{\langle \tau , \mathrm{one} \rangle , \langle \rho , \mathrm{one} \rangle \}$. 
Our remaining tasks are to show:
\begin{enumerate}
  \item \label{item:1}$\sigma \in M$, and
  \item \label{item:2} $\val(G,\sigma) = \{ \val(G,\tau) , \val(G,\rho) \}$
\end{enumerate}

By the implementation of pairs  in $\ZFC$, showing (\ref{item:1})
involves using that the
pairing axiom holds in $M$ and the absoluteness of pairing
thanks to $M$ being transitive. 

\begin{isabelle}
\isacommand{lemma}\isamarkupfalse%
\ pairs{\isacharunderscore}in{\isacharunderscore}M\ {\isacharcolon}\ \isanewline
\ \ {\isachardoublequoteopen}\ {\isasymlbrakk}\ a\ {\isasymin}\ M\ {\isacharsemicolon}\ b\ {\isasymin}\ M\ {\isacharsemicolon}\ c\ {\isasymin}\ M\ {\isacharsemicolon}\ d\ {\isasymin}\ M\ {\isasymrbrakk}\ {\isasymLongrightarrow}\ {\isacharbraceleft}{\isasymlangle}a{\isacharcomma}c{\isasymrangle}{\isacharcomma}{\isasymlangle}b{\isacharcomma}d{\isasymrangle}{\isacharbraceright}\ {\isasymin}\ M{\isachardoublequoteclose}
\end{isabelle}

Item (\ref{item:1}) then follows because \isa{\isasymtau}, \isa{\isasymrho} and
\isa{one} belong to \isa{M} (the last fact holds because \isa{one\isasymin P}, \isa{P\isasymin M} and
\isa{M} is transitive).

\begin{isabelle}
\isacommand{lemma}\isamarkupfalse%
\ sigma{\isacharunderscore}in{\isacharunderscore}M\ {\isacharcolon}\isanewline
\ \ {\isachardoublequoteopen} one\ {\isasymin}\ G\ {\isasymLongrightarrow}\ {\isasymtau}\ {\isasymin}\ M\ {\isasymLongrightarrow}\ {\isasymrho}\ {\isasymin}\ M\ {\isasymLongrightarrow}\ {\isacharbraceleft}{\isasymlangle}{\isasymtau}{\isacharcomma}one{\isasymrangle}{\isacharcomma}{\isasymlangle}{\isasymrho}{\isacharcomma}one{\isasymrangle}{\isacharbraceright}\ {\isasymin}\ M{\isachardoublequoteclose}

\isacommand{by}\isamarkupfalse%
\ {\isacharparenleft}rule\ pairs{\isacharunderscore}in{\isacharunderscore}M{\isacharcomma}simp{\isacharunderscore}all\ add{\isacharcolon}\ upair{\isacharunderscore}ax{\isacharunderscore}def\ one{\isacharunderscore}in{\isacharunderscore}M{\isacharparenright}%
\end{isabelle}

Under the assumption that \isa{one} belongs to the set \isa{G},
(\ref{item:2}) follows from \isa{def\_val} almost automatically:

\begin{isabelle}
\isacommand{lemma}\isamarkupfalse%
\ valsigma\ {\isacharcolon}\isanewline
\ \ {\isachardoublequoteopen}one\ {\isasymin}\ G\ {\isasymLongrightarrow}\ {\isacharbraceleft}{\isasymlangle}{\isasymtau}{\isacharcomma}one{\isasymrangle}{\isacharcomma}{\isasymlangle}{\isasymrho}{\isacharcomma}one{\isasymrangle}{\isacharbraceright}\ {\isasymin}\ M\ {\isasymLongrightarrow}\isanewline
\ \ \ val{\isacharparenleft}G{\isacharcomma}{\isacharbraceleft}{\isasymlangle}{\isasymtau}{\isacharcomma}one{\isasymrangle}{\isacharcomma}{\isasymlangle}{\isasymrho}{\isacharcomma}one{\isasymrangle}{\isacharbraceright}{\isacharparenright}\ {\isacharequal}\ {\isacharbraceleft}val{\isacharparenleft}G{\isacharcomma}{\isasymtau}{\isacharparenright}{\isacharcomma}val{\isacharparenleft}G{\isacharcomma}{\isasymrho}{\isacharparenright}{\isacharbraceright}{\isachardoublequoteclose}
\end{isabelle}


%%% Local Variables:
%%% mode: latex
%%% ispell-local-dictionary: "american"
%%% TeX-master: "first_steps_into_forcing"
%%% End:
