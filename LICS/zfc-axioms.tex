%%%%%%%%%%%%%%%%%%%%%%%%%%%%%%%%%%%%%%%%%%%%%%%%%%%%%%%%%%%%%%%%%%%%%%          
\section{Axioms of set theory}

The axioms of Zermelo and Fraenkel with Choice ($\ZFC$) are a
countably infinite list of first-order sentences.
\medskip
\fbox{\em \dots\ discussion of axioms \dots }
\medskip

We chose a presentation of the $\ZFC$ axioms that would be most
compatible with the development of relativization by Paulson:
\begin{isabelle}
\isacommand{locale}\isamarkupfalse%
\ M{\isacharunderscore}ZF\ {\isacharequal}\ \isanewline
\ \isakeyword{fixes}\ M\ \isanewline
\ \isakeyword{assumes}\ \isanewline
\ \ \ \ \ \ upair{\isacharunderscore}ax{\isacharcolon}\ \ \ \ {\isachardoublequoteopen}upair{\isacharunderscore}ax{\isacharparenleft}{\isacharhash}{\isacharhash}M{\isacharparenright}{\isachardoublequoteclose}\isanewline
\ \ \isakeyword{and}\ Union{\isacharunderscore}ax{\isacharcolon}\ \ \ \ {\isachardoublequoteopen}Union{\isacharunderscore}ax{\isacharparenleft}{\isacharhash}{\isacharhash}M{\isacharparenright}{\isachardoublequoteclose}\isanewline
\ \ \isakeyword{and}\ power{\isacharunderscore}ax{\isacharcolon}\ \ \ \ {\isachardoublequoteopen}power{\isacharunderscore}ax{\isacharparenleft}{\isacharhash}{\isacharhash}M{\isacharparenright}{\isachardoublequoteclose}\isanewline
\ \ \isakeyword{and}\ extensionality{\isacharcolon}\ \ \ {\isachardoublequoteopen}extensionality{\isacharparenleft}{\isacharhash}{\isacharhash}M{\isacharparenright}{\isachardoublequoteclose}\isanewline
\ \ \isakeyword{and}\ foundation{\isacharunderscore}ax{\isacharcolon}\ \ \ \ {\isachardoublequoteopen}foundation{\isacharunderscore}ax{\isacharparenleft}{\isacharhash}{\isacharhash}M{\isacharparenright}{\isachardoublequoteclose}\isanewline
\ \ \isakeyword{and}\ infinity{\isacharunderscore}ax{\isacharcolon}\ {\isachardoublequoteopen}infinity{\isacharunderscore}ax{\isacharparenleft}{\isacharhash}{\isacharhash}M{\isacharparenright}{\isachardoublequoteclose}\isanewline
\ \ \isakeyword{and}\ separation{\isacharunderscore}ax{\isacharcolon}\isanewline
\ \ {\isachardoublequoteopen}{\isasymlbrakk}\ {\isasymphi}\ {\isasymin}\ formula\ {\isacharsemicolon}\ arity{\isacharparenleft}{\isasymphi}{\isacharparenright}{\isacharequal}{\isadigit{1}}\ {\isasymor}\ arity{\isacharparenleft}{\isasymphi}{\isacharparenright}{\isacharequal}{\isadigit{2}}\ {\isasymrbrakk}\isanewline\ \ \ \ {\isasymLongrightarrow}\isanewline
\ \ {\isacharparenleft}{\isasymforall}a{\isasymin}M{\isachardot}\ separation{\isacharparenleft}{\isacharhash}{\isacharhash}M{\isacharcomma}{\isasymlambda}x{\isachardot}\ sats{\isacharparenleft}M{\isacharcomma}{\isasymphi}{\isacharcomma}{\isacharbrackleft}x{\isacharcomma}a{\isacharbrackright}{\isacharparenright}{\isacharparenright}{\isacharparenright}{\isachardoublequoteclose}\isanewline
\ \ \isakeyword{and}\ replacement{\isacharunderscore}ax{\isacharcolon}\isanewline 
\ \ {\isachardoublequoteopen}{\isasymlbrakk}\ {\isasymphi}\ {\isasymin}\ formula{\isacharsemicolon}\ arity{\isacharparenleft}{\isasymphi}{\isacharparenright}{\isacharequal}{\isadigit{2}}\ {\isasymor}\ arity{\isacharparenleft}{\isasymphi}{\isacharparenright}{\isacharequal}{\isadigit{3}}\ {\isasymrbrakk}\isanewline
\ \ \ \ {\isasymLongrightarrow}\isanewline
\ \ {\isacharparenleft}{\isasymforall}a{\isasymin}M{\isachardot}\ strong{\isacharunderscore}replacement{\isacharparenleft}{\isacharhash}{\isacharhash}M{\isacharcomma}\isanewline
\ \ \ \ {\isasymlambda}x\ y{\isachardot}\ sats{\isacharparenleft}M{\isacharcomma}{\isasymphi}{\isacharcomma}{\isacharbrackleft}x{\isacharcomma}y{\isacharcomma}a{\isacharbrackright}{\isacharparenright}{\isacharparenright}{\isacharparenright}{\isachardoublequoteclose}
\end{isabelle}


An equivalent formulation of the Foundation Axiom states that the
universe of sets can be decomposed in a transfinite hierarchy of
sets. 
\begin{theorem}
  Let $V_{\al}\defi\union\{\P(V_\be) : \be<\al\}$ for each ordinal
  $\al$. Then each $V_\al$ is a set and 
  $\forall x. \exists\al .\ \Ord(\al) \y x\in V_\al$.  
\end{theorem}

\medskip
\fbox{\em \dots\ discussion of axioms \dots }
\medskip

Models of the theory $\ZFC$ consist of a pair $\lb M,E\rb$ where $M$
is a set and $E$ is a binary relation on $M$. Forcing is a technique
to extend very special kind of models, where $M$ is a countable
transitive set (i.e., every element of $M$ is a subset of $M$) and
$E$ is the membership relation $\in$ restricted to $M$. In this case
we simply refer to $M$ as a \emph{countable transitive model} or
\emph{ctm}.

The existence of a ctm of $\ZFC$ can be proved from the existence of a
model  $\lb N,E\rb$ such that the relation $E$ is well founded. The
L\"owenheim-Skolem 
Theorem ensures that there is an countable elementary submodel 
$\lb N',E\restr N'\rb\preccurlyeq  \lb N,E\rb$ which must also be
well founded; then the 
Mostowski collapsing function \cite[Def.~I.9.31]{kunen2011set} sends $\lb
N',E\restr N'\rb$ 
isomorphically to some $\lb M,\in\rb$ with  $M$ transitive.

By G\"odel's Second Incompleteness Theorem we cannot  prove that
there exists a model of $\ZFC$  (assuming that our base theory is not
stronger than $\ZFC$ to start with), but for consistency proofs,
usually a model of only a finite subset of $\ZFC$ suffices. Hence the
following meta-theoretic result by Montague applies:
%
\begin{theorem}[Reflection Principle]\label{th:reflection-principle}
  For every finite $\Phi\sbq\ZFC$, $\ZFC$ proves: ``There exists
    unboundedly many $\al$ such that $V_\al\models \Phi$.''
\end{theorem}
%
Since the sets $V_\al$ are well founded, we can repeat the above
argument to obtain a ctm of $\Phi$.
%
%%% Local Variables: 
%%% mode: latex
%%% TeX-master: "Separation_In_MG"
%%% ispell-local-dictionary: "american"
%%% End: 
