\section{Lessons}\label{sec:lessons}

\subsection{Plan before formalizing}
\begin{enumerate}
\item Ask about your project
  \url{https://mathoverflow.net/q/265435/66044}

  Disclaimer: It may be useless.
\item Antecedents: Precise enumeration of what \session{ZF-Constructible} had.

\item We should had better used predicates for the forcing posets'
  order relations (the way DSL is written), and go for class forcing
  (with $\PP$ a definable subset of $M$). The latter change seems to be
  easy, but the former doesn't.
  
  “Sometimes it felt like sculpting on marble: The fear of having
  carved too deep and hence needing to change the whole stone!”

  Note: This is in direct contradiction with “code fever”
  (\ref{sec:beware-code-fever} below).
\item Read the fine print: Foundations of your proof assistant.
\end{enumerate}

\subsection{Control your bureaucracy. Automate early}
\begin{enumerate}
\item Bureaucracy vs ML programming.
\item The “math” was already formalized on 22 November 2020.
  We finished the last goal on 22 August 2021.
  (Update: 20 November 2021 \& 28 November 2021, for CH)
\item Missing: automation of closure of models under operations.
\item Missing: basic arithmetic for dealing with arities.
\end{enumerate}

\subsection{Beware of scale factors}
\begin{enumerate}
\item It is extremely misleading when automatic tools (\isatt{simp}, \isatt{auto}, etc)
  stop working just because of the sheer size of the goal. Oftentimes,
  in math, we disregard scale issues but they must always be taken
  into account in CS.
\item Example: $\forceisa(0\in 1)$ is expandable,
  $\forceisa(\neg\neg  0\in 1)$ is not.
\item Example: Synthesis of $\forceisa$; could have been fully synthesized,
  but that was dirty “strategy”.
\item The know-how of computer scientists on this kind of engineering is
  very important
\end{enumerate}

\subsection{You might have formalized it, and still be wrong}
\begin{enumerate}
\item Example: restriction of relations.
\item Pollack, “Pollack consistency” by Wiedijk. Opacity of automated
  proofs.
\item Plot twist: You can be right without knowing. Intuition may drive proofs
  even if we are not working on what we believe we are.
\end{enumerate}

\subsection{Beware of the “Code fever”}\label{sec:beware-code-fever}
\begin{enumerate}
\item “We know that doing math is fun---formalization is like DRUGS”
\item Feeling of accomplishment after seeing your writings
  validated beyond reasonable doubt (v.g. cofinality).

\item One easily forgets about the “Power of the Board.”
\end{enumerate}

\subsection{The Devil's on the shortcuts}
\begin{enumerate}
\item
  Our proofs of the “definition of forces” (and many
  consequences) and of the lemma for “forcing a value” of function
  depend on the countability of the ground model. 
\item
  Density arguments (look for “TODO”, “general versions”).
\end{enumerate}

\subsection{Document your project}
\begin{enumerate}
\item \theory{Definitions\_Main}, thanks to Vidnyánszky.
\end{enumerate}

%%% Local Variables: 
%%% mode: latex
%%% TeX-master: "independence_ch_isabelle"
%%% ispell-local-dictionary: "american"
%%% End: 
