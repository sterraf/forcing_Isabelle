%%%%%%%%%%%%%%%%%%%%%%%%%%%%%%%%%%%%%%%%%%%%%%%%%%%%%%%%%%%%%%%%%%%%%%          
\section{Introduction}
% no \IEEEPARstart
Zermelo-Fraenkel Set Theory ($\ZF$) has a prominent place among formal
theories. The reason for this is that it formalizes many intuitive
properties of the notion of set. As such, it can be used as a
foundation for mathematics and thus it has been
thoroughly studied. 
Considering the current trend 
of mechanization of mathematics~\cite{avigad2018mechanization}, it
seems natural to ask for a mechanization of the most salient results
of Set Theory.

The results we are interested in originally arose in connection to
\emph{relative consistency} proofs in set theory; that is, showing
that if $\ZF$ is 
consistent, the addition of a new axiom $A$ won't make the system
inconsistent; this is as much as we can expect to obtain, since
G\"odel's Incompleteness Theorems precludes a  
formal proof of the consistency of set theory in $\ZF$, unless the
latter is indeed inconsistent. There are statements $A$ which are
\emph{undecided} by $\ZF$, in the sense that both $A$ and $\neg A$ are
consistent relative to $\ZF$; perhaps the most prominent  is the
\emph{Continuum Hypothesis}, which led to the development of powerful
techniques for independence proofs. First, G\"odel inaugurated the
theory of \emph{inner models} by introducing his model $L$ of
the \emph{Axiom of Constructibility} \cite{godel-L} and proved the
relative consistency of the Axiom of Choice and the Generalized
Continuum Hypothesis with $\ZF$. More than twenty years later, Paul
J.~Cohen~\cite{Cohen-CH-PNAS} devised the technique of \emph{forcing},
which is the only known way of \emph{extending} models of $\ZF$; this
was used to prove the relative consistency of the 
negation of the Continuum Hypothesis. 

In this work we address a substantial part of formalizing the proof
that given a model $M$ of $\ZF$, any \emph{generic extension} $M[G]$
obtained by forcing also satisfies $\ZF$. As remarked by Kunen
\cite[p.250]{kunen2011set} \enquote{[...] in verifying that $M[G]$ is
  a model for set theory, the hardest axiom to verify is
  [Separation].}  The most important achievement of this paper is the
mechanization in the proof assistant \emph{Isabelle} of a proof of the
Axiom of Separation in generic extensions by using the ``fundamental''
theorems of forcing.  En route to this, we also formalized the
satisfaction by $M[G]$ of Extensionality, Foundation, and Union. As a
consequence of Separation we were able to formalize the proof of the
Powerset Axiom; finally, the Axiom of Infinity was proved under extra
assumptions. % We have already
% proved Pairing in the first report of our project
% \cite{2018arXiv180705174G}.
The theoretical support for this work has been 
the fine textbook by Kunen \cite{kunen2011set} and our development
benefited from the remarkable work done by Lawrence 
Paulson \cite{paulson_2003} on the formalization of G\"odel's
constructible universe in Isabelle. 

The
ultimate goal of our project is the formalization of the forcing
techniques needed to show the independence of the Continuum
Hypothesis. 
We think that this project constitutes an interesting test-case
for the current technology of formalization of mathematics, in
particular for the need of handling several layers of reasoning. 

The \emph{Formal Abstracts} project~\cite{hales-fabstracts} proposes
the formalization of complex pieces of mathematics by writing the
statements of 
results and the material upon which they are based (definitions,
propositions, lemmas), but omitting the proofs. In this work we
partially adhere to this vision to delineate our formalization
strategy: since the proofs that the  axioms hold in generic extensions
are independent of the \emph{proofs} of the fundamental theorems of
forcing, we assumed the latter for the time being. Let us remark
that those theorems depend on the definition of a function $\forceisa$
from formulas to formulas which is, by itself, quite demanding; the
formalization of it and of the fundamental theorems of forcing % roughly
comprises barely less than a half of our full project.

It might be a little surprising the lack of formalizations of forcing
and generic extensions. As far as we know, the development of
Quirin and Tabareau \cite{JFR6232} in homotopy type theory for constructing generic
extensions in a sheaf-theoretic setting is the unique mechanization of
forcing. This contrast with the fruitful use of forcing techniques to
extend the Curry-Howard isomorphism to classical axioms in order to
get programs for proofs using those axioms
\cite{Miquel:2011:FPT:2058525.2059614,lmcs:1070}. Moreover, the
combination of forcing with intuitionistic type theory
\cite{Coquand:2009:FTT:1807662.1807665,coquand2010note} gives rise
both to positive results (an algorithm to obtain witnesses of the
continuity of definable functionals \cite{coquand2012computational})
and also negative (the independence of Markov's principle
\cite{lmcs:3859}). In the same strand of forcing from the point of
view of proof theory \cite{avigad_2004} are the conservative
extensions of CoC with forcing conditions
\cite{jaber:hal-01319066,Jaber:2012:ETT:2358958.2359524}.

% \fbox{Parece mucho comienzo sólo para introducir a Kunen}
% \fbox{¿lo puedo achurar un poco?}

% In a gross simplification, there are two aspects to a formalization
% project like this one: thematic and programmatic. The first concerns
% the handling of all the theoretical concepts and results in the
% subject, while the second involves the practical issues of the
% implementation and design. In the case of forcing, the main intricacy
% lies in the first aspect. In this sense, following a sensible
% presentation of the material is key.  The authoritative reference 
% on the subject during the last 30 years has been Kunen's classical
% \cite{kunen1980}. In our
% formalizaton we have followed a recent rewrite \cite{kunen2011set}
% of that  textbook, which presents the material in the same sharp 
% style but offering a lot of details. In some sense this project
% wouldn't exist without this book. As alternative, introductory
% resources, the  interested reader can check
% \cite{chow-beginner-forcing}; also, the book \cite{weaver2014forcing}
% contains a thorough treament minimizing the technicalities.

In pursuing the proof of Separation on generic extensions we
extended Paulson's library with:
\begin{enumerate}
\item renaming of variables for \emph{internalized} formulas, which
  with little effort can be extended 
  to substitutions;
\item an improvement on definitions by recursion on well-founded
  relations; 
\item enhancements in the hierarchy of locales; and
\item a variant of the  principle of dependent choices and a version
  of Rasiowa-Sikorski's theorem, which 
  ensures the existence of generic filters for countable and transitive
  models of $\ZF$;
\end{enumerate}
the last item was already communicated in the first report
\cite{2018arXiv180705174G} of our project (where we also proved that
$M[G]$ satisfies Pairing; now we have redone this proof in Isar).
  
We briefly describe the contents of each
section. Section~\ref{sec:isabelle} contains the bare minimum
requirements to understand the (meta)logics used in Isabelle. Next, an
overview of the model theory of set theory is presented in
Section~\ref{sec:axioms-models-set-theory}. There is an ``internal''
representation of first-order formulas as sets, implemented by
Paulson; Section~\ref{sec:renaming} discusses syntactical
transformations of the former, mainly permutation of variables. 
In Section~\ref{sec:generic-extensions} the generic extensions are
succinctly reviewed and how the treatment of well founded recursion in
Isabelle was enhanced. We take care of the ``easy axioms'' in
Section~\ref{sec:easy-axioms}; these are the ones that
do not depend on the forcing theorems. We describe the latter in
Section~\ref{sec:forcing}. We adapted the  work by Paulson to our
needs, and this is described in
Section~\ref{sec:hack-constructible}. We present the proof
of the Separation Axiom Scheme in Section~\ref{sec:proof-separation},
which follows closely its implementation, and some comments on the
proof of the Powerset Axiom. A plan for future work and
some immediate conclusions are offered in
Section~\ref{sec:conclusions-future-work}.

%%% Local Variables: 
%%% mode: latex
%%% TeX-master: "Separation_In_MG"
%%% ispell-local-dictionary: "american"
%%% End: 
