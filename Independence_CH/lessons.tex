\section{Some lessons}\label{sec:lessons}

We want to finish this report by gathering some of the conclusions we
reached after the experience of formalizing the basics of forcing in a
proof assistant.

\subsection{Aims of a formalization and planning}

We believe that in every project of formalization of mathematics,
there is a tension between the haste to verify the target results and
the need to obtain a readable, albeit extremely detailed, corpus of
statements and proofs. This tension is mirrored in two differents
purposes of formalization: Developing new mathematics from scratch and
producing verified results en route to this, versus verifying and
documenting material that has already been produced on paper.

Our present project clearly belongs to the second category, so we
prioritized trying to obtain formal proofs that mimicked standard
prose (the highlight being the sample proof in
Section~\ref{sec:sample-formal-proof}). We feel that the Isar language
provided with Isabelle has the right balance between elegance and
efficacy. Another crucial aspect to achieve this goal is the level of
detail of the blueprint for the formalization. We must however confess
that we learned many of the subtleties of Isabelle in the making, and
many engineering decisions were also taken before it was clear the
precise way things would develop in the future.

A similar experience, but on an opposite side of the formalization
spectrum happened to the Liquid Tensor Project as described by Scholze
in \cite{LTE2021}. People involved in the formalization simply pushed
their way to reach the summit, formalizing lemma after lemma. They
actually wrote the blueprint for that formalization \emph{afterwards}
it was complete! From time to time, we were also frenziedly trying to
get the results formalized, going beyond what we had planned.

As a result from this, some design choices that seemed reasonable at
first were proved to be inconvenient. For instance, we should had
better used predicates (of type $\tyi\fun\tyi\fun\tyo$) for the
forcing posets' order relations; this is the way they
are presented in the \session{Delta\_System\_Lemma} session. A similar
problem is that we require the forcing poset to be an element of $M$,
so the present infrastructure does not allow class forcing out of the
box. (The latter change seems to be rather straightforward, but the
former does not.)

Nearly the final stage of the project, we decided to go for the minimal
set of definitions and versions of lemmas that were needed to obtain
our target results. For example, we only proved the Delta System Lemma
for $\aleph_1$-sized families (thus limiting us to the plain ccc) and
showed preservation of sequences by considering countably closed
forcings (in fact, we formalized the bare minimum requirement of being
$(<\omega+1)$-closed). In doing this we went against the tried and
true advice that one should formalize the most general version of the
results available.

\subsection{Bureaucracy and scale factors}

\begin{enumerate}
\item Bureaucracy vs ML programming.
\item The “math” was already formalized on 22 November 2020.
  We finished the last goal on 22 August 2021.
  (Update: 20 November 2021 \& 28 November 2021, for CH)
\item Missing: automation of closure of models under operations.
\item Missing: basic arithmetic for dealing with arities.
\end{enumerate}

\begin{enumerate}
\item It is extremely misleading when automatic tools (\isatt{simp}, \isatt{auto}, etc)
  stop working just because of the sheer size of the goal. Oftentimes,
  in math, we disregard scale issues but they must always be taken
  into account in CS.
\item Example: $\forceisa(0\in 1)$ is expandable,
  $\forceisa(\neg\neg  0\in 1)$ is not.
\item Example: Synthesis of $\forceisa$; could have been fully synthesized,
  but that was dirty “strategy”.
\item The know-how of computer scientists on this kind of engineering is
  very important
\end{enumerate}

\subsection{You might have formalized it, and still be wrong}
\begin{enumerate}
\item Example: restriction of relations.
\item Pollack, “Pollack consistency” by
  Wiedijk. Cf. \theory{Definitions\_Main} (thanks to discussions with
  Vidnyánszky). Opacity of automated proofs. 
\item Plot twist: You can be right without knowing. Intuition may drive proofs
  even if we are not working on what we believe we are.
\end{enumerate}

\subsection{Beware of the “Code fever”}\label{sec:beware-code-fever}
\begin{enumerate}
\item “We know that doing math is fun---formalization is like DRUGS”
\item Feeling of accomplishment after seeing your writings
  validated beyond reasonable doubt (v.g. cofinality).

\item One easily forgets about the “Power of the Board.”
\end{enumerate}

\subsection{The Devil's on the shortcuts}
\begin{enumerate}
\item
  Our proofs of the “definition of forces” (and many
  consequences) and of the lemma for “forcing a value” of function
  depend on the countability of the ground model. 
\item
  Density arguments (look for “TODO”, “general versions”).
\end{enumerate}

\end{enumerate}

%%% Local Variables: 
%%% mode: latex
%%% TeX-master: "independence_ch_isabelle"
%%% ispell-local-dictionary: "american"
%%% End: 
