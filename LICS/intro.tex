%%%%%%%%%%%%%%%%%%%%%%%%%%%%%%%%%%%%%%%%%%%%%%%%%%%%%%%%%%%%%%%%%%%%%%          
\section{Introduction}
% no \IEEEPARstart
Zermelo-Fraenkel Set Theory ($\ZF$) has a prominent place among formal
theories. The reason for this is that it formalizes many intuitive
properties of the notion of set. As such, it can be used as a
foundation for mathematics and thus it has been
thoroughly studied. 

One of the main initial concerns was to decide if this foundation
completely axiomatized all of mathematics. As it was shown by G\"odel
in 1931, no  such axiomatization (therefore, including artihmetic) can
be complete. A refinement of this result, G\"odel's Second
Incompleteness Theorem, says that no consistent formal system that
includes arithmetic can prove its own consistency. As a matter of
fact, many problems within the area of Set Theory, not directly related
to consistency or logic, have been proved to be undecided by the
theory. Two techniques have been developed to do this. One is Kurt G\"odel's
theory of \emph{inner models}, which started with his model $L$ of
\emph{Axiom of Constructibility} \cite{godel-L} and was used to proved the relative
consistency of the Axiom of Choice and the Generalized Continuum
Hypothesis with $\ZF$. The second one is the technique of
\emph{forcing}, which was developed 30 years later by Paul
J. Cohen~\cite{Cohen-CH-PNAS}. In essence, forcing is the only known
way of \emph{extending} models of $\ZF$. 

The constructible universe  has been formalized by Lawrence
Paulson \cite{paulson_2003} in the proof assistant \emph{Isabelle}, and
the present paper is part of a project of formalizing the whole of the
forcing machinery, continuing the work by Paulson. The starting point
of this project is our previous paper 
\cite{2018arXiv180705174G}, to which we 
refer the reader for further details on basic matters. 


In a gross simplification, there are two aspects to a formalization
project like this one: thematic and programmatic. The first concerns
the handling of all the theoretical concepts and results in the
subject, while the second involves the practical issues of the
implementation and design. In the case of forcing, the main intricacy
lies in the first aspect. In this sense, following a sensible
presentation of the material is key.  The authoritative reference 
on the subject during the last 30 years has been Kunen's classical
\cite{kunen1980}. In our
formalizaton we have followed a recent rewrite \cite{kunen2011set}
of that  textbook, which presents the material in the same sharp 
style but offering a lot of details. In some sense this project
wouldn't exist without this book. As alternative, introductory
resources, the  interested reader can check
\cite{chow-beginner-forcing}; also, the book \cite{weaver2014forcing}
contains a thorough treament minimizing the technicalities.

In this work we address a substantial part the problem of formalizing
the proof that given a model of $\ZF$, its extensions by forcing, 
\emph{generic extensions},
satisfy the same axioms. In particular, the first axiom that
genuinely calls for a use of the forcing machinery is that of
\emph{Separation} (\emph{Aussonderungsaxiom}), otherwise known as
``Axiom of (Restricted) Comprehension''  or ``Specification.'' In
\cite{2018arXiv180705174G} it was shown that  generic extensions of
models of $\ZF$ 
satisfy the Axiom of Pairing, and in this work we show that also
Extensionality, Foundation, Union, Infinity (under extra assumptions),
and Separation go through. The
Axiom of Powerset is not treated in the present paper but it poses no
extra difficulties; we discuss the Axiom of Replacement in the
conclusions. 

The proof that these axioms hold in generic extension is independent
of the \emph{proofs} of the ``fundamental'' theorems of forcing. In
particular, one of the most complex parts of those proofs is the
definition of the \emph{forcing relation}. It turns out that the
particular (``implementation'') details of this  definition  does not
appear in the statement of the fundamental theorems, and our
formalization works inside a \emph{locale} (set of assumptions) that
includes the statement of these theorems. The formalization of the
fundamental theorems of forcing roughly comprises little less than a
half of our full project. 

We briefly describe the contents of each
section. Section~\ref{sec:isabelle} contains the bare minimun
requirements to understand the (meta)logics used in Isabelle. Next, an
overview of the model theory of set theory is presented in
Section~\ref{sec:axioms-models-set-theory}. There is an ``internal''
representation of first-order formulas as sets, implemented by
Paulson; Section~\ref{sec:renaming} discusses syntactical
transformations of the former, mainly permutation of variables. 
In Section~\ref{sec:generic-extensions} the generic extensions are
succintly reviewed and how the treatment of well founded recursion in
Isabelle was enhanced. We take care of the ``easy axioms'' in
Section~\ref{sec:easy-axioms}; these are the ones that
do not depend on the forcing theorems. We describe the latter in
Section~\ref{sec:forcing}. We adapted the  work by Paulson to our
needs, and this is described in
Section~\ref{sec:hack-constructible}. We present the proof
of the Separation Axiom Scheme in Section~\ref{sec:proof-separation},
which follows closely its implementation. A plan for future work and
some immediate conclusions are offered in
Section~\ref{sec:conclusions-future-work}.

%%% Local Variables: 
%%% mode: latex
%%% TeX-master: "Separation_In_MG"
%%% ispell-local-dictionary: "american"
%%% End: 
