\section{Introduction}
\label{sec:introduction}

The present work reports on the third stage of our project of
formalizing the theory of forcing and its applications as presented in
one of the more
important references on the subject, Kunen's Set Theory
\cite{kunen2011set} (a rewrite of the classical \cite{kunen1980}). 

We
work using the  implementation of Zermelo-Fraenkel ($\ZF$)
set theory \emph{Isabelle/ZF} by Paulson and Grabczewski \cite{DBLP:journals/jar/PaulsonG96}. In
an early paper \cite{2018arXiv180705174G}, we set up the first
elements, defining forcing notions, names, generic extensions, and
showing the existence of generic filters via the Rasiowa-Sikorski
lemma (RSL). Our second (unpublished) technical report
\cite{2019arXiv190103313G} advanced by presenting the first accurate
\emph{formal abstract} of the Fundamental Theorems of Forcing, and
using them to show that that the $\ZF$ axioms apart from Replacement
and Infinity hold in all generic extensions.

This paper contains the proof of Fundamental Theorems and complete
proofs of the Axioms of Infinity, Replacement, and Choice in all
generic extensions. In particular, we were able to fulfill the
promised formal abstract for the Forcing 
Theorems almost to the letter. A requirement for Infinity and the
absoluteness of forcing for atomic formulas, we finished the inteface
between our development and
Paulson's constructibility library \cite{paulson_2003} which enables
us to do well-founded
recursion inside transitive models of an appropriate finite fragment
of $\ZF$. As a by-product, we honored a long debt: the fact that the
generic filter $G$ belongs to the extension $M[G]$ and $M\sbq M[G]$.
Finally, our development is now independent of $\AC$: We modularized
RSL in such a way that a version for countable 
posets does not require choice.

\begin{framed}
  We implemented an automatic ML machinery of (bounded) renaming
  of formulas.
\end{framed}

\subsection{Related work}
To the best of our knowledge, all of the previous works in
formalization of the method 
of forcing has been done in different variants of type theory. The
most important is the recent one by 
Han \& van Doorn
\cite{han_et_al:LIPIcs:2019:11074,DBLP:conf/cpp/HanD20}, which includes
a formalization of the independence of $\CH$ by means
the Boolean-valued approach to forcing, using the Lean
proof assistant \cite{DBLP:conf/cade/MouraKADR15}.

%%% Local Variables: 
%%% mode: latex
%%% TeX-master: "forcing_in_isabelle_zf"
%%% ispell-local-dictionary: "american"
%%% End: 
