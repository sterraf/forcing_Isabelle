\section{Introduction}\label{sec:introduction}
Set Theory plays a double role in Mathematics: It is one of its
possible foundations and also an active research area.
%
As it is widely known, Georg Cantor introduced its main concepts and in
particular showed the fundamental result that the real line, $\R$  is not
coordinable with the natural numbers. Soon after this, he posed the
most important question in the field, written as a conjecture:
\begin{quote}
  The \emph{Continuum Hypothesis} ($\CH$). Every uncountable subset of $\R$ is
  coordinable with $\R$.
\end{quote}

The current axiomatic foundation of Set Theory is through first-order
logic and uses the axioms devised by Zermelo and Fraenkel, including
the Axiom of Choice ($\AC$) among them. This theory is known by the
$\ZFC$ acronym. G\"odel \cite{godel-L} showed that $\CH$ cannot be refuted using
$\ZFC$, unless this theory itself is inconsistent (we say that
\emph{$\CH$ is relatively consistent with $\ZFC$}). For a while, this
result left the possibility that one might be able to show
$\ZFC\models \CH$, but in a groundbreaking work \cite{Cohen-CH-PNAS},
Paul Cohen discovered the technique of \emph{forcing} and proved that
$\neg\CH$ is relatively consistent with $\ZFC$. Forcing has been used
since then for showing innumerable independence results and to perform
mathematical constructions.

A great part of G\"odel's work on this subject has been formalized in
Isabelle by Larry Paulson. This paper formalizes a first part of the
machinery of forcing. In the rest of the introduction we discuss some
of the set-theoretical details involved and explain briefly Paulson's
formalization.

\subsection{Models of $\ZFC$}
By G\"odel's Second Incompleteness Theorem, we cannot  prove that
there exists a model of $\ZFC$. More formally, if we assume that
mathematical proofs can be encoded as theorems of $\ZFC$ and that
the latter do not lead to contradictions (i.e., $\ZFC$ is
\emph{consistent}), then we cannot prove that there exists a set $M$
and a binary relation $E$ such that $\lb M,E\rb$ satisfies the $\ZFC$
axioms.

A relative consistency proof for an axiom $A$ is then obtained by
assuming that there exists a model of ZFC, say $\lb M,E\rb$, and
constructing another model $\lb M',E'\rb$ for $\ZFC + A$. We single
out a very special kind of models:
%
\begin{definition}\label{def:transitive-model}
  \begin{enumerate}
  \item A set $M$ (of sets) is \emph{transitive} if for all $x\in M$ and 
    $y\in x$, we have $y\in M$ (i.e., every element of $M$ is a subset
    of $M$).
  \item $\lb M,E\rb$ is a \emph{transitive model} if $M$ is transitive
     and  $E$ is the membership relation $\in$ restricted to
    $M$. It is \emph{countable} if $M$ is coordinable with a subset of
    $\N$; we then say that the model  $M$ is a \emph{ctm}. 
  \end{enumerate}
\end{definition}
%
\noindent As in the last sentence, one usually refer to a transitive
model by the underlying set because the relation is fixed.
%% MIGUEL: feísimo este corte. Quizás refrasear la oración anterior
%% para que entre en un renglón.

%% MIGUEL: lo que sigue implica que para tener aritmética en ZFC son
%% necesarios una cantidad infinita de axiomas de ZFC?
In spite of G\"odel's Second Incompleteness Theorem, one can find
transitive models for every finite fragment of $\ZFC$. More precisely,
%
\begin{theorem}\label{th:ctm-finite-axioms}
  For each finite subset
  $\Phi\sbq \ZFC$, the statement \emph{``there exists a countable
    transitive model of $\Phi$''} is a theorem of $\ZFC$.
\end{theorem}
%
%% MIGUEL: controlar que tiene sentido lo que menciono en la nota al pie!
%% MIGUEL: Me da la sensación que podemos promover lo de la nota al pie
%% al cuerpo y luego explicar que por eso es importante el teorema.
This follows by a combination of the Reflection Principle, the
L\"owenheim-Skolem Theorem, and the Mostowksi Collapse. The reader can
consult the details in \cite{kunen2011set}. Consistency arguments that
assume the existence of a ctm $M$ of $\ZFC$ can usually be replaced by
a model as in Theorem~\ref{th:ctm-finite-axioms}, since a first-order
proof (e.g.\ of a contradiction)%
%%%%%%%%%%%%%
\footnote{It is relevant to this point that 
  both the approaches by G\"odel and Cohen for showing
  relative independence of an axiom $A$ 
  can be use to obtain an algorithm transforming a proof
  concluding a contradiction from $\ZFC+A$ to one from $\ZFC$.}
%%%%%%%%%%%%%
involves only finitely many axioms.

%% MIGUEL: Qué significa "portrayal convenient"?
%% MIGUEL: ¿por qué esto es conveniente para esta introducción?
Assume that $M$ is a ctm of $\ZFC$. In a portrayal convenient for this
introduction, G\"odel showed that $M$ contains a minimal submodel
$L^M$ of the same ``height'' (i.e.\ having the same ordinals)
that satisfies $\ZFC+\CH$. The sets in $L^M$ are called
\emph{constructible} and are in a sense ``definable.'' In fact, there
is a first-order formula $L$ such that $L^M = \{x\in M : M\models
L(x)\}$. To show that  $L^M\models \ZFC+\CH$, one uses that the
$\ZFC+\CH$ holds in $M$.

%% MIGUEL: acá me parece que hay un salto medio abrupto. Por que
%% es una necesidad primaria? Lo podés relacionar con lo que venía
%% del párrafo anterior?
%% MIGUEL: quizás lo que querés decir acá es para probar que vamos
%% a construir un modelo N a partir de un modelo M; y para probar
%% que N satisface un axioma, vamos a utilizar que M lo satisface.
It is therefore a primary need to have a means to correlate  (first-order)
properties satisfied by a model $M$ and those of a 
submodel $N\sbq M$. As a simple example on this, consider 
%% $M\defi \{\emptyset,\{\emptyset\},\{\{\emptyset\}\}\}$ and
%% $N\defi\{\{\emptyset\},\{\{\emptyset\}\}\}$, and let 
%% $\phi(x)\defi \neg\exists y (y \in x)$. It is clear that 
%% $M\models \phi(\emptyset)$; but we have
%% \[
%% M\not\models \phi(\{\emptyset\}) \quad\text{ but }\quad N\models \phi(\{\emptyset\}).
%% \]
%% There is a discrepancy between what $M$ and $N$ \emph{believe} about
%% their common element $\{\emptyset\}$.
$M\defi \{a,b,c, \{a,b\},\{a,b,c\}\}$ and
$N\defi\{a,b,\{a,b,c\}\}$, and let 
\[
\phi(x,y,z)\defi \forall w \,( w\in z \lsii w=x \o w=y).
\]
%% It is clear that  $M\models \phi(\{a,b\})$; but we have
Then we have
\[
M\not\models \phi(a,b,\{a,b,c\}) \quad\text{ but }\quad N\models \phi(a,b,\{a,b,c\}).
\]
There is a discrepancy between  $M$ and $N$ about $\{a,b,c\}$ being ``the
(unordered) pair of $a$ and $b$.'' We say that $\phi$ holds for
$a,b,\{a,b,c\}$ \emph{relative} to $N$. It is immediate to see that
$\phi$ holds  for $x,y,z$ relative to $N$ if and only if
\[
\phi^N(x,y,z)\defi \forall w.\ w\in N\limp ( w\in z \lsii w=x \o w=y)
\] 
holds. $\phi^N$ is called the \emph{relativization of $\phi$ to
  $N$}. One can generalize this operation of relativization to the
class of all sets satisfying a first-order predicate $C$ in a
straightforward way:
\[
\phi^C(x,y,z)\defi \forall w.\ C(w)\limp ( w\in z \lsii w=x \o w=y)
\] 

It can be shown elementarily that if $M$ and $N$ are transitive,
$\phi^N$ holds if and only if $\phi^M$ holds,  for $x,y,z\in N$. We
say then that $\phi$ is \emph{absolute between $N$ and $M$.}
%% MIGUEL: "truth of axioms in $M$" significa que $M$ satisface los
%% axiomas?
The concepts of relativization and absoluteness are central to the
task of transferring truth of axioms in $M$ to $L^M$, and constitute
the hardest part of Paulson's development.


\subsection{Forcing}
%% MIGUEL: no veo la necesidad del adversativo but, pondría and.
%% MIGUEL: Qué significa "a new set $G$ is adjoined to $M$"?
%% MIGUEL: este párrafo necesita correcciones.
Forcing is a technique to extend countable transitive models of
$\ZFC$. This process is guaranteed to preserve the $\ZFC$
axioms while allowing to fine-tune what other first order properties the
extension will have. Given a ctm $M$  of $\ZFC$ and a set $G$, a new
ctm  $M[G]$  that includes $M$ and
contains $G$ is constructed, and under some hypotheses ($G$ being ``generic''),
$M[G]$ satisfies $\ZFC$.

The easiest way to define genericity is by using a preorder with top
$\lb\PP,\leq,\1\rb$ in $M$.   
In Section~\ref{sec:forcing-notions} we formalize the definitions of
\emph{dense} subset and  \emph{filter} of  $\PP$, and we say that  $G$
is an $M$-generic filter
if it intersects every dense subset of $\PP$ that lies in $M$.

%% MIGUEL: Tenemos que dado un ctm M, existe un filtro genérico G.
%% ¿Dónde existe? Una primera lectura de ignorante me dice "existe
%% en M", pero eso me lo desmiente la siguiente oración. ¿Juega algún
%% rol absolutez? Si es así estaría bien explicitarlo.
%% MIGUEL: la última oración me resulta un poco difícil de entender.
The Rasiowa-Sikorski lemma states that, fixed a preorder $\PP$, for
any countable family of dense subsets of $\PP$ there is a filter
intersecting all of them. Thus, there are generic filters $G$ for
countable transitive models. In general, no such $G$ belongs to $M$ and
therefore the extension $M[G]$ is proper. We formalize the proof of
the Rasiowa-Sikorski lemma in
Section~\ref{sec:rasiowa-sikorski-lemma}. A requisite result on a
version of the Axiom of Choice is formalized in
Section~\ref{sec:sequence-version-dc}.

%% MIGUEL: redacción alternativa del párrafo siguiente.
Every  $y \in M[G]$ is obtained from an element $\dot y$ of $M$, thus
elements of $M$ are construed as \emph{names} or codes for elements of
$M[G]$.
The decoding is given by the function
$\val$ which takes the generic filter $G$ as a parameter. To
prove that $M$ is contained in $M[G]$ it suffices to give a name for
each element of $M$; we define the function $\chk$ which assigns
a name for each $x\in M$. Showing that $\chk(x)\in M$
when $x\in M$ involves some technical issues that will
be addressed in a further work. We explain names, $\val$, and
$\chk$ in Section~\ref{sec:names}.

A central part of this formalization project involves showing that
$\ZFC$ holds in the generic extension. This is most relevant since
forcing is essentially the only known way to \emph{extend} models of
$\ZFC$ (while preserving ordinals). The most difficult step to achieve
this goal is to define the \emph{forcing relation}, that allows to
connect satisfaction in $M$ to that of $M[G]$; this is needed to show
that the Powerset axiom and the axiom schemes of Separation and
Replacement hold in $M[G]$. In this work we tackle the Pairing
Axiom. This doesn't require the forcing relation, but provides an
illustration of the use of names.
   
%We continue with a brief description of the Isabelle framework. 

%%% Local Variables:
%%% mode: latex
%%% ispell-local-dictionary: "american"
%%% TeX-master: "first_steps_into_forcing"
%%% End:
