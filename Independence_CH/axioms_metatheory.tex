\section{Axioms of the meta-theory}
\label{appendix:axioms}

In this appendix we list the complete set of axioms of our meta-theory.

\subsection{Pure}
\begin{isabelle}
Pure.abstract\_rule: (⋀x. ?f(x) ≡ ?g(x)) ⟹ λx. ?f(x) ≡ λx. ?g(x)\isanewline
Pure.combination: ?f ≡ ?g ⟹ ?x ≡ ?y ⟹ ?f(?x) ≡ ?g(?y)\isanewline
Pure.equal\_elim: PROP ?A ≡ PROP ?B ⟹ PROP ?A ⟹ PROP ?B\isanewline
Pure.equal\_intr: (PROP ?A ⟹ PROP ?B) ⟹ (PROP ?B ⟹ PROP ?A) ⟹ \isanewline
\ \ \ \ \ \ \ \ \ \ PROP ?A ≡ PROP ?B\isanewline
Pure.reflexive: ?x ≡ ?x\isanewline
Pure.symmetric: ?x ≡ ?y ⟹ ?y ≡ ?x\isanewline
Pure.transitive: ?x ≡ ?y ⟹ ?y ≡ ?z ⟹ ?x ≡ ?z
\end{isabelle}

\subsection{IFOL and FOL}
In the axioms \isa{refl, subst, allI, spec, exE, exI,eq\_reflection} there is a constraint
for the \emph{type of} the variables \isa{a, b, x} to be in the class \isa{term\_class}.
\begin{isabelle}
IFOL.FalseE: ⋀P. False ⟹ P\isanewline
IFOL.refl:  (⋀a. a = a)\isanewline
IFOL.subst: (⋀a b P. a = b ⟹ P(a) ⟹ P(b))\isanewline
IFOL.allI:  (⋀P. (⋀x. P(x)) ⟹ ∀x. P(x))\isanewline
IFOL.spec:  (⋀P x. ∀x. P(x) ⟹ P(x))\isanewline
IFOL.exE:   (⋀P R. ∃x. P(x) ⟹ (⋀x. P(x) ⟹ R) ⟹ R)\isanewline
IFOL.exI:   (⋀P x. P(x) ⟹ ∃x. P(x))\isanewline
IFOL.conjI: ⋀P Q. P ⟹ Q ⟹ P ∧ Q\isanewline
IFOL.conjunct1: ⋀P Q. P ∧ Q ⟹ P\isanewline
IFOL.conjunct2: ⋀P Q. P ∧ Q ⟹ Q\isanewline
IFOL.disjE: ⋀P Q R. P ∨ Q ⟹ (P ⟹ R) ⟹ (Q ⟹ R) ⟹ R\isanewline
IFOL.disjI1: ⋀P Q. P ⟹ P ∨ Q\isanewline
IFOL.disjI2: ⋀P Q. Q ⟹ P ∨ Q\isanewline
IFOL.eq\_reflection: (⋀x y. x = y ⟹ x ≡ y)\isanewline
IFOL.iff\_reflection: ⋀P Q. P ⟷ Q ⟹ P ≡ Q\isanewline
IFOL.impI: ⋀P Q. (P ⟹ Q) ⟹ P ⟶ Q\isanewline
IFOL.mp: ⋀P Q. P ⟶ Q ⟹ P ⟹ Q\isanewline%\isanewline
FOL.classical: ⋀P. (¬ P ⟹ P) ⟹ P
\end{isabelle}

\subsection{ZF\_Base}
The following symbols are introduced in this theory:
\begin{isabelle}
axiomatization\isanewline
\ \ \ \      mem :: "[i, i] ⇒ o"  (infixl ‹∈› 50)  \isamarkupcmt{membership relation}\isanewline
  and zero :: "i"  (‹0›)  \isamarkupcmt{the empty set}\isanewline
  and Pow :: "i ⇒ i"  \isamarkupcmt{power sets}\isanewline
  and Inf :: "i"  \isamarkupcmt{infinite set}\isanewline
  and Union :: "i ⇒ i"  (‹⋃\_› [90] 90)\isanewline
  and PrimReplace :: "[i, [i, i] ⇒ o] ⇒ i"
\end{isabelle}
\noindent with these axioms
\begin{isabelle}
ZF\_Base.Pow\_iff: ⋀A B. A ∈ Pow(B) ⟷ A ⊆ B\isanewline
ZF\_Base.Union\_iff: ⋀A C. A ∈ ⋃C ⟷ (∃B∈C. A ∈ B)\isanewline
ZF\_Base.extension: ⋀A B. A = B ⟷ A ⊆ B ∧ B ⊆ A\isanewline
ZF\_Base.foundation: ⋀A. A = 0 ∨ (∃x∈A. ∀y∈x. y ∉ A)\isanewline
ZF\_Base.infinity: 0 ∈ Inf ∧ (∀y∈Inf. succ(y) ∈ Inf)\isanewline
ZF\_Base.replacement: ⋀A P b. ∀x∈A. ∀y z. P(x, y) ∧ P(x, z) ⟶ y = z ⟹ \isanewline
\ \ \ \ \ \ \ \ \ b ∈ PrimReplace(A, P) ⟷ (∃x∈A. P(x, b))
\end{isabelle}

\subsection{AC}

This theory \textit{AC} is only imported in the theory
\isa{Absolute{\uscore}Versions}; none of the main results
depends on \textit{AC}. The theory \isa{Absolute{\uscore}Versions}
shows that some absolut results can be obtained from the
relativized versions on $\mathcal{V}$.

\begin{isabelle}
AC.AC: ⋀a A B. a ∈ A ⟹ (⋀x. x ∈ A ⟹ ∃y. y ∈ B(x)) ⟹\isanewline
  \ \ \ \ \ \ ∃z. z ∈ Pi(A, B)
\end{isabelle}

%%% Local Variables:
%%% mode: latex
%%% TeX-master: "independence_ch_isabelle"
%%% ispell-local-dictionary: "american"
%%% End:
