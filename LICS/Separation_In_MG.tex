\documentclass[conference]{IEEEtran}
% Some Computer Society conferences also require the compsoc mode option,
% but others use the standard conference format.
%
% If IEEEtran.cls has not been installed into the LaTeX system files,
% manually specify the path to it like:
% \documentclass[conference]{../sty/IEEEtran}





% Some very useful LaTeX packages include:
% (uncomment the ones you want to load)
\usepackage[numbers]{natbib}
\usepackage{isabelle,isabellesym}
\usepackage{amsmath}
%\usepackage{amsthm}
\usepackage{amsfonts}
\usepackage{amssymb}
%\usepackage{bbm}  % Para el \bb{1}
%\usepackage[numbers]{natbib}
\usepackage{enumitem}
\usepackage{babel}
%\usepackage{babelbib}
\usepackage{multidef}
\usepackage{verbatim}
\usepackage{stmaryrd} %% para \llbracket
%%
%% \usepackage[bottom=2cm, top=2cm, left=2cm, right=2cm]{geometry}
%% \usepackage{titling}
%% \setlength{\droptitle}{-10ex} 
%%
\renewcommand{\o}{\vee}
\renewcommand{\O}{\bigvee}
\newcommand{\y}{\wedge}
\newcommand{\Y}{\bigwedge}
\newcommand{\limp}{\rightarrow}
\newcommand{\lsii}{\leftrightarrow}
%%

\DeclareMathOperator{\cf}{cf}
\DeclareMathOperator{\dom}{dom}
\DeclareMathOperator{\im}{img}
\DeclareMathOperator{\Fn}{Fn}
\DeclareMathOperator{\rk}{rk}
\DeclareMathOperator{\mos}{mos}
\DeclareMathOperator{\trcl}{trcl}
\DeclareMathOperator*{\diag}{\bigtriangleup}
\DeclareMathOperator{\Con}{Con}
\DeclareMathOperator{\Club}{Club}


\newcommand{\modelo}[1]{\mathbf{#1}}
\newcommand{\axiomas}[1]{\mathit{#1}}
\newcommand{\clase}[1]{\mathsf{#1}}
\newcommand{\poset}[1]{\mathbb{#1}}
\newcommand{\operador}[1]{\mathbf{#1}}

%% \newcommand{\Lim}{\clase{Lim}}
%% \newcommand{\Reg}{\clase{Reg}}
%% \newcommand{\Card}{\clase{Card}}
%% \newcommand{\On}{\clase{On}}
%% \newcommand{\WF}{\clase{WF}}
%% \newcommand{\HF}{\clase{HF}}
%% \newcommand{\HC}{\clase{HC}}
%%
%% El siguiente comando reemplaza todos los anteriores:
%%
\multidef{\clase{#1}}{Card,HC,HF,Lim,On->Ord,Reg,WF,Ord}
\newcommand{\ON}{\On}

%% En lugar de usar todo el paquete bbm:
\DeclareMathAlphabet{\mathbbm}{U}{bbm}{m}{n} 
\newcommand{\1}{\mathbbm{1}}

%%
%% \newcommand{\calD}{\mathcal{D}}
%% \newcommand{\calS}{\mathcal{S}}
%% \newcommand{\calU}{\mathcal{U}}
%% \newcommand{\calB}{\mathcal{B}}
%% \newcommand{\calL}{\mathcal{L}}
%% \newcommand{\calF}{\mathcal{F}}
%% \newcommand{\calT}{\mathcal{T}}
%% \newcommand{\calW}{\mathcal{W}}
%% \newcommand{\calA}{\mathcal{A}}
%%
%% El siguiente comando reemplaza todos los anteriores:
%%
\multidef[prefix=cal]{\mathcal{#1}}{A-Z}
%%
%% \newcommand{\A}{\modelo{A}}
%% \newcommand{\BB}{\modelo{B}}
%% \newcommand{\ZZ}{\modelo{Z}}
%% \newcommand{\PP}{\modelo{P}}
%% \newcommand{\QQ}{\modelo{Q}}
%% \newcommand{\RR}{\modelo{R}}
%%
%% El siguiente comando reemplaza todos los anteriores:
%%
\multidef{\modelo{#1}}{A,BB->B,CC->C,NN->N,PP->P,QQ->Q,RR->R,ZZ->Z}

\multidef[prefix=p]{\mathbb{#1}}{A-Z}
%% \newcommand{\B}{\modelo{B}}
%% \newcommand{\C}{\modelo{C}}
%% \newcommand{\F}{\modelo{F}}
%% \newcommand{\D}{\modelo{D}}

\newcommand{\Th}{\mb{Th}}
\newcommand{\Mod}{\mb{Mod}}

\newcommand{\Se}{\operador{S^\prec}}
\newcommand{\Pu}{\operador{P_u}}
\renewcommand{\Pr}{\operador{P_R}}
\renewcommand{\H}{\operador{H}}
\renewcommand{\S}{\operador{S}}
\newcommand{\I}{\operador{I}}
\newcommand{\E}{\operador{E}}

\newcommand{\se}{\preccurlyeq}
\newcommand{\ee}{\succ}
\newcommand{\id}{\approx}
\newcommand{\subm}{\subseteq}
\newcommand{\ext}{\supseteq}
\newcommand{\iso}{\cong}
%%
\renewcommand{\emptyset}{\varnothing}
\newcommand{\rel}{\mathcal{R}}
\newcommand{\Pow}{\mathop{\mathcal{P}}}
\renewcommand{\P}{\Pow}
\newcommand{\BP}{\mathrm{BP}}
\newcommand{\func}{\rightarrow}
\newcommand{\ord}{\mathrm{Ord}}
\newcommand{\R}{\mathbb{R}}
\newcommand{\N}{\mathbb{N}}
\newcommand{\Z}{\mathbb{Z}}
\renewcommand{\I}{\mathbb{I}}
\newcommand{\Q}{\mathbb{Q}}
\newcommand{\B}{\mathbf{B}}
\newcommand{\<}{\langle}
\renewcommand{\>}{\rangle}
\newcommand{\lb}{\langle}
\newcommand{\rb}{\rangle}
\newcommand{\impl}{\rightarrow}
\newcommand{\ent}{\Rightarrow}
\newcommand{\tne}{\Leftarrow}
\newcommand{\sii}{\Leftrightarrow}
\renewcommand{\phi}{\varphi}
\newcommand{\phis}{{\varphi^*}}
\renewcommand{\th}{\theta}
\newcommand{\Lda}{\Lambda}
\newcommand{\La}{\Lambda}
\newcommand{\lda}{\lambda}
\newcommand{\ka}{\kappa}
\newcommand{\del}{\delta}
\newcommand{\de}{\delta}
\newcommand{\ze}{\zeta}
%\newcommand{\ }{\ }
\newcommand{\la}{\lambda}
\newcommand{\al}{\alpha}
\newcommand{\be}{\beta}
\newcommand{\ga}{\gamma}
\newcommand{\Ga}{\Gamma}
\newcommand{\ep}{\varepsilon}
\newcommand{\De}{\Delta}
\newcommand{\defi}{\mathrel{\mathop:}=}
\newcommand{\forces}{\Vdash}
%\newcommand{\ap}{\mathbin{\wideparen{\ }}}
\newcommand{\Tree}{{\mathrm{Tr}_\N}}
\newcommand{\PTree}{{\mathrm{PTr}_\N}}
\newcommand{\NWO}{\mathit{NWO}}
\newcommand{\Suc}{{\N^{<\N}}}%
\newcommand{\init}{\mathsf{i}}
\newcommand{\ap}{\mathord{^\smallfrown}}
\newcommand{\Cantor}{\mathcal{C}}
%\newcommand{\C}{\Cantor}
\newcommand{\Baire}{\mathcal{N}}
\newcommand{\sig}{\ensuremath{\sigma}}
\newcommand{\fsig}{\ensuremath{F_\sigma}}
\newcommand{\gdel}{\ensuremath{G_\delta}}
\newcommand{\Sig}{\ensuremath{\boldsymbol{\Sigma}}}
\newcommand{\bPi}{\ensuremath{\boldsymbol{\Pi}}}
\newcommand{\Del}{\ensuremath{\boldsymbol\Delta}}
%\renewcommand{\F}{\operador{F}}
\newcommand{\ths}{{\theta^*}}
\newcommand{\om}{\ensuremath{\omega}}
%\renewcommand{\c}{\complement}
\newcommand{\comp}{\mathsf{c}}
\newcommand{\co}[1]{\left(#1\right)^\comp}
\newcommand{\len}[1]{\left|#1\right|}
\DeclareMathOperator{\tlim}{\overline{\mathrm{TLim}}}
\newcommand{\card}[1]{{\left|#1\right|}}
\newcommand{\bigcard}[1]{{\bigl|#1\bigr|}}
%
% Cardinality
%
\newcommand{\lec}{\leqslant_c}
\newcommand{\gec}{\geqslant_c}
\newcommand{\lc}{<_c}
\newcommand{\gc}{>_c}
\newcommand{\eqc}{=_c}
\newcommand{\biy}{\approx}
\newcommand*{\ale}[1]{\aleph_{#1}}
%
\newcommand{\Zerm}{\axiomas{Z}}
\newcommand{\ZC}{\axiomas{ZC}}
\newcommand{\AC}{\axiomas{AC}}
\newcommand{\DC}{\axiomas{DC}}
\newcommand{\MA}{\axiomas{MA}}
\newcommand{\CH}{\axiomas{CH}}
\newcommand{\ZFC}{\axiomas{ZFC}}
\newcommand{\ZF}{\axiomas{ZF}}
\newcommand{\Inf}{\axiomas{Inf}}
%
% Cardinal characteristics
%
\newcommand{\cont}{\mathfrak{c}}
\newcommand{\spl}{\mathfrak{s}}
\newcommand{\bound}{\mathfrak{b}}
\newcommand{\mad}{\mathfrak{a}}
\newcommand{\tower}{\mathfrak{t}}
%
\renewcommand{\hom}[2]{{}^{#1}\hskip-0.116ex{#2}}
\newcommand{\pred}[1][{}]{\mathop{\mathrm{pred}_{#1}}}
%% Postfix operator with supressable space:
%% \newcommand*{\iseg}{\relax\ifnum\lastnodetype>0 \mskip\medmuskip\fi{\downarrow}} %
\newcommand*{\iseg}{{\downarrow}}
\newcommand{\rr}{\mathrel{R}}
\newcommand{\restr}{\upharpoonright}
%\newcommand{\type}{\mathtt{}}
\newcommand{\app}{\mathop{\mathrm{Aprox}}}
\newcommand{\hess}{\triangleleft}
\newcommand{\bx}{\bar{x}}
\newcommand{\by}{\bar{y}}
\newcommand{\bz}{\bar{z}}
\newcommand{\union}{\mathop{\textstyle\bigcup}}
\newcommand{\sm}{\setminus}
\newcommand{\sbq}{\subseteq}
\newcommand{\nsbq}{\subseteq}
\newcommand{\mty}{\emptyset}
\newcommand{\dimg}{\text{\textup{``}}} % direct image
\newcommand{\quine}[1]{\ulcorner{\!#1\!}\urcorner}
%\newcommand{\ntrm}[1]{\textsl{\textbf{#1}}}
\newcommand{\Null}{\calN\!\mathit{ull}}
\DeclareMathOperator{\club}{Club}
\DeclareMathOperator{\otp}{otp}

%%%%%%%%%%%%%%%%%%%%%%%%%
% Variant aleph, beth, etc
% From http://tex.stackexchange.com/q/170476/69595
\makeatletter
\@ifpackageloaded{txfonts}\@tempswafalse\@tempswatrue
\if@tempswa
  \DeclareFontFamily{U}{txsymbols}{}
  \DeclareFontFamily{U}{txAMSb}{}
  \DeclareSymbolFont{txsymbols}{OMS}{txsy}{m}{n}
  \SetSymbolFont{txsymbols}{bold}{OMS}{txsy}{bx}{n}
  \DeclareFontSubstitution{OMS}{txsy}{m}{n}
  \DeclareSymbolFont{txAMSb}{U}{txsyb}{m}{n}
  \SetSymbolFont{txAMSb}{bold}{U}{txsyb}{bx}{n}
  \DeclareFontSubstitution{U}{txsyb}{m}{n}
  \DeclareMathSymbol{\aleph}{\mathord}{txsymbols}{64}
  \DeclareMathSymbol{\beth}{\mathord}{txAMSb}{105}
  \DeclareMathSymbol{\gimel}{\mathord}{txAMSb}{106}
  \DeclareMathSymbol{\daleth}{\mathord}{txAMSb}{107}
\fi
\makeatother

%%%%%%%%%%%%%%%%%%%%%%%%%%%%%%%%%%%%%%%%%%%%%%%%%%%%%%%%%%%%
%%
%% Theorem Environments
%%
%% \newtheorem{theorem}{Theorem}
%% \newtheorem{lemma}[theorem]{Lemma}
%% \newtheorem{prop}[theorem]{Proposition}
%% \newtheorem{corollary}[theorem]{Corollary}
%% \newtheorem{claim}{Claim}
%% \newtheorem*{claim*}{Claim}
%% \theoremstyle{definition}
%% \newtheorem{definition}[theorem]{Definition}
%% \newtheorem{remark}[theorem]{Remark}
%% \newtheorem{example}[theorem]{Example}
%% \theoremstyle{remark}
%% \newtheorem*{remark*}{Remark}
%%
%%%%%%%%%%%%%%%%%%%%%%%%%%%%%%%%%%%%%%%%%%%%%%%%%%%%%%%%%%%%%%%%%%%%%%

%% \newenvironment{inducc}{\begin{list}{}{\itemindent=2.5em \labelwidth=4em}}{\end{list}}
%% \newcommand{\caso}[1]{\item[\fbox{#1}]}
\newenvironment{proofofclaim}{\begin{proof}[Proof of Claim]}{\end{proof}}


%%% Local Variables: 
%%% mode: latex
%%% TeX-master: "first_steps_into_forcing"
%%% End: 


% *** MISC UTILITY PACKAGES ***
%
%\usepackage{ifpdf}
% Heiko Oberdiek's ifpdf.sty is very useful if you need conditional
% compilation based on whether the output is pdf or dvi.
% usage:
% \ifpdf
%   % pdf code
% \else
%   % dvi code
% \fi
% The latest version of ifpdf.sty can be obtained from:
% http://www.ctan.org/pkg/ifpdf
% Also, note that IEEEtran.cls V1.7 and later provides a builtin
% \ifCLASSINFOpdf conditional that works the same way.
% When switching from latex to pdflatex and vice-versa, the compiler may
% have to be run twice to clear warning/error messages.






% *** CITATION PACKAGES ***
%
%\usepackage{cite}
% cite.sty was written by Donald Arseneau
% V1.6 and later of IEEEtran pre-defines the format of the cite.sty package
% \cite{} output to follow that of the IEEE. Loading the cite package will
% result in citation numbers being automatically sorted and properly
% "compressed/ranged". e.g., [1], [9], [2], [7], [5], [6] without using
% cite.sty will become [1], [2], [5]--[7], [9] using cite.sty. cite.sty's
% \cite will automatically add leading space, if needed. Use cite.sty's
% noadjust option (cite.sty V3.8 and later) if you want to turn this off
% such as if a citation ever needs to be enclosed in parenthesis.
% cite.sty is already installed on most LaTeX systems. Be sure and use
% version 5.0 (2009-03-20) and later if using hyperref.sty.
% The latest version can be obtained at:
% http://www.ctan.org/pkg/cite
% The documentation is contained in the cite.sty file itself.






% *** GRAPHICS RELATED PACKAGES ***
%
\ifCLASSINFOpdf
  % \usepackage[pdftex]{graphicx}
  % declare the path(s) where your graphic files are
  % \graphicspath{{../pdf/}{../jpeg/}}
  % and their extensions so you won't have to specify these with
  % every instance of \includegraphics
  % \DeclareGraphicsExtensions{.pdf,.jpeg,.png}
\else
  % or other class option (dvipsone, dvipdf, if not using dvips). graphicx
  % will default to the driver specified in the system graphics.cfg if no
  % driver is specified.
  % \usepackage[dvips]{graphicx}
  % declare the path(s) where your graphic files are
  % \graphicspath{{../eps/}}
  % and their extensions so you won't have to specify these with
  % every instance of \includegraphics
  % \DeclareGraphicsExtensions{.eps}
\fi
% graphicx was written by David Carlisle and Sebastian Rahtz. It is
% required if you want graphics, photos, etc. graphicx.sty is already
% installed on most LaTeX systems. The latest version and documentation
% can be obtained at: 
% http://www.ctan.org/pkg/graphicx
% Another good source of documentation is "Using Imported Graphics in
% LaTeX2e" by Keith Reckdahl which can be found at:
% http://www.ctan.org/pkg/epslatex
%
% latex, and pdflatex in dvi mode, support graphics in encapsulated
% postscript (.eps) format. pdflatex in pdf mode supports graphics
% in .pdf, .jpeg, .png and .mps (metapost) formats. Users should ensure
% that all non-photo figures use a vector format (.eps, .pdf, .mps) and
% not a bitmapped formats (.jpeg, .png). The IEEE frowns on bitmapped formats
% which can result in "jaggedy"/blurry rendering of lines and letters as
% well as large increases in file sizes.
%
% You can find documentation about the pdfTeX application at:
% http://www.tug.org/applications/pdftex





% *** MATH PACKAGES ***
%
%\usepackage{amsmath}
% A popular package from the American Mathematical Society that provides
% many useful and powerful commands for dealing with mathematics.
%
% Note that the amsmath package sets \interdisplaylinepenalty to 10000
% thus preventing page breaks from occurring within multiline equations. Use:
%\interdisplaylinepenalty=2500
% after loading amsmath to restore such page breaks as IEEEtran.cls normally
% does. amsmath.sty is already installed on most LaTeX systems. The latest
% version and documentation can be obtained at:
% http://www.ctan.org/pkg/amsmath





% *** SPECIALIZED LIST PACKAGES ***
%
%\usepackage{algorithmic}
% algorithmic.sty was written by Peter Williams and Rogerio Brito.
% This package provides an algorithmic environment fo describing algorithms.
% You can use the algorithmic environment in-text or within a figure
% environment to provide for a floating algorithm. Do NOT use the algorithm
% floating environment provided by algorithm.sty (by the same authors) or
% algorithm2e.sty (by Christophe Fiorio) as the IEEE does not use dedicated
% algorithm float types and packages that provide these will not provide
% correct IEEE style captions. The latest version and documentation of
% algorithmic.sty can be obtained at:
% http://www.ctan.org/pkg/algorithms
% Also of interest may be the (relatively newer and more customizable)
% algorithmicx.sty package by Szasz Janos:
% http://www.ctan.org/pkg/algorithmicx




% *** ALIGNMENT PACKAGES ***
%
%\usepackage{array}
% Frank Mittelbach's and David Carlisle's array.sty patches and improves
% the standard LaTeX2e array and tabular environments to provide better
% appearance and additional user controls. As the default LaTeX2e table
% generation code is lacking to the point of almost being broken with
% respect to the quality of the end results, all users are strongly
% advised to use an enhanced (at the very least that provided by array.sty)
% set of table tools. array.sty is already installed on most systems. The
% latest version and documentation can be obtained at:
% http://www.ctan.org/pkg/array


% IEEEtran contains the IEEEeqnarray family of commands that can be used to
% generate multiline equations as well as matrices, tables, etc., of high
% quality.




% *** SUBFIGURE PACKAGES ***
%\ifCLASSOPTIONcompsoc
%  \usepackage[caption=false,font=normalsize,labelfont=sf,textfont=sf]{subfig}
%\else
%  \usepackage[caption=false,font=footnotesize]{subfig}
%\fi
% subfig.sty, written by Steven Douglas Cochran, is the modern replacement
% for subfigure.sty, the latter of which is no longer maintained and is
% incompatible with some LaTeX packages including fixltx2e. However,
% subfig.sty requires and automatically loads Axel Sommerfeldt's caption.sty
% which will override IEEEtran.cls' handling of captions and this will result
% in non-IEEE style figure/table captions. To prevent this problem, be sure
% and invoke subfig.sty's "caption=false" package option (available since
% subfig.sty version 1.3, 2005/06/28) as this is will preserve IEEEtran.cls
% handling of captions.
% Note that the Computer Society format requires a larger sans serif font
% than the serif footnote size font used in traditional IEEE formatting
% and thus the need to invoke different subfig.sty package options depending
% on whether compsoc mode has been enabled.
%
% The latest version and documentation of subfig.sty can be obtained at:
% http://www.ctan.org/pkg/subfig




% *** FLOAT PACKAGES ***
%
%\usepackage{fixltx2e}
% fixltx2e, the successor to the earlier fix2col.sty, was written by
% Frank Mittelbach and David Carlisle. This package corrects a few problems
% in the LaTeX2e kernel, the most notable of which is that in current
% LaTeX2e releases, the ordering of single and double column floats is not
% guaranteed to be preserved. Thus, an unpatched LaTeX2e can allow a
% single column figure to be placed prior to an earlier double column
% figure.
% Be aware that LaTeX2e kernels dated 2015 and later have fixltx2e.sty's
% corrections already built into the system in which case a warning will
% be issued if an attempt is made to load fixltx2e.sty as it is no longer
% needed.
% The latest version and documentation can be found at:
% http://www.ctan.org/pkg/fixltx2e


%\usepackage{stfloats}
% stfloats.sty was written by Sigitas Tolusis. This package gives LaTeX2e
% the ability to do double column floats at the bottom of the page as well
% as the top. (e.g., "\begin{figure*}[!b]" is not normally possible in
% LaTeX2e). It also provides a command:
%\fnbelowfloat
% to enable the placement of footnotes below bottom floats (the standard
% LaTeX2e kernel puts them above bottom floats). This is an invasive package
% which rewrites many portions of the LaTeX2e float routines. It may not work
% with other packages that modify the LaTeX2e float routines. The latest
% version and documentation can be obtained at:
% http://www.ctan.org/pkg/stfloats
% Do not use the stfloats baselinefloat ability as the IEEE does not allow
% \baselineskip to stretch. Authors submitting work to the IEEE should note
% that the IEEE rarely uses double column equations and that authors should try
% to avoid such use. Do not be tempted to use the cuted.sty or midfloat.sty
% packages (also by Sigitas Tolusis) as the IEEE does not format its papers in
% such ways.
% Do not attempt to use stfloats with fixltx2e as they are incompatible.
% Instead, use Morten Hogholm'a dblfloatfix which combines the features
% of both fixltx2e and stfloats:
%
% \usepackage{dblfloatfix}
% The latest version can be found at:
% http://www.ctan.org/pkg/dblfloatfix




% *** PDF, URL AND HYPERLINK PACKAGES ***
%
%\usepackage{url}
% url.sty was written by Donald Arseneau. It provides better support for
% handling and breaking URLs. url.sty is already installed on most LaTeX
% systems. The latest version and documentation can be obtained at:
% http://www.ctan.org/pkg/url
% Basically, \url{my_url_here}.




% *** Do not adjust lengths that control margins, column widths, etc. ***
% *** Do not use packages that alter fonts (such as pslatex).         ***
% There should be no need to do such things with IEEEtran.cls V1.6 and later.
% (Unless specifically asked to do so by the journal or conference you plan
% to submit to, of course. )


% correct bad hyphenation here
\hyphenation{op-tical net-works semi-conduc-tor}


\begin{document}
%
% paper title
% Titles are generally capitalized except for words such as a, an, and, as,
% at, but, by, for, in, nor, of, on, or, the, to and up, which are usually
% not capitalized unless they are the first or last word of the title.
% Linebreaks \\ can be used within to get better formatting as desired.
% Do not put math or special symbols in the title.
\title{Separation in generic extensions}


% author names and affiliations
% use a multiple column layout for up to three different
% affiliations
\author{\IEEEauthorblockN{Emmanuel Gunther}
\IEEEauthorblockA{FaMAF\\ Universidad Nacional de C\'ordoba\\
    C\'ordoba, Argentina\\
Email: gunther@famaf.unc.edu.ar}
\and
\IEEEauthorblockN{Miguel Pagano}
\IEEEauthorblockA{FaMAF\\ Universidad Nacional de C\'ordoba\\
    C\'ordoba, Argentina\\
Email: pagano@famaf.unc.edu.ar}
\and
\IEEEauthorblockN{Pedro S\'anchez Terraf}
\IEEEauthorblockA{CIEM-FaMAF\\Universidad Nacional de C\'ordoba\\
  C\'ordoba, Argentina\\
Email:  sterraf@famaf.unc.edu.ar}
}

% conference papers do not typically use \thanks and this command
% is locked out in conference mode. If really needed, such as for
% the acknowledgment of grants, issue a \IEEEoverridecommandlockouts
% after \documentclass

% for over three affiliations, or if they all won't fit within the width
% of the page, use this alternative format:
% 
%\author{\IEEEauthorblockN{Michael Shell\IEEEauthorrefmark{1},
%Homer Simpson\IEEEauthorrefmark{2},
%James Kirk\IEEEauthorrefmark{3}, 
%Montgomery Scott\IEEEauthorrefmark{3} and
%Eldon Tyrell\IEEEauthorrefmark{4}}
%\IEEEauthorblockA{\IEEEauthorrefmark{1}School of Electrical and Computer Engineering\\
%Georgia Institute of Technology,
%Atlanta, Georgia 30332--0250\\ Email: see http://www.michaelshell.org/contact.html}
%\IEEEauthorblockA{\IEEEauthorrefmark{2}Twentieth Century Fox, Springfield, USA\\
%Email: homer@thesimpsons.com}
%\IEEEauthorblockA{\IEEEauthorrefmark{3}Starfleet Academy, San Francisco, California 96678-2391\\
%Telephone: (800) 555--1212, Fax: (888) 555--1212}
%\IEEEauthorblockA{\IEEEauthorrefmark{4}Tyrell Inc., 123 Replicant Street, Los Angeles, California 90210--4321}}




% use for special paper notices
%\IEEEspecialpapernotice{(Invited Paper)}




% make the title area
\maketitle

% As a general rule, do not put math, special symbols or citations
% in the abstract
\begin{abstract}
We state the fundamental theorems of forcing and show that in the
Separation Axiom holds in generic extensions.
\end{abstract}

% no keywords




% For peer review papers, you can put extra information on the cover
% page as needed:
% \ifCLASSOPTIONpeerreview
% \begin{center} \bfseries EDICS Category: 3-BBND \end{center}
% \fi
%
% For peerreview papers, this IEEEtran command inserts a page break and
% creates the second title. It will be ignored for other modes.
\IEEEpeerreviewmaketitle



\section{Introduction}
% no \IEEEPARstart
We follow \cite{kunen2011set}. An introduction to the issues discussed
in this paper can be found in \cite{2018arXiv180705174G}. An
introduction to forcing can be found at \cite{2007arXiv0712.1320C},
and the book \cite{weaver2014forcing} contains a thorough treament
minimizing the technicalities.

\section{Forcing}
Given a ctm $M$, and an $M$-generic filter $G\sbq\PP$, the Forcing
Theorems relate satisfaction of a formula 
$\phi$ in the generic extension $M[G]$ to the satisfaction of another formula
$\phi'$ in $M$. The map $\phi\mapsto\phi'$ is defined by recursion on
the structure of $\phi$. The fact that this map works (in particular,
in order to show that $M[G]\models\ZFC$), we must use the fact that
$M\models\ZFC$. Since we can only assert the latter using internalized
formulas, the aformentioned map will be defined as a function from the
set \formula{} into itself. 

Up to this point, the main reason for working with internalized
versions is that  it is not possible to do recursion with formulas of
type $\tyo$ (i.e., the formulas in the first-order logic of
Isabelle/ZF).

We will now make more precise the definition of the map
$\phi\mapsto\phi'$ and how it relates satisfaction in $M$ to that in
$M[G]$. Actually, if the formula $\phi$ has $n$ free variables,
$\phi'$ will have $n+4$ free variables, where the first four account
for the forcing notion and a particular condition. If
$\phi=\phi(x_1,\dots,x_n)$, the standard notation for $\phi'$ is
\[
p\forces_{\PP,\leq,\1}^* \phi(t_1,\dots,t_n).
\]
Here, $\PP,\leq,\1,$ and $p\in\PP$ are the extra parameters. 
%% For the
%% time being, we just consider $\tau_1,\dots,\tau_n$ to be ordinary
%% variables. 
Given $\quine\phi\in\formula$, we will write this  in
Isabelle/ZF by using  
\[
\forceisa(\quine\phi,\PP,\leq,\1,p)
\]
Afterwards, the \emph{forcing relation} $\forces$ is defined by
relativizing/interpreting $\forces^*$ in a ctm $M$, for fixed
$\lb\PP,\leq,\1\rb\in M$:
\[
p\forces \phi(t_1,\dots,t_n) \defi 
\bigl(p\forces_{\PP,\leq,\1}^* \phi(t_1,\dots,t_n)\bigr)^M\!.
\]

% An example of a floating figure using the graphicx package.
% Note that \label must occur AFTER (or within) \caption.
% For figures, \caption should occur after the \includegraphics.
% Note that IEEEtran v1.7 and later has special internal code that
% is designed to preserve the operation of \label within \caption
% even when the captionsoff option is in effect. However, because
% of issues like this, it may be the safest practice to put all your
% \label just after \caption rather than within \caption{}.
%
% Reminder: the "draftcls" or "draftclsnofoot", not "draft", class
% option should be used if it is desired that the figures are to be
% displayed while in draft mode.
%
%\begin{figure}[!t]
%\centering
%\includegraphics[width=2.5in]{myfigure}
% where an .eps filename suffix will be assumed under latex, 
% and a .pdf suffix will be assumed for pdflatex; or what has been declared
% via \DeclareGraphicsExtensions.
%\caption{Simulation results for the network.}
%\label{fig_sim}
%\end{figure}

% Note that the IEEE typically puts floats only at the top, even when this
% results in a large percentage of a column being occupied by floats.


% An example of a double column floating figure using two subfigures.
% (The subfig.sty package must be loaded for this to work.)
% The subfigure \label commands are set within each subfloat command,
% and the \label for the overall figure must come after \caption.
% \hfil is used as a separator to get equal spacing.
% Watch out that the combined width of all the subfigures on a 
% line do not exceed the text width or a line break will occur.
%
%\begin{figure*}[!t]
%\centering
%\subfloat[Case I]{\includegraphics[width=2.5in]{box}%
%\label{fig_first_case}}
%\hfil
%\subfloat[Case II]{\includegraphics[width=2.5in]{box}%
%\label{fig_second_case}}
%\caption{Simulation results for the network.}
%\label{fig_sim}
%\end{figure*}
%
% Note that often IEEE papers with subfigures do not employ subfigure
% captions (using the optional argument to \subfloat[]), but instead will
% reference/describe all of them (a), (b), etc., within the main caption.
% Be aware that for subfig.sty to generate the (a), (b), etc., subfigure
% labels, the optional argument to \subfloat must be present. If a
% subcaption is not desired, just leave its contents blank,
% e.g., \subfloat[].


% An example of a floating table. Note that, for IEEE style tables, the
% \caption command should come BEFORE the table and, given that table
% captions serve much like titles, are usually capitalized except for words
% such as a, an, and, as, at, but, by, for, in, nor, of, on, or, the, to
% and up, which are usually not capitalized unless they are the first or
% last word of the caption. Table text will default to \footnotesize as
% the IEEE normally uses this smaller font for tables.
% The \label must come after \caption as always.
%
%\begin{table}[!t]
%% increase table row spacing, adjust to taste
%\renewcommand{\arraystretch}{1.3}
% if using array.sty, it might be a good idea to tweak the value of
% \extrarowheight as needed to properly center the text within the cells
%\caption{An Example of a Table}
%\label{table_example}
%\centering
%% Some packages, such as MDW tools, offer better commands for making tables
%% than the plain LaTeX2e tabular which is used here.
%\begin{tabular}{|c||c|}
%\hline
%One & Two\\
%\hline
%Three & Four\\
%\hline
%\end{tabular}
%\end{table}


% Note that the IEEE does not put floats in the very first column
% - or typically anywhere on the first page for that matter. Also,
% in-text middle ("here") positioning is typically not used, but it
% is allowed and encouraged for Computer Society conferences (but
% not Computer Society journals). Most IEEE journals/conferences use
% top floats exclusively. 
% Note that, LaTeX2e, unlike IEEE journals/conferences, places
% footnotes above bottom floats. This can be corrected via the
% \fnbelowfloat command of the stfloats package.




\section{Conclusion}



% conference papers do not normally have an appendix


% use section* for acknowledgment
\section*{Acknowledgment}





% trigger a \newpage just before the given reference
% number - used to balance the columns on the last page
% adjust value as needed - may need to be readjusted if
% the document is modified later
%\IEEEtriggeratref{8}
% The "triggered" command can be changed if desired:
%\IEEEtriggercmd{\enlargethispage{-5in}}

% references section

% can use a bibliography generated by BibTeX as a .bbl file
% BibTeX documentation can be easily obtained at:
% http://mirror.ctan.org/biblio/bibtex/contrib/doc/
% The IEEEtran BibTeX style support page is at:
% http://www.michaelshell.org/tex/ieeetran/bibtex/
\bibliographystyle{IEEEtranSN}
% argument is your BibTeX string definitions and bibliography database(s)
\bibliography{../LSFA/citados}
%
% <OR> manually copy in the resultant .bbl file
% set second argument of \begin to the number of references
% (used to reserve space for the reference number labels box)

% that's all folks
\end{document}


