\documentclass[11pt,english]{article}
\usepackage{babel}
\usepackage[numbers]{natbib}
%
\input{header-draft}
%
%
\begin{document}
\title{First steps towards a formalization of Forcing
  \\ Revisions made to the paper}
\author{E.~Gunther \and  M.~Pagano \and P.~S\'anchez~Terraf
  %% fourth author was partially
  %% supported by CONICET, ANPCyT project PICT 2012-1823, and by
  %% SeCyT-UNC project 05/B284.}
}
\date{}
\maketitle

In first place we want to thank the referees for reading the paper very
carefully and for his detailed and constructive criticism. Below we
detail the main changes that have been made%; also, a diff-PDF is
%provided that shows all the changes in color
.

We have spelled-check the manuscript, corrected punctuation, and took
notice of all the typos indicated in the report. 

The Introduction has many changes. A whole new section (``Related
work'') has been added, where the formalization of mathematics in
general and of forcing and set theory in particular are discussed.
Also, at the end of the subsection on Isabelle we describe the tools
available for implementation, and a comparison
between the two dialects available, the basic procedural style and the
declarative \emph{Isar}. At some point of the Introduction, we also
remark that our 
development follows the presentation by Kunen \cite{kunen2011set}. 

We added several comments on the implementation details, especially in
the section on forcing notions; this includes a description of the
complexity in number of lines and proof commands, and some hints on
the mechanics of the use Isabelle.

A thorough comparison between the available results on $\AC$ and
$\DC$, and the need for reimplementing some of those was included in
the subsection on Dependent Choices.

The section on ``Names'' includes two new applications. One is the
proof that the generic
extension $M[G]$ is a transitive set. The second one is even more basic  and
was missing in the two previous versions of the manuscript: The
generic filter $G$ belongs to $M[G]$. 
%
%
\bibliographystyle{arXiv/mi-estilo-else}
\bibliography{citados}
\end{document}
%%% Local Variables: 
%%% mode: latex
%%% ispell-local-dictionary: "american"
%%% End: 
