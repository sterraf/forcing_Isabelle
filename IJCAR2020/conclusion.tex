\section{Conclusion and future work}
\label{sec:conclusion}

We consider that the formalization of the definition of $\forceisa$
and its recursive characterization of forcing for atomic formulas is
turning point
in our project; the reason for this is that all further
developments will not involve such a daunting metamathematical
component. Even the proofs of the Fundamental Theorems of Forcing
turned out to follow rather smoothly after this initial setup was
ready, the only complicated affair being to show that various dense sets belong
to $M$. 
\begin{framed}
  Actually, this is a point to taken care of: the machinery of
  absoluteness and relativization must be developed for every new
  concept that is introduced.
\end{framed}

In the course of obtaining internal formulas for the atomic case of
forcing, a fruitful discussion
concerning complementary perspectives on the role of proof assistants
took place. An earlier approach relied more heavily in formula
synthesis, thus making the Isabelle simplifier an indispensable main
character. Following this line was quickier from the coding point of
view since few new primitives were introduced and thus fewer lemmas
concerning absoluteness and arities. On the downside, processing was a
bit slower, the formulas synthesized were gigantic and the process on
a whole as more error-prone. In fact, this approach was unsuccessful
and we opted for a more detailed engineering, defining all
intermediate steps. So the load on the assistant, in this part of the
development, balanced from code-production to code-verification. 

The next task in our path is pretty clear: To develop the forcing
notions to obtain the independence of $\CH$ 
along with the prerrequisite combinatorial results, v.g.\ the
$\Delta$-system lemma. A development of cofinality is under way in a
joint work with E.~Pacheco Rodríguez, which is needed for a general
statement of the latter. 

There are other debts, as well. In a second
release of \isatt{ZF-Constructible}, we intend to conform it to the
lines of \emph{Basic Set Theory (BST)} \cite[I.3.1]{kunen2011set} in
which elementary results have proofs using alternatively Powerset or
Replacement. The interest in this arises because many natural set
models
(rank-initial segments of the universe or the family $H(\kappa)$ of
sets hereditarily of cardinality less than $\kappa$) satisfy one of
those axioms and not the other. There are also still some older or
less significant proofs written in tactical (\textbf{apply}) format; we
hope we'll find the time to translate them to Isar. Finally, the
automation of formula synthesis is on an early stage of
development;  finishing that module will make writing our proofs of closure
under various operations faster.
\bigskip
 
COMMENTS:
\begin{enumerate}
\item No merit in ``not formalizing'' the Reflection Principle.
\end{enumerate}

%%% Local Variables: 
%%% mode: latex
%%% TeX-master: "forcing_in_isabelle_zf"
%%% ispell-local-dictionary: "american"
%%% End: 
