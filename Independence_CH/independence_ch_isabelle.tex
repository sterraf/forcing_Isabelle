% This is samplepaper.tex, a sample chapter demonstrating the
% LLNCS macro package for Springer Computer Science proceedings;
% Version 2.20 of 2017/10/04
%
\documentclass[runningheads]{llncs}
%
%\usepackage[utf8]{inputenc}
%\usepackage[final]{microtype}
\usepackage{isabelle_indepCH,isabellesym_indepCH}
\usepackage{booktabs,array,threeparttable}
\usepackage{amsmath,amsfonts,amssymb}
\usepackage{bbm}  % Para el \bb{1}
\usepackage{tikz}
\usepackage[english]{babel}
\usepackage{multidef}
\usepackage{verbatim}
\usepackage{stmaryrd} %% para \llbracket
\usepackage{hyperref}
\usepackage{xcolor}
\usepackage{framed}
\usepackage[numbers]{natbib}

%%
%% \usepackage[bottom=2cm, top=2cm, left=2cm, right=2cm]{geometry}
%% \usepackage{titling}
%% \setlength{\droptitle}{-10ex} 
%%
\renewcommand{\o}{\vee}
\renewcommand{\O}{\bigvee}
\newcommand{\y}{\wedge}
\newcommand{\Y}{\bigwedge}
\newcommand{\limp}{\longrightarrow}
\newcommand{\lsii}{\longleftrightarrow}
%%
%\newcommand{\DeclareMathOperator}[2]{\newcommand{#1}{\mathop{\mathrm{#2}}}}

\DeclareMathOperator{\cf}{cf}
\DeclareMathOperator{\dom}{domain}
\DeclareMathOperator{\im}{img}
\DeclareMathOperator{\Fn}{Fn}
\DeclareMathOperator{\rk}{rk}
\DeclareMathOperator{\mos}{mos}
\DeclareMathOperator{\trcl}{trcl}
\DeclareMathOperator{\Con}{Con}
\DeclareMathOperator{\Club}{Club}


\newcommand{\modelo}[1]{\mathbf{#1}}
\newcommand{\axiomas}[1]{\mathit{#1}}
\newcommand{\clase}[1]{\mathsf{#1}}
\newcommand{\poset}[1]{\mathbb{#1}}
\newcommand{\operador}[1]{\mathbf{#1}}

%% \newcommand{\Lim}{\clase{Lim}}
%% \newcommand{\Reg}{\clase{Reg}}
%% \newcommand{\Card}{\clase{Card}}
%% \newcommand{\On}{\clase{On}}
%% \newcommand{\WF}{\clase{WF}}
%% \newcommand{\HF}{\clase{HF}}
%% \newcommand{\HC}{\clase{HC}}
%%
%% El siguiente comando reemplaza todos los anteriores:
%%
\multidef{\clase{#1}}{Card,HC,HF,Lim,On->Ord,Reg,WF,Ord}
\newcommand{\ON}{\On}

%% En lugar de usar todo el paquete bbm:
\DeclareMathAlphabet{\mathbbm}{U}{bbm}{m}{n} 
\newcommand{\1}{\mathbbm{1}}
\newcommand{\PP}{\mathbbm{P}}

%%
%% \newcommand{\calD}{\mathcal{D}}
%% \newcommand{\calS}{\mathcal{S}}
%% \newcommand{\calU}{\mathcal{U}}
%% \newcommand{\calB}{\mathcal{B}}
%% \newcommand{\calL}{\mathcal{L}}
%% \newcommand{\calF}{\mathcal{F}}
%% \newcommand{\calT}{\mathcal{T}}
%% \newcommand{\calW}{\mathcal{W}}
%% \newcommand{\calA}{\mathcal{A}}
%%
%% El siguiente comando reemplaza todos los anteriores:
%%
\multidef[prefix=cal]{\mathcal{#1}}{A-Z}
%%
%% \newcommand{\A}{\modelo{A}}
%% \newcommand{\BB}{\modelo{B}}
%% \newcommand{\ZZ}{\modelo{Z}}
%% \newcommand{\PP}{\modelo{P}}
%% \newcommand{\QQ}{\modelo{Q}}
%% \newcommand{\RR}{\modelo{R}}
%%
%% El siguiente comando reemplaza todos los anteriores:
%%
\multidef{\modelo{#1}}{A,BB->B,CC->C,NN->N,QQ->Q,RR->R,ZZ->Z}

\multidef[prefix=p]{\mathbb{#1}}{A-Z}
%% \newcommand{\B}{\modelo{B}}
%% \newcommand{\C}{\modelo{C}}
%% \newcommand{\F}{\modelo{F}}
%% \newcommand{\D}{\modelo{D}}

\newcommand{\Th}{\mb{Th}}
\newcommand{\Mod}{\mb{Mod}}

\newcommand{\Se}{\operador{S^\prec}}
\newcommand{\Pu}{\operador{P_u}}
\renewcommand{\Pr}{\operador{P_R}}
\renewcommand{\H}{\operador{H}}
\renewcommand{\S}{\operador{S}}
\newcommand{\I}{\operador{I}}
\newcommand{\E}{\operador{E}}

\newcommand{\se}{\preccurlyeq}
\newcommand{\ee}{\succ}
\newcommand{\id}{\approx}
\newcommand{\subm}{\subseteq}
\newcommand{\ext}{\supseteq}
\newcommand{\iso}{\cong}
%%
\renewcommand{\emptyset}{\varnothing}
\newcommand{\rel}{\mathcal{R}}
\newcommand{\Pow}{\mathop{\mathcal{P}}}
\renewcommand{\P}{\Pow}
\newcommand{\BP}{\mathrm{BP}}
\newcommand{\func}{\rightarrow}
\newcommand{\ord}{\mathrm{Ord}}
\newcommand{\R}{\mathbb{R}}
\newcommand{\N}{\mathbb{N}}
\newcommand{\Z}{\mathbb{Z}}
\renewcommand{\I}{\mathbb{I}}
\newcommand{\Q}{\mathbb{Q}}
\newcommand{\B}{\mathbf{B}}
\newcommand{\lb}{\langle}
\newcommand{\rb}{\rangle}
\newcommand{\impl}{\rightarrow}
\newcommand{\ent}{\Rightarrow}
\newcommand{\tne}{\Leftarrow}
\newcommand{\sii}{\Leftrightarrow}
\renewcommand{\phi}{\varphi}
\newcommand{\phis}{{\varphi^*}}
\renewcommand{\th}{\theta}
\newcommand{\Lda}{\Lambda}
\newcommand{\La}{\Lambda}
\newcommand{\lda}{\lambda}
\newcommand{\ka}{\kappa}
\newcommand{\del}{\delta}
\newcommand{\de}{\delta}
\newcommand{\ze}{\zeta}
%\newcommand{\ }{\ }
\newcommand{\la}{\lambda}
\newcommand{\al}{\alpha}
\newcommand{\be}{\beta}
\newcommand{\ga}{\gamma}
\newcommand{\Ga}{\Gamma}
\newcommand{\ep}{\varepsilon}
\newcommand{\De}{\Delta}
\newcommand{\defi}{\mathrel{\mathop:}=}
\newcommand{\forces}{\Vdash}
%\newcommand{\ap}{\mathbin{\wideparen{\ }}}
\newcommand{\Tree}{{\mathrm{Tr}_\N}}
\newcommand{\PTree}{{\mathrm{PTr}_\N}}
\newcommand{\NWO}{\mathit{NWO}}
\newcommand{\Suc}{{\N^{<\N}}}%
\newcommand{\init}{\mathsf{i}}
\newcommand{\ap}{\mathord{^\smallfrown}}
\newcommand{\Cantor}{\mathcal{C}}
%\newcommand{\C}{\Cantor}
\newcommand{\Baire}{\mathcal{N}}
\newcommand{\sig}{\ensuremath{\sigma}}
\newcommand{\fsig}{\ensuremath{F_\sigma}}
\newcommand{\gdel}{\ensuremath{G_\delta}}
\newcommand{\Sig}{\ensuremath{\boldsymbol{\Sigma}}}
\newcommand{\bPi}{\ensuremath{\boldsymbol{\Pi}}}
\newcommand{\Del}{\ensuremath{\boldsymbol\Delta}}
%\renewcommand{\F}{\operador{F}}
\newcommand{\ths}{{\theta^*}}
\newcommand{\om}{\ensuremath{\omega}}
%\renewcommand{\c}{\complement}
\newcommand{\comp}{\mathsf{c}}
\newcommand{\co}[1]{\left(#1\right)^\comp}
\newcommand{\len}[1]{\left|#1\right|}
\DeclareMathOperator{\tlim}{\overline{\mathrm{TLim}}}
\newcommand{\card}[1]{{\left|#1\right|}}
\newcommand{\bigcard}[1]{{\bigl|#1\bigr|}}
%
% Cardinality
%
\newcommand{\lec}{\leqslant_c}
\newcommand{\gec}{\geqslant_c}
\newcommand{\lc}{<_c}
\newcommand{\gc}{>_c}
\newcommand{\eqc}{=_c}
\newcommand{\biy}{\approx}
\newcommand*{\ale}[1]{\aleph_{#1}}
%
\newcommand{\Zerm}{\axiomas{Z}}
\newcommand{\ZC}{\axiomas{ZC}}
\newcommand{\AC}{\axiomas{AC}}
\newcommand{\DC}{\axiomas{DC}}
\newcommand{\MA}{\axiomas{MA}}
\newcommand{\CH}{\axiomas{CH}}
\newcommand{\ZFC}{\axiomas{ZFC}}
\newcommand{\ZF}{\axiomas{ZF}}
\newcommand{\Inf}{\axiomas{Inf}}
%
% Cardinal characteristics
%
\newcommand{\cont}{\mathfrak{c}}
\newcommand{\spl}{\mathfrak{s}}
\newcommand{\bound}{\mathfrak{b}}
\newcommand{\mad}{\mathfrak{a}}
\newcommand{\tower}{\mathfrak{t}}
%
\renewcommand{\hom}[2]{{}^{#1}\hskip-0.116ex{#2}}
\newcommand{\pred}[1][{}]{\mathop{\mathrm{pred}_{#1}}}
%% Postfix operator with supressable space:
%% \newcommand*{\iseg}{\relax\ifnum\lastnodetype>0 \mskip\medmuskip\fi{\downarrow}} %
\newcommand*{\iseg}{{\downarrow}}
\newcommand{\rr}{\mathrel{R}}
\newcommand{\restr}{\upharpoonright}
%\newcommand{\type}{\mathtt{}}
\newcommand{\app}{\mathop{\mathrm{Aprox}}}
\newcommand{\hess}{\triangleleft}
\newcommand{\bx}{\bar{x}}
\newcommand{\by}{\bar{y}}
\newcommand{\bz}{\bar{z}}
\newcommand{\union}{\mathop{\textstyle\bigcup}}
\newcommand{\sm}{\setminus}
\newcommand{\sbq}{\subseteq}
\newcommand{\nsbq}{\subseteq}
\newcommand{\mty}{\emptyset}
\newcommand{\dimg}{\text{\textup{``}}} % direct image
\newcommand{\quine}[1]{\ulcorner{\!#1\!}\urcorner}
%\newcommand{\ntrm}[1]{\textsl{\textbf{#1}}}
\newcommand{\Null}{\calN\!\mathit{ull}}
\DeclareMathOperator{\club}{Club}
\DeclareMathOperator{\otp}{otp}
\DeclareMathOperator{\val}{\mathit{val}}
\DeclareMathOperator{\chk}{\mathit{check}}
\DeclareMathOperator{\edrel}{\mathit{edrel}}
\DeclareMathOperator{\eclose}{\mathit{eclose}}
\DeclareMathOperator{\Memrel}{\mathit{Memrel}}
\renewcommand{\PP}{\mathbb{P}}
\renewcommand{\app}{\mathrm{App}}
\newcommand{\formula}{\isatt{formula}}
\newcommand{\tyi}{\isatt{i}}
\newcommand{\tyo}{\isatt{o}}
\newcommand{\forceisa}{\mathop{\mathtt{forces}}}
\newcommand{\equ}{\mathbf{e}}
\newcommand{\bel}{\mathbf{b}}
\newcommand{\atr}{\mathit{atr}}
\newcommand{\concat}{\mathbin{@}}
\newcommand{\dB}[1]{\mathbf{#1}}
\newcommand{\ed}{\mathrel{\isatt{ed}}}
\newcommand{\frecR}{\mathrel{\isatt{frecR}}}
\newcommand{\forceseq}{\mathop{\isatt{forces{\isacharunderscore}eq}}}
\newcommand{\forcesmem}{\mathop{\isatt{forces{\isacharunderscore}mem}}}
\newcommand{\forcesat}{\mathop{\isatt{forces{\isacharunderscore}at}}}
\newcommand{\pleq}{\preceq}
%\renewcommand{\isacharunderscorekeyword}{\mbox{\_}}
%\renewcommand{\isacharunderscore}{\mbox{\_}}
\renewcommand{\isasymtturnstile}{\isamath{\Vdash}}
\renewcommand{\isacharminus}{-}
\newcommand{\uscore}{\isacharunderscore}
\newcommand{\session}[1]{\textit{#1}}
\newcommand{\theory}[1]{\texttt{#1}}
\newcommand{\punto}[1]{\overset{\tikz\draw[fill=black] (0,0) circle (0.6pt);}{#1}}

%%%%%%%%%%%%%%%%%%%%%%%%%
% Variant aleph, beth, etc
% From http://tex.stackexchange.com/q/170476/69595
\makeatletter
\@ifpackageloaded{txfonts}\@tempswafalse\@tempswatrue
\if@tempswa
  \DeclareFontFamily{U}{txsymbols}{}
  \DeclareFontFamily{U}{txAMSb}{}
  \DeclareSymbolFont{txsymbols}{OMS}{txsy}{m}{n}
  \SetSymbolFont{txsymbols}{bold}{OMS}{txsy}{bx}{n}
  \DeclareFontSubstitution{OMS}{txsy}{m}{n}
  \DeclareSymbolFont{txAMSb}{U}{txsyb}{m}{n}
  \SetSymbolFont{txAMSb}{bold}{U}{txsyb}{bx}{n}
  \DeclareFontSubstitution{U}{txsyb}{m}{n}
  \DeclareMathSymbol{\aleph}{\mathord}{txsymbols}{64}
  \DeclareMathSymbol{\beth}{\mathord}{txAMSb}{105}
  \DeclareMathSymbol{\gimel}{\mathord}{txAMSb}{106}
  \DeclareMathSymbol{\daleth}{\mathord}{txAMSb}{107}
\fi
\makeatother

%%%%%%%%%%%%%%%%%%%%%%%%%%%%%%%%%%%%%%%%%%%%%%%%%%%%%%%%%%%%
%%
%% Theorem Environments
%%
% \newtheorem{theorem}{Theorem}
% \newtheorem{lemma}[theorem]{Lemma}
% \newtheorem{prop}[theorem]{Proposition}
% \newtheorem{corollary}[theorem]{Corollary}
% \newtheorem{claim}{Claim}
% \newtheorem*{claim*}{Claim}
% \theoremstyle{definition}
% \newtheorem{definition}[theorem]{Definition}
% \newtheorem{remark}[theorem]{Remark}
% \newtheorem{example}[theorem]{Example}
% \theoremstyle{remark}
% \newtheorem*{remark*}{Remark}

%%%%%%%%%%%%%%%%%%%%%%%%%%%%%%%%%%%%%%%%%%%%%%%%%%%%%%%%%%%%%%%%%%%%%%

%% \newenvironment{inducc}{\begin{list}{}{\itemindent=2.5em \labelwidth=4em}}{\end{list}}
%% \newcommand{\caso}[1]{\item[\fbox{#1}]}
\newenvironment{proofofclaim}{\begin{proof}[Proof of Claim]}{\end{proof}}

\newcommand{\quantRel}[3]{#1 #2\kern -1pt[#3]}
\newcommand{\forallRel}[2]{\quantRel{\forall}{#1}{#2}}
\newcommand{\existsRel}[2]{\quantRel{\exists}{#1}{#2}}

\newif\ifarXiv
\newif\ifIEEE

%%% Local Variables: 
%%% mode: latex
%%% TeX-master: "independence_ch_isabelle"
%%% End: 

\usepackage{graphicx}
% Used for displaying a sample figure. If possible, figure files should
% be included in EPS format.
%
\hypersetup{
  colorlinks,
  urlcolor={blue},
  linkcolor={blue!50!black},
  citecolor={blue!50!black},
}
% If you use the hyperref package, please uncomment the following line
% to display URLs in blue roman font according to Springer's eBook style:
% \renewcommand\UrlFont{\color{blue}\rmfamily}
\DeclareUnicodeCharacter{2200}{\ensuremath{\forall}}
\DeclareUnicodeCharacter{2203}{\ensuremath{\exists}}
\DeclareUnicodeCharacter{2227}{\ensuremath{\wedge}}
\DeclareUnicodeCharacter{2228}{\ensuremath{\vee}}
\DeclareUnicodeCharacter{22C0}{\ensuremath{\bigwedge}}
\DeclareUnicodeCharacter{2261}{\ensuremath{\equiv}}
\DeclareUnicodeCharacter{27F9}{\ensuremath{\Longrightarrow}}
\DeclareUnicodeCharacter{27F7}{\ensuremath{\leftrightarrow}}
\DeclareUnicodeCharacter{27F6}{\ensuremath{\longrightarrow}}
\DeclareUnicodeCharacter{03BB}{\ensuremath{\lambda}}

\DeclareUnicodeCharacter{2208}{\ensuremath{\in}}
\DeclareUnicodeCharacter{2209}{\ensuremath{\not\in}}
\DeclareUnicodeCharacter{2286}{\ensuremath{\subseteq}}
\DeclareUnicodeCharacter{22C3}{\ensuremath{\bigcup}}
\DeclareUnicodeCharacter{21D2}{\ensuremath{\Rightarrow}}
\usepackage{array}
\newcounter{replInstCount}
\setcounter{replInstCount}{0}

\newcounter{LamReplCount}
\setcounter{LamReplCount}{0}
\begin{document}
%
\title{The formal verification of the ctm approach to forcing%Some lessons after the formalization of the ctm approach to forcing%
  \thanks{Supported by Secyt-UNC projects 33620180100855CB and 33620180100465CB, and Conicet.}%
}
%
\titlerunning{Formalization of ctm forcing}%Lessons after formalizing ctm forcing}
% If the paper title is too long for the running head, you can set
% an abbreviated paper title here
%
\author{Emmanuel Gunther\inst{1} \and
Miguel Pagano\inst{1} \and \\
Pedro Sánchez Terraf\inst{1,2}%\orcidID{0000-0003-3928-6942}
\and
Matías Steinberg\inst{1}
}
%
\authorrunning{E.~Gunther, M.~Pagano, P.~Sánchez Terraf, M.~Steinberg}
% First names are abbreviated in the running head.
% If there are more than two authors, 'et al.' is used.
%
\institute{Universidad Nacional de C\'ordoba. 
  \\  Facultad de Matem\'atica, Astronom\'{\i}a,  F\'{\i}sica y
  Computaci\'on.
  \and
    Centro de Investigaci\'on y Estudios de Matem\'atica (CIEM-FaMAF),
    Conicet. C\'ordoba. Argentina. \\
    \email{\{gunther,sterraf\}@famaf.unc.edu.ar\\
        \{miguel.pagano,matias.steinberg\}@unc.edu.ar}
}
%
\maketitle              % typeset the header of the contribution
%
\begin{abstract}
  We discuss some highlights of our computer-verified
  proof of the construction, given a countable transitive set-model $M$
  of $\ZFC$, of generic extensions  satisfying $\ZFC+\neg\CH$ and $\ZFC+\CH$.
  Moreover, let $\calR$ be the set of instances of the Axiom of
  Replacement. We isolated a 21-element subset $\Omega\sbq\calR$ and
  defined $\calF:\calR\to\calR$
  such that for every finite $\Phi\sbq\calR$
  we have
  $M\models \ZC \cup \calF\dimg\Phi \cup \Omega$ implies
  $M[G]\models \ZC \cup \Phi \cup \{ \neg \CH \}$ for any $M$-generic $G$, where $\ZC$ is
  Zermelo set theory with Choice.
  We also obtained the formulas yielded by the Forcing Definability Theorem
  explicitly.

  To achieve this, we worked in the proof assistant \emph{Isabelle},
  basing our development on the theory Isabelle/ZF by L.~Paulson and
  others.

  \keywords{forcing \and Isabelle/ZF \and countable transitive models
    \and continuum hypothesis
    \and proof assistants \and interactive theorem provers \and generic extension}
\end{abstract}
%
%
%
\section{Introduction}
\label{sec:introduction}

This paper is the culmination of our project on the computerized
formalization of the undecidability of the Continuum Hypothesis
($\CH$) from Zermelo-Fraenkel set theory with Choice ($\ZFC$), under the
assumption of the existence of a countable transitive model (ctm) of
$\ZFC$. In contrast to our reports of the previous steps towards this
goal
\cite{2018arXiv180705174G,2019arXiv190103313G,2020arXiv200109715G}, we
intend here to present our development to the mathematical logic
community. For this reason, we start with a general discussion around
the formalization of mathematics.

\subsection{Formalized mathematics}
The use of computers to assist the creation and verification of
mathematics has seen a steady grow. But the general awareness on the
matter still seems to be a bit scant (even among mathematicians
involved in foundations), and the venues devoted to the communication
of formalized mathematics are, mainly, computer science journals and
conferences: JAR, ITP, IJCAR, CPP, CICM, and others.

Nevertheless, the discussion about the subject in central mathematical
circles is increasing; there were some hints on the ICM2018 panel on
“machine-assisted” proofs
\cite{https://doi.org/10.48550/arxiv.1809.08062} and a lively
promotion by Kevin Buzzard, during his ICM2022 special plenary lecture
\cite{2021arXiv211211598B}.

%% These assistants provide several dialects, among which we single out:
%% \begin{enumerate}
%% \item Procedural: Useful for exploration/research.
%% \item Declarative: Only one that can be read by humans!
%% \end{enumerate}

Before we start an in-depth discussion, a point should be made clear:
A formalized proof is not the same as an \emph{automatic proof}. The
reader surely understands that, aside from results of a very specific sort, no current
technology allows us to write a reasonably complex (and correct)
theorem statement in a computer and expect to obtain a proof after hitting “Enter”, at
least not after a humanly feasible wait. On the other hand, it is
quite possible that the same reader has some mental image that
formalizing a proof requires making each application of Modus Ponens
explicit.

The fact is that \emph{proof assistants} are designed for the human prover to
be able to decompose a statement to be proved into smaller subgoals
which can actually be fed into some automatic tool. The balance between
what these tools are able to handle is not  easily appreciated by
intuition: Sometimes, ``trivial'' steps are not solved by them, which
can result in obvious frustration; but they would quickly solve some
goals that do not look like a ``mere computation.''

To appreciate the extent of mathematics formalizable, it is convenient to recall
some major successful projects, such as the Four Color Theorem
\cite{MR2463991}, the Odd Order Theorem
\cite{10.1007/978-3-642-39634-2_14}, and the proof the Kepler's
Conjecture \cite{MR3659768}. There is a vast mathematical corpus at
the Archive of Formal Proofs (AFP) based on Isabelle; and formalizations of
brand new mathematics like the Liquid Tensor Experiment
\cite{LTE2020,LTE2021} and the definition of perfectoid spaces \cite{10.1145/3372885.3373830}
have been achieved using Lean.

We will continue our description of proof assistants in
Section~\ref{sec:proof-assist-isabelle}. We kindly invite the reader
to enrich the previous exposition by reading the apt summary by
A.~Koutsoukou-Argyraki \cite{angeliki} and the interviews
therein; some of the experts consulted have also discussed
in \cite{2022arXiv220704779B} the status of formalized versus standard
proof in mathematics.

\subsection{Our achievements}
We formalized a model-theoretic rendition of forcing (Sect.~\ref{sec:forcing}), showing how to
construct proper extensions of ctms of $\ZF$ (respectively, with
$\AC$), and we formalized the basic forcing notions required to obtain
ctms of $\ZFC + \neg\CH$ and of $\ZFC + \CH$ (Sect.~\ref{sec:models-ch-negation}). No metatheoretic issues
(consistency, FOL calculi, etc) were formalized, so we were mainly
concerned with the mathematics of forcing. Nevertheless, by inspecting
the foundations underlying our proof assistant Isabelle
(Section~\ref{sec:isabelle-metalogic-meta}) it can be stated that our
formalization is a bona fide proof in $\ZF$ of the previous
constructions.

In order to reach our goals, we provided basic results that were
missing from Isabelle's $\ZF$ library, starting from ones
involving cardinal successors, countable sets, etc.
(Section~\ref{sec:extension-isabellezf}). We also extended the treatment of relativization of
set-theoretical concepts (Section~\ref{sec:tools-relativization}).
%% We redesigned Isabelle/ZF results on non-absolute concepts to work
%% relative to a class.

One added value that is obtained from the present formalization is
that we identified a handful of instances of Replacement which are
sufficient to set the forcing machinery up
(Section~\ref{sec:repl-instances}), on the basis of Zermelo set theory.
The eagerness to obtain this level of detail might be a consequence of
“an unnatural tendency to investigate, for the most part, trivial
minutiae of the formalism” on our part, as it was put by Cohen
\cite{zbMATH02012060}, but we would rather say that we were driven by
curiosity.

The code of our formalization can be accessed at the
AFP site, via the following link:
\begin{center}
  \url{https://www.isa-afp.org/entries/Independence_CH.html}
\end{center}

%%% Local Variables: 
%%% mode: latex
%%% TeX-master: "independence_ch_isabelle"
%%% ispell-local-dictionary: "american"
%%% End: 

 
\section{Proof assistants and Isabelle/ZF}
\label{sec:proof-assist-isabelle}

Let us briefly introduce Isabelle in the large landscape of proof
assistans (or ITPs for interactive theorem provers); we refer to the
excelent chapter by \citet{DBLP:series/hhl/HarrisonUW14} for a more
thorough recollection of the history of ITPs.

It is expected that an ITP assists the human user while mechanizing
some piece of mathematics; the interaction varies from system to
system, but a common interface consists on the system showing the
current goal and context. The user instructs the ITP to modify the
goal/context by means of tactics; a forward rule changes the context
with the intention of having in the new context a hypothesis closer to
the goal. A backward rule represents a deduction whose conclusion is
the goal, thus the new goal corresponds to the premises of the rule. A
proof is completed when the (current) goal is an instance of the
hypothesis (in the current context).

In that dialog, the user produces a script of tactics that can be
later reproduced step-by-step by the system (to check, for example,
that an imported theory is correct) or by the user to understand
the proof.\footnote{As we will see there are more declaratives
proof languages that aim to have intelligible proof scripts.}

To have any value at all, the system should be able to say if a tactic
makes sense (for instance, it must be forbidden to use the backward
tactic \textit{conjI} when the main connective is a
disjunction). Moreover one should be able to ascertain the validity of
the reaction of an ITP (be its rejection or its acceptance of some
alleged logical step). % Here I would merge the next paragraph.

Proof assistants provide diverse aids for the task
of formalizing a piece of mathematics. They are usually implemented
using a typed programming language; rigor is enforced by defining a
type of ``theorems,'' whose members can only be constructed using
operations stipulated in a small \emph{kernel} which encodes the
underlying foundation of the assistant. Further developments interact
with the type of theorems only through the kernel, and thus the latter
is the only ``trusted'' part of the assistant's code.

Several of the more established assistants (HOL Light, Coq, Isabelle)
are programmed in some variant of the ML language (which was
designed for this purpose); the newer Lean, on the other
hand, was originally conceived as a standalone functional programming
language with all the features of a standard assistant.

In the case of Isabelle, Standard ML is the first of the four layers
in which we work in this assistant. Both the kernel and automation
of proofs is coded in ML, sometimes as a substitute for induction on
formulas, as the next section explains.

\subsection{Isabelle metalogic $\Meta$}
\label{sec:isabelle-metalogic-meta}

The second layer of Isabelle is an
intuitionistic fragment of higher-order logic (or simple type theory)
called $\Meta$; its original version was described in \cite{Paulson1989},
and the addition of “sorts” appears in \cite{Nipkow-LF-91}.

The only predefined type is $\prop$ (“propositions”); new base types
can be postulated when defining objects logics. Types of higher order can be
assembled using the function space constructor $\fun$.

The type of propositions $\prop$ is equipped with a binary operation
$\implies$ (“meta-implication”) and a universal “meta-quantifier”
$\ALL$, that are used to represent the object
logic rules. As an example, the axiomatization of first-order logic
postulates a type $\tyo$ of booleans, and Modus Ponens
% https://isabelle.in.tum.de/dist/library/FOL/FOL/IFOL.html#IFOL.mp|axiom
is written as
\[
  \ALL P\,Q.\ \ [P\limp Q] \implies ([P] \implies [Q]).
\]
The square brackets (which are omitted in Isabelle theories) represent
an injection from $\tyo$ into $\prop$. % ($[P]$ can be read as “$P$
% holds”)
A consequence of this representation is that every formula of
the object logic appears atomic to $\Meta$.

%% Quantification is handled in the meta-level using a functional $\ALL$
%% with polymorphic type $(\alpha \fun \prop) \fun \prop$. 
Types in Isabelle are organized into \emph{classes} and \emph{sorts};
for ease of exposition, we will omit the former.  The axiomatization
of first-order logic postulates a sort $\{\type{term}\}$ (of
“individuals,” or elements of a first-order universe of discourse) and
stipulates that every further type variable $\alpha$ must be of that
sort. In particular, Isabelle/ZF only postulates one new type $\tyi$
(“sets”) of sort $\{\type{term}\}$. Hence, from the type of the universal
quantifier functional $\forall :: (\alpha \fun \tyo) \fun \tyo$, it
follows that it may only be applied to predicates with a variable of
type $\tyi$. This ensures that the object logic is effectively
first-order.

Paulson  \cite{Paulson1989} carried out a proof that the encoding
$\Meta_{\mathrm{IFOL}}$ of
intuitionistic first-order logic IFOL without equality  in the original $\Meta$ is
conservative (there is a correspondence between provable $\phi$ in
IFOL and provable $[\phi]$ in $\Meta_{\mathrm{IFOL}}$) by putting
$\Meta_{\mathrm{IFOL}}$ proofs in \emph{expanded normal form}
\cite{MR0387024}. Passing to classical logic does not present
difficulties, but the addition of meta-equality must be taken care of.
Even more so, since the treatment of equality differs between the
original and the present incarnation of $\Meta$; details for the
latter are exhaustively expounded in the recent formalization by
Nipkow and Roßkof \cite{10.1007/978-3-030-79876-5_6}.

The meta-logic $\Meta$ is rather weak; it has no induction/recursion
principles. Types are not inductively presented and, in particular, it
is not possible to prove by induction statements about
object-logic formulas (which are construed as terms of type $\tyi \fun
\dots \fun \tyi \fun \tyo$). Two ways to overcome this limitation are:
\begin{enumerate}
\item
  to construct the
  proof of each instance of the statement by hand or by programming on
  ML; or 
\item
  to encode formulas as sets and prove an internal version statement
  using induction of $\ZF$.
\end{enumerate}

For recursive definitions, only the second option is available, and
that is the way the Definition of Forcing is implemented in our
formalization.

\subsection{Isabelle/ZF}
\label{sec:isabellezf}

For the most part, the development of set theory in Isabelle is
carried out using its ZF object logic
\cite{DBLP:journals/jar/PaulsonG96}, which is the third logical layer
of the formalization and the most versatile one, since 
Isabelle's native automation is available at this level. Apart from
the type and sort
declarations detailed above, it features a finite axiomatization,
% https://isabelle.in.tum.de/dist/library/FOL/FOL/IFOL.html#ZF_Base.mem|const
with a predicate for membership, constants for the empty set and an
infinite set, and functions $\isatt{Pow}::\tyi\fun\tyi$,
$\union::\tyi\fun\tyi$, and $\isatt{PrimReplace} :: \tyi \fun (\tyi
\fun \tyi \fun\tyo)\fun \tyi$ (for Replacement). The Axiom of
Replacement
% https://isabelle.in.tum.de/dist/library/FOL/FOL/IFOL.html#ZF_Base.replacement|axiom
has a free predicate variable $P$: % $P ::\tyi\fun\tyi\fun\tyo$:
\begin{multline*}
  (\forall x \in A .\ \forall y\, z.\ P(x, y) \wedge P(x, z)
  \longrightarrow y=z) \implies \\
  b \in \isatt{PrimReplace}(A, P)
  \longleftrightarrow(\exists x \in A .\ P(x, b)) 
\end{multline*}
The restrictions on sorts described above ensure that it is not
possible that higher-order quantification gets used in $P$. The
statement of $\AC$ also has a higher-order free variable, but we only
use $\AC$ for demonstration purposes.

Isabelle/ZF reaches Hessenberg's $|A|\cdot|A| = |A|$. Our decision
(during 2017) to
use this assistant was triggered by its constructibility
library, \session{ZF-Constructible} \citep{paulson_2003},
% https://isabelle.in.tum.de/dist/library/ZF/ZF-Constructible/
which contains the development of $L$, the proof that it satisfies
$\AC$, and a version of the Reflection Principle. The latter was
actually encoded as a series of instructions to Isabelle automatic
proof tools that would prove each particular instance of reflection:
This is an example of what was said at the end of Section~\ref{sec:isabelle-metalogic-meta}.

The development of relativization and absoluteness for classes $C::
\isatt{i} \fun \isatt{o}$ follows the same pattern. Each particular
concept was manually written in a relational form and relativized.
Here, the contrast between the usual way one regards $\ZF$ as
first-order theory in the language $\{\in \}$ and the mathematical
practice of freely using defined concepts comes to the
forefront. Assistants have refined mechanisms to cope with defined
concepts (which also make their foundations a more complicated
than plain first-order logic), and this is the only that  nontrivial
mathematics can be formalized. But when studying relative interpretations, one
usually assumes a spartan syntax and defines relativization by
induction of formulas of the more succinct language. The approach
taken in \session{ZF-Constructible} is to consider relativizations of
the formulas (of type $\tyo$) that define each concept. For instance,
in the case of unions, we find a relativization
$\isatt{big{\uscore}union}:: (\isatt{i} \fun \isatt{o}) \fun \isatt{i}
\fun \isatt{i} \fun \isatt{o}$ of the statement
“$\union A = z$”:
\[
 \isatt{big{\uscore}union}(M,A,Z) \equiv \forall x[M].\ x \in z
 \longleftrightarrow (\exists y[M].\ y \in A \land x \in y)
\]
where $\forall x[M]\dots$ stands for $\forall x.\ M(x)\limp \dots$,
etc. The need to work with \emph{relational} presentations of defined
concepts stems from the fact that the model-theoretic definition of
$L$ requires working with set models and satisfaction, which is
defined for codes of formulas.

\subsection{Internalized formulas}
\label{sec:internalized-formulas}

\session{ZF-Constructible} defines the set $\formula$ of codes for
first-order formulas. These, alongside with lists, are instances of
Isabelle/ZF treatment of inductively defined (internal) datatypes; induction
and recursion theorems for them are proved automatically (this is in
constrast to general well-founded recursion, for which one has to work
with the fundamental recursor $\isatt{wfrec}$).

To avoid problems related to the binding of variables, de Bruijn
indices \cite{MR0321704} are used instead. The satisfaction predicate
$\isatt{sats}::\tyi\fun\tyi\fun\tyi\fun\tyo$ takes as arguments a set $M$, a list
$\mathit{env}\in\isatt{list}(M)$ for the assignment of free indices
(which are counted by the \isatt{arity} function),
and $\phi\in\formula$, and it is written
$M,\mathit{env}\models\phi$ in our formalization. This completes the
description of the fourth and last formal layer of the development.

Internalized formulas for most (but not all) of the relational
concepts can be obtained by guiding the automatic tactics. Actually,
great many of the concepts in \session{ZF-Constructible} where
internalized by hand; this is the case for union,
%% \begin{multline*}
%%   \isatt{big{\uscore}union{\uscore}fm}(A,z) \equiv \\
%%   ((\cdot\forall\cdot\cdot0 \in succ(z)\cdot \longleftrightarrow
%%   (\cdot\exists\cdot\cdot0 \in succ(succ(A))\cdot \land \cdot1 \in
%%   0\cdot\cdot\cdot)\cdot\cdot)
%% \end{multline*}
for which we have the following satisfaction lemma:
% https://isabelle.in.tum.de/dist/library/ZF/ZF-Constructible/L_axioms.html#L_axioms.sats_big_union_fm|fact
\begin{multline}\label{eq:sats_big_union_fm}
  x \in \omega \implies y \in \omega \implies \mathit{env} \in \isatt{list}(A)
  \implies \\
  A, \mathit{env} \models \isatt{big{\uscore}union{\uscore}fm}(x,y)%\cdot\union x \isatt{ is } y\cdot
  \longleftrightarrow \\
  \isatt{big{\uscore}union}(\isatt{\#\#} A, \isatt{nth}(x,
  \mathit{env}), \isatt{nth}(y, \mathit{env}))
\end{multline}
Note that $x$ and $y$ above are de Bruijn indices,
$\isatt{nth}(x,\mathit{env})$ is the $x$th element of $\mathit{env}$
and $\isatt{\#\#}A::\tyi\fun \tyo$ is the class corresponding to the
set $A::\tyi$.

%%% Local Variables:
%%% mode: latex
%%% TeX-master: "independence_ch_isabelle"
%%% ispell-local-dictionary: "american"
%%% End: 

 
\section{Relative versions of non-absolute concepts}
\label{sec:relat-vers-non-absol}

The treatment of relativization/internalization described in the
previous sections was enough for Paulson's treatment of
constructibility. This is the case because essentially all the
concepts in the way of proving the consistency of $\AC$ are
absolute, and the treatment of relational versions and relativized notions
could be minimized after proving the relevant absoluteness results:
For example, the lemma \isa{Union{\uscore}abs},
\[
  M(A) \implies M(z) \implies \isa{big{\uscore}union}(M, A, z) \longleftrightarrow z = \union
  A
\]
proved under the assumption that $M$ is transitive and nonempty.

Our first attempt to relativize cardinal arithmetic proceeded in the
same way
and we rapidly found out that stating and proving statements like $(||A||
= |A|) ^M$ in a completely relational language was extremely
cumbersome. This observation lead to the discovery of the discipline
expounded in the next subsection.

%% Working
%% in this relational 
%% way with powersets, cardinalities, and the like would be
%% unfeasible. As such, cardinal arithmetic was not put in relative form
%% in \session{ZF-Constructible}.

\subsection{Discipline and tools for relativization}
\label{sec:tools-relativization}
The missing step, that naturally appears in the literature, consists
of having relative \emph{functions} like $\Pow^M$, and the ability to
translate between the different presentations discussed so far.

To achieve this, we provide automatic tools to ease the definitions of
such relative versions, their fully relational counterparts, and the
internalized formulas. For instance, consider the
$\isa{cardinal}::\tyi \fun \tyi$ function defined in
\session{Isabelle/ZF}. Then the commands
\begin{isabelle}
  \isacommand{relativize}\isamarkupfalse%
  \ \isakeyword{functional}\ {\isachardoublequoteopen}cardinal{\isachardoublequoteclose}\ {\isachardoublequoteopen}cardinal{\isacharunderscore}{\kern0pt}rel{\isachardoublequoteclose}\ \isakeyword{external}\isanewline
  \isacommand{relationalize}\isamarkupfalse%
  \ {\isachardoublequoteopen}cardinal{\isacharunderscore}{\kern0pt}rel{\isachardoublequoteclose}\ {\isachardoublequoteopen}is{\isacharunderscore}{\kern0pt}cardinal{\isachardoublequoteclose}\isanewline
  \isacommand{synthesize}\isamarkupfalse%
  \ {\isachardoublequoteopen}is{\isacharunderscore}{\kern0pt}cardinal{\isachardoublequoteclose}\ \isakeyword{from{\isacharunderscore}{\kern0pt}definition}\ \isakeyword{assuming}\ {\isachardoublequoteopen}nonempty{\isachardoublequoteclose}%
\end{isabelle}
define the relative cardinal function
$\isa{cardinal{\uscore}rel}::(\tyi \fun \tyo) \fun \tyi \fun\tyi$
(denoted  $|\cdot|^M$, as expected),
the relational version $\isa{is{\uscore}cardinal}$ of the latter, the
internalized formula \isa{is{\uscore}cardinal{\uscore}fm} whose
satisfaction by a set is equivalent to the relational version, and
prove the previous statement (analogous to (\ref{eq:sats_big_union_fm})).
The proof that $\isa{is{\uscore}cardinal}(M,x,z)$  encodes the
statement $|x|^M = z$ must still be done by hand, since the definition
of $\isa{cardinal{\uscore}rel}$ already involves some tacit
absoluteness results (“\textit{the least $z \in \Ord$ such that $z
  \approx^M x$}” instead
of “\textit{the least $z \in \Ord^M$ such that $z
  \approx^M x$}”, and the like).

\subsection{Extension of Isabelle/ZF}
\label{sec:extension-isabellezf}
We extended \cite{Delta_System_Lemma-AFP} the material formalized in
Isabelle, from basic results involving function spaces and the
definition of cardinal exponentiation, to a treatment of cofinality
and the Delta System Lemma for $\omega_1$-families. We also included a
concise treatment of the axiom of Dependent Choices $\DC$ and the
general version of Rasiowa-Sikorski Lemma \cite{2018arXiv180705174G}
and a choiceless one for countable preorders.

This material was subsequently put in relative form in our formal
development on transitive class models \cite{Transitive_Models-AFP}
using as an aid the tools from
Section~\ref{sec:tools-relativization}. We also relativized many
original theories appearing in Isabelle/ZF, including the
fundamentals of cardinal arithmetic, the cumulative hierarchy, and the
definition of the $\ale{}$ function.


%%% Local Variables:
%%% mode: latex
%%% TeX-master: "independence_ch_isabelle"
%%% ispell-local-dictionary: "american"
%%% End: 


%%%%%%%%%%%%%%%%%%%%%%%%%%%%%%%%%%%%%%%%%%%%%%%%%%%%%%%%%%%%%%%%%%%%%%          
\section{Set models and forcing}
\label{sec:forcing}

\subsection{The $\ZFC$ axioms as locales}\label{sec:zfc-axioms-as-locales}
The description of set models of fragments of $\ZFC$ was performed
using Isabelle contexts (\emph{locales}) that fix a variable $M::\tyi$
and pack assumptions stating that $\lb M, \in\rb$ satisfy some
axioms. For example, the locale \isatt{M{\uscore}Z{\uscore}basic}
states that Zermelo set theory holds in $M$. The Union Axiom (\isatt{Union{\uscore}ax}), for
instance, is defined as follows:
\[
\forall x[\isatt{\#\#}M].\ \exists z[\isatt{\#\#}M].\ \isatt{big{\uscore}union}(\isatt{\#\#}M,x,z)
\]
%% % Same a above
%% \begin{isabelle}%
%% {\isasymforall}x{\isacharbrackleft}{\kern0pt}{\isacharhash}{\kern0pt}{\isacharhash}{\kern0pt}M{\isacharbrackright}{\kern0pt}{\isachardot}{\kern0pt}\ {\isasymexists}z{\isacharbrackleft}{\kern0pt}{\isacharhash}{\kern0pt}{\isacharhash}{\kern0pt}M{\isacharbrackright}{\kern0pt}{\isachardot}{\kern0pt}\ big{\isacharunderscore}{\kern0pt}union{\isacharparenleft}{\kern0pt}{\isacharhash}{\kern0pt}{\isacharhash}{\kern0pt}M{\isacharcomma}{\kern0pt}\ x{\isacharcomma}{\kern0pt}\ z{\isacharparenright}{\kern0pt}%
%% \end{isabelle}%
using relativized, relational versions of the axioms of Isabelle/ZF
since the interaction with \session{ZF-Constructible} was smoother
that way and, as it was mentioned in Section~\ref{sec:isabellezf},
this third layer of the formalization has more tools at our
disposal. Later, an equivalent statement “in the codes” (our fourth
layer) is obtained,
\[
  \isatt{Union{\uscore}ax}(\isatt{\#\#}M) \lsii A, [\,]\models \isatt{{\isasymcdot}Union\ Ax{\isasymcdot}}
\]
%% % Same a above
%% \begin{isabelle}
%% Union{\isacharunderscore}{\kern0pt}ax{\isacharparenleft}{\kern0pt}{\isacharhash}{\kern0pt}{\isacharhash}{\kern0pt}A{\isacharparenright}{\kern0pt}\ {\isasymlongleftrightarrow}\ A{\isacharcomma}{\kern0pt}\ {\isacharbrackleft}{\kern0pt}{\isacharbrackright}{\kern0pt}\ {\isasymTurnstile}\ {\isasymcdot}Union\ Ax{\isasymcdot}\isasep
%% \end{isabelle}
where \isatt{{\isasymcdot}Union\ Ax{\isasymcdot}} is the $\formula$ code for the
Union axiom. For the axiom schemes, \session{ZF-Constructible} defines
the expressions
\[
  \isatt{separation}(N,Q)
  \text{ and }
  \isatt{strong{\uscore}replacement}(N,R)
\]
that consume a class $N$ and predicates $Q::\tyi\fun\tyo$ and $R::\tyi\fun\tyi\fun\tyo$, and state that $N$
satisfies the $Q$-instance of Separation and the $R$-instance of
Replacement, respectively. In the statement of the Separation Axiom in
\isatt{M{\uscore}Z{\uscore}basic} and
the many replacement instances, predicates $Q$ and $R$ are actually
defined as the satisfaction of formulas of the appropriate arity; we can only show
that this form of the axioms hold in generic extensions. 

Further locales gather the assumption of transitivity of $M$ and
particular replacement instances expressed by means of the
\isatt{replacement{\uscore}assm(M,\isasymphi)} predicate ($@$ denotes
list concatenation):
\begin{multline}\label{eq:replacement_assm_def}
\varphi \in \formula  \limp \mathit{env} \in \isatt{list}(M) \limp \isatt{arity}(\varphi) \leq 2+ \isatt{length}(\mathit{env}) \limp \\
 \isatt{strong{\uscore}replacement}(\isatt{\#\#} M, \lambda x\, y.\ (M, [x,y]
\mathbin{@} \mathit{env}  \models \varphi))
\end{multline}
%% % Same as above
%% \begin{isabelle}
%% {\isasymphi}\;{\isasymin}\;formula\ {\isasymlongrightarrow}\ env\;{\isasymin}\;list{\isacharparenleft}{\kern0pt}M{\isacharparenright}{\kern0pt}\ {\isasymlongrightarrow} arity{\isacharparenleft}{\kern0pt}{\isasymphi}{\isacharparenright}{\kern0pt}\;{\isasymle}\;{\isadigit{2}}\ {\isacharplus}{\kern0pt}\isactrlsub {\isasymomega}\ length{\isacharparenleft}{\kern0pt}env{\isacharparenright}{\kern0pt}\ {\isasymlongrightarrow}\isanewline
%% \ \ \ \ strong{\isacharunderscore}{\kern0pt}replacement{\isacharparenleft}{\kern0pt}{\isacharhash}{\kern0pt}{\isacharhash}{\kern0pt}M{\isacharcomma}{\kern0pt}{\isasymlambda}x\ y{\isachardot}{\kern0pt}\ {\isacharparenleft}{\kern0pt}M\ {\isacharcomma}{\kern0pt}\ {\isacharbrackleft}{\kern0pt}x{\isacharcomma}{\kern0pt}y{\isacharbrackright}{\kern0pt}{\isacharat}{\kern0pt}env\ {\isasymTurnstile}\ {\isasymphi}{\isacharparenright}{\kern0pt}{\isacharparenright}
%% \end{isabelle}
%
%% Some of those instances, which we included for simplicity of design,
%% are actually “fake” instances in that they can be obtained from
%% Powerset, or from other replacement instances already available in the
%% respective context. An example of this is the sole instance of the
%% \isatt{M{\uscore}basic} locale, originally from
%% \session{ZF-Constructible}; we managed to eliminate it but in
%% doing so we had to reprove many lemmas from
%% that session in the weakened context. 
The locales gathering all 32 replacement instances necessary are named
\isatt{M{\uscore}ZF1} through \isatt{M{\uscore}ZF4},
\isatt{M{\uscore}ZF{\uscore}ground},
\isatt{M{\uscore}ZF{\uscore}ground{\uscore}notCH}, and
\isatt{M{\uscore}ZF{\uscore}ground{\uscore}CH} (see
Section~\ref{sec:repl-instances} for details).

%% Missing: Locales have to be interpreted (part of "big picture"?)

\subsection{The fundamental theorems}
Let $\lb \PP, {\preceq} ,\1\rb \in M$ be a forcing notion. In order to
fix the notation, the
$\val$ interpretation function takes $3$ arguments so that if $G\sbq \PP$, we have
$M[G]\defi \{ \val(\PP,G,\punto{a}) : \punto{a}\in M \}$.

The version of the Forcing Theorems that we formalized follows the
considerations on the $\forces^*$ relation as discussed in Kunen's new
\emph{Set Theory}
\cite[p.~257ff]{kunen2011set}. But in contrast to the point made on
p.~260 of this book, we internalized the recursion to define the forcing relation,
in that it involves codes for formulas, and the meta-theoretic formula
transformer $\phi\mapsto\mathit{Forces}_\phi$ is replaced by the
set-theoretic class function $\forceisa:: \tyi \fun \tyi$, which was defined by using
Isabelle/ZF facilities for primitive recursion.

Next we state this version of the fundamental theorems in a compact
way.
\begin{theorem}\label{th:forcing-thms}
  There exists a function  $\forceisa$
  such that for every
  $\phi\in\formula$ and $\punto{a}_0,\dots,\punto{a}_n\in M$,
  \begin{enumerate}
  \item\label{item:definability} (Definability)
    $\forceisa(\phi)\in\formula$, 
  \end{enumerate}
  where the 
  arity of $\forceisa(\phi)$ is at most $\isatt{arity}(\phi) + 4$; and if
  “$p \forces \phi\ [\punto{a}_0,\dots,\punto{a}_n]$”
  denotes
  “$M, [p,\PP,\preceq,\1, \punto{a}_0,\dots,\punto{a}_n]  \models
  \forceisa(\phi)$”, we have:
  \begin{enumerate}
    \setcounter{enumi}{1}
  \item\label{item:truth-lemma} (Truth Lemma) for every $M$-generic $G$,
    \[
      \exists p\in G.\ \; p \forces \phi\ [\punto{a}_0,\dots,\punto{a}_n]
    \]
    is equivalent to 
    \[
      M[G], [\val(\PP,G,\punto{a}_0),\dots,\val(\PP,G,\punto{a}_n)]
      \models \phi.
    \]
  \item \label{item:density-lemma} (Density Lemma) $p \forces \phi\ [\punto{a}_0,\dots,\punto{a}_n]$
    if and only if 
    $\{q\in \PP :  q \forces \phi\ [\punto{a}_0,\dots,\punto{a}_n]\}$
    is dense below $p$.
  \end{enumerate}
\end{theorem}
We actually have to define $\forceisa$ before we state the fundamental
theorems, so the main existential quantifier above does not appear in the
formalization.
Moreover, the items in Theorem~\ref{th:forcing-thms} appear as three separated lemmas in
\theory{Forcing{\uscore}Theorems} of our
\session{Independence\_CH} session \cite{Independence_CH-AFP},
and benefit from the $\isatt{map}$ function that applies a function to
each element of a list. For instance, the Truth Lemma is stated as
follows:
\begin{isabelle}
\isacommand{lemma}\isamarkupfalse%
\ truth{\isacharunderscore}{\kern0pt}lemma{\isacharcolon}{\kern0pt}\isanewline
\ \ \isakeyword{assumes}\isanewline
\ \ \ \ {\isachardoublequoteopen}{\isasymphi}{\isasymin}formula{\isachardoublequoteclose}\ {\isachardoublequoteopen}M{\isacharunderscore}{\kern0pt}generic{\isacharparenleft}{\kern0pt}G{\isacharparenright}{\kern0pt}{\isachardoublequoteclose}\isanewline
\ \ \ \ {\isachardoublequoteopen}env{\isasymin}list{\isacharparenleft}{\kern0pt}M{\isacharparenright}{\kern0pt}{\isachardoublequoteclose}\ {\isachardoublequoteopen}arity{\isacharparenleft}{\kern0pt}{\isasymphi}{\isacharparenright}{\kern0pt}{\isasymle}length{\isacharparenleft}{\kern0pt}env{\isacharparenright}{\kern0pt}{\isachardoublequoteclose}\isanewline
\ \ \isakeyword{shows}\isanewline
\ \ \ \ {\isachardoublequoteopen}{\isacharparenleft}{\kern0pt}{\isasymexists}p{\isasymin}G{\isachardot}{\kern0pt}\ p\ {\isasymtturnstile}\ {\isasymphi}\ env{\isacharparenright}{\kern0pt}\ \ \ {\isasymlongleftrightarrow}\ \ \ M{\isacharbrackleft}{\kern0pt}G{\isacharbrackright}{\kern0pt}{\isacharcomma}{\kern0pt}\ map{\isacharparenleft}{\kern0pt}val{\isacharparenleft}{\kern0pt}P{\isacharcomma}{\kern0pt}G{\isacharparenright}{\kern0pt}{\isacharcomma}{\kern0pt}env{\isacharparenright}{\kern0pt}\ {\isasymTurnstile}\ {\isasymphi}{\isachardoublequoteclose}
\end{isabelle}
where the $\forces$ notation (and its precedence) had already been set up in the
\theory{Forces{\uscore}Definition} theory as follows:
\begin{isabelle}
\isacommand{abbreviation}\isamarkupfalse%
\ Forces\ {\isacharcolon}{\kern0pt}{\isacharcolon}{\kern0pt}\ {\isachardoublequoteopen}{\isacharbrackleft}{\kern0pt}i{\isacharcomma}{\kern0pt}\ i{\isacharcomma}{\kern0pt}\ i{\isacharbrackright}{\kern0pt}\ {\isasymRightarrow}\ o{\isachardoublequoteclose}\ \ {\isacharparenleft}{\kern0pt}{\isachardoublequoteopen}{\isacharunderscore}{\kern0pt}\ {\isasymtturnstile}\ {\isacharunderscore}{\kern0pt}\ {\isacharunderscore}{\kern0pt}{\isachardoublequoteclose}\ {\isacharbrackleft}{\kern0pt}{\isadigit{3}}{\isadigit{6}}{\isacharcomma}{\kern0pt}{\isadigit{3}}{\isadigit{6}}{\isacharcomma}{\kern0pt}{\isadigit{3}}{\isadigit{6}}{\isacharbrackright}{\kern0pt}\ {\isadigit{6}}{\isadigit{0}}{\isacharparenright}{\kern0pt}\ \isakeyword{where}\isanewline
\ \ {\isachardoublequoteopen}p\ {\isasymtturnstile}\ {\isasymphi}\ env\ \ \ {\isasymequiv}\ \ \ M{\isacharcomma}{\kern0pt}\ {\isacharparenleft}{\kern0pt}{\isacharbrackleft}{\kern0pt}p{\isacharcomma}{\kern0pt}P{\isacharcomma}{\kern0pt}leq{\isacharcomma}{\kern0pt}{\isasymone}{\isacharbrackright}{\kern0pt}\ {\isacharat}{\kern0pt}\ env{\isacharparenright}{\kern0pt}\ {\isasymTurnstile}\ forces{\isacharparenleft}{\kern0pt}{\isasymphi}{\isacharparenright}{\kern0pt}{\isachardoublequoteclose}\isanewline
\end{isabelle}

Kunen first describes forcing for atomic formulas using a mutual
recursion
%% \begin{multline*}
%%   \forceseq (p,t_1,t_2) \defi 
%%   \forall s\in\dom(t_1)\cup\dom(t_2).\ \forall q\pleq p .\\
%%   \forcesmem(q,s,t_1)\lsii \forcesmem(q,s,t_2)
%% \end{multline*}
%% \begin{multline*}
%%   \forcesmem(p,t_1,t_2) \defi  \forall v\pleq p. \ \exists q\pleq v.\\  
%%   \exists s.\ \exists r\in \PP .\ \lb s,r\rb \in  t_2 \land q
%%   \pleq r \land \forceseq(q,t_1,s)
%% \end{multline*}
but then \cite[p.~257]{kunen2011set} it is cast as a single
recursively defined function $F$ over a wellfounded  relation $R$.
In our formalization, these are called $\frcat$ and 
$\isatt{frecR}$, respectively, and are defined on tuples $\lb \mathit{ft},t_1,t_2,p\rb$ (where
$\mathit{ft}\in\{0,1\}$ indicates whether the atomic formula being
forced is an equality or a membership, respectively).
Forcing for general formulas is then defined by recursion on the
datatype $\formula$ as indicated above. Technical details on the
implementation and proofs of the
Forcing Theorems have been spelled out in our
\cite{2020arXiv200109715G}.

%%% Local Variables: 
%%% mode: latex
%%% TeX-master: "independence_ch_isabelle"
%%% ispell-local-dictionary: "american"
%%% End: 


\section{A sample formal proof}
\label{sec:sample-formal-proof}

We present a fragment of the formal version of the proof that the
Powerset Axiom holds in a generic extension, which also serves to
illustrate the Isar dialect of Isabelle.

We quote the relevant
paragraph of Kunen's \cite[Thm.~IV.2.27]{kunen2011set}:
\begin{quote}
  For Power Set (similarly to Union above), it is sufficient to prove
  that whenever $a \in M[G]$, there is a $b \in M[G]$ such that
  $\mathcal{P}(a) \cap M[G] \subseteq b$. Fix $\tau \in
  M^{\mathbb{P}}$ such that $\tau_{G}=a$. Let
  $Q=(\mathcal{P}(\operatorname{dom}(\tau) \times
  \mathbb{P}))^{M}$. This is the set of all names $\vartheta \in
  M^{\mathbb{P}}$ such that $\operatorname{dom}(\vartheta) \subseteq
  \operatorname{dom}(\tau)$. Let $\pi=Q \times\{\1\}$ and let
  $b=\pi_{G}=$ $\left\{\vartheta_{G}: \vartheta \in Q\right\}$. Now,
  consider any $c \in \mathcal{P}(a) \cap M[G]$; we need to show that
  $c \in b$. Fix $\chi \in M^{\mathbb{P}}$ such that
  $\chi_{G}=c$, and let $\vartheta=\{\langle\sigma, p\rangle:
  \sigma \in \operatorname{dom}(\tau) \wedge p \Vdash \sigma \in
  \chi\}$; $\vartheta \in M$ by the Definability Lemma. Since
  $\vartheta \in Q$, we are done if we can show that
  $\vartheta_{G}=c$.
\end{quote}
The assumption $a\in M[G]$ appears in the lemma statement, and the
goal involving $b$ in the first sentence will appear below (signaled
by “{\small (**)}”); formalized
material necessarily tends to be much more linear than usual prose. In
what follows, we
will intersperse the relevant passages of the proof.
\begin{isabelle}
\isacommand{lemma}\isamarkupfalse%
\ Pow{\isacharunderscore}{\kern0pt}inter{\isacharunderscore}{\kern0pt}MG{\isacharcolon}{\kern0pt}\isanewline
\ \ \isakeyword{assumes}\ {\isachardoublequoteopen}a{\isasymin}M{\isacharbrackleft}{\kern0pt}G{\isacharbrackright}{\kern0pt}{\isachardoublequoteclose}\isanewline
\ \ \isakeyword{shows}\ {\isachardoublequoteopen}Pow{\isacharparenleft}{\kern0pt}a{\isacharparenright}{\kern0pt}\ {\isasyminter}\ M{\isacharbrackleft}{\kern0pt}G{\isacharbrackright}{\kern0pt}\ {\isasymin}\ M{\isacharbrackleft}{\kern0pt}G{\isacharbrackright}{\kern0pt}{\isachardoublequoteclose}\isanewline
%
\isacommand{proof}\isamarkupfalse%
\ {\isacharminus}{\kern0pt}
\end{isabelle}
\textit{Fix $\tau \in  M^{\mathbb{P}}$ such that $\tau_{G}=a$.}
\begin{isabelle}
\ \ \isacommand{from}\isamarkupfalse%
\ assms\isanewline
\ \ \isacommand{obtain}\isamarkupfalse%
\ {\isasymtau}\ \isakeyword{where}\ {\isachardoublequoteopen}{\isasymtau}\ {\isasymin}\ M{\isachardoublequoteclose}\ {\isachardoublequoteopen}val{\isacharparenleft}{\kern0pt}G{\isacharcomma}{\kern0pt}\ {\isasymtau}{\isacharparenright}{\kern0pt}\ {\isacharequal}{\kern0pt}\ a{\isachardoublequoteclose}\isanewline
\ \ \ \ \isacommand{using}\isamarkupfalse%
\ GenExtD\ \isacommand{by}\isamarkupfalse%
\ auto
\end{isabelle}
\textit{Let
  $Q=(\mathcal{P}(\operatorname{dom}(\tau) \times
  \mathbb{P}))^{M}$.}
\begin{isabelle}
\ \ \isacommand{let}\isamarkupfalse%
\ {\isacharquery}{\kern0pt}Q{\isacharequal}{\kern0pt}{\isachardoublequoteopen}Pow{\isacharparenleft}{\kern0pt}domain{\isacharparenleft}{\kern0pt}{\isasymtau}{\isacharparenright}{\kern0pt}{\isasymtimes}\isasymbbbP{\isacharparenright}{\kern0pt}\ {\isasyminter}\ M{\isachardoublequoteclose}
\end{isabelle}
\textit{This is the set of all names $\vartheta \in
  M^{\mathbb{P}}$} [\dots]---it is pretty laborious to show that things
are in $M$; we omit 17 lines of code to that effect,
that also apply previously proved lemmas.
\begin{isabelle}
\ \ \isacommand{from}\isamarkupfalse%
\ {\isacartoucheopen}{\isasymtau}{\isasymin}M{\isacartoucheclose}\isanewline
\ \ \isacommand{have}\isamarkupfalse%
\ {\isachardoublequoteopen}domain{\isacharparenleft}{\kern0pt}{\isasymtau}{\isacharparenright}{\kern0pt}{\isasymtimes}\isasymbbbP\ {\isasymin}\ M{\isachardoublequoteclose}\ {\isachardoublequoteopen}domain{\isacharparenleft}{\kern0pt}{\isasymtau}{\isacharparenright}{\kern0pt}\ {\isasymin}\ M{\isachardoublequoteclose}\isanewline
\ \ \ \ \isacommand{by}\isamarkupfalse%
\ simp{\isacharunderscore}{\kern0pt}all\isanewline
\ \ \isacommand{then}\isamarkupfalse%
\isanewline
\ \ \isacommand{have}\isamarkupfalse%
\ {\isachardoublequoteopen}{\isacharquery}{\kern0pt}Q\ {\isasymin}\ M{\isachardoublequoteclose}
\end{isabelle}
[\dots]\textit{ Let $\pi=Q \times\{\1\}$ and let
  $b=\pi_{G}=$ $\left\{\vartheta_{G}: \vartheta \in Q\right\}$.}
\begin{isabelle}
\ \ \isacommand{let}\isamarkupfalse%
\ {\isacharquery}{\kern0pt}{\isasympi}{\isacharequal}{\kern0pt}{\isachardoublequoteopen}{\isacharquery}{\kern0pt}Q{\isasymtimes}{\isacharbraceleft}{\kern0pt}{\isasymone}{\isacharbraceright}{\kern0pt}{\isachardoublequoteclose}\isanewline
\ \ \isacommand{let}\isamarkupfalse%
\ {\isacharquery}{\kern0pt}b{\isacharequal}{\kern0pt}{\isachardoublequoteopen}val{\isacharparenleft}{\kern0pt}G{\isacharcomma}{\kern0pt}{\isacharquery}{\kern0pt}{\isasympi}{\isacharparenright}{\kern0pt}{\isachardoublequoteclose}%% \isanewline
%% \ \ \isacommand{from}\isamarkupfalse%
%% \ {\isacartoucheopen}{\isacharquery}{\kern0pt}Q{\isasymin}M{\isacartoucheclose}\isanewline
%% \ \ \isacommand{have}\isamarkupfalse%
%% \ {\isachardoublequoteopen}{\isacharquery}{\kern0pt}{\isasympi}{\isasymin}M{\isachardoublequoteclose}\isanewline
%% \ \ \ \ \isacommand{using}\isamarkupfalse%
%% \ one{\isacharunderscore}{\kern0pt}in{\isacharunderscore}{\kern0pt}\isasymbbbP\ P{\isacharunderscore}{\kern0pt}in{\isacharunderscore}{\kern0pt}M\ transitivity\isanewline
%% \ \ \ \ \isacommand{by}\isamarkupfalse%
%% \ {\isacharparenleft}{\kern0pt}simp\ flip{\isacharcolon}{\kern0pt}\ setclass{\isacharunderscore}{\kern0pt}iff{\isacharparenright}{\kern0pt}\isanewline
%% \ \ \isacommand{then}\isamarkupfalse%
%% \isanewline
%% \ \ \isacommand{have}\isamarkupfalse%
%% \ {\isachardoublequoteopen}{\isacharquery}{\kern0pt}b\ {\isasymin}\ M{\isacharbrackleft}{\kern0pt}G{\isacharbrackright}{\kern0pt}{\isachardoublequoteclose}\isanewline
%% \ \ \ \ \isacommand{using}\isamarkupfalse%
%% \ GenExtI\ \isacommand{by}\isamarkupfalse%
%% \ simp
\end{isabelle}
\textit{Now,
  consider any $c \in \mathcal{P}(a) \cap M[G]$; we need to show that
  $c \in b$.}
\begin{isabelle}
  \label{goal-on-b}
\ \ \isacommand{have}\isamarkupfalse%
\ {\isachardoublequoteopen}Pow{\isacharparenleft}{\kern0pt}a{\isacharparenright}{\kern0pt}\ {\isasyminter}\ M{\isacharbrackleft}{\kern0pt}G{\isacharbrackright}{\kern0pt}\ {\isasymsubseteq}\ {\isacharquery}{\kern0pt}b{\isachardoublequoteclose}\hfill
\mbox{\rm\small(**)}\isanewline
\ \ \isacommand{proof}\isamarkupfalse%
\isanewline
\ \ \ \ \isacommand{fix}\isamarkupfalse%
\ c\isanewline
\ \ \ \ \isacommand{assume}\isamarkupfalse%
\ {\isachardoublequoteopen}c\ {\isasymin}\ Pow{\isacharparenleft}{\kern0pt}a{\isacharparenright}{\kern0pt}\ {\isasyminter}\ M{\isacharbrackleft}{\kern0pt}G{\isacharbrackright}{\kern0pt}{\isachardoublequoteclose}
\end{isabelle}
\textit{Fix $\chi \in M^{\mathbb{P}}$ such that
  $\chi_{G}=c$,}
\begin{isabelle}
\ \ \ \ \isacommand{then}\isamarkupfalse%
\isanewline
\ \ \ \ \isacommand{obtain}\isamarkupfalse%
\ {\isasymchi}\ \isakeyword{where}\ {\isachardoublequoteopen}c{\isasymin}M{\isacharbrackleft}{\kern0pt}G{\isacharbrackright}{\kern0pt}{\isachardoublequoteclose}\ {\isachardoublequoteopen}{\isasymchi}\ {\isasymin}\ M{\isachardoublequoteclose}\ {\isachardoublequoteopen}val{\isacharparenleft}{\kern0pt}G{\isacharcomma}{\kern0pt}{\isasymchi}{\isacharparenright}{\kern0pt}\ {\isacharequal}{\kern0pt}\ c{\isachardoublequoteclose}\isanewline
\ \ \ \ \ \ \isacommand{using}\isamarkupfalse%
\ GenExt{\isacharunderscore}{\kern0pt}iff\ \isacommand{by}\isamarkupfalse%
\ auto
\end{isabelle}
\textit{and let $\vartheta=\{\langle\sigma, p\rangle:
  \sigma \in \operatorname{dom}(\tau) \wedge p \Vdash \sigma \in
  \chi\}$;}
\begin{isabelle}
\ \ \ \ \isacommand{let}\isamarkupfalse%
\ {\isacharquery}{\kern0pt}{\isasymtheta}{\isacharequal}{\kern0pt}{\isachardoublequoteopen}{\isacharbraceleft}{\kern0pt}{\isasymlangle}{\isasymsigma}{\isacharcomma}{\kern0pt}p{\isasymrangle}\ {\isasymin}domain{\isacharparenleft}{\kern0pt}{\isasymtau}{\isacharparenright}{\kern0pt}{\isasymtimes}\isasymbbbP\ {\isachardot}{\kern0pt}\ p\ {\isasymtturnstile}\ {\isasymcdot}{\isadigit{0}}\ {\isasymin}\ {\isadigit{1}}{\isasymcdot}\ {\isacharbrackleft}{\kern0pt}{\isasymsigma}{\isacharcomma}{\kern0pt}{\isasymchi}{\isacharbrackright}{\kern0pt}\ {\isacharbraceright}{\kern0pt}{\isachardoublequoteclose}
\end{isabelle}
\textit{$\vartheta \in M$ by the Definability Lemma.}
\begin{isabelle}
\ \ \ \ \isacommand{have}\isamarkupfalse%
\ {\isachardoublequoteopen}arity{\isacharparenleft}{\kern0pt}forces{\isacharparenleft}{\kern0pt}Member{\isacharparenleft}{\kern0pt}{\isadigit{0}}{\isacharcomma}{\kern0pt}{\isadigit{1}}{\isacharparenright}{\kern0pt}{\isacharparenright}{\kern0pt}{\isacharparenright}{\kern0pt}\ {\isacharequal}{\kern0pt}\ {\isadigit{6}}{\isachardoublequoteclose}\isanewline
\ \ \ \ \ \ \isacommand{using}\isamarkupfalse%
\ arity{\isacharunderscore}{\kern0pt}forces{\isacharunderscore}{\kern0pt}at\ \isacommand{by}\isamarkupfalse%
\ auto\isanewline
\ \ \ \ \isacommand{with}\isamarkupfalse%
\ {\isacartoucheopen}domain{\isacharparenleft}{\kern0pt}{\isasymtau}{\isacharparenright}{\kern0pt}\ {\isasymin}\ M{\isacartoucheclose}\ {\isacartoucheopen}{\isasymchi}\ {\isasymin}\ M{\isacartoucheclose}\isanewline
\ \ \ \ \isacommand{have}\isamarkupfalse%
\ {\isachardoublequoteopen}{\isacharquery}{\kern0pt}{\isasymtheta}\ {\isasymin}\ M{\isachardoublequoteclose}\isanewline
\ \ \ \ \ \ \isacommand{using}\isamarkupfalse%
\ sats{\isacharunderscore}{\kern0pt}fst{\isacharunderscore}{\kern0pt}snd{\isacharunderscore}{\kern0pt}in{\isacharunderscore}{\kern0pt}M\isanewline
\ \ \ \ \ \ \isacommand{by}\isamarkupfalse%
\ simp
\end{isabelle}
\textit{Since
  $\vartheta \in Q$,}
\begin{isabelle}
\ \ \ \ \isacommand{then}\isamarkupfalse%
\isanewline
\ \ \ \ \isacommand{have}\isamarkupfalse%
\ {\isachardoublequoteopen}{\isacharquery}{\kern0pt}{\isasymtheta}\ {\isasymin}\ {\isacharquery}{\kern0pt}Q{\isachardoublequoteclose}\ \isacommand{by}\isamarkupfalse%
\ auto\isanewline
\ \ \ \ \isacommand{then}\isamarkupfalse%
\isanewline
\ \ \ \ \isacommand{have}\isamarkupfalse%
\ {\isachardoublequoteopen}val{\isacharparenleft}{\kern0pt}G{\isacharcomma}{\kern0pt}{\isacharquery}{\kern0pt}{\isasymtheta}{\isacharparenright}{\kern0pt}\ {\isasymin}\ {\isacharquery}{\kern0pt}b{\isachardoublequoteclose}\isanewline
\ \ \ \ \ \ \isacommand{using}\isamarkupfalse%
\ one{\isacharunderscore}{\kern0pt}in{\isacharunderscore}{\kern0pt}G\ generic\ val{\isacharunderscore}{\kern0pt}of{\isacharunderscore}{\kern0pt}elem\ {\isacharbrackleft}{\kern0pt}of\ {\isacharquery}{\kern0pt}{\isasymtheta}\ {\isasymone}\ {\isacharquery}{\kern0pt}{\isasympi}\ G{\isacharbrackright}{\kern0pt}\isanewline
\ \ \ \ \ \ \isacommand{by}\isamarkupfalse%
\ auto
\end{isabelle}
\textit{we are done if we can show that
  $\vartheta_{G}=c$.}
\begin{isabelle}
\ \ \ \ \isacommand{have}\isamarkupfalse%
\ {\isachardoublequoteopen}val{\isacharparenleft}{\kern0pt}G{\isacharcomma}{\kern0pt}{\isacharquery}{\kern0pt}{\isasymtheta}{\isacharparenright}{\kern0pt}\ {\isacharequal}{\kern0pt}\ c{\isachardoublequoteclose}\isanewline
%
\ \ \ \ \isacommand{proof} \ \mbox{$[\dots]$}
\end{isabelle}

Undoubtedly, even for this cherry-picked example, the formalization
looks a bit codish. It is therefore inevitable to compare this to the
magnificent results obtained by P.~Koepke and his team by using
Isabelle/Naproche \cite{10.1007/978-3-030-81097-9_2} (particularly,
the proof of König's Theorem). The trick there consists in presenting the result
being formalized as a restricted first-order problem, and then each
proof step can be handled by an automatic theorem prover.

%%% Local Variables: 
%%% mode: latex
%%% TeX-master: "independence_ch_isabelle"
%%% ispell-local-dictionary: "american"
%%% End: 


\section{Main achievements of the formalization}
\label{sec:main-achievements}

\subsection{A sufficient set of replacement instances}
\label{sec:repl-instances}

We isolated 22 instances of Replacement that are sufficient to force
$\CH$ or $\neg\CH$, which are enumerated below by the name of the
corresponding internalized first order formula. Many of these were already present in
relational form in the \session{ZF-Constructible} library.

The first 4 instances, collected in the subset
\isa{instances1{\uscore}fms} of \formula, consist of basic
constructions:

\begin{itemize}
\item 2 instances for transitive closure: one to prove closure under
  iteration of $X\mapsto\union X$ and an auxiliary one used to show absoluteness.
\item 1 instance to define $\in$-rank.
  %
\item 1 instance to construct the cumulative hierarchy (rank initial segments).
\end{itemize}

The next 4 instances (gathered in \isa{instances2{\uscore}fms})
are needed to set up
cardinal arithmetic in $M$:
\begin{itemize}
\item 2 instances for the definition of
  ordertypes: The relevant well-founded recursion and a technical
  auxiliary instance.
\item 2 instances for Aleph: Replacement through the ordertype function (for Hartogs' Theorem) and the well-founded recursion
  using it.
\end{itemize}

We also need a one extra replacement instance $\psi$ on $M$ for each
$\phi$ of the
previous ones to have them in $M[G]$:
\[
  \psi(x,\alpha,y_1,\dots,y_n) \defi \quine{\alpha = \min \bigl\{
    \beta \mid \exists\tau\in V_\beta.\  \mathit{snd}(x) \forces
    \phi\ [\mathit{fst}(x),\tau,y_1,\dots,y_n]\bigr\}}
\]
Here, $\mathit{fst}(\lb a,b\rb) = a$ and $\mathit{snd}(\lb a,b\rb) = b$.
% (with default value $0$ for non pairs).
The map $\phi\mapsto\psi$ is
the function $\calF$ referred to in the abstract.
All such “ground” replacement
instances appear in the locale \locale{M{\uscore}ZF3} and form the set
\isa{instances3{\uscore}fms}.

That makes 16 instances up to now. For the setup of forcing, we
require the following 3 instances, which form the set
\isa{instances{\uscore}ground{\uscore}fms}:
%
\begin{itemize}
\item Well-founded recursion to define check-names.
  %
\item Well-founded recursion for the definition of forcing for atomic formulas.
  %
\item Replacement through $x\mapsto \lb x,\check{x}\rb$ (for the
  definition of $\punto{G}$).
  %
\end{itemize}
The proof of the $\Delta$-System Lemma requires 2 instances which form the set
\isa{instances{\uscore}ground{\uscore}notCH{\uscore}fms}, that are
used for the recursive construction of sets using a choice function (as in the
construction of a wellorder of $X$ given a choice function on
$\Pow(X)$), and to show its absoluteness.

The $21$ formulas up to this point are collected into the set
\isa{overhead{\uscore}notCH} (called $\Omega$ in the abstract), which is enough to
force $\neg\CH$. To force $\CH$, we required one further instance for
the absoluteness of the recursive construction in the proof of
Dependent Choices from $\AC$. A listing with the names of all the formulas
can be found in Appendix~\ref{sec:repl-instances-appendix}.
  
The particular choice of some of the instances above arose from
Paulson's architecture on which we based our development.
This applies every time
a locale from \session{ZF-Constructible} has to be
interpreted (\locale{M{\uscore}eclose} and
\locale{M{\uscore}ordertype}, respectively, for the “auxiliary” instances).
%% For instance, the first
%% instance required for the definition of relative ordertypes arises
%% from Paulson's \session{ZF-Constructible}.
% https://isabelle.in.tum.de/dist/library/ZF/ZF-Constructible/Rank.html#offset_1123..1139

On the other hand, we replaced the original proof of the
Schröder-Bernstein Theorem by Zermelo's one
\cite[Exr. x4.27]{moschovakis1994notes}, because the former required
at least one extra instance
% (\isa{banach{\uscore}iterates{\uscore}fm})
arising from an iteration. We also managed to avoid 12 further
replacements by restructuring some of original theories in
\session{ZF-Constructible}, so these modifications are included as
part of our project.

It is to be noted that the proofs of the Forcing Theorems do not
require any extra replacement; actually, they only need the 7
instances appearing in \isa{instances1{\uscore}fms} and
\isa{instances{\uscore}ground{\uscore}fms}.  But this seems not be
the case for Separation, at least by inspecting our formalization:
More instances holding in $M$ are needed 
as the complexity of $\phi$ grows. One point where this is apparent is
in the proof of Theorem~\ref{th:forcing-thms}(\ref{item:truth-lemma}),
that appears as the \isa{truth{\uscore}lemma} in our development; it
depends on \isa{truth{\uscore}lemma'} and
\isa{truth{\uscore}lemma{\uscore}Neg}, which explicitly invoke
\isa{separation{\uscore}ax}. In any case, our intended grounds
(v.g., the transitive collapse of countable elementary submodels of a
rank initial segment $V_\alpha$ or an $H(\kappa)$) all satisfy full
Separation.


%-%-%-%-%-%-%-%-%-%-%-%-%-%-%-%-%-%-%-%-%-%-%-%-%-%-%-%-%-%-%-%-%
\subsection{Models for $\CH$ and its negation}
\label{sec:models-ch-negation}

The statements of the existence of models of $\ZFC + \neg\CH$ and of
$\ZFC + \CH$  appear in our formalization as follows:

\begin{isabelle}
\isacommand{corollary}\isamarkupfalse%
\ ctm{\isacharunderscore}{\kern0pt}ZFC{\isacharunderscore}{\kern0pt}imp{\isacharunderscore}{\kern0pt}ctm{\isacharunderscore}{\kern0pt}not{\isacharunderscore}{\kern0pt}CH{\isacharcolon}{\kern0pt}\isanewline
\ \ \isakeyword{assumes}\isanewline
\ \ \ \ {\isachardoublequoteopen}M\ {\isasymapprox}\ {\isasymomega}{\isachardoublequoteclose}\ {\isachardoublequoteopen}Transset{\isacharparenleft}{\kern0pt}M{\isacharparenright}{\kern0pt}{\isachardoublequoteclose}\ {\isachardoublequoteopen}M\ {\isasymTurnstile}\ ZFC{\isachardoublequoteclose}\isanewline
\ \ \isakeyword{shows}\isanewline
\ \ \ \ {\isachardoublequoteopen}{\isasymexists}N{\isachardot}{\kern0pt}\isanewline
\ \ \ \ \ \ M\ {\isasymsubseteq}\ N\ {\isasymand}\ N\ {\isasymapprox}\ {\isasymomega}\ {\isasymand}\ Transset{\isacharparenleft}{\kern0pt}N{\isacharparenright}{\kern0pt}\ {\isasymand}\ N\ {\isasymTurnstile}\ ZFC\ {\isasymunion}\ {\isacharbraceleft}{\kern0pt}{\isasymcdot}{\isasymnot}{\isasymcdot}CH{\isasymcdot}{\isasymcdot}{\isacharbraceright}{\kern0pt}\ {\isasymand}\isanewline
\ \ \ \ \ \ {\isacharparenleft}{\kern0pt}{\isasymforall}{\isasymalpha}{\isachardot}{\kern0pt}\ Ord{\isacharparenleft}{\kern0pt}{\isasymalpha}{\isacharparenright}{\kern0pt}\ {\isasymlongrightarrow}\ {\isacharparenleft}{\kern0pt}{\isasymalpha}\ {\isasymin}\ M\ {\isasymlongleftrightarrow}\ {\isasymalpha}\ {\isasymin}\ N{\isacharparenright}{\kern0pt}{\isacharparenright}{\kern0pt}{\isachardoublequoteclose}
\end{isabelle}

\begin{isabelle}
\isacommand{corollary}\isamarkupfalse%
\ ctm{\isacharunderscore}{\kern0pt}ZFC{\isacharunderscore}{\kern0pt}imp{\isacharunderscore}{\kern0pt}ctm{\isacharunderscore}{\kern0pt}CH{\isacharcolon}{\kern0pt}\isanewline
\ \ \isakeyword{assumes}\isanewline
\ \ \ \ {\isachardoublequoteopen}M\ {\isasymapprox}\ {\isasymomega}{\isachardoublequoteclose}\ {\isachardoublequoteopen}Transset{\isacharparenleft}{\kern0pt}M{\isacharparenright}{\kern0pt}{\isachardoublequoteclose}\ {\isachardoublequoteopen}M\ {\isasymTurnstile}\ ZFC{\isachardoublequoteclose}\isanewline
\ \ \isakeyword{shows}\isanewline
\ \ \ \ {\isachardoublequoteopen}{\isasymexists}N{\isachardot}{\kern0pt}\isanewline
\ \ \ \ \ \ M\ {\isasymsubseteq}\ N\ {\isasymand}\ N\ {\isasymapprox}\ {\isasymomega}\ {\isasymand}\ Transset{\isacharparenleft}{\kern0pt}N{\isacharparenright}{\kern0pt}\ {\isasymand}\ N\ {\isasymTurnstile}\ ZFC\ {\isasymunion}\ {\isacharbraceleft}{\kern0pt}{\isasymcdot}CH{\isasymcdot}{\isacharbraceright}{\kern0pt}\ {\isasymand}\isanewline
\ \ \ \ \ \ {\isacharparenleft}{\kern0pt}{\isasymforall}{\isasymalpha}{\isachardot}{\kern0pt}\ Ord{\isacharparenleft}{\kern0pt}{\isasymalpha}{\isacharparenright}{\kern0pt}\ {\isasymlongrightarrow}\ {\isacharparenleft}{\kern0pt}{\isasymalpha}\ {\isasymin}\ M\ {\isasymlongleftrightarrow}\ {\isasymalpha}\ {\isasymin}\ N{\isacharparenright}{\kern0pt}{\isacharparenright}{\kern0pt}{\isachardoublequoteclose}
\end{isabelle}
where $\approx$ is equipotency, and the predicate \isa{Transset}
holds for
transitive sets. Both results are proved without using Choice.

As the excerpts indicate, these results are obtained as corollaries of
two theorems in which only a subset of the aforementioned
replacement instances are assumed of the ground model. We begin the
discussion of these stronger results by
considering extensions of ctms of fragments of $\ZF$.
\begin{isabelle}
\isacommand{theorem}\isamarkupfalse%
\ extensions{\isacharunderscore}{\kern0pt}of{\isacharunderscore}{\kern0pt}ctms{\isacharcolon}{\kern0pt}\isanewline
\ \ \isakeyword{assumes}\isanewline
\ \ \ \ {\isachardoublequoteopen}M\ {\isasymapprox}\ {\isasymomega}{\isachardoublequoteclose}\ {\isachardoublequoteopen}Transset{\isacharparenleft}{\kern0pt}M{\isacharparenright}{\kern0pt}{\isachardoublequoteclose}\isanewline
\ \ \ \ {\isachardoublequoteopen}M\ {\isasymTurnstile}\ {\isasymcdot}Z{\isasymcdot}\ {\isasymunion}\ {\isacharbraceleft}{\kern0pt}{\isasymcdot}Replacement{\isacharparenleft}{\kern0pt}p{\isacharparenright}{\kern0pt}{\isasymcdot}\ {\isachardot}{\kern0pt}\ p\ {\isasymin}\ overhead{\isacharbraceright}{\kern0pt}{\isachardoublequoteclose}\isanewline
\ \ \ \ {\isachardoublequoteopen}{\isasymPhi}\ {\isasymsubseteq}\ formula{\isachardoublequoteclose}\isanewline%
\ \ \ \ {\isachardoublequoteopen}M\ {\isasymTurnstile}\ {\isacharbraceleft}{\kern0pt}\ {\isasymcdot}Replacement{\isacharparenleft}{\kern0pt}ground{\isacharunderscore}{\kern0pt}repl{\isacharunderscore}{\kern0pt}fm{\isacharparenleft}{\kern0pt}{\isasymphi}{\isacharparenright}{\kern0pt}{\isacharparenright}{\kern0pt}{\isasymcdot}\ {\isachardot}{\kern0pt}\ {\isasymphi}\ {\isasymin}\ {\isasymPhi}{\isacharbraceright}{\kern0pt}{\isachardoublequoteclose}\isanewline
\ \ \isakeyword{shows}\isanewline
\ \ \ \ {\isachardoublequoteopen}{\isasymexists}N{\isachardot}{\kern0pt}\isanewline
\ \ \ \ \ \ M\ {\isasymsubseteq}\ N\ {\isasymand}\ N\ {\isasymapprox}\ {\isasymomega}\ {\isasymand}\ Transset{\isacharparenleft}{\kern0pt}N{\isacharparenright}{\kern0pt}\ {\isasymand}\ M{\isasymnoteq}N\ {\isasymand}\isanewline
\ \ \ \ \ \ {\isacharparenleft}{\kern0pt}{\isasymforall}{\isasymalpha}{\isachardot}{\kern0pt}\ Ord{\isacharparenleft}{\kern0pt}{\isasymalpha}{\isacharparenright}{\kern0pt}\ {\isasymlongrightarrow}\ {\isacharparenleft}{\kern0pt}{\isasymalpha}\ {\isasymin}\ M\ {\isasymlongleftrightarrow}\ {\isasymalpha}\ {\isasymin}\ N{\isacharparenright}{\kern0pt}{\isacharparenright}{\kern0pt}\ {\isasymand}\isanewline
\ \ \ \ \ \ {\isacharparenleft}{\kern0pt}{\isacharparenleft}{\kern0pt}M{\isacharcomma}{\kern0pt}\ {\isacharbrackleft}{\kern0pt}{\isacharbrackright}{\kern0pt}{\isasymTurnstile}\ {\isasymcdot}AC{\isasymcdot}{\isacharparenright}{\kern0pt}\ {\isasymlongrightarrow}\ N{\isacharcomma}{\kern0pt}\ {\isacharbrackleft}{\kern0pt}{\isacharbrackright}{\kern0pt}\ {\isasymTurnstile}\ {\isasymcdot}AC{\isasymcdot}{\isacharparenright}{\kern0pt}\ {\isasymand}\isanewline
\ \ \ \ \ \ N\ {\isasymTurnstile}\ {\isasymcdot}Z{\isasymcdot}\ {\isasymunion}\ {\isacharbraceleft}{\kern0pt}\ {\isasymcdot}Replacement{\isacharparenleft}{\kern0pt}{\isasymphi}{\isacharparenright}{\kern0pt}{\isasymcdot}\ {\isachardot}{\kern0pt}\ {\isasymphi}\ {\isasymin}\ {\isasymPhi}{\isacharbraceright}{\kern0pt}{\isachardoublequoteclose}
\end{isabelle}

Here, the 7-element set \isa{overhead} is enough to construct a proper
extension. It is  the union of
\isa{instances{\isadigit{1}}{\isacharunderscore}{\kern0pt}fms} and
\isa{instances{\isacharunderscore}{\kern0pt}ground{\isacharunderscore}{\kern0pt}fms}.
Also,
\isa{{\isasymcdot}Z{\isasymcdot}} denotes Zermelo set theory and one
can use the parameter $\Phi$ to ensure those replacement instances in the extension.

In the
next theorem, the relevant set of formulas is
\isa{overhead{\isacharunderscore}{\kern0pt}notCH}, defined above in
Section~\ref{sec:repl-instances}, and \isa{ZC} denotes Zermelo set
theory plus Choice:

\begin{isabelle}
\isacommand{theorem}\isamarkupfalse%
\ ctm{\isacharunderscore}{\kern0pt}of{\isacharunderscore}{\kern0pt}not{\isacharunderscore}{\kern0pt}CH{\isacharcolon}{\kern0pt}\isanewline
\ \ \isakeyword{assumes}\isanewline
\ \ \ \ {\isachardoublequoteopen}M\ {\isasymapprox}\ {\isasymomega}{\isachardoublequoteclose}\ {\isachardoublequoteopen}Transset{\isacharparenleft}{\kern0pt}M{\isacharparenright}{\kern0pt}{\isachardoublequoteclose}\isanewline
\ \ \ \ {\isachardoublequoteopen}M\ {\isasymTurnstile}\ ZC\ {\isasymunion}\ {\isacharbraceleft}{\kern0pt}{\isasymcdot}Replacement{\isacharparenleft}{\kern0pt}p{\isacharparenright}{\kern0pt}{\isasymcdot}\ {\isachardot}{\kern0pt}\ p\ {\isasymin}\ overhead{\isacharunderscore}{\kern0pt}notCH{\isacharbraceright}{\kern0pt}{\isachardoublequoteclose}\isanewline
\ \ \ \ {\isachardoublequoteopen}{\isasymPhi}\ {\isasymsubseteq}\ formula{\isachardoublequoteclose}\isanewline
\ \ \ \ {\isachardoublequoteopen}M\ {\isasymTurnstile}\ {\isacharbraceleft}{\kern0pt}\ {\isasymcdot}Replacement{\isacharparenleft}{\kern0pt}ground{\isacharunderscore}{\kern0pt}repl{\isacharunderscore}{\kern0pt}fm{\isacharparenleft}{\kern0pt}{\isasymphi}{\isacharparenright}{\kern0pt}{\isacharparenright}{\kern0pt}{\isasymcdot}\ {\isachardot}{\kern0pt}\ {\isasymphi}\ {\isasymin}\ {\isasymPhi}{\isacharbraceright}{\kern0pt}{\isachardoublequoteclose}\isanewline
\ \ \isakeyword{shows}\isanewline
\ \ \ \ {\isachardoublequoteopen}{\isasymexists}N{\isachardot}{\kern0pt}\isanewline
\ \ \ \ \ \ M\ {\isasymsubseteq}\ N\ {\isasymand}\ N\ {\isasymapprox}\ {\isasymomega}\ {\isasymand}\ Transset{\isacharparenleft}{\kern0pt}N{\isacharparenright}{\kern0pt}\ {\isasymand}\isanewline
\ \ \ \ \ \ N\ {\isasymTurnstile}\ ZC\ {\isasymunion}\ {\isacharbraceleft}{\kern0pt}{\isasymcdot}{\isasymnot}{\isasymcdot}CH{\isasymcdot}{\isasymcdot}{\isacharbraceright}{\kern0pt}\ {\isasymunion}\ {\isacharbraceleft}{\kern0pt}\ {\isasymcdot}Replacement{\isacharparenleft}{\kern0pt}{\isasymphi}{\isacharparenright}{\kern0pt}{\isasymcdot}\ {\isachardot}{\kern0pt}\ {\isasymphi}\ {\isasymin}\ {\isasymPhi}{\isacharbraceright}{\kern0pt}\ {\isasymand}\isanewline
\ \ \ \ \ \ {\isacharparenleft}{\kern0pt}{\isasymforall}{\isasymalpha}{\isachardot}{\kern0pt}\ Ord{\isacharparenleft}{\kern0pt}{\isasymalpha}{\isacharparenright}{\kern0pt}\ {\isasymlongrightarrow}\ {\isacharparenleft}{\kern0pt}{\isasymalpha}\ {\isasymin}\ M\ {\isasymlongleftrightarrow}\ {\isasymalpha}\ {\isasymin}\ N{\isacharparenright}{\kern0pt}{\isacharparenright}{\kern0pt}{\isachardoublequoteclose}
\end{isabelle}

Finally, \isa{overhead{\isacharunderscore}{\kern0pt}CH} is the union
of \isa{overhead{\isacharunderscore}{\kern0pt}notCH} with the $\DC$
instance \isa{dc{\uscore}abs{\uscore}fm}:
\begin{isabelle}
\isacommand{theorem}\isamarkupfalse%
\ ctm{\isacharunderscore}{\kern0pt}of{\isacharunderscore}{\kern0pt}CH{\isacharcolon}{\kern0pt}\isanewline
\ \ \isakeyword{assumes}\isanewline
\ \ \ \ {\isachardoublequoteopen}M\ {\isasymapprox}\ {\isasymomega}{\isachardoublequoteclose}\ {\isachardoublequoteopen}Transset{\isacharparenleft}{\kern0pt}M{\isacharparenright}{\kern0pt}{\isachardoublequoteclose}\isanewline
\ \ \ \ {\isachardoublequoteopen}M\ {\isasymTurnstile}\ ZC\ {\isasymunion}\ {\isacharbraceleft}{\kern0pt}{\isasymcdot}Replacement{\isacharparenleft}{\kern0pt}p{\isacharparenright}{\kern0pt}{\isasymcdot}\ {\isachardot}{\kern0pt}\ p\ {\isasymin}\ overhead{\isacharunderscore}{\kern0pt}CH{\isacharbraceright}{\kern0pt}{\isachardoublequoteclose}\isanewline
\ \ \ \ {\isachardoublequoteopen}{\isasymPhi}\ {\isasymsubseteq}\ formula{\isachardoublequoteclose}\isanewline
\ \ \ \ {\isachardoublequoteopen}M\ {\isasymTurnstile}\ {\isacharbraceleft}{\kern0pt}\ {\isasymcdot}Replacement{\isacharparenleft}{\kern0pt}ground{\isacharunderscore}{\kern0pt}repl{\isacharunderscore}{\kern0pt}fm{\isacharparenleft}{\kern0pt}{\isasymphi}{\isacharparenright}{\kern0pt}{\isacharparenright}{\kern0pt}{\isasymcdot}\ {\isachardot}{\kern0pt}\ {\isasymphi}\ {\isasymin}\ {\isasymPhi}{\isacharbraceright}{\kern0pt}{\isachardoublequoteclose}\isanewline
\ \ \isakeyword{shows}\isanewline
\ \ \ \ {\isachardoublequoteopen}{\isasymexists}N{\isachardot}{\kern0pt}\isanewline
\ \ \ \ \ \ M\ {\isasymsubseteq}\ N\ {\isasymand}\ N\ {\isasymapprox}\ {\isasymomega}\ {\isasymand}\ Transset{\isacharparenleft}{\kern0pt}N{\isacharparenright}{\kern0pt}\ {\isasymand}\isanewline
\ \ \ \ \ \ N\ {\isasymTurnstile}\ ZC\ {\isasymunion}\ {\isacharbraceleft}{\kern0pt}{\isasymcdot}CH{\isasymcdot}{\isacharbraceright}{\kern0pt}\ {\isasymunion}\ {\isacharbraceleft}{\kern0pt}\ {\isasymcdot}Replacement{\isacharparenleft}{\kern0pt}{\isasymphi}{\isacharparenright}{\kern0pt}{\isasymcdot}\ {\isachardot}{\kern0pt}\ {\isasymphi}\ {\isasymin}\ {\isasymPhi}{\isacharbraceright}{\kern0pt}\ {\isasymand}\isanewline
\ \ \ \ \ \ {\isacharparenleft}{\kern0pt}{\isasymforall}{\isasymalpha}{\isachardot}{\kern0pt}\ Ord{\isacharparenleft}{\kern0pt}{\isasymalpha}{\isacharparenright}{\kern0pt}\ {\isasymlongrightarrow}\ {\isacharparenleft}{\kern0pt}{\isasymalpha}\ {\isasymin}\ M\ {\isasymlongleftrightarrow}\ {\isasymalpha}\ {\isasymin}\ N{\isacharparenright}{\kern0pt}{\isacharparenright}{\kern0pt}{\isachardoublequoteclose}
\end{isabelle}

%%% Local Variables: 
%%% mode: latex
%%% TeX-master: "independence_ch_isabelle"
%%% ispell-local-dictionary: "american"
%%% End: 


\section{Related work}
\label{sec:related-work}

\textbf{Reviewer's comments}
{\it
  \begin{itemize}
  \item There, it would be appropriate to contrast what was done in
    Paulson's work on constructibility with the current work on forcing.
  \item More to the point, the recent work by Han and van Doorn on
    forcing in Lean deserves more discussion.  They have gone further
    than the current authors, having proved the independence of the
    continuum hypothesis.  They prefer Boolean-valued models as being
    more direct in use than the authors' countable transitive models.
    \begin{itemize}
    \item Readers will want to know whether the type-theoretic approach
      is better/worse/just different than using Isabelle/ZF, and
    \item are there any benefits to the ctm approach?
    \item Is the type-theory encoding of ZF really accurate?
    \item How about comparing proofs of equivalent statements in the two
      approaches for length and readability?
    \end{itemize}
  \end{itemize}
}

To the best of our knowledge, all of the previous works in
formalization of the method 
of forcing have been done in different variants of type theory, and
none of them uses the ctm approach. The
most important is the recent one by 
Han and van Doorn
\cite{han_et_al:LIPIcs:2019:11074,DBLP:conf/cpp/HanD20}, which includes
a formalization of the independence of $\CH$ by means
the Boolean-valued approach to forcing, using the Lean
proof assistant \cite{DBLP:conf/cade/MouraKADR15}.


\begin{itemize}
\item The consistency strength of Lean requires infinitely
  many inaccessibles. More precisely, let Lean$_n$ be the theory of
  CiC foundations of Lean restricted to $n$ type universes.  Carneiro
  \cite{carneiro-ms-thesis}, proved the consistency of Lean$_n$ from
  $\ZFC$ plus the existence of $n$ inaccessible
  cardinals. It is also reported in \cite{carneiro-ms-thesis} that
  Werner's results in \cite{10.5555/645869.668660} can be adapted to
  show that Lean$_{n+2}$ proves the consistency of the latter theory. 

  On the other hand, Isabelle's \emph{Pure} is based on
  ``intuitionistic higher order logic.'' In Paulson
  \cite{Paulson1989} it is proved that \emph{Pure} is sound for
  intuitionistic first order logic, thus it does not add any strength
  to it. On top of this, the axiomatization of Isabelle/ZF results in
  a system equiconsistent with $\ZFC$. Our running assumption, that of
  the existence of a countable transitive model, is considerably
  weaker (directly and consistency-wise) than the existence of a
  single inaccesible cardinal. In that sense, directly obtain
  unprovability results in first order logic, the meta theoretic
  machinery used to obtain them is far heavier than the one we use to
  operate model-theoretically.
  %
\item We may discuss in finer detail the shape of the axioms of
  Isabelle/ZF. It is perhaps more correct to say it is an
  notational variant of NBG set theory, because the schemes of
  Replacement and Separation feature higher order (free) variables
  playing the role of formula variables. It can't be proved that the
  axioms thus written correspond to first order sentences. This is the
  reason that our relativized versions only apply to set models, where
  we can restrict the formula variables to predicates that actually
  come from first order variables. In that sense, the axioms of the
  locale \isatt{M{\isacharunderscore}ZF} correspond faithfully to the
  $ZF$ axioms.
\item \fbox{\bf take care of repetitions} In our opinion, one of the
  main benefits of using transitive models is that many fundamental
  notions are absolute and thus the many statements can be interpreted
  transparently. It also provides a very concrete way to understand
  generic objects: as sets that (in the non trivial case) are provably
  not in the original model; this dispells any mystical feel around
  this concept (contrary to the case when the ground model is the
  universe of all sets). In addition, two-valued semantics is
  closer to our intuition ($\leftarrow$ revise).
\end{itemize}
%%% Local Variables: 
%%% mode: latex
%%% TeX-master: "forcing_in_isabelle_zf"
%%% ispell-local-dictionary: "american"
%%% End: 


\section{Some lessons}\label{sec:lessons}

We want to finish this report by gathering some of the conclusions we
reached after the experience of formalizing the basics of forcing in a
proof assistant.

\subsection{Aims of a formalization and planning}

We believe that in every project of formalization of mathematics,
there is a tension between the haste to verify the target results and
the need to obtain a readable, albeit extremely detailed, corpus of
statements and proofs. This tension is mirrored in two differents
purposes of formalization: Developing new mathematics from scratch and
producing verified results en route to this, versus verifying and
documenting material that has already been produced on paper.

Our present project clearly belongs to the second category, so we
prioritized trying to obtain formal proofs that mimicked standard
prose (the highlight being the sample proof in
Section~\ref{sec:sample-formal-proof}). We feel that the Isar language
provided with Isabelle has the right balance between elegance and
efficacy. Another crucial aspect to achieve this goal is the level of
detail of the blueprint for the formalization. We must however confess
that we learned many of the subtleties of Isabelle in the making, and
many engineering decisions were also taken before it was clear the
precise way things would develop in the future.

A similar experience, but on an opposite side of the formalization
spectrum happened to the Liquid Tensor Project as described by Scholze
in \cite{LTE2021}. People involved in the formalization simply pushed
their way to reach the summit, formalizing lemma after lemma. They
actually wrote the blueprint for that formalization \emph{afterwards}
it was complete! From time to time, we were also frenziedly trying to
get the results formalized, going beyond what we had planned.

As a result from this, some design choices that seemed reasonable at
first were proved to be inconvenient. For instance, we should had
better used predicates (of type $\tyi\fun\tyi\fun\tyo$) for the
forcing posets' order relations; this is the way they
are presented in the \session{Delta\_System\_Lemma} session. A similar
problem is that we require the forcing poset to be an element of $M$,
so the present infrastructure does not allow class forcing out of the
box. (The latter change seems to be rather straightforward, but the
former does not.)

Nearly the final stage of the project, we decided to go for the minimal
set of definitions and versions of lemmas that were needed to obtain
our target results. For example, we only proved the Delta System Lemma
for $\aleph_1$-sized families (thus limiting us to the plain ccc) and
showed preservation of sequences by considering countably closed
forcings (in fact, we formalized the bare minimum requirement of being
$(<\omega+1)$-closed). In doing this we went against the tried and
true advice that one should formalize the most general version of the
results available.

\subsection{Bureaucracy and scale factors}

\begin{enumerate}
\item Bureaucracy vs ML programming.
\item The “math” was already formalized on 22 November 2020.
  We finished the last goal on 22 August 2021.
  (Update: 20 November 2021 \& 28 November 2021, for CH)
\item Missing: automation of closure of models under operations.
\item Missing: basic arithmetic for dealing with arities.
\end{enumerate}

\begin{enumerate}
\item It is extremely misleading when automatic tools (\isatt{simp}, \isatt{auto}, etc)
  stop working just because of the sheer size of the goal. Oftentimes,
  in math, we disregard scale issues but they must always be taken
  into account in CS.
\item Example: $\forceisa(0\in 1)$ is expandable,
  $\forceisa(\neg\neg  0\in 1)$ is not.
\item Example: Synthesis of $\forceisa$; could have been fully synthesized,
  but that was dirty “strategy”.
\item The know-how of computer scientists on this kind of engineering is
  very important
\end{enumerate}

\subsection{You might have formalized it, and still be wrong}
\begin{enumerate}
\item Example: restriction of relations.
\item Pollack, “Pollack consistency” by
  Wiedijk. Cf. \theory{Definitions\_Main} (thanks to discussions with
  Vidnyánszky). Opacity of automated proofs. 
\item Plot twist: You can be right without knowing. Intuition may drive proofs
  even if we are not working on what we believe we are.
\end{enumerate}

\subsection{Beware of the “Code fever”}\label{sec:beware-code-fever}
\begin{enumerate}
\item “We know that doing math is fun---formalization is like DRUGS”
\item Feeling of accomplishment after seeing your writings
  validated beyond reasonable doubt (v.g. cofinality).

\item One easily forgets about the “Power of the Board.”
\end{enumerate}

\subsection{The Devil's on the shortcuts}
\begin{enumerate}
\item
  Our proofs of the “definition of forces” (and many
  consequences) and of the lemma for “forcing a value” of function
  depend on the countability of the ground model. 
\item
  Density arguments (look for “TODO”, “general versions”).
\end{enumerate}

%%% Local Variables: 
%%% mode: latex
%%% TeX-master: "independence_ch_isabelle"
%%% ispell-local-dictionary: "american"
%%% End: 


\section{Conclusions and future work}
There are several technical milestones that have to be reached in the
course of a formalization of the theory of forcing. The first one, and most
obvious, is the bulk of set- and meta-theoretical concepts needed to work
with. This pushed us, in a sense,  into building on top of Isabelle/ZF,
since we know of no other development in set theory of such
depth (and breadth). In this paper we worked on setting the stage for the work with
generic extensions; in particular, this involves some purely mathematical
results, as the Rasiowa-Sikorski lemma. 

Other milestones in this formalization project
involve 
\begin{enumerate}
\item the definition
  of the forcing relation, 
\item proving the Fundamental Theorem of forcing
  (that relates truth in $M$ to that in $M[G]$), and 
\item using it to show
  that $M[G]\models \ZFC$. 
\end{enumerate}
The theory is very modular and this is
witnessed by the fact 
that the last goal does not depend on the proof of the Fundamental
Theorem nor on the definition of the forcing relation. Our next task
will be to obtain the last goal in that enumeration. 

To this end, we will develop an interface between Paulson's
relativization results and countable models of $\ZFC$. This will show
that every ctm $M$ is closed under well-founded recursion and, in
particular, that contains names for each of its
elements. Consequently, the proof of  $M\sbq M[G]$ will be
complete. A landmark will be to prove the Axiom Scheme
of Separation (the first that needs to use the machinery of forcing
nontrivially). As a part of the new formalization, we will provide
Isar versions of the longer applicative proofs presented in this work.

\ack{We'd like to thank the anonymous referees for reading the paper
  carefully and for their detailed and constructive criticism.}
%%% Local Variables:
%%% mode: latex
%%% ispell-local-dictionary: "american"
%%% TeX-master: "first_steps_into_forcing"
%%% End:

%
% ---- Bibliography ----
%
% BibTeX users should specify bibliography style 'splncs04'.
% References will then be sorted and formatted in the correct style.
%
%\bibliographystyle{splncs04}
\bibliographystyle{mi-estilo-else}
\bibliography{independence_ch_isabelle}

\appendix

\section{Main definitions of the formalization}\label{sec:definitions_main}

This section, which appears almost verbatim as
the theory \theory{Definitions\_Main} in \cite{Independence_CH-AFP},
might be considered as the bare minimum reading requisite to
trust that our development indeed formalizes the theory of
forcing.

The reader trusting
all the libraries on which our development is based, might jump
directly to Section~\ref{sec:relative-arith}, which treats relative
cardinal arithmetic as implemented in
\isa{T{\kern0pt}r{\kern0pt}a{\kern0pt}n{\kern0pt}s{\kern0pt}i{\kern0pt}t{\kern0pt}i{\kern0pt}v{\kern0pt}e{\kern0pt}{\char`\_}{\kern0pt}M{\kern0pt}o{\kern0pt}d{\kern0pt}e{\kern0pt}l{\kern0pt}s{\kern0pt}}. But in case one wants to dive deeper, the
following sections treat some basic concepts of the ZF logic
(Section~\ref{sec:def-main-ZF}) and in the
\session{ZF-Constructible} library (Section~\ref{sec:def-main-relative})
on which our definitions are built.

\subsection{ZF\label{sec:def-main-ZF}%
}
For the basic logic ZF we restrict ourselves to just a few
concepts (for its axioms, consult Appendix~\ref{appendix:axioms}).
%
\begin{isabelle}%
bij{\isacharparenleft}{\kern0pt}A{\isacharcomma}{\kern0pt}\ B{\isacharparenright}{\kern0pt}\ {\isasymequiv}\isanewline
{\isacharbraceleft}{\kern0pt}f\ {\isasymin}\ A\ {\isasymrightarrow}\ B\ {\isachardot}{\kern0pt}\ {\isasymforall}w{\isasymin}A{\isachardot}{\kern0pt}\ {\isasymforall}x{\isasymin}A{\isachardot}{\kern0pt}\ f\ {\isacharbackquote}{\kern0pt}\ w\ {\isacharequal}{\kern0pt}\ f\ {\isacharbackquote}{\kern0pt}\ x\ {\isasymlongrightarrow}\ w\ {\isacharequal}{\kern0pt}\ x{\isacharbraceright}{\kern0pt}\ {\isasyminter}\isanewline
{\isacharbraceleft}{\kern0pt}f\ {\isasymin}\ A\ {\isasymrightarrow}\ B\ {\isachardot}{\kern0pt}\ {\isasymforall}y{\isasymin}B{\isachardot}{\kern0pt}\ {\isasymexists}x{\isasymin}A{\isachardot}{\kern0pt}\ f\ {\isacharbackquote}{\kern0pt}\ x\ {\isacharequal}{\kern0pt}\ y{\isacharbraceright}{\kern0pt}%
\end{isabelle}%
\begin{isabelle}%
A\ {\isasymapprox}\ B\ {\isasymequiv}\ {\isasymexists}f{\isachardot}{\kern0pt}\ f\ {\isasymin}\ bij{\isacharparenleft}{\kern0pt}A{\isacharcomma}{\kern0pt}\ B{\isacharparenright}{\kern0pt}%
\end{isabelle}%
\begin{isabelle}%
Transset{\isacharparenleft}{\kern0pt}i{\isacharparenright}{\kern0pt}\ {\isasymequiv}\ {\isasymforall}x{\isasymin}i{\isachardot}{\kern0pt}\ x\ {\isasymsubseteq}\ i%
\end{isabelle}%
\begin{isabelle}%
Ord{\isacharparenleft}{\kern0pt}i{\isacharparenright}{\kern0pt}\ {\isasymequiv}\ Transset{\isacharparenleft}{\kern0pt}i{\isacharparenright}{\kern0pt}\ {\isasymand}\ {\isacharparenleft}{\kern0pt}{\isasymforall}x{\isasymin}i{\isachardot}{\kern0pt}\ Transset{\isacharparenleft}{\kern0pt}x{\isacharparenright}{\kern0pt}{\isacharparenright}{\kern0pt}%
\end{isabelle}%
\begin{isabelle}%
i\ {\isacharless}{\kern0pt}\ j\ {\isasymequiv}\ i\ {\isasymin}\ j\ {\isasymand}\ Ord{\isacharparenleft}{\kern0pt}j{\isacharparenright}{\kern0pt}\isasep\isanewline%
i\ {\isasymle}\ j\ {\isasymlongleftrightarrow}\ i\ {\isacharless}{\kern0pt}\ j\ {\isasymor}\ {\isacharparenleft}i\ {\isacharequal}{\kern0pt}\ j\ {\isasymand}\ Ord{\isacharparenleft}{\kern0pt}j{\isacharparenright}{\isacharparenright}{\kern0pt}%
\end{isabelle}%
With the concepts of empty set and successor in place,%

\begin{isabelle}
\isacommand{lemma}
\ empty{\uscore}{\kern0pt}def{\isacharprime}{\kern0pt}{\isacharcolon}{\kern0pt}\ {\isachardoublequoteopen}{\isasymforall}x{\isachardot}{\kern0pt}\ x\ {\isasymnotin}\ {\isadigit{0}}{\isachardoublequoteclose}%
\isanewline
\isacommand{lemma}
\ succ{\uscore}{\kern0pt}def{\isacharprime}{\kern0pt}{\isacharcolon}{\kern0pt}\ {\isachardoublequoteopen}succ{\isacharparenleft}{\kern0pt}i{\isacharparenright}{\kern0pt}\ {\isacharequal}{\kern0pt}\ i\ {\isasymunion}\ {\isacharbraceleft}{\kern0pt}i{\isacharbraceright}{\kern0pt}{\isachardoublequoteclose}%
\end{isabelle}
%
we can define the set of natural numbers \isa{{\isasymomega}}. In the
sources, it is  defined as a fixpoint, but here we just write
its characterization as the first limit ordinal.%
\begin{isabelle}%
Ord{\isacharparenleft}{\kern0pt}{\isasymomega}{\isacharparenright}{\kern0pt}\ {\isasymand}\ {\isadigit{0}}\ {\isacharless}{\kern0pt}\ {\isasymomega}\ {\isasymand}\ {\isacharparenleft}{\kern0pt}{\isasymforall}y{\isachardot}{\kern0pt}\ y\ {\isacharless}{\kern0pt}\ {\isasymomega}\ {\isasymlongrightarrow}\ succ{\isacharparenleft}{\kern0pt}y{\isacharparenright}{\kern0pt}\ {\isacharless}{\kern0pt}\ {\isasymomega}{\isacharparenright}{\kern0pt}\isasep\isanewline%
Ord{\isacharparenleft}{\kern0pt}i{\isacharparenright}{\kern0pt}\ {\isasymand}\ {\isadigit{0}}\ {\isacharless}{\kern0pt}\ i\ {\isasymand}\ {\isacharparenleft}{\kern0pt}{\isasymforall}y{\isachardot}{\kern0pt}\ y\ {\isacharless}{\kern0pt}\ i\ {\isasymlongrightarrow}\ succ{\isacharparenleft}{\kern0pt}y{\isacharparenright}{\kern0pt}\ {\isacharless}{\kern0pt}\ i{\isacharparenright}{\kern0pt}\ {\isasymLongrightarrow}\ {\isasymomega}\ {\isasymle}\ i%
\end{isabelle}%
Then, addition and predecessor on \isa{{\isasymomega}} are inductively
characterized as follows:%
\begin{isabelle}%
m\ {\isacharplus}{\kern0pt}\isactrlsub {\isasymomega}\ succ{\isacharparenleft}{\kern0pt}n{\isacharparenright}{\kern0pt}\ {\isacharequal}{\kern0pt}\ succ{\isacharparenleft}{\kern0pt}m\ {\isacharplus}{\kern0pt}\isactrlsub {\isasymomega}\ n{\isacharparenright}{\kern0pt}\isasep\isanewline%
m\ {\isasymin}\ {\isasymomega}\ {\isasymLongrightarrow}\ m\ {\isacharplus}{\kern0pt}\isactrlsub {\isasymomega}\ {\isadigit{0}}\ {\isacharequal}{\kern0pt}\ m\isasep\isanewline\isanewline%
pred{\isacharparenleft}{\kern0pt}{\isadigit{0}}{\isacharparenright}{\kern0pt}\ {\isacharequal}{\kern0pt}\ {\isadigit{0}}\isasep\isanewline%
pred{\isacharparenleft}{\kern0pt}succ{\isacharparenleft}{\kern0pt}y{\isacharparenright}{\kern0pt}{\isacharparenright}{\kern0pt}\ {\isacharequal}{\kern0pt}\ y%
\end{isabelle}%
Lists on a set \isa{A} can be characterized by being
recursively generated from the empty list \isa{{\isacharbrackleft}{\kern0pt}{\isacharbrackright}{\kern0pt}} and the
operation \isa{Cons} that adds a new element to the left end;
the induction theorem for them shows that the characterization is
“complete”. (Mind the
\isa{\isasymlbrakk P; Q\isasymrbrakk\ \isasymLongrightarrow\ R}
abbreviation for
\isa{P\ \isasymLongrightarrow\ Q\ \isasymLongrightarrow\ R}.)

\begin{isabelle}%
{\isacharbrackleft}{\kern0pt}{\isacharbrackright}{\kern0pt}\ {\isasymin}\ list{\isacharparenleft}{\kern0pt}A{\isacharparenright}{\kern0pt}\isasep\isanewline%
{\isasymlbrakk}a\ {\isasymin}\ A{\isacharsemicolon}{\kern0pt}\ l\ {\isasymin}\ list{\isacharparenleft}{\kern0pt}A{\isacharparenright}{\kern0pt}{\isasymrbrakk}\ {\isasymLongrightarrow}\ Cons{\isacharparenleft}{\kern0pt}a{\isacharcomma}{\kern0pt}\ l{\isacharparenright}{\kern0pt}\ {\isasymin}\ list{\isacharparenleft}{\kern0pt}A{\isacharparenright}{\kern0pt}\isasep\isanewline\isanewline%
{\isasymlbrakk}x\ {\isasymin}\ list{\isacharparenleft}{\kern0pt}A{\isacharparenright}{\kern0pt}{\isacharsemicolon}{\kern0pt}\ P{\isacharparenleft}{\kern0pt}{\isacharbrackleft}{\kern0pt}{\isacharbrackright}{\kern0pt}{\isacharparenright}{\kern0pt}{\isacharsemicolon}{\kern0pt}\ {\isasymAnd}a\ l{\isachardot}{\kern0pt}\ {\isasymlbrakk}a\ {\isasymin}\ A{\isacharsemicolon}{\kern0pt}\ l\ {\isasymin}\ list{\isacharparenleft}{\kern0pt}A{\isacharparenright}{\kern0pt}{\isacharsemicolon}{\kern0pt}\ P{\isacharparenleft}{\kern0pt}l{\isacharparenright}{\kern0pt}{\isasymrbrakk}\ {\isasymLongrightarrow}\isanewline
\ \ P{\isacharparenleft}{\kern0pt}Cons{\isacharparenleft}{\kern0pt}a{\isacharcomma}{\kern0pt}\ l{\isacharparenright}{\kern0pt}{\isacharparenright}{\kern0pt}{\isasymrbrakk}
{\isasymLongrightarrow}\ P{\isacharparenleft}{\kern0pt}x{\isacharparenright}{\kern0pt}%
\end{isabelle}%
Length, concatenation, and \isa{n}th element of lists are
recursively characterized as follows.%
\begin{isabelle}%
length{\isacharparenleft}{\kern0pt}{\isacharbrackleft}{\kern0pt}{\isacharbrackright}{\kern0pt}{\isacharparenright}{\kern0pt}\ {\isacharequal}{\kern0pt}\ {\isadigit{0}}\isasep\isanewline%
length{\isacharparenleft}{\kern0pt}Cons{\isacharparenleft}{\kern0pt}a{\isacharcomma}{\kern0pt}\ l{\isacharparenright}{\kern0pt}{\isacharparenright}{\kern0pt}\ {\isacharequal}{\kern0pt}\ succ{\isacharparenleft}{\kern0pt}length{\isacharparenleft}{\kern0pt}l{\isacharparenright}{\kern0pt}{\isacharparenright}{\kern0pt}\isasep\isanewline\isanewline%
{\isacharbrackleft}{\kern0pt}{\isacharbrackright}{\kern0pt}\ {\isacharat}{\kern0pt}\ ys\ {\isacharequal}{\kern0pt}\ ys\isasep\isanewline%
Cons{\isacharparenleft}{\kern0pt}a{\isacharcomma}{\kern0pt}\ l{\isacharparenright}{\kern0pt}\ {\isacharat}{\kern0pt}\ ys\ {\isacharequal}{\kern0pt}\ Cons{\isacharparenleft}{\kern0pt}a{\isacharcomma}{\kern0pt}\ l\ {\isacharat}{\kern0pt}\ ys{\isacharparenright}{\kern0pt}\isasep\isanewline\isanewline%
nth{\isacharparenleft}{\kern0pt}{\isadigit{0}}{\isacharcomma}{\kern0pt}\ Cons{\isacharparenleft}{\kern0pt}a{\isacharcomma}{\kern0pt}\ l{\isacharparenright}{\kern0pt}{\isacharparenright}{\kern0pt}\ {\isacharequal}{\kern0pt}\ a\isasep\isanewline%
n\ {\isasymin}\ {\isasymomega}\ {\isasymLongrightarrow}\ nth{\isacharparenleft}{\kern0pt}succ{\isacharparenleft}{\kern0pt}n{\isacharparenright}{\kern0pt}{\isacharcomma}{\kern0pt}\ Cons{\isacharparenleft}{\kern0pt}a{\isacharcomma}{\kern0pt}\ l{\isacharparenright}{\kern0pt}{\isacharparenright}{\kern0pt}\ {\isacharequal}{\kern0pt}\ nth{\isacharparenleft}{\kern0pt}n{\isacharcomma}{\kern0pt}\ l{\isacharparenright}{\kern0pt}%
\end{isabelle}%
We have the usual Haskell-like notation for iterated applications
of \isa{Cons}:%
\begin{isabelle}
\isacommand{lemma}\isamarkupfalse%
\ Cons{\isacharunderscore}{\kern0pt}app{\isacharcolon}{\kern0pt}\ {\isachardoublequoteopen}{\isacharbrackleft}{\kern0pt}a{\isacharcomma}{\kern0pt}b{\isacharcomma}{\kern0pt}c{\isacharbrackright}{\kern0pt}\ {\isacharequal}{\kern0pt}\ Cons{\isacharparenleft}{\kern0pt}a{\isacharcomma}{\kern0pt}Cons{\isacharparenleft}{\kern0pt}b{\isacharcomma}{\kern0pt}Cons{\isacharparenleft}{\kern0pt}c{\isacharcomma}{\kern0pt}{\isacharbrackleft}{\kern0pt}{\isacharbrackright}{\kern0pt}{\isacharparenright}{\kern0pt}{\isacharparenright}{\kern0pt}{\isacharparenright}{\kern0pt}{\isachardoublequoteclose}%
\end{isabelle}

%
%
Relative quantifiers restrict the range of the bound variable to a
class \isa{M} of type \isa{i\ {\isasymRightarrow}\ o}; that is, a truth-valued function with
set arguments.%
\begin{isabelle}
\isacommand{lemma}\isamarkupfalse%
\ {\isachardoublequoteopen}{\isasymforall}x{\isacharbrackleft}{\kern0pt}M{\isacharbrackright}{\kern0pt}{\isachardot}{\kern0pt}\ P{\isacharparenleft}{\kern0pt}x{\isacharparenright}{\kern0pt}\ {\isasymequiv}\ {\isasymforall}x{\isachardot}{\kern0pt}\ M{\isacharparenleft}{\kern0pt}x{\isacharparenright}{\kern0pt}\ {\isasymlongrightarrow}\ P{\isacharparenleft}{\kern0pt}x{\isacharparenright}{\kern0pt}{\isachardoublequoteclose}\isanewline
\ \ \ \ \ \ {\isachardoublequoteopen}{\isasymexists}x{\isacharbrackleft}{\kern0pt}M{\isacharbrackright}{\kern0pt}{\isachardot}{\kern0pt}\ P{\isacharparenleft}{\kern0pt}x{\isacharparenright}{\kern0pt}\ {\isasymequiv}\ {\isasymexists}x{\isachardot}{\kern0pt}\ M{\isacharparenleft}{\kern0pt}x{\isacharparenright}{\kern0pt}\ {\isasymand}\ P{\isacharparenleft}{\kern0pt}x{\isacharparenright}{\kern0pt}{\isachardoublequoteclose}
\end{isabelle}
%
%
Finally, a set can be viewed (“cast”) as a class using the
following function of type \isa{i\ {\isasymRightarrow}\ i\ {\isasymRightarrow}\ o}.%
\begin{isabelle}%
{\isacharparenleft}{\kern0pt}{\isacharhash}{\kern0pt}{\isacharhash}{\kern0pt}A{\isacharparenright}{\kern0pt}{\isacharparenleft}{\kern0pt}x{\isacharparenright}{\kern0pt}\ {\isasymlongleftrightarrow}\ x\ {\isasymin}\ A%
\end{isabelle}%
\subsection{Relative concepts\label{sec:def-main-relative}%
}
A list of relative concepts (mostly from the \session{ZF-Constructible}
    library) follows next.%
\begin{isabelle}%
big{\isacharunderscore}{\kern0pt}union{\isacharparenleft}{\kern0pt}M{\isacharcomma}{\kern0pt}\ A{\isacharcomma}{\kern0pt}\ z{\isacharparenright}{\kern0pt}\ {\isasymequiv}\ {\isasymforall}x{\isacharbrackleft}{\kern0pt}M{\isacharbrackright}{\kern0pt}{\isachardot}{\kern0pt}\ x\ {\isasymin}\ z\ {\isasymlongleftrightarrow}\ {\isacharparenleft}{\kern0pt}{\isasymexists}y{\isacharbrackleft}{\kern0pt}M{\isacharbrackright}{\kern0pt}{\isachardot}{\kern0pt}\ y\ {\isasymin}\ A\ {\isasymand}\ x\ {\isasymin}\ y{\isacharparenright}{\kern0pt}%
\end{isabelle}%
\begin{isabelle}%
upair{\isacharparenleft}{\kern0pt}M{\isacharcomma}{\kern0pt}\ a{\isacharcomma}{\kern0pt}\ b{\isacharcomma}{\kern0pt}\ z{\isacharparenright}{\kern0pt}\ {\isasymequiv}\ a\ {\isasymin}\ z\ {\isasymand}\ b\ {\isasymin}\ z\ {\isasymand}\ {\isacharparenleft}{\kern0pt}{\isasymforall}x{\isacharbrackleft}{\kern0pt}M{\isacharbrackright}{\kern0pt}{\isachardot}{\kern0pt}\ x\ {\isasymin}\ z\ {\isasymlongrightarrow}\ x\ {\isacharequal}{\kern0pt}\ a\ {\isasymor}\ x\ {\isacharequal}{\kern0pt}\ b{\isacharparenright}{\kern0pt}%
\end{isabelle}%
\begin{isabelle}%
pair{\isacharparenleft}{\kern0pt}M{\isacharcomma}{\kern0pt}\ a{\isacharcomma}{\kern0pt}\ b{\isacharcomma}{\kern0pt}\ z{\isacharparenright}{\kern0pt}\ {\isasymequiv}\isanewline
{\isasymexists}x{\isacharbrackleft}{\kern0pt}M{\isacharbrackright}{\kern0pt}{\isachardot}{\kern0pt}\ upair{\isacharparenleft}{\kern0pt}M{\isacharcomma}{\kern0pt}\ a{\isacharcomma}{\kern0pt}\ a{\isacharcomma}{\kern0pt}\ x{\isacharparenright}{\kern0pt}\ {\isasymand}\ {\isacharparenleft}{\kern0pt}{\isasymexists}y{\isacharbrackleft}{\kern0pt}M{\isacharbrackright}{\kern0pt}{\isachardot}{\kern0pt}\ upair{\isacharparenleft}{\kern0pt}M{\isacharcomma}{\kern0pt}\ a{\isacharcomma}{\kern0pt}\ b{\isacharcomma}{\kern0pt}\ y{\isacharparenright}{\kern0pt}\ {\isasymand}\ upair{\isacharparenleft}{\kern0pt}M{\isacharcomma}{\kern0pt}\ x{\isacharcomma}{\kern0pt}\ y{\isacharcomma}{\kern0pt}\ z{\isacharparenright}{\kern0pt}{\isacharparenright}{\kern0pt}%
\end{isabelle}%
\begin{isabelle}%
successor{\isacharparenleft}{\kern0pt}M{\isacharcomma}{\kern0pt}\ a{\isacharcomma}{\kern0pt}\ z{\isacharparenright}{\kern0pt}\ {\isasymequiv}\isanewline
{\isasymexists}x{\isacharbrackleft}{\kern0pt}M{\isacharbrackright}{\kern0pt}{\isachardot}{\kern0pt}\ upair{\isacharparenleft}{\kern0pt}M{\isacharcomma}{\kern0pt}\ a{\isacharcomma}{\kern0pt}\ a{\isacharcomma}{\kern0pt}\ x{\isacharparenright}{\kern0pt}\ {\isasymand}\ {\isacharparenleft}{\kern0pt}{\isasymforall}xa{\isacharbrackleft}{\kern0pt}M{\isacharbrackright}{\kern0pt}{\isachardot}{\kern0pt}\ xa\ {\isasymin}\ z\ {\isasymlongleftrightarrow}\ xa\ {\isasymin}\ x\ {\isasymor}\ xa\ {\isasymin}\ a{\isacharparenright}{\kern0pt}%
\end{isabelle}%
\begin{isabelle}%
empty{\isacharparenleft}{\kern0pt}M{\isacharcomma}{\kern0pt}\ z{\isacharparenright}{\kern0pt}\ {\isasymequiv}\ {\isasymforall}x{\isacharbrackleft}{\kern0pt}M{\isacharbrackright}{\kern0pt}{\isachardot}{\kern0pt}\ x\ {\isasymnotin}\ z%
\end{isabelle}%
\begin{isabelle}%
transitive{\isacharunderscore}{\kern0pt}set{\isacharparenleft}{\kern0pt}M{\isacharcomma}{\kern0pt}\ a{\isacharparenright}{\kern0pt}\ {\isasymequiv}\ {\isasymforall}x{\isacharbrackleft}{\kern0pt}M{\isacharbrackright}{\kern0pt}{\isachardot}{\kern0pt}\ x\ {\isasymin}\ a\ {\isasymlongrightarrow}\ {\isacharparenleft}{\kern0pt}{\isasymforall}xa{\isacharbrackleft}{\kern0pt}M{\isacharbrackright}{\kern0pt}{\isachardot}{\kern0pt}\ xa\ {\isasymin}\ x\ {\isasymlongrightarrow}\ xa\ {\isasymin}\ a{\isacharparenright}{\kern0pt}%
\end{isabelle}%
\begin{isabelle}%
ordinal{\isacharparenleft}{\kern0pt}M{\isacharcomma}{\kern0pt}\ a{\isacharparenright}{\kern0pt}\ {\isasymequiv}\isanewline
transitive{\isacharunderscore}{\kern0pt}set{\isacharparenleft}{\kern0pt}M{\isacharcomma}{\kern0pt}\ a{\isacharparenright}{\kern0pt}\ {\isasymand}\ {\isacharparenleft}{\kern0pt}{\isasymforall}x{\isacharbrackleft}{\kern0pt}M{\isacharbrackright}{\kern0pt}{\isachardot}{\kern0pt}\ x\ {\isasymin}\ a\ {\isasymlongrightarrow}\ transitive{\isacharunderscore}{\kern0pt}set{\isacharparenleft}{\kern0pt}M{\isacharcomma}{\kern0pt}\ x{\isacharparenright}{\kern0pt}{\isacharparenright}{\kern0pt}%
\end{isabelle}%
\begin{isabelle}%
image{\isacharparenleft}{\kern0pt}M{\isacharcomma}{\kern0pt}\ r{\isacharcomma}{\kern0pt}\ A{\isacharcomma}{\kern0pt}\ z{\isacharparenright}{\kern0pt}\ {\isasymequiv}\isanewline
{\isasymforall}y{\isacharbrackleft}{\kern0pt}M{\isacharbrackright}{\kern0pt}{\isachardot}{\kern0pt}\ y\ {\isasymin}\ z\ {\isasymlongleftrightarrow}\ {\isacharparenleft}{\kern0pt}{\isasymexists}w{\isacharbrackleft}{\kern0pt}M{\isacharbrackright}{\kern0pt}{\isachardot}{\kern0pt}\ w\ {\isasymin}\ r\ {\isasymand}\ {\isacharparenleft}{\kern0pt}{\isasymexists}x{\isacharbrackleft}{\kern0pt}M{\isacharbrackright}{\kern0pt}{\isachardot}{\kern0pt}\ x\ {\isasymin}\ A\ {\isasymand}\ pair{\isacharparenleft}{\kern0pt}M{\isacharcomma}{\kern0pt}\ x{\isacharcomma}{\kern0pt}\ y{\isacharcomma}{\kern0pt}\ w{\isacharparenright}{\kern0pt}{\isacharparenright}{\kern0pt}{\isacharparenright}{\kern0pt}%
\end{isabelle}%
\begin{isabelle}%
is{\isacharunderscore}{\kern0pt}apply{\isacharparenleft}{\kern0pt}M{\isacharcomma}{\kern0pt}\ f{\isacharcomma}{\kern0pt}\ x{\isacharcomma}{\kern0pt}\ y{\isacharparenright}{\kern0pt}\ {\isasymequiv}\isanewline
{\isasymexists}xs{\isacharbrackleft}{\kern0pt}M{\isacharbrackright}{\kern0pt}{\isachardot}{\kern0pt}\isanewline
\isaindent{\ \ \ }{\isasymexists}fxs{\isacharbrackleft}{\kern0pt}M{\isacharbrackright}{\kern0pt}{\isachardot}{\kern0pt}\ upair{\isacharparenleft}{\kern0pt}M{\isacharcomma}{\kern0pt}\ x{\isacharcomma}{\kern0pt}\ x{\isacharcomma}{\kern0pt}\ xs{\isacharparenright}{\kern0pt}\ {\isasymand}\ image{\isacharparenleft}{\kern0pt}M{\isacharcomma}{\kern0pt}\ f{\isacharcomma}{\kern0pt}\ xs{\isacharcomma}{\kern0pt}\ fxs{\isacharparenright}{\kern0pt}\ {\isasymand}\isanewline
\isaindent{\ \ \ \ \ }big{\isacharunderscore}{\kern0pt}union{\isacharparenleft}{\kern0pt}M{\isacharcomma}{\kern0pt}\ fxs{\isacharcomma}{\kern0pt}\ y{\isacharparenright}{\kern0pt}%
\end{isabelle}%
\begin{isabelle}%
is{\isacharunderscore}{\kern0pt}function{\isacharparenleft}{\kern0pt}M{\isacharcomma}{\kern0pt}\ r{\isacharparenright}{\kern0pt}\ {\isasymequiv}\isanewline
{\isasymforall}x{\isacharbrackleft}{\kern0pt}M{\isacharbrackright}{\kern0pt}{\isachardot}{\kern0pt}\isanewline
\isaindent{\ \ \ }{\isasymforall}y{\isacharbrackleft}{\kern0pt}M{\isacharbrackright}{\kern0pt}{\isachardot}{\kern0pt}\isanewline
\isaindent{\ \ \ \ \ \ }{\isasymforall}y{\isacharprime}{\kern0pt}{\isacharbrackleft}{\kern0pt}M{\isacharbrackright}{\kern0pt}{\isachardot}{\kern0pt}\isanewline
\isaindent{\ \ \ \ \ \ \ \ \ }{\isasymforall}p{\isacharbrackleft}{\kern0pt}M{\isacharbrackright}{\kern0pt}{\isachardot}{\kern0pt}\isanewline
\isaindent{\ \ \ \ \ \ \ \ \ \ \ \ }{\isasymforall}p{\isacharprime}{\kern0pt}{\isacharbrackleft}{\kern0pt}M{\isacharbrackright}{\kern0pt}{\isachardot}{\kern0pt}\isanewline
\isaindent{\ \ \ \ \ \ \ \ \ \ \ \ \ \ \ }pair{\isacharparenleft}{\kern0pt}M{\isacharcomma}{\kern0pt}\ x{\isacharcomma}{\kern0pt}\ y{\isacharcomma}{\kern0pt}\ p{\isacharparenright}{\kern0pt}\ {\isasymlongrightarrow}\isanewline
\isaindent{\ \ \ \ \ \ \ \ \ \ \ \ \ \ \ }pair{\isacharparenleft}{\kern0pt}M{\isacharcomma}{\kern0pt}\ x{\isacharcomma}{\kern0pt}\ y{\isacharprime}{\kern0pt}{\isacharcomma}{\kern0pt}\ p{\isacharprime}{\kern0pt}{\isacharparenright}{\kern0pt}\ {\isasymlongrightarrow}\ p\ {\isasymin}\ r\ {\isasymlongrightarrow}\ p{\isacharprime}{\kern0pt}\ {\isasymin}\ r\ {\isasymlongrightarrow}\ y\ {\isacharequal}{\kern0pt}\ y{\isacharprime}{\kern0pt}%
\end{isabelle}%
\begin{isabelle}%
is{\isacharunderscore}{\kern0pt}relation{\isacharparenleft}{\kern0pt}M{\isacharcomma}{\kern0pt}\ r{\isacharparenright}{\kern0pt}\ {\isasymequiv}\ {\isasymforall}z{\isacharbrackleft}{\kern0pt}M{\isacharbrackright}{\kern0pt}{\isachardot}{\kern0pt}\ z\ {\isasymin}\ r\ {\isasymlongrightarrow}\ {\isacharparenleft}{\kern0pt}{\isasymexists}x{\isacharbrackleft}{\kern0pt}M{\isacharbrackright}{\kern0pt}{\isachardot}{\kern0pt}\ {\isasymexists}y{\isacharbrackleft}{\kern0pt}M{\isacharbrackright}{\kern0pt}{\isachardot}{\kern0pt}\ pair{\isacharparenleft}{\kern0pt}M{\isacharcomma}{\kern0pt}\ x{\isacharcomma}{\kern0pt}\ y{\isacharcomma}{\kern0pt}\ z{\isacharparenright}{\kern0pt}{\isacharparenright}{\kern0pt}%
\end{isabelle}%
\begin{isabelle}%
is{\isacharunderscore}{\kern0pt}domain{\isacharparenleft}{\kern0pt}M{\isacharcomma}{\kern0pt}\ r{\isacharcomma}{\kern0pt}\ z{\isacharparenright}{\kern0pt}\ {\isasymequiv}\isanewline
{\isasymforall}x{\isacharbrackleft}{\kern0pt}M{\isacharbrackright}{\kern0pt}{\isachardot}{\kern0pt}\ x\ {\isasymin}\ z\ {\isasymlongleftrightarrow}\ {\isacharparenleft}{\kern0pt}{\isasymexists}w{\isacharbrackleft}{\kern0pt}M{\isacharbrackright}{\kern0pt}{\isachardot}{\kern0pt}\ w\ {\isasymin}\ r\ {\isasymand}\ {\isacharparenleft}{\kern0pt}{\isasymexists}y{\isacharbrackleft}{\kern0pt}M{\isacharbrackright}{\kern0pt}{\isachardot}{\kern0pt}\ pair{\isacharparenleft}{\kern0pt}M{\isacharcomma}{\kern0pt}\ x{\isacharcomma}{\kern0pt}\ y{\isacharcomma}{\kern0pt}\ w{\isacharparenright}{\kern0pt}{\isacharparenright}{\kern0pt}{\isacharparenright}{\kern0pt}%
\end{isabelle}%
\begin{isabelle}%
typed{\isacharunderscore}{\kern0pt}function{\isacharparenleft}{\kern0pt}M{\isacharcomma}{\kern0pt}\ A{\isacharcomma}{\kern0pt}\ B{\isacharcomma}{\kern0pt}\ r{\isacharparenright}{\kern0pt}\ {\isasymequiv}\isanewline
is{\isacharunderscore}{\kern0pt}function{\isacharparenleft}{\kern0pt}M{\isacharcomma}{\kern0pt}\ r{\isacharparenright}{\kern0pt}\ {\isasymand}\isanewline
is{\isacharunderscore}{\kern0pt}relation{\isacharparenleft}{\kern0pt}M{\isacharcomma}{\kern0pt}\ r{\isacharparenright}{\kern0pt}\ {\isasymand}\isanewline
is{\isacharunderscore}{\kern0pt}domain{\isacharparenleft}{\kern0pt}M{\isacharcomma}{\kern0pt}\ r{\isacharcomma}{\kern0pt}\ A{\isacharparenright}{\kern0pt}\ {\isasymand}\isanewline
{\isacharparenleft}{\kern0pt}{\isasymforall}u{\isacharbrackleft}{\kern0pt}M{\isacharbrackright}{\kern0pt}{\isachardot}{\kern0pt}\ u\ {\isasymin}\ r\ {\isasymlongrightarrow}\ {\isacharparenleft}{\kern0pt}{\isasymforall}x{\isacharbrackleft}{\kern0pt}M{\isacharbrackright}{\kern0pt}{\isachardot}{\kern0pt}\ {\isasymforall}y{\isacharbrackleft}{\kern0pt}M{\isacharbrackright}{\kern0pt}{\isachardot}{\kern0pt}\ pair{\isacharparenleft}{\kern0pt}M{\isacharcomma}{\kern0pt}\ x{\isacharcomma}{\kern0pt}\ y{\isacharcomma}{\kern0pt}\ u{\isacharparenright}{\kern0pt}\ {\isasymlongrightarrow}\ y\ {\isasymin}\ B{\isacharparenright}{\kern0pt}{\isacharparenright}{\kern0pt}%
\end{isabelle}%
\begin{isabelle}%
is{\isacharunderscore}{\kern0pt}function{\isacharunderscore}{\kern0pt}space{\isacharparenleft}{\kern0pt}M{\isacharcomma}{\kern0pt}\ A{\isacharcomma}{\kern0pt}\ B{\isacharcomma}{\kern0pt}\ fs{\isacharparenright}{\kern0pt}\ {\isasymequiv}\isanewline
M{\isacharparenleft}{\kern0pt}fs{\isacharparenright}{\kern0pt}\ {\isasymand}\ {\isacharparenleft}{\kern0pt}{\isasymforall}f{\isacharbrackleft}{\kern0pt}M{\isacharbrackright}{\kern0pt}{\isachardot}{\kern0pt}\ f\ {\isasymin}\ fs\ {\isasymlongleftrightarrow}\ typed{\isacharunderscore}{\kern0pt}function{\isacharparenleft}{\kern0pt}M{\isacharcomma}{\kern0pt}\ A{\isacharcomma}{\kern0pt}\ B{\isacharcomma}{\kern0pt}\ f{\isacharparenright}{\kern0pt}{\isacharparenright}{\kern0pt}\isasep\isanewline\isanewline%
A\ {\isasymrightarrow}\isactrlbsup M\isactrlesup \ B\ {\isasymequiv}\ THE\ d{\isachardot}{\kern0pt}\ is{\isacharunderscore}{\kern0pt}function{\isacharunderscore}{\kern0pt}space{\isacharparenleft}{\kern0pt}M{\isacharcomma}{\kern0pt}\ A{\isacharcomma}{\kern0pt}\ B{\isacharcomma}{\kern0pt}\ d{\isacharparenright}{\kern0pt}\isasep\isanewline\isanewline%
surjection{\isacharparenleft}{\kern0pt}M{\isacharcomma}{\kern0pt}\ A{\isacharcomma}{\kern0pt}\ B{\isacharcomma}{\kern0pt}\ f{\isacharparenright}{\kern0pt}\ {\isasymequiv}\isanewline
typed{\isacharunderscore}{\kern0pt}function{\isacharparenleft}{\kern0pt}M{\isacharcomma}{\kern0pt}\ A{\isacharcomma}{\kern0pt}\ B{\isacharcomma}{\kern0pt}\ f{\isacharparenright}{\kern0pt}\ {\isasymand}\isanewline
{\isacharparenleft}{\kern0pt}{\isasymforall}y{\isacharbrackleft}{\kern0pt}M{\isacharbrackright}{\kern0pt}{\isachardot}{\kern0pt}\ y\ {\isasymin}\ B\ {\isasymlongrightarrow}\ {\isacharparenleft}{\kern0pt}{\isasymexists}x{\isacharbrackleft}{\kern0pt}M{\isacharbrackright}{\kern0pt}{\isachardot}{\kern0pt}\ x\ {\isasymin}\ A\ {\isasymand}\ is{\isacharunderscore}{\kern0pt}apply{\isacharparenleft}{\kern0pt}M{\isacharcomma}{\kern0pt}\ f{\isacharcomma}{\kern0pt}\ x{\isacharcomma}{\kern0pt}\ y{\isacharparenright}{\kern0pt}{\isacharparenright}{\kern0pt}{\isacharparenright}{\kern0pt}%
\end{isabelle}%

\subsubsection*{Relative version of the $\ZFC$ axioms}
\begin{isabelle}%
extensionality{\isacharparenleft}{\kern0pt}M{\isacharparenright}{\kern0pt}\ {\isasymequiv}\ {\isasymforall}x{\isacharbrackleft}{\kern0pt}M{\isacharbrackright}{\kern0pt}{\isachardot}{\kern0pt}\ {\isasymforall}y{\isacharbrackleft}{\kern0pt}M{\isacharbrackright}{\kern0pt}{\isachardot}{\kern0pt}\ {\isacharparenleft}{\kern0pt}{\isasymforall}z{\isacharbrackleft}{\kern0pt}M{\isacharbrackright}{\kern0pt}{\isachardot}{\kern0pt}\ z\ {\isasymin}\ x\ {\isasymlongleftrightarrow}\ z\ {\isasymin}\ y{\isacharparenright}{\kern0pt}\ {\isasymlongrightarrow}\ x\ {\isacharequal}{\kern0pt}\ y%
\end{isabelle}%
\begin{isabelle}%
foundation{\isacharunderscore}{\kern0pt}ax{\isacharparenleft}{\kern0pt}M{\isacharparenright}{\kern0pt}\ {\isasymequiv}\isanewline
{\isasymforall}x{\isacharbrackleft}{\kern0pt}M{\isacharbrackright}{\kern0pt}{\isachardot}{\kern0pt}\ {\isacharparenleft}{\kern0pt}{\isasymexists}y{\isacharbrackleft}{\kern0pt}M{\isacharbrackright}{\kern0pt}{\isachardot}{\kern0pt}\ y\ {\isasymin}\ x{\isacharparenright}{\kern0pt}\ {\isasymlongrightarrow}\ {\isacharparenleft}{\kern0pt}{\isasymexists}y{\isacharbrackleft}{\kern0pt}M{\isacharbrackright}{\kern0pt}{\isachardot}{\kern0pt}\ y\ {\isasymin}\ x\ {\isasymand}\ {\isasymnot}\ {\isacharparenleft}{\kern0pt}{\isasymexists}z{\isacharbrackleft}{\kern0pt}M{\isacharbrackright}{\kern0pt}{\isachardot}{\kern0pt}\ z\ {\isasymin}\ x\ {\isasymand}\ z\ {\isasymin}\ y{\isacharparenright}{\kern0pt}{\isacharparenright}{\kern0pt}%
\end{isabelle}%
\begin{isabelle}%
upair{\isacharunderscore}{\kern0pt}ax{\isacharparenleft}{\kern0pt}M{\isacharparenright}{\kern0pt}\ {\isasymequiv}\ {\isasymforall}x{\isacharbrackleft}{\kern0pt}M{\isacharbrackright}{\kern0pt}{\isachardot}{\kern0pt}\ {\isasymforall}y{\isacharbrackleft}{\kern0pt}M{\isacharbrackright}{\kern0pt}{\isachardot}{\kern0pt}\ {\isasymexists}z{\isacharbrackleft}{\kern0pt}M{\isacharbrackright}{\kern0pt}{\isachardot}{\kern0pt}\ upair{\isacharparenleft}{\kern0pt}M{\isacharcomma}{\kern0pt}\ x{\isacharcomma}{\kern0pt}\ y{\isacharcomma}{\kern0pt}\ z{\isacharparenright}{\kern0pt}%
\end{isabelle}%
\begin{isabelle}%
Union{\isacharunderscore}{\kern0pt}ax{\isacharparenleft}{\kern0pt}M{\isacharparenright}{\kern0pt}\ {\isasymequiv}\ {\isasymforall}x{\isacharbrackleft}{\kern0pt}M{\isacharbrackright}{\kern0pt}{\isachardot}{\kern0pt}\ {\isasymexists}z{\isacharbrackleft}{\kern0pt}M{\isacharbrackright}{\kern0pt}{\isachardot}{\kern0pt}\ big{\isacharunderscore}{\kern0pt}union{\isacharparenleft}{\kern0pt}M{\isacharcomma}{\kern0pt}\ x{\isacharcomma}{\kern0pt}\ z{\isacharparenright}{\kern0pt}%
\end{isabelle}%
\begin{isabelle}%
power{\isacharunderscore}{\kern0pt}ax{\isacharparenleft}{\kern0pt}M{\isacharparenright}{\kern0pt}\ {\isasymequiv}\ {\isasymforall}x{\isacharbrackleft}{\kern0pt}M{\isacharbrackright}{\kern0pt}{\isachardot}{\kern0pt}\ {\isasymexists}z{\isacharbrackleft}{\kern0pt}M{\isacharbrackright}{\kern0pt}{\isachardot}{\kern0pt}\ {\isasymforall}xa{\isacharbrackleft}{\kern0pt}M{\isacharbrackright}{\kern0pt}{\isachardot}{\kern0pt}\ xa\ {\isasymin}\ z\ {\isasymlongleftrightarrow}\ {\isacharparenleft}{\kern0pt}{\isasymforall}xb{\isacharbrackleft}{\kern0pt}M{\isacharbrackright}{\kern0pt}{\isachardot}{\kern0pt}\ xb\ {\isasymin}\ xa\ {\isasymlongrightarrow}\ xb\ {\isasymin}\ x{\isacharparenright}{\kern0pt}%
\end{isabelle}%
\begin{isabelle}%
infinity{\isacharunderscore}{\kern0pt}ax{\isacharparenleft}{\kern0pt}M{\isacharparenright}{\kern0pt}\ {\isasymequiv}\isanewline
{\isasymexists}I{\isacharbrackleft}{\kern0pt}M{\isacharbrackright}{\kern0pt}{\isachardot}{\kern0pt}\isanewline
\isaindent{\ \ \ }{\isacharparenleft}{\kern0pt}{\isasymexists}z{\isacharbrackleft}{\kern0pt}M{\isacharbrackright}{\kern0pt}{\isachardot}{\kern0pt}\ empty{\isacharparenleft}{\kern0pt}M{\isacharcomma}{\kern0pt}\ z{\isacharparenright}{\kern0pt}\ {\isasymand}\ z\ {\isasymin}\ I{\isacharparenright}{\kern0pt}\ {\isasymand}\isanewline
\isaindent{\ \ \ }{\isacharparenleft}{\kern0pt}{\isasymforall}y{\isacharbrackleft}{\kern0pt}M{\isacharbrackright}{\kern0pt}{\isachardot}{\kern0pt}\ y\ {\isasymin}\ I\ {\isasymlongrightarrow}\ {\isacharparenleft}{\kern0pt}{\isasymexists}sy{\isacharbrackleft}{\kern0pt}M{\isacharbrackright}{\kern0pt}{\isachardot}{\kern0pt}\ successor{\isacharparenleft}{\kern0pt}M{\isacharcomma}{\kern0pt}\ y{\isacharcomma}{\kern0pt}\ sy{\isacharparenright}{\kern0pt}\ {\isasymand}\ sy\ {\isasymin}\ I{\isacharparenright}{\kern0pt}{\isacharparenright}{\kern0pt}%
\end{isabelle}%
\begin{isabelle}%
choice{\isacharunderscore}{\kern0pt}ax{\isacharparenleft}{\kern0pt}M{\isacharparenright}{\kern0pt}\ {\isasymequiv}\ {\isasymforall}x{\isacharbrackleft}{\kern0pt}M{\isacharbrackright}{\kern0pt}{\isachardot}{\kern0pt}\ {\isasymexists}a{\isacharbrackleft}{\kern0pt}M{\isacharbrackright}{\kern0pt}{\isachardot}{\kern0pt}\ {\isasymexists}f{\isacharbrackleft}{\kern0pt}M{\isacharbrackright}{\kern0pt}{\isachardot}{\kern0pt}\ ordinal{\isacharparenleft}{\kern0pt}M{\isacharcomma}{\kern0pt}\ a{\isacharparenright}{\kern0pt}\ {\isasymand}\ surjection{\isacharparenleft}{\kern0pt}M{\isacharcomma}{\kern0pt}\ a{\isacharcomma}{\kern0pt}\ x{\isacharcomma}{\kern0pt}\ f{\isacharparenright}{\kern0pt}%
\end{isabelle}%
\begin{isabelle}%
separation{\isacharparenleft}{\kern0pt}M{\isacharcomma}{\kern0pt}\ P{\isacharparenright}{\kern0pt}\ {\isasymequiv}\ {\isasymforall}z{\isacharbrackleft}{\kern0pt}M{\isacharbrackright}{\kern0pt}{\isachardot}{\kern0pt}\ {\isasymexists}y{\isacharbrackleft}{\kern0pt}M{\isacharbrackright}{\kern0pt}{\isachardot}{\kern0pt}\ {\isasymforall}x{\isacharbrackleft}{\kern0pt}M{\isacharbrackright}{\kern0pt}{\isachardot}{\kern0pt}\ x\ {\isasymin}\ y\ {\isasymlongleftrightarrow}\ x\ {\isasymin}\ z\ {\isasymand}\ P{\isacharparenleft}{\kern0pt}x{\isacharparenright}{\kern0pt}%
\end{isabelle}%
\begin{isabelle}%
univalent{\isacharparenleft}{\kern0pt}M{\isacharcomma}{\kern0pt}\ A{\isacharcomma}{\kern0pt}\ P{\isacharparenright}{\kern0pt}\ {\isasymequiv}\isanewline
{\isasymforall}x{\isacharbrackleft}{\kern0pt}M{\isacharbrackright}{\kern0pt}{\isachardot}{\kern0pt}\ x\ {\isasymin}\ A\ {\isasymlongrightarrow}\ {\isacharparenleft}{\kern0pt}{\isasymforall}y{\isacharbrackleft}{\kern0pt}M{\isacharbrackright}{\kern0pt}{\isachardot}{\kern0pt}\ {\isasymforall}z{\isacharbrackleft}{\kern0pt}M{\isacharbrackright}{\kern0pt}{\isachardot}{\kern0pt}\ P{\isacharparenleft}{\kern0pt}x{\isacharcomma}{\kern0pt}\ y{\isacharparenright}{\kern0pt}\ {\isasymand}\ P{\isacharparenleft}{\kern0pt}x{\isacharcomma}{\kern0pt}\ z{\isacharparenright}{\kern0pt}\ {\isasymlongrightarrow}\ y\ {\isacharequal}{\kern0pt}\ z{\isacharparenright}{\kern0pt}%
\end{isabelle}%
\begin{isabelle}%
strong{\isacharunderscore}{\kern0pt}replacement{\isacharparenleft}{\kern0pt}M{\isacharcomma}{\kern0pt}\ P{\isacharparenright}{\kern0pt}\ {\isasymequiv}\isanewline
{\isasymforall}A{\isacharbrackleft}{\kern0pt}M{\isacharbrackright}{\kern0pt}{\isachardot}{\kern0pt}\isanewline
\isaindent{\ \ \ }univalent{\isacharparenleft}{\kern0pt}M{\isacharcomma}{\kern0pt}\ A{\isacharcomma}{\kern0pt}\ P{\isacharparenright}{\kern0pt}\ {\isasymlongrightarrow}\ {\isacharparenleft}{\kern0pt}{\isasymexists}Y{\isacharbrackleft}{\kern0pt}M{\isacharbrackright}{\kern0pt}{\isachardot}{\kern0pt}\ {\isasymforall}b{\isacharbrackleft}{\kern0pt}M{\isacharbrackright}{\kern0pt}{\isachardot}{\kern0pt}\ b\ {\isasymin}\ Y\ {\isasymlongleftrightarrow}\ {\isacharparenleft}{\kern0pt}{\isasymexists}x{\isacharbrackleft}{\kern0pt}M{\isacharbrackright}{\kern0pt}{\isachardot}{\kern0pt}\ x\ {\isasymin}\ A\ {\isasymand}\ P{\isacharparenleft}{\kern0pt}x{\isacharcomma}{\kern0pt}\ b{\isacharparenright}{\kern0pt}{\isacharparenright}{\kern0pt}{\isacharparenright}{\kern0pt}%
\end{isabelle}%
\subsubsection*{Internalized formulas}
“Codes” for formulas (as sets) are constructed from natural
numbers using \isa{Member}, \isa{Equal}, \isa{Nand},
and \isa{Forall}.%
\begin{isabelle}%
{\isasymlbrakk}x\ {\isasymin}\ {\isasymomega}{\isacharsemicolon}{\kern0pt}\ y\ {\isasymin}\ {\isasymomega}{\isasymrbrakk}\ {\isasymLongrightarrow}\ {\isasymcdot}x\ {\isasymin}\ y{\isasymcdot}\ {\isasymin}\ formula\isasep\isanewline%
{\isasymlbrakk}x\ {\isasymin}\ {\isasymomega}{\isacharsemicolon}{\kern0pt}\ y\ {\isasymin}\ {\isasymomega}{\isasymrbrakk}\ {\isasymLongrightarrow}\ {\isasymcdot}x\ {\isacharequal}{\kern0pt}\ y{\isasymcdot}\ {\isasymin}\ formula\isasep\isanewline%
{\isasymlbrakk}p\ {\isasymin}\ formula{\isacharsemicolon}{\kern0pt}\ q\ {\isasymin}\ formula{\isasymrbrakk}\ {\isasymLongrightarrow}\ {\isasymcdot}{\isasymnot}{\isacharparenleft}{\kern0pt}p\ {\isasymand}\ q{\isacharparenright}{\kern0pt}{\isasymcdot}\ {\isasymin}\ formula\isasep\isanewline%
p\ {\isasymin}\ formula\ {\isasymLongrightarrow}\ {\isacharparenleft}{\kern0pt}{\isasymcdot}{\isasymforall}p{\isasymcdot}{\isacharparenright}{\kern0pt}\ {\isasymin}\ formula\isasep\isanewline\isanewline%
{\isasymlbrakk}x\ {\isasymin}\ formula{\isacharsemicolon}{\kern0pt}\ {\isasymAnd}x\ y{\isachardot}{\kern0pt}\ {\isasymlbrakk}x\ {\isasymin}\ {\isasymomega}{\isacharsemicolon}{\kern0pt}\ y\ {\isasymin}\ {\isasymomega}{\isasymrbrakk}\ {\isasymLongrightarrow}\ P{\isacharparenleft}{\kern0pt}{\isasymcdot}x\ {\isasymin}\ y{\isasymcdot}{\isacharparenright}{\kern0pt}{\isacharsemicolon}{\kern0pt}\isanewline
\isaindent{\ }{\isasymAnd}x\ y{\isachardot}{\kern0pt}\ {\isasymlbrakk}x\ {\isasymin}\ {\isasymomega}{\isacharsemicolon}{\kern0pt}\ y\ {\isasymin}\ {\isasymomega}{\isasymrbrakk}\ {\isasymLongrightarrow}\ P{\isacharparenleft}{\kern0pt}{\isasymcdot}x\ {\isacharequal}{\kern0pt}\ y{\isasymcdot}{\isacharparenright}{\kern0pt}{\isacharsemicolon}{\kern0pt}\isanewline
\isaindent{\ }{\isasymAnd}p\ q{\isachardot}{\kern0pt}\ {\isasymlbrakk}p\ {\isasymin}\ formula{\isacharsemicolon}{\kern0pt}\ P{\isacharparenleft}{\kern0pt}p{\isacharparenright}{\kern0pt}{\isacharsemicolon}{\kern0pt}\ q\ {\isasymin}\ formula{\isacharsemicolon}{\kern0pt}\ P{\isacharparenleft}{\kern0pt}q{\isacharparenright}{\kern0pt}{\isasymrbrakk}\ {\isasymLongrightarrow}\ P{\isacharparenleft}{\kern0pt}{\isasymcdot}{\isasymnot}{\isacharparenleft}{\kern0pt}p\ {\isasymand}\ q{\isacharparenright}{\kern0pt}{\isasymcdot}{\isacharparenright}{\kern0pt}{\isacharsemicolon}{\kern0pt}\isanewline
\isaindent{\ }{\isasymAnd}p{\isachardot}{\kern0pt}\ {\isasymlbrakk}p\ {\isasymin}\ formula{\isacharsemicolon}{\kern0pt}\ P{\isacharparenleft}{\kern0pt}p{\isacharparenright}{\kern0pt}{\isasymrbrakk}\ {\isasymLongrightarrow}\ P{\isacharparenleft}{\kern0pt}{\isacharparenleft}{\kern0pt}{\isasymcdot}{\isasymforall}p{\isasymcdot}{\isacharparenright}{\kern0pt}{\isacharparenright}{\kern0pt}{\isasymrbrakk}\isanewline
{\isasymLongrightarrow}\ P{\isacharparenleft}{\kern0pt}x{\isacharparenright}{\kern0pt}%
\end{isabelle}%
Definitions for the other connectives and the internal existential
quantifier are also provided. For instance, negation:%
\begin{isabelle}%
{\isasymcdot}{\isasymnot}p{\isasymcdot}\ {\isasymequiv}\ {\isasymcdot}{\isasymnot}{\isacharparenleft}{\kern0pt}p\ {\isasymand}\ p{\isacharparenright}{\kern0pt}{\isasymcdot}%
\end{isabelle}%
The \isa{arity} function strictly bounding the free de Bruijn
indices of a formula is defined below:
\begin{isabelle}%
arity{\isacharparenleft}{\kern0pt}{\isasymcdot}x\ {\isasymin}\ y{\isasymcdot}{\isacharparenright}{\kern0pt}\ {\isacharequal}{\kern0pt}\ succ{\isacharparenleft}{\kern0pt}x{\isacharparenright}{\kern0pt}\ {\isasymunion}\ succ{\isacharparenleft}{\kern0pt}y{\isacharparenright}{\kern0pt}\isasep\isanewline%
arity{\isacharparenleft}{\kern0pt}{\isasymcdot}x\ {\isacharequal}{\kern0pt}\ y{\isasymcdot}{\isacharparenright}{\kern0pt}\ {\isacharequal}{\kern0pt}\ succ{\isacharparenleft}{\kern0pt}x{\isacharparenright}{\kern0pt}\ {\isasymunion}\ succ{\isacharparenleft}{\kern0pt}y{\isacharparenright}{\kern0pt}\isasep\isanewline%
arity{\isacharparenleft}{\kern0pt}{\isasymcdot}{\isasymnot}{\isacharparenleft}{\kern0pt}p\ {\isasymand}\ q{\isacharparenright}{\kern0pt}{\isasymcdot}{\isacharparenright}{\kern0pt}\ {\isacharequal}{\kern0pt}\ arity{\isacharparenleft}{\kern0pt}p{\isacharparenright}{\kern0pt}\ {\isasymunion}\ arity{\isacharparenleft}{\kern0pt}q{\isacharparenright}{\kern0pt}\isasep\isanewline%
arity{\isacharparenleft}{\kern0pt}{\isacharparenleft}{\kern0pt}{\isasymcdot}{\isasymforall}p{\isasymcdot}{\isacharparenright}{\kern0pt}{\isacharparenright}{\kern0pt}\ {\isacharequal}{\kern0pt}\ pred{\isacharparenleft}{\kern0pt}arity{\isacharparenleft}{\kern0pt}p{\isacharparenright}{\kern0pt}{\isacharparenright}{\kern0pt}%
\end{isabelle}%
We have the satisfaction relation between $\in$-models and
    first order formulas (given a “environment” list representing
    the assignment of free variables),%
\begin{isabelle}%
{\isasymlbrakk}nth{\isacharparenleft}{\kern0pt}i{\isacharcomma}{\kern0pt}\ env{\isacharparenright}{\kern0pt}\ {\isacharequal}{\kern0pt}\ x{\isacharsemicolon}{\kern0pt}\ nth{\isacharparenleft}{\kern0pt}j{\isacharcomma}{\kern0pt}\ env{\isacharparenright}{\kern0pt}\ {\isacharequal}{\kern0pt}\ y{\isacharsemicolon}{\kern0pt}\ env\ {\isasymin}\ list{\isacharparenleft}{\kern0pt}A{\isacharparenright}{\kern0pt}{\isasymrbrakk}\isanewline
{\isasymLongrightarrow}\ x\ {\isasymin}\ y\ {\isasymlongleftrightarrow}\ A{\isacharcomma}{\kern0pt}\ env\ {\isasymTurnstile}\ {\isasymcdot}i\ {\isasymin}\ j{\isasymcdot}\isasep\isanewline\isanewline%
{\isasymlbrakk}nth{\isacharparenleft}{\kern0pt}i{\isacharcomma}{\kern0pt}\ env{\isacharparenright}{\kern0pt}\ {\isacharequal}{\kern0pt}\ x{\isacharsemicolon}{\kern0pt}\ nth{\isacharparenleft}{\kern0pt}j{\isacharcomma}{\kern0pt}\ env{\isacharparenright}{\kern0pt}\ {\isacharequal}{\kern0pt}\ y{\isacharsemicolon}{\kern0pt}\ env\ {\isasymin}\ list{\isacharparenleft}{\kern0pt}A{\isacharparenright}{\kern0pt}{\isasymrbrakk}\isanewline
{\isasymLongrightarrow}\ x\ {\isacharequal}{\kern0pt}\ y\ {\isasymlongleftrightarrow}\ A{\isacharcomma}{\kern0pt}\ env\ {\isasymTurnstile}\ {\isasymcdot}i\ {\isacharequal}{\kern0pt}\ j{\isasymcdot}\isasep\isanewline\isanewline%
env\ {\isasymin}\ list{\isacharparenleft}{\kern0pt}A{\isacharparenright}{\kern0pt}\ {\isasymLongrightarrow}\ {\isacharparenleft}{\kern0pt}A{\isacharcomma}{\kern0pt}\ env\ {\isasymTurnstile}\ {\isasymcdot}{\isasymnot}{\isacharparenleft}{\kern0pt}p\ {\isasymand}\ q{\isacharparenright}{\kern0pt}{\isasymcdot}{\isacharparenright}{\kern0pt}\ {\isasymlongleftrightarrow}\ {\isasymnot}\ {\isacharparenleft}{\kern0pt}{\isacharparenleft}{\kern0pt}A{\isacharcomma}{\kern0pt}\ env\ {\isasymTurnstile}\ p{\isacharparenright}{\kern0pt}\ {\isasymand}\isanewline%
\ \ {\isacharparenleft}{\kern0pt}A{\isacharcomma}{\kern0pt}\ env\ {\isasymTurnstile}\ q{\isacharparenright}{\kern0pt}{\isacharparenright}{\kern0pt}\isasep\isanewline\isanewline%
env\ {\isasymin}\ list{\isacharparenleft}{\kern0pt}A{\isacharparenright}{\kern0pt}\ {\isasymLongrightarrow}\ {\isacharparenleft}{\kern0pt}A{\isacharcomma}{\kern0pt}\ env\ {\isasymTurnstile}\ {\isacharparenleft}{\kern0pt}{\isasymcdot}{\isasymforall}p{\isasymcdot}{\isacharparenright}{\kern0pt}{\isacharparenright}{\kern0pt}\ {\isasymlongleftrightarrow}\ {\isacharparenleft}{\kern0pt}{\isasymforall}x{\isasymin}A{\isachardot}{\kern0pt}\ A{\isacharcomma}{\kern0pt}\ Cons{\isacharparenleft}{\kern0pt}x{\isacharcomma}{\kern0pt}\ env{\isacharparenright}{\kern0pt}\ {\isasymTurnstile}\ p{\isacharparenright}{\kern0pt}%
\end{isabelle}%
as well as the satisfaction of an arbitrary set of sentences.%
\begin{isabelle}%
A\ {\isasymTurnstile}\ {\isasymPhi}\ {\isasymequiv}\ {\isasymforall}{\isasymphi}{\isasymin}{\isasymPhi}{\isachardot}{\kern0pt}\ A{\isacharcomma}{\kern0pt}\ {\isacharbrackleft}{\kern0pt}{\isacharbrackright}{\kern0pt}\ {\isasymTurnstile}\ {\isasymphi}%
\end{isabelle}%
The internalized (viz. as elements of the set \isa{formula})
versions of the axioms are checked next against the relative statements.%
\begin{isabelle}%
Union{\isacharunderscore}{\kern0pt}ax{\isacharparenleft}{\kern0pt}{\isacharhash}{\kern0pt}{\isacharhash}{\kern0pt}A{\isacharparenright}{\kern0pt}\ {\isasymlongleftrightarrow}\ A{\isacharcomma}{\kern0pt}\ {\isacharbrackleft}{\kern0pt}{\isacharbrackright}{\kern0pt}\ {\isasymTurnstile}\ {\isasymcdot}Union\ Ax{\isasymcdot}\isasep\isanewline%
power{\isacharunderscore}{\kern0pt}ax{\isacharparenleft}{\kern0pt}{\isacharhash}{\kern0pt}{\isacharhash}{\kern0pt}A{\isacharparenright}{\kern0pt}\ {\isasymlongleftrightarrow}\ A{\isacharcomma}{\kern0pt}\ {\isacharbrackleft}{\kern0pt}{\isacharbrackright}{\kern0pt}\ {\isasymTurnstile}\ {\isasymcdot}Powerset\ Ax{\isasymcdot}\isasep\isanewline%
upair{\isacharunderscore}{\kern0pt}ax{\isacharparenleft}{\kern0pt}{\isacharhash}{\kern0pt}{\isacharhash}{\kern0pt}A{\isacharparenright}{\kern0pt}\ {\isasymlongleftrightarrow}\ A{\isacharcomma}{\kern0pt}\ {\isacharbrackleft}{\kern0pt}{\isacharbrackright}{\kern0pt}\ {\isasymTurnstile}\ {\isasymcdot}Pairing{\isasymcdot}\isasep\isanewline%
foundation{\isacharunderscore}{\kern0pt}ax{\isacharparenleft}{\kern0pt}{\isacharhash}{\kern0pt}{\isacharhash}{\kern0pt}A{\isacharparenright}{\kern0pt}\ {\isasymlongleftrightarrow}\ A{\isacharcomma}{\kern0pt}\ {\isacharbrackleft}{\kern0pt}{\isacharbrackright}{\kern0pt}\ {\isasymTurnstile}\ {\isasymcdot}Foundation{\isasymcdot}\isasep\isanewline%
extensionality{\isacharparenleft}{\kern0pt}{\isacharhash}{\kern0pt}{\isacharhash}{\kern0pt}A{\isacharparenright}{\kern0pt}\ {\isasymlongleftrightarrow}\ A{\isacharcomma}{\kern0pt}\ {\isacharbrackleft}{\kern0pt}{\isacharbrackright}{\kern0pt}\ {\isasymTurnstile}\ {\isasymcdot}Extensionality{\isasymcdot}\isasep\isanewline%
infinity{\isacharunderscore}{\kern0pt}ax{\isacharparenleft}{\kern0pt}{\isacharhash}{\kern0pt}{\isacharhash}{\kern0pt}A{\isacharparenright}{\kern0pt}\ {\isasymlongleftrightarrow}\ A{\isacharcomma}{\kern0pt}\ {\isacharbrackleft}{\kern0pt}{\isacharbrackright}{\kern0pt}\ {\isasymTurnstile}\ {\isasymcdot}Infinity{\isasymcdot}\isasep\isanewline\isanewline%
{\isasymphi}\ {\isasymin}\ formula\ {\isasymLongrightarrow}\isanewline
{\isacharparenleft}{\kern0pt}M{\isacharcomma}{\kern0pt}\ {\isacharbrackleft}{\kern0pt}{\isacharbrackright}{\kern0pt}\ {\isasymTurnstile}\ {\isasymcdot}Separation{\isacharparenleft}{\kern0pt}{\isasymphi}{\isacharparenright}{\kern0pt}{\isasymcdot}{\isacharparenright}{\kern0pt}\ {\isasymlongleftrightarrow}\isanewline
{\isacharparenleft}{\kern0pt}{\isasymforall}env{\isasymin}list{\isacharparenleft}{\kern0pt}M{\isacharparenright}{\kern0pt}{\isachardot}{\kern0pt}\isanewline
\isaindent{{\isacharparenleft}{\kern0pt}\ \ \ }arity{\isacharparenleft}{\kern0pt}{\isasymphi}{\isacharparenright}{\kern0pt}\ {\isasymle}\ {\isadigit{1}}\ {\isacharplus}{\kern0pt}\isactrlsub {\isasymomega}\ length{\isacharparenleft}{\kern0pt}env{\isacharparenright}{\kern0pt}\ {\isasymlongrightarrow}\isanewline
\ \ \ \ separation{\isacharparenleft}{\kern0pt}{\isacharhash}{\kern0pt}{\isacharhash}{\kern0pt}M{\isacharcomma}{\kern0pt}\ {\isasymlambda}x{\isachardot}{\kern0pt}\ M{\isacharcomma}{\kern0pt}\ {\isacharbrackleft}{\kern0pt}x{\isacharbrackright}{\kern0pt}\ {\isacharat}{\kern0pt}\ env\ {\isasymTurnstile}\ {\isasymphi}{\isacharparenright}{\kern0pt}{\isacharparenright}{\kern0pt}\isasep\isanewline\isanewline%
{\isasymphi}\ {\isasymin}\ formula\ {\isasymLongrightarrow}\isanewline
{\isacharparenleft}{\kern0pt}M{\isacharcomma}{\kern0pt}\ {\isacharbrackleft}{\kern0pt}{\isacharbrackright}{\kern0pt}\ {\isasymTurnstile}\ {\isasymcdot}Replacement{\isacharparenleft}{\kern0pt}{\isasymphi}{\isacharparenright}{\kern0pt}{\isasymcdot}{\isacharparenright}{\kern0pt}\ {\isasymlongleftrightarrow}\ {\isacharparenleft}{\kern0pt}{\isasymforall}env{\isachardot}{\kern0pt}\ replacement{\isacharunderscore}{\kern0pt}assm{\isacharparenleft}{\kern0pt}M{\isacharcomma}{\kern0pt}\ env{\isacharcomma}{\kern0pt}\ {\isasymphi}{\isacharparenright}{\kern0pt}{\isacharparenright}\isanewline\isanewline%
choice{\isacharunderscore}{\kern0pt}ax{\isacharparenleft}{\kern0pt}{\isacharhash}{\kern0pt}{\isacharhash}{\kern0pt}A{\isacharparenright}{\kern0pt}\ {\isasymlongleftrightarrow}\ A{\isacharcomma}{\kern0pt}\ {\isacharbrackleft}{\kern0pt}{\isacharbrackright}{\kern0pt}\ {\isasymTurnstile}\ {\isasymcdot}AC{\isasymcdot}%
\end{isabelle}%

Finally, the axiom sets are defined as follows.

\begin{isabelle}%
ZF{\isacharunderscore}{\kern0pt}fin\ {\isasymequiv}\isanewline
{\isacharbraceleft}{\kern0pt}{\isasymcdot}Extensionality{\isasymcdot}{\isacharcomma}{\kern0pt}\ {\isasymcdot}Foundation{\isasymcdot}{\isacharcomma}{\kern0pt}\ {\isasymcdot}Pairing{\isasymcdot}{\isacharcomma}{\kern0pt}\ {\isasymcdot}Union\ Ax{\isasymcdot}{\isacharcomma}{\kern0pt}\ {\isasymcdot}Infinity{\isasymcdot}{\isacharcomma}{\kern0pt}\isanewline
\isaindent{{\isacharbraceleft}{\kern0pt}}{\isasymcdot}Powerset\ Ax{\isasymcdot}{\isacharbraceright}{\kern0pt}\isasep\isanewline\isanewline%
ZF{\isacharunderscore}{\kern0pt}schemes\ {\isasymequiv}\isanewline
{\isacharbraceleft}{\kern0pt}{\isasymcdot}Separation{\isacharparenleft}{\kern0pt}p{\isacharparenright}{\kern0pt}{\isasymcdot}\ {\isachardot}{\kern0pt}\ p\ {\isasymin}\ formula{\isacharbraceright}{\kern0pt}\ {\isasymunion}\ {\isacharbraceleft}{\kern0pt}{\isasymcdot}Replacement{\isacharparenleft}{\kern0pt}p{\isacharparenright}{\kern0pt}{\isasymcdot}\ {\isachardot}{\kern0pt}\ p\ {\isasymin}\ formula{\isacharbraceright}{\kern0pt}\isasep\isanewline\isanewline%
{\isasymcdot}Z{\isasymcdot}\ {\isasymequiv}\ ZF{\isacharunderscore}{\kern0pt}fin\ {\isasymunion}\ {\isacharbraceleft}{\kern0pt}{\isasymcdot}Separation{\isacharparenleft}{\kern0pt}p{\isacharparenright}{\kern0pt}{\isasymcdot}\ {\isachardot}{\kern0pt}\ p\ {\isasymin}\ formula{\isacharbraceright}{\kern0pt}\isasep\isanewline%
ZC\ {\isasymequiv}\ {\isasymcdot}Z{\isasymcdot}\ {\isasymunion}\ {\isacharbraceleft}{\kern0pt}{\isasymcdot}AC{\isasymcdot}{\isacharbraceright}{\kern0pt}\isasep\isanewline%
ZF\ {\isasymequiv}\ ZF{\isacharunderscore}{\kern0pt}schemes\ {\isasymunion}\ ZF{\isacharunderscore}{\kern0pt}fin\isasep\isanewline%
ZFC\ {\isasymequiv}\ ZF\ {\isasymunion}\ {\isacharbraceleft}{\kern0pt}{\isasymcdot}AC{\isasymcdot}{\isacharbraceright}{\kern0pt}%
\end{isabelle}%

\subsection{Relativization of infinitary arithmetic\label{sec:relative-arith}%
}
In order to state the defining property of the relative
equipotence relation, we work under the assumptions of the
locale \isa{M{\isacharunderscore}{\kern0pt}cardinals}. They comprise a finite set
of instances of Separation and Replacement to prove
closure properties of the transitive class \isa{M}.%
\begin{isabelle}
\isacommand{lemma}\isamarkupfalse%
\ {\isacharparenleft}{\kern0pt}\isakeyword{in}\ M{\isacharunderscore}{\kern0pt}cardinals{\isacharparenright}{\kern0pt}\ eqpoll{\isacharunderscore}{\kern0pt}def{\isacharprime}{\kern0pt}{\isacharcolon}{\kern0pt}\isanewline
\ \ \isakeyword{assumes}\ {\isachardoublequoteopen}M{\isacharparenleft}{\kern0pt}A{\isacharparenright}{\kern0pt}{\isachardoublequoteclose}\ {\isachardoublequoteopen}M{\isacharparenleft}{\kern0pt}B{\isacharparenright}{\kern0pt}{\isachardoublequoteclose}\ \isakeyword{shows}\ {\isachardoublequoteopen}A\ {\isasymapprox}\isactrlbsup M\isactrlesup \ B\ {\isasymlongleftrightarrow}\ {\isacharparenleft}{\kern0pt}{\isasymexists}f{\isacharbrackleft}{\kern0pt}M{\isacharbrackright}{\kern0pt}{\isachardot}{\kern0pt}\ f\ {\isasymin}\ bij{\isacharparenleft}{\kern0pt}A{\isacharcomma}{\kern0pt}B{\isacharparenright}{\kern0pt}{\isacharparenright}{\kern0pt}{\isachardoublequoteclose}
\end{isabelle}

%
Below, $\mu$ denotes the minimum operator on the ordinals.%
\begin{isabelle}
  \isacommand{lemma}\isamarkupfalse%
\ cardinalities{\isacharunderscore}{\kern0pt}defs{\isacharcolon}{\kern0pt}\isanewline
\ \ \isakeyword{fixes}\ M{\isacharcolon}{\kern0pt}{\isacharcolon}{\kern0pt}{\isachardoublequoteopen}i{\isasymRightarrow}o{\isachardoublequoteclose}\isanewline
\ \ \isakeyword{shows}\isanewline
\ \ \ \ {\isachardoublequoteopen}{\isacharbar}{\kern0pt}A{\isacharbar}{\kern0pt}\isactrlbsup M\isactrlesup \ {\isasymequiv}\ {\isasymmu}\ i{\isachardot}{\kern0pt}\ M{\isacharparenleft}{\kern0pt}i{\isacharparenright}{\kern0pt}\ {\isasymand}\ i\ {\isasymapprox}\isactrlbsup M\isactrlesup \ A{\isachardoublequoteclose}\isanewline
\ \ \ \ {\isachardoublequoteopen}Card\isactrlbsup M\isactrlesup {\isacharparenleft}{\kern0pt}{\isasymalpha}{\isacharparenright}{\kern0pt}\ {\isasymequiv}\ {\isasymalpha}\ {\isacharequal}{\kern0pt}\ {\isacharbar}{\kern0pt}{\isasymalpha}{\isacharbar}{\kern0pt}\isactrlbsup M\isactrlesup {\isachardoublequoteclose}\isanewline
\ \ \ \ {\isachardoublequoteopen}{\isasymkappa}\isactrlbsup {\isasymup}{\isasymnu}{\isacharcomma}{\kern0pt}M\isactrlesup \ {\isasymequiv}\ {\isacharbar}{\kern0pt}{\isasymnu}\ {\isasymrightarrow}\isactrlbsup M\isactrlesup \ {\isasymkappa}{\isacharbar}{\kern0pt}\isactrlbsup M\isactrlesup {\isachardoublequoteclose}\isanewline
\ \ \ \ {\isachardoublequoteopen}{\isacharparenleft}{\kern0pt}{\isasymkappa}\isactrlsup {\isacharplus}{\kern0pt}{\isacharparenright}{\kern0pt}\isactrlbsup M\isactrlesup \ {\isasymequiv}\ {\isasymmu}\ x{\isachardot}{\kern0pt}\ M{\isacharparenleft}{\kern0pt}x{\isacharparenright}{\kern0pt}\ {\isasymand}\ Card\isactrlbsup M\isactrlesup {\isacharparenleft}{\kern0pt}x{\isacharparenright}{\kern0pt}\ {\isasymand}\ {\isasymkappa}\ {\isacharless}{\kern0pt}\ x{\isachardoublequoteclose}
\end{isabelle}
Analogous to the previous Lemma
\isa{eqpoll{\isacharunderscore}{\kern0pt}def{\isacharprime}{\kern0pt}},
the next lemma holds under
the assumptions of the locale \isa{M{\isacharunderscore}{\kern0pt}aleph}. The axiom instances
included are sufficient to state and prove the defining
properties of the relativized \isa{Aleph} function
(in particular, the required ability to perform transfinite recursions).%
\begin{isabelle}%
\isacommand{context}\isamarkupfalse%
\ M{\isacharunderscore}{\kern0pt}aleph\isanewline
\isakeyword{begin}%
\isanewline
\isanewline
{\isasymaleph}\isactrlbsub {\isadigit{0}}\isactrlesub \isactrlbsup M\isactrlesup \ {\isacharequal}{\kern0pt}\ {\isasymomega}\isasep\isanewline%
{\isasymlbrakk}Ord{\isacharparenleft}{\kern0pt}{\isasymalpha}{\isacharparenright}{\kern0pt}{\isacharsemicolon}{\kern0pt}\ M{\isacharparenleft}{\kern0pt}{\isasymalpha}{\isacharparenright}{\kern0pt}{\isasymrbrakk}\ {\isasymLongrightarrow}\ {\isasymaleph}\isactrlbsub succ{\isacharparenleft}{\kern0pt}{\isasymalpha}{\isacharparenright}{\kern0pt}\isactrlesub \isactrlbsup M\isactrlesup \ {\isacharequal}{\kern0pt}\ {\isacharparenleft}{\kern0pt}{\isasymaleph}\isactrlbsub {\isasymalpha}\isactrlesub \isactrlbsup M\isactrlesup \isactrlsup {\isacharplus}{\kern0pt}{\isacharparenright}{\kern0pt}\isactrlbsup M\isactrlesup \isasep\isanewline%
{\isasymlbrakk}Limit{\isacharparenleft}{\kern0pt}{\isasymalpha}{\isacharparenright}{\kern0pt}{\isacharsemicolon}{\kern0pt}\ M{\isacharparenleft}{\kern0pt}{\isasymalpha}{\isacharparenright}{\kern0pt}{\isasymrbrakk}\ {\isasymLongrightarrow}\ {\isasymaleph}\isactrlbsub {\isasymalpha}\isactrlesub \isactrlbsup M\isactrlesup \ {\isacharequal}{\kern0pt}\ {\isacharparenleft}{\kern0pt}{\isasymUnion}j{\isasymin}{\isasymalpha}{\isachardot}{\kern0pt}\ {\isasymaleph}\isactrlbsub j\isactrlesub \isactrlbsup M\isactrlesup {\isacharparenright}{\kern0pt}%
\end{isabelle}%
\isacommand{end}\isamarkupfalse%
\ %
\isamarkupcmt{\isa{M{\isacharunderscore}{\kern0pt}aleph}%
}
\begin{isabelle}
\isacommand{lemma}\isamarkupfalse%
\ ContHyp{\isacharunderscore}{\kern0pt}rel{\isacharunderscore}{\kern0pt}def{\isacharprime}{\kern0pt}{\isacharcolon}{\kern0pt}\isanewline
\ \ \isakeyword{fixes}\ N{\isacharcolon}{\kern0pt}{\isacharcolon}{\kern0pt}{\isachardoublequoteopen}i{\isasymRightarrow}o{\isachardoublequoteclose}\isanewline
\ \ \isakeyword{shows}\isanewline
\ \ \ \ {\isachardoublequoteopen}CH\isactrlbsup N\isactrlesup \ {\isasymequiv}\ {\isasymaleph}\isactrlbsub {\isadigit{1}}\isactrlesub \isactrlbsup N\isactrlesup \ {\isacharequal}{\kern0pt}\ {\isadigit{2}}\isactrlbsup {\isasymup}{\isasymaleph}\isactrlbsub {\isadigit{0}}\isactrlesub \isactrlbsup N\isactrlesup {\isacharcomma}{\kern0pt}N\isactrlesup {\isachardoublequoteclose}
\end{isabelle}

%
%
Under appropriate hypotheses (this time, from the locale \isa{M{\isacharunderscore}{\kern0pt}ZF{\isacharunderscore}{\kern0pt}library}),
   \isa{CH\isactrlbsup M\isactrlesup } is equivalent to its fully relational version \isa{is{\isacharunderscore}{\kern0pt}ContHyp}.
    As a sanity check, we see that if the transitive class is indeed \isa{{\isasymV}},
    we recover the original $\CH$.%
\begin{isabelle}%
M{\isacharunderscore}{\kern0pt}ZF{\isacharunderscore}{\kern0pt}library{\isacharparenleft}{\kern0pt}M{\isacharparenright}{\kern0pt}\ {\isasymLongrightarrow}\ is{\isacharunderscore}{\kern0pt}ContHyp{\isacharparenleft}{\kern0pt}M{\isacharparenright}{\kern0pt}\ {\isasymlongleftrightarrow}\ CH\isactrlbsup M\isactrlesup \isasep\isanewline%
is{\isacharunderscore}{\kern0pt}ContHyp{\isacharparenleft}{\kern0pt}{\isasymV}{\isacharparenright}{\kern0pt}\ {\isasymlongleftrightarrow}\ {\isasymaleph}\isactrlbsub {\isadigit{1}}\isactrlesub \ {\isacharequal}{\kern0pt}\ {\isadigit{2}}\isactrlbsup {\isasymup}{\isasymaleph}\isactrlbsub {\isadigit{0}}\isactrlesub \isactrlesup %
\end{isabelle}%
In turn, the fully relational version evaluated on a nonempty
transitive \isa{A} is equivalent to the satisfaction of the
first-order formula \isa{{\isasymcdot}CH{\isasymcdot}} (since it
actually is a sentence, it does not depend on \isa{env}, which
appears only because the definition of $\models$ requires that argument).%
\begin{isabelle}%
{\isasymlbrakk}env\ {\isasymin}\ list{\isacharparenleft}{\kern0pt}A{\isacharparenright}{\kern0pt}{\isacharsemicolon}{\kern0pt}\ {\isadigit{0}}\ {\isasymin}\ A{\isasymrbrakk}\ {\isasymLongrightarrow}\ is{\isacharunderscore}{\kern0pt}ContHyp{\isacharparenleft}{\kern0pt}{\isacharhash}{\kern0pt}{\isacharhash}{\kern0pt}A{\isacharparenright}{\kern0pt}\ {\isasymlongleftrightarrow}\ A{\isacharcomma}{\kern0pt}\ env\ {\isasymTurnstile}\ {\isasymcdot}CH{\isasymcdot}%
\end{isabelle}%
%% \subsection{Forcing \label{sec:def-main-forcing}%
%% }
%% Our first milestone was to obtain a proper extension using forcing.
%% Its original proof didn't required the previous developments involving
%% below.%
%% \begin{isabelle}%
%% {\isasymlbrakk}M\ {\isasymapprox}\ {\isasymomega}{\isacharsemicolon}{\kern0pt}\ Transset{\isacharparenleft}{\kern0pt}M{\isacharparenright}{\kern0pt}{\isacharsemicolon}{\kern0pt}\ M\ {\isasymTurnstile}\ ZF{\isasymrbrakk}\isanewline
%% {\isasymLongrightarrow}\ {\isasymexists}N{\isachardot}{\kern0pt}\ M\ {\isasymsubseteq}\ N\ {\isasymand}\isanewline
%% \isaindent{{\isasymLongrightarrow}\ {\isasymexists}N{\isachardot}{\kern0pt}\ }N\ {\isasymapprox}\ {\isasymomega}\ {\isasymand}\isanewline
%% \isaindent{{\isasymLongrightarrow}\ {\isasymexists}N{\isachardot}{\kern0pt}\ }Transset{\isacharparenleft}{\kern0pt}N{\isacharparenright}{\kern0pt}\ {\isasymand}\isanewline
%% \isaindent{{\isasymLongrightarrow}\ {\isasymexists}N{\isachardot}{\kern0pt}\ }N\ {\isasymTurnstile}\ ZF\ {\isasymand}\isanewline
%% \isaindent{{\isasymLongrightarrow}\ {\isasymexists}N{\isachardot}{\kern0pt}\ }M\ {\isasymnoteq}\ N\ {\isasymand}\ {\isacharparenleft}{\kern0pt}{\isasymforall}{\isasymalpha}{\isachardot}{\kern0pt}\ Ord{\isacharparenleft}{\kern0pt}{\isasymalpha}{\isacharparenright}{\kern0pt}\ {\isasymlongrightarrow}\ {\isasymalpha}\ {\isasymin}\ M\ {\isasymlongleftrightarrow}\ {\isasymalpha}\ {\isasymin}\ N{\isacharparenright}{\kern0pt}\ {\isasymand}\ {\isacharparenleft}{\kern0pt}{\isacharparenleft}{\kern0pt}M{\isacharcomma}{\kern0pt}\ {\isacharbrackleft}{\kern0pt}{\isacharbrackright}{\kern0pt}\ {\isasymTurnstile}\ {\isasymcdot}AC{\isasymcdot}{\isacharparenright}{\kern0pt}\ {\isasymlongrightarrow}\ N\ {\isasymTurnstile}\ ZFC{\isacharparenright}{\kern0pt}%
%% \end{isabelle}%
%% We can finally state our main results, namely, the existence of models
%% for $\ZFC + \CH$ and $\ZFC + \neg\CH$ under the assumption of a ctm of $\ZFC$.%
%% \begin{isabelle}%
%% {\isasymlbrakk}M\ {\isasymapprox}\ {\isasymomega}{\isacharsemicolon}{\kern0pt}\ Transset{\isacharparenleft}{\kern0pt}M{\isacharparenright}{\kern0pt}{\isacharsemicolon}{\kern0pt}\ M\ {\isasymTurnstile}\ ZFC{\isasymrbrakk}\isanewline
%% {\isasymLongrightarrow}\ {\isasymexists}N{\isachardot}{\kern0pt}\ M\ {\isasymsubseteq}\ N\ {\isasymand}\isanewline
%% \isaindent{{\isasymLongrightarrow}\ {\isasymexists}N{\isachardot}{\kern0pt}\ }N\ {\isasymapprox}\ {\isasymomega}\ {\isasymand}\isanewline
%% \isaindent{{\isasymLongrightarrow}\ {\isasymexists}N{\isachardot}{\kern0pt}\ }Transset{\isacharparenleft}{\kern0pt}N{\isacharparenright}{\kern0pt}\ {\isasymand}\ N\ {\isasymTurnstile}\ ZFC\ {\isasymunion}\ {\isacharbraceleft}{\kern0pt}{\isasymcdot}{\isasymnot}{\isasymcdot}CH{\isasymcdot}{\isasymcdot}{\isacharbraceright}{\kern0pt}\ {\isasymand}\ {\isacharparenleft}{\kern0pt}{\isasymforall}{\isasymalpha}{\isachardot}{\kern0pt}\ Ord{\isacharparenleft}{\kern0pt}{\isasymalpha}{\isacharparenright}{\kern0pt}\ {\isasymlongrightarrow}\ {\isasymalpha}\ {\isasymin}\ M\ {\isasymlongleftrightarrow}\ {\isasymalpha}\ {\isasymin}\ N{\isacharparenright}{\kern0pt}%
%% \end{isabelle}%
%% \begin{isabelle}%
%% {\isasymlbrakk}M\ {\isasymapprox}\ {\isasymomega}{\isacharsemicolon}{\kern0pt}\ Transset{\isacharparenleft}{\kern0pt}M{\isacharparenright}{\kern0pt}{\isacharsemicolon}{\kern0pt}\ M\ {\isasymTurnstile}\ ZFC{\isasymrbrakk}\isanewline
%% {\isasymLongrightarrow}\ {\isasymexists}N{\isachardot}{\kern0pt}\ M\ {\isasymsubseteq}\ N\ {\isasymand}\isanewline
%% \isaindent{{\isasymLongrightarrow}\ {\isasymexists}N{\isachardot}{\kern0pt}\ }N\ {\isasymapprox}\ {\isasymomega}\ {\isasymand}\isanewline
%% \isaindent{{\isasymLongrightarrow}\ {\isasymexists}N{\isachardot}{\kern0pt}\ }Transset{\isacharparenleft}{\kern0pt}N{\isacharparenright}{\kern0pt}\ {\isasymand}\ N\ {\isasymTurnstile}\ ZFC\ {\isasymunion}\ {\isacharbraceleft}{\kern0pt}{\isasymcdot}CH{\isasymcdot}{\isacharbraceright}{\kern0pt}\ {\isasymand}\ {\isacharparenleft}{\kern0pt}{\isasymforall}{\isasymalpha}{\isachardot}{\kern0pt}\ Ord{\isacharparenleft}{\kern0pt}{\isasymalpha}{\isacharparenright}{\kern0pt}\ {\isasymlongrightarrow}\ {\isasymalpha}\ {\isasymin}\ M\ {\isasymlongleftrightarrow}\ {\isasymalpha}\ {\isasymin}\ N{\isacharparenright}{\kern0pt}%
%% \end{isabelle}%
%% In the above three statements, the function \isa{ground{\isacharunderscore}{\kern0pt}repl{\isacharunderscore}{\kern0pt}fm}
%% takes an element \isa{{\isasymphi}} of \isa{formula} and returns the
%% replacement instance in the ground model that produces the
%% \isa{{\isasymphi}}-replacement instance in the generic extension. The next
%% result is stated in the context \isa{G{\isacharunderscore}{\kern0pt}generic{\isadigit{1}}}, which assumes
%% the existence of a generic filter.
%% %
%% \begin{isabelle}%
%% \isacommand{context}\isamarkupfalse%
%% \ G{\isacharunderscore}{\kern0pt}generic{\isadigit{1}}\isanewline
%% \isakeyword{begin}\isanewline
%% \isanewline
%% {\isasymlbrakk}{\isasymphi}\ {\isasymin}\ formula{\isacharsemicolon}{\kern0pt}\ M{\isacharcomma}{\kern0pt}\ {\isacharbrackleft}{\kern0pt}{\isacharbrackright}{\kern0pt}\ {\isasymTurnstile}\ {\isasymcdot}Replacement{\isacharparenleft}{\kern0pt}ground{\isacharunderscore}{\kern0pt}repl{\isacharunderscore}{\kern0pt}fm{\isacharparenleft}{\kern0pt}{\isasymphi}{\isacharparenright}{\kern0pt}{\isacharparenright}{\kern0pt}{\isasymcdot}{\isasymrbrakk}\isanewline
%% {\isasymLongrightarrow}\ M{\isacharbrackleft}{\kern0pt}G{\isacharbrackright}{\kern0pt}{\isacharcomma}{\kern0pt}\ {\isacharbrackleft}{\kern0pt}{\isacharbrackright}{\kern0pt}\ {\isasymTurnstile}\ {\isasymcdot}Replacement{\isacharparenleft}{\kern0pt}{\isasymphi}{\isacharparenright}{\kern0pt}{\isasymcdot}%
%% \end{isabelle}%
%% \isacommand{end}\isamarkupfalse%
%% \ %
%% \isamarkupcmt{\isa{G{\isacharunderscore}{\kern0pt}generic{\isadigit{1}}}%
%% }

%%% Local Variables:
%%% mode: latex
%%% TeX-master: "independence_ch_isabelle"
%%% ispell-local-dictionary: "american"
%%% End:


\section{Discipline for relativization}
\label{sec:discipline-relativization}

As we said in Sec.~\ref{sec:relat-vers-non-absol}, in order to force
$\CH$ and its negation we depended on having relativized versions of
cardinals, Alephs, etc. It was clear for us that our efforts would be
more efficient if we set up a discipline for relativizing sets (terms
of type $\tyi$) and predicates/relations (terms of type $\tyo$).

Paulson only had, for each set, the relational version. It seemed
clearer to us to have a functional version of the relativized concept.
Going back to our example in \ref{sec:tools-relativization}, for the
concept $\isa{cardinal}::\tyi \fun \tyi$ we want its relative
version
$\isa{cardinal{\uscore}rel}::(\tyi \fun \tyo) \fun \tyi \fun\tyi$
and the relational version of the latter
$\isa{is{\uscore}cardinal}::(\tyi \fun \tyo) \fun \tyi \fun \tyi
\fun \tyo$.

Our first attempt of defining a discipline was inspired by
mathematical considerations: if we might prove that
$\isa{is{\uscore}cardinal}$ is functional and also prove the
existence of a witness $\isa{c}$ such that $\isa{M(c)}$ and
$\isa{is{\uscore}cardinal(M,x,c)}$ then
$\isa{cardinal{\uscore}rel(M,x)}$ can be obtained by the operator of
definite descriptions.

Soon we realized that resorting to definite descriptions was needed
only for the most primitive concepts. In fact, once we have a
relativized concept, we can use it to define other relativizations.
For instance, $\isa{cardinal{\uscore}rel}$ depends on having
relative versions of $\isa{bij}$. Instead of relationalizing
$\isa{bij}$ to get $\isa{is{\uscore}bij}$ and then prove
uniqueness and existence of a witness, we define
$\isa{bij{\uscore}rel}$ using $\isa{inj{\uscore}rel}$ and
$\isa{surj{\uscore}rel}$.

%%% Local Variables: 
%%% mode: latex
%%% TeX-master: "independence_ch_isabelle"
%%% ispell-local-dictionary: "american"
%%% End: 


\section{Recursions in cofinality and the Delta System Lemma}\label{sec:recursions-cofinality}

As we mentioned near the end of
Section~\ref{sec:aims-formalization-planning}, we decided to minimize
the requirements being formalized in order to achieve our immediate
goal. In particular, the treatment of cofinality in the companion
project \cite{Delta_System_Lemma-AFP} was left behind.

We already observed that well-founded, and in particular transfinite,
recursion is not easily dealt with in Isabelle/ZF. Nevertheless, and
mainly as a curiosity, we found out that only one recursive
construction is needed for the development of the basic theory of
cofinality (as in \cite[Sect.~I.13]{kunen2011set}), which is used in
the proof of the following “factorization” lemma:

\begin{lemma}
  Let $\del,\ga\in\Ord$ and assume $f:\del\to\ga$ is cofinal.  There exists
  a strictly increasing $g:\cf(\ga)\to \del$ such that $f\circ g$ is
  strictly increasing and cofinal in $\ga$. Moreover, if $f$ is
  strictly increasing, then $g$ must also be cofinal.
\end{lemma}

It turns out that the rest of the basic results on cofinality (namely,
idempotence of $\cf$, that regular ordinals are cardinals, the
cofinality of Alephs, König's Theorem) follow easily from the previous
Lemma by “algebraic” reasoning only.
We expect the relativization of these
results to be straightforward.

The AFP entry \cite{Delta_System_Lemma-AFP} also includes the
formalization of the (absolute) Delta System Lemma (DSL). Formalizing its
proof was rather straightforward, once the many prerequisites were
taken care of. Some of those were really basic, for instance:
\begin{enumerate}
\item \label{item:1}$\omega$ injects into every infinite set;
\item \label{item:2}surjective images of countable sets are countable;
\item \label{item:3}the union over a countable index set $J$ of a family $X :: \tyi
  \fun \tyi$ of countable sets is countable.
\end{enumerate}
It was also convenient to isolate the relevant recursive construction
principle (\isatt{bounded{\uscore}cardinal{\uscore}selection}) that
appears in the proof of DSL, which was also useful for showing
Item~\ref{item:1}:
\begin{lemma}\label{lem:bdd-card-selection}
  Assume $F$ is nonempty,
  $\gamma$ is a cardinal, and $Q$ is a binary predicate over $F$ satisfying 
  \[
    \forall Y \sbq F.\ |Y| < \gamma \implies \exists a\in F.\ \forall
    y\in Y.\ Q(y,a).
  \]
  Then there exists  $S:\gamma \to F$ such that
  $Q(S(\alpha),S(\beta))$ for all $\alpha<\beta<\gamma$.
\end{lemma}

Concerning the relativization of the proof of DSL for its use in
forcing, it required inserting proofs that all the relevant objects
lied in the model $M$; this was only tedious.
A bit more effort was required at the point where Item~\ref{item:3}
was used, because it involved an application of Replacement (being $X$
a class function); relativizing
Lemma~\ref{lem:bdd-card-selection} also required some work, because of the
recursion.


%%% Local Variables:
%%% mode: latex
%%% TeX-master: "independence_ch_isabelle"
%%% ispell-local-dictionary: "american"
%%% End:


\section{Axioms of Isabelle/ZF}
\label{appendix:axioms}

In this appendix we list the complete set of axioms of Isabelle's
metatheory and logic.

\subsection{The metatheory Pure}
\begin{isabelle}
Pure.abstract\_rule: (⋀x. ?f(x) ≡ ?g(x)) ⟹ λx. ?f(x) ≡ λx. ?g(x)\isanewline
Pure.combination: ?f ≡ ?g ⟹ ?x ≡ ?y ⟹ ?f(?x) ≡ ?g(?y)\isanewline
Pure.equal\_elim: PROP ?A ≡ PROP ?B ⟹ PROP ?A ⟹ PROP ?B\isanewline
Pure.equal\_intr: (PROP ?A ⟹ PROP ?B) ⟹ (PROP ?B ⟹ PROP ?A) ⟹ \isanewline
\ \ \ \ \ \ \ \ \ \ PROP ?A ≡ PROP ?B\isanewline
Pure.reflexive: ?x ≡ ?x\isanewline
Pure.symmetric: ?x ≡ ?y ⟹ ?y ≡ ?x\isanewline
Pure.transitive: ?x ≡ ?y ⟹ ?y ≡ ?z ⟹ ?x ≡ ?z
\end{isabelle}

\subsection{IFOL and FOL}
In the axioms \isa{refl, subst, allI, spec, exE, exI,eq\_reflection} there is a constraint
for the \emph{type of} the variables \isa{a, b, x} to be in the class \isa{term\_class}.
\begin{isabelle}
IFOL.FalseE: ⋀P. False ⟹ P\isanewline
IFOL.refl:  (⋀a. a = a)\isanewline
IFOL.subst: (⋀a b P. a = b ⟹ P(a) ⟹ P(b))\isanewline
IFOL.allI:  (⋀P. (⋀x. P(x)) ⟹ ∀x. P(x))\isanewline
IFOL.spec:  (⋀P x. ∀x. P(x) ⟹ P(x))\isanewline
IFOL.exE:   (⋀P R. ∃x. P(x) ⟹ (⋀x. P(x) ⟹ R) ⟹ R)\isanewline
IFOL.exI:   (⋀P x. P(x) ⟹ ∃x. P(x))\isanewline
IFOL.conjI: ⋀P Q. P ⟹ Q ⟹ P ∧ Q\isanewline
IFOL.conjunct1: ⋀P Q. P ∧ Q ⟹ P\isanewline
IFOL.conjunct2: ⋀P Q. P ∧ Q ⟹ Q\isanewline
IFOL.disjE: ⋀P Q R. P ∨ Q ⟹ (P ⟹ R) ⟹ (Q ⟹ R) ⟹ R\isanewline
IFOL.disjI1: ⋀P Q. P ⟹ P ∨ Q\isanewline
IFOL.disjI2: ⋀P Q. Q ⟹ P ∨ Q\isanewline
IFOL.eq\_reflection: (⋀x y. x = y ⟹ x ≡ y)\isanewline
IFOL.iff\_reflection: ⋀P Q. P ⟷ Q ⟹ P ≡ Q\isanewline
IFOL.impI: ⋀P Q. (P ⟹ Q) ⟹ P ⟶ Q\isanewline
IFOL.mp: ⋀P Q. P ⟶ Q ⟹ P ⟹ Q\isanewline%\isanewline
FOL.classical: ⋀P. (¬ P ⟹ P) ⟹ P
\end{isabelle}

\subsection{ZF\_Base}
The following symbols are introduced in this theory:
\begin{isabelle}
axiomatization\isanewline
\ \ \ \      mem :: "[i, i] ⇒ o"  (infixl ‹∈› 50)  \isamarkupcmt{membership relation}\isanewline
  and zero :: "i"  (‹0›)  \isamarkupcmt{the empty set}\isanewline
  and Pow :: "i ⇒ i"  \isamarkupcmt{power sets}\isanewline
  and Inf :: "i"  \isamarkupcmt{infinite set}\isanewline
  and Union :: "i ⇒ i"  (‹⋃\_› [90] 90)\isanewline
  and PrimReplace :: "[i, [i, i] ⇒ o] ⇒ i"
\end{isabelle}
\noindent After the definitions of $\notin$, $\subseteq$, $\isa{succ}$,
and relative quantifications are presented, the following axioms are postulated:
\begin{isabelle}
ZF\_Base.Pow\_iff: ⋀A B. A ∈ Pow(B) ⟷ A ⊆ B\isanewline
ZF\_Base.Union\_iff: ⋀A C. A ∈ ⋃C ⟷ (∃B∈C. A ∈ B)\isanewline
ZF\_Base.extension: ⋀A B. A = B ⟷ A ⊆ B ∧ B ⊆ A\isanewline
ZF\_Base.foundation: ⋀A. A = 0 ∨ (∃x∈A. ∀y∈x. y ∉ A)\isanewline
ZF\_Base.infinity: 0 ∈ Inf ∧ (∀y∈Inf. succ(y) ∈ Inf)\isanewline
ZF\_Base.replacement: ⋀A P b. ∀x∈A. ∀y z. P(x, y) ∧ P(x, z) ⟶ y = z \isanewline
\ \ \ \ \ \ \ \ \ ⟹ b ∈ PrimReplace(A, P) ⟷ (∃x∈A. P(x, b))
\end{isabelle}

\subsection{AC}

The theory \theory{AC} is only imported in the theory
\theory{Absolute\_Versions}; none of the main results
depends on $\AC$. The latter theory
shows that some absolute results can be obtained from the
relativized versions on $\mathcal{V}$.

\begin{isabelle}
AC.AC: ⋀a A B. a ∈ A ⟹ (⋀x. x ∈ A ⟹ ∃y. y ∈ B(x)) ⟹\isanewline
  \ \ \ \ \ \ ∃z. z ∈ Pi(A, B)
\end{isabelle}

%%% Local Variables:
%%% mode: latex
%%% TeX-master: "independence_ch_isabelle"
%%% ispell-local-dictionary: "american"
%%% End:


\section{Lambda replacements}\label{sec:lambda-replacements}

The development of the locale structure of the project was a dynamical
process. As further properties of closure of the ground $M$ were
required, we gathered the relevant instances of Separation and
Replacement into a new locale (always assuming a class model, for
added generality), and proceded to apply them to those closure proofs.

This procedure lead to a steady grow in the number of interpretation
obligations and therefore, of formula synthesis (since the two axiom
schemes were postulated using codes for formulas). That number would
easily surpass the hundred, and the automatic tools at our disposal
for that task were rudimentary (as discussed in
Section~\ref{sec:bureaucracy-scale-factors}).

Faced with this situation, we decided that we needed some sort of
\emph{compositionality} in order to obtain new instances from the ones
already proved: Having Replacement for class functions $F$ and $G$
does not entail immediately replacement under $F\circ G$ (unless you
use one further instance of Separation, and the net gain is zero). The
solution was to postulate for the relevant $F$s, instead of
replacement through $x\mapsto F(x)$, a \emph{lambda replacement}
through $x\mapsto \lb x,F(x)\rb$. The name “lambda” corresponds to the
fact that this type of replacement is equivalent to closure under
$(\lambda x\in A.\ F(x)) \defi \{ \lb x,F(x)\rb : x\in A \}$ for every
$A\in M$.

Now, a fixed set of six replacements and one separation (apart from
those in \locale{M{\uscore}basic}, which also assumes the Powerset
Axiom for the class $M$) is sufficient to obtain the lambda
replacement under $x\mapsto \lb x,F(G(x))\rb$ given those for $F$ and
$G$. To obtain compositions with binary class functions $H$, it is
enough to assume the lambda replacement
$x\mapsto \lb x,H(\mathit{fst}(x),\mathit{snd}(x)))\rb$. We summarize
the assumptions in Table~\ref{tab:m-repl-instances}.

\newcommand{\lamRepl}[2][x]{#1 \mapsto\langle #1,#2\rangle}
\begin{table}[!h]
\centering
\begin{threeparttable}
\begin{tabular}{r<{\stepcounter{LamReplCount}\theLamReplCount.} >{\hspace{1ex}}l @{\hspace{0.8em}} l}
  \toprule
  \multicolumn{1}{r}{No.} & Name & Instance  \\
  \midrule
  \multicolumn{3}{@{}l}{\hspace{0.3em}{\textit{{Replacement Instances}}}}\\
  & \isa{lam{\uscore}replacement{\uscore}fst} & $\lamRepl{\mathit{fst}(x)}$ \\
  & \isa{lam{\uscore}replacement{\uscore}snd} & $\lamRepl{\mathit{snd}(x)}$ \\
  & \isa{lam{\uscore}replacement{\uscore}Union} & $\lamRepl{\bigcup(x)}$ \\
  & \isa{lam{\uscore}replacement{\uscore}Image} & $\lamRepl{\mathit{fst}(x)``\mathit{snd}(x)}$ \\
  & \isa{lam{\uscore}replacement{\uscore}middle{\uscore}del} &
        $\lamRepl{\langle \mathit{fst}(\mathit{fst}(x)),\mathit{snd}(\mathit{snd}(x)) \rangle}$ \\
  & \isa{lam{\uscore}replacement{\uscore}prodRepl} &
        $\lamRepl{\langle \mathit{snd}(\mathit{fst}(x)),\langle \mathit{fst}(\mathit{fst}(x)),\mathit{snd}(\mathit{snd}(x))\rangle \rangle}$\\
  \midrule
  \multicolumn{3}{@{}l}{\hspace{0.3em}{\textit{{Separation Instances}}}}\\
  & \isa{middle{\uscore}separation} & $\mathit{snd}(\mathit{fst}(x))=\mathit{fst}(\mathit{snd}(x))$ \\
  & \isa{separation{\uscore}fst{\uscore}in{\uscore}snd} & $\mathit{fst}(\mathit{snd}(x)) \in \mathit{snd}(\mathit{snd}(x))$\\
  \bottomrule
\end{tabular}
\caption{Replacement and Separation instances of the locale \isa{M{\uscore}remplacement}}
\label{tab:m-repl-instances}
\end{threeparttable}
\end{table}

%%% Local Variables:
%%% mode: latex
%%% TeX-master: "independence_ch_isabelle"
%%% ispell-local-dictionary: "american"
%%% End:


\section{22 replacement instances to rule them all}
\label{sec:repl-instances-appendix}

In Table~\ref{tab:instances1} we show the name of the fourteen
formulas involved in the twenty two instances of replacement needed in
our mechanization. The formulas marked with ($\dagger$) are needed
twice: one by themselves and the other as the argument for
$\isa{ground{\uscore}repl{\uscore}fm}$. The
$\isa{ground{\uscore}repl{\uscore}fm}$ function maps $\phi$ to $\psi$
as described in Sect.~\ref{sec:repl-instances}. These eight instances form
the set \isa{instances3{\uscore}fms}.

% These are expressed using the following predicate:
% \[
%   \isa{ground{\uscore}replacement{\uscore}assm}(M,\mathit{env},\phi)
% \]
% defined as
% $\isa{replacement{\uscore}assm}(M,\mathit{env},\isa{ground{\uscore}repl{\uscore}fm}(\phi))$.
\newcommand{\groundRepl}{\ensuremath{{}^\dagger}}
\newcommand{\replInstSet}[1]{\multicolumn{3}{@{}l}{\hspace{0.3em}{$\mathbf{\mathit{#1}}$}}}

\begin{table}[!h]
\centering
\begin{threeparttable}
\begin{tabular}{r<{\stepcounter{replInstCount}\thereplInstCount.} >{\hspace{2pt}}l @{\hspace{1ex}} p{6cm}}
  \toprule
  \multicolumn{1}{r}{No.} & Formula's name & Comment \\
  \midrule
  \replInstSet{instances1{\uscore}fms}\\
%  \addlinespace
  & \isa{eclose{\uscore}closed{\uscore}fm} \groundRepl & Closure under iteration of $X\mapsto\union X$. \\
  & \isa{eclose{\uscore}abs{\uscore}fm} \groundRepl & Absoluteness of the previous construction.\\
%  \addlinespace
  & \isa{wfrec{\uscore}rank{\uscore}fm} \groundRepl &  For $\in$-rank.\\
%  \addlinespace
  & \isa{transrec{\uscore}VFrom{\uscore}fm} \groundRepl & For rank initial segments.\\
  \midrule
  \replInstSet{instances2{\uscore}fms}\\
%  \addlinespace
  & \isa{wfrec{\uscore}ordertype{\uscore}fm} \groundRepl & Well-founded recursion for the construction of ordertypes. \\
  & \isa{omap{\uscore}replacement{\uscore}fm} \groundRepl & Auxiliary instance for the definition of ordertypes. \\
%  \addlinespace
  & \isa{ordtype{\uscore}replacement{\uscore}fm} \groundRepl& Replacement through the ordertype function, for Hartogs' Theorem.\\
%  \addlinespace
  & \isa{wfrec{\uscore}Aleph{\uscore}fm} \groundRepl& Well-founded recursion to define Aleph.\\
  \midrule
  \replInstSet{instances{\uscore}ground{\uscore}fms}\\
%  \addlinespace
  & \isa{wfrec{\uscore}Hcheck{\uscore}fm} & Well-founded recursion to define check.\\
  & \isa{wfrec{\uscore}Hfrc{\uscore}at{\uscore}fm}. & Well-founded recursion for the definition of forcing for atomic formulas.\\
  & \isa{lam{\uscore}replacement{\uscore}check{\uscore}fm} & Replacement through $x\mapsto \lb x,\check{x}\rb$, for $\punto{G}$.\\
  \midrule
  \replInstSet{instances{\uscore}ground{\uscore}notCH{\uscore}fms}\\
%  \addlinespace
  &   \isa{rec{\uscore}constr{\uscore}fm} &
  Recursive construction of sets using a choice function.\\
  & \isa{rec{\uscore}constr{\uscore}abs{\uscore}fm} &
  Absoluteness of the previous construction.\\
  \midrule
  \replInstSet{instances{\uscore}ground{\uscore}CH{\uscore}fms}\\
%  \addlinespace
  & \isa{dc{\uscore}abs{\uscore}fm} &  Absoluteness of the recursive construction in the proof of
  Dependent Choice from $\AC$. \\
  \midrule
  \replInstSet{instances3{\uscore}fms}\\
%  \addlinespace
  \multicolumn{1}{r}{15-22.} & $\isa{ground{\uscore}repl{\uscore}fm}(\phi)$ & one for each formula $\phi$ marked with $\dagger$ \\
  \bottomrule
\end{tabular}
\caption{Replacement Instances used in our mechanization}
\label{tab:instances1}
\end{threeparttable}
\end{table}



%%% Local Variables:
%%% mode: latex
%%% TeX-master: "independence_ch_isabelle"
%%% ispell-local-dictionary: "american"
%%% End: 


\end{document}

%%% Local Variables: 
%%% mode: latex
%%% ispell-local-dictionary: "american"
%%% End: 
