\subsection{Names}
\label{sec:names}
In this section we formalize the function $\val$ that allows to
construct the elements of the generic extension $M[G]$ from elements
of the ctm $M$ and the generic filter $G$. The definition of $\val$
can be written succintly as
%
\begin{equation}\label{eq:def-val}
\val(G,\tau)\defi \{\val(G,\sigma) : \exists p\in \PP
(\lb\sigma,p\rb\in\tau \y p\in G)\}.
\end{equation}
%
This definition proceeds by well-founded recursion on the relation
$\mathit{ed}$:
\[
x \mathrel{\mathit{ed}} y \iff \exists p : \lb x,p\rb\in y.
\]
Recall that in $\ZFC$, an ordered pair $\lb x,y \rb$ is the set
$\{\{x\},\{x,y\}\}$. Hence this relation is well-founded since $\in$
is by the Foundation Axiom.

The  formalization of $\val$ in Isabelle/ZF involves the function
\isa{wfrec}, which takes as an argument a well-founded relation of
type \isa{i}, i.e.\ a set of pairs. Therefore we restrict
$\mathit{ed}$ above to a set.
%
\begin{isabelle}
\isacommand{definition}\isamarkupfalse%
\isanewline
\ \ edrel\ {\isacharcolon}{\isacharcolon}\ {\isachardoublequoteopen}i\ {\isasymRightarrow}\ i{\isachardoublequoteclose}\ \isakeyword{where}\isanewline
\ \ {\isachardoublequoteopen}edrel{\isacharparenleft}A{\isacharparenright}\ {\isacharequal}{\isacharequal}\ {\isacharbraceleft}{\isacharless}x{\isacharcomma}y{\isachargreater}\ {\isasymin}\ A{\isacharasterisk}A\ {\isachardot}\ x\ {\isasymin}\ domain{\isacharparenleft}y{\isacharparenright}{\isacharbraceright}{\isachardoublequoteclose}
\end{isabelle}
%
Since \isa{edrel(A)} is a subset of a  well-founded relation (the
transitive closure of the membership relation restricted to \isa{A}),
then it is well-founded as well.

\begin{isabelle}
\isacommand{lemma}\isamarkupfalse%
\ edrel{\isacharunderscore}sub{\isacharunderscore}memrel{\isacharcolon}\ {\isachardoublequoteopen}edrel{\isacharparenleft}A{\isacharparenright}\ {\isasymsubseteq}\ trancl{\isacharparenleft}Memrel{\isacharparenleft}eclose{\isacharparenleft}A{\isacharparenright}{\isacharparenright}{\isacharparenright}{\isachardoublequoteclose}
\end{isabelle}
\dots
\begin{isabelle}
\isacommand{lemma}\isamarkupfalse%
\ wf{\isacharunderscore}edrel\ {\isacharcolon}\ {\isachardoublequoteopen}wf{\isacharparenleft}edrel{\isacharparenleft}A{\isacharparenright}{\isacharparenright}{\isachardoublequoteclose}\isanewline
%
%
\isacommand{apply}\isamarkupfalse%
\ {\isacharparenleft}rule{\isacharunderscore}tac\ wf{\isacharunderscore}subset\ {\isacharbrackleft}of\ {\isachardoublequoteopen}trancl{\isacharparenleft}Memrel{\isacharparenleft}eclose{\isacharparenleft}A{\isacharparenright}{\isacharparenright}{\isacharparenright}{\isachardoublequoteclose}{\isacharbrackright}{\isacharparenright}\isanewline
\ \ \isacommand{apply}\isamarkupfalse%
\ {\isacharparenleft}auto\ simp\ add{\isacharcolon}edrel{\isacharunderscore}sub{\isacharunderscore}memrel\ wf{\isacharunderscore}trancl\ wf{\isacharunderscore}Memrel{\isacharparenright}\isanewline
\ \ \isacommand{done}\isamarkupfalse%
%
\end{isabelle}

We now turn to the definition of $\val$.

\begin{isabelle}
\isacommand{definition}\isamarkupfalse%
\isanewline
\ \ Hv\ {\isacharcolon}{\isacharcolon}\ {\isachardoublequoteopen}i{\isasymRightarrow}i{\isasymRightarrow}i{\isasymRightarrow}i{\isachardoublequoteclose}\ \isakeyword{where}\isanewline
\ \ {\isachardoublequoteopen}Hv{\isacharparenleft}G{\isacharcomma}x{\isacharcomma}f{\isacharparenright}\ {\isacharequal}{\isacharequal}\ {\isacharbraceleft}\ f{\isacharbackquote}y\ {\isachardot}{\isachardot}\ y{\isasymin}\ domain{\isacharparenleft}x{\isacharparenright}{\isacharcomma}\ {\isasymexists}p{\isasymin}P{\isachardot}\ {\isacharless}y{\isacharcomma}p{\isachargreater}\ {\isasymin}\ x\ {\isasymand}\ p\ {\isasymin}\ G\ {\isacharbraceright}{\isachardoublequoteclose}\isanewline
\isanewline
\isacommand{definition}\isamarkupfalse%
\isanewline
\ \ val\ {\isacharcolon}{\isacharcolon}\ {\isachardoublequoteopen}i{\isasymRightarrow}i{\isasymRightarrow}i{\isachardoublequoteclose}\ \isakeyword{where}\isanewline
\ \ {\isachardoublequoteopen}val{\isacharparenleft}G{\isacharcomma}{\isasymtau}{\isacharparenright}\ {\isacharequal}{\isacharequal}\ wfrec{\isacharparenleft}edrel{\isacharparenleft}eclose{\isacharparenleft}M{\isacharparenright}{\isacharparenright}{\isacharcomma}\ {\isasymtau}\ {\isacharcomma}Hv{\isacharparenleft}G{\isacharparenright}{\isacharparenright}{\isachardoublequoteclose}
\end{isabelle}
Then we can obtain the recursive expression~(\ref{eq:def-val}) in the
following lemma:
%
\begin{isabelle}
\isacommand{lemma}\isamarkupfalse%
\ def{\isacharunderscore}val{\isacharcolon}\ \ {\isachardoublequoteopen}x{\isasymin}M\ {\isasymLongrightarrow}\ val{\isacharparenleft}G{\isacharcomma}x{\isacharparenright}\ {\isacharequal}\ {\isacharbraceleft}val{\isacharparenleft}G{\isacharcomma}t{\isacharparenright}\ {\isachardot}{\isachardot}\ t{\isasymin}domain{\isacharparenleft}x{\isacharparenright}\ {\isacharcomma}\ {\isasymexists}p{\isasymin}P\ {\isachardot}\ \ {\isasymlangle}t{\isacharcomma}\ p{\isasymrangle}{\isasymin}x\ {\isasymand}\ p\ {\isasymin}\ G\ {\isacharbraceright}{\isachardoublequoteclose}
\end{isabelle}

We can finally define the generic extension of $M$ by $G$, also
setting up the notation $M[G]$ for it:
\begin{isabelle}
\isacommand{definition}\isamarkupfalse%
\isanewline
\ \ GenExt\ {\isacharcolon}{\isacharcolon}\ {\isachardoublequoteopen}i{\isasymRightarrow}i{\isachardoublequoteclose}\ \ \ \ \ {\isacharparenleft}{\isachardoublequoteopen}M{\isacharbrackleft}{\isacharunderscore}{\isacharbrackright}{\isachardoublequoteclose}\ {\isadigit{9}}{\isadigit{0}}{\isacharparenright}\isanewline
\ \ \isakeyword{where}\ {\isachardoublequoteopen}GenExt{\isacharparenleft}G{\isacharparenright}{\isacharequal}{\isacharequal}\ {\isacharbraceleft}val{\isacharparenleft}G{\isacharcomma}{\isasymtau}{\isacharparenright}{\isachardot}\ {\isasymtau}\ {\isasymin}\ M{\isacharbraceright}{\isachardoublequoteclose}
\end{isabelle}

%% It is to be noted that this results holds under the assumption that
%% the name \isa{t} is in $M$.

We now provide names for elements in $M$. That is, for each $x\in M$,
we define $\chk(x)$ (usually denoted by $\check{x}$ in the literature)
such that $\val(G,\chk(x))=x$. This will show that $M\sbq M[G]$, with
a caveat we make explicit in the end of this section. As explained in
the introduction, the fact that $M[G]$ extends $M$ is crucial to show
that $\ZFC$ holds in the former.

The definition of $\chk(x)$ is a straightforward $\in$-recursion:
\begin{equation}
  \label{eq:def-check}
  \chk(x)\defi\{\lb\chk(y),\1\rb : y\in x\}
\end{equation}
Now the set-relation argument for \isa{wfrec} is the membership
relation restricted to a set \isa{A}, \isa{Memrel(A)}.

\begin{isabelle}
\isacommand{definition}\isamarkupfalse%
\ \isanewline
\ \ Hcheck\ {\isacharcolon}{\isacharcolon}\ {\isachardoublequoteopen}{\isacharbrackleft}i{\isacharcomma}i{\isacharbrackright}\ {\isasymRightarrow}\ i{\isachardoublequoteclose}\ \isakeyword{where}\isanewline
\ \ {\isachardoublequoteopen}Hcheck{\isacharparenleft}z{\isacharcomma}f{\isacharparenright}\ \ {\isacharequal}{\isacharequal}\ {\isacharbraceleft}\ {\isacharless}f{\isacharbackquote}y{\isacharcomma}one{\isachargreater}\ {\isachardot}\ y\ {\isasymin}\ z{\isacharbraceright}{\isachardoublequoteclose}\isanewline
\isanewline
\isacommand{definition}\isamarkupfalse%
\isanewline
\ \ check\ {\isacharcolon}{\isacharcolon}\ {\isachardoublequoteopen}i\ {\isasymRightarrow}\ i{\isachardoublequoteclose}\ \isakeyword{where}\isanewline
\ \ {\isachardoublequoteopen}check{\isacharparenleft}x{\isacharparenright}\ {\isacharequal}{\isacharequal}\ wfrec{\isacharparenleft}Memrel{\isacharparenleft}eclose{\isacharparenleft}{\isacharbraceleft}x{\isacharbraceright}{\isacharparenright}{\isacharparenright}{\isacharcomma}\ x\ {\isacharcomma}\ Hcheck{\isacharparenright}{\isachardoublequoteclose}
\end{isabelle}
Here, \isa{eclose} returns the (downward) $\in$-closure of its
argument. The main result is stated as follows:

\begin{isabelle}
\isacommand{lemma}\isamarkupfalse%
\ valcheck\ {\isacharcolon}\ {\isachardoublequoteopen}y\ {\isasymin}\ M\ {\isasymLongrightarrow}\ one\ {\isasymin}\ G\ {\isasymLongrightarrow}\ {\isasymforall}x{\isasymin}M{\isachardot}\ check{\isacharparenleft}x{\isacharparenright}\ {\isasymin}\ M\ {\isasymLongrightarrow}\ val{\isacharparenleft}G{\isacharcomma}check{\isacharparenleft}y{\isacharparenright}{\isacharparenright}\ \ {\isacharequal}\ y{\isachardoublequoteclose}
\end{isabelle}

On the one hand, the only requirement on the set  \isatt{G} is that it 
contains the top \isa{one} of the poset \isa{P}. On the other hand, it
is necessary that all the relevant names are indeed in $M$. It
requires a serious development  to fulfill this assumption. One of the
hardest parts of Paulson's formalization of constructibility involves
showing that models are closed under recursive construction. We will
eventually formalize the that if $M\models\ZFC$ and the arguments of
\isa{wfrec} are in $M$, then its value also is. This will require
to adapt to models $M$ several locales defined in \cite{paulson_2003}
that were intended to be used for the class of constructible sets.




%%% Local Variables:
%%% mode: latex
%%% ispell-local-dictionary: "american"
%%% TeX-master: "first_steps_into_forcing"
%%% End:
