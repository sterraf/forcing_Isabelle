%%%%%%%%%%%%%%%%%%%%%%%%%%%%%%%%%%%%%%%%%%%%%%%%%%%%%%%%%%%%%%%%%%%%%%          
\section{Axioms and models of set theory}
\label{sec:axioms-models-set-theory}

The axioms of Zermelo and Fraenkel ($\ZF$) form a
countably infinite list of first-order sentences in a language
consisting of an only binary relation symbol $\in$. These include the
axioms of Extensionality, Pairing, Union, Powerset, Foundation, Infinity, and two
axiom-schemes collectively referred as
\begin{inlinelist}[label=(\emph{\alph*})]
\item Axiom of Separation:
\emph{For every $A$, $a_1,\dots,a_n$,  and  a formula
  $\psi(x_0,x_1,\dots,x_n)$, there exists $\{a\in  A:
  \psi(a,a_1,\dots,a_n)\}$}, 
and \item Axiom of Replacement: \emph{For every $A$, $a_1,\dots,a_n$,  and
  a formula   $\psi(x,z,x_1\dots,x_n)$, if 
  $\forall x.\exists!z.\psi(x,z,x_1,\dots,x_n)$,  there exists 
  $\{b : \exists a\in A. \psi(a,b,a_1,\dots,a_n)\}$}.
\end{inlinelist}
An excellent introduction to the axioms and the motivation behind them
can be found in Shoenfield \cite{MR3727410}. 

A model of the theory $\ZF$ consists of a pair $\lb M,E\rb$ where $M$
is a set and $E$ is a binary relation on $M$ satisfying the
axioms. Forcing is a technique 
to extend very special kind of models, where $M$ is a countable
transitive set (i.e., every element of $M$ is a subset of $M$) and
$E$ is the membership relation $\in$ restricted to $M$. In this case
we simply refer to $M$ as a \emph{countable transitive model} or
\emph{ctm}. The following result shows how to obtain ctms from weaker
hypotheses. 
%
\begin{lemma}\label{lem:wf-model-implies-ctm}
  If there exists  a
  model  $\lb N,E\rb$  of $\ZF$ such that the relation $E$ is well
  founded, then there exists a countable transitive one.
\end{lemma}
\begin{proof}
  (Sketch) The L\"owenheim-Skolem 
  Theorem ensures that there is an countable elementary submodel 
  $\lb N',E\restr N'\rb\preccurlyeq  \lb N,E\rb$ which must also be
  well founded; then the 
  Mostowski collapsing function \cite[Def.~I.9.31]{kunen2011set} sends $\lb
  N',E\restr N'\rb$ 
  isomorphically to some $\lb M,\in\rb$ with  $M$ transitive.
\end{proof}

In this stage of our implementation, we chose a presentation of the
$\ZF$ axioms that would be most
compatible with the development by Paulson. For
instance, the predicate
\isatt{upair{\isacharunderscore}ax{\isacharcolon}{\isacharcolon}{\isachardoublequoteopen}{\isacharparenleft}i{\isacharequal}{\isachargreater}o{\isacharparenright}{\isacharequal}{\isachargreater}o{\isachardoublequoteclose}}
takes a ``class'' (unary predicate) $C$ as an argument and states that
$C$ satisfies the Pairing Axiom.
\begin{isabelle}
upair{\isacharunderscore}ax{\isacharparenleft}C{\isacharparenright}{\isacharequal}{\isacharequal}{\isasymforall}x{\isacharbrackleft}C{\isacharbrackright}{\isachardot}{\isasymforall}y{\isacharbrackleft}C{\isacharbrackright}{\isachardot}{\isasymexists}z{\isacharbrackleft}C{\isacharbrackright}{\isachardot}\ upair{\isacharparenleft}C{\isacharcomma}x{\isacharcomma}y{\isacharcomma}z{\isacharparenright}
\end{isabelle}
Here, $\forall x[C]. \phi$ stands for
$\forall x. C(x) \longrightarrow \phi$, \emph{relative}
quantification. All of the development of relativization by Paulson is
written for a class model, so we set up a locale fixing a set $M$
and using the class $\#\#M\defi \lambda x. \ x\in M$ as the argument. 
\begin{isabelle}
\isacommand{locale}\isamarkupfalse%
\ M{\isacharunderscore}ZF\ {\isacharequal}\ \isanewline
\ \isakeyword{fixes}\ M\ \isanewline
\ \isakeyword{assumes}\ \isanewline
\ \ \ \ \ \ upair{\isacharunderscore}ax{\isacharcolon}\ \ \ \ {\isachardoublequoteopen}upair{\isacharunderscore}ax{\isacharparenleft}{\isacharhash}{\isacharhash}M{\isacharparenright}{\isachardoublequoteclose}\isanewline
%% \ \ \isakeyword{and}\ Union{\isacharunderscore}ax{\isacharcolon}\ \ \ \ {\isachardoublequoteopen}Union{\isacharunderscore}ax{\isacharparenleft}{\isacharhash}{\isacharhash}M{\isacharparenright}{\isachardoublequoteclose}\isanewline
%% \ \ \isakeyword{and}\ power{\isacharunderscore}ax{\isacharcolon}\ \ \ \ {\isachardoublequoteopen}power{\isacharunderscore}ax{\isacharparenleft}{\isacharhash}{\isacharhash}M{\isacharparenright}{\isachardoublequoteclose}\isanewline
%% \ \ \isakeyword{and}\ extensionality{\isacharcolon}\ \ \ {\isachardoublequoteopen}extensionality{\isacharparenleft}{\isacharhash}{\isacharhash}M{\isacharparenright}{\isachardoublequoteclose}\isanewline
%% \ \ \isakeyword{and}\ foundation{\isacharunderscore}ax{\isacharcolon}\ \ \ \ {\isachardoublequoteopen}foundation{\isacharunderscore}ax{\isacharparenleft}{\isacharhash}{\isacharhash}M{\isacharparenright}{\isachardoublequoteclose}\isanewline
%% \ \ \isakeyword{and}\ infinity{\isacharunderscore}ax{\isacharcolon}\ {\isachardoublequoteopen}infinity{\isacharunderscore}ax{\isacharparenleft}{\isacharhash}{\isacharhash}M{\isacharparenright}{\isachardoublequoteclose}\isanewline
\ \ \isakeyword{and}\ \dots \isanewline
\ \ \isakeyword{and}\ separation{\isacharunderscore}ax{\isacharcolon}\isanewline
\ \ {\isachardoublequoteopen}{\isasymlbrakk}\ {\isasymphi}\ {\isasymin}\ formula\ {\isacharsemicolon}\ arity{\isacharparenleft}{\isasymphi}{\isacharparenright}{\isacharequal}{\isadigit{1}}\ {\isasymor}\ arity{\isacharparenleft}{\isasymphi}{\isacharparenright}{\isacharequal}{\isadigit{2}}\ {\isasymrbrakk}\isanewline\ \ \ \ {\isasymLongrightarrow}\isanewline
\ \ {\isacharparenleft}{\isasymforall}a{\isasymin}M{\isachardot}\ separation{\isacharparenleft}{\isacharhash}{\isacharhash}M{\isacharcomma}{\isasymlambda}x{\isachardot}\ sats{\isacharparenleft}M{\isacharcomma}{\isasymphi}{\isacharcomma}{\isacharbrackleft}x{\isacharcomma}a{\isacharbrackright}{\isacharparenright}{\isacharparenright}{\isacharparenright}{\isachardoublequoteclose}\isanewline
\ \ \isakeyword{and}\ replacement{\isacharunderscore}ax{\isacharcolon}\isanewline 
\ \ {\isachardoublequoteopen}{\isasymlbrakk}\ {\isasymphi}\ {\isasymin}\ formula{\isacharsemicolon}\ arity{\isacharparenleft}{\isasymphi}{\isacharparenright}{\isacharequal}{\isadigit{2}}\ {\isasymor}\ arity{\isacharparenleft}{\isasymphi}{\isacharparenright}{\isacharequal}{\isadigit{3}}\ {\isasymrbrakk}\isanewline
\ \ \ \ {\isasymLongrightarrow}\isanewline
\ \ {\isacharparenleft}{\isasymforall}a{\isasymin}M{\isachardot}\ strong{\isacharunderscore}replacement{\isacharparenleft}{\isacharhash}{\isacharhash}M{\isacharcomma}\isanewline
\ \ \ \ {\isasymlambda}x\ y{\isachardot}\ sats{\isacharparenleft}M{\isacharcomma}{\isasymphi}{\isacharcomma}{\isacharbrackleft}x{\isacharcomma}y{\isacharcomma}a{\isacharbrackright}{\isacharparenright}{\isacharparenright}{\isacharparenright}{\isachardoublequoteclose}
\end{isabelle}
The rest of the axioms are also
included. We single out Separation and Replacement: These are written
for formulas with at most one extra parameter (meaning $n\leq 1$ in the
above $\psi$). Thanks to Pairing, these 
versions are equivalent to the usual formulations. We are only able to
prove that the generic extension satisfies Separation for any particular
number of parameters, but not in general. This is a consequence that
induction on terms of type \tyo{} is not available.

It is also possible define a predicate that states that a set
satisfies a (possibly infinite) set of formulas, and then to state
that ``$M$ satisfies $\ZF$'' in a standard way. With the
aforementioned restriction on parameters, it can be shown that this
statement is equivalent to the set of assumptions of the locale
\isatt{M\_ZF}.
%
%%% Local Variables: 
%%% mode: latex
%%% TeX-master: "Separation_In_MG"
%%% ispell-local-dictionary: "american"
%%% End: 
