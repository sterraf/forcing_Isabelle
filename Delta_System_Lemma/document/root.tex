\documentclass[11pt,a4paper]{article}
\usepackage{isabelle,isabellesym}
\usepackage[numbers]{natbib}

\usepackage{relsize}
\DeclareRobustCommand{\isactrlbsub}{\emph\bgroup\math{}\sb\bgroup\mbox\bgroup\isaspacing\itshape\smaller}
\DeclareRobustCommand{\isactrlesub}{\egroup\egroup\endmath\egroup}
\DeclareRobustCommand{\isactrlbsup}{\emph\bgroup\math{}\sp\bgroup\mbox\bgroup\isaspacing\itshape\smaller}
\DeclareRobustCommand{\isactrlesup}{\egroup\egroup\endmath\egroup}

% further packages required for unusual symbols (see also
% isabellesym.sty), use only when needed

\usepackage{amssymb}
  %for \<leadsto>, \<box>, \<diamond>, \<sqsupset>, \<mho>, \<Join>,
  %\<lhd>, \<lesssim>, \<greatersim>, \<lessapprox>, \<greaterapprox>,
  %\<triangleq>, \<yen>, \<lozenge>

%\usepackage{eurosym}
  %for \<euro>

%\usepackage[only,bigsqcap]{stmaryrd}
  %for \<Sqinter>

%\usepackage{eufrak}
  %for \<AA> ... \<ZZ>, \<aa> ... \<zz> (also included in amssymb)

%\usepackage{textcomp}
  %for \<onequarter>, \<onehalf>, \<threequarters>, \<degree>, \<cent>,
  %\<currency>

\input{header-delta-system}

% this should be the last package used
\usepackage{pdfsetup}

% urls in roman style, theory text in math-similar italics
\urlstyle{rm}
\isabellestyle{it}

% for uniform font size
%\renewcommand{\isastyle}{\isastyleminor}

\renewcommand{\isacharunderscorekeyword}{\mbox{\_}}
\renewcommand{\isacharunderscore}{\mbox{\_}}
\renewcommand{\isasymtturnstile}{\isamath{\Vdash}}
\renewcommand{\isacharminus}{-}

\begin{document}

\title{Cofinality and the Delta System Lemma}
\author{
  Pedro S\'anchez Terraf\thanks{Universidad Nacional de C\'ordoba. 
    Facultad de Matem\'atica, Astronom\'{\i}a,  F\'{\i}sica y
    Computaci\'on.}
  \thanks{%
    Centro de Investigaci\'on y Estudios de Matem\'atica
    (CIEM-FaMAF), Conicet. C\'ordoba. Argentina.
    Supported by Secyt-UNC project 33620180100465CB.}
}
\maketitle

\begin{abstract}
  We formalize the basic results on cofinality of linearly ordered
  sets and ordinals and \v{S}anin's Lemma for uncountable families of
  finite sets. We work in the set theory framework of
  Isabelle/ZF, using the Axiom of Choice as needed.
\end{abstract}


\tableofcontents

% sane default for proof documents
\parindent 0pt\parskip 0.5ex

% generated text of all theories
\input{session}

% optional bibliography
\bibliographystyle{root}
\bibliography{root}

\end{document}

%%% Local Variables:
%%% mode: latex
%%% TeX-master: t
%%% End:
