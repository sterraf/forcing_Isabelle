\section{Forcing notions}\label{sec:forcing-posets}
%\textit{Para no confundir, directamente usamos este nombre en lugar de
% ``forcing posets'' para los preórdenes con un máximo distinguido}

Our first goal is to formalize some version of Martin's Axiom (MA), for
example to show that if the cardinal $\kappa$ satisfies some property
about partial orders then $\kappa < \aleph_1$. Let us first introduce
some definitions about orders. We will also use this section to exhibit
how to encode some definitions in Isabelle/ZF.

\begin{definition}
  A pre-order on a set $P$ is a binary relation $\leqslant$ which is
  reflexive and transitive.
\end{definition}

In Isabelle/ZF sets are \emph{individuals}, that is terms of type $i$; since
relations are also sets, the relation $\leqslant$ cannot be defined as
a predicate on $P$.

\begin{example}
  Isabelle/ZF has already defined the set of naturals \isatt{nat}.
\end{example}

%%% Local Variables:
%%% mode: latex
%%% ispell-local-dictionary: "american"
%%% TeX-master: "first_steps_into_forcing"
%%% End:


