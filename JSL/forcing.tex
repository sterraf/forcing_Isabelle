%%%%%%%%%%%%%%%%%%%%%%%%%%%%%%%%%%%%%%%%%%%%%%%%%%%%%%%%%%%%%%%%%%%%%%          
\section{Forcing}
\label{sec:forcing}
Let $\lb \PP, {\preceq} ,\1\rb \in M$ be a forcing notion. Given $G\sbq \PP$, we have
$M[G]\defi \{ \val(\PP,G,\punto{a}) : \punto{a}\in M \}$.

The following form of the Forcing Theorems  is the one
that we formalized.
\begin{theorem}
  There exists a function  $\forceisa: \formula \to  \formula$
  such that for every
  $\phi$ and $\punto{a}_0,\dots,\punto{a}_n\in M$,
  \begin{enumerate}
  \item (Truth Lemma) for every $M$-generic $G$,
    \[
      M[G], [\val(\PP,G,\punto{a}_0),\dots,\val(\PP,G,\punto{a}_n)]
      \models \phi
    \]
    is equivalent to 
    \[
      \exists p\in G.\ \; M, [p,\PP,\preceq,\1, \punto{a}_0,\dots,\punto{a}_n]  \models
      \forceisa(\phi).\]

    We use the notation $p \forces
    \phi\ [\punto{a}_0,\dots,\punto{a}_n]$ for this last assertion.
  \item (Density Lemma) $p \forces \phi\ [\punto{a}_0,\dots,\punto{a}_n]$
    if and only if 
    $\{q\in \PP :  q \forces \phi\ [\punto{a}_0,\dots,\punto{a}_n]\}$
    is dense below $p$.
  \end{enumerate}
\end{theorem}

Note: $\forceisa$ was formalized as a class-function (i.e., of type
$\tyi\Rightarrow\tyi$), to take advantage of the Isabelle's
conveniences for primitive recursion.

%%% Local Variables: 
%%% mode: latex
%%% TeX-master: "independence_ch_isabelle"
%%% ispell-local-dictionary: "american"
%%% End: 
