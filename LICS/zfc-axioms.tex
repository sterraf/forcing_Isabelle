%%%%%%%%%%%%%%%%%%%%%%%%%%%%%%%%%%%%%%%%%%%%%%%%%%%%%%%%%%%%%%%%%%%%%%          
\section{Models and axioms of set theory}

The axioms of Zermelo and Fraenkel ($\ZF$) form a
countably infinite list of first-order sentences in a language
consisting of an only binary relation symbol $\in$. These include the
axioms of Pairing, Union, Powerset, Foundation, Infinity, and two
axiom-schemes collectively referred to as Axiom of Separation:
\emph{For every $A$, $a_1,\dots,a_n$,  and  a formula
  $\psi(x_0,x_1,\dots,x_n)$, there exists $\{x\in  A:
  \psi(a,a_1,\dots,a_n)\}$}, 
and Axiom of Replacement: \emph{For every $A$, $a_1,\dots,a_n$,  and
  a formula   $\psi(x,z,x_1\dots,x_n)$, if 
  $\forall x.\exists!z.\psi(x,z,x_1,\dots,x_n)$,  there exists 
  $\{b : \exists a\in A. \psi(a,b,a_1,\dots,a_n)\}$}.
An excellent introduction to the axioms and the motivation behind them
can be found in Shoenfield \cite{MR3727410}. 

Models of the theory $\ZF$ consist of a pair $\lb M,E\rb$ where $M$
is a set and $E$ is a binary relation on $M$. Forcing is a technique
to extend very special kind of models, where $M$ is a countable
transitive set (i.e., every element of $M$ is a subset of $M$) and
$E$ is the membership relation $\in$ restricted to $M$. In this case
we simply refer to $M$ as a \emph{countable transitive model} or
\emph{ctm}.

The existence of a ctm of $\ZF$ can be proved from the existence of a
model  $\lb N,E\rb$ such that the relation $E$ is well founded. The
L\"owenheim-Skolem 
Theorem ensures that there is an countable elementary submodel 
$\lb N',E\restr N'\rb\preccurlyeq  \lb N,E\rb$ which must also be
well founded; then the 
Mostowski collapsing function \cite[Def.~I.9.31]{kunen2011set} sends $\lb
N',E\restr N'\rb$ 
isomorphically to some $\lb M,\in\rb$ with  $M$ transitive.

By G\"odel's Second Incompleteness Theorem we cannot  prove that
there exists a model of $\ZF$  (assuming that our base theory is not
stronger than $\ZF$ to start with), but for consistency proofs,
usually a model of only a finite subset of $\ZF$ suffices. Recall that
an equivalent formulation of the Foundation Axiom states that the 
universe of sets can be decomposed in a transfinite hierarchy of
sets. 
\begin{theorem}
  Let $V_{\al}\defi\union\{\P(V_\be) : \be<\al\}$ for each ordinal
  $\al$. Then each $V_\al$ is a set and 
  $\forall x. \exists\al .\ \Ord(\al) \y x\in V_\al$.  
\end{theorem}
%
Then the following meta-theoretic result by Montague applies:
%
\begin{theorem}[Reflection Principle]\label{th:reflection-principle}
  For every finite $\Phi\sbq\ZF$, $\ZF$ proves: ``There exists
    unboundedly many $\al$ such that $V_\al\models \Phi$.''
\end{theorem}
%
Since the sets $V_\al$ are well founded, we can repeat the above
argument to obtain a ctm of $\Phi$.

In this stage of our implementation, we chose a presentation of the
$\ZF$ axioms that would be most
compatible with the development of relativization by Paulson. For
instance, the predicate
\isatt{upair{\isacharunderscore}ax{\isacharcolon}{\isacharcolon}{\isachardoublequoteopen}{\isacharparenleft}i{\isacharequal}{\isachargreater}o{\isacharparenright}{\isacharequal}{\isachargreater}o{\isachardoublequoteclose}}
takes a ``class'' (unary predicate) $C$ as an argument and states that
$C$ satisfies the Pairing Axiom.
\begin{isabelle}
upair{\isacharunderscore}ax{\isacharparenleft}C{\isacharparenright}{\isacharequal}{\isacharequal}{\isasymforall}x{\isacharbrackleft}C{\isacharbrackright}{\isachardot}{\isasymforall}y{\isacharbrackleft}C{\isacharbrackright}{\isachardot}{\isasymexists}z{\isacharbrackleft}C{\isacharbrackright}{\isachardot}\ upair{\isacharparenleft}C{\isacharcomma}x{\isacharcomma}y{\isacharcomma}z{\isacharparenright}
\end{isabelle}
Here, $\forall x[C]. \phi$ stands for
$\forall x. C(x) \longrightarrow \phi$, \emph{relative}
quantification. All of the development of relativization by Paulson is
written for class arguments, so we set up a locale fixing a set $M$
and using the class $\#\#M\defi \lambda x. \ x\in M$ as the argument. 
\begin{isabelle}
\isacommand{locale}\isamarkupfalse%
\ M{\isacharunderscore}ZF\ {\isacharequal}\ \isanewline
\ \isakeyword{fixes}\ M\ \isanewline
\ \isakeyword{assumes}\ \isanewline
\ \ \ \ \ \ upair{\isacharunderscore}ax{\isacharcolon}\ \ \ \ {\isachardoublequoteopen}upair{\isacharunderscore}ax{\isacharparenleft}{\isacharhash}{\isacharhash}M{\isacharparenright}{\isachardoublequoteclose}\isanewline
%% \ \ \isakeyword{and}\ Union{\isacharunderscore}ax{\isacharcolon}\ \ \ \ {\isachardoublequoteopen}Union{\isacharunderscore}ax{\isacharparenleft}{\isacharhash}{\isacharhash}M{\isacharparenright}{\isachardoublequoteclose}\isanewline
%% \ \ \isakeyword{and}\ power{\isacharunderscore}ax{\isacharcolon}\ \ \ \ {\isachardoublequoteopen}power{\isacharunderscore}ax{\isacharparenleft}{\isacharhash}{\isacharhash}M{\isacharparenright}{\isachardoublequoteclose}\isanewline
%% \ \ \isakeyword{and}\ extensionality{\isacharcolon}\ \ \ {\isachardoublequoteopen}extensionality{\isacharparenleft}{\isacharhash}{\isacharhash}M{\isacharparenright}{\isachardoublequoteclose}\isanewline
%% \ \ \isakeyword{and}\ foundation{\isacharunderscore}ax{\isacharcolon}\ \ \ \ {\isachardoublequoteopen}foundation{\isacharunderscore}ax{\isacharparenleft}{\isacharhash}{\isacharhash}M{\isacharparenright}{\isachardoublequoteclose}\isanewline
%% \ \ \isakeyword{and}\ infinity{\isacharunderscore}ax{\isacharcolon}\ {\isachardoublequoteopen}infinity{\isacharunderscore}ax{\isacharparenleft}{\isacharhash}{\isacharhash}M{\isacharparenright}{\isachardoublequoteclose}\isanewline
\ \ \isakeyword{and}\ \dots \isanewline
\ \ \isakeyword{and}\ separation{\isacharunderscore}ax{\isacharcolon}\isanewline
\ \ {\isachardoublequoteopen}{\isasymlbrakk}\ {\isasymphi}\ {\isasymin}\ formula\ {\isacharsemicolon}\ arity{\isacharparenleft}{\isasymphi}{\isacharparenright}{\isacharequal}{\isadigit{1}}\ {\isasymor}\ arity{\isacharparenleft}{\isasymphi}{\isacharparenright}{\isacharequal}{\isadigit{2}}\ {\isasymrbrakk}\isanewline\ \ \ \ {\isasymLongrightarrow}\isanewline
\ \ {\isacharparenleft}{\isasymforall}a{\isasymin}M{\isachardot}\ separation{\isacharparenleft}{\isacharhash}{\isacharhash}M{\isacharcomma}{\isasymlambda}x{\isachardot}\ sats{\isacharparenleft}M{\isacharcomma}{\isasymphi}{\isacharcomma}{\isacharbrackleft}x{\isacharcomma}a{\isacharbrackright}{\isacharparenright}{\isacharparenright}{\isacharparenright}{\isachardoublequoteclose}\isanewline
\ \ \isakeyword{and}\ replacement{\isacharunderscore}ax{\isacharcolon}\isanewline 
\ \ {\isachardoublequoteopen}{\isasymlbrakk}\ {\isasymphi}\ {\isasymin}\ formula{\isacharsemicolon}\ arity{\isacharparenleft}{\isasymphi}{\isacharparenright}{\isacharequal}{\isadigit{2}}\ {\isasymor}\ arity{\isacharparenleft}{\isasymphi}{\isacharparenright}{\isacharequal}{\isadigit{3}}\ {\isasymrbrakk}\isanewline
\ \ \ \ {\isasymLongrightarrow}\isanewline
\ \ {\isacharparenleft}{\isasymforall}a{\isasymin}M{\isachardot}\ strong{\isacharunderscore}replacement{\isacharparenleft}{\isacharhash}{\isacharhash}M{\isacharcomma}\isanewline
\ \ \ \ {\isasymlambda}x\ y{\isachardot}\ sats{\isacharparenleft}M{\isacharcomma}{\isasymphi}{\isacharcomma}{\isacharbrackleft}x{\isacharcomma}y{\isacharcomma}a{\isacharbrackright}{\isacharparenright}{\isacharparenright}{\isacharparenright}{\isachardoublequoteclose}
\end{isabelle}
The rest of the axioms are also
included. We single out Separation and Replacement: these are written
for formulas with at most one extra parameter (meaning $n\leq 1$ in the
above $\phi$). Thanks to Pairing, these
versions are equivalent to the usual formulations. We are only able to
prove that the extension satisfies Separation for any particular
number of parameters, but not in general. This is a consequence that
induction on terms of typo \tyo{} is not available.


%
%%% Local Variables: 
%%% mode: latex
%%% TeX-master: "Separation_In_MG"
%%% ispell-local-dictionary: "american"
%%% End: 
