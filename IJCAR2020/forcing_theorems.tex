\section{The forcing theorems}
\label{sec:forcing-theorems}

After the definition of $\forceisa$ is complete, the proof of the
Fundamental Theorems of Forcing is comparatively straightforward, and
we were able to follow Kunen very closely. The more involved points of
this part of the development were those where we needed to proved that
various (dense) subsets of $\PP$ were in $M$; for this, we had to
recourse to several ad-hoc absoluteness lemmas.

The first results concern characterizations of the forcing
relation. Two of them are \isatt{Forces{\isacharunderscore}Member}:
\begin{center}
  \isatt{{\isacharparenleft}p\ {\isasymtturnstile}\ Member{\isacharparenleft}n{\isacharcomma}m{\isacharparenright}\ env{\isacharparenright}\ {\isasymlongleftrightarrow}\ forces{\isacharunderscore}mem{\isacharparenleft}p{\isacharcomma}t{\isadigit{1}}{\isacharcomma}t{\isadigit{2}}{\isacharparenright}},
\end{center}
where \isatt{t{\isadigit{1}}} and \isatt{t{\isadigit{1}}} are the
\isatt{n}th resp.\ \isatt{m}th elements of \isatt{env}, and  \isatt{Forces{\isacharunderscore}Forall}:
\begin{center}
  \isatt{{\isacharparenleft}p\ {\isasymtturnstile}\ Forall{\isacharparenleft}{\isasymphi}{\isacharparenright}\ env{\isacharparenright}\ {\isasymlongleftrightarrow}\ {\isacharparenleft}{\isasymforall}x{\isasymin}M{\isachardot}\ {\isacharparenleft}p\ {\isasymtturnstile}\ {\isasymphi}\ {\isacharparenleft}{\isacharbrackleft}x{\isacharbrackright}\ {\isacharat}\ env{\isacharparenright}{\isacharparenright}{\isacharparenright}}.
\end{center}
Equivalent statements, along with the ones corresponding to \isatt{Forces{\isacharunderscore}Equal} and
\isatt{Forces{\isacharunderscore}Nand}, appear in Kunen as the
inductive definition of the forcing relation \cite[Def.~IV.2.42]{kunen2011set}.

As with the previous section, the proofs of the forcing theorems have two different
flavors: The ones for the atomic formulas proceed by using the
principle of 
\isatt{forces{\isacharunderscore}induction}, and then an induction on
$\formula$ wraps the former with the remaining cases (\isatt{Nand} and \isatt{Forall}). 

As an example of the first class, we can take a look at our
formalization of \cite[Lem.~IV.2.40(a)]{kunen2011set}. Note that the
context (a ``locale,'' in Isabelle terminology, namely \isatt{forcing{\isacharunderscore}data}) of the lemma 
includes the assumption of \isatt{P} being
a forcing notion, and the predicate of being $M$-generic is defined in
terms of \isatt{P}:

\begin{isabelle}
  \isacommand{lemma}\isamarkupfalse%
  \ IV{\isadigit{2}}{\isadigit{4}}{\isadigit{0}}a{\isacharcolon}\isanewline
  \ \ \isakeyword{assumes}\isanewline
  \ \ \ \ {\isachardoublequoteopen}M{\isacharunderscore}generic{\isacharparenleft}G{\isacharparenright}{\isachardoublequoteclose}\isanewline
  \ \ \isakeyword{shows}\ \isanewline
  \ \ {\isachardoublequoteopen}{\isacharparenleft}{\isasymtau}{\isasymin}M{\isasymlongrightarrow}{\isasymtheta}{\isasymin}M{\isasymlongrightarrow}{\isacharparenleft}{\isasymforall}p{\isasymin}G{\isachardot}forces{\isacharunderscore}eq{\isacharparenleft}p{\isacharcomma}{\isasymtau}{\isacharcomma}{\isasymtheta}{\isacharparenright}{\isasymlongrightarrow}val{\isacharparenleft}G{\isacharcomma}{\isasymtau}{\isacharparenright}{\isacharequal}val{\isacharparenleft}G{\isacharcomma}{\isasymtheta}{\isacharparenright}{\isacharparenright}{\isacharparenright}\isanewline
  \ \  {\isasymand}\isanewline
  \ \ {\isacharparenleft}{\isasymtau}{\isasymin}M{\isasymlongrightarrow}{\isasymtheta}{\isasymin}M{\isasymlongrightarrow}{\isacharparenleft}{\isasymforall}p{\isasymin}G{\isachardot}forces{\isacharunderscore}mem{\isacharparenleft}p{\isacharcomma}{\isasymtau}{\isacharcomma}{\isasymtheta}{\isacharparenright}{\isasymlongrightarrow}val{\isacharparenleft}G{\isacharcomma}{\isasymtau}{\isacharparenright}{\isasymin}val{\isacharparenleft}G{\isacharcomma}{\isasymtheta}{\isacharparenright}{\isacharparenright}{\isacharparenright}{\isachardoublequoteclose}
\end{isabelle}
%
Its proof starts by an introduction of \isatt{forces{\isacharunderscore}induction};
the  inductive cases for each atomic type were handled before as
separate lemmas (\isatt{IV240a{\isacharunderscore}mem} and \isatt{IV240a{\isacharunderscore}eq}). We
illustrate with the statement of the latter.
%
\begin{isabelle}
\isacommand{lemma}\isamarkupfalse%
\ IV{\isadigit{2}}{\isadigit{4}}{\isadigit{0}}a{\isacharunderscore}eq{\isacharcolon}\isanewline
\ \ \isakeyword{assumes}\isanewline
\ \ \ \ {\isachardoublequoteopen}M{\isacharunderscore}generic{\isacharparenleft}G{\isacharparenright}{\isachardoublequoteclose}\ {\isachardoublequoteopen}p{\isasymin}G{\isachardoublequoteclose}\ {\isachardoublequoteopen}forces{\isacharunderscore}eq{\isacharparenleft}p{\isacharcomma}{\isasymtau}{\isacharcomma}{\isasymtheta}{\isacharparenright}{\isachardoublequoteclose}\isanewline
\ \ \ \ \isakeyword{and}\isanewline
\ \ \ \ IH{\isacharcolon}{\isachardoublequoteopen}{\isasymAnd}q\ {\isasymsigma}{\isachardot}\ q{\isasymin}P\ {\isasymLongrightarrow}\ q{\isasymin}G\ {\isasymLongrightarrow}\ {\isasymsigma}{\isasymin}domain{\isacharparenleft}{\isasymtau}{\isacharparenright}\ {\isasymunion}\ domain{\isacharparenleft}{\isasymtheta}{\isacharparenright}\ {\isasymLongrightarrow}\ \isanewline
\ \ \ \ \ \ \ \ {\isacharparenleft}forces{\isacharunderscore}mem{\isacharparenleft}q{\isacharcomma}{\isasymsigma}{\isacharcomma}{\isasymtau}{\isacharparenright}\ {\isasymlongrightarrow}\ val{\isacharparenleft}G{\isacharcomma}{\isasymsigma}{\isacharparenright}\ {\isasymin}\ val{\isacharparenleft}G{\isacharcomma}{\isasymtau}{\isacharparenright}{\isacharparenright}\ {\isasymand}\isanewline
\ \ \ \ \ \ \ \ {\isacharparenleft}forces{\isacharunderscore}mem{\isacharparenleft}q{\isacharcomma}{\isasymsigma}{\isacharcomma}{\isasymtheta}{\isacharparenright}\ {\isasymlongrightarrow}\ val{\isacharparenleft}G{\isacharcomma}{\isasymsigma}{\isacharparenright}\ {\isasymin}\ val{\isacharparenleft}G{\isacharcomma}{\isasymtheta}{\isacharparenright}{\isacharparenright}{\isachardoublequoteclose}\isanewline
\ \ \isakeyword{shows}\isanewline
\ \ \ \ {\isachardoublequoteopen}val{\isacharparenleft}G{\isacharcomma}{\isasymtau}{\isacharparenright}\ {\isacharequal}\ val{\isacharparenleft}G{\isacharcomma}{\isasymtheta}{\isacharparenright}{\isachardoublequoteclose}
\end{isabelle}

Examples of proofs  using the second kind of induction include
the basic \isatt{strengthening{\isacharunderscore}lemma} and the main
results in this section, the lemmas of  Density  (actually, its nontrivial
direction
\isatt{dense{\isacharunderscore}below{\isacharunderscore}imp{\isacharunderscore}forces})
and Truth, 
which we state next.
%
\begin{isabelle}
\isacommand{lemma}\isamarkupfalse%
\ density{\isacharunderscore}lemma{\isacharcolon}\isanewline
\ \ \isakeyword{assumes}\isanewline
\ \ \ \ {\isachardoublequoteopen}p{\isasymin}P{\isachardoublequoteclose}\ {\isachardoublequoteopen}{\isasymphi}{\isasymin}formula{\isachardoublequoteclose}\ {\isachardoublequoteopen}env{\isasymin}list{\isacharparenleft}M{\isacharparenright}{\isachardoublequoteclose}\ {\isachardoublequoteopen}arity{\isacharparenleft}{\isasymphi}{\isacharparenright}{\isasymle}length{\isacharparenleft}env{\isacharparenright}{\isachardoublequoteclose}\isanewline
\ \ \isakeyword{shows}\isanewline
\ \ \ \ {\isachardoublequoteopen}{\isacharparenleft}p\ {\isasymtturnstile}\ {\isasymphi}\ env{\isacharparenright}\ {\isasymlongleftrightarrow}\ dense{\isacharunderscore}below{\isacharparenleft}{\isacharbraceleft}q{\isasymin}P{\isachardot}\ {\isacharparenleft}q\ {\isasymtturnstile}\ {\isasymphi}\ env{\isacharparenright}{\isacharbraceright}{\isacharcomma}p{\isacharparenright}{\isachardoublequoteclose}
\end{isabelle}
\begin{isabelle}
\isacommand{lemma}\isamarkupfalse%
\ truth{\isacharunderscore}lemma{\isacharcolon}\isanewline
\ \ \isakeyword{assumes}\ \isanewline
\ \ \ \ {\isachardoublequoteopen}{\isasymphi}{\isasymin}formula{\isachardoublequoteclose}\ {\isachardoublequoteopen}M{\isacharunderscore}generic{\isacharparenleft}G{\isacharparenright}{\isachardoublequoteclose}\isanewline
\ \ \isakeyword{shows}\ \isanewline
\ \ \ \ \ {\isachardoublequoteopen}{\isasymAnd}env{\isachardot}\ env{\isasymin}list{\isacharparenleft}M{\isacharparenright}\ {\isasymLongrightarrow}\ arity{\isacharparenleft}{\isasymphi}{\isacharparenright}{\isasymle}length{\isacharparenleft}env{\isacharparenright}\ {\isasymLongrightarrow}\ \isanewline
\ \ \ \ \ \ {\isacharparenleft}{\isasymexists}p{\isasymin}G{\isachardot}\ {\isacharparenleft}p\ {\isasymtturnstile}\ {\isasymphi}\ env{\isacharparenright}{\isacharparenright}\ \ {\isasymlongleftrightarrow}\ \ M{\isacharbrackleft}G{\isacharbrackright}{\isacharcomma}\ map{\isacharparenleft}val{\isacharparenleft}G{\isacharparenright}{\isacharcomma}env{\isacharparenright}\ {\isasymTurnstile}\ {\isasymphi}{\isachardoublequoteclose}
\end{isabelle}
%
From these results, the semantical characterization of the forcing
relation (the ``definition of $\forces$'' in 
\cite[IV.2.22]{kunen2011set}) follows easily:
\begin{isabelle}
\isacommand{lemma}\isamarkupfalse%
\ definition{\isacharunderscore}of{\isacharunderscore}forcing{\isacharcolon}\isanewline
\ \ \isakeyword{assumes}\isanewline
\ \ \ \ {\isachardoublequoteopen}p{\isasymin}P{\isachardoublequoteclose}\ {\isachardoublequoteopen}{\isasymphi}{\isasymin}formula{\isachardoublequoteclose}\ {\isachardoublequoteopen}env{\isasymin}list{\isacharparenleft}M{\isacharparenright}{\isachardoublequoteclose}\ {\isachardoublequoteopen}arity{\isacharparenleft}{\isasymphi}{\isacharparenright}{\isasymle}length{\isacharparenleft}env{\isacharparenright}{\isachardoublequoteclose}\isanewline
\ \ \isakeyword{shows}\isanewline
\ \ \ \ {\isachardoublequoteopen}{\isacharparenleft}p\ {\isasymtturnstile}\ {\isasymphi}\ env{\isacharparenright}\ {\isasymlongleftrightarrow}\isanewline
\ \ \ \ \ {\isacharparenleft}{\isasymforall}G{\isachardot}\ M{\isacharunderscore}generic{\isacharparenleft}G{\isacharparenright}{\isasymand}\ p{\isasymin}G\ {\isasymlongrightarrow}\ M{\isacharbrackleft}G{\isacharbrackright}{\isacharcomma}\ map{\isacharparenleft}val{\isacharparenleft}G{\isacharparenright}{\isacharcomma}env{\isacharparenright}\ {\isasymTurnstile}\ {\isasymphi}{\isacharparenright}{\isachardoublequoteclose}
\end{isabelle}

The present statement of the Fundamental Theorems is almost exactly
the same of those in our previous report \cite{2019arXiv190103313G},
with the only modification being the bound on arities and a missing
typing constraint. This implied only minor adjustments in the proofs
of the satisfaction of axioms.

%%% Local Variables: 
%%% mode: latex
%%% TeX-master: "forcing_in_isabelle_zf"
%%% ispell-local-dictionary: "american"
%%% End: 
