\documentclass[9pt]{entcs} \usepackage{entcsmacro}
\usepackage{graphicx}
\sloppy

\input{header-draft}
\def\lastname{Gunther, Pagano, S\'anchez Terraf}
\begin{document}
\begin{frontmatter}
  \title{First steps towards a formalization of Forcing}
  \author{Emmanuel Gunther%\thanksref{ALL}
    \thanksref{myemail}}
  \address{FaMAF\\ Universidad Nacional de C\'ordoba\\
    C\'ordoba, Argentina} \author{Miguel Pagano\thanksref{coemail}}
  \address{FaMAF\\Universidad Nacional de C\'ordoba\\
    C\'ordoba, Argentina}
  \author{Pedro S\'anchez Terraf\thanksref{co2email}}
  \address{CIEM-FaMAF\\Universidad Nacional de C\'ordoba\\
    C\'ordoba, Argentina}
 \thanks[myemail]{Email:
    \href{mailto:gunther@famaf.unc.edu.ar} {\texttt{\normalshape
        gunther@famaf.unc.edu.ar}}} \thanks[coemail]{Email:
    \href{mailto:pagano@famaf.unc.edu.ar} {\texttt{\normalshape
        pagano@famaf.unc.edu.ar}}}  \thanks[co2email]{Email:
    \href{mailto:sterraf@famaf.unc.edu.ar} {\texttt{\normalshape
        sterraf@famaf.unc.edu.ar}}} 
 \thanks[ALL]{Supported by Secyt-UNC project 33620180100465CB.} 
\begin{abstract} 
\end{abstract}
\begin{keyword}
Isabelle/ZF, forcing, preorder, Rasiowa-Sikorski lemma, names, generic extension.
\end{keyword}
\end{frontmatter}

%%%%%%%%%%%%%%%%%%%%%%%%%%%%%%%%%%%%%%%%%%%%%%%%%%%%%%%%%%%%%%%%%%%%%%%%%%%%%%%%
\section{Introduction}\label{sec:introduction}

%-%-%-%-%-%-%-%-%-%-%-%-%-%-%-%-%-%-%-%-%-%-%-%-%-%-%-%-%-%-%-%-%-%-%-%-%-%-%-%
\subsection{Models of ZFC}

%-%-%-%-%-%-%-%-%-%-%-%-%-%-%-%-%-%-%-%-%-%-%-%-%-%-%-%-%-%-%-%-%-%-%-%-%-%-%-%
\subsection{Isabelle/ZF}



%%%%%%%%%%%%%%%%%%%%%%%%%%%%%%%%%%%%%%%%%%%%%%%%%%%%%%%%%%%%%%%%%%%%%%%%%%%%%%%%
\section{Forcing notions}\label{sec:forcing-posets}
\textit{Para no confundir, directamente usamos este nombre en lugar de
  ``forcing posets'' para los preórdenes con un máximo distinguido}

%-%-%-%-%-%-%-%-%-%-%-%-%-%-%-%-%-%-%-%-%-%-%-%-%-%-%-%-%-%-%-%-%-%-%-%-%-%-%-%
\subsection{The Rasiowa-Sikorski lemma}
In order to formalize of the proof of the Rasiowa-Sikorski lemma in
general, a version of the Axiom of Choice ($\AC$) must be proved. This
version is known as the \emph{Principle of Dependent choices}:
\begin{quote}
  ($\DC$) Let $\rr$ be a binary relation on $A$, and $a\in A$. If
  $\forall x\in A\,  \exists y\in A\, x\rr y$, then there exists
  $f:\om\to A$ such that $f(0)=a$ and $f(n)\rr f(n+1)$ for all
  $n\in\om$.
\end{quote}

A different version of $\DC$ (without the constraint $f(0)=a$, but
allowing a parameter that allows to prove an equivalence with $\AC$) was
formalized as part of the Isabelle/ZF library, along with many
statements that follow from or are equivalent to $\AC$. Since the
proofs there required a greater generality, it would have been very
contrived to derived this ``pointed'' version of $\DC$ from that
development. Instead, we preferred to give a direct proof based in the
one that appears in Moschovakis \cite{moschovakis1994notes}. Another
advantage is that this proof depends on the following version of $\AC$
(existence of choice functions on $\P(X)\sm \{\mty\}$):
\[
\exists f:\P(X)\sm \{\mty\}\to X \, \forall A\in \P(X)\sm \{\mty\}\ f(A)
    \in A.
\]
This statement is more standard than the one deemed ``$\AC$'' in the
Isabelle/ZF development, that involves a formula parameter and thus
involves a silent use of Replacement.


%%%%%%%%%%%%%%%%%%%%%%%%%%%%%%%%%%%%%%%%%%%%%%%%%%%%%%%%%%%%%%%%%%%%%%%%%%%%%%%%
\section{Names}


%%%%%%%%%%%%%%%%%%%%%%%%%%%%%%%%%%%%%%%%%%%%%%%%%%%%%%%%%%%%%%%%%%%%%%%%%%%%%%%%
\section{The generic extension}
Forcing is a technique to extend countable transitive models of
$\ZFC$. This process is guaranteed to preserve the $\ZFC$
axioms but allows to fine-tune what other first order properties the
extension will have. The first, both historically and in importance,
application of forcing was the proof that $\AC$ and $\CH$ were not
derivable from $\ZFC$; this was done by Cohen in its seminal work
\cite{Cohen-CH-PNAS}.

 

%%%%%%%%%%%%%%%%%%%%%%%%%%%%%%%%%%%%%%%%%%%%%%%%%%%%%%%%%%%%%%%%%%%%%%%%%%%%%%%%
\section{Future work}
\begin{itemize}
\item Develop an interface between Paulson's relativization results
  and countable models $M$ of $\ZFC$. This will show
  that every such $M$ is closed under well-founded recursion and, in
  particular, that contains names for each of its
  elements. Consequently, $M\sbq M[G]$.
\item Prove that the basic axioms of $\ZFC$ hold in the generic
  extension. 
\item After prototyping the forcing relation, prove the Axiom Scheme
  of Separation (the first that needs to use the machinery of forcing
  nontrivially).
\end{itemize}
\bibliographystyle{entcs}
\bibliography{citados,journals&preprints,conferencias}
\end{document}

%%% Local Variables:
%%% mode: latex
%%% ispell-local-dictionary: "american"
%%% End:
