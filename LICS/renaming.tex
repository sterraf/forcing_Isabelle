%%%%%%%%%%%%%%%%%%%%%%%%%%%%%%%%%%%%%%%%%%%%%%%%%%%%%%%%%%%%%%%%%%%%%%          
\section{Renaming}
\label{sec:renaming}
\newcommand{\renaming}[2]{(#1)[#2]}
\newcommand{\inFm}[2]{#1 \in #2}
\newcommand{\eqFm}[2]{#1 = #2}
\newcommand{\negFm}[1]{\neg #1}
\newcommand{\andFm}[2]{#1 \wedge #2}
\newcommand{\forallFm}[1]{\forall #1}

\newcommand{\inIFm}[2]{\mathsf{Member}(#1,#2)}
\newcommand{\eqIFm}[2]{\mathsf{Equal}(#1,#2)}
\newcommand{\nandIFm}[2]{\mathsf{Nand}(#1,#2)}
\newcommand{\forallIFm}[1]{\mathsf{Forall(#1)}}


In the course of our work we need to reason about renaming of formulas
and its effect on their satisfiability. Internalized formulas are
implemented using de Bruijn indices for variables and the arity of a
formula $\phi$ gives the least natural number containing all the free
variables in $\phi$. Following \citet{fiore-abssyn}, one can
understand the arity of a formula as the context of the free
variables; notice that the arity of $\forallFm{\phi}$ is the
predecessor of the arity of $\phi$. Renamings are, consequently,
mappings between finite sets; since we can think of $\mathsf{succ}(n)$
as the coproduct $1+n = \{0\} \cup \{1,\dots,n\}$, then given a
renaming $f \colon n \to m$, the 
unique morphism $\mathsf{id}_1+f \colon 1+n \to 1+m$ is used to rename
free variables in a quantified formula. 

\begin{definition}[Renaming]
  Let $\phi$ be a formula of arity $n$ and let $f \colon n \to m$, the
  renaming of $\phi$ by $f$, denoted $\renaming{\phi}{f}$, is defined
  by recursion on $\phi$:
  \begin{gather*}
    \renaming{\inFm{i}{j}}{f} = \inFm{f\,i}{f\,j}\\
    \renaming{\eqFm{i}{j}}{f} = \eqFm{f\,i}{f\,j}\\
    \renaming{\negFm{\phi}}{f} = \negFm{\renaming{\phi}{f}}\\
    \renaming{\andFm{\phi}{\psi}}{f} = \andFm{\renaming{\phi}{f}}{\renaming{\psi}{f}}\\
    \renaming{\forallFm{\phi}}{f} = \forallFm{\renaming{\phi}{\mathsf{id}_1+f}}
  \end{gather*}
\end{definition}

As usual, if $M$ is a set, $a_0,\dots,a_{n-1}$ are elements of $M$, and
$\phi$ is a formula of arity $n$, we write
\[
M,[a_0,\dots,a_{n-1}] \models \phi
\]
to denote that $\phi$ is satisfied by $M$ when $i$ is interpreted
as $a_i$ ($i=0,\dots,n-1$). We call the list $[a_0,\dots,a_{n-1}]$ the
\emph{environment}.

The action of renaming on environments re-indexes the variables. An
easy proof connects satisfaction with renamings.
\begin{lemma}
  \label{lem:renaming}
  Let $\phi$ be a formula of arity $n$, $f \colon n \to m$ be a
  renaming, and let $\rho=[a_1,\ldots,a_n]$ and
  $\rho'=[b_1,\ldots,b_m]$ be environments of length $n$ and $m$,
  respectively. If for all $i \in n$, $a_i = b_{j}$ where $j=f\,i$,
  then $M,\rho\models \phi$ is equivalent to
  $M,\rho' \models \renaming{\phi}{f}$.
\end{lemma}

An important resource in Isabelle/ZF is the facility for defining
inductive sets \cite{paulson2000fixedpoint,paulson1995set} together
with a principle for defining functions by structural recursion.
Internalized formulas are a prime example of this, so we define
a function \isa{ren} that associates to each formula an internalized
function that can be later applied to suitable arguments. Notice that
Paulson used \isa{Nand} because it is more economical.
\begin{isabelle}
\isamarkuptrue%
\isacommand{consts}\isamarkupfalse%
\ ren\ {\isacharcolon}{\isacharcolon}\ {\isachardoublequoteopen}i{\isacharequal}{\isachargreater}i{\isachardoublequoteclose}\isanewline
\isacommand{primrec}\isamarkupfalse%
\isanewline
\ {\isachardoublequoteopen}ren{\isacharparenleft}Member{\isacharparenleft}x{\isacharcomma}y{\isacharparenright}{\isacharparenright}\ {\isacharequal}\isanewline
\ \ {\isacharparenleft}{\isasymlambda}\ n\ {\isasymin}\ nat\ {\isachardot}\ {\isasymlambda}\ m\ {\isasymin}\ nat{\isachardot}\ {\isasymlambda}f\ {\isasymin}\ n\ {\isasymrightarrow}\ m{\isachardot}\ Member\ {\isacharparenleft}f{\isacharbackquote}x{\isacharcomma}\ f{\isacharbackquote}y{\isacharparenright}{\isacharparenright}{\isachardoublequoteclose}\isanewline
\ \isanewline
\ {\isachardoublequoteopen}ren{\isacharparenleft}Equal{\isacharparenleft}x{\isacharcomma}y{\isacharparenright}{\isacharparenright}\ {\isacharequal}\isanewline
\ \ {\isacharparenleft}{\isasymlambda}\ n\ {\isasymin}\ nat\ {\isachardot}\ {\isasymlambda}\ m\ {\isasymin}\ nat{\isachardot}\ {\isasymlambda}f\ {\isasymin}\ n\ {\isasymrightarrow}\ m{\isachardot}\ Equal\ {\isacharparenleft}f{\isacharbackquote}x{\isacharcomma}\ f{\isacharbackquote}y{\isacharparenright}{\isacharparenright}{\isachardoublequoteclose}\isanewline
\ \isanewline
\ {\isachardoublequoteopen}ren{\isacharparenleft}Nand{\isacharparenleft}p{\isacharcomma}q{\isacharparenright}{\isacharparenright}\ {\isacharequal}\isanewline
\ \ {\isacharparenleft}{\isasymlambda}\ n\ {\isasymin}\ nat\ {\isachardot}\ {\isasymlambda}\ m\ {\isasymin}\ nat{\isachardot}\ {\isasymlambda}f\ {\isasymin}\ n\ {\isasymrightarrow}\ m{\isachardot}\ \isanewline
\ \ \ Nand\ {\isacharparenleft}ren{\isacharparenleft}p{\isacharparenright}{\isacharbackquote}n{\isacharbackquote}m{\isacharbackquote}f{\isacharcomma}\ ren{\isacharparenleft}q{\isacharparenright}{\isacharbackquote}n{\isacharbackquote}m{\isacharbackquote}f{\isacharparenright}{\isacharparenright}{\isachardoublequoteclose}\isanewline
\ \isanewline
\ {\isachardoublequoteopen}ren{\isacharparenleft}Forall{\isacharparenleft}p{\isacharparenright}{\isacharparenright}\ {\isacharequal}\isanewline
\ \  {\isacharparenleft}{\isasymlambda}\ n\ {\isasymin}\ nat\ {\isachardot}\ {\isasymlambda}\ m\ {\isasymin}\ nat{\isachardot}\ {\isasymlambda}f\ {\isasymin}\ n\ {\isasymrightarrow}\ m{\isachardot}\ \isanewline
\ \ \ Forall\ {\isacharparenleft}ren{\isacharparenleft}p{\isacharparenright}{\isacharbackquote}succ{\isacharparenleft}n{\isacharparenright}{\isacharbackquote}succ{\isacharparenleft}m{\isacharparenright}{\isacharbackquote}sum{\isacharunderscore}id{\isacharparenleft}n{\isacharcomma}f{\isacharparenright}{\isacharparenright}{\isacharparenright}{\isachardoublequoteclose}
\end{isabelle}

In the last equation, \isa{sum{\isacharunderscore}id} corresponds to
the coproduct morphism $\mathsf{id}_{1}+f \colon 1 + n \to 1 +
n$. Since the schema for recursively defined functions does not allow
parameters, we are forced to return a function of three arguments
(\isa{n,m,f}). This also exposes some inconveniences of working in the
untyped realm of set theory; for example to use \isa{ren} we will need
to prove that the renaming is a function. Besides some auxiliary
results (for example that the application of renaming to suitable
arguments yields a formula), the main result corresponding to
Lemma~\ref{lem:renaming} is:
\begin{isabelle}
\isacommand{lemma}\isamarkupfalse%
\ sats{\isacharunderscore}iff{\isacharunderscore}sats{\isacharunderscore}ren\ {\isacharcolon}\ \isanewline
\ \ \isakeyword{fixes}\ {\isasymphi}\isanewline
\ \ \isakeyword{assumes}\ {\isachardoublequoteopen}{\isasymphi}\ {\isasymin}\ formula{\isachardoublequoteclose}\isanewline
\ \ \isakeyword{shows}\ \ {\isachardoublequoteopen}{\isasymAnd}\ n\ m\ {\isasymrho}\ {\isasymrho}{\isacharprime}\ f\ {\isachardot}\ \isanewline
\ \ {\isasymlbrakk}n{\isasymin}nat\ {\isacharsemicolon}\ m{\isasymin}nat\ {\isacharsemicolon}\ f\ {\isasymin}\ n{\isasymrightarrow}m\ {\isacharsemicolon}\ arity{\isacharparenleft}{\isasymphi}{\isacharparenright}\ {\isasymle}\ n\ {\isacharsemicolon}\isanewline
\ \ \ \ \ {\isasymrho}\ {\isasymin}\ list{\isacharparenleft}M{\isacharparenright}\ {\isacharsemicolon}\ {\isasymrho}{\isacharprime}\ {\isasymin}\ list{\isacharparenleft}M{\isacharparenright}\ {\isacharsemicolon}\ \isanewline
\ \ \ {\isasymAnd}\ i\ {\isachardot}\ i{\isacharless}n\ {\isasymLongrightarrow}\ nth{\isacharparenleft}i{\isacharcomma}{\isasymrho}{\isacharparenright}\ {\isacharequal}\ nth{\isacharparenleft}f{\isacharbackquote}i{\isacharcomma}{\isasymrho}{\isacharprime}{\isacharparenright}\ {\isasymrbrakk}\ {\isasymLongrightarrow}\isanewline
\ \ sats{\isacharparenleft}M{\isacharcomma}{\isasymphi}{\isacharcomma}{\isasymrho}{\isacharparenright}\ {\isasymlongleftrightarrow}\ sats{\isacharparenleft}M{\isacharcomma}ren{\isacharparenleft}{\isasymphi}{\isacharparenright}{\isacharbackquote}n{\isacharbackquote}m{\isacharbackquote}f{\isacharcomma}{\isasymrho}{\isacharprime}{\isacharparenright}{\isachardoublequoteclose}\end{isabelle}

All our uses of this lemma involve concrete renamings on small
numbers, but we also tested it with more abstract ones for arbitrary
numbers. All the renamings of the first kind follow the same pattern
and, more importantly, share equal proofs. We would like to develop
some \texttt{ML} tools in order to automatize this.
%% We think that it should be possible, and clearer,
%% to express all the current renamings in the theory \isa{Formula} using
%% our approach. 


%%% Local Variables: 
%%% mode: latex
%%% TeX-master: "Separation_In_MG"
%%% ispell-local-dictionary: "american"
%%% End: 
