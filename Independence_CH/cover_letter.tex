% This is samplepaper.tex, a sample chapter demonstrating the
% LLNCS macro package for Springer Computer Science proceedings;
% Version 2.20 of 2017/10/04
%
\documentclass[runningheads]{llncs}
%
%\usepackage[utf8]{inputenc}
%\usepackage[final]{microtype}
\usepackage{isabelle_indepCH,isabellesym_indepCH}
\usepackage{booktabs,array,threeparttable}
\input{header-indepCH}
\usepackage{graphicx}
% Used for displaying a sample figure. If possible, figure files should
% be included in EPS format.
%
\hypersetup{
  colorlinks,
  urlcolor={blue},
  linkcolor={blue!50!black},
  citecolor={blue!50!black},
}
% If you use the hyperref package, please uncomment the following line
% to display URLs in blue roman font according to Springer's eBook style:
% \renewcommand\UrlFont{\color{blue}\rmfamily}
\DeclareUnicodeCharacter{2200}{\ensuremath{\forall}}
\DeclareUnicodeCharacter{2203}{\ensuremath{\exists}}
\DeclareUnicodeCharacter{2227}{\ensuremath{\wedge}}
\DeclareUnicodeCharacter{2228}{\ensuremath{\vee}}
\DeclareUnicodeCharacter{22C0}{\ensuremath{\bigwedge}}
\DeclareUnicodeCharacter{2261}{\ensuremath{\equiv}}
\DeclareUnicodeCharacter{27F9}{\ensuremath{\Longrightarrow}}
\DeclareUnicodeCharacter{27F7}{\ensuremath{\leftrightarrow}}
\DeclareUnicodeCharacter{27F6}{\ensuremath{\longrightarrow}}
\DeclareUnicodeCharacter{03BB}{\ensuremath{\lambda}}

\DeclareUnicodeCharacter{2208}{\ensuremath{\in}}
\DeclareUnicodeCharacter{2209}{\ensuremath{\not\in}}
\DeclareUnicodeCharacter{2286}{\ensuremath{\subseteq}}
\DeclareUnicodeCharacter{22C3}{\ensuremath{\bigcup}}
\DeclareUnicodeCharacter{21D2}{\ensuremath{\Rightarrow}}
\usepackage{array}
\newcounter{replInstCount}
\setcounter{replInstCount}{0}

\newcounter{LamReplCount}
\setcounter{LamReplCount}{0}
\begin{document}
%
\title{Cover Letter: Replies to Referee
}
%
\titlerunning{Formalization of ctm forcing}%Lessons after formalizing ctm forcing}
% If the paper title is too long for the running head, you can set
% an abbreviated paper title here
%
\author{Emmanuel Gunther\inst{1} \and
Miguel Pagano\inst{1} \and \\
Pedro Sánchez Terraf\inst{1,2}%\orcidID{0000-0003-3928-6942}
\and
Matías Steinberg\inst{1}
}
%
\authorrunning{E.~Gunther, M.~Pagano, P.~Sánchez Terraf, M.~Steinberg}
% First names are abbreviated in the running head.
% If there are more than two authors, 'et al.' is used.
%
\institute{Universidad Nacional de C\'ordoba. 
  \\  Facultad de Matem\'atica, Astronom\'{\i}a,  F\'{\i}sica y
  Computaci\'on.
  \and
    Centro de Investigaci\'on y Estudios de Matem\'atica (CIEM-FaMAF),
    Conicet. C\'ordoba. Argentina. \\
    \email{\{gunther,sterraf\}@famaf.unc.edu.ar\\
        \{miguel.pagano,matias.steinberg\}@unc.edu.ar}
}
%
\maketitle              % typeset the header of the contribution
%
First and foremost, we are very grateful for your comments and
pointers. After the replies in this cover letter, we show the
fragments of the paper with changes; additions and deletions are
highlighted in context. We hope they are useful for you.

Please give us notice if any of the corrections is insufficient.

\begin{verbatim}
p 5 l 33: Isabelle Z/F reaches Hessenberg's |A|.|A| = |A|: I
 have no idea what this means.
\end{verbatim}

We expanded that section.

\begin{verbatim}
p 7 l 40: It would help to explain the type of cardinal_rel, in
 particular explain that the first argument is M.
\end{verbatim}

Same.

\begin{verbatim}
p 8 l 19: You haven't said what a locale is; it would help to
 explain that the text that follows the term is an explanation
 of that.
\end{verbatim}

We did, in the last paragraph of Sect.~2.2. Should we expand on
that?

\begin{verbatim}
p 8 bottom: it would help to show separation(N, Q) and
 strong_replacement(N, R) and explain the formula at the bottom
 of the page.
\end{verbatim}

Please indicate if the edition makes things clearer.

\begin{verbatim}
The example on page 11 is nice.
\end{verbatim}

Much appreciated!

\begin{verbatim}
p 13: it would be nice to say more about the Delta system lemma
 and its proof (and whether it was relatively hard or easy
 compared to other aspects of the formalization).
\end{verbatim}

We added comments to Appendix C.

\begin{verbatim}
p 16 last sentence of the first full paragraph ("Even if we
 intended ..."): I don't understand what is going on here. Do
 you mean relative consistency in terms of probability?  Is the
 difficult proving soundness and completeness of first-order
 logic?  You should clarify what this paragraph aims to get
 across.
\end{verbatim}

We meant a finitary, relative consistency proof. We hope now
that is better expressed on the manuscript.

\begin{verbatim}
p 17 toward the bottom: "They actually wrote the blueprint for
 the formalization *afterwards* it was complete": this is
 false. Commelin and Topaz have given many talks explaining how
 the blueprint was used, and how essential it was to guiding the
 collaboration. The project made key use of an online version of
 the blueprint that showed the status of the formalization and
 provided links from the informal text to the formal proof. The
 prior sentence sounds wrong too. Even Scholze's post describes
 the process of revising the proof and discovering new
 approaches as they proceeded. If the authors want to make
 claims about LTE, they should check the claims with a member of
 the project to make sure they are accurate.
\end{verbatim}

We checked with Commelin and he acknowledged that there was some
back and forth between the formalization effort and the
blueprint. But he did not confirm that they shared the same kind of
“feeling” we were trying to convey in that paragraph, so we deleted
the comment. 

\begin{verbatim}
p 19 l -15: It would be helpful to say something about where the
 blowup in length comes from.
\end{verbatim}

Done.

\begin{verbatim}
  If length is an issue, the appendices could be shortened, but
 the data is interesting and I recommend keeping them if
 possible.
\end{verbatim}

Thanks again for your support.
\end{document}

%%% Local Variables: 
%%% mode: latex
%%% ispell-local-dictionary: "american"
%%% End: 
