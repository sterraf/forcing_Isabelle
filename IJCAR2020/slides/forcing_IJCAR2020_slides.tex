\documentclass[english]{beamer}
%\usepackage{beamerarticle}
\usepackage{graphicx}
\usepackage{babel}
\usepackage{tikz}
\usetikzlibrary{arrows,shapes}
\usepackage[all]{xy}
\usepackage[invisible]{arxiv-beamer-theme}

\usepackage[utf8]{inputenc}
%%%
%%% Natbib must be here!! (after Beamer stuff)
\usepackage[authoryear,square]{natbib}
\input{header-slides}
\usepackage{isabelle,isabellesym}
\newcommand{\uscore}{\isacharunderscore}

\CompileMatrices

\title[Forcing in Isabelle/ZF]{Formalization of Forcing in Isabelle/ZF}

\author[E.~Gunther, M.~Pagano, PST]{E.~Gunther \qquad M.~Pagano \qquad P.~Sánchez Terraf\thanks{Supported by
    CONICET and SeCyT-UNC} 
}
\institute[UNC]{CIEM-FaMAF --- Universidad Nacional de Córdoba}

\date[IJCAR 2020]{International Joint Conference on Automated
  Reasoning\\ Paris (Virtual),\ \  2020 / 07 / 02}

\subject{Logic in Computer Science}

\pgfdeclareimage[height=3ex]{UNC-logo}{unc_400_Blanco_Negro}
\logo{\pgfuseimage{UNC-logo}}

%% Para usar overlays con tikz
\tikzset{onslide/.code args={<#1>#2}{%
    \only<#1>{\pgfkeysalso{#2}} % \pgfkeysalso doesn't change the path
}}

\begin{document}
%\nobibliography*

\begin{frame}[plain]
  \titlepage
  \begin{center}
    \insertlogo
  \end{center}
\end{frame}
%
\begin{frame}
  \frametitle{Summary}
  \tableofcontents
  \transwipe
\end{frame}
%
%%%%%%%%%%%%%%%%%%%%%%%%%%%%%%%%%%%%%%%%%%%%%%%%%%%
\section{Introduction}

%-%-%-%-%-%-%-%-%-%-%-%-%-%-%-%-%-%-%-%-%-%-%-%-%-%
\subsection{Why Isabelle/ZF?}

\begin{frame}
  \frametitle{Why Isabelle/ZF?}
  \begin{block}{Pros}
    \begin{itemize}
    \item Most advanced set theory formalized (around 2017).
    \item Closer to the mathematical practice of forcing.
    \item Structured proof language Isar
      \citep{DBLP:conf/tphol/Wenzel99}.
    \item Weak metatheory.
    \end{itemize}
  \end{block}
  %
  \pause
  %
  \begin{block}{Cons}
    \begin{itemize}
    \item A fraction of automation of Isabelle (\texttt{sledgehammer},
      etc).
    \item Untyped, and too weak a metatheory.
    \end{itemize}
  \end{block}
\end{frame}

\begin{frame}
  \frametitle{Isabelle/ZF \citep{DBLP:journals/jar/PaulsonG96}}
  \begin{itemize}
  \item<+-> An object logic of Isabelle axiomatized over the
    intuitionistic fragment of HOL \emph{Pure}. 
  \item<+-> It postulates \alert<3->{two types}: \isatt{i}
    (sets) and \isatt{o} (booleans). \visible<3->{\alert{Not
      inductively defined!}}
    \pause
  \item<+-> More stuff.
  \end{itemize}
\end{frame}

%-%-%-%-%-%-%-%-%-%-%-%-%-%-%-%-%-%-%-%-%-%-%-%-%-%
\subsection{Paulson's \texttt{ZF-Constructible}}

\begin{frame}
  \frametitle{Paulson's \texttt{ZF-Constructible}}

\end{frame}


%-%-%-%-%-%-%-%-%-%-%-%-%-%-%-%-%-%-%-%-%-%-%-%-%-%
\subsection{The ctm approach to forcing}

\begin{frame}
  \frametitle{The ctm approach to forcing, I}
  \begin{block}{Countable transitive model (ctm) of $\ZF$}
    $\lb M, E\rb \models \ZF$ where
    \begin{itemize}
    \item $M$ is countable.
    \item $E \defi {\in}\restriction M$.
    \item $x\in y \in M \implies x\in M$.
    \end{itemize}
  \end{block}
  %
  \pause%
  %
  \textbf{Note}. If $\lb N, R\rb \models \ZF$ with $R$ wellfounded, then there
  exists a ctm $M$ of $\ZF$.
  %
  \pause
  %
  \begin{block}{Forcing notions and generics}
    \begin{itemize}
    \item $\lb \PP, {\preceq} ,\1\rb$ preorder with top.
    \item A filter $G\sbq \PP$ is \textbf{$M$-generic} if it meets
      every dense $D\sbq \PP$ in $M$.
    \end{itemize}
  \end{block}
\end{frame}

\begin{frame}
  \frametitle{The ctm approach to forcing, II}
  Given $G\sbq \PP$, we can adjoin it to $M$ to form the
  \textbf{generic extension} $M[G]$.
  \pause
  Every $a\in M[G]$ is coded by
  some $\punto{a}\in M$ through the definable (in $M$) function $\val$:
  \[
  M[G]\defi \{ \val(G,\punto{a}) : \punto{a}\in M\}
  \]
  \pause
  \emph{Fundamentally}, \textbf{truth} in $M[G]$ is coded in $M$ by the
  function $\forceisa$.
  \pause
  \begin{theorem}[Cohen, 1963]
    \begin{enumerate}
    \item $\bigl(\forall G\visible<5->{\alert{\ni p}}.\quad M[G], [\val(G,\punto{a})]
      \models \phi\bigr)
      \iff
      M,
      [\only<-4>{\1}\alert{\only<5->{p\hspace{0.2ex}}},\PP,\preceq,\punto{a}]\models
      \forceisa(\phi)$.
    \item $\forceisa$ is definable over $M$.\phantom{$\punto{a}).\models \phi\bigr)
      \iff M,$}\visible<6->{%
      $\overset{\searrow}{}$ \ 
      $\alert{p} \forces^M_{\PP,\preceq} \phi(\punto{a})$
      \  $\overset{\swarrow}{}$}
      \pause% 
      \pause% 
      \pause%       
    \item For $env\in\mathsf{list}(M)$,
      \[
      M[G], \mathrm{map}(\val(G),env) \models \phi)
      \iff
      \exists p\in G.\ M, [p,\preceq,\PP] \mathbin{{+}\!\!\!\!{+}} env \models
      \forceisa(\phi).
      \]
    \end{enumerate}
  \end{theorem} 
\end{frame}

\begin{frame}
  \frametitle{The ctm approach to forcing, III}
  By choosing the forcing notion $\lb \PP, {\preceq} ,\1\rb$
  appropriately one can tune the first order properties of  $M[G]$ (for
  any generic $G$).
  \pause

  
\end{frame}

\begin{frame}
  \frametitle{Other approaches}
  \begin{itemize}
  \item<+-> Set theory over Isabelle/HOL:
    \begin{itemize}
    \item \isatt{HOLZF} \citep{DBLP:conf/ictac/Obua06}
    \item \isatt{ZFC{\uscore}in{\uscore}HOL}
      \citep{ZFC_in_HOL-AFP}
    \end{itemize}
  \item<+-> \alert{Lean}: Full formalization of the Boolean-valued approach to
    forcing and the independence of $\CH$ \citep{DBLP:conf/cpp/HanD20}.
  \end{itemize}
  %
  \begin{block}{A word on consistency strength}<+->
    \begin{description}
      \setlength{\labelwidth}{12em}
      \setlength{\labelsep}{2em}
      \setlength{\itemindent}{6em}
    \item[Isabelle/ZF + ctm]  (far) less than $\ZF$ +  one
      inaccessible.
    \item[\isatt{HOLZF},
      \isatt{ZFC{\uscore}in{\uscore}HOL}]
      approximately $\ZF$ +  one   inaccessible. 
    \item[Lean (CiC)] $\ZF$ + $\omega$ inaccessibles \citep{carneiro-ms-thesis}. 
    \end{description}
  \end{block}
\end{frame}

%% \begin{frame}
%%   \frametitle{A word on consistency}
%% \end{frame}

%%%%%%%%%%%%%%%%%%%%%%%%%%%%%%%%%%%%%%%%%%%%%%%%%%%
\section{The development}

%-%-%-%-%-%-%-%-%-%-%-%-%-%-%-%-%-%-%-%-%-%-%-%-%-%
\subsection{What did we accomplish?}

\begin{frame}
  \frametitle{What did we accomplish?}
  %
  \begin{shadowblock}{}
    \begin{enumerate}
    \item<+-> We adapted \texttt{ZF-Constructible} to obtain
      absoluteness results for nonempty transitive classes ($\leftarrow$
      \alert{in Isabelle2020}).
      %
      \pause
      
      Around 40 absoluteness/closure lemmas now hold using weaker
      hypotheses on the class $M$ (most of them, just that $M$ is
      transitive and nonempty). 
      %
      \pause
    \item<+-> We formalized the formula transformer $\forceisa$ and
      hence the forcing relation $\forces$, and proved the Fundamental
      Theorems.
    \item<+-> We showed that generic extensions of ctms of $\ZF$ are also
      ctms of $\ZF$ (respectively, adding $\AC$).
    \item<+-> We provided the forcing notion that adds a Cohen real,
      therefore proving the existence of a nontrivial extension.
    \end{enumerate}
  \end{shadowblock}
\end{frame}

\begin{frame}
  \frametitle{Some details}
  \begin{enumerate}
  \item<+-> Modifications to \texttt{ZF-Constructible}.


  \item<+-> We formalized the formula transformer $\forceisa$ and
    hence the forcing relation $\forces$, and proved the Fundamental
    Theorems.
  \item<+-> We showed that generic extensions of ctms of $\ZF$ are also
    ctms of $\ZF$ (respectively, adding $\AC$).
  \item<+-> We provided the forcing notion that adds a Cohen real,
    therefore proving the existence of a nontrivial extension.
  \end{enumerate}
\end{frame}

%-%-%-%-%-%-%-%-%-%-%-%-%-%-%-%-%-%-%-%-%-%-%-%-%-%
\subsection{A map}

\begin{frame}
  \begin{center}
    \only<01>{\includegraphics[scale=0.23]{kunen_powerset_01.png} \\ \includegraphics[scale=0.26]{isa_converted_01.png}}%
    \only<02>{\includegraphics[scale=0.23]{kunen_powerset_02.png} \\ \includegraphics[scale=0.26]{isa_converted_02.png}}%
    \only<03>{\includegraphics[scale=0.23]{kunen_powerset_03.png} \\ \includegraphics[scale=0.26]{isa_converted_03.png}}%
    \only<04>{\includegraphics[scale=0.23]{kunen_powerset_04.png} \\ \includegraphics[scale=0.26]{isa_converted_04.png}}%
    \only<05>{\includegraphics[scale=0.23]{kunen_powerset_05.png} \\ \includegraphics[scale=0.26]{isa_converted_05.png}}%
    \only<06>{\includegraphics[scale=0.23]{kunen_powerset_06.png} \\ \includegraphics[scale=0.26]{isa_converted_06.png}}%
    \only<07>{\includegraphics[scale=0.23]{kunen_powerset_07.png} \\ \includegraphics[scale=0.26]{isa_converted_07.png}}%
    \only<08>{\includegraphics[scale=0.23]{kunen_powerset_08.png} \\ \includegraphics[scale=0.26]{isa_converted_08.png}}%
    \only<09>{\includegraphics[scale=0.23]{kunen_powerset_09.png} \\ \includegraphics[scale=0.26]{isa_converted_09.png}}%
    \only<10>{\includegraphics[scale=0.23]{kunen_powerset_10.png} \\ \includegraphics[scale=0.26]{isa_converted_10.png}}%
    \only<11>{\includegraphics[scale=0.23]{kunen_powerset_11.png} \\ \includegraphics[scale=0.26]{isa_converted_11.png}}%
    \only<12>{\includegraphics[scale=0.23]{kunen_powerset_12.png} \\ \includegraphics[scale=0.26]{isa_converted_12.png}}%
    \only<13>{\includegraphics[scale=0.23]{kunen_powerset_13.png} \\ \includegraphics[scale=0.26]{isa_converted_13.png}}%
    \only<14>{\includegraphics[scale=0.23]{kunen_powerset_14.png} \\ \includegraphics[scale=0.26]{isa_converted_14.png}}%
    \only<15>{\includegraphics[scale=0.23]{kunen_powerset_15.png} \\ \includegraphics[scale=0.26]{isa_converted_15.png}}%
    \only<16>{\includegraphics[scale=0.23]{kunen_powerset_16.png} \\ \includegraphics[scale=0.26]{isa_converted_16.png}}%
    \only<17>{\includegraphics[scale=0.23]{kunen_powerset_17.png} \\ \includegraphics[scale=0.26]{isa_converted_17.png}}%
    \only<18>{\includegraphics[scale=0.23]{kunen_powerset_18.png} \\ \includegraphics[scale=0.26]{isa_converted_18.png}}%
    \only<19>{\includegraphics[scale=0.23]{kunen_powerset_19.png} \\ \includegraphics[scale=0.26]{isa_converted_19.png}}%
    \only<20>{\includegraphics[scale=0.23]{kunen_powerset_20.png} \\ \includegraphics[scale=0.26]{isa_converted_20.png}}%
  \end{center}
\end{frame}

\begin{frame}
  \frametitle{Locale structure} 
             {  
               \renewcommand{\arraystretch}{1.5}               \begin{tabular}{rcl}
                 \texttt{forcing{\uscore}notion} & = & preorder $\PP$ with top. \\
                 %    \texttt{M{\uscore}ZF} & = & set model of the $\ZF$ axioms \\ 
                 \texttt{M{\uscore}ZF{\uscore}trans} & = & set model $M$ of the $\ZF$
                 axioms \alert{+}  $M$ transitive \\ 
                 \texttt{M{\uscore}ctm} & = &  \texttt{M{\uscore}ZF{\uscore}trans} \alert{+}
                 $M$ countable \\
                 \texttt{forcing{\uscore}data} & =  & \texttt{M{\uscore}ctm} \alert{+}
                 \texttt{forcing{\uscore}notion} in $M$\\
                 \texttt{separative{\uscore}notion} & = &
                 \texttt{forcing{\uscore}notion} \alert{+} $\PP$ separative \\
                 \texttt{M{\uscore}ctm{\uscore}separative} & = &
                 \texttt{forcing{\uscore}data} \alert{+} \texttt{separative{\uscore}notion}
               \end{tabular} 
             }
\end{frame}

%-%-%-%-%-%-%-%-%-%-%-%-%-%-%-%-%-%-%-%-%-%-%-%-%-%
\subsection{}

\begin{frame}
  \frametitle{}
\end{frame}

%%%%%%%%%%%%%%%%%%%%%%%%%%%%%%%%%%%%%%%%%%%%%%%%%%%
\section{}
%-%-%-%-%-%-%-%-%-%-%-%-%-%-%-%-%-%-%-%-%-%-%-%-%-%
\subsection{}

\begin{frame}
  \frametitle{}
\end{frame}

%%%%%%%%%%%%%%%%%%%%%%%%%%%%%%%%%%%%%%%%%%%%%%%%%%%
\section{Looking forward}

\begin{frame}
  \frametitle{Looking forward}
  \begin{block}{Formalizing math}
    \begin{itemize}
    \item<+-> Cofinality, K\H{o}nig's Theorem, Shanin's $\Delta$-system Lemma.
    \item<+-> Forcing notion for adding $\kappa$ Cohen reals.
    \item<+-> Theorems on preservation of cardinals.
    \end{itemize}
  \end{block}
  %
  \begin{block}{Technical aids}<+->
    \begin{itemize}
    \item<.-> Automatic relativization and proof of absoluteness of concepts.
    \item<+-> ``Relative functions'' (e.g., $\P^M$, $\card{\cdot}^M$, $\cf^M$).
    \item<+-> 
    \end{itemize}
  \end{block}
\end{frame}

%-%-%-%-%-%-%-%-%-%-%-%-%-%-%-%-%-%-%-%-%-%-%-%-%-%
\begin{frame}
  \begin{shadowblock}{}
    \begin{center}
      {\Huge Thank you!}
    \end{center}
  \end{shadowblock}
\end{frame}

\begin{frame}
  \frametitle{References}
  \bibliographystyle{mi-estilo-else}
  \bibliography{../forcing_in_isabelle_zf}
\end{frame}

\end{document}

%% - Más bola a la metateoría (ZF).
%% 
%% * es weak ---> falta de inducción/recursión, y el objeto
%% central (forces) se define así.
%% * too strong ---> not possible to have finitary consistency proofs
%% 
%% - ctm approach: bondades 
%% 
%% * absolutez de ordinal, etc
%% * Modelos a dos valores.
%% 
%% - no definir locales, hablar menos de la arquitectura modular
%% (no tanto del codigo en si)
%% 
%% - Pruebas mostrar:
%% 
%% density_mem ?
%% powerset?
%% choice? 



%%% Local Variables: 
%%% mode: latex
%%% ispell-local-dictionary: "american"
%%% End: 
