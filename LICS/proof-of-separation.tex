%%%%%%%%%%%%%%%%%%%%%%%%%%%%%%%%%%%%%%%%%%%%%%%%%%%%%%%%%%%%%%%%%%%%%%          
\section{Proof of Separation}

This proof can be found in the file \verb|Separation_Axiom.thy| of the
development, which we proceed to discuss.

The key technical result is the following:
\begin{isabelle}
  \isacommand{lemma}\isamarkupfalse%
  \ Collect{\isacharunderscore}sats{\isacharunderscore}in{\isacharunderscore}MG\ {\isacharcolon}\isanewline
  \ \ \isakeyword{assumes}\isanewline
  \ \ \ \ {\isachardoublequoteopen}{\isasympi}\ {\isasymin}\ M{\isachardoublequoteclose}\ {\isachardoublequoteopen}{\isasymsigma}\ {\isasymin}\ M{\isachardoublequoteclose}\ {\isachardoublequoteopen}val{\isacharparenleft}G{\isacharcomma}\ {\isasympi}{\isacharparenright}\ {\isacharequal}\ c{\isachardoublequoteclose}\ {\isachardoublequoteopen}val{\isacharparenleft}G{\isacharcomma}\ {\isasymsigma}{\isacharparenright}\ {\isacharequal}\ w{\isachardoublequoteclose}\isanewline
  \ \ \ \ {\isachardoublequoteopen}{\isasymphi}\ {\isasymin}\ formula{\isachardoublequoteclose}\ {\isachardoublequoteopen}arity{\isacharparenleft}{\isasymphi}{\isacharparenright}\ {\isasymle}\ {\isadigit{2}}{\isachardoublequoteclose}\isanewline
  \ \ \isakeyword{shows}\ \ \ \ \isanewline
  \ \ \ \ {\isachardoublequoteopen}{\isacharbraceleft}x{\isasymin}c{\isachardot}\ sats{\isacharparenleft}M{\isacharbrackleft}G{\isacharbrackright}{\isacharcomma}\ {\isasymphi}{\isacharcomma}\ {\isacharbrackleft}x{\isacharcomma}\ w{\isacharbrackright}{\isacharparenright}{\isacharbraceright}{\isasymin}\ M{\isacharbrackleft}G{\isacharbrackright}{\isachardoublequoteclose}
\end{isabelle}
%
From this, using absoluteness, we will be able to derive the
$\phi$-instance of Separation. 

To show that   
\[
S\defi\{x\in c : M[G]\models \phi(x,w)\} \in M[G],
\]
it is enough to provide a name $n\in M$ for this set.
 
The candidate name is
\[
n \defi \{u \in\dom(\pi)\times\PP :M,[u,\PP,\leq,\1,\sig,\pi]\models \psi\}
\]
where
\[
\psi \defi \exists \th\, p.\ x_0=\lb\th,p\rb \y 
   \forceisa(\th\in x_5\y\phi(\th,x_4)).
\]
The fact that $n\in M$ follows by an application of a six-variable
instance of Separation in $M$ (lemma \isatt{six{\isacharunderscore}sep{\isacharunderscore}aux}).

Almost a third part of the proof involves the syntactic handling of
internalized formulas and permutation of variables. The more
substantive portion concerns proving that actually $\val(G,n)=S$.

Let's first focus into the predicate 
\[
M,[u,\PP,\leq,\1,\sig,\pi]\models \psi
\]
defining $n$ by separation. By definition of the satisfaction
relation and permuting variables, we have it is equivalent to the fact
that there exist $\th,p\in M$ with   $u=\lb\th,p\rb$  and 
\[
M,[\PP,\leq,\1,p,\th,\sig,\pi]\models \forceisa(x_4\in
x_6\y\phi(x_4,x_5)). 
\]
% (Note that the variable $x_7$ is not used.)
This, in turn is equivalent by the Definition of Forcing to: For all $M$-generic
filters $F$ such that $p\in F$, 
\[
M,[\val(G,\th),\val(G,\sig),\val(G,\pi)]\models x_4\in
x_6\y\phi(x_4,x_5). 
\] 


%% The proof of
%% \isatt{Collect{\isacharunderscore}sats{\isacharunderscore}in{\isacharunderscore}MG}
%% has three parts:
%% \begin{enumerate}
%% \item Definition of the name $n$;
%% \item Proving that 
%% \item 
%% \end{enumerate}

%%% Local Variables: 
%%% mode: latex
%%% TeX-master: "Separation_In_MG"
%%% ispell-local-dictionary: "american"
%%% End: 
