\section{Recursions in cofinality}\label{sec:recursions-cofinality}

As we mentioned near the end of
Section~\ref{sec:aims-formalization-planning}, we decided to minimize
the requirements being formalized in order to achieve our immediate
goal. In particular, the treatment of cofinality in the companion
project \cite{Delta_System_Lemma-AFP} was left behind.

We already observed that well-founded, and in particular transfinite,
recursion is not easily dealt with in Isabelle/ZF. Nevertheless, and
mainly as a curiosity, we found out that only one recursive
construction is needed for the development of the basic theory of
cofinality (as in \cite[Sect.~I.13]{kunen2011set}), which is used in
the proof of the following “factorization” lemma:

\begin{lemma}
  Let $\del,\ga\in\Ord$ and assume $f:\del\to\ga$ is cofinal.  There exists
  a strictly increasing $g:\cf(\ga)\to \del$ such that $f\circ g$ is
  strictly increasing and cofinal in $\ga$. Moreover, if $f$ is
  strictly increasing, then $g$ must also be cofinal.
\end{lemma}

It turns out that the rest of the basic results on cofinality (namely,
idempotence of $\cf$, that regular ordinals are cardinals, the
cofinality of Alephs, König's Theorem) follow easily from the previous
Lemma by “algebraic” reasoning only.

We therefore expect that the relativization of these
results be straightforward, when time permits.

%%% Local Variables:
%%% mode: latex
%%% TeX-master: "independence_ch_isabelle"
%%% ispell-local-dictionary: "american"
%%% End:
