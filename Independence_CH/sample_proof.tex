\section{A sample formal proof}
\label{sec:sample-formal-proof}

We present a fragment of the formal version of the proof that the
Powerset Axiom holds in a generic extension, which also serves to
illustrate the Isar dialect of Isabelle.

We quote the relevant
paragraph of Kunen's \cite[Thm.~IV.2.27]{kunen2011set}:
\begin{quote}
  For Power Set (similarly to Union above), it is sufficient to prove
  that whenever $a \in M[G]$, there is a $b \in M[G]$ such that
  $\mathcal{P}(a) \cap M[G] \subseteq b$. Fix $\tau \in
  M^{\mathbb{P}}$ such that $\tau_{G}=a$. Let
  $Q=(\mathcal{P}(\operatorname{dom}(\tau) \times
  \mathbb{P}))^{M}$. This is the set of all names $\vartheta \in
  M^{\mathbb{P}}$ such that $\operatorname{dom}(\vartheta) \subseteq
  \operatorname{dom}(\tau)$. Let $\pi=Q \times\{\1\}$ and let
  $b=\pi_{G}=$ $\left\{\vartheta_{G}: \vartheta \in Q\right\}$. Now,
  consider any $c \in \mathcal{P}(a) \cap M[G]$; we need to show that
  $c \in b$. Fix $\chi \in M^{\mathbb{P}}$ such that
  $\chi_{G}=c$, and let $\vartheta=\{\langle\sigma, p\rangle:
  \sigma \in \operatorname{dom}(\tau) \wedge p \Vdash \sigma \in
  \chi\}$; $\vartheta \in M$ by the Definability Lemma. Since
  $\vartheta \in Q$, we are done if we can show that
  $\vartheta_{G}=c$.
\end{quote}
The assumption $a\in M[G]$ appears in the lemma statement, and the
goal involving $b$ in the first sentence will appear below (signaled
by “{\small (**)}”); formalized
material necessarily tends to be much more linear than usual prose. In
what follows, we
will intersperse the relevant passages of the proof.
\begin{isabelle}
\isacommand{lemma}\isamarkupfalse%
\ Pow{\isacharunderscore}{\kern0pt}inter{\isacharunderscore}{\kern0pt}MG{\isacharcolon}{\kern0pt}\isanewline
\ \ \isakeyword{assumes}\ {\isachardoublequoteopen}a{\isasymin}M{\isacharbrackleft}{\kern0pt}G{\isacharbrackright}{\kern0pt}{\isachardoublequoteclose}\isanewline
\ \ \isakeyword{shows}\ {\isachardoublequoteopen}Pow{\isacharparenleft}{\kern0pt}a{\isacharparenright}{\kern0pt}\ {\isasyminter}\ M{\isacharbrackleft}{\kern0pt}G{\isacharbrackright}{\kern0pt}\ {\isasymin}\ M{\isacharbrackleft}{\kern0pt}G{\isacharbrackright}{\kern0pt}{\isachardoublequoteclose}\isanewline
%
\isacommand{proof}\isamarkupfalse%
\ {\isacharminus}{\kern0pt}
\end{isabelle}
\textit{Fix $\tau \in  M^{\mathbb{P}}$ such that $\tau_{G}=a$.}
\begin{isabelle}
\ \ \isacommand{from}\isamarkupfalse%
\ assms\isanewline
\ \ \isacommand{obtain}\isamarkupfalse%
\ {\isasymtau}\ \isakeyword{where}\ {\isachardoublequoteopen}{\isasymtau}\ {\isasymin}\ M{\isachardoublequoteclose}\ {\isachardoublequoteopen}val{\isacharparenleft}{\kern0pt}G{\isacharcomma}{\kern0pt}\ {\isasymtau}{\isacharparenright}{\kern0pt}\ {\isacharequal}{\kern0pt}\ a{\isachardoublequoteclose}\isanewline
\ \ \ \ \isacommand{using}\isamarkupfalse%
\ GenExtD\ \isacommand{by}\isamarkupfalse%
\ auto
\end{isabelle}
\textit{Let
  $Q=(\mathcal{P}(\operatorname{dom}(\tau) \times
  \mathbb{P}))^{M}$. This is the set of all names $\vartheta \in
  M^{\mathbb{P}}$} [\dots]%% ---it is pretty laborious to show that things
%% are in $M$; we omit 17 lines of code to that effect,
%% that also apply previously proved lemmas.
\begin{isabelle}
\ \ \isacommand{let}\isamarkupfalse%
\ {\isacharquery}{\kern0pt}Q{\isacharequal}{\kern0pt}{\isachardoublequoteopen}Pow\isactrlbsup M\isactrlesup {\isacharparenleft}{\kern0pt}domain{\isacharparenleft}{\kern0pt}{\isasymtau}{\isacharparenright}{\kern0pt}{\isasymtimes}{\isasymbbbP}{\isacharparenright}{\kern0pt}{\isachardoublequoteclose}
\end{isabelle}
\textit{ Let $\pi=Q \times\{\1\}$ and let
  $b=\pi_{G}=$ $\left\{\vartheta_{G}: \vartheta \in Q\right\}$.}
\begin{isabelle}
\ \ \isacommand{let}\isamarkupfalse%
\ {\isacharquery}{\kern0pt}{\isasympi}{\isacharequal}{\kern0pt}{\isachardoublequoteopen}{\isacharquery}{\kern0pt}Q{\isasymtimes}{\isacharbraceleft}{\kern0pt}{\isasymone}{\isacharbraceright}{\kern0pt}{\isachardoublequoteclose}\isanewline
\ \ \isacommand{let}\isamarkupfalse%
\ {\isacharquery}{\kern0pt}b{\isacharequal}{\kern0pt}{\isachardoublequoteopen}val{\isacharparenleft}{\kern0pt}G{\isacharcomma}{\kern0pt}{\isacharquery}{\kern0pt}{\isasympi}{\isacharparenright}{\kern0pt}{\isachardoublequoteclose}
\end{isabelle}
(Recall: \textit{\dots there is a $b\in M[G]$ such that\dots})
\begin{isabelle}
\ \ \isacommand{from}\isamarkupfalse%
\ {\isacartoucheopen}{\isasymtau}{\isasymin}M{\isacartoucheclose}\isanewline
\ \ \isacommand{have}\isamarkupfalse%
\ {\isachardoublequoteopen}domain{\isacharparenleft}{\kern0pt}{\isasymtau}{\isacharparenright}{\kern0pt}{\isasymtimes}{\isasymbbbP}\ {\isasymin}\ M{\isachardoublequoteclose}\ {\isachardoublequoteopen}domain{\isacharparenleft}{\kern0pt}{\isasymtau}{\isacharparenright}{\kern0pt}\ {\isasymin}\ M{\isachardoublequoteclose}\isanewline
\ \ \ \ \isacommand{by}\isamarkupfalse%
\ simp{\isacharunderscore}{\kern0pt}all\isanewline
\ \ \isacommand{then}\isamarkupfalse%
\isanewline
\ \ \isacommand{have}\isamarkupfalse%
\ {\isachardoublequoteopen}{\isacharquery}{\kern0pt}b\ {\isasymin}\ M{\isacharbrackleft}{\kern0pt}G{\isacharbrackright}{\kern0pt}{\isachardoublequoteclose}\isanewline
\ \ \ \ \isacommand{by}\isamarkupfalse%
\ {\isacharparenleft}{\kern0pt}auto\ intro{\isacharbang}{\kern0pt}{\isacharcolon}{\kern0pt}GenExtI{\isacharparenright}{\kern0pt}
\end{isabelle}
\textit{Now,
  consider any $c \in \mathcal{P}(a) \cap M[G]$; we need to show that
  $c \in b$.}
\begin{isabelle}
  \label{goal-on-b}
\ \ \isacommand{have}\isamarkupfalse%
\ {\isachardoublequoteopen}Pow{\isacharparenleft}{\kern0pt}a{\isacharparenright}{\kern0pt}\ {\isasyminter}\ M{\isacharbrackleft}{\kern0pt}G{\isacharbrackright}{\kern0pt}\ {\isasymsubseteq}\ {\isacharquery}{\kern0pt}b{\isachardoublequoteclose}\hfill
\mbox{\rm\small(**)}\isanewline
\ \ \isacommand{proof}\isamarkupfalse%
\isanewline
\ \ \ \ \isacommand{fix}\isamarkupfalse%
\ c\isanewline
\ \ \ \ \isacommand{assume}\isamarkupfalse%
\ {\isachardoublequoteopen}c\ {\isasymin}\ Pow{\isacharparenleft}{\kern0pt}a{\isacharparenright}{\kern0pt}\ {\isasyminter}\ M{\isacharbrackleft}{\kern0pt}G{\isacharbrackright}{\kern0pt}{\isachardoublequoteclose}
\end{isabelle}
\textit{Fix $\chi \in M^{\mathbb{P}}$ such that
  $\chi_{G}=c$,}
\begin{isabelle}
\ \ \ \ \isacommand{then}\isamarkupfalse%
\isanewline
\ \ \ \ \isacommand{obtain}\isamarkupfalse%
\ {\isasymchi}\ \isakeyword{where}\ {\isachardoublequoteopen}c{\isasymin}M{\isacharbrackleft}{\kern0pt}G{\isacharbrackright}{\kern0pt}{\isachardoublequoteclose}\ {\isachardoublequoteopen}{\isasymchi}\ {\isasymin}\ M{\isachardoublequoteclose}\ {\isachardoublequoteopen}val{\isacharparenleft}{\kern0pt}G{\isacharcomma}{\kern0pt}{\isasymchi}{\isacharparenright}{\kern0pt}\ {\isacharequal}{\kern0pt}\ c{\isachardoublequoteclose}\isanewline
\ \ \ \ \ \ \isacommand{using}\isamarkupfalse%
\ GenExt{\isacharunderscore}{\kern0pt}iff\ \isacommand{by}\isamarkupfalse%
\ auto
\end{isabelle}
\textit{and let $\vartheta=\{\langle\sigma, p\rangle:
  \sigma \in \operatorname{dom}(\tau) \wedge p \Vdash \sigma \in
  \chi\}$;}
\begin{isabelle}
\ \ \ \ \isacommand{let}\isamarkupfalse%
\ {\isacharquery}{\kern0pt}{\isasymtheta}{\isacharequal}{\kern0pt}{\isachardoublequoteopen}{\isacharbraceleft}{\kern0pt}{\isasymlangle}{\isasymsigma}{\isacharcomma}{\kern0pt}p{\isasymrangle}\ {\isasymin}domain{\isacharparenleft}{\kern0pt}{\isasymtau}{\isacharparenright}{\kern0pt}{\isasymtimes}\isasymbbbP\ {\isachardot}{\kern0pt}\ p\ {\isasymtturnstile}\ {\isasymcdot}{\isadigit{0}}\ {\isasymin}\ {\isadigit{1}}{\isasymcdot}\ {\isacharbrackleft}{\kern0pt}{\isasymsigma}{\isacharcomma}{\kern0pt}{\isasymchi}{\isacharbrackright}{\kern0pt}\ {\isacharbraceright}{\kern0pt}{\isachardoublequoteclose}
\end{isabelle}
\textit{$\vartheta \in M$ by the Definability Lemma.}
\begin{isabelle}
\ \ \ \ \isacommand{have}\isamarkupfalse%
\ {\isachardoublequoteopen}arity{\isacharparenleft}{\kern0pt}forces{\isacharparenleft}{\kern0pt}\ {\isasymcdot}{\isadigit{0}}\ {\isasymin}\ {\isadigit{1}}{\isasymcdot}\ {\isacharparenright}{\kern0pt}{\isacharparenright}{\kern0pt}\ {\isacharequal}{\kern0pt}\ {\isadigit{6}}{\isachardoublequoteclose}\isanewline
\ \ \ \ \ \ \isacommand{using}\isamarkupfalse%
\ arity{\isacharunderscore}{\kern0pt}forces{\isacharunderscore}{\kern0pt}at\ \isacommand{by}\isamarkupfalse%
\ auto\isanewline
\ \ \ \ \isacommand{with}\isamarkupfalse%
\ {\isacartoucheopen}domain{\isacharparenleft}{\kern0pt}{\isasymtau}{\isacharparenright}{\kern0pt}\ {\isasymin}\ M{\isacartoucheclose}\ {\isacartoucheopen}{\isasymchi}\ {\isasymin}\ M{\isacartoucheclose}\isanewline
\ \ \ \ \isacommand{have}\isamarkupfalse%
\ {\isachardoublequoteopen}{\isacharquery}{\kern0pt}{\isasymtheta}\ {\isasymin}\ M{\isachardoublequoteclose}\isanewline
\ \ \ \ \ \ \isacommand{using}\isamarkupfalse%
\ sats{\isacharunderscore}{\kern0pt}fst{\isacharunderscore}{\kern0pt}snd{\isacharunderscore}{\kern0pt}in{\isacharunderscore}{\kern0pt}M\isanewline
\ \ \ \ \ \ \isacommand{by}\isamarkupfalse%
\ simp\end{isabelle}
\textit{Since
  $\vartheta \in Q$,}
\begin{isabelle}
\ \ \ \ \isacommand{with}\isamarkupfalse%
\ {\isacartoucheopen}domain{\isacharparenleft}{\kern0pt}{\isasymtau}{\isacharparenright}{\kern0pt}{\isasymtimes}{\isasymbbbP}\ {\isasymin}\ M{\isacartoucheclose}\isanewline
\ \ \ \ \isacommand{have}\isamarkupfalse%
\ {\isachardoublequoteopen}{\isacharquery}{\kern0pt}{\isasymtheta}\ {\isasymin}\ {\isacharquery}{\kern0pt}Q{\isachardoublequoteclose}\isanewline
\ \ \ \ \ \ \isacommand{using}\isamarkupfalse%
\ Pow{\isacharunderscore}{\kern0pt}rel{\isacharunderscore}{\kern0pt}char\ \isacommand{by}\isamarkupfalse%
\ auto
\end{isabelle}
\textit{we are done if we can show that
  $\vartheta_{G}=c$.}
\begin{isabelle}
\ \ \ \ \isacommand{have}\isamarkupfalse%
\ {\isachardoublequoteopen}val{\isacharparenleft}{\kern0pt}G{\isacharcomma}{\kern0pt}{\isacharquery}{\kern0pt}{\isasymtheta}{\isacharparenright}{\kern0pt}\ {\isacharequal}{\kern0pt}\ c{\isachardoublequoteclose}\isanewline
%
\ \ \ \ \isacommand{proof} \ \mbox{$[\dots]$}
\end{isabelle}

This cherry-picked example shows that the formalization can be close
to the mathematical exposition and might be useful to reconstruct the
proof from the book; nonetheless, it also has significantly more
details than the mathematical prose, even with some indications to
direct the automatic tools.

There has been some progress on assistants where one writes statements
and proofs in natural language; recently P.~Koepke and his team
achieved magnificient results by using Isabelle/Naproche
\cite{10.1007/978-3-030-81097-9_2} to formalize proofs of several
results (particularly, the proof of König's Theorem). The input
language of Isabelle/Naproche is a \emph{controlled} natural language
that presents the result being formalized as a deduction in
first-order logic, where every assumption and the ``whole logical
scenario'' are explicitly given. From the input language,
Isabelle/Naproche builds ``proof tasks'' that are handled to automatic
theorem provers. As far as we can tell, Isabelle/Naproche is
promising but still unsuitable for a project of the magnitude of ours.

% Undoubtedly, even for this cherry-picked example, the formalization
% looks a bit codish. It is therefore inevitable to compare this to the
% magnificent results obtained by P.~Koepke and his team by using
% Isabelle/Naproche \cite{10.1007/978-3-030-81097-9_2} (particularly,
% the proof of König's Theorem). The trick there consists in presenting the result
% being formalized as a restricted first-order problem, and then each
% proof step can be handled by an automatic theorem prover.

%%% Local Variables: 
%%% mode: latex
%%% TeX-master: "independence_ch_isabelle"
%%% ispell-local-dictionary: "american"
%%% End: 
