\section{Introduction}\label{sec:introduction}
Set Theory play a double role in Mathematics: It is one of its
possible foundations, and also an active research area.

As it is widely known, Georg Cantor introduced its main concepts and in
particular showed the fundamental result that the real line, $\R$  is not
coordinable with the natural numbers. Soon after this, he posed the
most important question in the field, written as a conjecture:
\begin{quote}
  The \emph{Continuum Hypothesis} ($\CH$). Every uncountable subset of $\R$ is
  coordinable with $\R$.
\end{quote}
\bigskip

\hrule
\bigskip


The most widely accepted foundation of mathematics is through Set
Theory. More precisely, using the first-order theory axiomatized by
Zermelo, Fraenkel, and including the Axiom of Choice ($\AC$) among its
axioms. This theory is known by the $\ZFC$ acronym.
\bigskip

\hrule
\bigskip

The first, both historically and in importance,
    application of forcing was the proof that $\AC$ and $\CH$ were not
    derivable from $\ZFC$; this was done by Cohen in its seminal work \cite{Cohen-CH-PNAS}.

%-%-%-%-%-%-%-%-%-%-%-%-%-%-%-%-%-%-%-%-%-%-%-%-%-%-%-%-%-%-%-%-%-%-%-%-%-%-%-%
\subsection{Models of ZFC}
\begin{enumerate}
\item $\CH$ is (Platonistically) undecidable.
  \begin{enumerate}
  \item By G\"odel's Second Incompleteness Theorems, we can't show
    that a model of $\ZFC$ exists.
  \item G\"odel showed that $\CH$ is relatively consistent with
    $\ZFC$. ``Easy.''
  \item Cohen showed that  $\neg\CH$ is relatively consistent with
    $\ZFC$. Not ``easy.'' Formalization Open.
  \end{enumerate}
\item Using $\ZF$, G\"odel defined the constructible universe,
  $L$. G\"odel showed that $\lb L,\in\rb\models \ZFC+\CH$. Paulson formalized $L$
  for the consistency of $\AC$.
\item The main concept for G\"odel's proofs is
  \emph{relativization}. This is the core of Paulson's development.
  \begin{enumerate}
  \item \dots
  \end{enumerate}
  %
\item 
\item Cohen invented Forcing.
  \begin{enumerate}
  \item Countable transitive models $M$. These are enough for consistency
    proofs by the L\"owenheim-Skolem Theorem, the Reflection
    Principle, and the Mostowksi Collapse.
    %
  \item 
    %
  \item Forcing is a technique to extend countable transitive models of
    $\ZFC$. This process is guaranteed to preserve the $\ZFC$
    axioms but allows to fine-tune what other first order properties the
    extension will have. 
    %
  \item Given a preorder with top $\lb\PP,\leq,\1\rb$ in $M$, define
    \emph{filter} $G$, \emph{dense}. In general, there is no  $G\in M$
    that intersects every dense
    subset of $\PP$ that lies in $M$ (``generic'').
  \item Rasiowa-Sikorski states that there is such $G$ for every
    family of countable dense subsets. Thus, there are generic filters
    for countable transitive models.
  \end{enumerate}

\end{enumerate}

%-%-%-%-%-%-%-%-%-%-%-%-%-%-%-%-%-%-%-%-%-%-%-%-%-%-%-%-%-%-%-%-%-%-%-%-%-%-%-%
\subsection{Isabelle/ZF}


%%% Local Variables:
%%% mode: latex
%%% ispell-local-dictionary: "american"
%%% TeX-master: "first_steps_into_forcing"
%%% End:
