\appendix

\section{Discipline for relativization}

As we said in Sec.~\ref{sec:relat-vers-non-absol}, in order to force
CH and its negation we depended on having relativized versions of
cardinals, Alephs, etc. It was clear for us that our efforts would be
more efficient if we set up a discipline for relativizing sets (terms
of type $\tyi$) and predicates/relations (terms of type $\tyo$).

Paulson only had, for each set, the relational version. It seemed
clearer to us to have a functional version of the relativized concept.
Going back to our example in \ref{sec:tools-relativization}, for the
concept $\isatt{cardinal}::\tyi \fun \tyi$ we want its relative
version
$\isatt{cardinal{\uscore}rel}::(\tyi \fun \tyo) \fun \tyi \fun\tyi$
and the relational version of the latter
$\isatt{is{\uscore}cardinal}::(\tyi \fun \tyo) \fun \tyi \fun \tyi
\fun \tyo$.

Our first attempt of defining a discipline was inspired by
mathematical considerations: if we might prove that
$\isatt{is{\uscore}cardinal}$ is functional and also prove the
existence of a witness $\isatt{c}$ such that $\isatt{M(c)}$ and
$\isatt{is{\uscore}cardinal(M,x,c)}$ then
$\isatt{cardinal{\uscore}rel(M,x)}$ can be obtained by the operator of
definite descriptions.

Soon we realized that resorting to definite descriptions was needed
only for the most primitive concepts. In fact, once we have a
relativized concept, we can use it to define other relativizations.
For instance, $\isatt{cardinal{\uscore}rel}$ depends on having
relative versions of $\isatt{bij}$. Instead of relationalizing
$\isatt{bij}$ to get $\isatt{is{\uscore}bij}$ and then prove
uniqueness and existence of a witness, we define
$\isatt{bij{\uscore}rel}$ using $\isatt{inj{\uscore}rel}$ and
$\isatt{surj{\uscore}rel}$.

%%% Local Variables: 
%%% mode: latex
%%% TeX-master: "independence_ch_isabelle"
%%% ispell-local-dictionary: "american"
%%% End: 
