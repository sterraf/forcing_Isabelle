\documentclass{article}
\usepackage{isabelle,isabellesym}
\renewcommand{\ttdefault}{cmtt}
\usepackage{xcolor}
\usepackage{csquotes}
\usepackage{enumitem}

\newlist{inlinelist}{enumerate*}{1}
\setlist*[inlinelist,1]{%
  label=(\roman*),
}
\usepackage{hyperref}
\usepackage[numbers]{natbib}
\input{header-draft}
\makeatletter
\def\foottext{\gdef\@thefnmark{}\@footnotetext}
\makeatother
\newcommand{\keywords}[1]{\foottext{\emph{Keywords:} #1}}
\newcommand{\ack}[1]{\par\bigskip \noindent \emph{Acknowledgment:} #1}

\hypersetup{
  pdftitle={Separation in Generic Extensions for Isabelle},
  pdfsubject={Computer Science},
  pdfkeywords={Isabelle/ZF, forcing, names, generic extension, constructibility},
  colorlinks,
  linkcolor={blue!40!black},
  citecolor={blue!40!black},
  urlcolor={blue!40!black}
}

\begin{document}
\title{Separation in Generic Extensions for Isabelle}
\author{Emmanuel Gunther
  \and 
  Miguel Pagano
  \and 
  Pedro S\'anchez Terraf}
\maketitle

\begin{abstract}
  We mechanize, in the proof assistant
  Isabelle, a proof of the
  axiom-scheme of Separation in 
  generic extensions of models of set theory  
  by using the fundamental theorems of forcing.
  We also formalize the satisfaction of the axioms of
  Extensionality, Foundation, Union, and Powerset. The axiom of
  Infinity is likewise treated, under additional assumptions on the ground
  model.
  We also  extend Paulson's library on constructibility  with
  renaming of variables for internalized formulas, an improvement on
  definitions by recursion on well-founded  relations and sharpening
  of the hypotheses in his development of relativization and
  absoluteness.
\end{abstract}
\keywords{
Isabelle/ZF, forcing, names, generic extension, constructibility.
}

%%%%%%%%%%%%%%%%%%%%%%%%%%%%%%%%%%%%%%%%%%%%%%%%%%%%%%%%%%%%%%%%%%%%%%%%%%%%%%%%
%%%%%%%%%%%%%%%%%%%%%%%%%%%%%%%%%%%%%%%%%%%%%%%%%%%%%%%%%%%%%%%%%%%%%%          
\section{Introduction}
% no \IEEEPARstart
Zermelo-Fraenkel Set Theory ($\ZF$) has a prominent place among formal
theories; in particular, it provides a foundation for mathematics and
most of the formal toolkit used everyday by the computer scientist has
also Set Theory at its base (cf.~\cite{paulson1995set}). In this time
of mechanization of mathematics~\cite{avigad2018mechanization}, it
seems natural to ask for a mechanization of the most salient results
of Set Theory.

After G\"odel's Incompleteness Theorems, we cannot expect to have a
formal proof of the consistency of Set Theory in $\ZF$. Besides its own
consistency, there are other results which are undecided by $\ZF$: the
undecidability of Continuum Hypothesis leads to the development of
techniques for independence proofs. First G\"odel introduced the
theory of \emph{inner models}, which gives rise to his model $L$ of
the \emph{Axiom of Constructibility} \cite{godel-L} and proved the
relative consistency of the Axiom of Choice and the Generalized
Continuum Hypothesis with $\ZF$. Thirty years later Paul
J. Cohen~\cite{Cohen-CH-PNAS} devised the technique of \emph{forcing},
which is the only known way of \emph{extending} models of $\ZF$; in
particular, it can be used to prove the relative consistency of the
negation of the Continuum Hypothesis. 

In this work we address a substantial part of formalizing the proof
that given a model $M$ of $\ZF$, any \emph{generic extension} $M[G]$
given by forcing also satisfies $\ZF$. As remarked by
\citet[][p.250]{kunen2011set} \enquote{[...] in verifying that $M[G]$
  is a model for set theory, the hardest axiom to verify is
  Comprehension.}  The most important achievement of this paper is the
mechanization in Isabelle a considerable part of this result; en route
to this, we also formalized the satisfaction by $M[G]$ of
Extensionality, Foundation, Union, and Infinity. % We have already
% proved Pairing in the first report of our project
% \cite{2018arXiv180705174G}.

Our development benefited from the remarkable work done by Lawrence
Paulson \cite{paulson_2003} on the formalization of G\"odel's
constructible universe in the proof assistant \emph{Isabelle}. The
ultimate goal of our project is the formalization of forcing to
complete the mechanization of the independence of the Continuum
Hypothesis. We think that this project constitutes an interesting case
which stresses how feasible is to formally implement mathematics that
involve several levels of reasoning.

The \emph{Formal Abstract} project~\cite{hales-fabstracts} proposes
that the formalization of mathematics by writing the statements of
results and the material upon which they are based (definitions,
propositions, lemmas), but ommiting the proofs. We took a less drastic
position: since the proof that those axioms hold in generic extension
is independent of the \emph{proofs} of the ``fundamental'' theorems of
forcing, we assumed these results. Let us remark that the definition
of the forces relation is, by itself, quite demanding; the
formalization of it and of the fundamental theorems of forcing roughly
comprises little less than a half of our full project.

It might be a little surprising the lack of formalizations of forcing
and generic extensions. As far as we know, the development of
\citet{JFR6232} in homotopy type theory for constructing generic
extensions in a sheaf-theoretic setting is the unique mechanization of
forcing. This contrast with the fruitful use of forcing techniques to
extend the Curry-Howard isomorphism to classical axioms
\cite{Miquel:2011:FPT:2058525.2059614,lmcs:1070}. Moreover, the
combination of forcing with intuitionistic type theory
\cite{Coquand:2009:FTT:1807662.1807665,coquand2010note} gives rise
both to positive results (an algorithm to obtain witnesses of the
continuity of definable functionals \cite{coquand2012computational})
and also negative (the independence of Markov's principle
\cite{lmcs:3859}). In the same strand of forcing from the point of
view of proof theory \cite{avigad_2004} are the conservative
extensions of CoC with forcing conditions
\cite{jaber:hal-01319066,Jaber:2012:ETT:2358958.2359524}.

% \fbox{Parece mucho comienzo sólo para introducir a Kunen}
% \fbox{¿lo puedo achurar un poco?}

% In a gross simplification, there are two aspects to a formalization
% project like this one: thematic and programmatic. The first concerns
% the handling of all the theoretical concepts and results in the
% subject, while the second involves the practical issues of the
% implementation and design. In the case of forcing, the main intricacy
% lies in the first aspect. In this sense, following a sensible
% presentation of the material is key.  The authoritative reference 
% on the subject during the last 30 years has been Kunen's classical
% \cite{kunen1980}. In our
% formalizaton we have followed a recent rewrite \cite{kunen2011set}
% of that  textbook, which presents the material in the same sharp 
% style but offering a lot of details. In some sense this project
% wouldn't exist without this book. As alternative, introductory
% resources, the  interested reader can check
% \cite{chow-beginner-forcing}; also, the book \cite{weaver2014forcing}
% contains a thorough treament minimizing the technicalities.

In pursuing the proof of preservation on generic extensions we
extended Paulson's library with:
\begin{inlinelist}
\item renaming of variables, which with little effort can be extended
  to substitutions;
\item an improvement on definitions by recursion on well-founded
  relations; 
\item a better hierarchy of locales; and
\item a new choice principle and a version of Rassiowa-Sikorski which
  ensure the existence of generic filters for countable and transitive
  models of $\ZF$; these points were already communicated in the
  first report \cite{2018arXiv180705174G}.
\end{inlinelist} 

We briefly describe the contents of each
section. Section~\ref{sec:isabelle} contains the bare minimun
requirements to understand the (meta)logics used in Isabelle. Next, an
overview of the model theory of set theory is presented in
Section~\ref{sec:axioms-models-set-theory}. There is an ``internal''
representation of first-order formulas as sets, implemented by
Paulson; Section~\ref{sec:renaming} discusses syntactical
transformations of the former, mainly permutation of variables. 
In Section~\ref{sec:generic-extensions} the generic extensions are
succintly reviewed and how the treatment of well founded recursion in
Isabelle was enhanced. We take care of the ``easy axioms'' in
Section~\ref{sec:easy-axioms}; these are the ones that
do not depend on the forcing theorems. We describe the latter in
Section~\ref{sec:forcing}. We adapted the  work by Paulson to our
needs, and this is described in
Section~\ref{sec:hack-constructible}. We present the proof
of the Separation Axiom Scheme in Section~\ref{sec:proof-separation},
which follows closely its implementation. A plan for future work and
some immediate conclusions are offered in
Section~\ref{sec:conclusions-future-work}.

%%% Local Variables: 
%%% mode: latex
%%% TeX-master: "Separation_In_MG"
%%% ispell-local-dictionary: "american"
%%% End: 


%%%%%%%%%%%%%%%%%%%%%%%%%%%%%%%%%%%%%%%%%%%%%%%%%%%%%%%%%%%%%%%%%%%%%%          
\section{Isabelle}
\label{sec:isabelle}
%-%-%-%-%-%-%-%-%-%-%-%-%-%-%-%-%-%-%-%-%-%-%-%-%-%-%-%-%-%-%-%-%-%-%-
\subsection{Logics}
\label{sec:logics}
Isabelle provides a meta-language called \emph{Pure} that consists of
a fragment of higher order logic, where \isatt{\isasymRightarrow} is
the function-space arrow. The meta-Boolean type is called
\isatt{prop}. Meta-connectives
\isatt{\isasymLongrightarrow} and \isatt{\&\&\&} fulfill the role of
implication and conjunction, and the meta-binder \isatt{\isasymAnd}
corresponds to universal quantification. 

On top of \emph{Pure}, theories/object logics can be defined, with
their own types, connectives and rules. Rules can be written  using
meta-implication: ``$P$, $Q$, and $R$ yield $S$'' can be written
\[
P \ \isatt{\isasymLongrightarrow}\ Q\ \isatt{\isasymLongrightarrow}\ R\ \isatt{\isasymLongrightarrow}\ S
\]
(as usual,  \isatt{\isasymLongrightarrow} associates to the right), and
syntactic sugar is provided to curry the previous rule as follows:
\[
\isasymlbrakk P; Q; R \isasymrbrakk \ \isatt{\isasymLongrightarrow}\ S.
\]
One further example is given by induction on the natural numbers
\isatt{nat},
\[
\isasymlbrakk P(0);\ (\textstyle\isasymAnd
x.\ P(x)\ \isasymLongrightarrow\ P(\isatt{succ}(x))) \isasymrbrakk
\ \isasymLongrightarrow\ P(n), 
\]
where we are omitting the ``typing'' assumtions on $n$ and $x$.

We work in the object theory \emph{Isabelle/ZF}. Two types are defined
in this theory: \tyo, the object-Booleans, and \tyi,
sets. It must be insisted that the types are defined axiomatically, not
recursively. That is, although there are constants and functions that
generate elements of both types, neither of them are 
\emph{initial}, in the sense that they are not the least types
obtained by combining the given constants and operations. This will
have concrete consequences in our strategy to approach the
development. From the beginning, we had to resort to
\emph{internalized} formulas, i.e.\ elements of type $\tyi$ that
encode first-order formulas with a binary relation symbol, and the
satisfaction predicate \isatt{sats\,::\,"i\isasymRightarrow i\isasymRightarrow i\isasymRightarrow o"}  between a set
model with an environment and an internalized formula (where the
relation symbol is interpreted as membership). The set 
\isatt{formula::"\tyi"}
 of internalized
formulas is defined by recursion and hence it is possible to perform
inductive arguments using them. In this sense, the object-logic level
is further divided into \emph{internal} and \emph{external}
sublevels. 

\medskip
\fbox{Portar a \textbf{Isabelle2018} antes de enviar.}

\medskip

The source code is written for the 2018 version of Isabelle (with
minor modifications, it can be run in Isabelle2016-1). Most of it is
presented in the (nowadays standard) declarative flavour called
\emph{Isar} \cite{DBLP:conf/tphol/Wenzel99}, where intermediate
statements in the course 
of a proof are explicitly stated, interspersed with automatic 
tactics handling more trivial steps. The goal is that the resulting
text, a \emph{proof document}, can be understood without the need of
running it.

%-%-%-%-%-%-%-%-%-%-%-%-%-%-%-%-%-%-%-%-%-%-%-%-%-%-%-%-%-%-%-%-%-%-%-
\subsection{Locales}
\label{sec:locales}
Locales \cite{ballarin2010tutorial} provide a neat facility to
encapsulate a context (fixed objects and assumptions on them) that is
to be used in proving several theorems, as in usual mathematical
practice. 

In this paper, locales have a further use. The \emph{Fundamental
  Theorems of Forcing} we use talk about a specific map $\forceisa$
from formulas to formulas. The definition of $\forceisa$ is involved
and we will not dwell on this now; but applications of those theorems
do not require to know how it is defined. Therefore, we black-box it
and pack everything in a locale called \texttt{forcing\_thms} that
assumes that there is such a 
map that satisfies the Fundamental Theorems.

%% \begin{description}
%% \item[\texttt{forcing\_notion}] preorden con top
%% \item[\texttt{countable\_generic}] lo anterior con una familia contable de densos.
%% \item[\texttt{M\_ZF}] axiomas.
%% \item[\texttt{forcing\_data}]: lo anterior contable transitivo y una notion.
%% \item[\texttt{forcing\_thms}]: eso.
%% \item[\texttt{G\_generic}]: lo anterior y G es genérico.
%% \item[\texttt{M\_extra\_assms}]: check in M e instancia de reemplazo para G.
%% \item[\texttt{G\_generic\_extra}]: los dos anteriores (no sé si sigue estando)
%% \end{description}


%%% Local Variables: 
%%% mode: latex
%%% TeX-master: "Separation_In_MG"
%%% ispell-local-dictionary: "american"
%%% End: 


%%%%%%%%%%%%%%%%%%%%%%%%%%%%%%%%%%%%%%%%%%%%%%%%%%%%%%%%%%%%%%%%%%%%%%          
\section{Axioms of set theory}


%%% Local Variables: 
%%% mode: latex
%%% TeX-master: "Separation_In_MG"
%%% ispell-local-dictionary: "american"
%%% End: 


%%%%%%%%%%%%%%%%%%%%%%%%%%%%%%%%%%%%%%%%%%%%%%%%%%%%%%%%%%%%%%%%%%%%%%          
\section{Renaming}
\label{sec:renaming}
\newcommand{\renaming}[2]{(#1)[#2]}
\newcommand{\inFm}[2]{#1 \in #2}
\newcommand{\eqFm}[2]{#1 = #2}
\newcommand{\negFm}[1]{\neg #1}
\newcommand{\andFm}[2]{#1 \wedge #2}
\newcommand{\forallFm}[1]{\forall #1}

\newcommand{\inIFm}[2]{\mathsf{Member}(#1,#2)}
\newcommand{\eqIFm}[2]{\mathsf{Equal}(#1,#2)}
\newcommand{\nandIFm}[2]{\mathsf{Nand}(#1,#2)}
\newcommand{\forallIFm}[1]{\mathsf{Forall(#1)}}


In the course of our work we need to reason about renaming of formulas
and its effect on their satisfiability. Internalized formulas are
implemented using de Bruijn indices for variables and the arity of a
formula $\phi$ gives the least natural number containing all the free
variables in $\phi$. Following Fiore et al. \cite{fiore-abssyn}, one
can understand the arity of a formula as the context of the free
variables; notice that the arity of $\forallFm{\phi}$ is the
predecessor of the arity of $\phi$. In order to understand renamings,
it is helpful to think of $\mathsf{succ}(n)$ as the coproduct
$1+n = \{0\} \cup \{1,\dots,n\}$; given a renaming $f \colon n \to m$,
the unique morphism $\mathsf{id}_1+f \colon 1+n \to 1+m$ is used to
rename free variables in a quantified formula.

\begin{definition}[Renaming]
  Let $\phi$ be a formula of arity $n$ and let $f \colon n \to m$, the
  renaming of $\phi$ by $f$, denoted $\renaming{\phi}{f}$, is defined
  by recursion on $\phi$:
  \begin{gather*}
    \renaming{\inFm{i}{j}}{f} = \inFm{f\,i}{f\,j}\\
    \renaming{\eqFm{i}{j}}{f} = \eqFm{f\,i}{f\,j}\\
    \renaming{\negFm{\phi}}{f} = \negFm{\renaming{\phi}{f}}\\
    \renaming{\andFm{\phi}{\psi}}{f} = \andFm{\renaming{\phi}{f}}{\renaming{\psi}{f}}\\
    \renaming{\forallFm{\phi}}{f} = \forallFm{\renaming{\phi}{\mathsf{id}_1+f}}
  \end{gather*}
\end{definition}

As usual, if $M$ is a set, $a_0,\dots,a_{n-1}$ are elements of $M$, and
$\phi$ is a formula of arity $n$, we write
\[
M,[a_0,\dots,a_{n-1}] \models \phi
\]
to denote that $\phi$ is satisfied by $M$ when $i$ is interpreted
as $a_i$ ($i=0,\dots,n-1$). We call the list $[a_0,\dots,a_{n-1}]$ the
\emph{environment}.

The action of renaming on environments re-indexes the variables. An
easy proof connects satisfaction with renamings.
\begin{lemma}
  \label{lem:renaming}
  Let $\phi$ be a formula of arity $n$, $f \colon n \to m$ be a
  renaming, and let $\rho=[a_1,\ldots,a_n]$ and
  $\rho'=[b_1,\ldots,b_m]$ be environments of length $n$ and $m$,
  respectively. If for all $i \in n$, $a_i = b_{j}$ where $j=f\,i$,
  then $M,\rho\models \phi$ is equivalent to
  $M,\rho' \models \renaming{\phi}{f}$.
\end{lemma}

An important resource in Isabelle/ZF is the facility for defining
inductive sets \cite{paulson2000fixedpoint,paulson1995set} together
with a principle for defining functions by structural recursion.
Internalized formulas are a prime example of this, so we define
a function \isa{ren} that associates to each formula an internalized
function that can be later applied to suitable arguments. Notice that
Paulson used \isa{Nand} because it is more economical.
\begin{isabelle}
\isamarkuptrue%
\isacommand{consts}\isamarkupfalse%
\ ren\ {\isacharcolon}{\isacharcolon}\ {\isachardoublequoteopen}i{\isacharequal}{\isachargreater}i{\isachardoublequoteclose}\isanewline
\isacommand{primrec}\isamarkupfalse%
\isanewline
\ {\isachardoublequoteopen}ren{\isacharparenleft}Member{\isacharparenleft}x{\isacharcomma}y{\isacharparenright}{\isacharparenright}\ {\isacharequal}\isanewline
\ \ {\isacharparenleft}{\isasymlambda}n\ {\isasymin}\ nat\ {\isachardot}\ {\isasymlambda}\ m\ {\isasymin}\ nat{\isachardot}\ {\isasymlambda}f\ {\isasymin}\ n\ {\isasymrightarrow}\ m{\isachardot}\ Member\ {\isacharparenleft}f{\isacharbackquote}x{\isacharcomma}\ f{\isacharbackquote}y{\isacharparenright}{\isacharparenright}{\isachardoublequoteclose}\isanewline
\ \isanewline
\ {\isachardoublequoteopen}ren{\isacharparenleft}Equal{\isacharparenleft}x{\isacharcomma}y{\isacharparenright}{\isacharparenright}\ {\isacharequal}\isanewline
\ \ {\isacharparenleft}{\isasymlambda}n\ {\isasymin}\ nat\ {\isachardot}\ {\isasymlambda}\ m\ {\isasymin}\ nat{\isachardot}\ {\isasymlambda}f\ {\isasymin}\ n\ {\isasymrightarrow}\ m{\isachardot}\ Equal\ {\isacharparenleft}f{\isacharbackquote}x{\isacharcomma}\ f{\isacharbackquote}y{\isacharparenright}{\isacharparenright}{\isachardoublequoteclose}\isanewline
\ \isanewline
\ {\isachardoublequoteopen}ren{\isacharparenleft}Nand{\isacharparenleft}p{\isacharcomma}q{\isacharparenright}{\isacharparenright}\ {\isacharequal}\isanewline
\ \ {\isacharparenleft}{\isasymlambda}n\ {\isasymin}\ nat\ {\isachardot}\ {\isasymlambda}\ m\ {\isasymin}\ nat{\isachardot}\ {\isasymlambda}f\ {\isasymin}\ n\ {\isasymrightarrow}\ m{\isachardot}\ 
Nand\ {\isacharparenleft}ren{\isacharparenleft}p{\isacharparenright}{\isacharbackquote}n{\isacharbackquote}m{\isacharbackquote}f{\isacharcomma}\ ren{\isacharparenleft}q{\isacharparenright}{\isacharbackquote}n{\isacharbackquote}m{\isacharbackquote}f{\isacharparenright}{\isacharparenright}{\isachardoublequoteclose}\isanewline
\ \isanewline
\ {\isachardoublequoteopen}ren{\isacharparenleft}Forall{\isacharparenleft}p{\isacharparenright}{\isacharparenright}\ {\isacharequal}\isanewline
\ \  {\isacharparenleft}{\isasymlambda}n\ {\isasymin}\ nat\ {\isachardot}\ {\isasymlambda}\ m\ {\isasymin}\ nat{\isachardot}\ {\isasymlambda}f\ {\isasymin}\ n\ {\isasymrightarrow}\ m{\isachardot}\ \isanewline
\ \ \ Forall\ {\isacharparenleft}ren{\isacharparenleft}p{\isacharparenright}{\isacharbackquote}succ{\isacharparenleft}n{\isacharparenright}{\isacharbackquote}succ{\isacharparenleft}m{\isacharparenright}{\isacharbackquote}sum{\isacharunderscore}id{\isacharparenleft}n{\isacharcomma}f{\isacharparenright}{\isacharparenright}{\isacharparenright}{\isachardoublequoteclose}
\end{isabelle}

In the last equation, \isa{sum{\isacharunderscore}id} corresponds to
the coproduct morphism $\mathsf{id}_{1}+f \colon 1 + n \to 1 +
n$. Since the schema for recursively defined functions does not allow
parameters, we are forced to return a function of three arguments
(\isa{n,m,f}). This also exposes some inconveniences of working in the
untyped realm of set theory; for example to use \isa{ren} we will need
to prove that the renaming is a function. Besides some auxiliary
results (for example that the application of renaming to suitable
arguments yields a formula), the main result corresponding to
Lemma~\ref{lem:renaming} is:
\begin{isabelle}
\isacommand{lemma}\isamarkupfalse%
\ sats{\isacharunderscore}iff{\isacharunderscore}sats{\isacharunderscore}ren\ {\isacharcolon}\ \isanewline
\ \ \isakeyword{fixes}\ {\isasymphi}\isanewline
\ \ \isakeyword{assumes}\ {\isachardoublequoteopen}{\isasymphi}\ {\isasymin}\ formula{\isachardoublequoteclose}\isanewline
\ \ \isakeyword{shows}\ \ {\isachardoublequoteopen}{\isasymAnd}\ n\ m\ {\isasymrho}\ {\isasymrho}{\isacharprime}\ f\ {\isachardot}\ \isanewline
\ \ {\isasymlbrakk}n{\isasymin}nat\ {\isacharsemicolon}\ m{\isasymin}nat\ {\isacharsemicolon}\ f\ {\isasymin}\ n{\isasymrightarrow}m\ {\isacharsemicolon}\ arity{\isacharparenleft}{\isasymphi}{\isacharparenright}\ {\isasymle}\ n\ {\isacharsemicolon}\isanewline
\ \ \ \ \ {\isasymrho}\ {\isasymin}\ list{\isacharparenleft}M{\isacharparenright}\ {\isacharsemicolon}\ {\isasymrho}{\isacharprime}\ {\isasymin}\ list{\isacharparenleft}M{\isacharparenright}\ {\isacharsemicolon}\ \isanewline
\ \ \ {\isasymAnd}\ i\ {\isachardot}\ i{\isacharless}n\ {\isasymLongrightarrow}\ nth{\isacharparenleft}i{\isacharcomma}{\isasymrho}{\isacharparenright}\ {\isacharequal}\ nth{\isacharparenleft}f{\isacharbackquote}i{\isacharcomma}{\isasymrho}{\isacharprime}{\isacharparenright}\ {\isasymrbrakk}\ {\isasymLongrightarrow}\isanewline
\ \ sats{\isacharparenleft}M{\isacharcomma}{\isasymphi}{\isacharcomma}{\isasymrho}{\isacharparenright}\ {\isasymlongleftrightarrow}\ sats{\isacharparenleft}M{\isacharcomma}ren{\isacharparenleft}{\isasymphi}{\isacharparenright}{\isacharbackquote}n{\isacharbackquote}m{\isacharbackquote}f{\isacharcomma}{\isasymrho}{\isacharprime}{\isacharparenright}{\isachardoublequoteclose}\end{isabelle}

The use of this lemma involves some repetitive tasks (mainly proving
that the renaming is in fact a function). We would like to develop
some \texttt{ML} tools in order to automatize this.
%% We think that it should be possible, and clearer,
%% to express all the current renamings in the theory \isa{Formula} using
%% our approach. 


%%% Local Variables: 
%%% mode: latex
%%% TeX-master: "Separation_In_MG"
%%% ispell-local-dictionary: "american"
%%% End: 


%%%%%%%%%%%%%%%%%%%%%%%%%%%%%%%%%%%%%%%%%%%%%%%%%%%%%%%%%%%%%%%%%%%%%%          
\section{Generic extensions}
\label{sec:generic-extensions}
We will swiftly review some definitions in order to reach the concept
of \emph{generic extension}. As first preliminary definitions, a \emph{forcing
notion} $\lb\PP,\leq,\1\rb$ is simply a preorder with top ($\1$), and a \emph{filter}
$G\sbq\PP$ is an increasing subset which is downwards
compatible. Given a ctm $M$ of $\ZF$, a forcing
notion in $M$, and a filter $G$, a new set $M[G]$ is defined. Each
element $a\in M[G]$ is 
determined by its \emph{name} $\dot a$ in $M$. Actually, the structure of
each $\dot a$ is used to construct $a$. They are related by a
map $\val$ that takes $G$ as a parameter:
\[
\val(G,\dot a) = a.
\] 
Then the extension is defined by the image of the map $\val(G,\cdot)$:
\[
M[G] \defi \{\val(G,\tau): \tau\in M\}.
\]
Metatheoretically, it is straightforward to see that $M[G]$ is a
transitive set that satisfies some axioms of $\ZF$ (see
Section~\ref{sec:easy-axioms}) and includes $M\cup\{G\}$. Nevertheless
there is no a priori reason for $M[G]$ to satisfy either Separation, Powerset
or Replacement. The original insight by Cohen was to define the notion
of \emph{genericity} for a filter $G\sbq\PP$ and to prove that
whenever $G$ is generic, $M[G]$ will satisfy $\ZF$. Remember that a
filter is generic if it intersects all the dense sets in $M$; in
\cite{2018arXiv180705174G} we formalized the Rasiowa-Sikorski lemma which
proves the existence of generic filters for ctms.

The Separation Axiom  is the first that requires the notion of
genericity and the use of the forcing machinery, which we review in
the Section~\ref{sec:forcing}.

%%% Local Variables: 
%%% mode: latex
%%% TeX-master: "Separation_In_MG"
%%% ispell-local-dictionary: "american"
%%% End: 


%%%%%%%%%%%%%%%%%%%%%%%%%%%%%%%%%%%%%%%%%%%%%%%%%%%%%%%%%%%%%%%%%%%%%%          
\subsection*{Recursion and Values of Names}

The map $\val$ used in the definition of the generic extension is
characterized by the recursive equation
\begin{equation}
  \label{eq:val}
  \val(G,\tau) = \{val(G,\sigma) :\exists p \in\PP .%
  \lb\sigma,p\rb \in \tau \wedge p \in G \}
\end{equation}

As is well-known, the principle of transfinite recursion on
well-founded relations~\cite[p. 48]{kunen2011set}) allows us to define
a recursive function $F \colon A\to A$ by choosing a well-founded
relation $R \subseteq A\times A$ and a functional
$H\colon A\times (A \to A) \to A$ satisfying
$F(a)=H(a,F\!\upharpoonright\!(R^{-1}(a)))$. \citet{paulson1995set}
made this principle available in Isabelle/ZF via the the operator
\isa{wfrec}. The formalization of the functional $\mathit{Hv}$ is
straightforward:
%
\begin{isabelle}
\isacommand{definition}\isamarkupfalse%
\isanewline
\ \ Hv\ {\isacharcolon}{\isacharcolon}\ {\isachardoublequoteopen}i{\isasymRightarrow}i{\isasymRightarrow}i{\isasymRightarrow}i{\isachardoublequoteclose}\ \isakeyword{where}\isanewline
\ \ {\isachardoublequoteopen}Hv{\isacharparenleft}G{\isacharcomma}y{\isacharcomma}f{\isacharparenright}\ {\isacharequal}{\isacharequal}\ {\isacharbraceleft}\ f{\isacharbackquote}x\ {\isachardot}{\isachardot}\ x{\isasymin}\ domain{\isacharparenleft}y{\isacharparenright}{\isacharcomma}\ {\isasymexists}p{\isasymin}P{\isachardot}\ {\isacharless}x{\isacharcomma}p{\isachargreater}\ {\isasymin}\ y\ {\isasymand}\ p\ {\isasymin}\ G\ {\isacharbraceright}{\isachardoublequoteclose}
\end{isabelle}
In the references \cite{kunen2011set,weaver2014forcing} $\val$ is
applied only to \emph{names}, that are certain elements of $M$
characterized by a recursively defined predicate. The well-founded
relation used to justify Equation~\eqref{eq:val} is
\[ x \mathrel{\mathit{ed}} y \iff \exists p . \lb x,p\rb\in y. \] In
order to use \isa{wfrec} the relation should be expressed as a set, so
we took the restriction of $\mathit{ed}$ to the whole universe
$M$; i.e. $\mathit{ed}\cap M\times M$.  It was inconvenient to work
with that definition because it imposed an obligation showing that
some set is in $M$. The remedy was to restrict $\mathit{ed}$ to the
transitive closure of the actual parameter:
\begin{isabelle}
\isacommand{definition}\isamarkupfalse%
\isanewline
\ val\ {\isacharcolon}{\isacharcolon}\ {\isachardoublequoteopen}i{\isasymRightarrow}i{\isasymRightarrow}i{\isachardoublequoteclose}\ \isakeyword{where}\isanewline
\ {\isachardoublequoteopen}val{\isacharparenleft}G{\isacharcomma}{\isasymtau}{\isacharparenright}{\isacharequal}{\isacharequal}\ wfrec{\isacharparenleft}edrel{\isacharparenleft}eclose{\isacharparenleft}{\isacharbraceleft}{\isasymtau}{\isacharbraceright}{\isacharparenright}{\isacharparenright}{\isacharcomma}{\isasymtau}{\isacharcomma}Hv{\isacharparenleft}G{\isacharparenright}{\isacharparenright}{\isachardoublequoteclose}
\end{isabelle}
A key result, albeit intuitive and rather easy, to make this
definition work is that when computing the value of a recursive
function on some argument $a$, one can restrict the relation to some
ambient set if it includes $a$ and all of its predecessesors.
\begin{isabelle}
\isacommand{lemma}\isamarkupfalse%
\ wfrec{\isacharunderscore}restr\ {\isacharcolon}\isanewline
\ \ \isakeyword{assumes}\ {\isachardoublequoteopen}relation{\isacharparenleft}r{\isacharparenright}{\isachardoublequoteclose}\ {\isachardoublequoteopen}wf{\isacharparenleft}r{\isacharparenright}{\isachardoublequoteclose}\ \isanewline
\ \ \isakeyword{shows}\ \ {\isachardoublequoteopen}a{\isasymin}A\ {\isasymLongrightarrow}\ {\isacharparenleft}r{\isacharcircum}{\isacharplus}{\isacharparenright}{\isacharminus}{\isacharbackquote}{\isacharbackquote}{\isacharbraceleft}a{\isacharbraceright}\ {\isasymsubseteq}\ A\ {\isasymLongrightarrow}\ \isanewline
\ \ \ \ \ \ \ \ \ \ wfrec{\isacharparenleft}r{\isacharcomma}a{\isacharcomma}H{\isacharparenright}\ {\isacharequal}\ wfrec{\isacharparenleft}r{\isasyminter}A{\isasymtimes}A{\isacharcomma}a{\isacharcomma}H{\isacharparenright}{\isachardoublequoteclose}
\end{isabelle}

This lemma gives rise to some useful results about values of
names. First, we characterize $\val$ as in Equation~(\ref{eq:val}):
\begin{isabelle}
  \isacommand{lemma}\isamarkupfalse%
  \ def{\isacharunderscore}val{\isacharcolon}\isanewline
  \ {\isachardoublequoteopen}val{\isacharparenleft}G{\isacharcomma}x{\isacharparenright}\ {\isacharequal}\ {\isacharbraceleft}val{\isacharparenleft}G{\isacharcomma}t{\isacharparenright}\ {\isachardot}{\isachardot}\ t{\isasymin}domain{\isacharparenleft}x{\isacharparenright}\ {\isacharcomma}\isanewline
\ \ \ \ \ \ \ \ \ \ \ \ \ \ \ \ \ \ \ {\isasymexists}p{\isasymin}P{\isachardot}\ {\isacharless}t{\isacharcomma}p{\isachargreater}{\isasymin}x\ {\isasymand}\ p{\isasymin}G\ {\isacharbraceright}{\isachardoublequoteclose}
\end{isabelle}
and the monotonicity of $\val$ follows automatically after a
substitution.
\begin{isabelle}
\isacommand{lemma}\isamarkupfalse%
\ val{\isacharunderscore}mono{\isacharcolon}\ {\isachardoublequoteopen}x{\isasymsubseteq}y\ {\isasymLongrightarrow}\ val{\isacharparenleft}G{\isacharcomma}x{\isacharparenright}\ {\isasymsubseteq}\ val{\isacharparenleft}G{\isacharcomma}y{\isacharparenright}{\isachardoublequoteclose}\isanewline
%
\ \ \isacommand{by}\isamarkupfalse%
\ {\isacharparenleft}subst\ {\isacharparenleft}{\isadigit{1}}\ {\isadigit{2}}{\isacharparenright}\ def{\isacharunderscore}val{\isacharcomma}\ force{\isacharparenright}%
\end{isabelle}
More interestingly we can give a neat equation for values of
names defined by Separation, say $B = \{x\in A\times \PP.\ Q(x)\}$,
then
\begin{equation}
\val(G,B) = \{\val(G,t) : t\in A , \exists p\in \PP \cap G.\ Q(\lb t,p\rb) \} \label{eq:val-name-sep}
\end{equation}


\fbox{Creo que deberíamos poner acá \isa{check} y \isa{check\_in\_M} }

%%% Local Variables: 
%%% mode: latex
%%% TeX-master: "Separation_In_MG"
%%% ispell-local-dictionary: "american"
%%% End: 


% %%%%%%%%%%%%%%%%%%%%%%%%%%%%%%%%%%%%%%%%%%%%%%%%%%%%%%%%%%%%%%%%%%%%%%          
\section{Prior results}

Paulson's work provides an appropriate starting point for our
developments.

There are other formalizations of set theory (Isabelle/HOL, Mizar, Automath).

To the best of our knowledge, only preliminary approaches to the
formalization of forcing exist; v.g. \cite{Quirin}, which has already been
discussed in our previous work \cite{2018arXiv180705174G}. 

%%% Local Variables: 
%%% mode: latex
%%% TeX-master: "Separation_In_MG"
%%% ispell-local-dictionary: "american"
%%% End: 


%%%%%%%%%%%%%%%%%%%%%%%%%%%%%%%%%%%%%%%%%%%%%%%%%%%%%%%%%%%%%%%%%%%%%%          
\section{Hacking of \isatt{ZF-Constructible}}

In \cite{paulson_2003}, Paulson presented his formalization of the
relative consistency of the Axiom of Choice. This development is
included inside the Isabelle distribution with session name
\isatt{ZF-Constructible}. The main technical devices, invented by
G\"odel for this purpose, are \emph{relativization} and
\emph{absoluteness}. In a nutshell, to relativize a formula $\phi$ to
a class $C$, it is enough to restrict its quantifiers to $C$. The
example of \isatt{upair\_ax} in
Section~\ref{sec:axioms-models-set-theory}, the relativized version of
the Pairing Axiom, is extracted from \texttt{Relative}, one of the
core theories of \isatt{ZF-Constructible}. On the other hand, $\phi$
is \emph{absolute} for $C$ if it is equivalent to its relativization,
meaning that the statement made $\phi$ coincides with what $C$
``believes'' $\phi$ is saying. Paulson shows that under certain
hypothesis  on a class $M$ (condensed in the locale \isatt{M\_trivial}), a plethora of
absoluteness and closure results can be proved about $M$.

The development of forcing, and the study of ctms in general, takes
absoluteness as a starting point. We were not able to work with
\isatt{ZF-Constructible} right out-of-the-box. The main reason is that
we can't expect state the ``class version'' of Replacement for a
\emph{set} $M$ by
using first-order formulas, since predicates \isatt{P::"i=>o"} can't
be proved to be only the definable ones. Therefore, we had to make
some modifications to the various sets of hypothesis in several
locales to make the results available as tools for the present and
future developments.

%% There are several lemmas that were declared, in later developments, as
%% introduction/simplification rules (notably, the rule
%% \isatt{equalityI}). They raised a warning, and we have 
%% eliminated some of them, but in some cases we had to keep them because
%% the proof works because it is insisted that the rule is \emph{safe}.

The most notable changes, located in the theory \texttt{Relative}, are
the following:
\begin{enumerate}
\item\label{item:1} We eliminated the requirement that the relative Axiom of Replacement
  is satisfied by a class $M$ to be in the locale \isatt{M\_trivial}. 
\item\label{item:2} We moved the requirement of the Powerset Axiom to \isatt{M\_basic}. 
\item\label{item:3} We replaced the need that the set of natural numbers is in $M$ by the
  milder hypothesis that $M(0)$. Actually, most results should follow
  by only assuming that $M$ is nonempty.
\end{enumerate}

As a consequence of Item~\ref{item:1}, the lemma
\isatt{strong\_replacementI} is no longer valid and was commented
out.

We moved the requirement $M(\mathtt{nat})$ to the locale
\isatt{M\_trancl} (inside the theory \isatt{WF\_absolute}), where it is needed for the first time. Some results,
for instance \isatt{rtran\_closure\_mem\_iff} and 
\isatt{iterates\_imp\_wfrec\_replacement} had to be moved inside that
locale.

The proof, for instance, that the constructible universe $L$ satisfies
the modified locale \isatt{M\_trivial} holds with minor
modifications. Nevertheless, in order to have a neater presentation,
we have stripped off several sections concerning $L$ from the theories
\isatt{L\_axioms} and \isatt{Internalize}, and we merged them to form
the new file  \isatt{Internalizations}. 

\medskip
\fbox{>>>Poner lo que sigue en las conclusiones???}
\medskip

We believe that the
\isatt{ZF-Constructible} session  would benefit from some changes in
the way some results are organized accross theory files; for
intance, by cataloging all or most of the internalized formulas in one
file. Another, most basic example would be to start out with an even
more locale that only assumes $M$ to be a nonempty transitive class,
as many absoluteness results follow from these hypothesis.  
It is our desire to advocate a future work to a thorough
revision of the development of constructibility to maximize modularity.

%%% Local Variables: 
%%% mode: latex
%%% TeX-master: "Separation_In_MG"
%%% ispell-local-dictionary: "american"
%%% End: 


%%%%%%%%%%%%%%%%%%%%%%%%%%%%%%%%%%%%%%%%%%%%%%%%%%%%%%%%%%%%%%%%%%%%%%          
\section{Foundation, Union, Infinity}
\label{sec:easy-axioms}

It is straightforward to show that the generic extension $M[G]$
satisfies foundation, union, and infinity. The easiest one is
foundation which does not depend on the satisfaction by $M$.


\fbox{Hablar de \texttt{M\_trivial} y del uso de interfaz para instanciar}

\fbox{Falta \texttt{check\_in\_M} para completar Infinity}

%%% Local Variables: 
%%% mode: latex
%%% TeX-master: "Separation_In_MG"
%%% ispell-local-dictionary: "american"
%%% End: 


%%%%%%%%%%%%%%%%%%%%%%%%%%%%%%%%%%%%%%%%%%%%%%%%%%%%%%%%%%%%%%%%%%%%%%          
\section{Forcing}
\label{sec:forcing}
Let $\lb \PP, {\preceq} ,\1\rb \in M$ be a forcing notion. Given $G\sbq \PP$, we have
$M[G]\defi \{ \val(\PP,G,\punto{a}) : \punto{a}\in M \}$.

The following form of the Forcing Theorems  is the one
that we formalized.
\begin{theorem}
  There exists a function  $\forceisa:: \tyi \fun  \tyi$
  such that for every
  $\phi\in\formula$ and $\punto{a}_0,\dots,\punto{a}_n\in M$,
  \begin{enumerate}
  \item (Definability) $\forceisa(\phi)\in\formula$;
  \item (Truth Lemma) for every $M$-generic $G$,
    \[
      M[G], [\val(\PP,G,\punto{a}_0),\dots,\val(\PP,G,\punto{a}_n)]
      \models \phi
    \]
    is equivalent to 
    \[
      \exists p\in G.\ \; M, [p,\PP,\preceq,\1, \punto{a}_0,\dots,\punto{a}_n]  \models
      \forceisa(\phi).\]

    We use the notation $p \forces
    \phi\ [\punto{a}_0,\dots,\punto{a}_n]$ for this last assertion.
  \item (Density Lemma) $p \forces \phi\ [\punto{a}_0,\dots,\punto{a}_n]$
    if and only if 
    $\{q\in \PP :  q \forces \phi\ [\punto{a}_0,\dots,\punto{a}_n]\}$
    is dense below $p$.
  \end{enumerate}
\end{theorem}

We followed the new Kunen's book to define
$\forceisa$.  Forcing for atomic formulas is described as a mutual
recursion
%% \begin{multline*}
%%   \forceseq (p,t_1,t_2) \defi 
%%   \forall s\in\dom(t_1)\cup\dom(t_2).\ \forall q\pleq p .\\
%%   \forcesmem(q,s,t_1)\lsii \forcesmem(q,s,t_2)
%% \end{multline*}
%% \begin{multline*}
%%   \forcesmem(p,t_1,t_2) \defi  \forall v\pleq p. \ \exists q\pleq v.\\  
%%   \exists s.\ \exists r\in \PP .\ \lb s,r\rb \in  t_2 \land q
%%   \pleq r \land \forceseq(q,t_1,s)
%% \end{multline*}
but then \cite[p.~257]{kunen2011set} it is cast as a single
recursively defined function $\frcat$ over the wellfounded relation
$\isatt{frecR}$ on tuples $\lb \mathit{ft},t_1,t_2,p\rb$ (where
$\mathit{ft}\in\{0,1\}$ indicates the type of the atomic formula being
forced). Forcing for general formulas is defined by recursion on the
datatype $\formula$. Details on the implementation and proofs of the
Forcing Theorems have been spelled out in our
\cite{2020arXiv200109715G}.


It is to be noted that application of the Forcing theorems do not
require any extra Replacement instances on $M$.

%%% Local Variables: 
%%% mode: latex
%%% TeX-master: "independence_ch_isabelle"
%%% ispell-local-dictionary: "american"
%%% End: 


%%%%%%%%%%%%%%%%%%%%%%%%%%%%%%%%%%%%%%%%%%%%%%%%%%%%%%%%%%%%%%%%%%%%%%          
\section{Proof of Separation}

This proof can be found in the file \verb|Separation_Axiom.thy| of the
development, which we proceed to discuss.

The key technical result is the following:
\begin{isabelle}
  \isacommand{lemma}\isamarkupfalse%
  \ Collect{\isacharunderscore}sats{\isacharunderscore}in{\isacharunderscore}MG\ {\isacharcolon}\isanewline
  \ \ \isakeyword{assumes}\isanewline
  \ \ \ \ {\isachardoublequoteopen}{\isasympi}\ {\isasymin}\ M{\isachardoublequoteclose}\ {\isachardoublequoteopen}{\isasymsigma}\ {\isasymin}\ M{\isachardoublequoteclose}\ {\isachardoublequoteopen}val{\isacharparenleft}G{\isacharcomma}\ {\isasympi}{\isacharparenright}\ {\isacharequal}\ c{\isachardoublequoteclose}\ {\isachardoublequoteopen}val{\isacharparenleft}G{\isacharcomma}\ {\isasymsigma}{\isacharparenright}\ {\isacharequal}\ w{\isachardoublequoteclose}\isanewline
  \ \ \ \ {\isachardoublequoteopen}{\isasymphi}\ {\isasymin}\ formula{\isachardoublequoteclose}\ {\isachardoublequoteopen}arity{\isacharparenleft}{\isasymphi}{\isacharparenright}\ {\isasymle}\ {\isadigit{2}}{\isachardoublequoteclose}\isanewline
  \ \ \isakeyword{shows}\ \ \ \ \isanewline
  \ \ \ \ {\isachardoublequoteopen}{\isacharbraceleft}x{\isasymin}c{\isachardot}\ sats{\isacharparenleft}M{\isacharbrackleft}G{\isacharbrackright}{\isacharcomma}\ {\isasymphi}{\isacharcomma}\ {\isacharbrackleft}x{\isacharcomma}\ w{\isacharbrackright}{\isacharparenright}{\isacharbraceright}{\isasymin}\ M{\isacharbrackleft}G{\isacharbrackright}{\isachardoublequoteclose}
\end{isabelle}
%
From this, using absoluteness, we will be able to derive the
$\phi$-instance of Separation. 

To show that   
\[
S\defi\{x\in c : M[G]\models \phi(x,w)\} \in M[G],
\]
it is enough to provide a name $n\in M$ for this set.
 
The candidate name is
\[
n \defi \{u \in\dom(\pi)\times\PP :M,[u,\PP,\leq,\1,\sig,\pi]\models \psi\}
\]
where
\[
\psi \defi \exists \th\, p.\ x_0=\lb\th,p\rb \y 
   \forceisa(\th\in x_5\y\phi(\th,x_4)).
\]
The fact that $n\in M$ follows by an application of a six-variable
instance of Separation in $M$ (lemma \isatt{six{\isacharunderscore}sep{\isacharunderscore}aux}).

Almost a third part of the proof involves the syntactic handling of
internalized formulas and permutation of variables. The more
substantive portion concerns proving that actually $\val(G,n)=S$.

Let's first focus into the predicate 
\[
M,[u,\PP,\leq,\1,\sig,\pi]\models \psi
\]
defining $n$ by separation. By definition of the satisfaction
relation and permuting variables, we have it is equivalent to the fact
that there exist $\th,p\in M$ with   $u=\lb\th,p\rb$  and 
\[
M,[\PP,\leq,\1,p,\th,\sig,\pi]\models \forceisa(x_4\in
x_6\y\phi(x_4,x_5)). 
\]
% (Note that the variable $x_7$ is not used.)
This, in turn is equivalent by the Definition of Forcing to: For all $M$-generic
filters $F$ such that $p\in F$, 
\[
M,[\val(G,\th),\val(G,\sig),\val(G,\pi)]\models x_4\in
x_6\y\phi(x_4,x_5). 
\] 


%% The proof of
%% \isatt{Collect{\isacharunderscore}sats{\isacharunderscore}in{\isacharunderscore}MG}
%% has three parts:
%% \begin{enumerate}
%% \item Definition of the name $n$;
%% \item Proving that 
%% \item 
%% \end{enumerate}

%%% Local Variables: 
%%% mode: latex
%%% TeX-master: "Separation_In_MG"
%%% ispell-local-dictionary: "american"
%%% End: 


\section{Conclusion}

TODO:
\begin{enumerate}
\item Implement \emph{Basic Set Theory (BST)} by Kunen in
  Constructible: the use of alternatively Replacement or Powerset to
  prove basic absoluteness and closure resuls.
\item Enhance the automatization of formulas
\item Develop the forcing notions to obtain the independence of $\CH$,
  along with the prerrequisite combinatorial results (v.g.\ the
  $\Delta$-system lemma).
\end{enumerate}

%%% Local Variables: 
%%% mode: latex
%%% TeX-master: "forcing_in_isabelle_zf"
%%% ispell-local-dictionary: "american"
%%% End: 


%%%%%%%%%%%%%%%%%%%%%%%%%%%%%%%%%%%%%%%%%%%%%%%%%%%%%%%%%%%%%%%%%%%%%%%%%%%%%%%%

\bibliographystyle{mi-estilo-else}
\bibliography{../LSFA/citados}
%\documentclass{article}
\usepackage{isabelle,isabellesym}
\renewcommand{\ttdefault}{cmtt}
\usepackage{xcolor}
\usepackage{csquotes}
\usepackage{enumitem}

\newlist{inlinelist}{enumerate*}{1}
\setlist*[inlinelist,1]{%
  label=(\roman*),
}
\usepackage{hyperref}
\usepackage[numbers]{natbib}
\input{header-draft}
\makeatletter
\def\foottext{\gdef\@thefnmark{}\@footnotetext}
\makeatother
\newcommand{\keywords}[1]{\foottext{\emph{Keywords:} #1}}
\newcommand{\ack}[1]{\par\bigskip \noindent \emph{Acknowledgment:} #1}

\hypersetup{
  pdftitle={Separation in Generic Extensions for Isabelle},
  pdfsubject={Computer Science},
  pdfkeywords={Isabelle/ZF, forcing, names, generic extension, constructibility},
  colorlinks,
  linkcolor={blue!40!black},
  citecolor={blue!40!black},
  urlcolor={blue!40!black}
}

\begin{document}
\title{Separation in Generic Extensions for Isabelle}
\author{Emmanuel Gunther
  \and 
  Miguel Pagano
  \and 
  Pedro S\'anchez Terraf}
\maketitle

\begin{abstract}
  We mechanize, in the proof assistant
  Isabelle, a proof of the
  axiom-scheme of Separation in 
  generic extensions of models of set theory  
  by using the fundamental theorems of forcing.
  We also formalize the satisfaction of the axioms of
  Extensionality, Foundation, Union, and Powerset. The axiom of
  Infinity is likewise treated, under additional assumptions on the ground
  model.
  We also  extend Paulson's library on constructibility  with
  renaming of variables for internalized formulas, an improvement on
  definitions by recursion on well-founded  relations and sharpening
  of the hypotheses in his development of relativization and
  absoluteness.
\end{abstract}
\keywords{
Isabelle/ZF, forcing, names, generic extension, constructibility.
}

%%%%%%%%%%%%%%%%%%%%%%%%%%%%%%%%%%%%%%%%%%%%%%%%%%%%%%%%%%%%%%%%%%%%%%%%%%%%%%%%
%%%%%%%%%%%%%%%%%%%%%%%%%%%%%%%%%%%%%%%%%%%%%%%%%%%%%%%%%%%%%%%%%%%%%%          
\section{Introduction}
% no \IEEEPARstart
Zermelo-Fraenkel Set Theory ($\ZF$) has a prominent place among formal
theories; in particular, it provides a foundation for mathematics and
most of the formal toolkit used everyday by the computer scientist has
also Set Theory at its base (cf.~\cite{paulson1995set}). In this time
of mechanization of mathematics~\cite{avigad2018mechanization}, it
seems natural to ask for a mechanization of the most salient results
of Set Theory.

After G\"odel's Incompleteness Theorems, we cannot expect to have a
formal proof of the consistency of Set Theory in $\ZF$. Besides its own
consistency, there are other results which are undecided by $\ZF$: the
undecidability of Continuum Hypothesis leads to the development of
techniques for independence proofs. First G\"odel introduced the
theory of \emph{inner models}, which gives rise to his model $L$ of
the \emph{Axiom of Constructibility} \cite{godel-L} and proved the
relative consistency of the Axiom of Choice and the Generalized
Continuum Hypothesis with $\ZF$. Thirty years later Paul
J. Cohen~\cite{Cohen-CH-PNAS} devised the technique of \emph{forcing},
which is the only known way of \emph{extending} models of $\ZF$; in
particular, it can be used to prove the relative consistency of the
negation of the Continuum Hypothesis. 

In this work we address a substantial part of formalizing the proof
that given a model $M$ of $\ZF$, any \emph{generic extension} $M[G]$
given by forcing also satisfies $\ZF$. As remarked by
\citet[][p.250]{kunen2011set} \enquote{[...] in verifying that $M[G]$
  is a model for set theory, the hardest axiom to verify is
  Comprehension.}  The most important achievement of this paper is the
mechanization in Isabelle a considerable part of this result; en route
to this, we also formalized the satisfaction by $M[G]$ of
Extensionality, Foundation, Union, and Infinity. % We have already
% proved Pairing in the first report of our project
% \cite{2018arXiv180705174G}.

Our development benefited from the remarkable work done by Lawrence
Paulson \cite{paulson_2003} on the formalization of G\"odel's
constructible universe in the proof assistant \emph{Isabelle}. The
ultimate goal of our project is the formalization of forcing to
complete the mechanization of the independence of the Continuum
Hypothesis. We think that this project constitutes an interesting case
which stresses how feasible is to formally implement mathematics that
involve several levels of reasoning.

The \emph{Formal Abstract} project~\cite{hales-fabstracts} proposes
that the formalization of mathematics by writing the statements of
results and the material upon which they are based (definitions,
propositions, lemmas), but ommiting the proofs. We took a less drastic
position: since the proof that those axioms hold in generic extension
is independent of the \emph{proofs} of the ``fundamental'' theorems of
forcing, we assumed these results. Let us remark that the definition
of the forces relation is, by itself, quite demanding; the
formalization of it and of the fundamental theorems of forcing roughly
comprises little less than a half of our full project.

It might be a little surprising the lack of formalizations of forcing
and generic extensions. As far as we know, the development of
\citet{JFR6232} in homotopy type theory for constructing generic
extensions in a sheaf-theoretic setting is the unique mechanization of
forcing. This contrast with the fruitful use of forcing techniques to
extend the Curry-Howard isomorphism to classical axioms
\cite{Miquel:2011:FPT:2058525.2059614,lmcs:1070}. Moreover, the
combination of forcing with intuitionistic type theory
\cite{Coquand:2009:FTT:1807662.1807665,coquand2010note} gives rise
both to positive results (an algorithm to obtain witnesses of the
continuity of definable functionals \cite{coquand2012computational})
and also negative (the independence of Markov's principle
\cite{lmcs:3859}). In the same strand of forcing from the point of
view of proof theory \cite{avigad_2004} are the conservative
extensions of CoC with forcing conditions
\cite{jaber:hal-01319066,Jaber:2012:ETT:2358958.2359524}.

% \fbox{Parece mucho comienzo sólo para introducir a Kunen}
% \fbox{¿lo puedo achurar un poco?}

% In a gross simplification, there are two aspects to a formalization
% project like this one: thematic and programmatic. The first concerns
% the handling of all the theoretical concepts and results in the
% subject, while the second involves the practical issues of the
% implementation and design. In the case of forcing, the main intricacy
% lies in the first aspect. In this sense, following a sensible
% presentation of the material is key.  The authoritative reference 
% on the subject during the last 30 years has been Kunen's classical
% \cite{kunen1980}. In our
% formalizaton we have followed a recent rewrite \cite{kunen2011set}
% of that  textbook, which presents the material in the same sharp 
% style but offering a lot of details. In some sense this project
% wouldn't exist without this book. As alternative, introductory
% resources, the  interested reader can check
% \cite{chow-beginner-forcing}; also, the book \cite{weaver2014forcing}
% contains a thorough treament minimizing the technicalities.

In pursuing the proof of preservation on generic extensions we
extended Paulson's library with:
\begin{inlinelist}
\item renaming of variables, which with little effort can be extended
  to substitutions;
\item an improvement on definitions by recursion on well-founded
  relations; 
\item a better hierarchy of locales; and
\item a new choice principle and a version of Rassiowa-Sikorski which
  ensure the existence of generic filters for countable and transitive
  models of $\ZF$; these points were already communicated in the
  first report \cite{2018arXiv180705174G}.
\end{inlinelist} 

We briefly describe the contents of each
section. Section~\ref{sec:isabelle} contains the bare minimun
requirements to understand the (meta)logics used in Isabelle. Next, an
overview of the model theory of set theory is presented in
Section~\ref{sec:axioms-models-set-theory}. There is an ``internal''
representation of first-order formulas as sets, implemented by
Paulson; Section~\ref{sec:renaming} discusses syntactical
transformations of the former, mainly permutation of variables. 
In Section~\ref{sec:generic-extensions} the generic extensions are
succintly reviewed and how the treatment of well founded recursion in
Isabelle was enhanced. We take care of the ``easy axioms'' in
Section~\ref{sec:easy-axioms}; these are the ones that
do not depend on the forcing theorems. We describe the latter in
Section~\ref{sec:forcing}. We adapted the  work by Paulson to our
needs, and this is described in
Section~\ref{sec:hack-constructible}. We present the proof
of the Separation Axiom Scheme in Section~\ref{sec:proof-separation},
which follows closely its implementation. A plan for future work and
some immediate conclusions are offered in
Section~\ref{sec:conclusions-future-work}.

%%% Local Variables: 
%%% mode: latex
%%% TeX-master: "Separation_In_MG"
%%% ispell-local-dictionary: "american"
%%% End: 


%%%%%%%%%%%%%%%%%%%%%%%%%%%%%%%%%%%%%%%%%%%%%%%%%%%%%%%%%%%%%%%%%%%%%%          
\section{Isabelle}
\label{sec:isabelle}
%-%-%-%-%-%-%-%-%-%-%-%-%-%-%-%-%-%-%-%-%-%-%-%-%-%-%-%-%-%-%-%-%-%-%-
\subsection{Logics}
\label{sec:logics}
Isabelle provides a meta-language called \emph{Pure} that consists of
a fragment of higher order logic, where \isatt{\isasymRightarrow} is
the function-space arrow. The meta-Boolean type is called
\isatt{prop}. Meta-connectives
\isatt{\isasymLongrightarrow} and \isatt{\&\&\&} fulfill the role of
implication and conjunction, and the meta-binder \isatt{\isasymAnd}
corresponds to universal quantification. 

On top of \emph{Pure}, theories/object logics can be defined, with
their own types, connectives and rules. Rules can be written  using
meta-implication: ``$P$, $Q$, and $R$ yield $S$'' can be written
\[
P \ \isatt{\isasymLongrightarrow}\ Q\ \isatt{\isasymLongrightarrow}\ R\ \isatt{\isasymLongrightarrow}\ S
\]
(as usual,  \isatt{\isasymLongrightarrow} associates to the right), and
syntactic sugar is provided to curry the previous rule as follows:
\[
\isasymlbrakk P; Q; R \isasymrbrakk \ \isatt{\isasymLongrightarrow}\ S.
\]
One further example is given by induction on the natural numbers
\isatt{nat},
\[
\isasymlbrakk P(0);\ (\textstyle\isasymAnd
x.\ P(x)\ \isasymLongrightarrow\ P(\isatt{succ}(x))) \isasymrbrakk
\ \isasymLongrightarrow\ P(n), 
\]
where we are omitting the ``typing'' assumtions on $n$ and $x$.

We work in the object theory \emph{Isabelle/ZF}. Two types are defined
in this theory: \tyo, the object-Booleans, and \tyi,
sets. It must be insisted that the types are defined axiomatically, not
recursively. That is, although there are constants and functions that
generate elements of both types, neither of them are 
\emph{initial}, in the sense that they are not the least types
obtained by combining the given constants and operations. This will
have concrete consequences in our strategy to approach the
development. From the beginning, we had to resort to
\emph{internalized} formulas, i.e.\ elements of type $\tyi$ that
encode first-order formulas with a binary relation symbol, and the
satisfaction predicate \isatt{sats\,::\,"i\isasymRightarrow i\isasymRightarrow i\isasymRightarrow o"}  between a set
model with an environment and an internalized formula (where the
relation symbol is interpreted as membership). The set 
\isatt{formula::"\tyi"}
 of internalized
formulas is defined by recursion and hence it is possible to perform
inductive arguments using them. In this sense, the object-logic level
is further divided into \emph{internal} and \emph{external}
sublevels. 

\medskip
\fbox{Portar a \textbf{Isabelle2018} antes de enviar.}

\medskip

The source code is written for the 2018 version of Isabelle (with
minor modifications, it can be run in Isabelle2016-1). Most of it is
presented in the (nowadays standard) declarative flavour called
\emph{Isar} \cite{DBLP:conf/tphol/Wenzel99}, where intermediate
statements in the course 
of a proof are explicitly stated, interspersed with automatic 
tactics handling more trivial steps. The goal is that the resulting
text, a \emph{proof document}, can be understood without the need of
running it.

%-%-%-%-%-%-%-%-%-%-%-%-%-%-%-%-%-%-%-%-%-%-%-%-%-%-%-%-%-%-%-%-%-%-%-
\subsection{Locales}
\label{sec:locales}
Locales \cite{ballarin2010tutorial} provide a neat facility to
encapsulate a context (fixed objects and assumptions on them) that is
to be used in proving several theorems, as in usual mathematical
practice. 

In this paper, locales have a further use. The \emph{Fundamental
  Theorems of Forcing} we use talk about a specific map $\forceisa$
from formulas to formulas. The definition of $\forceisa$ is involved
and we will not dwell on this now; but applications of those theorems
do not require to know how it is defined. Therefore, we black-box it
and pack everything in a locale called \texttt{forcing\_thms} that
assumes that there is such a 
map that satisfies the Fundamental Theorems.

%% \begin{description}
%% \item[\texttt{forcing\_notion}] preorden con top
%% \item[\texttt{countable\_generic}] lo anterior con una familia contable de densos.
%% \item[\texttt{M\_ZF}] axiomas.
%% \item[\texttt{forcing\_data}]: lo anterior contable transitivo y una notion.
%% \item[\texttt{forcing\_thms}]: eso.
%% \item[\texttt{G\_generic}]: lo anterior y G es genérico.
%% \item[\texttt{M\_extra\_assms}]: check in M e instancia de reemplazo para G.
%% \item[\texttt{G\_generic\_extra}]: los dos anteriores (no sé si sigue estando)
%% \end{description}


%%% Local Variables: 
%%% mode: latex
%%% TeX-master: "Separation_In_MG"
%%% ispell-local-dictionary: "american"
%%% End: 


%%%%%%%%%%%%%%%%%%%%%%%%%%%%%%%%%%%%%%%%%%%%%%%%%%%%%%%%%%%%%%%%%%%%%%          
\section{Axioms of set theory}


%%% Local Variables: 
%%% mode: latex
%%% TeX-master: "Separation_In_MG"
%%% ispell-local-dictionary: "american"
%%% End: 


%%%%%%%%%%%%%%%%%%%%%%%%%%%%%%%%%%%%%%%%%%%%%%%%%%%%%%%%%%%%%%%%%%%%%%          
\section{Renaming}
\label{sec:renaming}
\newcommand{\renaming}[2]{(#1)[#2]}
\newcommand{\inFm}[2]{#1 \in #2}
\newcommand{\eqFm}[2]{#1 = #2}
\newcommand{\negFm}[1]{\neg #1}
\newcommand{\andFm}[2]{#1 \wedge #2}
\newcommand{\forallFm}[1]{\forall #1}

\newcommand{\inIFm}[2]{\mathsf{Member}(#1,#2)}
\newcommand{\eqIFm}[2]{\mathsf{Equal}(#1,#2)}
\newcommand{\nandIFm}[2]{\mathsf{Nand}(#1,#2)}
\newcommand{\forallIFm}[1]{\mathsf{Forall(#1)}}


In the course of our work we need to reason about renaming of formulas
and its effect on their satisfiability. Internalized formulas are
implemented using de Bruijn indices for variables and the arity of a
formula $\phi$ gives the least natural number containing all the free
variables in $\phi$. Following Fiore et al. \cite{fiore-abssyn}, one
can understand the arity of a formula as the context of the free
variables; notice that the arity of $\forallFm{\phi}$ is the
predecessor of the arity of $\phi$. In order to understand renamings,
it is helpful to think of $\mathsf{succ}(n)$ as the coproduct
$1+n = \{0\} \cup \{1,\dots,n\}$; given a renaming $f \colon n \to m$,
the unique morphism $\mathsf{id}_1+f \colon 1+n \to 1+m$ is used to
rename free variables in a quantified formula.

\begin{definition}[Renaming]
  Let $\phi$ be a formula of arity $n$ and let $f \colon n \to m$, the
  renaming of $\phi$ by $f$, denoted $\renaming{\phi}{f}$, is defined
  by recursion on $\phi$:
  \begin{gather*}
    \renaming{\inFm{i}{j}}{f} = \inFm{f\,i}{f\,j}\\
    \renaming{\eqFm{i}{j}}{f} = \eqFm{f\,i}{f\,j}\\
    \renaming{\negFm{\phi}}{f} = \negFm{\renaming{\phi}{f}}\\
    \renaming{\andFm{\phi}{\psi}}{f} = \andFm{\renaming{\phi}{f}}{\renaming{\psi}{f}}\\
    \renaming{\forallFm{\phi}}{f} = \forallFm{\renaming{\phi}{\mathsf{id}_1+f}}
  \end{gather*}
\end{definition}

As usual, if $M$ is a set, $a_0,\dots,a_{n-1}$ are elements of $M$, and
$\phi$ is a formula of arity $n$, we write
\[
M,[a_0,\dots,a_{n-1}] \models \phi
\]
to denote that $\phi$ is satisfied by $M$ when $i$ is interpreted
as $a_i$ ($i=0,\dots,n-1$). We call the list $[a_0,\dots,a_{n-1}]$ the
\emph{environment}.

The action of renaming on environments re-indexes the variables. An
easy proof connects satisfaction with renamings.
\begin{lemma}
  \label{lem:renaming}
  Let $\phi$ be a formula of arity $n$, $f \colon n \to m$ be a
  renaming, and let $\rho=[a_1,\ldots,a_n]$ and
  $\rho'=[b_1,\ldots,b_m]$ be environments of length $n$ and $m$,
  respectively. If for all $i \in n$, $a_i = b_{j}$ where $j=f\,i$,
  then $M,\rho\models \phi$ is equivalent to
  $M,\rho' \models \renaming{\phi}{f}$.
\end{lemma}

An important resource in Isabelle/ZF is the facility for defining
inductive sets \cite{paulson2000fixedpoint,paulson1995set} together
with a principle for defining functions by structural recursion.
Internalized formulas are a prime example of this, so we define
a function \isa{ren} that associates to each formula an internalized
function that can be later applied to suitable arguments. Notice that
Paulson used \isa{Nand} because it is more economical.
\begin{isabelle}
\isamarkuptrue%
\isacommand{consts}\isamarkupfalse%
\ ren\ {\isacharcolon}{\isacharcolon}\ {\isachardoublequoteopen}i{\isacharequal}{\isachargreater}i{\isachardoublequoteclose}\isanewline
\isacommand{primrec}\isamarkupfalse%
\isanewline
\ {\isachardoublequoteopen}ren{\isacharparenleft}Member{\isacharparenleft}x{\isacharcomma}y{\isacharparenright}{\isacharparenright}\ {\isacharequal}\isanewline
\ \ {\isacharparenleft}{\isasymlambda}n\ {\isasymin}\ nat\ {\isachardot}\ {\isasymlambda}\ m\ {\isasymin}\ nat{\isachardot}\ {\isasymlambda}f\ {\isasymin}\ n\ {\isasymrightarrow}\ m{\isachardot}\ Member\ {\isacharparenleft}f{\isacharbackquote}x{\isacharcomma}\ f{\isacharbackquote}y{\isacharparenright}{\isacharparenright}{\isachardoublequoteclose}\isanewline
\ \isanewline
\ {\isachardoublequoteopen}ren{\isacharparenleft}Equal{\isacharparenleft}x{\isacharcomma}y{\isacharparenright}{\isacharparenright}\ {\isacharequal}\isanewline
\ \ {\isacharparenleft}{\isasymlambda}n\ {\isasymin}\ nat\ {\isachardot}\ {\isasymlambda}\ m\ {\isasymin}\ nat{\isachardot}\ {\isasymlambda}f\ {\isasymin}\ n\ {\isasymrightarrow}\ m{\isachardot}\ Equal\ {\isacharparenleft}f{\isacharbackquote}x{\isacharcomma}\ f{\isacharbackquote}y{\isacharparenright}{\isacharparenright}{\isachardoublequoteclose}\isanewline
\ \isanewline
\ {\isachardoublequoteopen}ren{\isacharparenleft}Nand{\isacharparenleft}p{\isacharcomma}q{\isacharparenright}{\isacharparenright}\ {\isacharequal}\isanewline
\ \ {\isacharparenleft}{\isasymlambda}n\ {\isasymin}\ nat\ {\isachardot}\ {\isasymlambda}\ m\ {\isasymin}\ nat{\isachardot}\ {\isasymlambda}f\ {\isasymin}\ n\ {\isasymrightarrow}\ m{\isachardot}\ 
Nand\ {\isacharparenleft}ren{\isacharparenleft}p{\isacharparenright}{\isacharbackquote}n{\isacharbackquote}m{\isacharbackquote}f{\isacharcomma}\ ren{\isacharparenleft}q{\isacharparenright}{\isacharbackquote}n{\isacharbackquote}m{\isacharbackquote}f{\isacharparenright}{\isacharparenright}{\isachardoublequoteclose}\isanewline
\ \isanewline
\ {\isachardoublequoteopen}ren{\isacharparenleft}Forall{\isacharparenleft}p{\isacharparenright}{\isacharparenright}\ {\isacharequal}\isanewline
\ \  {\isacharparenleft}{\isasymlambda}n\ {\isasymin}\ nat\ {\isachardot}\ {\isasymlambda}\ m\ {\isasymin}\ nat{\isachardot}\ {\isasymlambda}f\ {\isasymin}\ n\ {\isasymrightarrow}\ m{\isachardot}\ \isanewline
\ \ \ Forall\ {\isacharparenleft}ren{\isacharparenleft}p{\isacharparenright}{\isacharbackquote}succ{\isacharparenleft}n{\isacharparenright}{\isacharbackquote}succ{\isacharparenleft}m{\isacharparenright}{\isacharbackquote}sum{\isacharunderscore}id{\isacharparenleft}n{\isacharcomma}f{\isacharparenright}{\isacharparenright}{\isacharparenright}{\isachardoublequoteclose}
\end{isabelle}

In the last equation, \isa{sum{\isacharunderscore}id} corresponds to
the coproduct morphism $\mathsf{id}_{1}+f \colon 1 + n \to 1 +
n$. Since the schema for recursively defined functions does not allow
parameters, we are forced to return a function of three arguments
(\isa{n,m,f}). This also exposes some inconveniences of working in the
untyped realm of set theory; for example to use \isa{ren} we will need
to prove that the renaming is a function. Besides some auxiliary
results (for example that the application of renaming to suitable
arguments yields a formula), the main result corresponding to
Lemma~\ref{lem:renaming} is:
\begin{isabelle}
\isacommand{lemma}\isamarkupfalse%
\ sats{\isacharunderscore}iff{\isacharunderscore}sats{\isacharunderscore}ren\ {\isacharcolon}\ \isanewline
\ \ \isakeyword{fixes}\ {\isasymphi}\isanewline
\ \ \isakeyword{assumes}\ {\isachardoublequoteopen}{\isasymphi}\ {\isasymin}\ formula{\isachardoublequoteclose}\isanewline
\ \ \isakeyword{shows}\ \ {\isachardoublequoteopen}{\isasymAnd}\ n\ m\ {\isasymrho}\ {\isasymrho}{\isacharprime}\ f\ {\isachardot}\ \isanewline
\ \ {\isasymlbrakk}n{\isasymin}nat\ {\isacharsemicolon}\ m{\isasymin}nat\ {\isacharsemicolon}\ f\ {\isasymin}\ n{\isasymrightarrow}m\ {\isacharsemicolon}\ arity{\isacharparenleft}{\isasymphi}{\isacharparenright}\ {\isasymle}\ n\ {\isacharsemicolon}\isanewline
\ \ \ \ \ {\isasymrho}\ {\isasymin}\ list{\isacharparenleft}M{\isacharparenright}\ {\isacharsemicolon}\ {\isasymrho}{\isacharprime}\ {\isasymin}\ list{\isacharparenleft}M{\isacharparenright}\ {\isacharsemicolon}\ \isanewline
\ \ \ {\isasymAnd}\ i\ {\isachardot}\ i{\isacharless}n\ {\isasymLongrightarrow}\ nth{\isacharparenleft}i{\isacharcomma}{\isasymrho}{\isacharparenright}\ {\isacharequal}\ nth{\isacharparenleft}f{\isacharbackquote}i{\isacharcomma}{\isasymrho}{\isacharprime}{\isacharparenright}\ {\isasymrbrakk}\ {\isasymLongrightarrow}\isanewline
\ \ sats{\isacharparenleft}M{\isacharcomma}{\isasymphi}{\isacharcomma}{\isasymrho}{\isacharparenright}\ {\isasymlongleftrightarrow}\ sats{\isacharparenleft}M{\isacharcomma}ren{\isacharparenleft}{\isasymphi}{\isacharparenright}{\isacharbackquote}n{\isacharbackquote}m{\isacharbackquote}f{\isacharcomma}{\isasymrho}{\isacharprime}{\isacharparenright}{\isachardoublequoteclose}\end{isabelle}

The use of this lemma involves some repetitive tasks (mainly proving
that the renaming is in fact a function). We would like to develop
some \texttt{ML} tools in order to automatize this.
%% We think that it should be possible, and clearer,
%% to express all the current renamings in the theory \isa{Formula} using
%% our approach. 


%%% Local Variables: 
%%% mode: latex
%%% TeX-master: "Separation_In_MG"
%%% ispell-local-dictionary: "american"
%%% End: 


%%%%%%%%%%%%%%%%%%%%%%%%%%%%%%%%%%%%%%%%%%%%%%%%%%%%%%%%%%%%%%%%%%%%%%          
\section{Generic extensions}
\label{sec:generic-extensions}
We will swiftly review some definitions in order to reach the concept
of \emph{generic extension}. As first preliminary definitions, a \emph{forcing
notion} $\lb\PP,\leq,\1\rb$ is simply a preorder with top ($\1$), and a \emph{filter}
$G\sbq\PP$ is an increasing subset which is downwards
compatible. Given a ctm $M$ of $\ZF$, a forcing
notion in $M$, and a filter $G$, a new set $M[G]$ is defined. Each
element $a\in M[G]$ is 
determined by its \emph{name} $\dot a$ in $M$. Actually, the structure of
each $\dot a$ is used to construct $a$. They are related by a
map $\val$ that takes $G$ as a parameter:
\[
\val(G,\dot a) = a.
\] 
Then the extension is defined by the image of the map $\val(G,\cdot)$:
\[
M[G] \defi \{\val(G,\tau): \tau\in M\}.
\]
Metatheoretically, it is straightforward to see that $M[G]$ is a
transitive set that satisfies some axioms of $\ZF$ (see
Section~\ref{sec:easy-axioms}) and includes $M\cup\{G\}$. Nevertheless
there is no a priori reason for $M[G]$ to satisfy either Separation, Powerset
or Replacement. The original insight by Cohen was to define the notion
of \emph{genericity} for a filter $G\sbq\PP$ and to prove that
whenever $G$ is generic, $M[G]$ will satisfy $\ZF$. Remember that a
filter is generic if it intersects all the dense sets in $M$; in
\cite{2018arXiv180705174G} we formalized the Rasiowa-Sikorski lemma which
proves the existence of generic filters for ctms.

The Separation Axiom  is the first that requires the notion of
genericity and the use of the forcing machinery, which we review in
the Section~\ref{sec:forcing}.

%%% Local Variables: 
%%% mode: latex
%%% TeX-master: "Separation_In_MG"
%%% ispell-local-dictionary: "american"
%%% End: 


%%%%%%%%%%%%%%%%%%%%%%%%%%%%%%%%%%%%%%%%%%%%%%%%%%%%%%%%%%%%%%%%%%%%%%          
\subsection*{Recursion and Values of Names}

The map $\val$ used in the definition of the generic extension is
characterized by the recursive equation
\begin{equation}
  \label{eq:val}
  \val(G,\tau) = \{val(G,\sigma) :\exists p \in\PP .%
  \lb\sigma,p\rb \in \tau \wedge p \in G \}
\end{equation}

As is well-known, the principle of transfinite recursion on
well-founded relations~\cite[p. 48]{kunen2011set}) allows us to define
a recursive function $F \colon A\to A$ by choosing a well-founded
relation $R \subseteq A\times A$ and a functional
$H\colon A\times (A \to A) \to A$ satisfying
$F(a)=H(a,F\!\upharpoonright\!(R^{-1}(a)))$. \citet{paulson1995set}
made this principle available in Isabelle/ZF via the the operator
\isa{wfrec}. The formalization of the functional $\mathit{Hv}$ is
straightforward:
%
\begin{isabelle}
\isacommand{definition}\isamarkupfalse%
\isanewline
\ \ Hv\ {\isacharcolon}{\isacharcolon}\ {\isachardoublequoteopen}i{\isasymRightarrow}i{\isasymRightarrow}i{\isasymRightarrow}i{\isachardoublequoteclose}\ \isakeyword{where}\isanewline
\ \ {\isachardoublequoteopen}Hv{\isacharparenleft}G{\isacharcomma}y{\isacharcomma}f{\isacharparenright}\ {\isacharequal}{\isacharequal}\ {\isacharbraceleft}\ f{\isacharbackquote}x\ {\isachardot}{\isachardot}\ x{\isasymin}\ domain{\isacharparenleft}y{\isacharparenright}{\isacharcomma}\ {\isasymexists}p{\isasymin}P{\isachardot}\ {\isacharless}x{\isacharcomma}p{\isachargreater}\ {\isasymin}\ y\ {\isasymand}\ p\ {\isasymin}\ G\ {\isacharbraceright}{\isachardoublequoteclose}
\end{isabelle}
In the references \cite{kunen2011set,weaver2014forcing} $\val$ is
applied only to \emph{names}, that are certain elements of $M$
characterized by a recursively defined predicate. The well-founded
relation used to justify Equation~\eqref{eq:val} is
\[ x \mathrel{\mathit{ed}} y \iff \exists p . \lb x,p\rb\in y. \] In
order to use \isa{wfrec} the relation should be expressed as a set, so
we took the restriction of $\mathit{ed}$ to the whole universe
$M$; i.e. $\mathit{ed}\cap M\times M$.  It was inconvenient to work
with that definition because it imposed an obligation showing that
some set is in $M$. The remedy was to restrict $\mathit{ed}$ to the
transitive closure of the actual parameter:
\begin{isabelle}
\isacommand{definition}\isamarkupfalse%
\isanewline
\ val\ {\isacharcolon}{\isacharcolon}\ {\isachardoublequoteopen}i{\isasymRightarrow}i{\isasymRightarrow}i{\isachardoublequoteclose}\ \isakeyword{where}\isanewline
\ {\isachardoublequoteopen}val{\isacharparenleft}G{\isacharcomma}{\isasymtau}{\isacharparenright}{\isacharequal}{\isacharequal}\ wfrec{\isacharparenleft}edrel{\isacharparenleft}eclose{\isacharparenleft}{\isacharbraceleft}{\isasymtau}{\isacharbraceright}{\isacharparenright}{\isacharparenright}{\isacharcomma}{\isasymtau}{\isacharcomma}Hv{\isacharparenleft}G{\isacharparenright}{\isacharparenright}{\isachardoublequoteclose}
\end{isabelle}
A key result, albeit intuitive and rather easy, to make this
definition work is that when computing the value of a recursive
function on some argument $a$, one can restrict the relation to some
ambient set if it includes $a$ and all of its predecessesors.
\begin{isabelle}
\isacommand{lemma}\isamarkupfalse%
\ wfrec{\isacharunderscore}restr\ {\isacharcolon}\isanewline
\ \ \isakeyword{assumes}\ {\isachardoublequoteopen}relation{\isacharparenleft}r{\isacharparenright}{\isachardoublequoteclose}\ {\isachardoublequoteopen}wf{\isacharparenleft}r{\isacharparenright}{\isachardoublequoteclose}\ \isanewline
\ \ \isakeyword{shows}\ \ {\isachardoublequoteopen}a{\isasymin}A\ {\isasymLongrightarrow}\ {\isacharparenleft}r{\isacharcircum}{\isacharplus}{\isacharparenright}{\isacharminus}{\isacharbackquote}{\isacharbackquote}{\isacharbraceleft}a{\isacharbraceright}\ {\isasymsubseteq}\ A\ {\isasymLongrightarrow}\ \isanewline
\ \ \ \ \ \ \ \ \ \ wfrec{\isacharparenleft}r{\isacharcomma}a{\isacharcomma}H{\isacharparenright}\ {\isacharequal}\ wfrec{\isacharparenleft}r{\isasyminter}A{\isasymtimes}A{\isacharcomma}a{\isacharcomma}H{\isacharparenright}{\isachardoublequoteclose}
\end{isabelle}

This lemma gives rise to some useful results about values of
names. First, we characterize $\val$ as in Equation~(\ref{eq:val}):
\begin{isabelle}
  \isacommand{lemma}\isamarkupfalse%
  \ def{\isacharunderscore}val{\isacharcolon}\isanewline
  \ {\isachardoublequoteopen}val{\isacharparenleft}G{\isacharcomma}x{\isacharparenright}\ {\isacharequal}\ {\isacharbraceleft}val{\isacharparenleft}G{\isacharcomma}t{\isacharparenright}\ {\isachardot}{\isachardot}\ t{\isasymin}domain{\isacharparenleft}x{\isacharparenright}\ {\isacharcomma}\isanewline
\ \ \ \ \ \ \ \ \ \ \ \ \ \ \ \ \ \ \ {\isasymexists}p{\isasymin}P{\isachardot}\ {\isacharless}t{\isacharcomma}p{\isachargreater}{\isasymin}x\ {\isasymand}\ p{\isasymin}G\ {\isacharbraceright}{\isachardoublequoteclose}
\end{isabelle}
and the monotonicity of $\val$ follows automatically after a
substitution.
\begin{isabelle}
\isacommand{lemma}\isamarkupfalse%
\ val{\isacharunderscore}mono{\isacharcolon}\ {\isachardoublequoteopen}x{\isasymsubseteq}y\ {\isasymLongrightarrow}\ val{\isacharparenleft}G{\isacharcomma}x{\isacharparenright}\ {\isasymsubseteq}\ val{\isacharparenleft}G{\isacharcomma}y{\isacharparenright}{\isachardoublequoteclose}\isanewline
%
\ \ \isacommand{by}\isamarkupfalse%
\ {\isacharparenleft}subst\ {\isacharparenleft}{\isadigit{1}}\ {\isadigit{2}}{\isacharparenright}\ def{\isacharunderscore}val{\isacharcomma}\ force{\isacharparenright}%
\end{isabelle}
More interestingly we can give a neat equation for values of
names defined by Separation, say $B = \{x\in A\times \PP.\ Q(x)\}$,
then
\begin{equation}
\val(G,B) = \{\val(G,t) : t\in A , \exists p\in \PP \cap G.\ Q(\lb t,p\rb) \} \label{eq:val-name-sep}
\end{equation}


\fbox{Creo que deberíamos poner acá \isa{check} y \isa{check\_in\_M} }

%%% Local Variables: 
%%% mode: latex
%%% TeX-master: "Separation_In_MG"
%%% ispell-local-dictionary: "american"
%%% End: 


% %%%%%%%%%%%%%%%%%%%%%%%%%%%%%%%%%%%%%%%%%%%%%%%%%%%%%%%%%%%%%%%%%%%%%%          
\section{Prior results}

Paulson's work provides an appropriate starting point for our
developments.

There are other formalizations of set theory (Isabelle/HOL, Mizar, Automath).

To the best of our knowledge, only preliminary approaches to the
formalization of forcing exist; v.g. \cite{Quirin}, which has already been
discussed in our previous work \cite{2018arXiv180705174G}. 

%%% Local Variables: 
%%% mode: latex
%%% TeX-master: "Separation_In_MG"
%%% ispell-local-dictionary: "american"
%%% End: 


%%%%%%%%%%%%%%%%%%%%%%%%%%%%%%%%%%%%%%%%%%%%%%%%%%%%%%%%%%%%%%%%%%%%%%          
\section{Hacking of \isatt{ZF-Constructible}}

In \cite{paulson_2003}, Paulson presented his formalization of the
relative consistency of the Axiom of Choice. This development is
included inside the Isabelle distribution with session name
\isatt{ZF-Constructible}. The main technical devices, invented by
G\"odel for this purpose, are \emph{relativization} and
\emph{absoluteness}. In a nutshell, to relativize a formula $\phi$ to
a class $C$, it is enough to restrict its quantifiers to $C$. The
example of \isatt{upair\_ax} in
Section~\ref{sec:axioms-models-set-theory}, the relativized version of
the Pairing Axiom, is extracted from \texttt{Relative}, one of the
core theories of \isatt{ZF-Constructible}. On the other hand, $\phi$
is \emph{absolute} for $C$ if it is equivalent to its relativization,
meaning that the statement made $\phi$ coincides with what $C$
``believes'' $\phi$ is saying. Paulson shows that under certain
hypothesis  on a class $M$ (condensed in the locale \isatt{M\_trivial}), a plethora of
absoluteness and closure results can be proved about $M$.

The development of forcing, and the study of ctms in general, takes
absoluteness as a starting point. We were not able to work with
\isatt{ZF-Constructible} right out-of-the-box. The main reason is that
we can't expect state the ``class version'' of Replacement for a
\emph{set} $M$ by
using first-order formulas, since predicates \isatt{P::"i=>o"} can't
be proved to be only the definable ones. Therefore, we had to make
some modifications to the various sets of hypothesis in several
locales to make the results available as tools for the present and
future developments.

%% There are several lemmas that were declared, in later developments, as
%% introduction/simplification rules (notably, the rule
%% \isatt{equalityI}). They raised a warning, and we have 
%% eliminated some of them, but in some cases we had to keep them because
%% the proof works because it is insisted that the rule is \emph{safe}.

The most notable changes, located in the theory \texttt{Relative}, are
the following:
\begin{enumerate}
\item\label{item:1} We eliminated the requirement that the relative Axiom of Replacement
  is satisfied by a class $M$ to be in the locale \isatt{M\_trivial}. 
\item\label{item:2} We moved the requirement of the Powerset Axiom to \isatt{M\_basic}. 
\item\label{item:3} We replaced the need that the set of natural numbers is in $M$ by the
  milder hypothesis that $M(0)$. Actually, most results should follow
  by only assuming that $M$ is nonempty.
\end{enumerate}

As a consequence of Item~\ref{item:1}, the lemma
\isatt{strong\_replacementI} is no longer valid and was commented
out.

We moved the requirement $M(\mathtt{nat})$ to the locale
\isatt{M\_trancl} (inside the theory \isatt{WF\_absolute}), where it is needed for the first time. Some results,
for instance \isatt{rtran\_closure\_mem\_iff} and 
\isatt{iterates\_imp\_wfrec\_replacement} had to be moved inside that
locale.

The proof, for instance, that the constructible universe $L$ satisfies
the modified locale \isatt{M\_trivial} holds with minor
modifications. Nevertheless, in order to have a neater presentation,
we have stripped off several sections concerning $L$ from the theories
\isatt{L\_axioms} and \isatt{Internalize}, and we merged them to form
the new file  \isatt{Internalizations}. 

\medskip
\fbox{>>>Poner lo que sigue en las conclusiones???}
\medskip

We believe that the
\isatt{ZF-Constructible} session  would benefit from some changes in
the way some results are organized accross theory files; for
intance, by cataloging all or most of the internalized formulas in one
file. Another, most basic example would be to start out with an even
more locale that only assumes $M$ to be a nonempty transitive class,
as many absoluteness results follow from these hypothesis.  
It is our desire to advocate a future work to a thorough
revision of the development of constructibility to maximize modularity.

%%% Local Variables: 
%%% mode: latex
%%% TeX-master: "Separation_In_MG"
%%% ispell-local-dictionary: "american"
%%% End: 


%%%%%%%%%%%%%%%%%%%%%%%%%%%%%%%%%%%%%%%%%%%%%%%%%%%%%%%%%%%%%%%%%%%%%%          
\section{Foundation, Union, Infinity}
\label{sec:easy-axioms}

It is straightforward to show that the generic extension $M[G]$
satisfies foundation, union, and infinity. The easiest one is
foundation which does not depend on the satisfaction by $M$.


\fbox{Hablar de \texttt{M\_trivial} y del uso de interfaz para instanciar}

\fbox{Falta \texttt{check\_in\_M} para completar Infinity}

%%% Local Variables: 
%%% mode: latex
%%% TeX-master: "Separation_In_MG"
%%% ispell-local-dictionary: "american"
%%% End: 


%%%%%%%%%%%%%%%%%%%%%%%%%%%%%%%%%%%%%%%%%%%%%%%%%%%%%%%%%%%%%%%%%%%%%%          
\section{Forcing}
\label{sec:forcing}
Let $\lb \PP, {\preceq} ,\1\rb \in M$ be a forcing notion. Given $G\sbq \PP$, we have
$M[G]\defi \{ \val(\PP,G,\punto{a}) : \punto{a}\in M \}$.

The following form of the Forcing Theorems  is the one
that we formalized.
\begin{theorem}
  There exists a function  $\forceisa:: \tyi \fun  \tyi$
  such that for every
  $\phi\in\formula$ and $\punto{a}_0,\dots,\punto{a}_n\in M$,
  \begin{enumerate}
  \item (Definability) $\forceisa(\phi)\in\formula$;
  \item (Truth Lemma) for every $M$-generic $G$,
    \[
      M[G], [\val(\PP,G,\punto{a}_0),\dots,\val(\PP,G,\punto{a}_n)]
      \models \phi
    \]
    is equivalent to 
    \[
      \exists p\in G.\ \; M, [p,\PP,\preceq,\1, \punto{a}_0,\dots,\punto{a}_n]  \models
      \forceisa(\phi).\]

    We use the notation $p \forces
    \phi\ [\punto{a}_0,\dots,\punto{a}_n]$ for this last assertion.
  \item (Density Lemma) $p \forces \phi\ [\punto{a}_0,\dots,\punto{a}_n]$
    if and only if 
    $\{q\in \PP :  q \forces \phi\ [\punto{a}_0,\dots,\punto{a}_n]\}$
    is dense below $p$.
  \end{enumerate}
\end{theorem}

We followed the new Kunen's book to define
$\forceisa$.  Forcing for atomic formulas is described as a mutual
recursion
%% \begin{multline*}
%%   \forceseq (p,t_1,t_2) \defi 
%%   \forall s\in\dom(t_1)\cup\dom(t_2).\ \forall q\pleq p .\\
%%   \forcesmem(q,s,t_1)\lsii \forcesmem(q,s,t_2)
%% \end{multline*}
%% \begin{multline*}
%%   \forcesmem(p,t_1,t_2) \defi  \forall v\pleq p. \ \exists q\pleq v.\\  
%%   \exists s.\ \exists r\in \PP .\ \lb s,r\rb \in  t_2 \land q
%%   \pleq r \land \forceseq(q,t_1,s)
%% \end{multline*}
but then \cite[p.~257]{kunen2011set} it is cast as a single
recursively defined function $\frcat$ over the wellfounded relation
$\isatt{frecR}$ on tuples $\lb \mathit{ft},t_1,t_2,p\rb$ (where
$\mathit{ft}\in\{0,1\}$ indicates the type of the atomic formula being
forced). Forcing for general formulas is defined by recursion on the
datatype $\formula$. Details on the implementation and proofs of the
Forcing Theorems have been spelled out in our
\cite{2020arXiv200109715G}.


It is to be noted that application of the Forcing theorems do not
require any extra Replacement instances on $M$.

%%% Local Variables: 
%%% mode: latex
%%% TeX-master: "independence_ch_isabelle"
%%% ispell-local-dictionary: "american"
%%% End: 


%%%%%%%%%%%%%%%%%%%%%%%%%%%%%%%%%%%%%%%%%%%%%%%%%%%%%%%%%%%%%%%%%%%%%%          
\section{Proof of Separation}

This proof can be found in the file \verb|Separation_Axiom.thy| of the
development, which we proceed to discuss.

The key technical result is the following:
\begin{isabelle}
  \isacommand{lemma}\isamarkupfalse%
  \ Collect{\isacharunderscore}sats{\isacharunderscore}in{\isacharunderscore}MG\ {\isacharcolon}\isanewline
  \ \ \isakeyword{assumes}\isanewline
  \ \ \ \ {\isachardoublequoteopen}{\isasympi}\ {\isasymin}\ M{\isachardoublequoteclose}\ {\isachardoublequoteopen}{\isasymsigma}\ {\isasymin}\ M{\isachardoublequoteclose}\ {\isachardoublequoteopen}val{\isacharparenleft}G{\isacharcomma}\ {\isasympi}{\isacharparenright}\ {\isacharequal}\ c{\isachardoublequoteclose}\ {\isachardoublequoteopen}val{\isacharparenleft}G{\isacharcomma}\ {\isasymsigma}{\isacharparenright}\ {\isacharequal}\ w{\isachardoublequoteclose}\isanewline
  \ \ \ \ {\isachardoublequoteopen}{\isasymphi}\ {\isasymin}\ formula{\isachardoublequoteclose}\ {\isachardoublequoteopen}arity{\isacharparenleft}{\isasymphi}{\isacharparenright}\ {\isasymle}\ {\isadigit{2}}{\isachardoublequoteclose}\isanewline
  \ \ \isakeyword{shows}\ \ \ \ \isanewline
  \ \ \ \ {\isachardoublequoteopen}{\isacharbraceleft}x{\isasymin}c{\isachardot}\ sats{\isacharparenleft}M{\isacharbrackleft}G{\isacharbrackright}{\isacharcomma}\ {\isasymphi}{\isacharcomma}\ {\isacharbrackleft}x{\isacharcomma}\ w{\isacharbrackright}{\isacharparenright}{\isacharbraceright}{\isasymin}\ M{\isacharbrackleft}G{\isacharbrackright}{\isachardoublequoteclose}
\end{isabelle}
%
From this, using absoluteness, we will be able to derive the
$\phi$-instance of Separation. 

To show that   
\[
S\defi\{x\in c : M[G]\models \phi(x,w)\} \in M[G],
\]
it is enough to provide a name $n\in M$ for this set.
 
The candidate name is
\[
n \defi \{u \in\dom(\pi)\times\PP :M,[u,\PP,\leq,\1,\sig,\pi]\models \psi\}
\]
where
\[
\psi \defi \exists \th\, p.\ x_0=\lb\th,p\rb \y 
   \forceisa(\th\in x_5\y\phi(\th,x_4)).
\]
The fact that $n\in M$ follows by an application of a six-variable
instance of Separation in $M$ (lemma \isatt{six{\isacharunderscore}sep{\isacharunderscore}aux}).

Almost a third part of the proof involves the syntactic handling of
internalized formulas and permutation of variables. The more
substantive portion concerns proving that actually $\val(G,n)=S$.

Let's first focus into the predicate 
\[
M,[u,\PP,\leq,\1,\sig,\pi]\models \psi
\]
defining $n$ by separation. By definition of the satisfaction
relation and permuting variables, we have it is equivalent to the fact
that there exist $\th,p\in M$ with   $u=\lb\th,p\rb$  and 
\[
M,[\PP,\leq,\1,p,\th,\sig,\pi]\models \forceisa(x_4\in
x_6\y\phi(x_4,x_5)). 
\]
% (Note that the variable $x_7$ is not used.)
This, in turn is equivalent by the Definition of Forcing to: For all $M$-generic
filters $F$ such that $p\in F$, 
\[
M,[\val(G,\th),\val(G,\sig),\val(G,\pi)]\models x_4\in
x_6\y\phi(x_4,x_5). 
\] 


%% The proof of
%% \isatt{Collect{\isacharunderscore}sats{\isacharunderscore}in{\isacharunderscore}MG}
%% has three parts:
%% \begin{enumerate}
%% \item Definition of the name $n$;
%% \item Proving that 
%% \item 
%% \end{enumerate}

%%% Local Variables: 
%%% mode: latex
%%% TeX-master: "Separation_In_MG"
%%% ispell-local-dictionary: "american"
%%% End: 


\section{Conclusion}

TODO:
\begin{enumerate}
\item Implement \emph{Basic Set Theory (BST)} by Kunen in
  Constructible: the use of alternatively Replacement or Powerset to
  prove basic absoluteness and closure resuls.
\item Enhance the automatization of formulas
\item Develop the forcing notions to obtain the independence of $\CH$,
  along with the prerrequisite combinatorial results (v.g.\ the
  $\Delta$-system lemma).
\end{enumerate}

%%% Local Variables: 
%%% mode: latex
%%% TeX-master: "forcing_in_isabelle_zf"
%%% ispell-local-dictionary: "american"
%%% End: 


%%%%%%%%%%%%%%%%%%%%%%%%%%%%%%%%%%%%%%%%%%%%%%%%%%%%%%%%%%%%%%%%%%%%%%%%%%%%%%%%

\bibliographystyle{mi-estilo-else}
\bibliography{../LSFA/citados}
%\documentclass{article}
\usepackage{isabelle,isabellesym}
\renewcommand{\ttdefault}{cmtt}
\usepackage{xcolor}
\usepackage{csquotes}
\usepackage{enumitem}

\newlist{inlinelist}{enumerate*}{1}
\setlist*[inlinelist,1]{%
  label=(\roman*),
}
\usepackage{hyperref}
\usepackage[numbers]{natbib}
\input{header-draft}
\makeatletter
\def\foottext{\gdef\@thefnmark{}\@footnotetext}
\makeatother
\newcommand{\keywords}[1]{\foottext{\emph{Keywords:} #1}}
\newcommand{\ack}[1]{\par\bigskip \noindent \emph{Acknowledgment:} #1}

\hypersetup{
  pdftitle={Separation in Generic Extensions for Isabelle},
  pdfsubject={Computer Science},
  pdfkeywords={Isabelle/ZF, forcing, names, generic extension, constructibility},
  colorlinks,
  linkcolor={blue!40!black},
  citecolor={blue!40!black},
  urlcolor={blue!40!black}
}

\begin{document}
\title{Separation in Generic Extensions for Isabelle}
\author{Emmanuel Gunther
  \and 
  Miguel Pagano
  \and 
  Pedro S\'anchez Terraf}
\maketitle

\begin{abstract}
  We mechanize, in the proof assistant
  Isabelle, a proof of the
  axiom-scheme of Separation in 
  generic extensions of models of set theory  
  by using the fundamental theorems of forcing.
  We also formalize the satisfaction of the axioms of
  Extensionality, Foundation, Union, and Powerset. The axiom of
  Infinity is likewise treated, under additional assumptions on the ground
  model.
  We also  extend Paulson's library on constructibility  with
  renaming of variables for internalized formulas, an improvement on
  definitions by recursion on well-founded  relations and sharpening
  of the hypotheses in his development of relativization and
  absoluteness.
\end{abstract}
\keywords{
Isabelle/ZF, forcing, names, generic extension, constructibility.
}

%%%%%%%%%%%%%%%%%%%%%%%%%%%%%%%%%%%%%%%%%%%%%%%%%%%%%%%%%%%%%%%%%%%%%%%%%%%%%%%%
%%%%%%%%%%%%%%%%%%%%%%%%%%%%%%%%%%%%%%%%%%%%%%%%%%%%%%%%%%%%%%%%%%%%%%          
\section{Introduction}
% no \IEEEPARstart
Zermelo-Fraenkel Set Theory ($\ZF$) has a prominent place among formal
theories; in particular, it provides a foundation for mathematics and
most of the formal toolkit used everyday by the computer scientist has
also Set Theory at its base (cf.~\cite{paulson1995set}). In this time
of mechanization of mathematics~\cite{avigad2018mechanization}, it
seems natural to ask for a mechanization of the most salient results
of Set Theory.

After G\"odel's Incompleteness Theorems, we cannot expect to have a
formal proof of the consistency of Set Theory in $\ZF$. Besides its own
consistency, there are other results which are undecided by $\ZF$: the
undecidability of Continuum Hypothesis leads to the development of
techniques for independence proofs. First G\"odel introduced the
theory of \emph{inner models}, which gives rise to his model $L$ of
the \emph{Axiom of Constructibility} \cite{godel-L} and proved the
relative consistency of the Axiom of Choice and the Generalized
Continuum Hypothesis with $\ZF$. Thirty years later Paul
J. Cohen~\cite{Cohen-CH-PNAS} devised the technique of \emph{forcing},
which is the only known way of \emph{extending} models of $\ZF$; in
particular, it can be used to prove the relative consistency of the
negation of the Continuum Hypothesis. 

In this work we address a substantial part of formalizing the proof
that given a model $M$ of $\ZF$, any \emph{generic extension} $M[G]$
given by forcing also satisfies $\ZF$. As remarked by
\citet[][p.250]{kunen2011set} \enquote{[...] in verifying that $M[G]$
  is a model for set theory, the hardest axiom to verify is
  Comprehension.}  The most important achievement of this paper is the
mechanization in Isabelle a considerable part of this result; en route
to this, we also formalized the satisfaction by $M[G]$ of
Extensionality, Foundation, Union, and Infinity. % We have already
% proved Pairing in the first report of our project
% \cite{2018arXiv180705174G}.

Our development benefited from the remarkable work done by Lawrence
Paulson \cite{paulson_2003} on the formalization of G\"odel's
constructible universe in the proof assistant \emph{Isabelle}. The
ultimate goal of our project is the formalization of forcing to
complete the mechanization of the independence of the Continuum
Hypothesis. We think that this project constitutes an interesting case
which stresses how feasible is to formally implement mathematics that
involve several levels of reasoning.

The \emph{Formal Abstract} project~\cite{hales-fabstracts} proposes
that the formalization of mathematics by writing the statements of
results and the material upon which they are based (definitions,
propositions, lemmas), but ommiting the proofs. We took a less drastic
position: since the proof that those axioms hold in generic extension
is independent of the \emph{proofs} of the ``fundamental'' theorems of
forcing, we assumed these results. Let us remark that the definition
of the forces relation is, by itself, quite demanding; the
formalization of it and of the fundamental theorems of forcing roughly
comprises little less than a half of our full project.

It might be a little surprising the lack of formalizations of forcing
and generic extensions. As far as we know, the development of
\citet{JFR6232} in homotopy type theory for constructing generic
extensions in a sheaf-theoretic setting is the unique mechanization of
forcing. This contrast with the fruitful use of forcing techniques to
extend the Curry-Howard isomorphism to classical axioms
\cite{Miquel:2011:FPT:2058525.2059614,lmcs:1070}. Moreover, the
combination of forcing with intuitionistic type theory
\cite{Coquand:2009:FTT:1807662.1807665,coquand2010note} gives rise
both to positive results (an algorithm to obtain witnesses of the
continuity of definable functionals \cite{coquand2012computational})
and also negative (the independence of Markov's principle
\cite{lmcs:3859}). In the same strand of forcing from the point of
view of proof theory \cite{avigad_2004} are the conservative
extensions of CoC with forcing conditions
\cite{jaber:hal-01319066,Jaber:2012:ETT:2358958.2359524}.

% \fbox{Parece mucho comienzo sólo para introducir a Kunen}
% \fbox{¿lo puedo achurar un poco?}

% In a gross simplification, there are two aspects to a formalization
% project like this one: thematic and programmatic. The first concerns
% the handling of all the theoretical concepts and results in the
% subject, while the second involves the practical issues of the
% implementation and design. In the case of forcing, the main intricacy
% lies in the first aspect. In this sense, following a sensible
% presentation of the material is key.  The authoritative reference 
% on the subject during the last 30 years has been Kunen's classical
% \cite{kunen1980}. In our
% formalizaton we have followed a recent rewrite \cite{kunen2011set}
% of that  textbook, which presents the material in the same sharp 
% style but offering a lot of details. In some sense this project
% wouldn't exist without this book. As alternative, introductory
% resources, the  interested reader can check
% \cite{chow-beginner-forcing}; also, the book \cite{weaver2014forcing}
% contains a thorough treament minimizing the technicalities.

In pursuing the proof of preservation on generic extensions we
extended Paulson's library with:
\begin{inlinelist}
\item renaming of variables, which with little effort can be extended
  to substitutions;
\item an improvement on definitions by recursion on well-founded
  relations; 
\item a better hierarchy of locales; and
\item a new choice principle and a version of Rassiowa-Sikorski which
  ensure the existence of generic filters for countable and transitive
  models of $\ZF$; these points were already communicated in the
  first report \cite{2018arXiv180705174G}.
\end{inlinelist} 

We briefly describe the contents of each
section. Section~\ref{sec:isabelle} contains the bare minimun
requirements to understand the (meta)logics used in Isabelle. Next, an
overview of the model theory of set theory is presented in
Section~\ref{sec:axioms-models-set-theory}. There is an ``internal''
representation of first-order formulas as sets, implemented by
Paulson; Section~\ref{sec:renaming} discusses syntactical
transformations of the former, mainly permutation of variables. 
In Section~\ref{sec:generic-extensions} the generic extensions are
succintly reviewed and how the treatment of well founded recursion in
Isabelle was enhanced. We take care of the ``easy axioms'' in
Section~\ref{sec:easy-axioms}; these are the ones that
do not depend on the forcing theorems. We describe the latter in
Section~\ref{sec:forcing}. We adapted the  work by Paulson to our
needs, and this is described in
Section~\ref{sec:hack-constructible}. We present the proof
of the Separation Axiom Scheme in Section~\ref{sec:proof-separation},
which follows closely its implementation. A plan for future work and
some immediate conclusions are offered in
Section~\ref{sec:conclusions-future-work}.

%%% Local Variables: 
%%% mode: latex
%%% TeX-master: "Separation_In_MG"
%%% ispell-local-dictionary: "american"
%%% End: 


%%%%%%%%%%%%%%%%%%%%%%%%%%%%%%%%%%%%%%%%%%%%%%%%%%%%%%%%%%%%%%%%%%%%%%          
\section{Isabelle}
\label{sec:isabelle}
%-%-%-%-%-%-%-%-%-%-%-%-%-%-%-%-%-%-%-%-%-%-%-%-%-%-%-%-%-%-%-%-%-%-%-
\subsection{Logics}
\label{sec:logics}
Isabelle provides a meta-language called \emph{Pure} that consists of
a fragment of higher order logic, where \isatt{\isasymRightarrow} is
the function-space arrow. The meta-Boolean type is called
\isatt{prop}. Meta-connectives
\isatt{\isasymLongrightarrow} and \isatt{\&\&\&} fulfill the role of
implication and conjunction, and the meta-binder \isatt{\isasymAnd}
corresponds to universal quantification. 

On top of \emph{Pure}, theories/object logics can be defined, with
their own types, connectives and rules. Rules can be written  using
meta-implication: ``$P$, $Q$, and $R$ yield $S$'' can be written
\[
P \ \isatt{\isasymLongrightarrow}\ Q\ \isatt{\isasymLongrightarrow}\ R\ \isatt{\isasymLongrightarrow}\ S
\]
(as usual,  \isatt{\isasymLongrightarrow} associates to the right), and
syntactic sugar is provided to curry the previous rule as follows:
\[
\isasymlbrakk P; Q; R \isasymrbrakk \ \isatt{\isasymLongrightarrow}\ S.
\]
One further example is given by induction on the natural numbers
\isatt{nat},
\[
\isasymlbrakk P(0);\ (\textstyle\isasymAnd
x.\ P(x)\ \isasymLongrightarrow\ P(\isatt{succ}(x))) \isasymrbrakk
\ \isasymLongrightarrow\ P(n), 
\]
where we are omitting the ``typing'' assumtions on $n$ and $x$.

We work in the object theory \emph{Isabelle/ZF}. Two types are defined
in this theory: \tyo, the object-Booleans, and \tyi,
sets. It must be insisted that the types are defined axiomatically, not
recursively. That is, although there are constants and functions that
generate elements of both types, neither of them are 
\emph{initial}, in the sense that they are not the least types
obtained by combining the given constants and operations. This will
have concrete consequences in our strategy to approach the
development. From the beginning, we had to resort to
\emph{internalized} formulas, i.e.\ elements of type $\tyi$ that
encode first-order formulas with a binary relation symbol, and the
satisfaction predicate \isatt{sats\,::\,"i\isasymRightarrow i\isasymRightarrow i\isasymRightarrow o"}  between a set
model with an environment and an internalized formula (where the
relation symbol is interpreted as membership). The set 
\isatt{formula::"\tyi"}
 of internalized
formulas is defined by recursion and hence it is possible to perform
inductive arguments using them. In this sense, the object-logic level
is further divided into \emph{internal} and \emph{external}
sublevels. 

\medskip
\fbox{Portar a \textbf{Isabelle2018} antes de enviar.}

\medskip

The source code is written for the 2018 version of Isabelle (with
minor modifications, it can be run in Isabelle2016-1). Most of it is
presented in the (nowadays standard) declarative flavour called
\emph{Isar} \cite{DBLP:conf/tphol/Wenzel99}, where intermediate
statements in the course 
of a proof are explicitly stated, interspersed with automatic 
tactics handling more trivial steps. The goal is that the resulting
text, a \emph{proof document}, can be understood without the need of
running it.

%-%-%-%-%-%-%-%-%-%-%-%-%-%-%-%-%-%-%-%-%-%-%-%-%-%-%-%-%-%-%-%-%-%-%-
\subsection{Locales}
\label{sec:locales}
Locales \cite{ballarin2010tutorial} provide a neat facility to
encapsulate a context (fixed objects and assumptions on them) that is
to be used in proving several theorems, as in usual mathematical
practice. 

In this paper, locales have a further use. The \emph{Fundamental
  Theorems of Forcing} we use talk about a specific map $\forceisa$
from formulas to formulas. The definition of $\forceisa$ is involved
and we will not dwell on this now; but applications of those theorems
do not require to know how it is defined. Therefore, we black-box it
and pack everything in a locale called \texttt{forcing\_thms} that
assumes that there is such a 
map that satisfies the Fundamental Theorems.

%% \begin{description}
%% \item[\texttt{forcing\_notion}] preorden con top
%% \item[\texttt{countable\_generic}] lo anterior con una familia contable de densos.
%% \item[\texttt{M\_ZF}] axiomas.
%% \item[\texttt{forcing\_data}]: lo anterior contable transitivo y una notion.
%% \item[\texttt{forcing\_thms}]: eso.
%% \item[\texttt{G\_generic}]: lo anterior y G es genérico.
%% \item[\texttt{M\_extra\_assms}]: check in M e instancia de reemplazo para G.
%% \item[\texttt{G\_generic\_extra}]: los dos anteriores (no sé si sigue estando)
%% \end{description}


%%% Local Variables: 
%%% mode: latex
%%% TeX-master: "Separation_In_MG"
%%% ispell-local-dictionary: "american"
%%% End: 


%%%%%%%%%%%%%%%%%%%%%%%%%%%%%%%%%%%%%%%%%%%%%%%%%%%%%%%%%%%%%%%%%%%%%%          
\section{Axioms of set theory}


%%% Local Variables: 
%%% mode: latex
%%% TeX-master: "Separation_In_MG"
%%% ispell-local-dictionary: "american"
%%% End: 


%%%%%%%%%%%%%%%%%%%%%%%%%%%%%%%%%%%%%%%%%%%%%%%%%%%%%%%%%%%%%%%%%%%%%%          
\section{Renaming}
\label{sec:renaming}
\newcommand{\renaming}[2]{(#1)[#2]}
\newcommand{\inFm}[2]{#1 \in #2}
\newcommand{\eqFm}[2]{#1 = #2}
\newcommand{\negFm}[1]{\neg #1}
\newcommand{\andFm}[2]{#1 \wedge #2}
\newcommand{\forallFm}[1]{\forall #1}

\newcommand{\inIFm}[2]{\mathsf{Member}(#1,#2)}
\newcommand{\eqIFm}[2]{\mathsf{Equal}(#1,#2)}
\newcommand{\nandIFm}[2]{\mathsf{Nand}(#1,#2)}
\newcommand{\forallIFm}[1]{\mathsf{Forall(#1)}}


In the course of our work we need to reason about renaming of formulas
and its effect on their satisfiability. Internalized formulas are
implemented using de Bruijn indices for variables and the arity of a
formula $\phi$ gives the least natural number containing all the free
variables in $\phi$. Following Fiore et al. \cite{fiore-abssyn}, one
can understand the arity of a formula as the context of the free
variables; notice that the arity of $\forallFm{\phi}$ is the
predecessor of the arity of $\phi$. In order to understand renamings,
it is helpful to think of $\mathsf{succ}(n)$ as the coproduct
$1+n = \{0\} \cup \{1,\dots,n\}$; given a renaming $f \colon n \to m$,
the unique morphism $\mathsf{id}_1+f \colon 1+n \to 1+m$ is used to
rename free variables in a quantified formula.

\begin{definition}[Renaming]
  Let $\phi$ be a formula of arity $n$ and let $f \colon n \to m$, the
  renaming of $\phi$ by $f$, denoted $\renaming{\phi}{f}$, is defined
  by recursion on $\phi$:
  \begin{gather*}
    \renaming{\inFm{i}{j}}{f} = \inFm{f\,i}{f\,j}\\
    \renaming{\eqFm{i}{j}}{f} = \eqFm{f\,i}{f\,j}\\
    \renaming{\negFm{\phi}}{f} = \negFm{\renaming{\phi}{f}}\\
    \renaming{\andFm{\phi}{\psi}}{f} = \andFm{\renaming{\phi}{f}}{\renaming{\psi}{f}}\\
    \renaming{\forallFm{\phi}}{f} = \forallFm{\renaming{\phi}{\mathsf{id}_1+f}}
  \end{gather*}
\end{definition}

As usual, if $M$ is a set, $a_0,\dots,a_{n-1}$ are elements of $M$, and
$\phi$ is a formula of arity $n$, we write
\[
M,[a_0,\dots,a_{n-1}] \models \phi
\]
to denote that $\phi$ is satisfied by $M$ when $i$ is interpreted
as $a_i$ ($i=0,\dots,n-1$). We call the list $[a_0,\dots,a_{n-1}]$ the
\emph{environment}.

The action of renaming on environments re-indexes the variables. An
easy proof connects satisfaction with renamings.
\begin{lemma}
  \label{lem:renaming}
  Let $\phi$ be a formula of arity $n$, $f \colon n \to m$ be a
  renaming, and let $\rho=[a_1,\ldots,a_n]$ and
  $\rho'=[b_1,\ldots,b_m]$ be environments of length $n$ and $m$,
  respectively. If for all $i \in n$, $a_i = b_{j}$ where $j=f\,i$,
  then $M,\rho\models \phi$ is equivalent to
  $M,\rho' \models \renaming{\phi}{f}$.
\end{lemma}

An important resource in Isabelle/ZF is the facility for defining
inductive sets \cite{paulson2000fixedpoint,paulson1995set} together
with a principle for defining functions by structural recursion.
Internalized formulas are a prime example of this, so we define
a function \isa{ren} that associates to each formula an internalized
function that can be later applied to suitable arguments. Notice that
Paulson used \isa{Nand} because it is more economical.
\begin{isabelle}
\isamarkuptrue%
\isacommand{consts}\isamarkupfalse%
\ ren\ {\isacharcolon}{\isacharcolon}\ {\isachardoublequoteopen}i{\isacharequal}{\isachargreater}i{\isachardoublequoteclose}\isanewline
\isacommand{primrec}\isamarkupfalse%
\isanewline
\ {\isachardoublequoteopen}ren{\isacharparenleft}Member{\isacharparenleft}x{\isacharcomma}y{\isacharparenright}{\isacharparenright}\ {\isacharequal}\isanewline
\ \ {\isacharparenleft}{\isasymlambda}n\ {\isasymin}\ nat\ {\isachardot}\ {\isasymlambda}\ m\ {\isasymin}\ nat{\isachardot}\ {\isasymlambda}f\ {\isasymin}\ n\ {\isasymrightarrow}\ m{\isachardot}\ Member\ {\isacharparenleft}f{\isacharbackquote}x{\isacharcomma}\ f{\isacharbackquote}y{\isacharparenright}{\isacharparenright}{\isachardoublequoteclose}\isanewline
\ \isanewline
\ {\isachardoublequoteopen}ren{\isacharparenleft}Equal{\isacharparenleft}x{\isacharcomma}y{\isacharparenright}{\isacharparenright}\ {\isacharequal}\isanewline
\ \ {\isacharparenleft}{\isasymlambda}n\ {\isasymin}\ nat\ {\isachardot}\ {\isasymlambda}\ m\ {\isasymin}\ nat{\isachardot}\ {\isasymlambda}f\ {\isasymin}\ n\ {\isasymrightarrow}\ m{\isachardot}\ Equal\ {\isacharparenleft}f{\isacharbackquote}x{\isacharcomma}\ f{\isacharbackquote}y{\isacharparenright}{\isacharparenright}{\isachardoublequoteclose}\isanewline
\ \isanewline
\ {\isachardoublequoteopen}ren{\isacharparenleft}Nand{\isacharparenleft}p{\isacharcomma}q{\isacharparenright}{\isacharparenright}\ {\isacharequal}\isanewline
\ \ {\isacharparenleft}{\isasymlambda}n\ {\isasymin}\ nat\ {\isachardot}\ {\isasymlambda}\ m\ {\isasymin}\ nat{\isachardot}\ {\isasymlambda}f\ {\isasymin}\ n\ {\isasymrightarrow}\ m{\isachardot}\ 
Nand\ {\isacharparenleft}ren{\isacharparenleft}p{\isacharparenright}{\isacharbackquote}n{\isacharbackquote}m{\isacharbackquote}f{\isacharcomma}\ ren{\isacharparenleft}q{\isacharparenright}{\isacharbackquote}n{\isacharbackquote}m{\isacharbackquote}f{\isacharparenright}{\isacharparenright}{\isachardoublequoteclose}\isanewline
\ \isanewline
\ {\isachardoublequoteopen}ren{\isacharparenleft}Forall{\isacharparenleft}p{\isacharparenright}{\isacharparenright}\ {\isacharequal}\isanewline
\ \  {\isacharparenleft}{\isasymlambda}n\ {\isasymin}\ nat\ {\isachardot}\ {\isasymlambda}\ m\ {\isasymin}\ nat{\isachardot}\ {\isasymlambda}f\ {\isasymin}\ n\ {\isasymrightarrow}\ m{\isachardot}\ \isanewline
\ \ \ Forall\ {\isacharparenleft}ren{\isacharparenleft}p{\isacharparenright}{\isacharbackquote}succ{\isacharparenleft}n{\isacharparenright}{\isacharbackquote}succ{\isacharparenleft}m{\isacharparenright}{\isacharbackquote}sum{\isacharunderscore}id{\isacharparenleft}n{\isacharcomma}f{\isacharparenright}{\isacharparenright}{\isacharparenright}{\isachardoublequoteclose}
\end{isabelle}

In the last equation, \isa{sum{\isacharunderscore}id} corresponds to
the coproduct morphism $\mathsf{id}_{1}+f \colon 1 + n \to 1 +
n$. Since the schema for recursively defined functions does not allow
parameters, we are forced to return a function of three arguments
(\isa{n,m,f}). This also exposes some inconveniences of working in the
untyped realm of set theory; for example to use \isa{ren} we will need
to prove that the renaming is a function. Besides some auxiliary
results (for example that the application of renaming to suitable
arguments yields a formula), the main result corresponding to
Lemma~\ref{lem:renaming} is:
\begin{isabelle}
\isacommand{lemma}\isamarkupfalse%
\ sats{\isacharunderscore}iff{\isacharunderscore}sats{\isacharunderscore}ren\ {\isacharcolon}\ \isanewline
\ \ \isakeyword{fixes}\ {\isasymphi}\isanewline
\ \ \isakeyword{assumes}\ {\isachardoublequoteopen}{\isasymphi}\ {\isasymin}\ formula{\isachardoublequoteclose}\isanewline
\ \ \isakeyword{shows}\ \ {\isachardoublequoteopen}{\isasymAnd}\ n\ m\ {\isasymrho}\ {\isasymrho}{\isacharprime}\ f\ {\isachardot}\ \isanewline
\ \ {\isasymlbrakk}n{\isasymin}nat\ {\isacharsemicolon}\ m{\isasymin}nat\ {\isacharsemicolon}\ f\ {\isasymin}\ n{\isasymrightarrow}m\ {\isacharsemicolon}\ arity{\isacharparenleft}{\isasymphi}{\isacharparenright}\ {\isasymle}\ n\ {\isacharsemicolon}\isanewline
\ \ \ \ \ {\isasymrho}\ {\isasymin}\ list{\isacharparenleft}M{\isacharparenright}\ {\isacharsemicolon}\ {\isasymrho}{\isacharprime}\ {\isasymin}\ list{\isacharparenleft}M{\isacharparenright}\ {\isacharsemicolon}\ \isanewline
\ \ \ {\isasymAnd}\ i\ {\isachardot}\ i{\isacharless}n\ {\isasymLongrightarrow}\ nth{\isacharparenleft}i{\isacharcomma}{\isasymrho}{\isacharparenright}\ {\isacharequal}\ nth{\isacharparenleft}f{\isacharbackquote}i{\isacharcomma}{\isasymrho}{\isacharprime}{\isacharparenright}\ {\isasymrbrakk}\ {\isasymLongrightarrow}\isanewline
\ \ sats{\isacharparenleft}M{\isacharcomma}{\isasymphi}{\isacharcomma}{\isasymrho}{\isacharparenright}\ {\isasymlongleftrightarrow}\ sats{\isacharparenleft}M{\isacharcomma}ren{\isacharparenleft}{\isasymphi}{\isacharparenright}{\isacharbackquote}n{\isacharbackquote}m{\isacharbackquote}f{\isacharcomma}{\isasymrho}{\isacharprime}{\isacharparenright}{\isachardoublequoteclose}\end{isabelle}

The use of this lemma involves some repetitive tasks (mainly proving
that the renaming is in fact a function). We would like to develop
some \texttt{ML} tools in order to automatize this.
%% We think that it should be possible, and clearer,
%% to express all the current renamings in the theory \isa{Formula} using
%% our approach. 


%%% Local Variables: 
%%% mode: latex
%%% TeX-master: "Separation_In_MG"
%%% ispell-local-dictionary: "american"
%%% End: 


%%%%%%%%%%%%%%%%%%%%%%%%%%%%%%%%%%%%%%%%%%%%%%%%%%%%%%%%%%%%%%%%%%%%%%          
\section{Generic extensions}
\label{sec:generic-extensions}
We will swiftly review some definitions in order to reach the concept
of \emph{generic extension}. As first preliminary definitions, a \emph{forcing
notion} $\lb\PP,\leq,\1\rb$ is simply a preorder with top ($\1$), and a \emph{filter}
$G\sbq\PP$ is an increasing subset which is downwards
compatible. Given a ctm $M$ of $\ZF$, a forcing
notion in $M$, and a filter $G$, a new set $M[G]$ is defined. Each
element $a\in M[G]$ is 
determined by its \emph{name} $\dot a$ in $M$. Actually, the structure of
each $\dot a$ is used to construct $a$. They are related by a
map $\val$ that takes $G$ as a parameter:
\[
\val(G,\dot a) = a.
\] 
Then the extension is defined by the image of the map $\val(G,\cdot)$:
\[
M[G] \defi \{\val(G,\tau): \tau\in M\}.
\]
Metatheoretically, it is straightforward to see that $M[G]$ is a
transitive set that satisfies some axioms of $\ZF$ (see
Section~\ref{sec:easy-axioms}) and includes $M\cup\{G\}$. Nevertheless
there is no a priori reason for $M[G]$ to satisfy either Separation, Powerset
or Replacement. The original insight by Cohen was to define the notion
of \emph{genericity} for a filter $G\sbq\PP$ and to prove that
whenever $G$ is generic, $M[G]$ will satisfy $\ZF$. Remember that a
filter is generic if it intersects all the dense sets in $M$; in
\cite{2018arXiv180705174G} we formalized the Rasiowa-Sikorski lemma which
proves the existence of generic filters for ctms.

The Separation Axiom  is the first that requires the notion of
genericity and the use of the forcing machinery, which we review in
the Section~\ref{sec:forcing}.

%%% Local Variables: 
%%% mode: latex
%%% TeX-master: "Separation_In_MG"
%%% ispell-local-dictionary: "american"
%%% End: 


%%%%%%%%%%%%%%%%%%%%%%%%%%%%%%%%%%%%%%%%%%%%%%%%%%%%%%%%%%%%%%%%%%%%%%          
\subsection*{Recursion and Values of Names}

The map $\val$ used in the definition of the generic extension is
characterized by the recursive equation
\begin{equation}
  \label{eq:val}
  \val(G,\tau) = \{val(G,\sigma) :\exists p \in\PP .%
  \lb\sigma,p\rb \in \tau \wedge p \in G \}
\end{equation}

As is well-known, the principle of transfinite recursion on
well-founded relations~\cite[p. 48]{kunen2011set}) allows us to define
a recursive function $F \colon A\to A$ by choosing a well-founded
relation $R \subseteq A\times A$ and a functional
$H\colon A\times (A \to A) \to A$ satisfying
$F(a)=H(a,F\!\upharpoonright\!(R^{-1}(a)))$. \citet{paulson1995set}
made this principle available in Isabelle/ZF via the the operator
\isa{wfrec}. The formalization of the functional $\mathit{Hv}$ is
straightforward:
%
\begin{isabelle}
\isacommand{definition}\isamarkupfalse%
\isanewline
\ \ Hv\ {\isacharcolon}{\isacharcolon}\ {\isachardoublequoteopen}i{\isasymRightarrow}i{\isasymRightarrow}i{\isasymRightarrow}i{\isachardoublequoteclose}\ \isakeyword{where}\isanewline
\ \ {\isachardoublequoteopen}Hv{\isacharparenleft}G{\isacharcomma}y{\isacharcomma}f{\isacharparenright}\ {\isacharequal}{\isacharequal}\ {\isacharbraceleft}\ f{\isacharbackquote}x\ {\isachardot}{\isachardot}\ x{\isasymin}\ domain{\isacharparenleft}y{\isacharparenright}{\isacharcomma}\ {\isasymexists}p{\isasymin}P{\isachardot}\ {\isacharless}x{\isacharcomma}p{\isachargreater}\ {\isasymin}\ y\ {\isasymand}\ p\ {\isasymin}\ G\ {\isacharbraceright}{\isachardoublequoteclose}
\end{isabelle}
In the references \cite{kunen2011set,weaver2014forcing} $\val$ is
applied only to \emph{names}, that are certain elements of $M$
characterized by a recursively defined predicate. The well-founded
relation used to justify Equation~\eqref{eq:val} is
\[ x \mathrel{\mathit{ed}} y \iff \exists p . \lb x,p\rb\in y. \] In
order to use \isa{wfrec} the relation should be expressed as a set, so
we took the restriction of $\mathit{ed}$ to the whole universe
$M$; i.e. $\mathit{ed}\cap M\times M$.  It was inconvenient to work
with that definition because it imposed an obligation showing that
some set is in $M$. The remedy was to restrict $\mathit{ed}$ to the
transitive closure of the actual parameter:
\begin{isabelle}
\isacommand{definition}\isamarkupfalse%
\isanewline
\ val\ {\isacharcolon}{\isacharcolon}\ {\isachardoublequoteopen}i{\isasymRightarrow}i{\isasymRightarrow}i{\isachardoublequoteclose}\ \isakeyword{where}\isanewline
\ {\isachardoublequoteopen}val{\isacharparenleft}G{\isacharcomma}{\isasymtau}{\isacharparenright}{\isacharequal}{\isacharequal}\ wfrec{\isacharparenleft}edrel{\isacharparenleft}eclose{\isacharparenleft}{\isacharbraceleft}{\isasymtau}{\isacharbraceright}{\isacharparenright}{\isacharparenright}{\isacharcomma}{\isasymtau}{\isacharcomma}Hv{\isacharparenleft}G{\isacharparenright}{\isacharparenright}{\isachardoublequoteclose}
\end{isabelle}
A key result, albeit intuitive and rather easy, to make this
definition work is that when computing the value of a recursive
function on some argument $a$, one can restrict the relation to some
ambient set if it includes $a$ and all of its predecessesors.
\begin{isabelle}
\isacommand{lemma}\isamarkupfalse%
\ wfrec{\isacharunderscore}restr\ {\isacharcolon}\isanewline
\ \ \isakeyword{assumes}\ {\isachardoublequoteopen}relation{\isacharparenleft}r{\isacharparenright}{\isachardoublequoteclose}\ {\isachardoublequoteopen}wf{\isacharparenleft}r{\isacharparenright}{\isachardoublequoteclose}\ \isanewline
\ \ \isakeyword{shows}\ \ {\isachardoublequoteopen}a{\isasymin}A\ {\isasymLongrightarrow}\ {\isacharparenleft}r{\isacharcircum}{\isacharplus}{\isacharparenright}{\isacharminus}{\isacharbackquote}{\isacharbackquote}{\isacharbraceleft}a{\isacharbraceright}\ {\isasymsubseteq}\ A\ {\isasymLongrightarrow}\ \isanewline
\ \ \ \ \ \ \ \ \ \ wfrec{\isacharparenleft}r{\isacharcomma}a{\isacharcomma}H{\isacharparenright}\ {\isacharequal}\ wfrec{\isacharparenleft}r{\isasyminter}A{\isasymtimes}A{\isacharcomma}a{\isacharcomma}H{\isacharparenright}{\isachardoublequoteclose}
\end{isabelle}

This lemma gives rise to some useful results about values of
names. First, we characterize $\val$ as in Equation~(\ref{eq:val}):
\begin{isabelle}
  \isacommand{lemma}\isamarkupfalse%
  \ def{\isacharunderscore}val{\isacharcolon}\isanewline
  \ {\isachardoublequoteopen}val{\isacharparenleft}G{\isacharcomma}x{\isacharparenright}\ {\isacharequal}\ {\isacharbraceleft}val{\isacharparenleft}G{\isacharcomma}t{\isacharparenright}\ {\isachardot}{\isachardot}\ t{\isasymin}domain{\isacharparenleft}x{\isacharparenright}\ {\isacharcomma}\isanewline
\ \ \ \ \ \ \ \ \ \ \ \ \ \ \ \ \ \ \ {\isasymexists}p{\isasymin}P{\isachardot}\ {\isacharless}t{\isacharcomma}p{\isachargreater}{\isasymin}x\ {\isasymand}\ p{\isasymin}G\ {\isacharbraceright}{\isachardoublequoteclose}
\end{isabelle}
and the monotonicity of $\val$ follows automatically after a
substitution.
\begin{isabelle}
\isacommand{lemma}\isamarkupfalse%
\ val{\isacharunderscore}mono{\isacharcolon}\ {\isachardoublequoteopen}x{\isasymsubseteq}y\ {\isasymLongrightarrow}\ val{\isacharparenleft}G{\isacharcomma}x{\isacharparenright}\ {\isasymsubseteq}\ val{\isacharparenleft}G{\isacharcomma}y{\isacharparenright}{\isachardoublequoteclose}\isanewline
%
\ \ \isacommand{by}\isamarkupfalse%
\ {\isacharparenleft}subst\ {\isacharparenleft}{\isadigit{1}}\ {\isadigit{2}}{\isacharparenright}\ def{\isacharunderscore}val{\isacharcomma}\ force{\isacharparenright}%
\end{isabelle}
More interestingly we can give a neat equation for values of
names defined by Separation, say $B = \{x\in A\times \PP.\ Q(x)\}$,
then
\begin{equation}
\val(G,B) = \{\val(G,t) : t\in A , \exists p\in \PP \cap G.\ Q(\lb t,p\rb) \} \label{eq:val-name-sep}
\end{equation}


\fbox{Creo que deberíamos poner acá \isa{check} y \isa{check\_in\_M} }

%%% Local Variables: 
%%% mode: latex
%%% TeX-master: "Separation_In_MG"
%%% ispell-local-dictionary: "american"
%%% End: 


% %%%%%%%%%%%%%%%%%%%%%%%%%%%%%%%%%%%%%%%%%%%%%%%%%%%%%%%%%%%%%%%%%%%%%%          
\section{Prior results}

Paulson's work provides an appropriate starting point for our
developments.

There are other formalizations of set theory (Isabelle/HOL, Mizar, Automath).

To the best of our knowledge, only preliminary approaches to the
formalization of forcing exist; v.g. \cite{Quirin}, which has already been
discussed in our previous work \cite{2018arXiv180705174G}. 

%%% Local Variables: 
%%% mode: latex
%%% TeX-master: "Separation_In_MG"
%%% ispell-local-dictionary: "american"
%%% End: 


%%%%%%%%%%%%%%%%%%%%%%%%%%%%%%%%%%%%%%%%%%%%%%%%%%%%%%%%%%%%%%%%%%%%%%          
\section{Hacking of \isatt{ZF-Constructible}}

In \cite{paulson_2003}, Paulson presented his formalization of the
relative consistency of the Axiom of Choice. This development is
included inside the Isabelle distribution with session name
\isatt{ZF-Constructible}. The main technical devices, invented by
G\"odel for this purpose, are \emph{relativization} and
\emph{absoluteness}. In a nutshell, to relativize a formula $\phi$ to
a class $C$, it is enough to restrict its quantifiers to $C$. The
example of \isatt{upair\_ax} in
Section~\ref{sec:axioms-models-set-theory}, the relativized version of
the Pairing Axiom, is extracted from \texttt{Relative}, one of the
core theories of \isatt{ZF-Constructible}. On the other hand, $\phi$
is \emph{absolute} for $C$ if it is equivalent to its relativization,
meaning that the statement made $\phi$ coincides with what $C$
``believes'' $\phi$ is saying. Paulson shows that under certain
hypothesis  on a class $M$ (condensed in the locale \isatt{M\_trivial}), a plethora of
absoluteness and closure results can be proved about $M$.

The development of forcing, and the study of ctms in general, takes
absoluteness as a starting point. We were not able to work with
\isatt{ZF-Constructible} right out-of-the-box. The main reason is that
we can't expect state the ``class version'' of Replacement for a
\emph{set} $M$ by
using first-order formulas, since predicates \isatt{P::"i=>o"} can't
be proved to be only the definable ones. Therefore, we had to make
some modifications to the various sets of hypothesis in several
locales to make the results available as tools for the present and
future developments.

%% There are several lemmas that were declared, in later developments, as
%% introduction/simplification rules (notably, the rule
%% \isatt{equalityI}). They raised a warning, and we have 
%% eliminated some of them, but in some cases we had to keep them because
%% the proof works because it is insisted that the rule is \emph{safe}.

The most notable changes, located in the theory \texttt{Relative}, are
the following:
\begin{enumerate}
\item\label{item:1} We eliminated the requirement that the relative Axiom of Replacement
  is satisfied by a class $M$ to be in the locale \isatt{M\_trivial}. 
\item\label{item:2} We moved the requirement of the Powerset Axiom to \isatt{M\_basic}. 
\item\label{item:3} We replaced the need that the set of natural numbers is in $M$ by the
  milder hypothesis that $M(0)$. Actually, most results should follow
  by only assuming that $M$ is nonempty.
\end{enumerate}

As a consequence of Item~\ref{item:1}, the lemma
\isatt{strong\_replacementI} is no longer valid and was commented
out.

We moved the requirement $M(\mathtt{nat})$ to the locale
\isatt{M\_trancl} (inside the theory \isatt{WF\_absolute}), where it is needed for the first time. Some results,
for instance \isatt{rtran\_closure\_mem\_iff} and 
\isatt{iterates\_imp\_wfrec\_replacement} had to be moved inside that
locale.

The proof, for instance, that the constructible universe $L$ satisfies
the modified locale \isatt{M\_trivial} holds with minor
modifications. Nevertheless, in order to have a neater presentation,
we have stripped off several sections concerning $L$ from the theories
\isatt{L\_axioms} and \isatt{Internalize}, and we merged them to form
the new file  \isatt{Internalizations}. 

\medskip
\fbox{>>>Poner lo que sigue en las conclusiones???}
\medskip

We believe that the
\isatt{ZF-Constructible} session  would benefit from some changes in
the way some results are organized accross theory files; for
intance, by cataloging all or most of the internalized formulas in one
file. Another, most basic example would be to start out with an even
more locale that only assumes $M$ to be a nonempty transitive class,
as many absoluteness results follow from these hypothesis.  
It is our desire to advocate a future work to a thorough
revision of the development of constructibility to maximize modularity.

%%% Local Variables: 
%%% mode: latex
%%% TeX-master: "Separation_In_MG"
%%% ispell-local-dictionary: "american"
%%% End: 


%%%%%%%%%%%%%%%%%%%%%%%%%%%%%%%%%%%%%%%%%%%%%%%%%%%%%%%%%%%%%%%%%%%%%%          
\section{Foundation, Union, Infinity}
\label{sec:easy-axioms}

It is straightforward to show that the generic extension $M[G]$
satisfies foundation, union, and infinity. The easiest one is
foundation which does not depend on the satisfaction by $M$.


\fbox{Hablar de \texttt{M\_trivial} y del uso de interfaz para instanciar}

\fbox{Falta \texttt{check\_in\_M} para completar Infinity}

%%% Local Variables: 
%%% mode: latex
%%% TeX-master: "Separation_In_MG"
%%% ispell-local-dictionary: "american"
%%% End: 


%%%%%%%%%%%%%%%%%%%%%%%%%%%%%%%%%%%%%%%%%%%%%%%%%%%%%%%%%%%%%%%%%%%%%%          
\section{Forcing}
\label{sec:forcing}
Let $\lb \PP, {\preceq} ,\1\rb \in M$ be a forcing notion. Given $G\sbq \PP$, we have
$M[G]\defi \{ \val(\PP,G,\punto{a}) : \punto{a}\in M \}$.

The following form of the Forcing Theorems  is the one
that we formalized.
\begin{theorem}
  There exists a function  $\forceisa:: \tyi \fun  \tyi$
  such that for every
  $\phi\in\formula$ and $\punto{a}_0,\dots,\punto{a}_n\in M$,
  \begin{enumerate}
  \item (Definability) $\forceisa(\phi)\in\formula$;
  \item (Truth Lemma) for every $M$-generic $G$,
    \[
      M[G], [\val(\PP,G,\punto{a}_0),\dots,\val(\PP,G,\punto{a}_n)]
      \models \phi
    \]
    is equivalent to 
    \[
      \exists p\in G.\ \; M, [p,\PP,\preceq,\1, \punto{a}_0,\dots,\punto{a}_n]  \models
      \forceisa(\phi).\]

    We use the notation $p \forces
    \phi\ [\punto{a}_0,\dots,\punto{a}_n]$ for this last assertion.
  \item (Density Lemma) $p \forces \phi\ [\punto{a}_0,\dots,\punto{a}_n]$
    if and only if 
    $\{q\in \PP :  q \forces \phi\ [\punto{a}_0,\dots,\punto{a}_n]\}$
    is dense below $p$.
  \end{enumerate}
\end{theorem}

We followed the new Kunen's book to define
$\forceisa$.  Forcing for atomic formulas is described as a mutual
recursion
%% \begin{multline*}
%%   \forceseq (p,t_1,t_2) \defi 
%%   \forall s\in\dom(t_1)\cup\dom(t_2).\ \forall q\pleq p .\\
%%   \forcesmem(q,s,t_1)\lsii \forcesmem(q,s,t_2)
%% \end{multline*}
%% \begin{multline*}
%%   \forcesmem(p,t_1,t_2) \defi  \forall v\pleq p. \ \exists q\pleq v.\\  
%%   \exists s.\ \exists r\in \PP .\ \lb s,r\rb \in  t_2 \land q
%%   \pleq r \land \forceseq(q,t_1,s)
%% \end{multline*}
but then \cite[p.~257]{kunen2011set} it is cast as a single
recursively defined function $\frcat$ over the wellfounded relation
$\isatt{frecR}$ on tuples $\lb \mathit{ft},t_1,t_2,p\rb$ (where
$\mathit{ft}\in\{0,1\}$ indicates the type of the atomic formula being
forced). Forcing for general formulas is defined by recursion on the
datatype $\formula$. Details on the implementation and proofs of the
Forcing Theorems have been spelled out in our
\cite{2020arXiv200109715G}.


It is to be noted that application of the Forcing theorems do not
require any extra Replacement instances on $M$.

%%% Local Variables: 
%%% mode: latex
%%% TeX-master: "independence_ch_isabelle"
%%% ispell-local-dictionary: "american"
%%% End: 


%%%%%%%%%%%%%%%%%%%%%%%%%%%%%%%%%%%%%%%%%%%%%%%%%%%%%%%%%%%%%%%%%%%%%%          
\section{Proof of Separation}

This proof can be found in the file \verb|Separation_Axiom.thy| of the
development, which we proceed to discuss.

The key technical result is the following:
\begin{isabelle}
  \isacommand{lemma}\isamarkupfalse%
  \ Collect{\isacharunderscore}sats{\isacharunderscore}in{\isacharunderscore}MG\ {\isacharcolon}\isanewline
  \ \ \isakeyword{assumes}\isanewline
  \ \ \ \ {\isachardoublequoteopen}{\isasympi}\ {\isasymin}\ M{\isachardoublequoteclose}\ {\isachardoublequoteopen}{\isasymsigma}\ {\isasymin}\ M{\isachardoublequoteclose}\ {\isachardoublequoteopen}val{\isacharparenleft}G{\isacharcomma}\ {\isasympi}{\isacharparenright}\ {\isacharequal}\ c{\isachardoublequoteclose}\ {\isachardoublequoteopen}val{\isacharparenleft}G{\isacharcomma}\ {\isasymsigma}{\isacharparenright}\ {\isacharequal}\ w{\isachardoublequoteclose}\isanewline
  \ \ \ \ {\isachardoublequoteopen}{\isasymphi}\ {\isasymin}\ formula{\isachardoublequoteclose}\ {\isachardoublequoteopen}arity{\isacharparenleft}{\isasymphi}{\isacharparenright}\ {\isasymle}\ {\isadigit{2}}{\isachardoublequoteclose}\isanewline
  \ \ \isakeyword{shows}\ \ \ \ \isanewline
  \ \ \ \ {\isachardoublequoteopen}{\isacharbraceleft}x{\isasymin}c{\isachardot}\ sats{\isacharparenleft}M{\isacharbrackleft}G{\isacharbrackright}{\isacharcomma}\ {\isasymphi}{\isacharcomma}\ {\isacharbrackleft}x{\isacharcomma}\ w{\isacharbrackright}{\isacharparenright}{\isacharbraceright}{\isasymin}\ M{\isacharbrackleft}G{\isacharbrackright}{\isachardoublequoteclose}
\end{isabelle}
%
From this, using absoluteness, we will be able to derive the
$\phi$-instance of Separation. 

To show that   
\[
S\defi\{x\in c : M[G]\models \phi(x,w)\} \in M[G],
\]
it is enough to provide a name $n\in M$ for this set.
 
The candidate name is
\[
n \defi \{u \in\dom(\pi)\times\PP :M,[u,\PP,\leq,\1,\sig,\pi]\models \psi\}
\]
where
\[
\psi \defi \exists \th\, p.\ x_0=\lb\th,p\rb \y 
   \forceisa(\th\in x_5\y\phi(\th,x_4)).
\]
The fact that $n\in M$ follows by an application of a six-variable
instance of Separation in $M$ (lemma \isatt{six{\isacharunderscore}sep{\isacharunderscore}aux}).

Almost a third part of the proof involves the syntactic handling of
internalized formulas and permutation of variables. The more
substantive portion concerns proving that actually $\val(G,n)=S$.

Let's first focus into the predicate 
\[
M,[u,\PP,\leq,\1,\sig,\pi]\models \psi
\]
defining $n$ by separation. By definition of the satisfaction
relation and permuting variables, we have it is equivalent to the fact
that there exist $\th,p\in M$ with   $u=\lb\th,p\rb$  and 
\[
M,[\PP,\leq,\1,p,\th,\sig,\pi]\models \forceisa(x_4\in
x_6\y\phi(x_4,x_5)). 
\]
% (Note that the variable $x_7$ is not used.)
This, in turn is equivalent by the Definition of Forcing to: For all $M$-generic
filters $F$ such that $p\in F$, 
\[
M,[\val(G,\th),\val(G,\sig),\val(G,\pi)]\models x_4\in
x_6\y\phi(x_4,x_5). 
\] 


%% The proof of
%% \isatt{Collect{\isacharunderscore}sats{\isacharunderscore}in{\isacharunderscore}MG}
%% has three parts:
%% \begin{enumerate}
%% \item Definition of the name $n$;
%% \item Proving that 
%% \item 
%% \end{enumerate}

%%% Local Variables: 
%%% mode: latex
%%% TeX-master: "Separation_In_MG"
%%% ispell-local-dictionary: "american"
%%% End: 


\section{Conclusion}

TODO:
\begin{enumerate}
\item Implement \emph{Basic Set Theory (BST)} by Kunen in
  Constructible: the use of alternatively Replacement or Powerset to
  prove basic absoluteness and closure resuls.
\item Enhance the automatization of formulas
\item Develop the forcing notions to obtain the independence of $\CH$,
  along with the prerrequisite combinatorial results (v.g.\ the
  $\Delta$-system lemma).
\end{enumerate}

%%% Local Variables: 
%%% mode: latex
%%% TeX-master: "forcing_in_isabelle_zf"
%%% ispell-local-dictionary: "american"
%%% End: 


%%%%%%%%%%%%%%%%%%%%%%%%%%%%%%%%%%%%%%%%%%%%%%%%%%%%%%%%%%%%%%%%%%%%%%%%%%%%%%%%

\bibliographystyle{mi-estilo-else}
\bibliography{../LSFA/citados}
%\documentclass{article}
\usepackage{isabelle,isabellesym}
\renewcommand{\ttdefault}{cmtt}
\usepackage{xcolor}
\usepackage{csquotes}
\usepackage{enumitem}

\newlist{inlinelist}{enumerate*}{1}
\setlist*[inlinelist,1]{%
  label=(\roman*),
}
\usepackage{hyperref}
\usepackage[numbers]{natbib}
\input{header-draft}
\makeatletter
\def\foottext{\gdef\@thefnmark{}\@footnotetext}
\makeatother
\newcommand{\keywords}[1]{\foottext{\emph{Keywords:} #1}}
\newcommand{\ack}[1]{\par\bigskip \noindent \emph{Acknowledgment:} #1}

\hypersetup{
  pdftitle={Separation in Generic Extensions for Isabelle},
  pdfsubject={Computer Science},
  pdfkeywords={Isabelle/ZF, forcing, names, generic extension, constructibility},
  colorlinks,
  linkcolor={blue!40!black},
  citecolor={blue!40!black},
  urlcolor={blue!40!black}
}

\begin{document}
\title{Separation in Generic Extensions for Isabelle}
\author{Emmanuel Gunther
  \and 
  Miguel Pagano
  \and 
  Pedro S\'anchez Terraf}
\maketitle

\begin{abstract}
  We mechanize, in the proof assistant
  Isabelle, a proof of the
  axiom-scheme of Separation in 
  generic extensions of models of set theory  
  by using the fundamental theorems of forcing.
  We also formalize the satisfaction of the axioms of
  Extensionality, Foundation, Union, and Powerset. The axiom of
  Infinity is likewise treated, under additional assumptions on the ground
  model.
  We also  extend Paulson's library on constructibility  with
  renaming of variables for internalized formulas, an improvement on
  definitions by recursion on well-founded  relations and sharpening
  of the hypotheses in his development of relativization and
  absoluteness.
\end{abstract}
\keywords{
Isabelle/ZF, forcing, names, generic extension, constructibility.
}

%%%%%%%%%%%%%%%%%%%%%%%%%%%%%%%%%%%%%%%%%%%%%%%%%%%%%%%%%%%%%%%%%%%%%%%%%%%%%%%%
\input{intro_m}

\input{isabelle}

\input{zfc-axioms}

\input{renaming}

\input{generic-extensions}

\input{recursion}

% \input{prior-results}

\input{hacking}

\input{easy-axioms}

\input{forcing}

\input{proof-of-separation}

\input{conclusion}

%%%%%%%%%%%%%%%%%%%%%%%%%%%%%%%%%%%%%%%%%%%%%%%%%%%%%%%%%%%%%%%%%%%%%%%%%%%%%%%%

\bibliographystyle{mi-estilo-else}
\bibliography{../LSFA/citados}
%\input{separation_arxiv.bbl}
\end{document}

%%% Local Variables:
%%% mode: latex
%%% ispell-local-dictionary: "american"
%%% End:

\end{document}

%%% Local Variables:
%%% mode: latex
%%% ispell-local-dictionary: "american"
%%% End:

\end{document}

%%% Local Variables:
%%% mode: latex
%%% ispell-local-dictionary: "american"
%%% End:

\end{document}

%%% Local Variables:
%%% mode: latex
%%% ispell-local-dictionary: "american"
%%% End:
