% This is samplepaper.tex, a sample chapter demonstrating the
% LLNCS macro package for Springer Computer Science proceedings;
% Version 2.20 of 2017/10/04
%
\documentclass[runningheads]{llncs}
%
\usepackage[utf8]{inputenc}
\usepackage{isabelle_indepCH,isabellesym_indepCH}
\usepackage{amsmath,amsfonts,amssymb}
\usepackage{bbm}  % Para el \bb{1}
\usepackage{tikz}
\usepackage[english]{babel}
\usepackage{multidef}
\usepackage{verbatim}
\usepackage{stmaryrd} %% para \llbracket
\usepackage{hyperref}
\usepackage{xcolor}
\usepackage{framed}
\usepackage[numbers]{natbib}

%%
%% \usepackage[bottom=2cm, top=2cm, left=2cm, right=2cm]{geometry}
%% \usepackage{titling}
%% \setlength{\droptitle}{-10ex} 
%%
\renewcommand{\o}{\vee}
\renewcommand{\O}{\bigvee}
\newcommand{\y}{\wedge}
\newcommand{\Y}{\bigwedge}
\newcommand{\limp}{\longrightarrow}
\newcommand{\lsii}{\longleftrightarrow}
%%
%\newcommand{\DeclareMathOperator}[2]{\newcommand{#1}{\mathop{\mathrm{#2}}}}

\DeclareMathOperator{\cf}{cf}
\DeclareMathOperator{\dom}{domain}
\DeclareMathOperator{\im}{img}
\DeclareMathOperator{\Fn}{Fn}
\DeclareMathOperator{\rk}{rk}
\DeclareMathOperator{\mos}{mos}
\DeclareMathOperator{\trcl}{trcl}
\DeclareMathOperator{\Con}{Con}
\DeclareMathOperator{\Club}{Club}


\newcommand{\modelo}[1]{\mathbf{#1}}
\newcommand{\axiomas}[1]{\mathit{#1}}
\newcommand{\clase}[1]{\mathsf{#1}}
\newcommand{\poset}[1]{\mathbb{#1}}
\newcommand{\operador}[1]{\mathbf{#1}}

%% \newcommand{\Lim}{\clase{Lim}}
%% \newcommand{\Reg}{\clase{Reg}}
%% \newcommand{\Card}{\clase{Card}}
%% \newcommand{\On}{\clase{On}}
%% \newcommand{\WF}{\clase{WF}}
%% \newcommand{\HF}{\clase{HF}}
%% \newcommand{\HC}{\clase{HC}}
%%
%% El siguiente comando reemplaza todos los anteriores:
%%
\multidef{\clase{#1}}{Card,HC,HF,Lim,On->Ord,Reg,WF,Ord}
\newcommand{\ON}{\On}

%% En lugar de usar todo el paquete bbm:
\DeclareMathAlphabet{\mathbbm}{U}{bbm}{m}{n} 
\newcommand{\1}{\mathbbm{1}}
\newcommand{\PP}{\mathbbm{P}}

%%
%% \newcommand{\calD}{\mathcal{D}}
%% \newcommand{\calS}{\mathcal{S}}
%% \newcommand{\calU}{\mathcal{U}}
%% \newcommand{\calB}{\mathcal{B}}
%% \newcommand{\calL}{\mathcal{L}}
%% \newcommand{\calF}{\mathcal{F}}
%% \newcommand{\calT}{\mathcal{T}}
%% \newcommand{\calW}{\mathcal{W}}
%% \newcommand{\calA}{\mathcal{A}}
%%
%% El siguiente comando reemplaza todos los anteriores:
%%
\multidef[prefix=cal]{\mathcal{#1}}{A-Z}
%%
%% \newcommand{\A}{\modelo{A}}
%% \newcommand{\BB}{\modelo{B}}
%% \newcommand{\ZZ}{\modelo{Z}}
%% \newcommand{\PP}{\modelo{P}}
%% \newcommand{\QQ}{\modelo{Q}}
%% \newcommand{\RR}{\modelo{R}}
%%
%% El siguiente comando reemplaza todos los anteriores:
%%
\multidef{\modelo{#1}}{A,BB->B,CC->C,NN->N,QQ->Q,RR->R,ZZ->Z}

\multidef[prefix=p]{\mathbb{#1}}{A-Z}
%% \newcommand{\B}{\modelo{B}}
%% \newcommand{\C}{\modelo{C}}
%% \newcommand{\F}{\modelo{F}}
%% \newcommand{\D}{\modelo{D}}

\newcommand{\Th}{\mb{Th}}
\newcommand{\Mod}{\mb{Mod}}

\newcommand{\Se}{\operador{S^\prec}}
\newcommand{\Pu}{\operador{P_u}}
\renewcommand{\Pr}{\operador{P_R}}
\renewcommand{\H}{\operador{H}}
\renewcommand{\S}{\operador{S}}
\newcommand{\I}{\operador{I}}
\newcommand{\E}{\operador{E}}

\newcommand{\se}{\preccurlyeq}
\newcommand{\ee}{\succ}
\newcommand{\id}{\approx}
\newcommand{\subm}{\subseteq}
\newcommand{\ext}{\supseteq}
\newcommand{\iso}{\cong}
%%
\renewcommand{\emptyset}{\varnothing}
\newcommand{\rel}{\mathcal{R}}
\newcommand{\Pow}{\mathop{\mathcal{P}}}
\renewcommand{\P}{\Pow}
\newcommand{\BP}{\mathrm{BP}}
\newcommand{\func}{\rightarrow}
\newcommand{\ord}{\mathrm{Ord}}
\newcommand{\R}{\mathbb{R}}
\newcommand{\N}{\mathbb{N}}
\newcommand{\Z}{\mathbb{Z}}
\renewcommand{\I}{\mathbb{I}}
\newcommand{\Q}{\mathbb{Q}}
\newcommand{\B}{\mathbf{B}}
\newcommand{\lb}{\langle}
\newcommand{\rb}{\rangle}
\newcommand{\impl}{\rightarrow}
\newcommand{\ent}{\Rightarrow}
\newcommand{\tne}{\Leftarrow}
\newcommand{\sii}{\Leftrightarrow}
\renewcommand{\phi}{\varphi}
\newcommand{\phis}{{\varphi^*}}
\renewcommand{\th}{\theta}
\newcommand{\Lda}{\Lambda}
\newcommand{\La}{\Lambda}
\newcommand{\lda}{\lambda}
\newcommand{\ka}{\kappa}
\newcommand{\del}{\delta}
\newcommand{\de}{\delta}
\newcommand{\ze}{\zeta}
%\newcommand{\ }{\ }
\newcommand{\la}{\lambda}
\newcommand{\al}{\alpha}
\newcommand{\be}{\beta}
\newcommand{\ga}{\gamma}
\newcommand{\Ga}{\Gamma}
\newcommand{\ep}{\varepsilon}
\newcommand{\De}{\Delta}
\newcommand{\defi}{\mathrel{\mathop:}=}
\newcommand{\forces}{\Vdash}
%\newcommand{\ap}{\mathbin{\wideparen{\ }}}
\newcommand{\Tree}{{\mathrm{Tr}_\N}}
\newcommand{\PTree}{{\mathrm{PTr}_\N}}
\newcommand{\NWO}{\mathit{NWO}}
\newcommand{\Suc}{{\N^{<\N}}}%
\newcommand{\init}{\mathsf{i}}
\newcommand{\ap}{\mathord{^\smallfrown}}
\newcommand{\Cantor}{\mathcal{C}}
%\newcommand{\C}{\Cantor}
\newcommand{\Baire}{\mathcal{N}}
\newcommand{\sig}{\ensuremath{\sigma}}
\newcommand{\fsig}{\ensuremath{F_\sigma}}
\newcommand{\gdel}{\ensuremath{G_\delta}}
\newcommand{\Sig}{\ensuremath{\boldsymbol{\Sigma}}}
\newcommand{\bPi}{\ensuremath{\boldsymbol{\Pi}}}
\newcommand{\Del}{\ensuremath{\boldsymbol\Delta}}
%\renewcommand{\F}{\operador{F}}
\newcommand{\ths}{{\theta^*}}
\newcommand{\om}{\ensuremath{\omega}}
%\renewcommand{\c}{\complement}
\newcommand{\comp}{\mathsf{c}}
\newcommand{\co}[1]{\left(#1\right)^\comp}
\newcommand{\len}[1]{\left|#1\right|}
\DeclareMathOperator{\tlim}{\overline{\mathrm{TLim}}}
\newcommand{\card}[1]{{\left|#1\right|}}
\newcommand{\bigcard}[1]{{\bigl|#1\bigr|}}
%
% Cardinality
%
\newcommand{\lec}{\leqslant_c}
\newcommand{\gec}{\geqslant_c}
\newcommand{\lc}{<_c}
\newcommand{\gc}{>_c}
\newcommand{\eqc}{=_c}
\newcommand{\biy}{\approx}
\newcommand*{\ale}[1]{\aleph_{#1}}
%
\newcommand{\Zerm}{\axiomas{Z}}
\newcommand{\ZC}{\axiomas{ZC}}
\newcommand{\AC}{\axiomas{AC}}
\newcommand{\DC}{\axiomas{DC}}
\newcommand{\MA}{\axiomas{MA}}
\newcommand{\CH}{\axiomas{CH}}
\newcommand{\ZFC}{\axiomas{ZFC}}
\newcommand{\ZF}{\axiomas{ZF}}
\newcommand{\Inf}{\axiomas{Inf}}
%
% Cardinal characteristics
%
\newcommand{\cont}{\mathfrak{c}}
\newcommand{\spl}{\mathfrak{s}}
\newcommand{\bound}{\mathfrak{b}}
\newcommand{\mad}{\mathfrak{a}}
\newcommand{\tower}{\mathfrak{t}}
%
\renewcommand{\hom}[2]{{}^{#1}\hskip-0.116ex{#2}}
\newcommand{\pred}[1][{}]{\mathop{\mathrm{pred}_{#1}}}
%% Postfix operator with supressable space:
%% \newcommand*{\iseg}{\relax\ifnum\lastnodetype>0 \mskip\medmuskip\fi{\downarrow}} %
\newcommand*{\iseg}{{\downarrow}}
\newcommand{\rr}{\mathrel{R}}
\newcommand{\restr}{\upharpoonright}
%\newcommand{\type}{\mathtt{}}
\newcommand{\app}{\mathop{\mathrm{Aprox}}}
\newcommand{\hess}{\triangleleft}
\newcommand{\bx}{\bar{x}}
\newcommand{\by}{\bar{y}}
\newcommand{\bz}{\bar{z}}
\newcommand{\union}{\mathop{\textstyle\bigcup}}
\newcommand{\sm}{\setminus}
\newcommand{\sbq}{\subseteq}
\newcommand{\nsbq}{\subseteq}
\newcommand{\mty}{\emptyset}
\newcommand{\dimg}{\text{\textup{``}}} % direct image
\newcommand{\quine}[1]{\ulcorner{\!#1\!}\urcorner}
%\newcommand{\ntrm}[1]{\textsl{\textbf{#1}}}
\newcommand{\Null}{\calN\!\mathit{ull}}
\DeclareMathOperator{\club}{Club}
\DeclareMathOperator{\otp}{otp}
\DeclareMathOperator{\val}{\mathit{val}}
\DeclareMathOperator{\chk}{\mathit{check}}
\DeclareMathOperator{\edrel}{\mathit{edrel}}
\DeclareMathOperator{\eclose}{\mathit{eclose}}
\DeclareMathOperator{\Memrel}{\mathit{Memrel}}
\renewcommand{\PP}{\mathbb{P}}
\renewcommand{\app}{\mathrm{App}}
\newcommand{\formula}{\isatt{formula}}
\newcommand{\tyi}{\isatt{i}}
\newcommand{\tyo}{\isatt{o}}
\newcommand{\forceisa}{\mathop{\mathtt{forces}}}
\newcommand{\equ}{\mathbf{e}}
\newcommand{\bel}{\mathbf{b}}
\newcommand{\atr}{\mathit{atr}}
\newcommand{\concat}{\mathbin{@}}
\newcommand{\dB}[1]{\mathbf{#1}}
\newcommand{\ed}{\mathrel{\isatt{ed}}}
\newcommand{\frecR}{\mathrel{\isatt{frecR}}}
\newcommand{\forceseq}{\mathop{\isatt{forces{\isacharunderscore}eq}}}
\newcommand{\forcesmem}{\mathop{\isatt{forces{\isacharunderscore}mem}}}
\newcommand{\forcesat}{\mathop{\isatt{forces{\isacharunderscore}at}}}
\newcommand{\pleq}{\preceq}
%\renewcommand{\isacharunderscorekeyword}{\mbox{\_}}
%\renewcommand{\isacharunderscore}{\mbox{\_}}
\renewcommand{\isasymtturnstile}{\isamath{\Vdash}}
\renewcommand{\isacharminus}{-}
\newcommand{\uscore}{\isacharunderscore}
\newcommand{\session}[1]{\textit{#1}}
\newcommand{\theory}[1]{\texttt{#1}}
\newcommand{\punto}[1]{\overset{\tikz\draw[fill=black] (0,0) circle (0.6pt);}{#1}}

%%%%%%%%%%%%%%%%%%%%%%%%%
% Variant aleph, beth, etc
% From http://tex.stackexchange.com/q/170476/69595
\makeatletter
\@ifpackageloaded{txfonts}\@tempswafalse\@tempswatrue
\if@tempswa
  \DeclareFontFamily{U}{txsymbols}{}
  \DeclareFontFamily{U}{txAMSb}{}
  \DeclareSymbolFont{txsymbols}{OMS}{txsy}{m}{n}
  \SetSymbolFont{txsymbols}{bold}{OMS}{txsy}{bx}{n}
  \DeclareFontSubstitution{OMS}{txsy}{m}{n}
  \DeclareSymbolFont{txAMSb}{U}{txsyb}{m}{n}
  \SetSymbolFont{txAMSb}{bold}{U}{txsyb}{bx}{n}
  \DeclareFontSubstitution{U}{txsyb}{m}{n}
  \DeclareMathSymbol{\aleph}{\mathord}{txsymbols}{64}
  \DeclareMathSymbol{\beth}{\mathord}{txAMSb}{105}
  \DeclareMathSymbol{\gimel}{\mathord}{txAMSb}{106}
  \DeclareMathSymbol{\daleth}{\mathord}{txAMSb}{107}
\fi
\makeatother

%%%%%%%%%%%%%%%%%%%%%%%%%%%%%%%%%%%%%%%%%%%%%%%%%%%%%%%%%%%%
%%
%% Theorem Environments
%%
% \newtheorem{theorem}{Theorem}
% \newtheorem{lemma}[theorem]{Lemma}
% \newtheorem{prop}[theorem]{Proposition}
% \newtheorem{corollary}[theorem]{Corollary}
% \newtheorem{claim}{Claim}
% \newtheorem*{claim*}{Claim}
% \theoremstyle{definition}
% \newtheorem{definition}[theorem]{Definition}
% \newtheorem{remark}[theorem]{Remark}
% \newtheorem{example}[theorem]{Example}
% \theoremstyle{remark}
% \newtheorem*{remark*}{Remark}

%%%%%%%%%%%%%%%%%%%%%%%%%%%%%%%%%%%%%%%%%%%%%%%%%%%%%%%%%%%%%%%%%%%%%%

%% \newenvironment{inducc}{\begin{list}{}{\itemindent=2.5em \labelwidth=4em}}{\end{list}}
%% \newcommand{\caso}[1]{\item[\fbox{#1}]}
\newenvironment{proofofclaim}{\begin{proof}[Proof of Claim]}{\end{proof}}

\newcommand{\quantRel}[3]{#1 #2\kern -1pt[#3]}
\newcommand{\forallRel}[2]{\quantRel{\forall}{#1}{#2}}
\newcommand{\existsRel}[2]{\quantRel{\exists}{#1}{#2}}

\newif\ifarXiv
\newif\ifIEEE

%%% Local Variables: 
%%% mode: latex
%%% TeX-master: "independence_ch_isabelle"
%%% End: 

\usepackage{graphicx}
% Used for displaying a sample figure. If possible, figure files should
% be included in EPS format.
%
\hypersetup{
  colorlinks,
  urlcolor={blue},
  linkcolor={blue!50!black},
  citecolor={blue!50!black},
}
% If you use the hyperref package, please uncomment the following line
% to display URLs in blue roman font according to Springer's eBook style:
% \renewcommand\UrlFont{\color{blue}\rmfamily}

\begin{document}
%
\title{The formal verification of the ctm approach to forcing%Some lessons after the formalization of the ctm approach to forcing%
  \thanks{Supported by Secyt-UNC project 33620180100465CB and Conicet.}%
}
%
\titlerunning{Formalizing of ctm forcing}%Lessons after formalizing ctm forcing}
% If the paper title is too long for the running head, you can set
% an abbreviated paper title here
%
\author{Emmanuel Gunther\inst{1} \and
Miguel Pagano\inst{1} \and \\
Pedro Sánchez Terraf\inst{1,2}%\orcidID{0000-0003-3928-6942}
\and
Matías Steinberg\inst{1}
}
%
\authorrunning{E.~Gunther, M.~Pagano, P.~Sánchez Terraf, M.~Steinberg}
% First names are abbreviated in the running head.
% If there are more than two authors, 'et al.' is used.
%
\institute{Universidad Nacional de C\'ordoba. 
  \\  Facultad de Matem\'atica, Astronom\'{\i}a,  F\'{\i}sica y
  Computaci\'on.
  \and
    Centro de Investigaci\'on y Estudios de Matem\'atica (CIEM-FaMAF),
    Conicet. C\'ordoba. Argentina. \\
    \email{\{gunther,pagano,sterraf\}@famaf.unc.edu.ar\\
        matias.steinberg@mi.unc.edu.ar}
}
%
\maketitle              % typeset the header of the contribution
%
\begin{abstract}
  We'll discuss some highlights of our computer-verified
  proof of the construction, given a countable transitive set model $M$
  of $\ZFC$, of generic extensions  satisfying $\ZFC+\neg\CH$ and $\ZFC+\CH$.
  In particular,
  we isolated a set $\Delta$ of 39 instances
  of Axiom of Replacement and a function $F$
  such that such that for any finite fragment $\Phi\sbq\ZFC$,
  $F(\Phi)\sbq\ZFC$ is also finite and if
  $M\models F(\Phi) + \Delta$ then $M[G]\models \Phi + \neg \CH$.
  We also obtained the formulas yielded by the Forcing Definability Theorem
  explicitly.

  To achieve this, we worked in the proof assistant \emph{Isabelle},
  basing our development on the theory Isabelle/ZF by L.~Paulson and
  others.

  %% The vantage point of the talk will be that of a mathematician but
  %% elements from the computer science perspective will be
  %% present. Perhaps some myths regarding what can effectively be done
  %% using  proof assistants/checkers will
  %% be dispelled.

  We'll also compare our formalization with the recent one by Jesse
  M.~Han and Floris van Doorn in the proof assistant \emph{Lean}.

  \keywords{forcing \and Isabelle/ZF \and countable transitive models
    \and absoluteness \and generic extension \and constructibility.}
\end{abstract}
%
%
%
\section{Introduction}
\label{sec:introduction}

This paper is the culmination of our project on the computerized
formalization of the undecidability of the Continuum Hypothesis
($\CH$) from Zermelo-Fraenkel set theory with Choice ($\ZFC$), under the
assumption of the existence of a countable transitive model (ctm) of
$\ZFC$. In contrast to our reports of the previous steps towards this
goal
\cite{2018arXiv180705174G,2019arXiv190103313G,2020arXiv200109715G}, we
intend here to present our development to the mathematical logic
community. For this reason, we start with a general discussion around
the formalization of mathematics.

\subsection{Formalized mathematics}
The use of computers to assist the creation and verification of
mathematics has seen a steady grow. But the general awareness on the
matter still seems to be a bit scant (even among mathematicians
involved in foundations), and the venues devoted to the communication
of formalized mathematics are, mainly, computer science journals and
conferences: JAR, ITP, IJCAR, CPP, CICM, and others.

Nevertheless, the discussion about the subject in central mathematical
circles is increasing; there were some hints on the ICM2018 panel on
“machine-assisted” proofs
\cite{https://doi.org/10.48550/arxiv.1809.08062} and a lively
promotion by Kevin Buzzard, during his ICM2022 special plenary lecture
\cite{2021arXiv211211598B}.

%% These assistants provide several dialects, among which we single out:
%% \begin{enumerate}
%% \item Procedural: Useful for exploration/research.
%% \item Declarative: Only one that can be read by humans!
%% \end{enumerate}

Before we start an in-depth discussion, a point should be made clear:
A formalized proof is not the same as an \emph{automatic proof}. The
reader surely understands that, aside from results of a very specific sort, no current
technology allows us to write a reasonably complex (and correct)
theorem statement in a computer and expect to obtain a proof after hitting “Enter”, at
least not after a humanly feasible wait. On the other hand, it is
quite possible that the same reader has some mental image that
formalizing a proof requires making each application of Modus Ponens
explicit.

The fact is that \emph{proof assistants} are designed for the human prover to
be able to decompose a statement to be proved into smaller subgoals
which can actually be fed into some automatic tool. The balance between
what these tools are able to handle is not  easily appreciated by
intuition: Sometimes, ``trivial'' steps are not solved by them, which
can result in obvious frustration; but they would quickly solve some
goals that do not look like a ``mere computation.''

To appreciate the extent of mathematics formalizable, it is convenient to recall
some major successful projects, such as the Four Color Theorem
\cite{MR2463991}, the Odd Order Theorem
\cite{10.1007/978-3-642-39634-2_14}, and the proof the Kepler's
Conjecture \cite{MR3659768}. There is a vast mathematical corpus at
the Archive of Formal Proofs (AFP) based on Isabelle; and formalizations of
brand new mathematics like the Liquid Tensor Experiment
\cite{LTE2020,LTE2021} and the definition of perfectoid spaces \cite{10.1145/3372885.3373830}
have been achieved using Lean.

We will continue our description of proof assistants in
Section~\ref{sec:proof-assist-isabelle}. We kindly invite the reader
to enrich the previous exposition by reading the apt summary by
A.~Koutsoukou-Argyraki \cite{angeliki} and the interviews
therein; some of the experts consulted have also discussed
in \cite{2022arXiv220704779B} the status of formalized versus standard
proof in mathematics.

\subsection{Our achievements}
We formalized a model-theoretic rendition of forcing (Sect.~\ref{sec:forcing}), showing how to
construct proper extensions of ctms of $\ZF$ (respectively, with
$\AC$), and we formalized the basic forcing notions required to obtain
ctms of $\ZFC + \neg\CH$ and of $\ZFC + \CH$ (Sect.~\ref{sec:models-ch-negation}). No metatheoretic issues
(consistency, FOL calculi, etc) were formalized, so we were mainly
concerned with the mathematics of forcing. Nevertheless, by inspecting
the foundations underlying our proof assistant Isabelle
(Section~\ref{sec:isabelle-metalogic-meta}) it can be stated that our
formalization is a bona fide proof in $\ZF$ of the previous
constructions.

In order to reach our goals, we provided basic results that were
missing from Isabelle's $\ZF$ library, starting from ones
involving cardinal successors, countable sets, etc.
(Section~\ref{sec:extension-isabellezf}). We also extended the treatment of relativization of
set-theoretical concepts (Section~\ref{sec:tools-relativization}).
%% We redesigned Isabelle/ZF results on non-absolute concepts to work
%% relative to a class.

One added value that is obtained from the present formalization is
that we identified a handful of instances of Replacement which are
sufficient to set the forcing machinery up
(Section~\ref{sec:repl-instances}), on the basis of Zermelo set theory.
The eagerness to obtain this level of detail might be a consequence of
“an unnatural tendency to investigate, for the most part, trivial
minutiae of the formalism” on our part, as it was put by Cohen
\cite{zbMATH02012060}, but we would rather say that we were driven by
curiosity.

The code of our formalization can be accessed at the
AFP site, via the following link:
\begin{center}
  \url{https://www.isa-afp.org/entries/Independence_CH.html}
\end{center}

%%% Local Variables: 
%%% mode: latex
%%% TeX-master: "independence_ch_isabelle"
%%% ispell-local-dictionary: "american"
%%% End: 

 
\section{Proof assistants and Isabelle/ZF}
\label{sec:proof-assist-isabelle}

Let us briefly introduce Isabelle in the large landscape of proof
assistans (or ITPs for interactive theorem provers); we refer to the
excelent chapter by \citet{DBLP:series/hhl/HarrisonUW14} for a more
thorough recollection of the history of ITPs.

It is expected that an ITP assists the human user while mechanizing
some piece of mathematics; the interaction varies from system to
system, but a common interface consists on the system showing the
current goal and context. The user instructs the ITP to modify the
goal/context by means of tactics; a forward rule changes the context
with the intention of having in the new context a hypothesis closer to
the goal. A backward rule represents a deduction whose conclusion is
the goal, thus the new goal corresponds to the premises of the rule. A
proof is completed when the (current) goal is an instance of the
hypothesis (in the current context).

In that dialog, the user produces a script of tactics that can be
later reproduced step-by-step by the system (to check, for example,
that an imported theory is correct) or by the user to understand
the proof.\footnote{As we will see there are more declaratives
proof languages that aim to have intelligible proof scripts.}

To have any value at all, the system should be able to say if a tactic
makes sense (for instance, it must be forbidden to use the backward
tactic \textit{conjI} when the main connective is a
disjunction). Moreover one should be able to ascertain the validity of
the reaction of an ITP (be its rejection or its acceptance of some
alleged logical step). % Here I would merge the next paragraph.

Proof assistants provide diverse aids for the task
of formalizing a piece of mathematics. They are usually implemented
using a typed programming language; rigor is enforced by defining a
type of ``theorems,'' whose members can only be constructed using
operations stipulated in a small \emph{kernel} which encodes the
underlying foundation of the assistant. Further developments interact
with the type of theorems only through the kernel, and thus the latter
is the only ``trusted'' part of the assistant's code.

Several of the more established assistants (HOL Light, Coq, Isabelle)
are programmed in some variant of the ML language (which was
designed for this purpose); the newer Lean, on the other
hand, was originally conceived as a standalone functional programming
language with all the features of a standard assistant.

In the case of Isabelle, Standard ML is the first of the four layers
in which we work in this assistant. Both the kernel and automation
of proofs is coded in ML, sometimes as a substitute for induction on
formulas, as the next section explains.

\subsection{Isabelle metalogic $\Meta$}
\label{sec:isabelle-metalogic-meta}

The second layer of Isabelle is an
intuitionistic fragment of higher-order logic (or simple type theory)
called $\Meta$; its original version was described in \cite{Paulson1989},
and the addition of “sorts” appears in \cite{Nipkow-LF-91}.

The only predefined type is $\prop$ (“propositions”); new base types
can be postulated when defining objects logics. Types of higher order can be
assembled using the function space constructor $\fun$.

The type of propositions $\prop$ is equipped with a binary operation
$\implies$ (“meta-implication”) and a universal “meta-quantifier”
$\ALL$, that are used to represent the object
logic rules. As an example, the axiomatization of first-order logic
postulates a type $\tyo$ of booleans, and Modus Ponens
% https://isabelle.in.tum.de/dist/library/FOL/FOL/IFOL.html#IFOL.mp|axiom
is written as
\[
  \ALL P\,Q.\ \ [P\limp Q] \implies ([P] \implies [Q]).
\]
The square brackets (which are omitted in Isabelle theories) represent
an injection from $\tyo$ into $\prop$. % ($[P]$ can be read as “$P$
% holds”)
A consequence of this representation is that every formula of
the object logic appears atomic to $\Meta$.

%% Quantification is handled in the meta-level using a functional $\ALL$
%% with polymorphic type $(\alpha \fun \prop) \fun \prop$. 
Types in Isabelle are organized into \emph{classes} and \emph{sorts};
for ease of exposition, we will omit the former.  The axiomatization
of first-order logic postulates a sort $\{\type{term}\}$ (of
“individuals,” or elements of a first-order universe of discourse) and
stipulates that every further type variable $\alpha$ must be of that
sort. In particular, Isabelle/ZF only postulates one new type $\tyi$
(“sets”) of sort $\{\type{term}\}$. Hence, from the type of the universal
quantifier functional $\forall :: (\alpha \fun \tyo) \fun \tyo$, it
follows that it may only be applied to predicates with a variable of
type $\tyi$. This ensures that the object logic is effectively
first-order.

Paulson  \cite{Paulson1989} carried out a proof that the encoding
$\Meta_{\mathrm{IFOL}}$ of
intuitionistic first-order logic IFOL without equality  in the original $\Meta$ is
conservative (there is a correspondence between provable $\phi$ in
IFOL and provable $[\phi]$ in $\Meta_{\mathrm{IFOL}}$) by putting
$\Meta_{\mathrm{IFOL}}$ proofs in \emph{expanded normal form}
\cite{MR0387024}. Passing to classical logic does not present
difficulties, but the addition of meta-equality must be taken care of.
Even more so, since the treatment of equality differs between the
original and the present incarnation of $\Meta$; details for the
latter are exhaustively expounded in the recent formalization by
Nipkow and Roßkof \cite{10.1007/978-3-030-79876-5_6}.

The meta-logic $\Meta$ is rather weak; it has no induction/recursion
principles. Types are not inductively presented and, in particular, it
is not possible to prove by induction statements about
object-logic formulas (which are construed as terms of type $\tyi \fun
\dots \fun \tyi \fun \tyo$). Two ways to overcome this limitation are:
\begin{enumerate}
\item
  to construct the
  proof of each instance of the statement by hand or by programming on
  ML; or 
\item
  to encode formulas as sets and prove an internal version statement
  using induction of $\ZF$.
\end{enumerate}

For recursive definitions, only the second option is available, and
that is the way the Definition of Forcing is implemented in our
formalization.

\subsection{Isabelle/ZF}
\label{sec:isabellezf}

For the most part, the development of set theory in Isabelle is
carried out using its ZF object logic
\cite{DBLP:journals/jar/PaulsonG96}, which is the third logical layer
of the formalization and the most versatile one, since 
Isabelle's native automation is available at this level. Apart from
the type and sort
declarations detailed above, it features a finite axiomatization,
% https://isabelle.in.tum.de/dist/library/FOL/FOL/IFOL.html#ZF_Base.mem|const
with a predicate for membership, constants for the empty set and an
infinite set, and functions $\isatt{Pow}::\tyi\fun\tyi$,
$\union::\tyi\fun\tyi$, and $\isatt{PrimReplace} :: \tyi \fun (\tyi
\fun \tyi \fun\tyo)\fun \tyi$ (for Replacement). The Axiom of
Replacement
% https://isabelle.in.tum.de/dist/library/FOL/FOL/IFOL.html#ZF_Base.replacement|axiom
has a free predicate variable $P$: % $P ::\tyi\fun\tyi\fun\tyo$:
\begin{multline*}
  (\forall x \in A .\ \forall y\, z.\ P(x, y) \wedge P(x, z)
  \longrightarrow y=z) \implies \\
  b \in \isatt{PrimReplace}(A, P)
  \longleftrightarrow(\exists x \in A .\ P(x, b)) 
\end{multline*}
The restrictions on sorts described above ensure that it is not
possible that higher-order quantification gets used in $P$. The
statement of $\AC$ also has a higher-order free variable, but we only
use $\AC$ for demonstration purposes.

Isabelle/ZF reaches Hessenberg's $|A|\cdot|A| = |A|$. Our decision
(during 2017) to
use this assistant was triggered by its constructibility
library, \session{ZF-Constructible} \citep{paulson_2003},
% https://isabelle.in.tum.de/dist/library/ZF/ZF-Constructible/
which contains the development of $L$, the proof that it satisfies
$\AC$, and a version of the Reflection Principle. The latter was
actually encoded as a series of instructions to Isabelle automatic
proof tools that would prove each particular instance of reflection:
This is an example of what was said at the end of Section~\ref{sec:isabelle-metalogic-meta}.

The development of relativization and absoluteness for classes $C::
\isatt{i} \fun \isatt{o}$ follows the same pattern. Each particular
concept was manually written in a relational form and relativized.
Here, the contrast between the usual way one regards $\ZF$ as
first-order theory in the language $\{\in \}$ and the mathematical
practice of freely using defined concepts comes to the
forefront. Assistants have refined mechanisms to cope with defined
concepts (which also make their foundations a more complicated
than plain first-order logic), and this is the only that  nontrivial
mathematics can be formalized. But when studying relative interpretations, one
usually assumes a spartan syntax and defines relativization by
induction of formulas of the more succinct language. The approach
taken in \session{ZF-Constructible} is to consider relativizations of
the formulas (of type $\tyo$) that define each concept. For instance,
in the case of unions, we find a relativization
$\isatt{big{\uscore}union}:: (\isatt{i} \fun \isatt{o}) \fun \isatt{i}
\fun \isatt{i} \fun \isatt{o}$ of the statement
“$\union A = z$”:
\[
 \isatt{big{\uscore}union}(M,A,Z) \equiv \forall x[M].\ x \in z
 \longleftrightarrow (\exists y[M].\ y \in A \land x \in y)
\]
where $\forall x[M]\dots$ stands for $\forall x.\ M(x)\limp \dots$,
etc. The need to work with \emph{relational} presentations of defined
concepts stems from the fact that the model-theoretic definition of
$L$ requires working with set models and satisfaction, which is
defined for codes of formulas.

\subsection{Internalized formulas}
\label{sec:internalized-formulas}

\session{ZF-Constructible} defines the set $\formula$ of codes for
first-order formulas. These, alongside with lists, are instances of
Isabelle/ZF treatment of inductively defined (internal) datatypes; induction
and recursion theorems for them are proved automatically (this is in
constrast to general well-founded recursion, for which one has to work
with the fundamental recursor $\isatt{wfrec}$).

To avoid problems related to the binding of variables, de Bruijn
indices \cite{MR0321704} are used instead. The satisfaction predicate
$\isatt{sats}::\tyi\fun\tyi\fun\tyi\fun\tyo$ takes as arguments a set $M$, a list
$\mathit{env}\in\isatt{list}(M)$ for the assignment of free indices
(which are counted by the \isatt{arity} function),
and $\phi\in\formula$, and it is written
$M,\mathit{env}\models\phi$ in our formalization. This completes the
description of the fourth and last formal layer of the development.

Internalized formulas for most (but not all) of the relational
concepts can be obtained by guiding the automatic tactics. Actually,
great many of the concepts in \session{ZF-Constructible} where
internalized by hand; this is the case for union,
%% \begin{multline*}
%%   \isatt{big{\uscore}union{\uscore}fm}(A,z) \equiv \\
%%   ((\cdot\forall\cdot\cdot0 \in succ(z)\cdot \longleftrightarrow
%%   (\cdot\exists\cdot\cdot0 \in succ(succ(A))\cdot \land \cdot1 \in
%%   0\cdot\cdot\cdot)\cdot\cdot)
%% \end{multline*}
for which we have the following satisfaction lemma:
% https://isabelle.in.tum.de/dist/library/ZF/ZF-Constructible/L_axioms.html#L_axioms.sats_big_union_fm|fact
\begin{multline}\label{eq:sats_big_union_fm}
  x \in \omega \implies y \in \omega \implies \mathit{env} \in \isatt{list}(A)
  \implies \\
  A, \mathit{env} \models \isatt{big{\uscore}union{\uscore}fm}(x,y)%\cdot\union x \isatt{ is } y\cdot
  \longleftrightarrow \\
  \isatt{big{\uscore}union}(\isatt{\#\#} A, \isatt{nth}(x,
  \mathit{env}), \isatt{nth}(y, \mathit{env}))
\end{multline}
Note that $x$ and $y$ above are de Bruijn indices,
$\isatt{nth}(x,\mathit{env})$ is the $x$th element of $\mathit{env}$
and $\isatt{\#\#}A::\tyi\fun \tyo$ is the class corresponding to the
set $A::\tyi$.

%%% Local Variables:
%%% mode: latex
%%% TeX-master: "independence_ch_isabelle"
%%% ispell-local-dictionary: "american"
%%% End: 

 
\section{Relative versions of non-absolute concepts}
\label{sec:relat-vers-non-absol}

The treatment of relativization/internalization described in the
previous sections was enough for Paulson's treatment of
constructibility. This is the case because essentially all the
concepts in the way of proving the consistency of $\AC$ are
absolute, and the treatment of relational versions and relativized notions
could be minimized after proving the relevant absoluteness results:
For example, the lemma \isa{Union{\uscore}abs},
\[
  M(A) \implies M(z) \implies \isa{big{\uscore}union}(M, A, z) \longleftrightarrow z = \union
  A
\]
proved under the assumption that $M$ is transitive and nonempty.

Our first attempt to relativize cardinal arithmetic proceeded in the
same way
and we rapidly found out that stating and proving statements like $(||A||
= |A|) ^M$ in a completely relational language was extremely
cumbersome. This observation lead to the discovery of the discipline
expounded in the next subsection.

%% Working
%% in this relational 
%% way with powersets, cardinalities, and the like would be
%% unfeasible. As such, cardinal arithmetic was not put in relative form
%% in \session{ZF-Constructible}.

\subsection{Discipline and tools for relativization}
\label{sec:tools-relativization}
The missing step, that naturally appears in the literature, consists
of having relative \emph{functions} like $\Pow^M$, and the ability to
translate between the different presentations discussed so far.

To achieve this, we provide automatic tools to ease the definitions of
such relative versions, their fully relational counterparts, and the
internalized formulas. For instance, consider the
$\isa{cardinal}::\tyi \fun \tyi$ function defined in
\session{Isabelle/ZF}. Then the commands
\begin{isabelle}
  \isacommand{relativize}\isamarkupfalse%
  \ \isakeyword{functional}\ {\isachardoublequoteopen}cardinal{\isachardoublequoteclose}\ {\isachardoublequoteopen}cardinal{\isacharunderscore}{\kern0pt}rel{\isachardoublequoteclose}\ \isakeyword{external}\isanewline
  \isacommand{relationalize}\isamarkupfalse%
  \ {\isachardoublequoteopen}cardinal{\isacharunderscore}{\kern0pt}rel{\isachardoublequoteclose}\ {\isachardoublequoteopen}is{\isacharunderscore}{\kern0pt}cardinal{\isachardoublequoteclose}\isanewline
  \isacommand{synthesize}\isamarkupfalse%
  \ {\isachardoublequoteopen}is{\isacharunderscore}{\kern0pt}cardinal{\isachardoublequoteclose}\ \isakeyword{from{\isacharunderscore}{\kern0pt}definition}\ \isakeyword{assuming}\ {\isachardoublequoteopen}nonempty{\isachardoublequoteclose}%
\end{isabelle}
define the relative cardinal function
$\isa{cardinal{\uscore}rel}::(\tyi \fun \tyo) \fun \tyi \fun\tyi$
(denoted  $|\cdot|^M$, as expected),
the relational version $\isa{is{\uscore}cardinal}$ of the latter, the
internalized formula \isa{is{\uscore}cardinal{\uscore}fm} whose
satisfaction by a set is equivalent to the relational version, and
prove the previous statement (analogous to (\ref{eq:sats_big_union_fm})).
The proof that $\isa{is{\uscore}cardinal}(M,x,z)$  encodes the
statement $|x|^M = z$ must still be done by hand, since the definition
of $\isa{cardinal{\uscore}rel}$ already involves some tacit
absoluteness results (“\textit{the least $z \in \Ord$ such that $z
  \approx^M x$}” instead
of “\textit{the least $z \in \Ord^M$ such that $z
  \approx^M x$}”, and the like).

\subsection{Extension of Isabelle/ZF}
\label{sec:extension-isabellezf}
We extended \cite{Delta_System_Lemma-AFP} the material formalized in
Isabelle, from basic results involving function spaces and the
definition of cardinal exponentiation, to a treatment of cofinality
and the Delta System Lemma for $\omega_1$-families. We also included a
concise treatment of the axiom of Dependent Choices $\DC$ and the
general version of Rasiowa-Sikorski Lemma \cite{2018arXiv180705174G}
and a choiceless one for countable preorders.

This material was subsequently put in relative form in our formal
development on transitive class models \cite{Transitive_Models-AFP}
using as an aid the tools from
Section~\ref{sec:tools-relativization}. We also relativized many
original theories appearing in Isabelle/ZF, including the
fundamentals of cardinal arithmetic, the cumulative hierarchy, and the
definition of the $\ale{}$ function.


%%% Local Variables:
%%% mode: latex
%%% TeX-master: "independence_ch_isabelle"
%%% ispell-local-dictionary: "american"
%%% End: 


%% \section{The definition of $\forceisa$}
\label{sec:definition-forces}

The core of the development is showing the definability of the
relation of forcing. As we explained in our previous
report~\cite[Sect.~8]{2019arXiv190103313G}, this comprises the
definition of a function $\forceisa$ that maps the set of internal
formulas into itself. It is the very reason of applicability of
forcing that the satisfaction of a first-order formula $\phi$ in all
of the generic extensions of a ctm $M$ can be ``controlled'' in a
definable way from $M$ (viz., by satisfaction of the formula
$\forceisa(\phi)$).

In fact, given a forcing notion $\PP$ (i.e. a preorder with a top element)
in a ctm $M$,
Kunen defines the \emph{forcing relation} model-theoretically 
by considering all extensions $M[G]$ with $G$ generic for $\PP$.
Then two fundamental results are proved, the Truth Lemma and the
Definability Lemma; but the proof of the first one is based on the
formula that witnesses Definability. To make sense of this in our 
formalization, we started with the internalized relation and then
proved that it is equivalent to the semantic version 
(``\isatt{definition{\isacharunderscore}of{\isacharunderscore}forces},'' in
the next section).
For that reason, the usual notation of the forcing relation 
$p \Vdash \phi\ \mathit{env}$ (for $\mathit{env}$ a list of
``names''), abbreviates in our code the
satisfaction by $M$ of $\forceisa(\phi)$:
\begin{isabelle}
\ \ {\isachardoublequoteopen}p\ {\isasymtturnstile}\ {\isasymphi}\ env\ \ \ {\isasymequiv}\ \ \ M{\isacharcomma}\ {\isacharparenleft}{\isacharbrackleft}p{\isacharcomma}P{\isacharcomma}leq{\isacharcomma}one{\isacharbrackright}\ {\isacharat}\ env{\isacharparenright}
    {\isasymTurnstile}\ forces{\isacharparenleft}{\isasymphi}{\isacharparenright}{\isachardoublequoteclose}
\end{isabelle}

The definition of $\forceisa$ proceeds by recursion
over the set $\formula$ and its base case, that is, for
atomic formulas, is (in)famously the most complicated one. Actually,
newcomers can be puzzled by the fact that forcing for atomic
formulas is also defined by (mutual) recursion: to know if $\tau_1\in\tau_2$ is
forced by $p$ (notation: $\forcesmem(p,\tau_1,\tau_2)$), one must check if $\tau_1=\sigma$ is forced for $\sigma$
moving in the transitive closure of $\tau_2$. To disentangle this, one
must realize that this last recursion must be described syntactically:
the definition of $\forceisa(\phi)$ for atomic $\phi$ is then an
internal definition of the alleged recursion on names. 

Our aim was to follow the definition proposed by Kunen
in~\cite[p.~257]{kunen2011set}, where the following mutual recursion
is given:
\begin{multline}\label{eq:def-forcing-equality}
  \forceseq (p,t_1,t_2) \defi 
  \forall s\in\dom(t_1)\cup\dom(t_2).\ \forall q\pleq p .\\
  \forcesmem(q,s,t_1)\lsii 
  \forcesmem(q,s,t_2),
\end{multline}
\begin{multline}\label{eq:def-forcing-membership}
  \forcesmem(p,t_1,t_2) \defi  \forall v\pleq p. \ \exists q\pleq v. \\
  \exists s.\ \exists r\in \PP .\ \lb s,r\rb \in
      t_2 \land q \pleq r \land \forceseq(q,t_1,s)
\end{multline}
Note that the definition of $\forcesmem$ is equivalent to require 
 the set 
\[
\{q\pleq p : \exists \lb s,r\rb\in t_2 . \ q\pleq r \land \forceseq(q,t_1,s)\}
\]
to be dense below $p$.

It was not straightforward to use the recursion machinery of
Isabelle/ZF to define $\forceseq$ and $\forcesmem$. For this, we
defined a relation $\frecR$ on 4-tuples of elements of $M$, proved
that it is well-founded and, more important, we also proved an
induction principle for this relation:
%
\begin{isabelle}
\isacommand{lemma}\isamarkupfalse%
\ forces{\isacharunderscore}induction{\isacharcolon}\isanewline
\ \ \isakeyword{assumes}\isanewline
\ \ \ \ {\isachardoublequoteopen}{\isasymAnd}{\isasymtau}\ {\isasymtheta}{\isachardot}\ {\isasymlbrakk}{\isasymAnd}{\isasymsigma}{\isachardot}\ {\isasymsigma}{\isasymin}domain{\isacharparenleft}{\isasymtheta}{\isacharparenright}\ {\isasymLongrightarrow}\ Q{\isacharparenleft}{\isasymtau}{\isacharcomma}{\isasymsigma}{\isacharparenright}{\isasymrbrakk}\ {\isasymLongrightarrow}\ R{\isacharparenleft}{\isasymtau}{\isacharcomma}{\isasymtheta}{\isacharparenright}{\isachardoublequoteclose}\footnotemark\isanewline
\ \ \ \ {\isachardoublequoteopen}{\isasymAnd}{\isasymtau}\ {\isasymtheta}{\isachardot}\ {\isasymlbrakk}{\isasymAnd}{\isasymsigma}{\isachardot}\ {\isasymsigma}{\isasymin}domain{\isacharparenleft}{\isasymtau}{\isacharparenright}\ {\isasymunion}\ domain{\isacharparenleft}{\isasymtheta}{\isacharparenright}\ {\isasymLongrightarrow}\ R{\isacharparenleft}{\isasymsigma}{\isacharcomma}{\isasymtau}{\isacharparenright}\ {\isasymand}\ R{\isacharparenleft}{\isasymsigma}{\isacharcomma}{\isasymtheta}{\isacharparenright}{\isasymrbrakk}\isanewline
\ \ \ \ \ \  {\isasymLongrightarrow}\ Q{\isacharparenleft}{\isasymtau}{\isacharcomma}{\isasymtheta}{\isacharparenright}{\isachardoublequoteclose}\isanewline
\ \ \isakeyword{shows}\isanewline
\ \ \ \ {\isachardoublequoteopen}Q{\isacharparenleft}{\isasymtau}{\isacharcomma}{\isasymtheta}{\isacharparenright}\ {\isasymand}\ R{\isacharparenleft}{\isasymtau}{\isacharcomma}{\isasymtheta}{\isacharparenright}{\isachardoublequoteclose}
\end{isabelle}
\footnotetext{The logical primitives of \emph{Pure} are
\isatt{\isasymLongrightarrow}, \isatt{\&\&\&}, and \isatt{\isasymAnd}
(implication, conjunction, and universal
quantification, resp.), which operate on the meta-Booleans
\isatt{prop}.}
%
and 
obtained both functions as cases of a another one, 
$\forcesat$, using a single recursion on $\frecR$. Then we obtained 
(\ref{eq:def-forcing-equality}) and (\ref{eq:def-forcing-membership})
as our corollaries \isatt{def{\isacharunderscore}forces{\isacharunderscore}eq} and
\isatt{def{\isacharunderscore}forces{\isacharunderscore}mem}.

Other approaches, like the one in Neeman~\cite{neeman-course} (and
Kunen's older book \cite{kunen1980}), prefer
to have a single, more complicated, definition by simple recursion for
$\forceseq$ and then define $\forcesmem$ explicitly. On hindsight,
this might have been a little simpler to do, but we preferred to be as
faithful to the text as possible concerning this point.

Once $\forcesat$ and its relativized version
$\isatt{is{\isacharunderscore}forces{\isacharunderscore}at}$ were
defined, we proceeded to show absoluteness and provided internal
definitions for the recursion on names using results in
\isatt{ZF-Constructible}. This finished the atomic case of the
formula-transformer $\forceisa$. The characterization of $\forceisa$
for negated and universal quantified formulas is given by the
following lemmas, respectively:
%
\begin{isabelle}
\isacommand{lemma}\isamarkupfalse%
\ sats{\isacharunderscore}forces{\isacharunderscore}Neg{\isacharcolon}\isanewline
\ \ \isakeyword{assumes}\isanewline
\ \ \ \ {\isachardoublequoteopen}p{\isasymin}P{\isachardoublequoteclose}\ {\isachardoublequoteopen}env\ {\isasymin}\ list{\isacharparenleft}M{\isacharparenright}{\isachardoublequoteclose}\ {\isachardoublequoteopen}{\isasymphi}{\isasymin}formula{\isachardoublequoteclose}\isanewline
\ \ \isakeyword{shows}\isanewline
\ \ \ \ {\isachardoublequoteopen}M{\isacharcomma}\ {\isacharbrackleft}p{\isacharcomma}P{\isacharcomma}leq{\isacharcomma}one{\isacharbrackright}\ {\isacharat}\ env\ {\isasymTurnstile}\ forces{\isacharparenleft}Neg{\isacharparenleft}{\isasymphi}{\isacharparenright}{\isacharparenright}\ \ \ {\isasymlongleftrightarrow}\ \isanewline
\ \ \ \ \ {\isasymnot}{\isacharparenleft}{\isasymexists}q{\isasymin}M{\isachardot}\ q{\isasymin}P\ {\isasymand}\ is{\isacharunderscore}leq{\isacharparenleft}{\isacharhash}{\isacharhash}M{\isacharcomma}leq{\isacharcomma}q{\isacharcomma}p{\isacharparenright}\ {\isasymand}\ \isanewline
\ \ \ \ \ \ \ \ \ \ M{\isacharcomma}\ {\isacharbrackleft}q{\isacharcomma}P{\isacharcomma}leq{\isacharcomma}one{\isacharbrackright}{\isacharat}env\ {\isasymTurnstile}\ forces{\isacharparenleft}{\isasymphi}{\isacharparenright}{\isacharparenright}{\isachardoublequoteclose}\isanewline

\isacommand{lemma}\isamarkupfalse%
\ sats{\isacharunderscore}forces{\isacharunderscore}Forall{\isacharcolon}\isanewline
\ \ \isakeyword{assumes}\isanewline
\ \ \ \ {\isachardoublequoteopen}p{\isasymin}P{\isachardoublequoteclose}\ {\isachardoublequoteopen}env\ {\isasymin}\ list{\isacharparenleft}M{\isacharparenright}{\isachardoublequoteclose}\ {\isachardoublequoteopen}{\isasymphi}{\isasymin}formula{\isachardoublequoteclose}\isanewline
\ \ \isakeyword{shows}\isanewline
\ \ \ \ {\isachardoublequoteopen}M{\isacharcomma}{\isacharbrackleft}p{\isacharcomma}P{\isacharcomma}leq{\isacharcomma}one{\isacharbrackright}\ {\isacharat}\ env\ {\isasymTurnstile}\ forces{\isacharparenleft}Forall{\isacharparenleft}{\isasymphi}{\isacharparenright}{\isacharparenright}\ {\isasymlongleftrightarrow}\ \isanewline
\ \ \ \ \ {\isacharparenleft}{\isasymforall}x{\isasymin}M{\isachardot}\ \ \ M{\isacharcomma}\ {\isacharbrackleft}p{\isacharcomma}P{\isacharcomma}leq{\isacharcomma}one{\isacharcomma}x{\isacharbrackright}\ {\isacharat}\ env\ {\isasymTurnstile}\ forces{\isacharparenleft}{\isasymphi}{\isacharparenright}{\isacharparenright}{\isachardoublequoteclose}
\end{isabelle}

Let us note in passing another improvement over our previous report:
we made a couple of new technical results concerning recursive
definitions. Paulson proved absoluteness of functions defined by
well-founded recursion over a transitive relation. Some of our
definitions by recursion (\emph{check} and \emph{forces}) do not fit
in that scheme.  One can replace the relation $R$ for its transitive
closure $R^+$ in the recursive definition because one can prove, in
general, that
$F\!\upharpoonright\!(R^{-1}(x))(y) =
F\!\upharpoonright\!({R^+}^{-1}(x))(y)$ whenever $(x,y) \in R$.


%%% Local Variables: 
%%% mode: latex
%%% TeX-master: "forcing_in_isabelle_zf"
%%% ispell-local-dictionary: "american"
%%% End: 


%%%%%%%%%%%%%%%%%%%%%%%%%%%%%%%%%%%%%%%%%%%%%%%%%%%%%%%%%%%%%%%%%%%%%%          
\section{Set models and forcing}
\label{sec:forcing}

\subsection{The $\ZFC$ axioms as locales}\label{sec:zfc-axioms-as-locales}
The description of set models of fragments of $\ZFC$ was performed
using Isabelle contexts (\emph{locales}) that fix a variable $M::\tyi$
and pack assumptions stating that $\lb M, \in\rb$ satisfy some
axioms. For example, the locale \isatt{M{\uscore}Z{\uscore}basic}
states that Zermelo set theory holds in $M$. The Union Axiom (\isatt{Union{\uscore}ax}), for
instance, is defined as follows:
\[
\forall x[\isatt{\#\#}M].\ \exists z[\isatt{\#\#}M].\ \isatt{big{\uscore}union}(\isatt{\#\#}M,x,z)
\]
%% % Same a above
%% \begin{isabelle}%
%% {\isasymforall}x{\isacharbrackleft}{\kern0pt}{\isacharhash}{\kern0pt}{\isacharhash}{\kern0pt}M{\isacharbrackright}{\kern0pt}{\isachardot}{\kern0pt}\ {\isasymexists}z{\isacharbrackleft}{\kern0pt}{\isacharhash}{\kern0pt}{\isacharhash}{\kern0pt}M{\isacharbrackright}{\kern0pt}{\isachardot}{\kern0pt}\ big{\isacharunderscore}{\kern0pt}union{\isacharparenleft}{\kern0pt}{\isacharhash}{\kern0pt}{\isacharhash}{\kern0pt}M{\isacharcomma}{\kern0pt}\ x{\isacharcomma}{\kern0pt}\ z{\isacharparenright}{\kern0pt}%
%% \end{isabelle}%
using relativized, relational versions of the axioms of Isabelle/ZF
since the interaction with \session{ZF-Constructible} was smoother
that way and, as it was mentioned in Section~\ref{sec:isabellezf},
this third layer of the formalization has more tools at our
disposal. Later, an equivalent statement “in the codes” (our fourth
layer) is obtained,
\[
  \isatt{Union{\uscore}ax}(\isatt{\#\#}M) \lsii A, [\,]\models \isatt{{\isasymcdot}Union\ Ax{\isasymcdot}}
\]
%% % Same a above
%% \begin{isabelle}
%% Union{\isacharunderscore}{\kern0pt}ax{\isacharparenleft}{\kern0pt}{\isacharhash}{\kern0pt}{\isacharhash}{\kern0pt}A{\isacharparenright}{\kern0pt}\ {\isasymlongleftrightarrow}\ A{\isacharcomma}{\kern0pt}\ {\isacharbrackleft}{\kern0pt}{\isacharbrackright}{\kern0pt}\ {\isasymTurnstile}\ {\isasymcdot}Union\ Ax{\isasymcdot}\isasep
%% \end{isabelle}
where \isatt{{\isasymcdot}Union\ Ax{\isasymcdot}} is the $\formula$ code for the
Union axiom. For the axiom schemes, \session{ZF-Constructible} defines
the expressions
\[
  \isatt{separation}(N,Q)
  \text{ and }
  \isatt{strong{\uscore}replacement}(N,R)
\]
that consume a class $N$ and predicates $Q::\tyi\fun\tyo$ and $R::\tyi\fun\tyi\fun\tyo$, and state that $N$
satisfies the $Q$-instance of Separation and the $R$-instance of
Replacement, respectively. In the statement of the Separation Axiom in
\isatt{M{\uscore}Z{\uscore}basic} and
the many replacement instances, predicates $Q$ and $R$ are actually
defined as the satisfaction of formulas of the appropriate arity; we can only show
that this form of the axioms hold in generic extensions. 

Further locales gather the assumption of transitivity of $M$ and
particular replacement instances expressed by means of the
\isatt{replacement{\uscore}assm(M,\isasymphi)} predicate ($@$ denotes
list concatenation):
\begin{multline}\label{eq:replacement_assm_def}
\varphi \in \formula  \limp \mathit{env} \in \isatt{list}(M) \limp \isatt{arity}(\varphi) \leq 2+ \isatt{length}(\mathit{env}) \limp \\
 \isatt{strong{\uscore}replacement}(\isatt{\#\#} M, \lambda x\, y.\ (M, [x,y]
\mathbin{@} \mathit{env}  \models \varphi))
\end{multline}
%% % Same as above
%% \begin{isabelle}
%% {\isasymphi}\;{\isasymin}\;formula\ {\isasymlongrightarrow}\ env\;{\isasymin}\;list{\isacharparenleft}{\kern0pt}M{\isacharparenright}{\kern0pt}\ {\isasymlongrightarrow} arity{\isacharparenleft}{\kern0pt}{\isasymphi}{\isacharparenright}{\kern0pt}\;{\isasymle}\;{\isadigit{2}}\ {\isacharplus}{\kern0pt}\isactrlsub {\isasymomega}\ length{\isacharparenleft}{\kern0pt}env{\isacharparenright}{\kern0pt}\ {\isasymlongrightarrow}\isanewline
%% \ \ \ \ strong{\isacharunderscore}{\kern0pt}replacement{\isacharparenleft}{\kern0pt}{\isacharhash}{\kern0pt}{\isacharhash}{\kern0pt}M{\isacharcomma}{\kern0pt}{\isasymlambda}x\ y{\isachardot}{\kern0pt}\ {\isacharparenleft}{\kern0pt}M\ {\isacharcomma}{\kern0pt}\ {\isacharbrackleft}{\kern0pt}x{\isacharcomma}{\kern0pt}y{\isacharbrackright}{\kern0pt}{\isacharat}{\kern0pt}env\ {\isasymTurnstile}\ {\isasymphi}{\isacharparenright}{\kern0pt}{\isacharparenright}
%% \end{isabelle}
%
%% Some of those instances, which we included for simplicity of design,
%% are actually “fake” instances in that they can be obtained from
%% Powerset, or from other replacement instances already available in the
%% respective context. An example of this is the sole instance of the
%% \isatt{M{\uscore}basic} locale, originally from
%% \session{ZF-Constructible}; we managed to eliminate it but in
%% doing so we had to reprove many lemmas from
%% that session in the weakened context. 
The locales gathering all 32 replacement instances necessary are named
\isatt{M{\uscore}ZF1} through \isatt{M{\uscore}ZF4},
\isatt{M{\uscore}ZF{\uscore}ground},
\isatt{M{\uscore}ZF{\uscore}ground{\uscore}notCH}, and
\isatt{M{\uscore}ZF{\uscore}ground{\uscore}CH} (see
Section~\ref{sec:repl-instances} for details).

%% Missing: Locales have to be interpreted (part of "big picture"?)

\subsection{The fundamental theorems}
Let $\lb \PP, {\preceq} ,\1\rb \in M$ be a forcing notion. In order to
fix the notation, the
$\val$ interpretation function takes $3$ arguments so that if $G\sbq \PP$, we have
$M[G]\defi \{ \val(\PP,G,\punto{a}) : \punto{a}\in M \}$.

The version of the Forcing Theorems that we formalized follows the
considerations on the $\forces^*$ relation as discussed in Kunen's new
\emph{Set Theory}
\cite[p.~257ff]{kunen2011set}. But in contrast to the point made on
p.~260 of this book, we internalized the recursion to define the forcing relation,
in that it involves codes for formulas, and the meta-theoretic formula
transformer $\phi\mapsto\mathit{Forces}_\phi$ is replaced by the
set-theoretic class function $\forceisa:: \tyi \fun \tyi$, which was defined by using
Isabelle/ZF facilities for primitive recursion.

Next we state this version of the fundamental theorems in a compact
way.
\begin{theorem}\label{th:forcing-thms}
  There exists a function  $\forceisa$
  such that for every
  $\phi\in\formula$ and $\punto{a}_0,\dots,\punto{a}_n\in M$,
  \begin{enumerate}
  \item\label{item:definability} (Definability)
    $\forceisa(\phi)\in\formula$, 
  \end{enumerate}
  where the 
  arity of $\forceisa(\phi)$ is at most $\isatt{arity}(\phi) + 4$; and if
  “$p \forces \phi\ [\punto{a}_0,\dots,\punto{a}_n]$”
  denotes
  “$M, [p,\PP,\preceq,\1, \punto{a}_0,\dots,\punto{a}_n]  \models
  \forceisa(\phi)$”, we have:
  \begin{enumerate}
    \setcounter{enumi}{1}
  \item\label{item:truth-lemma} (Truth Lemma) for every $M$-generic $G$,
    \[
      \exists p\in G.\ \; p \forces \phi\ [\punto{a}_0,\dots,\punto{a}_n]
    \]
    is equivalent to 
    \[
      M[G], [\val(\PP,G,\punto{a}_0),\dots,\val(\PP,G,\punto{a}_n)]
      \models \phi.
    \]
  \item \label{item:density-lemma} (Density Lemma) $p \forces \phi\ [\punto{a}_0,\dots,\punto{a}_n]$
    if and only if 
    $\{q\in \PP :  q \forces \phi\ [\punto{a}_0,\dots,\punto{a}_n]\}$
    is dense below $p$.
  \end{enumerate}
\end{theorem}
We actually have to define $\forceisa$ before we state the fundamental
theorems, so the main existential quantifier above does not appear in the
formalization.
Moreover, the items in Theorem~\ref{th:forcing-thms} appear as three separated lemmas in
\theory{Forcing{\uscore}Theorems} of our
\session{Independence\_CH} session \cite{Independence_CH-AFP},
and benefit from the $\isatt{map}$ function that applies a function to
each element of a list. For instance, the Truth Lemma is stated as
follows:
\begin{isabelle}
\isacommand{lemma}\isamarkupfalse%
\ truth{\isacharunderscore}{\kern0pt}lemma{\isacharcolon}{\kern0pt}\isanewline
\ \ \isakeyword{assumes}\isanewline
\ \ \ \ {\isachardoublequoteopen}{\isasymphi}{\isasymin}formula{\isachardoublequoteclose}\ {\isachardoublequoteopen}M{\isacharunderscore}{\kern0pt}generic{\isacharparenleft}{\kern0pt}G{\isacharparenright}{\kern0pt}{\isachardoublequoteclose}\isanewline
\ \ \ \ {\isachardoublequoteopen}env{\isasymin}list{\isacharparenleft}{\kern0pt}M{\isacharparenright}{\kern0pt}{\isachardoublequoteclose}\ {\isachardoublequoteopen}arity{\isacharparenleft}{\kern0pt}{\isasymphi}{\isacharparenright}{\kern0pt}{\isasymle}length{\isacharparenleft}{\kern0pt}env{\isacharparenright}{\kern0pt}{\isachardoublequoteclose}\isanewline
\ \ \isakeyword{shows}\isanewline
\ \ \ \ {\isachardoublequoteopen}{\isacharparenleft}{\kern0pt}{\isasymexists}p{\isasymin}G{\isachardot}{\kern0pt}\ p\ {\isasymtturnstile}\ {\isasymphi}\ env{\isacharparenright}{\kern0pt}\ \ \ {\isasymlongleftrightarrow}\ \ \ M{\isacharbrackleft}{\kern0pt}G{\isacharbrackright}{\kern0pt}{\isacharcomma}{\kern0pt}\ map{\isacharparenleft}{\kern0pt}val{\isacharparenleft}{\kern0pt}P{\isacharcomma}{\kern0pt}G{\isacharparenright}{\kern0pt}{\isacharcomma}{\kern0pt}env{\isacharparenright}{\kern0pt}\ {\isasymTurnstile}\ {\isasymphi}{\isachardoublequoteclose}
\end{isabelle}
where the $\forces$ notation (and its precedence) had already been set up in the
\theory{Forces{\uscore}Definition} theory as follows:
\begin{isabelle}
\isacommand{abbreviation}\isamarkupfalse%
\ Forces\ {\isacharcolon}{\kern0pt}{\isacharcolon}{\kern0pt}\ {\isachardoublequoteopen}{\isacharbrackleft}{\kern0pt}i{\isacharcomma}{\kern0pt}\ i{\isacharcomma}{\kern0pt}\ i{\isacharbrackright}{\kern0pt}\ {\isasymRightarrow}\ o{\isachardoublequoteclose}\ \ {\isacharparenleft}{\kern0pt}{\isachardoublequoteopen}{\isacharunderscore}{\kern0pt}\ {\isasymtturnstile}\ {\isacharunderscore}{\kern0pt}\ {\isacharunderscore}{\kern0pt}{\isachardoublequoteclose}\ {\isacharbrackleft}{\kern0pt}{\isadigit{3}}{\isadigit{6}}{\isacharcomma}{\kern0pt}{\isadigit{3}}{\isadigit{6}}{\isacharcomma}{\kern0pt}{\isadigit{3}}{\isadigit{6}}{\isacharbrackright}{\kern0pt}\ {\isadigit{6}}{\isadigit{0}}{\isacharparenright}{\kern0pt}\ \isakeyword{where}\isanewline
\ \ {\isachardoublequoteopen}p\ {\isasymtturnstile}\ {\isasymphi}\ env\ \ \ {\isasymequiv}\ \ \ M{\isacharcomma}{\kern0pt}\ {\isacharparenleft}{\kern0pt}{\isacharbrackleft}{\kern0pt}p{\isacharcomma}{\kern0pt}P{\isacharcomma}{\kern0pt}leq{\isacharcomma}{\kern0pt}{\isasymone}{\isacharbrackright}{\kern0pt}\ {\isacharat}{\kern0pt}\ env{\isacharparenright}{\kern0pt}\ {\isasymTurnstile}\ forces{\isacharparenleft}{\kern0pt}{\isasymphi}{\isacharparenright}{\kern0pt}{\isachardoublequoteclose}\isanewline
\end{isabelle}

Kunen first describes forcing for atomic formulas using a mutual
recursion
%% \begin{multline*}
%%   \forceseq (p,t_1,t_2) \defi 
%%   \forall s\in\dom(t_1)\cup\dom(t_2).\ \forall q\pleq p .\\
%%   \forcesmem(q,s,t_1)\lsii \forcesmem(q,s,t_2)
%% \end{multline*}
%% \begin{multline*}
%%   \forcesmem(p,t_1,t_2) \defi  \forall v\pleq p. \ \exists q\pleq v.\\  
%%   \exists s.\ \exists r\in \PP .\ \lb s,r\rb \in  t_2 \land q
%%   \pleq r \land \forceseq(q,t_1,s)
%% \end{multline*}
but then \cite[p.~257]{kunen2011set} it is cast as a single
recursively defined function $F$ over a wellfounded  relation $R$.
In our formalization, these are called $\frcat$ and 
$\isatt{frecR}$, respectively, and are defined on tuples $\lb \mathit{ft},t_1,t_2,p\rb$ (where
$\mathit{ft}\in\{0,1\}$ indicates whether the atomic formula being
forced is an equality or a membership, respectively).
Forcing for general formulas is then defined by recursion on the
datatype $\formula$ as indicated above. Technical details on the
implementation and proofs of the
Forcing Theorems have been spelled out in our
\cite{2020arXiv200109715G}.

%%% Local Variables: 
%%% mode: latex
%%% TeX-master: "independence_ch_isabelle"
%%% ispell-local-dictionary: "american"
%%% End: 


\section{Details of the formalization}

\subsection{32 replacement instances to rule them all}
\label{sec:repl-instances}

We isolated 32 instances of Replacement that are sufficient to force
$\CH$ or $\neg\CH$. The first 13 instances are needed to set up
cardinal arithmetic in $M$. Many of these were already present in
relational form in the \session{ZF-Constructible} library.

\begin{itemize}
\item 6 instances for iteration of operations. For each construct,
  Paulson used one replacement instance in order to have absoluteness,
  and a second one to obtain closure.
  \begin{itemize}
  \item
    Lists
    \isatt{list{\uscore}repl1{\uscore}intf{\uscore}fm},
    \isatt{list{\uscore}repl2{\uscore}intf{\uscore}fm}.
  \item
    Formulas
    \isatt{formula{\uscore}repl1{\uscore}intf{\uscore}fm},
    \isatt{formula{\uscore}repl1{\uscore}intf{\uscore}fm}.
  \item
    Transitive closure %(iteration of $\union$)
    \isatt{eclose{\uscore}repl1{\uscore}intf{\uscore}fm},
    \isatt{eclose{\uscore}repl2{\uscore}intf{\uscore}fm}.
  \end{itemize}
\item 1 instance for absoluteness of the definition of the
  $n$th element of a list (iteration of the tail operation)
  \isatt{tl{\uscore}repl{\uscore}intf{\uscore}fm}. By transitivity,
  we do not need an instance for closure.
\end{itemize}
The instances so far
are needed to interpret locales \isatt{M{\uscore}datatypes} and
\isatt{M{\uscore}eclose}. The former is used in the relative definition of
\isatt{Vset}.
\begin{itemize}
\item
  2 instances for ordertypes.
  \begin{itemize}
  \item Auxiliary for the definition of ordertypes
    \isatt{replacement{\uscore}is{\uscore}order{\uscore}eq{\uscore}map{\uscore}fm}.
    
    This one is needed for the interpretation of
    \isatt{M{\uscore}ordertype}.
    
  \item Replacement through $x\mapsto \otype(x)$, for Hartogs' Theorem
    \isatt{replacement{\uscore}is{\uscore}order{\uscore}body{\uscore}fm}.
  \end{itemize}
\item
  4 instances for definitions by well-founded recursion.
  \begin{itemize}
  \item Ordertypes
    \isatt{wfrec{\uscore}replacement{\uscore}order{\uscore}pred{\uscore}fm}.
  \item Rank \isatt{wfrec{\uscore}rank{\uscore}fm}.
  \item Cumulative hierarchy (rank initial segments) \isatt{trans{\uscore}repl{\uscore}HVFrom{\uscore}fm}.
  \item Aleph \isatt{replacement{\uscore}HAleph{\uscore}wfrec{\uscore}repl{\uscore}body{\uscore}fm}.
  \end{itemize}
\end{itemize}

We also need a one extra replacement instance $\psi$ on $M$ for each
$\phi$ of the
previous ones to have them in $M[G]$.
\[
  \psi(x,\alpha,y_1,\dots,y_n) \defi \quine{\alpha = \min \bigl\{
    \beta \mid \exists\tau\in V_\beta.\  \mathit{snd}(x) \forces
    \phi\ [\mathit{fst}(x),\tau,y_1,\dots,y_n]\bigr\}}
\]
In our development, the mapping $\phi\mapsto\psi$ defined above is given by the
$\isatt{ground{\uscore}repl{\uscore}fm}$ function, and all ground replacement
instances appear in the locale \isatt{M{\uscore}ZF4}. These are expressed using
the \isatt{ground{\uscore}replacement{\uscore}assm(M,\isasymphi)} predicate
obtained by replacing $\phi$ by
$\isatt{ground{\uscore}repl{\uscore}fm}(\phi)$ in Eq.~(\ref{eq:replacement_assm_def}).

That makes 26 instances up to now. For the setup of forcing, we
require the following 5 instances (the last two are needed for the $\Delta$-System Lemma):

\begin{itemize}
\item 2 instances for definitions by well-founded recursion.
  \begin{itemize}
  \item
    Check names
    \isatt{wfrec{\uscore}Hcheck{\uscore}fm}.
  \item Forcing for atomic formulas
    \isatt{wfrec{\uscore}Hfrc{\uscore}at{\uscore}fm}.
  \end{itemize}
\item Replacement through $x\mapsto \lb x,\check{x}\rb$ (for the
  definition of $\punto{G}$)
  \isatt{Lambda{\uscore}in{\uscore}M{\uscore}fm(check{\uscore}fm(2,0,1),1)}.
\item Recursive construction of sets using a choice function
  \isatt{replacement{\uscore}is{\uscore}trans{\uscore}apply{\uscore}image{\uscore}fm}.
\item Absoluteness of the previous construction \isatt{replacement{\uscore}transrec{\uscore}apply{\uscore}image{\uscore}body{\uscore}fm}.
\end{itemize}
%
This is enough to force $\neg\CH$. To force $\CH$, one further instance is needed:
%
\begin{itemize}
\item Absoluteness of the recursive construction in the proof of the
  Dependent Choices \isatt{replacement{\uscore}dcwit{\uscore}repl{\uscore}body{\uscore}fm}.
\end{itemize}

The particular choice of some of the instances above arose from
Paulson's architecture on which we based our development.
This applies every time
a locale from \session{ZF-Constructible} has to be
interpreted (namely \isatt{M{\uscore}datatypes}
\isatt{M{\uscore}eclose}, and \isatt{M{\uscore}ordertype}).
%% For instance, the first
%% instance required for the definition of relative ordertypes arises
%% from Paulson's \session{ZF-Constructible}.
% https://isabelle.in.tum.de/dist/library/ZF/ZF-Constructible/Rank.html#offset_1123..1139

On the other hand, as explained in
Section~\ref{sec:zfc-axioms-as-locales}, we managed to eliminate the
instance arising from the \isatt{M{\uscore}basic} locale. Similarly,
we replaced the original proof of the Schröder-Bernstein Theorem by
Zermelo's one \cite[Exr. x4.27]{moschovakis1994notes}, because the
former required at least one extra instance
% (\isatt{banach{\uscore}iterates{\uscore}fm})
arising from an iteration.

It is to be noted that application of the Forcing Theorems do not
require any extra replacement instances on $M$ (independently of the
formula $\phi$ for which they are invoked). This is not the case for
Separation, at least from our formalization: More instances are needed
as the complexity of $\phi$ grows. One point where this is apparent is
in the proof of Theorem~\ref{th:forcing-thms}(\ref{item:truth-lemma}),
that appears as the \isatt{truth{\uscore}lemma} in our development; it
depends on \isatt{truth{\uscore}lemma'} and
\isatt{truth{\uscore}lemma{\uscore}Neg} which explicitly invoke
\isatt{separation{\uscore}ax}.

\subsection{A sample formal proof}
\label{sec:sample-formal-proof}

We present a fragment of the formal version of Kunen's proof that the
Powerset Axiom holds in a generic extension. We quote the relevant
paragraph of \cite[Thm.~IV.2.27]{kunen2011set}:
\begin{quote}
  For Power Set (similarly to Union above), it is sufficient to prove
  that whenever $a \in M[G]$, there is a $b \in M[G]$ such that
  $\mathcal{P}(a) \cap M[G] \subseteq b$. Fix $\tau \in
  M^{\mathbb{P}}$ such that $\tau_{G}=a$. Let
  $Q=(\mathcal{P}(\operatorname{dom}(\tau) \times
  \mathbb{P}))^{M}$. This is the set of all names $\vartheta \in
  M^{\mathbb{P}}$ such that $\operatorname{dom}(\vartheta) \subseteq
  \operatorname{dom}(\tau)$. Let $\pi=Q \times\{\1\}$ and let
  $b=\pi_{G}=$ $\left\{\vartheta_{G}: \vartheta \in Q\right\}$. Now,
  consider any $c \in \mathcal{P}(a) \cap M[G]$; we need to show that
  $c \in b$. Fix $\varkappa \in M^{\mathbb{P}}$ such that
  $\varkappa_{G}=c$, and let $\vartheta=\{\langle\sigma, p\rangle:
  \sigma \in \operatorname{dom}(\tau) \wedge p \Vdash \sigma \in
  \varkappa\}$; $\vartheta \in M$ by the Definability Lemma. Since
  $\vartheta \in Q$, we are done if we can show that
  $\vartheta_{G}=c$.
\end{quote}
The assumption $a\in M[G]$ appears in the lemma statement, and the
goal involving $b$ in the first sentence will appear below (page \pageref{goal-on-b}); formalized
material necessarily tends to be much more linear than usual prose. In
what follows, we
will intersperse the relevant passages of the proof.
\begin{isabelle}
\isacommand{lemma}\isamarkupfalse%
\ Pow{\isacharunderscore}{\kern0pt}inter{\isacharunderscore}{\kern0pt}MG{\isacharcolon}{\kern0pt}\isanewline
\ \ \isakeyword{assumes}\ {\isachardoublequoteopen}a{\isasymin}M{\isacharbrackleft}{\kern0pt}G{\isacharbrackright}{\kern0pt}{\isachardoublequoteclose}\isanewline
\ \ \isakeyword{shows}\ {\isachardoublequoteopen}Pow{\isacharparenleft}{\kern0pt}a{\isacharparenright}{\kern0pt}\ {\isasyminter}\ M{\isacharbrackleft}{\kern0pt}G{\isacharbrackright}{\kern0pt}\ {\isasymin}\ M{\isacharbrackleft}{\kern0pt}G{\isacharbrackright}{\kern0pt}{\isachardoublequoteclose}\isanewline
%
\isacommand{proof}\isamarkupfalse%
\ {\isacharminus}{\kern0pt}
\end{isabelle}
\textit{Fix $\tau \in  M^{\mathbb{P}}$ such that $\tau_{G}=a$.}
\begin{isabelle}
\ \ \isacommand{from}\isamarkupfalse%
\ assms\isanewline
\ \ \isacommand{obtain}\isamarkupfalse%
\ {\isasymtau}\ \isakeyword{where}\ {\isachardoublequoteopen}{\isasymtau}\ {\isasymin}\ M{\isachardoublequoteclose}\ {\isachardoublequoteopen}val{\isacharparenleft}{\kern0pt}P{\isacharcomma}{\kern0pt}G{\isacharcomma}{\kern0pt}\ {\isasymtau}{\isacharparenright}{\kern0pt}\ {\isacharequal}{\kern0pt}\ a{\isachardoublequoteclose}\isanewline
\ \ \ \ \isacommand{using}\isamarkupfalse%
\ GenExtD\ \isacommand{by}\isamarkupfalse%
\ auto
\end{isabelle}
\textit{Let
  $Q=(\mathcal{P}(\operatorname{dom}(\tau) \times
  \mathbb{P}))^{M}$.}
\begin{isabelle}
\ \ \isacommand{let}\isamarkupfalse%
\ {\isacharquery}{\kern0pt}Q{\isacharequal}{\kern0pt}{\isachardoublequoteopen}Pow{\isacharparenleft}{\kern0pt}domain{\isacharparenleft}{\kern0pt}{\isasymtau}{\isacharparenright}{\kern0pt}{\isasymtimes}P{\isacharparenright}{\kern0pt}\ {\isasyminter}\ M{\isachardoublequoteclose}
\end{isabelle}
\textit{This is the set of all names $\vartheta \in
  M^{\mathbb{P}}$} [\dots]---it is pretty laborious to show that things
are in $M$; it takes 17 more lines of code (not shown below) that
apply further previously proved lemmas.
\begin{isabelle}
\ \ \isacommand{from}\isamarkupfalse%
\ {\isacartoucheopen}{\isasymtau}{\isasymin}M{\isacartoucheclose}\isanewline
\ \ \isacommand{have}\isamarkupfalse%
\ {\isachardoublequoteopen}domain{\isacharparenleft}{\kern0pt}{\isasymtau}{\isacharparenright}{\kern0pt}{\isasymtimes}P\ {\isasymin}\ M{\isachardoublequoteclose}\ {\isachardoublequoteopen}domain{\isacharparenleft}{\kern0pt}{\isasymtau}{\isacharparenright}{\kern0pt}\ {\isasymin}\ M{\isachardoublequoteclose}\isanewline
\ \ \ \ \isacommand{using}\isamarkupfalse%
\ domain{\isacharunderscore}{\kern0pt}closed\ cartprod{\isacharunderscore}{\kern0pt}closed\ P{\isacharunderscore}{\kern0pt}in{\isacharunderscore}{\kern0pt}M\isanewline
\ \ \ \ \isacommand{by}\isamarkupfalse%
\ simp{\isacharunderscore}{\kern0pt}all\isanewline
\ \ \isacommand{then}\isamarkupfalse%
\isanewline
\ \ \isacommand{have}\isamarkupfalse%
\ {\isachardoublequoteopen}{\isacharquery}{\kern0pt}Q\ {\isasymin}\ M{\isachardoublequoteclose}
\end{isabelle}
[\dots]\textit{ Let $\pi=Q \times\{\1\}$ and let
  $b=\pi_{G}=$ $\left\{\vartheta_{G}: \vartheta \in Q\right\}$.}
\begin{isabelle}
\ \ \isacommand{let}\isamarkupfalse%
\ {\isacharquery}{\kern0pt}{\isasympi}{\isacharequal}{\kern0pt}{\isachardoublequoteopen}{\isacharquery}{\kern0pt}Q{\isasymtimes}{\isacharbraceleft}{\kern0pt}{\isasymone}{\isacharbraceright}{\kern0pt}{\isachardoublequoteclose}\isanewline
\ \ \isacommand{let}\isamarkupfalse%
\ {\isacharquery}{\kern0pt}b{\isacharequal}{\kern0pt}{\isachardoublequoteopen}val{\isacharparenleft}{\kern0pt}P{\isacharcomma}{\kern0pt}G{\isacharcomma}{\kern0pt}{\isacharquery}{\kern0pt}{\isasympi}{\isacharparenright}{\kern0pt}{\isachardoublequoteclose}%% \isanewline
%% \ \ \isacommand{from}\isamarkupfalse%
%% \ {\isacartoucheopen}{\isacharquery}{\kern0pt}Q{\isasymin}M{\isacartoucheclose}\isanewline
%% \ \ \isacommand{have}\isamarkupfalse%
%% \ {\isachardoublequoteopen}{\isacharquery}{\kern0pt}{\isasympi}{\isasymin}M{\isachardoublequoteclose}\isanewline
%% \ \ \ \ \isacommand{using}\isamarkupfalse%
%% \ one{\isacharunderscore}{\kern0pt}in{\isacharunderscore}{\kern0pt}P\ P{\isacharunderscore}{\kern0pt}in{\isacharunderscore}{\kern0pt}M\ transitivity\isanewline
%% \ \ \ \ \isacommand{by}\isamarkupfalse%
%% \ {\isacharparenleft}{\kern0pt}simp\ flip{\isacharcolon}{\kern0pt}\ setclass{\isacharunderscore}{\kern0pt}iff{\isacharparenright}{\kern0pt}\isanewline
%% \ \ \isacommand{then}\isamarkupfalse%
%% \isanewline
%% \ \ \isacommand{have}\isamarkupfalse%
%% \ {\isachardoublequoteopen}{\isacharquery}{\kern0pt}b\ {\isasymin}\ M{\isacharbrackleft}{\kern0pt}G{\isacharbrackright}{\kern0pt}{\isachardoublequoteclose}\isanewline
%% \ \ \ \ \isacommand{using}\isamarkupfalse%
%% \ GenExtI\ \isacommand{by}\isamarkupfalse%
%% \ simp
\end{isabelle}
\textit{Now,
  consider any $c \in \mathcal{P}(a) \cap M[G]$; we need to show that
  $c \in b$.}
\begin{isabelle}
  \label{goal-on-b}
\ \ \isacommand{have}\isamarkupfalse%
\ {\isachardoublequoteopen}Pow{\isacharparenleft}{\kern0pt}a{\isacharparenright}{\kern0pt}\ {\isasyminter}\ M{\isacharbrackleft}{\kern0pt}G{\isacharbrackright}{\kern0pt}\ {\isasymsubseteq}\ {\isacharquery}{\kern0pt}b{\isachardoublequoteclose}\isanewline
\ \ \isacommand{proof}\isamarkupfalse%
\isanewline
\ \ \ \ \isacommand{fix}\isamarkupfalse%
\ c\isanewline
\ \ \ \ \isacommand{assume}\isamarkupfalse%
\ {\isachardoublequoteopen}c\ {\isasymin}\ Pow{\isacharparenleft}{\kern0pt}a{\isacharparenright}{\kern0pt}\ {\isasyminter}\ M{\isacharbrackleft}{\kern0pt}G{\isacharbrackright}{\kern0pt}{\isachardoublequoteclose}
\end{isabelle}
\textit{Fix $\varkappa \in M^{\mathbb{P}}$ such that
  $\varkappa_{G}=c$,}
\begin{isabelle}
\ \ \ \ \isacommand{then}\isamarkupfalse%
\isanewline
\ \ \ \ \isacommand{obtain}\isamarkupfalse%
\ {\isasymchi}\ \isakeyword{where}\ {\isachardoublequoteopen}c{\isasymin}M{\isacharbrackleft}{\kern0pt}G{\isacharbrackright}{\kern0pt}{\isachardoublequoteclose}\ {\isachardoublequoteopen}{\isasymchi}\ {\isasymin}\ M{\isachardoublequoteclose}\ {\isachardoublequoteopen}val{\isacharparenleft}{\kern0pt}P{\isacharcomma}{\kern0pt}G{\isacharcomma}{\kern0pt}{\isasymchi}{\isacharparenright}{\kern0pt}\ {\isacharequal}{\kern0pt}\ c{\isachardoublequoteclose}\isanewline
\ \ \ \ \ \ \isacommand{using}\isamarkupfalse%
\ GenExt{\isacharunderscore}{\kern0pt}iff\ \isacommand{by}\isamarkupfalse%
\ auto
\end{isabelle}
\textit{and let $\vartheta=\{\langle\sigma, p\rangle:
  \sigma \in \operatorname{dom}(\tau) \wedge p \Vdash \sigma \in
  \varkappa\}$;}
\begin{isabelle}
\ \ \ \ \isacommand{let}\isamarkupfalse%
\ {\isacharquery}{\kern0pt}{\isasymtheta}{\isacharequal}{\kern0pt}{\isachardoublequoteopen}{\isacharbraceleft}{\kern0pt}{\isasymlangle}{\isasymsigma}{\isacharcomma}{\kern0pt}p{\isasymrangle}\ {\isasymin}domain{\isacharparenleft}{\kern0pt}{\isasymtau}{\isacharparenright}{\kern0pt}{\isasymtimes}P\ {\isachardot}{\kern0pt}\ p\ {\isasymtturnstile}\ {\isasymcdot}{\isadigit{0}}\ {\isasymin}\ {\isadigit{1}}{\isasymcdot}\ {\isacharbrackleft}{\kern0pt}{\isasymsigma}{\isacharcomma}{\kern0pt}{\isasymchi}{\isacharbrackright}{\kern0pt}\ {\isacharbraceright}{\kern0pt}{\isachardoublequoteclose}
\end{isabelle}
\textit{$\vartheta \in M$ by the Definability Lemma.}
\begin{isabelle}
\ \ \ \ \isacommand{have}\isamarkupfalse%
\ {\isachardoublequoteopen}arity{\isacharparenleft}{\kern0pt}forces{\isacharparenleft}{\kern0pt}Member{\isacharparenleft}{\kern0pt}{\isadigit{0}}{\isacharcomma}{\kern0pt}{\isadigit{1}}{\isacharparenright}{\kern0pt}{\isacharparenright}{\kern0pt}{\isacharparenright}{\kern0pt}\ {\isacharequal}{\kern0pt}\ {\isadigit{6}}{\isachardoublequoteclose}\isanewline
\ \ \ \ \ \ \isacommand{using}\isamarkupfalse%
\ arity{\isacharunderscore}{\kern0pt}forces{\isacharunderscore}{\kern0pt}at\ \isacommand{by}\isamarkupfalse%
\ auto\isanewline
\ \ \ \ \isacommand{with}\isamarkupfalse%
\ {\isacartoucheopen}domain{\isacharparenleft}{\kern0pt}{\isasymtau}{\isacharparenright}{\kern0pt}\ {\isasymin}\ M{\isacartoucheclose}\ {\isacartoucheopen}{\isasymchi}\ {\isasymin}\ M{\isacartoucheclose}\isanewline
\ \ \ \ \isacommand{have}\isamarkupfalse%
\ {\isachardoublequoteopen}{\isacharquery}{\kern0pt}{\isasymtheta}\ {\isasymin}\ M{\isachardoublequoteclose}\isanewline
\ \ \ \ \ \ \isacommand{using}\isamarkupfalse%
\ P{\isacharunderscore}{\kern0pt}in{\isacharunderscore}{\kern0pt}M\ one{\isacharunderscore}{\kern0pt}in{\isacharunderscore}{\kern0pt}M\ leq{\isacharunderscore}{\kern0pt}in{\isacharunderscore}{\kern0pt}M\ sats{\isacharunderscore}{\kern0pt}fst{\isacharunderscore}{\kern0pt}snd{\isacharunderscore}{\kern0pt}in{\isacharunderscore}{\kern0pt}M\isanewline
\ \ \ \ \ \ \isacommand{by}\isamarkupfalse%
\ simp
\end{isabelle}
\textit{Since
  $\vartheta \in Q$,}
\begin{isabelle}
\ \ \ \ \isacommand{then}\isamarkupfalse%
\isanewline
\ \ \ \ \isacommand{have}\isamarkupfalse%
\ {\isachardoublequoteopen}{\isacharquery}{\kern0pt}{\isasymtheta}\ {\isasymin}\ {\isacharquery}{\kern0pt}Q{\isachardoublequoteclose}\ \isacommand{by}\isamarkupfalse%
\ auto\isanewline
\ \ \ \ \isacommand{then}\isamarkupfalse%
\isanewline
\ \ \ \ \isacommand{have}\isamarkupfalse%
\ {\isachardoublequoteopen}val{\isacharparenleft}{\kern0pt}P{\isacharcomma}{\kern0pt}G{\isacharcomma}{\kern0pt}{\isacharquery}{\kern0pt}{\isasymtheta}{\isacharparenright}{\kern0pt}\ {\isasymin}\ {\isacharquery}{\kern0pt}b{\isachardoublequoteclose}\isanewline
\ \ \ \ \ \ \isacommand{using}\isamarkupfalse%
\ one{\isacharunderscore}{\kern0pt}in{\isacharunderscore}{\kern0pt}G\ one{\isacharunderscore}{\kern0pt}in{\isacharunderscore}{\kern0pt}P\ generic\ val{\isacharunderscore}{\kern0pt}of{\isacharunderscore}{\kern0pt}elem\ {\isacharbrackleft}{\kern0pt}of\ {\isacharquery}{\kern0pt}{\isasymtheta}\ {\isasymone}\ {\isacharquery}{\kern0pt}{\isasympi}\ G{\isacharbrackright}{\kern0pt}\isanewline
\ \ \ \ \ \ \isacommand{by}\isamarkupfalse%
\ auto
\end{isabelle}
\textit{we are done if we can show that
  $\vartheta_{G}=c$.}
\begin{isabelle}
\ \ \ \ \isacommand{have}\isamarkupfalse%
\ {\isachardoublequoteopen}val{\isacharparenleft}{\kern0pt}P{\isacharcomma}{\kern0pt}G{\isacharcomma}{\kern0pt}{\isacharquery}{\kern0pt}{\isasymtheta}{\isacharparenright}{\kern0pt}\ {\isacharequal}{\kern0pt}\ c{\isachardoublequoteclose}
\end{isabelle}

%%% Local Variables: 
%%% mode: latex
%%% TeX-master: "independence_ch_isabelle"
%%% ispell-local-dictionary: "american"
%%% End: 

%% 
%% \section{The forcing theorems}
\label{sec:forcing-theorems}

After the definition of $\forceisa$ is complete, the proof of the
Fundamental Theorems of Forcing is comparatively straightforward, and
we were able to follow Kunen very closely. The more involved points of
this part of the development were those where we needed to proved that
various (dense) subsets of $\PP$ were in $M$; for this, we had to
recourse to several absoluteness ad-hoc lemmas.

The first results concern characterizations of the forcing
relation. Two of them are \isatt{Forces{\isacharunderscore}Member}:
\begin{center}
  \isatt{{\isacharparenleft}p\ {\isasymtturnstile}\ Member{\isacharparenleft}n{\isacharcomma}m{\isacharparenright}\ env{\isacharparenright}\ {\isasymlongleftrightarrow}\ forces{\isacharunderscore}mem{\isacharparenleft}p{\isacharcomma}t{\isadigit{1}}{\isacharcomma}t{\isadigit{2}}{\isacharparenright}},
\end{center}
where \isatt{t{\isadigit{1}}} and \isatt{t{\isadigit{1}}} are the
\isatt{n}th resp.\ \isatt{m}th elements of \isatt{env}, and  \isatt{Forces{\isacharunderscore}Forall}:
\begin{center}
  \isatt{{\isacharparenleft}p\ {\isasymtturnstile}\ Forall{\isacharparenleft}{\isasymphi}{\isacharparenright}\ env{\isacharparenright}\ {\isasymlongleftrightarrow}\ {\isacharparenleft}{\isasymforall}x{\isasymin}M{\isachardot}\ {\isacharparenleft}p\ {\isasymtturnstile}\ {\isasymphi}\ {\isacharparenleft}{\isacharbrackleft}x{\isacharbrackright}\ {\isacharat}\ env{\isacharparenright}{\isacharparenright}{\isacharparenright}}.
\end{center}
These two, along with  \isatt{Forces{\isacharunderscore}Equal} and
\isatt{Forces{\isacharunderscore}Nand}, appear in Kunen as the
inductive definition of the forcing relation \cite[Def.~IV.2.42]{kunen2011set}.

As with the previous section, the proofs of the forcing theorems have two different
flavours: The ones for the atomic formulas proceed by using the
principle of 
\isatt{forces{\isacharunderscore}induction}, and then an induction on
$\formula$ wraps the former with the remaining cases (\isatt{Nand} and \isatt{Forall}). 

As an example of the first class, we can take a look at our
formalization of \cite[Lem.~IV.2.40(a)]{kunen2011set}:

\begin{isabelle}
  \isacommand{lemma}\isamarkupfalse%
  \ IV{\isadigit{2}}{\isadigit{4}}{\isadigit{0}}a{\isacharcolon}\isanewline
  \ \ \isakeyword{assumes}\isanewline
  \ \ \ \ {\isachardoublequoteopen}M{\isacharunderscore}generic{\isacharparenleft}G{\isacharparenright}{\isachardoublequoteclose}\isanewline
  \ \ \isakeyword{shows}\ \isanewline
  \ \ \ \ {\isachardoublequoteopen}{\isacharparenleft}{\isasymtau}{\isasymin}M{\isasymlongrightarrow}{\isasymtheta}{\isasymin}M{\isasymlongrightarrow}{\isacharparenleft}{\isasymforall}p{\isasymin}G{\isachardot}forces{\isacharunderscore}eq{\isacharparenleft}p{\isacharcomma}{\isasymtau}{\isacharcomma}{\isasymtheta}{\isacharparenright}{\isasymlongrightarrow}val{\isacharparenleft}G{\isacharcomma}{\isasymtau}{\isacharparenright}{\isacharequal}val{\isacharparenleft}G{\isacharcomma}{\isasymtheta}{\isacharparenright}{\isacharparenright}{\isacharparenright}{\isasymand}\isanewline
  \ \ \ \ \ {\isacharparenleft}{\isasymtau}{\isasymin}M{\isasymlongrightarrow}{\isasymtheta}{\isasymin}M{\isasymlongrightarrow}{\isacharparenleft}{\isasymforall}p{\isasymin}G{\isachardot}forces{\isacharunderscore}mem{\isacharparenleft}p{\isacharcomma}{\isasymtau}{\isacharcomma}{\isasymtheta}{\isacharparenright}{\isasymlongrightarrow}val{\isacharparenleft}G{\isacharcomma}{\isasymtau}{\isacharparenright}{\isasymin}val{\isacharparenleft}G{\isacharcomma}{\isasymtheta}{\isacharparenright}{\isacharparenright}{\isacharparenright}{\isachardoublequoteclose}
\end{isabelle}
%
Its proof starts by an introduction of \isatt{forces{\isacharunderscore}induction};
the  inductive cases for each atomic type were handled before as
separate lemmas (\isatt{IV240a{\isacharunderscore}mem} and \isatt{IV240a{\isacharunderscore}eq}). We
illustrate with the statement of the latter.
%
\begin{isabelle}
\isacommand{lemma}\isamarkupfalse%
\ IV{\isadigit{2}}{\isadigit{4}}{\isadigit{0}}a{\isacharunderscore}eq{\isacharcolon}\isanewline
\ \ \isakeyword{assumes}\isanewline
\ \ \ \ {\isachardoublequoteopen}M{\isacharunderscore}generic{\isacharparenleft}G{\isacharparenright}{\isachardoublequoteclose}\ {\isachardoublequoteopen}p{\isasymin}G{\isachardoublequoteclose}\ {\isachardoublequoteopen}forces{\isacharunderscore}eq{\isacharparenleft}p{\isacharcomma}{\isasymtau}{\isacharcomma}{\isasymtheta}{\isacharparenright}{\isachardoublequoteclose}\isanewline
\ \ \ \ \isakeyword{and}\isanewline
\ \ \ \ IH{\isacharcolon}{\isachardoublequoteopen}{\isasymAnd}q\ {\isasymsigma}{\isachardot}\ q{\isasymin}P\ {\isasymLongrightarrow}\ q{\isasymin}G\ {\isasymLongrightarrow}\ {\isasymsigma}{\isasymin}domain{\isacharparenleft}{\isasymtau}{\isacharparenright}\ {\isasymunion}\ domain{\isacharparenleft}{\isasymtheta}{\isacharparenright}\ {\isasymLongrightarrow}\ \isanewline
\ \ \ \ \ \ \ \ {\isacharparenleft}forces{\isacharunderscore}mem{\isacharparenleft}q{\isacharcomma}{\isasymsigma}{\isacharcomma}{\isasymtau}{\isacharparenright}\ {\isasymlongrightarrow}\ val{\isacharparenleft}G{\isacharcomma}{\isasymsigma}{\isacharparenright}\ {\isasymin}\ val{\isacharparenleft}G{\isacharcomma}{\isasymtau}{\isacharparenright}{\isacharparenright}\ {\isasymand}\isanewline
\ \ \ \ \ \ \ \ {\isacharparenleft}forces{\isacharunderscore}mem{\isacharparenleft}q{\isacharcomma}{\isasymsigma}{\isacharcomma}{\isasymtheta}{\isacharparenright}\ {\isasymlongrightarrow}\ val{\isacharparenleft}G{\isacharcomma}{\isasymsigma}{\isacharparenright}\ {\isasymin}\ val{\isacharparenleft}G{\isacharcomma}{\isasymtheta}{\isacharparenright}{\isacharparenright}{\isachardoublequoteclose}\isanewline
\ \ \isakeyword{shows}\isanewline
\ \ \ \ {\isachardoublequoteopen}val{\isacharparenleft}G{\isacharcomma}{\isasymtau}{\isacharparenright}\ {\isacharequal}\ val{\isacharparenleft}G{\isacharcomma}{\isasymtheta}{\isacharparenright}{\isachardoublequoteclose}
\end{isabelle}

As an example of the second kind of induction (on formulas), we choose the
following relatively simple result:

\begin{isabelle}
\isacommand{lemma}\isamarkupfalse%
\ strengthening{\isacharunderscore}lemma{\isacharcolon}\isanewline
\ \ \isakeyword{assumes}\ \isanewline
\ \ \ \ {\isachardoublequoteopen}p{\isasymin}P{\isachardoublequoteclose}\ {\isachardoublequoteopen}{\isasymphi}{\isasymin}formula{\isachardoublequoteclose}\ {\isachardoublequoteopen}r{\isasymin}P{\isachardoublequoteclose}\ {\isachardoublequoteopen}r{\isasympreceq}p{\isachardoublequoteclose}\isanewline
\ \ \isakeyword{shows}\isanewline
\ \ \ \ {\isachardoublequoteopen}{\isasymAnd}env{\isachardot}\ env{\isasymin}list{\isacharparenleft}M{\isacharparenright}\ {\isasymLongrightarrow}\ arity{\isacharparenleft}{\isasymphi}{\isacharparenright}{\isasymle}length{\isacharparenleft}env{\isacharparenright}\ {\isasymLongrightarrow}\ p\ {\isasymtturnstile}\ {\isasymphi}\ env\isanewline 
\ \ \ \ \ {\isasymLongrightarrow}\ r\ {\isasymtturnstile}\ {\isasymphi}\ env{\isachardoublequoteclose}\isanewline
%
%
\isacommand{using}\isamarkupfalse%
\ assms{\isacharparenleft}{\isadigit{2}}{\isacharparenright}
\end{isabelle}
%
The proof is divided in the 4 cases of definition of an element of $\formula$,
%
\begin{isabelle}
\isacommand{proof}\isamarkupfalse%
\ {\isacharparenleft}induct{\isacharparenright}\isanewline
\ \ \isacommand{case}\isamarkupfalse%
\ {\isacharparenleft}Member\ n\ m{\isacharparenright}\isanewline
\ \ \isacommand{then}\isamarkupfalse%
\isanewline
\ \ \dots
\isanewline
\ \ \isacommand{show}\isamarkupfalse%
\ {\isacharquery}case\ \isanewline
\ \ \ \ \isacommand{using}\isamarkupfalse%
\ Forces{\isacharunderscore}Member{\isacharbrackleft}of\ {\isacharunderscore}\ {\isachardoublequoteopen}nth{\isacharparenleft}n{\isacharcomma}env{\isacharparenright}{\isachardoublequoteclose}\ {\isachardoublequoteopen}nth{\isacharparenleft}m{\isacharcomma}env{\isacharparenright}{\isachardoublequoteclose}\ env\ n\ m{\isacharbrackright}\isanewline
\ \ \ \ \ \ strengthening{\isacharunderscore}mem{\isacharbrackleft}of\ p\ r\ {\isachardoublequoteopen}nth{\isacharparenleft}n{\isacharcomma}env{\isacharparenright}{\isachardoublequoteclose}\ {\isachardoublequoteopen}nth{\isacharparenleft}m{\isacharcomma}env{\isacharparenright}{\isachardoublequoteclose}{\isacharbrackright}\ \isacommand{by}\isamarkupfalse%
\ simp
\end{isabelle}
%
where the final step depends on previously proved
\isatt{strengthening{\isacharunderscore}mem} and the characterization of
$\forceisa$ for membership 


The case of equality is entirely analogous, and the \isatt{Nand} and
\isatt{Forall} cases are handled very simply.
%
\begin{isabelle}
\isacommand{next}\isamarkupfalse%
\isanewline
\ \ \isacommand{case}\isamarkupfalse%
\ {\isacharparenleft}Equal\ n\ m{\isacharparenright}\isanewline
\ \ \dots\isanewline
\isacommand{next}\isamarkupfalse%
\isanewline
\ \ \isacommand{case}\isamarkupfalse%
\ {\isacharparenleft}Nand\ {\isasymphi}\ {\isasympsi}{\isacharparenright}\isanewline
\ \ \isacommand{with}\isamarkupfalse%
\ assms\isanewline
\ \ \isacommand{show}\isamarkupfalse%
\ {\isacharquery}case\ \isanewline
\ \ \ \ \isacommand{using}\isamarkupfalse%
\ Forces{\isacharunderscore}Nand\ Transset{\isacharunderscore}intf{\isacharbrackleft}OF\ trans{\isacharunderscore}M\ {\isacharunderscore}\ P{\isacharunderscore}in{\isacharunderscore}M{\isacharbrackright}\ pair{\isacharunderscore}in{\isacharunderscore}M{\isacharunderscore}iff\isanewline
\ \ \ \ \ \ Transset{\isacharunderscore}intf{\isacharbrackleft}OF\ trans{\isacharunderscore}M\ {\isacharunderscore}\ leq{\isacharunderscore}in{\isacharunderscore}M{\isacharbrackright}\ leq{\isacharunderscore}transD\ \isacommand{by}\isamarkupfalse%
\ auto\isanewline
\isacommand{next}\isamarkupfalse%
\isanewline
\ \ \isacommand{case}\isamarkupfalse%
\ {\isacharparenleft}Forall\ {\isasymphi}{\isacharparenright}\isanewline
\ \ \isacommand{with}\isamarkupfalse%
\ assms\isanewline
\ \ \isacommand{have}\isamarkupfalse%
\ {\isachardoublequoteopen}p\ {\isasymtturnstile}\ {\isasymphi}\ {\isacharparenleft}{\isacharbrackleft}x{\isacharbrackright}\ {\isacharat}\ env{\isacharparenright}{\isachardoublequoteclose}\ \isakeyword{if}\ {\isachardoublequoteopen}x{\isasymin}M{\isachardoublequoteclose}\ \isakeyword{for}\ x\isanewline
\ \ \ \ \isacommand{using}\isamarkupfalse%
\ that\ Forces{\isacharunderscore}Forall\ \isacommand{by}\isamarkupfalse%
\ simp\isanewline
\ \ \isacommand{with}\isamarkupfalse%
\ \underline{Forall}\ \isanewline
\ \ \isacommand{have}\isamarkupfalse%
\ {\isachardoublequoteopen}r\ {\isasymtturnstile}\ {\isasymphi}\ {\isacharparenleft}{\isacharbrackleft}x{\isacharbrackright}\ {\isacharat}\ env{\isacharparenright}{\isachardoublequoteclose}\ \isakeyword{if}\ {\isachardoublequoteopen}x{\isasymin}M{\isachardoublequoteclose}\ \isakeyword{for}\ x\isanewline
\ \ \ \ \isacommand{using}\isamarkupfalse%
\ that\ pred{\isacharunderscore}le{\isadigit{2}}\ \isacommand{by}\isamarkupfalse%
\ {\isacharparenleft}simp{\isacharparenright}\isanewline
\ \ \isacommand{with}\isamarkupfalse%
\ assms\ \underline{Forall}\isanewline
\ \ \isacommand{show}\isamarkupfalse%
\ {\isacharquery}case\ \isanewline
\ \ \ \ \isacommand{using}\isamarkupfalse%
\ Forces{\isacharunderscore}Forall\ \isacommand{by}\isamarkupfalse%
\ simp\isanewline
\isacommand{qed}\isamarkupfalse
\end{isabelle}
%
It can be noted that the inductive hypothesis
gets used in the last case (underlined here as
\isatt{\underline{Forall}}), but not in the case for \isatt{Nand}.



%%% Local Variables: 
%%% mode: latex
%%% TeX-master: "forcing_in_isabelle_zf"
%%% ispell-local-dictionary: "american"
%%% End: 

%% 
%% \section{Example of proper extension}
\label{sec:example-proper-extension}

Even when the axioms of $\ZFC$ are proved in the generic extension,
one cannot claim that the magic of forcing has taken place unless one
is able to provide some \emph{proper} extension with the \emph{same
ordinals}. After all, one is assuming from starters a model $M$ of $\ZFC$,
and in some trivial cases $M[G]$ might end up to be exactly $M$; this
is where \emph{proper} enters the stage. But, for instance, in the
presence of large cardinals, a model $M'\supsetneq M$ might be an
end-extension of $M$ ---this is were we ask the two models to have the
same ordinals, the same \emph{height}. 

Three theory files contain the relevant
results. \verb|Ordinals_In_MG.thy| shows, using the closure of $M$
under ranks, that $M$ and $M[G]$ share the same ordinals (actually,
ranks of elements of $M[G]$ are bounded by the ranks of their names in
$M$):
\begin{isabelle}
\isacommand{lemma}\isamarkupfalse%
\ rank{\isacharunderscore}val{\isacharcolon}\ {\isachardoublequoteopen}rank{\isacharparenleft}val{\isacharparenleft}G{\isacharcomma}x{\isacharparenright}{\isacharparenright}\ {\isasymle}\ rank{\isacharparenleft}x{\isacharparenright}{\isachardoublequoteclose}\isanewline
\isacommand{lemma}\isamarkupfalse%
\ Ord{\isacharunderscore}MG{\isacharunderscore}iff{\isacharcolon}\isanewline
\ \ \isakeyword{assumes}\ {\isachardoublequoteopen}Ord{\isacharparenleft}{\isasymalpha}{\isacharparenright}{\isachardoublequoteclose}\ \isanewline
\ \ \isakeyword{shows}\ {\isachardoublequoteopen}{\isasymalpha}\ {\isasymin}\ M\ {\isasymlongleftrightarrow}\ {\isasymalpha}\ {\isasymin}\ M{\isacharbrackleft}G{\isacharbrackright}{\isachardoublequoteclose}
\end{isabelle}

To prove these results, we found it useful to formalize induction over
the relation \isatt{ed}$(x,y) \defi x\in\dom(y)$, which is key
to arguments involving names.
\begin{isabelle}
\isacommand{lemma}\isamarkupfalse%
\ ed{\isacharunderscore}induction{\isacharcolon}\isanewline
\ \ \isakeyword{assumes}\ {\isachardoublequoteopen}{\isasymAnd}x{\isachardot}\ {\isasymlbrakk}{\isasymAnd}y{\isachardot}\ \ ed{\isacharparenleft}y{\isacharcomma}x{\isacharparenright}\ {\isasymLongrightarrow}\ Q{\isacharparenleft}y{\isacharparenright}\ {\isasymrbrakk}\ {\isasymLongrightarrow}\ Q{\isacharparenleft}x{\isacharparenright}{\isachardoublequoteclose}\isanewline
\ \ \isakeyword{shows}\ {\isachardoublequoteopen}Q{\isacharparenleft}a{\isacharparenright}{\isachardoublequoteclose}
\end{isabelle}

\verb|Succession_Poset.thy| contains our first example of a poset
that interprets the locale
\isatt{forcing{\isacharunderscore}notion}, essentially the notion for
adding one Cohen real. It is the set $2^{<\om}$ of all finite binary
sequences partially  ordered by reverse inclusion.
The sufficient condition for a proper extension is that
the forcing poset is \emph{separative}: every element has two
incompatible (\isatt{{\isasymbottom}s}) extensions. Here,
\isatt{seq{\isacharunderscore}upd{\isacharparenleft}f{\isacharcomma}x{\isacharparenright}}
adds \isatt{x} to the end of the sequence \isatt{f}.

\begin{isabelle}
\isacommand{lemma}\isamarkupfalse%
\ seqspace{\isacharunderscore}separative{\isacharcolon}\isanewline
\ \ \isakeyword{assumes}\ {\isachardoublequoteopen}f{\isasymin}{\isadigit{2}}{\isacharcircum}{\isacharless}{\isasymomega}{\isachardoublequoteclose}\isanewline
\ \ \isakeyword{shows}\ {\isachardoublequoteopen}seq{\isacharunderscore}upd{\isacharparenleft}f{\isacharcomma}{\isadigit{0}}{\isacharparenright}\ {\isasymbottom}s\ seq{\isacharunderscore}upd{\isacharparenleft}f{\isacharcomma}{\isadigit{1}}{\isacharparenright}{\isachardoublequoteclose}
\end{isabelle}
 
We prove in the theory file \verb|Proper_Extension.thy| that, in
general, every separative forcing notion gives rise to a proper
extension.

%%% Local Variables: 
%%% mode: latex
%%% TeX-master: "forcing_in_isabelle_zf"
%%% ispell-local-dictionary: "american"
%%% End: 

%% 
%% \section{The axioms of replacement and choice}
\label{sec:axioms-replacement-choice}

In \cite{2019arXiv190103313G} we proved that any generic extension
preserves the satisfaction of almost all the axioms, including the separation scheme
(we adapted those proofs to the current statement of the axiom
schemes). Our proofs that Replacement and choice hold in every generic
extension depend on further relativized concepts and closure properties.

\subsection{Replacement}

The proof of the Replacement Axiom scheme in $M[G]$ in Kunen uses the
Reflection Principle relativized to $M$. We took an alternative
pathway, following Neeman \cite{neeman-course}. In his course notes,
he uses the relativization of the cumulative hierarchy of sets. 

The
family of all sets of rank less than $\alpha$ is called
\isatt{Vset}$(\alpha)$ in Isabelle/ZF. We showed, in the theory file
\verb|Relative_Univ.thy|
 the following
relativization and closure results concerning this function, for a
class $M$ satisfying the locale \isatt{M{\isacharunderscore}eclose}
plus the Powerset Axiom and four instances of replacement.
%
\begin{isabelle}
\isacommand{lemma}\isamarkupfalse%
\ Vset{\isacharunderscore}abs{\isacharcolon}\ {\isachardoublequoteopen}{\isasymlbrakk}\ M{\isacharparenleft}i{\isacharparenright}{\isacharsemicolon}\ M{\isacharparenleft}V{\isacharparenright}{\isacharsemicolon}\ Ord{\isacharparenleft}i{\isacharparenright}\ {\isasymrbrakk}\ {\isasymLongrightarrow}\ \isanewline
\ \ \ \ \ \ \ \  \ \  \ \ \ \ \ \ \ \ \ \ is{\isacharunderscore}Vset{\isacharparenleft}M{\isacharcomma}i{\isacharcomma}V{\isacharparenright}\ {\isasymlongleftrightarrow}\ V\ {\isacharequal}\ {\isacharbraceleft}x{\isasymin}Vset{\isacharparenleft}i{\isacharparenright}{\isachardot}\ M{\isacharparenleft}x{\isacharparenright}{\isacharbraceright}{\isachardoublequoteclose}
\end{isabelle}
\begin{isabelle}
\isacommand{lemma}\isamarkupfalse%
\ Vset{\isacharunderscore}closed{\isacharcolon}\ {\isachardoublequoteopen}{\isasymlbrakk}\ M{\isacharparenleft}i{\isacharparenright}{\isacharsemicolon}\ Ord{\isacharparenleft}i{\isacharparenright}\ {\isasymrbrakk}\ {\isasymLongrightarrow}\ M{\isacharparenleft}{\isacharbraceleft}x{\isasymin}Vset{\isacharparenleft}i{\isacharparenright}{\isachardot}\ M{\isacharparenleft}x{\isacharparenright}{\isacharbraceright}{\isacharparenright}{\isachardoublequoteclose}
\end{isabelle}
We also have the basic result
\begin{isabelle}
\isacommand{lemma}\isamarkupfalse%
\ M{\isacharunderscore}into{\isacharunderscore}Vset{\isacharcolon}\isanewline
\ \ \isakeyword{assumes}\ {\isachardoublequoteopen}M{\isacharparenleft}a{\isacharparenright}{\isachardoublequoteclose}\isanewline
\ \ \isakeyword{shows}\ {\isachardoublequoteopen}{\isasymexists}i{\isacharbrackleft}M{\isacharbrackright}{\isachardot}\ {\isasymexists}V{\isacharbrackleft}M{\isacharbrackright}{\isachardot}\ ordinal{\isacharparenleft}M{\isacharcomma}i{\isacharparenright}\ {\isasymand}\ is{\isacharunderscore}Vfrom{\isacharparenleft}M{\isacharcomma}{\isadigit{0}}{\isacharcomma}i{\isacharcomma}V{\isacharparenright}\ {\isasymand}\ a{\isasymin}V{\isachardoublequoteclose}
\end{isabelle}
stating that $M$ is included in 
$\union\{\isatt{Vset}(\alpha) : \alpha\in M\}$ (it's actually equal).

For the proof of the Replacement Axiom, we assume that $\phi$ is
functional in its first two variables when interpreted in $M[G]$ and
the first ranges over the domain \isatt{c}${}\in M[G]$. Then we show
that the collection of
all values of the second variable, when the first ranges over
\isatt{c}, belongs to $M[G]$:
%
\begin{isabelle}
\isacommand{lemma}\isamarkupfalse%
\ Replace{\isacharunderscore}sats{\isacharunderscore}in{\isacharunderscore}MG{\isacharcolon}\isanewline
\ \ \isakeyword{assumes}\isanewline
\ \ \ \ {\isachardoublequoteopen}c{\isasymin}M{\isacharbrackleft}G{\isacharbrackright}{\isachardoublequoteclose}\ {\isachardoublequoteopen}env\ {\isasymin}\ list{\isacharparenleft}M{\isacharbrackleft}G{\isacharbrackright}{\isacharparenright}{\isachardoublequoteclose}\isanewline
\ \ \ \ {\isachardoublequoteopen}{\isasymphi}\ {\isasymin}\ formula{\isachardoublequoteclose}\ {\isachardoublequoteopen}arity{\isacharparenleft}{\isasymphi}{\isacharparenright}\ {\isasymle}\ {\isadigit{2}}\ {\isacharhash}{\isacharplus}\ length{\isacharparenleft}env{\isacharparenright}{\isachardoublequoteclose}\isanewline
\ \ \ \ {\isachardoublequoteopen}univalent{\isacharparenleft}{\isacharhash}{\isacharhash}M{\isacharbrackleft}G{\isacharbrackright}{\isacharcomma}\ c{\isacharcomma}\ {\isasymlambda}x\ v{\isachardot}\ {\isacharparenleft}M{\isacharbrackleft}G{\isacharbrackright}{\isacharcomma}\ {\isacharbrackleft}x{\isacharcomma}v{\isacharbrackright}{\isacharat}env\ {\isasymTurnstile}\ {\isasymphi}{\isacharparenright}{\isacharparenright}{\isachardoublequoteclose}\isanewline
\ \ \isakeyword{shows}\isanewline
\ \ \ \ {\isachardoublequoteopen}{\isacharbraceleft}v{\isachardot}\ x{\isasymin}c{\isacharcomma}\ v{\isasymin}M{\isacharbrackleft}G{\isacharbrackright}\ {\isasymand}\ {\isacharparenleft}M{\isacharbrackleft}G{\isacharbrackright}{\isacharcomma}\ {\isacharbrackleft}x{\isacharcomma}v{\isacharbrackright}{\isacharat}env\ {\isasymTurnstile}\ {\isasymphi}{\isacharparenright}{\isacharbraceright}\ {\isasymin}\ M{\isacharbrackleft}G{\isacharbrackright}{\isachardoublequoteclose}
\end{isabelle}
%
From this, the satisfaction of the Replacement Axiom in $M[G]$ follows
very easily.

The proof of the previous lemma, following Neeman, proceeds as usual
by turning an argument concerning elements of $M[G]$ to one involving
names lying in $M$, and connecting both worlds by using the forcing
theorems. In the case at hand, by functionality of $\phi$ we know that
for every $x\in c\cap M[G]$ there exists exactly one $v\in M[G]$ such
that
$M[G], [x,v]\mathbin{@} \mathit{env} \models \phi$. Now,
given a name $\pi'\in M$ for $c$, every name of an element of $c$
belongs to $\pi\defi \dom(\pi')\times \PP$, which is easily seen to be
in $M$. We will use $\pi$ to be the domain in an application of the
Replacement Axiom in $M$. But now, obviously, we have lost
functionality since there are many names $\dot v\in M$ for a fixed $v$
in $M[G]$. To solve this issue, for each $\rho p \defi\lb\rho,p\rb\in
\pi$ we calculate the
minimum rank of some $\tau\in M$ such that 
$p\forces \phi(\rho,\tau,\dots)$ if there is one, or $0$ otherwise. By
Replacement in $M$, we can show that the supremum \isatt{?sup} of these ordinals
belongs to $M$ and we can construct a \isatt{?bigname} $\defi$ 
\isatt{{\isacharbraceleft}x{\isasymin}Vset{\isacharparenleft}{\isacharquery}sup{\isacharparenright}{\isachardot}\ x\ {\isasymin}\
}$M$\isatt{{\isacharbraceright}\ {\isasymtimes}\ {\isacharbraceleft}one{\isacharbraceright}}
whose interpretation by (any generic) $G$ will include all possible elements
as $v$ above.

The previous calculation required some absoluteness and closure
results regarding the minimum ordinal binder, \isatt{Least}$(Q)$, also
denoted $\mu x. Q(x)$, that can be found in the theory file
\verb|Least.thy|.

\subsection{Choice}
A first important observation is that the proof of $\AC$ in $M[G]$
only requires the assumption that $M$ satisfies (a finite fragment of)
$\ZFC$. There is no need to invoke Choice in the metatheory.

Although our previous version of the development used $\AC$, that was
only needed to show the Rasiowa-Sikorski Lemma (RSL) for
arbitrary posets. We have modularized the proof of the latter
and now the version for countable posets that we use to show the
existence of generic filters
does not require Choice (as it is well known). We also bundled the
full RSL along with our implementation of the principle of dependent
choices in an independent branch of the dependency graph, which is the
only place where the theory \texttt{ZF.AC} is invoked.

Our statement of the Axiom of Choice is the one preferred for
arguments involving transitive classes satisfying $\ZF$:
%
\begin{center}
\isatt{{\isasymforall}x{\isacharbrackleft}M{\isacharbrackright}{\isachardot}\ {\isasymexists}a{\isacharbrackleft}M{\isacharbrackright}{\isachardot}\ {\isasymexists}f{\isacharbrackleft}M{\isacharbrackright}{\isachardot}\ ordinal{\isacharparenleft}M{\isacharcomma}a{\isacharparenright}\ {\isasymand}\ surjection{\isacharparenleft}M{\isacharcomma}a{\isacharcomma}x{\isacharcomma}f{\isacharparenright}}
\end{center}
%
The Simplifier is able to show automatically that this
statement is equivalent to the next one, in which the real notions of
ordinal and surjection appear:
%
\begin{center}
\isatt{{\isasymforall}x{\isacharbrackleft}M{\isacharbrackright}{\isachardot}\ {\isasymexists}a{\isacharbrackleft}M{\isacharbrackright}{\isachardot}\ {\isasymexists}f{\isacharbrackleft}M{\isacharbrackright}{\isachardot}\ Ord{\isacharparenleft}a{\isacharparenright}\ {\isasymand}\ f\ {\isasymin}\ surj{\isacharparenleft}a{\isacharcomma}x{\isacharparenright}}
\end{center}

As with the forcing axioms, the proof of $\AC$ in $M[G]$ follows the pattern of Kunen
\cite[IV.2.27]{kunen2011set} and is rather
straightforward; the only complicated technical point being to show
that the relevant name belongs to $M$. We assume that \isatt{a}${}\neq\emptyset$
belongs to $M[G]$ and has a name $\tau\in M$. By $\AC$ in $M$, there
is a surjection \isatt{s} from an ordinal $\alpha\in M$ ($\subseteq M[G]$) onto
$\dom(\tau)$. Now
%
\begin{center}
\isatt{{\isacharbraceleft}opair{\isacharunderscore}name{\isacharparenleft}check{\isacharparenleft}{\isasymbeta}{\isacharparenright}{\isacharcomma}s{\isacharbackquote}{\isasymbeta}{\isacharparenright}{\isachardot}\ {\isasymbeta}{\isasymin}{\isasymalpha}{\isacharbraceright}\ {\isasymtimes}\ {\isacharbraceleft}one{\isacharbraceright}}
\end{center}
%
is a name for a function \isatt{f} with domain $\alpha$ such that \isatt{a}
is included in its range, and where
\isatt{opair{\isacharunderscore}name}$(\sig,\rho)$ is a name for the
ordered pair $\lb\val(G,\sig),\val(G,\rho)\rb$. From this, $\AC$ in
$M[G]$ follows easily.

\subsection{The main theorem}
With all these elements in place, we are able to transcript the main
theorem of our formalization:
\begin{isabelle}
\isacommand{theorem}\isamarkupfalse%
\ extensions{\isacharunderscore}of{\isacharunderscore}ctms{\isacharcolon}\isanewline
\ \ \isakeyword{assumes}\ \isanewline
\ \ \ \ {\isachardoublequoteopen}M\ {\isasymapprox}\ nat{\isachardoublequoteclose}\ {\isachardoublequoteopen}Transset{\isacharparenleft}M{\isacharparenright}{\isachardoublequoteclose}\ {\isachardoublequoteopen}M\ {\isasymTurnstile}\ ZF{\isachardoublequoteclose}\isanewline
\ \ \isakeyword{shows}\ \isanewline
\ \ \ \ {\isachardoublequoteopen}{\isasymexists}N{\isachardot}\ \isanewline
\ \ \ \ \ \ M\ {\isasymsubseteq}\ N\ {\isasymand}\ N\ {\isasymapprox}\ nat\ {\isasymand}\ Transset{\isacharparenleft}N{\isacharparenright}\ {\isasymand}\ N\ {\isasymTurnstile}\ ZF\ {\isasymand}\ M{\isasymnoteq}N\ {\isasymand}\isanewline
\ \ \ \ \ \ {\isacharparenleft}{\isasymforall}{\isasymalpha}{\isachardot}\ Ord{\isacharparenleft}{\isasymalpha}{\isacharparenright}\ {\isasymlongrightarrow}\ {\isacharparenleft}{\isasymalpha}\ {\isasymin}\ M\ {\isasymlongleftrightarrow}\ {\isasymalpha}\ {\isasymin}\ N{\isacharparenright}{\isacharparenright}\ {\isasymand}\isanewline
\ \ \ \ \ \ {\isacharparenleft}M{\isacharcomma}\ {\isacharbrackleft}{\isacharbrackright}{\isasymTurnstile}\ AC\ {\isasymlongrightarrow}\ N\ {\isasymTurnstile}\ ZFC{\isacharparenright}{\isachardoublequoteclose}
\end{isabelle}
Here, \isatt{\isasymapprox} stands for equipotency, \isatt{nat} is the
set of natural numbers, and the predicate 
\isatt{Transset} indicates transitivity; and as usual, \isatt{AC}
denotes the Axiom of Choice, and \isatt{ZF} and \isatt{ZFC} the
corresponding subsets of \isatt{formula}.

%%% Local Variables: 
%%% mode: latex
%%% TeX-master: "forcing_in_isabelle_zf"
%%% ispell-local-dictionary: "american"
%%% End: 

%% 

\section{Related work}
\label{sec:related-work}

\textbf{Reviewer's comments}
{\it
  \begin{itemize}
  \item There, it would be appropriate to contrast what was done in
    Paulson's work on constructibility with the current work on forcing.
  \item More to the point, the recent work by Han and van Doorn on
    forcing in Lean deserves more discussion.  They have gone further
    than the current authors, having proved the independence of the
    continuum hypothesis.  They prefer Boolean-valued models as being
    more direct in use than the authors' countable transitive models.
    \begin{itemize}
    \item Readers will want to know whether the type-theoretic approach
      is better/worse/just different than using Isabelle/ZF, and
    \item are there any benefits to the ctm approach?
    \item Is the type-theory encoding of ZF really accurate?
    \item How about comparing proofs of equivalent statements in the two
      approaches for length and readability?
    \end{itemize}
  \end{itemize}
}

To the best of our knowledge, all of the previous works in
formalization of the method 
of forcing have been done in different variants of type theory, and
none of them uses the ctm approach. The
most important is the recent one by 
Han and van Doorn
\cite{han_et_al:LIPIcs:2019:11074,DBLP:conf/cpp/HanD20}, which includes
a formalization of the independence of $\CH$ by means
the Boolean-valued approach to forcing, using the Lean
proof assistant \cite{DBLP:conf/cade/MouraKADR15}.


\begin{itemize}
\item The consistency strength of Lean requires infinitely
  many inaccessibles. More precisely, let Lean$_n$ be the theory of
  CiC foundations of Lean restricted to $n$ type universes.  Carneiro
  \cite{carneiro-ms-thesis}, proved the consistency of Lean$_n$ from
  $\ZFC$ plus the existence of $n$ inaccessible
  cardinals. It is also reported in \cite{carneiro-ms-thesis} that
  Werner's results in \cite{10.5555/645869.668660} can be adapted to
  show that Lean$_{n+2}$ proves the consistency of the latter theory. 

  On the other hand, Isabelle's \emph{Pure} is based on
  ``intuitionistic higher order logic.'' In Paulson
  \cite{Paulson1989} it is proved that \emph{Pure} is sound for
  intuitionistic first order logic, thus it does not add any strength
  to it. On top of this, the axiomatization of Isabelle/ZF results in
  a system equiconsistent with $\ZFC$. Our running assumption, that of
  the existence of a countable transitive model, is considerably
  weaker (directly and consistency-wise) than the existence of a
  single inaccesible cardinal. In that sense, directly obtain
  unprovability results in first order logic, the meta theoretic
  machinery used to obtain them is far heavier than the one we use to
  operate model-theoretically.
  %
\item We may discuss in finer detail the shape of the axioms of
  Isabelle/ZF. It is perhaps more correct to say it is an
  notational variant of NBG set theory, because the schemes of
  Replacement and Separation feature higher order (free) variables
  playing the role of formula variables. It can't be proved that the
  axioms thus written correspond to first order sentences. This is the
  reason that our relativized versions only apply to set models, where
  we can restrict the formula variables to predicates that actually
  come from first order variables. In that sense, the axioms of the
  locale \isatt{M{\isacharunderscore}ZF} correspond faithfully to the
  $ZF$ axioms.
\item \fbox{\bf take care of repetitions} In our opinion, one of the
  main benefits of using transitive models is that many fundamental
  notions are absolute and thus the many statements can be interpreted
  transparently. It also provides a very concrete way to understand
  generic objects: as sets that (in the non trivial case) are provably
  not in the original model; this dispells any mystical feel around
  this concept (contrary to the case when the ground model is the
  universe of all sets). In addition, two-valued semantics is
  closer to our intuition ($\leftarrow$ revise).
\end{itemize}
%%% Local Variables: 
%%% mode: latex
%%% TeX-master: "forcing_in_isabelle_zf"
%%% ispell-local-dictionary: "american"
%%% End: 


\section{Some lessons}\label{sec:lessons}

We want to finish this report by gathering some of the conclusions we
reached after the experience of formalizing the basics of forcing in a
proof assistant.

\subsection{Aims of a formalization and planning}

We believe that in every project of formalization of mathematics,
there is a tension between the haste to verify the target results and
the need to obtain a readable, albeit extremely detailed, corpus of
statements and proofs. This tension is mirrored in two differents
purposes of formalization: Developing new mathematics from scratch and
producing verified results en route to this, versus verifying and
documenting material that has already been produced on paper.

Our present project clearly belongs to the second category, so we
prioritized trying to obtain formal proofs that mimicked standard
prose (the highlight being the sample proof in
Section~\ref{sec:sample-formal-proof}). We feel that the Isar language
provided with Isabelle has the right balance between elegance and
efficacy. Another crucial aspect to achieve this goal is the level of
detail of the blueprint for the formalization. We must however confess
that we learned many of the subtleties of Isabelle in the making, and
many engineering decisions were also taken before it was clear the
precise way things would develop in the future.

A similar experience, but on an opposite side of the formalization
spectrum happened to the Liquid Tensor Project as described by Scholze
in \cite{LTE2021}. People involved in the formalization simply pushed
their way to reach the summit, formalizing lemma after lemma. They
actually wrote the blueprint for that formalization \emph{afterwards}
it was complete! From time to time, we were also frenziedly trying to
get the results formalized, going beyond what we had planned.

As a result from this, some design choices that seemed reasonable at
first were proved to be inconvenient. For instance, we should had
better used predicates (of type $\tyi\fun\tyi\fun\tyo$) for the
forcing posets' order relations; this is the way they
are presented in the \session{Delta\_System\_Lemma} session. A similar
problem is that we require the forcing poset to be an element of $M$,
so the present infrastructure does not allow class forcing out of the
box. (The latter change seems to be rather straightforward, but the
former does not.)

Nearly the final stage of the project, we decided to go for the minimal
set of definitions and versions of lemmas that were needed to obtain
our target results. For example, we only proved the Delta System Lemma
for $\aleph_1$-sized families (thus limiting us to the plain ccc) and
showed preservation of sequences by considering countably closed
forcings (in fact, we formalized the bare minimum requirement of being
$(<\omega+1)$-closed). In doing this we went against the tried and
true advice that one should formalize the most general version of the
results available.

\subsection{Bureaucracy and scale factors}

\begin{enumerate}
\item Bureaucracy vs ML programming.
\item The “math” was already formalized on 22 November 2020.
  We finished the last goal on 22 August 2021.
  (Update: 20 November 2021 \& 28 November 2021, for CH)
\item Missing: automation of closure of models under operations.
\item Missing: basic arithmetic for dealing with arities.
\end{enumerate}

\begin{enumerate}
\item It is extremely misleading when automatic tools (\isatt{simp}, \isatt{auto}, etc)
  stop working just because of the sheer size of the goal. Oftentimes,
  in math, we disregard scale issues but they must always be taken
  into account in CS.
\item Example: $\forceisa(0\in 1)$ is expandable,
  $\forceisa(\neg\neg  0\in 1)$ is not.
\item Example: Synthesis of $\forceisa$; could have been fully synthesized,
  but that was dirty “strategy”.
\item The know-how of computer scientists on this kind of engineering is
  very important
\end{enumerate}

\subsection{You might have formalized it, and still be wrong}
\begin{enumerate}
\item Example: restriction of relations.
\item Pollack, “Pollack consistency” by
  Wiedijk. Cf. \theory{Definitions\_Main} (thanks to discussions with
  Vidnyánszky). Opacity of automated proofs. 
\item Plot twist: You can be right without knowing. Intuition may drive proofs
  even if we are not working on what we believe we are.
\end{enumerate}

\subsection{Beware of the “Code fever”}\label{sec:beware-code-fever}
\begin{enumerate}
\item “We know that doing math is fun---formalization is like DRUGS”
\item Feeling of accomplishment after seeing your writings
  validated beyond reasonable doubt (v.g. cofinality).

\item One easily forgets about the “Power of the Board.”
\end{enumerate}

\subsection{The Devil's on the shortcuts}
\begin{enumerate}
\item
  Our proofs of the “definition of forces” (and many
  consequences) and of the lemma for “forcing a value” of function
  depend on the countability of the ground model. 
\item
  Density arguments (look for “TODO”, “general versions”).
\end{enumerate}

%%% Local Variables: 
%%% mode: latex
%%% TeX-master: "independence_ch_isabelle"
%%% ispell-local-dictionary: "american"
%%% End: 


\section{Conclusions and future work}
There are several technical milestones that have to be reached in the
course of a formalization of the theory of forcing. The first one, and most
obvious, is the bulk of set- and meta-theoretical concepts needed to work
with. This pushed us, in a sense,  into building on top of Isabelle/ZF,
since we know of no other development in set theory of such
depth (and breadth). In this paper we worked on setting the stage for the work with
generic extensions; in particular, this involves some purely mathematical
results, as the Rasiowa-Sikorski lemma. 

Other milestones in this formalization project
involve 
\begin{enumerate}
\item the definition
  of the forcing relation, 
\item proving the Fundamental Theorem of forcing
  (that relates truth in $M$ to that in $M[G]$), and 
\item using it to show
  that $M[G]\models \ZFC$. 
\end{enumerate}
The theory is very modular and this is
witnessed by the fact 
that the last goal does not depend on the proof of the Fundamental
Theorem nor on the definition of the forcing relation. Our next task
will be to obtain the last goal in that enumeration. 

To this end, we will develop an interface between Paulson's
relativization results and countable models of $\ZFC$. This will show
that every ctm $M$ is closed under well-founded recursion and, in
particular, that contains names for each of its
elements. Consequently, the proof of  $M\sbq M[G]$ will be
complete. A landmark will be to prove the Axiom Scheme
of Separation (the first that needs to use the machinery of forcing
nontrivially). As a part of the new formalization, we will provide
Isar versions of the longer applicative proofs presented in this work.

\ack{We'd like to thank the anonymous referees for reading the paper
  carefully and for their detailed and constructive criticism.}
%%% Local Variables:
%%% mode: latex
%%% ispell-local-dictionary: "american"
%%% TeX-master: "first_steps_into_forcing"
%%% End:

%
% ---- Bibliography ----
%
% BibTeX users should specify bibliography style 'splncs04'.
% References will then be sorted and formatted in the correct style.
%
%\bibliographystyle{splncs04}
\bibliographystyle{mi-estilo-else}
\bibliography{independence_ch_isabelle}

\end{document}

%%% Local Variables: 
%%% mode: latex
%%% ispell-local-dictionary: "american"
%%% End: 
