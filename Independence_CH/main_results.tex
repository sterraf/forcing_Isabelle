\section{Main achievements of the formalization}
\label{sec:main-achievements}

\subsection{A sufficient set of replacement instances}
\label{sec:repl-instances}

We isolated 22 instances of Replacement that are sufficient to force
$\CH$ or $\neg\CH$, which are enumerated below by the name of the
corresponding internalized first order formula. Many of these were already present in
relational form in the \session{ZF-Constructible} library.

The first 4 instances, collected in the subset
\isatt{instances1{\uscore}fms} of \formula, consist of basic
constructions:

\begin{itemize}
\item 2 instances for transitive closure: one to prove closure under
  iteration of $X\mapsto\union X$ and an auxiliary one used to show absoluteness.
\item 1 instance to define $\in$-rank.
  %
\item 1 instance to construct the cumulative hierarchy (rank initial segments).
\end{itemize}

The next 4 instances (gathered in \isatt{instances2{\uscore}fms})
are needed to set up
cardinal arithmetic in $M$:
\begin{itemize}
\item 2 instances for the definition of
  ordertypes;
\item 2 instances for Aleph: Replacement through $x\mapsto
  \otype(x)$ (for Hartogs' Theorem) and the well-founded recursion
  using it.
\end{itemize}

We also need a one extra replacement instance $\psi$ on $M$ for each
$\phi$ of the
previous ones to have them in $M[G]$:
\[
  \psi(x,\alpha,y_1,\dots,y_n) \defi \quine{\alpha = \min \bigl\{
    \beta \mid \exists\tau\in V_\beta.\  \mathit{snd}(x) \forces
    \phi\ [\mathit{fst}(x),\tau,y_1,\dots,y_n]\bigr\}}
\]
Here, $\mathit{fst}(\lb a,b\rb) = a$, $\mathit{snd}(\lb a,b\rb) = b$
(with default value $0$ for non pairs).
In our development, the mapping $\phi\mapsto\psi$ defined above is given by the
$\isatt{ground{\uscore}repl{\uscore}fm}$ function, and all ground replacement
instances appear in the locale \locale{M{\uscore}ZF3} and form the set
\isatt{instances3{\uscore}fms}. These are expressed using
the
\[
  \isatt{ground{\uscore}replacement{\uscore}assm}(M,\mathit{env},\phi)
\]
predicate
obtained by replacing $\phi$ by
$\isatt{ground{\uscore}repl{\uscore}fm}(\phi)$ in Eq.~(\ref{eq:replacement_assm_def}).

That makes 16 instances up to now. For the setup of forcing, we
require the following 3 instances, which form the set
\isatt{instances{\uscore}ground{\uscore}fms}:
%
\begin{itemize}
\item Well-founded recursion to define check-names.
  %
\item Well-founded recursion for the definition of forcing for atomic formulas.
  %
\item Replacement through $x\mapsto \lb x,\check{x}\rb$ (for the
  definition of $\punto{G}$).
  %
\end{itemize}
The proof of the $\Delta$-System Lemma requires 2 instances which form the set
\isatt{instances{\uscore}ground{\uscore}notCH{\uscore}fms}, that are
used for the recursive construction of sets using a choice function (as in the
construction of a wellorder of $X$ given a choice function on
$\Pow(X)$), and to show its absoluteness.

The $21$ formulas up to this point are collected into the set
\isatt{overhead{\uscore}notCH}, which is enough to
force $\neg\CH$. To force $\CH$, we required one further instance for
the absoluteness of the recursive construction in the proof of
Dependent Choices from $\AC$.
  
The particular choice of some of the instances above arose from
Paulson's architecture on which we based our development.
This applies every time
a locale from \session{ZF-Constructible} has to be
interpreted (\locale{M{\uscore}eclose} and
\locale{M{\uscore}ordertype}, respectively, for the “auxiliary” instances).
%% For instance, the first
%% instance required for the definition of relative ordertypes arises
%% from Paulson's \session{ZF-Constructible}.
% https://isabelle.in.tum.de/dist/library/ZF/ZF-Constructible/Rank.html#offset_1123..1139

On the other hand, we replaced the original proof of the
Schröder-Bernstein Theorem by Zermelo's one
\cite[Exr. x4.27]{moschovakis1994notes}, because the former required
at least one extra instance
% (\isatt{banach{\uscore}iterates{\uscore}fm})
arising from an iteration. We also managed to avoid 12 further
replacements by restructuring some of original theories in
\session{ZF-Constructible}, so these modifications are included as
part of our project.

It is to be noted that the proofs of the Forcing Theorems do not
require any extra replacement; actually, they only need the 7
instances appearing in \isatt{instances1{\uscore}fms} and
\isatt{instances{\uscore}ground{\uscore}fms}.  But this seems not be
the case for Separation, at least by inspecting our formalization:
More instances holding in $M$ are needed 
as the complexity of $\phi$ grows. One point where this is apparent is
in the proof of Theorem~\ref{th:forcing-thms}(\ref{item:truth-lemma}),
that appears as the \isatt{truth{\uscore}lemma} in our development; it
depends on \isatt{truth{\uscore}lemma'} and
\isatt{truth{\uscore}lemma{\uscore}Neg}, which explicitly invoke
\isatt{separation{\uscore}ax}. In any case, our intended grounds
(v.g., the transitive collapse of countable elementary submodels of a
rank initial segment $V_\alpha$ or an $H(\kappa)$) all satisfy full
Separation.


%-%-%-%-%-%-%-%-%-%-%-%-%-%-%-%-%-%-%-%-%-%-%-%-%-%-%-%-%-%-%-%-%
\subsection{Models for $\CH$ and its negation}
\label{sec:models-ch-negation}

The statements of the existence of models of $\ZFC + \neg\CH$ and of
$\ZFC + \CH$  appear in our formalization as follows:

\begin{isabelle}
\isacommand{corollary}\isamarkupfalse%
\ ctm{\isacharunderscore}{\kern0pt}ZFC{\isacharunderscore}{\kern0pt}imp{\isacharunderscore}{\kern0pt}ctm{\isacharunderscore}{\kern0pt}not{\isacharunderscore}{\kern0pt}CH{\isacharcolon}{\kern0pt}\isanewline
\ \ \isakeyword{assumes}\isanewline
\ \ \ \ {\isachardoublequoteopen}M\ {\isasymapprox}\ {\isasymomega}{\isachardoublequoteclose}\ {\isachardoublequoteopen}Transset{\isacharparenleft}{\kern0pt}M{\isacharparenright}{\kern0pt}{\isachardoublequoteclose}\ {\isachardoublequoteopen}M\ {\isasymTurnstile}\ ZFC{\isachardoublequoteclose}\isanewline
\ \ \isakeyword{shows}\isanewline
\ \ \ \ {\isachardoublequoteopen}{\isasymexists}N{\isachardot}{\kern0pt}\isanewline
\ \ \ \ \ \ M\ {\isasymsubseteq}\ N\ {\isasymand}\ N\ {\isasymapprox}\ {\isasymomega}\ {\isasymand}\ Transset{\isacharparenleft}{\kern0pt}N{\isacharparenright}{\kern0pt}\ {\isasymand}\ N\ {\isasymTurnstile}\ ZFC\ {\isasymunion}\ {\isacharbraceleft}{\kern0pt}{\isasymcdot}{\isasymnot}{\isasymcdot}CH{\isasymcdot}{\isasymcdot}{\isacharbraceright}{\kern0pt}\ {\isasymand}\isanewline
\ \ \ \ \ \ {\isacharparenleft}{\kern0pt}{\isasymforall}{\isasymalpha}{\isachardot}{\kern0pt}\ Ord{\isacharparenleft}{\kern0pt}{\isasymalpha}{\isacharparenright}{\kern0pt}\ {\isasymlongrightarrow}\ {\isacharparenleft}{\kern0pt}{\isasymalpha}\ {\isasymin}\ M\ {\isasymlongleftrightarrow}\ {\isasymalpha}\ {\isasymin}\ N{\isacharparenright}{\kern0pt}{\isacharparenright}{\kern0pt}{\isachardoublequoteclose}
\end{isabelle}

\begin{isabelle}
\isacommand{corollary}\isamarkupfalse%
\ ctm{\isacharunderscore}{\kern0pt}ZFC{\isacharunderscore}{\kern0pt}imp{\isacharunderscore}{\kern0pt}ctm{\isacharunderscore}{\kern0pt}CH{\isacharcolon}{\kern0pt}\isanewline
\ \ \isakeyword{assumes}\isanewline
\ \ \ \ {\isachardoublequoteopen}M\ {\isasymapprox}\ {\isasymomega}{\isachardoublequoteclose}\ {\isachardoublequoteopen}Transset{\isacharparenleft}{\kern0pt}M{\isacharparenright}{\kern0pt}{\isachardoublequoteclose}\ {\isachardoublequoteopen}M\ {\isasymTurnstile}\ ZFC{\isachardoublequoteclose}\isanewline
\ \ \isakeyword{shows}\isanewline
\ \ \ \ {\isachardoublequoteopen}{\isasymexists}N{\isachardot}{\kern0pt}\isanewline
\ \ \ \ \ \ M\ {\isasymsubseteq}\ N\ {\isasymand}\ N\ {\isasymapprox}\ {\isasymomega}\ {\isasymand}\ Transset{\isacharparenleft}{\kern0pt}N{\isacharparenright}{\kern0pt}\ {\isasymand}\ N\ {\isasymTurnstile}\ ZFC\ {\isasymunion}\ {\isacharbraceleft}{\kern0pt}{\isasymcdot}CH{\isasymcdot}{\isacharbraceright}{\kern0pt}\ {\isasymand}\isanewline
\ \ \ \ \ \ {\isacharparenleft}{\kern0pt}{\isasymforall}{\isasymalpha}{\isachardot}{\kern0pt}\ Ord{\isacharparenleft}{\kern0pt}{\isasymalpha}{\isacharparenright}{\kern0pt}\ {\isasymlongrightarrow}\ {\isacharparenleft}{\kern0pt}{\isasymalpha}\ {\isasymin}\ M\ {\isasymlongleftrightarrow}\ {\isasymalpha}\ {\isasymin}\ N{\isacharparenright}{\kern0pt}{\isacharparenright}{\kern0pt}{\isachardoublequoteclose}
\end{isabelle}
where $\approx$ is equipotence, and the predicate \isatt{Transset}
holds for
transitive sets. Both results are proved without using Choice.

As the excerpts indicate, these results are obtained as corollaries of
two theorems in which only a subset of the aforementioned
replacement instances are assumed of the ground model. We begin the
discussion of these stronger results by
considering extensions of ctms of fragments of $\ZF$.
\begin{isabelle}
\isacommand{theorem}\isamarkupfalse%
\ extensions{\isacharunderscore}{\kern0pt}of{\isacharunderscore}{\kern0pt}ctms{\isacharcolon}{\kern0pt}\isanewline
\ \ \isakeyword{assumes}\isanewline
\ \ \ \ {\isachardoublequoteopen}M\ {\isasymapprox}\ {\isasymomega}{\isachardoublequoteclose}\ {\isachardoublequoteopen}Transset{\isacharparenleft}{\kern0pt}M{\isacharparenright}{\kern0pt}{\isachardoublequoteclose}\isanewline
\ \ \ \ {\isachardoublequoteopen}M\ {\isasymTurnstile}\ {\isasymcdot}Z{\isasymcdot}\ {\isasymunion}\ {\isacharbraceleft}{\kern0pt}{\isasymcdot}Replacement{\isacharparenleft}{\kern0pt}p{\isacharparenright}{\kern0pt}{\isasymcdot}\ {\isachardot}{\kern0pt}\ p\ {\isasymin}\ overhead{\isacharbraceright}{\kern0pt}{\isachardoublequoteclose}\isanewline
\ \ \ \ {\isachardoublequoteopen}{\isasymPhi}\ {\isasymsubseteq}\ formula{\isachardoublequoteclose}\isanewline%
\ \ \ \ {\isachardoublequoteopen}M\ {\isasymTurnstile}\ {\isacharbraceleft}{\kern0pt}\ {\isasymcdot}Replacement{\isacharparenleft}{\kern0pt}ground{\isacharunderscore}{\kern0pt}repl{\isacharunderscore}{\kern0pt}fm{\isacharparenleft}{\kern0pt}{\isasymphi}{\isacharparenright}{\kern0pt}{\isacharparenright}{\kern0pt}{\isasymcdot}\ {\isachardot}{\kern0pt}\ {\isasymphi}\ {\isasymin}\ {\isasymPhi}{\isacharbraceright}{\kern0pt}{\isachardoublequoteclose}\isanewline
\ \ \isakeyword{shows}\isanewline
\ \ \ \ {\isachardoublequoteopen}{\isasymexists}N{\isachardot}{\kern0pt}\isanewline
\ \ \ \ \ \ M\ {\isasymsubseteq}\ N\ {\isasymand}\ N\ {\isasymapprox}\ {\isasymomega}\ {\isasymand}\ Transset{\isacharparenleft}{\kern0pt}N{\isacharparenright}{\kern0pt}\ {\isasymand}\ M{\isasymnoteq}N\ {\isasymand}\isanewline
\ \ \ \ \ \ {\isacharparenleft}{\kern0pt}{\isasymforall}{\isasymalpha}{\isachardot}{\kern0pt}\ Ord{\isacharparenleft}{\kern0pt}{\isasymalpha}{\isacharparenright}{\kern0pt}\ {\isasymlongrightarrow}\ {\isacharparenleft}{\kern0pt}{\isasymalpha}\ {\isasymin}\ M\ {\isasymlongleftrightarrow}\ {\isasymalpha}\ {\isasymin}\ N{\isacharparenright}{\kern0pt}{\isacharparenright}{\kern0pt}\ {\isasymand}\isanewline
\ \ \ \ \ \ {\isacharparenleft}{\kern0pt}{\isacharparenleft}{\kern0pt}M{\isacharcomma}{\kern0pt}\ {\isacharbrackleft}{\kern0pt}{\isacharbrackright}{\kern0pt}{\isasymTurnstile}\ {\isasymcdot}AC{\isasymcdot}{\isacharparenright}{\kern0pt}\ {\isasymlongrightarrow}\ N{\isacharcomma}{\kern0pt}\ {\isacharbrackleft}{\kern0pt}{\isacharbrackright}{\kern0pt}\ {\isasymTurnstile}\ {\isasymcdot}AC{\isasymcdot}{\isacharparenright}{\kern0pt}\ {\isasymand}\isanewline
\ \ \ \ \ \ N\ {\isasymTurnstile}\ {\isasymcdot}Z{\isasymcdot}\ {\isasymunion}\ {\isacharbraceleft}{\kern0pt}\ {\isasymcdot}Replacement{\isacharparenleft}{\kern0pt}{\isasymphi}{\isacharparenright}{\kern0pt}{\isasymcdot}\ {\isachardot}{\kern0pt}\ {\isasymphi}\ {\isasymin}\ {\isasymPhi}{\isacharbraceright}{\kern0pt}{\isachardoublequoteclose}
\end{isabelle}

Here, the 7-element set \isatt{overhead} is enough to construct a proper
extension. It is  the union of
\isa{instances{\isadigit{1}}{\isacharunderscore}{\kern0pt}fms} and
\isa{instances{\isacharunderscore}{\kern0pt}ground{\isacharunderscore}{\kern0pt}fms}.
Also,
\isatt{{\isasymcdot}Z{\isasymcdot}} denotes Zermelo set theory. In the
next theorem, the relevant set of formulas is
\isatt{overhead{\isacharunderscore}{\kern0pt}notCH}, defined above in
Section~\ref{sec:repl-instances}, and \isatt{ZC} denotes Zermelo set
theory plus Choice:

\begin{isabelle}
\isacommand{theorem}\isamarkupfalse%
\ ctm{\isacharunderscore}{\kern0pt}of{\isacharunderscore}{\kern0pt}not{\isacharunderscore}{\kern0pt}CH{\isacharcolon}{\kern0pt}\isanewline
\ \ \isakeyword{assumes}\isanewline
\ \ \ \ {\isachardoublequoteopen}M\ {\isasymapprox}\ {\isasymomega}{\isachardoublequoteclose}\ {\isachardoublequoteopen}Transset{\isacharparenleft}{\kern0pt}M{\isacharparenright}{\kern0pt}{\isachardoublequoteclose}\isanewline
\ \ \ \ {\isachardoublequoteopen}M\ {\isasymTurnstile}\ ZC\ {\isasymunion}\ {\isacharbraceleft}{\kern0pt}{\isasymcdot}Replacement{\isacharparenleft}{\kern0pt}p{\isacharparenright}{\kern0pt}{\isasymcdot}\ {\isachardot}{\kern0pt}\ p\ {\isasymin}\ overhead{\isacharunderscore}{\kern0pt}notCH{\isacharbraceright}{\kern0pt}{\isachardoublequoteclose}\isanewline
\ \ \ \ {\isachardoublequoteopen}{\isasymPhi}\ {\isasymsubseteq}\ formula{\isachardoublequoteclose}\isanewline
\ \ \ \ {\isachardoublequoteopen}M\ {\isasymTurnstile}\ {\isacharbraceleft}{\kern0pt}\ {\isasymcdot}Replacement{\isacharparenleft}{\kern0pt}ground{\isacharunderscore}{\kern0pt}repl{\isacharunderscore}{\kern0pt}fm{\isacharparenleft}{\kern0pt}{\isasymphi}{\isacharparenright}{\kern0pt}{\isacharparenright}{\kern0pt}{\isasymcdot}\ {\isachardot}{\kern0pt}\ {\isasymphi}\ {\isasymin}\ {\isasymPhi}{\isacharbraceright}{\kern0pt}{\isachardoublequoteclose}\isanewline
\ \ \isakeyword{shows}\isanewline
\ \ \ \ {\isachardoublequoteopen}{\isasymexists}N{\isachardot}{\kern0pt}\isanewline
\ \ \ \ \ \ M\ {\isasymsubseteq}\ N\ {\isasymand}\ N\ {\isasymapprox}\ {\isasymomega}\ {\isasymand}\ Transset{\isacharparenleft}{\kern0pt}N{\isacharparenright}{\kern0pt}\ {\isasymand}\isanewline
\ \ \ \ \ \ N\ {\isasymTurnstile}\ ZC\ {\isasymunion}\ {\isacharbraceleft}{\kern0pt}{\isasymcdot}{\isasymnot}{\isasymcdot}CH{\isasymcdot}{\isasymcdot}{\isacharbraceright}{\kern0pt}\ {\isasymunion}\ {\isacharbraceleft}{\kern0pt}\ {\isasymcdot}Replacement{\isacharparenleft}{\kern0pt}{\isasymphi}{\isacharparenright}{\kern0pt}{\isasymcdot}\ {\isachardot}{\kern0pt}\ {\isasymphi}\ {\isasymin}\ {\isasymPhi}{\isacharbraceright}{\kern0pt}\ {\isasymand}\isanewline
\ \ \ \ \ \ {\isacharparenleft}{\kern0pt}{\isasymforall}{\isasymalpha}{\isachardot}{\kern0pt}\ Ord{\isacharparenleft}{\kern0pt}{\isasymalpha}{\isacharparenright}{\kern0pt}\ {\isasymlongrightarrow}\ {\isacharparenleft}{\kern0pt}{\isasymalpha}\ {\isasymin}\ M\ {\isasymlongleftrightarrow}\ {\isasymalpha}\ {\isasymin}\ N{\isacharparenright}{\kern0pt}{\isacharparenright}{\kern0pt}{\isachardoublequoteclose}
\end{isabelle}

Finally, \isatt{overhead{\isacharunderscore}{\kern0pt}CH} is the union
of \isatt{overhead{\isacharunderscore}{\kern0pt}CH} with the $\DC$
instance \isatt{replacement{\isacharunderscore}{\kern0pt}dcwit{\isacharunderscore}{\kern0pt}repl{\isacharunderscore}{\kern0pt}body{\isacharunderscore}{\kern0pt}fm}:
\begin{isabelle}
\isacommand{theorem}\isamarkupfalse%
\ ctm{\isacharunderscore}{\kern0pt}of{\isacharunderscore}{\kern0pt}CH{\isacharcolon}{\kern0pt}\isanewline
\ \ \isakeyword{assumes}\isanewline
\ \ \ \ {\isachardoublequoteopen}M\ {\isasymapprox}\ {\isasymomega}{\isachardoublequoteclose}\ {\isachardoublequoteopen}Transset{\isacharparenleft}{\kern0pt}M{\isacharparenright}{\kern0pt}{\isachardoublequoteclose}\isanewline
\ \ \ \ {\isachardoublequoteopen}M\ {\isasymTurnstile}\ ZC\ {\isasymunion}\ {\isacharbraceleft}{\kern0pt}{\isasymcdot}Replacement{\isacharparenleft}{\kern0pt}p{\isacharparenright}{\kern0pt}{\isasymcdot}\ {\isachardot}{\kern0pt}\ p\ {\isasymin}\ overhead{\isacharunderscore}{\kern0pt}CH{\isacharbraceright}{\kern0pt}{\isachardoublequoteclose}\isanewline
\ \ \ \ {\isachardoublequoteopen}{\isasymPhi}\ {\isasymsubseteq}\ formula{\isachardoublequoteclose}\isanewline
\ \ \ \ {\isachardoublequoteopen}M\ {\isasymTurnstile}\ {\isacharbraceleft}{\kern0pt}\ {\isasymcdot}Replacement{\isacharparenleft}{\kern0pt}ground{\isacharunderscore}{\kern0pt}repl{\isacharunderscore}{\kern0pt}fm{\isacharparenleft}{\kern0pt}{\isasymphi}{\isacharparenright}{\kern0pt}{\isacharparenright}{\kern0pt}{\isasymcdot}\ {\isachardot}{\kern0pt}\ {\isasymphi}\ {\isasymin}\ {\isasymPhi}{\isacharbraceright}{\kern0pt}{\isachardoublequoteclose}\isanewline
\ \ \isakeyword{shows}\isanewline
\ \ \ \ {\isachardoublequoteopen}{\isasymexists}N{\isachardot}{\kern0pt}\isanewline
\ \ \ \ \ \ M\ {\isasymsubseteq}\ N\ {\isasymand}\ N\ {\isasymapprox}\ {\isasymomega}\ {\isasymand}\ Transset{\isacharparenleft}{\kern0pt}N{\isacharparenright}{\kern0pt}\ {\isasymand}\isanewline
\ \ \ \ \ \ N\ {\isasymTurnstile}\ ZC\ {\isasymunion}\ {\isacharbraceleft}{\kern0pt}{\isasymcdot}CH{\isasymcdot}{\isacharbraceright}{\kern0pt}\ {\isasymunion}\ {\isacharbraceleft}{\kern0pt}\ {\isasymcdot}Replacement{\isacharparenleft}{\kern0pt}{\isasymphi}{\isacharparenright}{\kern0pt}{\isasymcdot}\ {\isachardot}{\kern0pt}\ {\isasymphi}\ {\isasymin}\ {\isasymPhi}{\isacharbraceright}{\kern0pt}\ {\isasymand}\isanewline
\ \ \ \ \ \ {\isacharparenleft}{\kern0pt}{\isasymforall}{\isasymalpha}{\isachardot}{\kern0pt}\ Ord{\isacharparenleft}{\kern0pt}{\isasymalpha}{\isacharparenright}{\kern0pt}\ {\isasymlongrightarrow}\ {\isacharparenleft}{\kern0pt}{\isasymalpha}\ {\isasymin}\ M\ {\isasymlongleftrightarrow}\ {\isasymalpha}\ {\isasymin}\ N{\isacharparenright}{\kern0pt}{\isacharparenright}{\kern0pt}{\isachardoublequoteclose}
\end{isabelle}

%%% Local Variables: 
%%% mode: latex
%%% TeX-master: "independence_ch_isabelle"
%%% ispell-local-dictionary: "american"
%%% End: 
