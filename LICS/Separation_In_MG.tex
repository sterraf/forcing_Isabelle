\documentclass[conference]{IEEEtran}
% Some Computer Society conferences also require the compsoc mode option,
% but others use the standard conference format.
%
% If IEEEtran.cls has not been installed into the LaTeX system files,
% manually specify the path to it like:
% \documentclass[conference]{../sty/IEEEtran}





% Some very useful LaTeX packages include:
% (uncomment the ones you want to load)
\usepackage[numbers]{natbib}
\usepackage{isabelle,isabellesym}
\usepackage{amsmath}
%\usepackage{amsthm}
\usepackage{amsfonts}
\usepackage{amssymb}
%\usepackage{bbm}  % Para el \bb{1}
%\usepackage[numbers]{natbib}
\usepackage{enumitem}
\usepackage{babel}
%\usepackage{babelbib}
\usepackage{multidef}
\usepackage{verbatim}
\usepackage{stmaryrd} %% para \llbracket
%%
%% \usepackage[bottom=2cm, top=2cm, left=2cm, right=2cm]{geometry}
%% \usepackage{titling}
%% \setlength{\droptitle}{-10ex} 
%%
\renewcommand{\o}{\vee}
\renewcommand{\O}{\bigvee}
\newcommand{\y}{\wedge}
\newcommand{\Y}{\bigwedge}
\newcommand{\limp}{\rightarrow}
\newcommand{\lsii}{\leftrightarrow}
%%

\DeclareMathOperator{\cf}{cf}
\DeclareMathOperator{\dom}{dom}
\DeclareMathOperator{\im}{img}
\DeclareMathOperator{\Fn}{Fn}
\DeclareMathOperator{\rk}{rk}
\DeclareMathOperator{\mos}{mos}
\DeclareMathOperator{\trcl}{trcl}
\DeclareMathOperator*{\diag}{\bigtriangleup}
\DeclareMathOperator{\Con}{Con}
\DeclareMathOperator{\Club}{Club}


\newcommand{\modelo}[1]{\mathbf{#1}}
\newcommand{\axiomas}[1]{\mathit{#1}}
\newcommand{\clase}[1]{\mathsf{#1}}
\newcommand{\poset}[1]{\mathbb{#1}}
\newcommand{\operador}[1]{\mathbf{#1}}

%% \newcommand{\Lim}{\clase{Lim}}
%% \newcommand{\Reg}{\clase{Reg}}
%% \newcommand{\Card}{\clase{Card}}
%% \newcommand{\On}{\clase{On}}
%% \newcommand{\WF}{\clase{WF}}
%% \newcommand{\HF}{\clase{HF}}
%% \newcommand{\HC}{\clase{HC}}
%%
%% El siguiente comando reemplaza todos los anteriores:
%%
\multidef{\clase{#1}}{Card,HC,HF,Lim,On->Ord,Reg,WF,Ord}
\newcommand{\ON}{\On}

%% En lugar de usar todo el paquete bbm:
\DeclareMathAlphabet{\mathbbm}{U}{bbm}{m}{n} 
\newcommand{\1}{\mathbbm{1}}

%%
%% \newcommand{\calD}{\mathcal{D}}
%% \newcommand{\calS}{\mathcal{S}}
%% \newcommand{\calU}{\mathcal{U}}
%% \newcommand{\calB}{\mathcal{B}}
%% \newcommand{\calL}{\mathcal{L}}
%% \newcommand{\calF}{\mathcal{F}}
%% \newcommand{\calT}{\mathcal{T}}
%% \newcommand{\calW}{\mathcal{W}}
%% \newcommand{\calA}{\mathcal{A}}
%%
%% El siguiente comando reemplaza todos los anteriores:
%%
\multidef[prefix=cal]{\mathcal{#1}}{A-Z}
%%
%% \newcommand{\A}{\modelo{A}}
%% \newcommand{\BB}{\modelo{B}}
%% \newcommand{\ZZ}{\modelo{Z}}
%% \newcommand{\PP}{\modelo{P}}
%% \newcommand{\QQ}{\modelo{Q}}
%% \newcommand{\RR}{\modelo{R}}
%%
%% El siguiente comando reemplaza todos los anteriores:
%%
\multidef{\modelo{#1}}{A,BB->B,CC->C,NN->N,PP->P,QQ->Q,RR->R,ZZ->Z}

\multidef[prefix=p]{\mathbb{#1}}{A-Z}
%% \newcommand{\B}{\modelo{B}}
%% \newcommand{\C}{\modelo{C}}
%% \newcommand{\F}{\modelo{F}}
%% \newcommand{\D}{\modelo{D}}

\newcommand{\Th}{\mb{Th}}
\newcommand{\Mod}{\mb{Mod}}

\newcommand{\Se}{\operador{S^\prec}}
\newcommand{\Pu}{\operador{P_u}}
\renewcommand{\Pr}{\operador{P_R}}
\renewcommand{\H}{\operador{H}}
\renewcommand{\S}{\operador{S}}
\newcommand{\I}{\operador{I}}
\newcommand{\E}{\operador{E}}

\newcommand{\se}{\preccurlyeq}
\newcommand{\ee}{\succ}
\newcommand{\id}{\approx}
\newcommand{\subm}{\subseteq}
\newcommand{\ext}{\supseteq}
\newcommand{\iso}{\cong}
%%
\renewcommand{\emptyset}{\varnothing}
\newcommand{\rel}{\mathcal{R}}
\newcommand{\Pow}{\mathop{\mathcal{P}}}
\renewcommand{\P}{\Pow}
\newcommand{\BP}{\mathrm{BP}}
\newcommand{\func}{\rightarrow}
\newcommand{\ord}{\mathrm{Ord}}
\newcommand{\R}{\mathbb{R}}
\newcommand{\N}{\mathbb{N}}
\newcommand{\Z}{\mathbb{Z}}
\renewcommand{\I}{\mathbb{I}}
\newcommand{\Q}{\mathbb{Q}}
\newcommand{\B}{\mathbf{B}}
\newcommand{\<}{\langle}
\renewcommand{\>}{\rangle}
\newcommand{\lb}{\langle}
\newcommand{\rb}{\rangle}
\newcommand{\impl}{\rightarrow}
\newcommand{\ent}{\Rightarrow}
\newcommand{\tne}{\Leftarrow}
\newcommand{\sii}{\Leftrightarrow}
\renewcommand{\phi}{\varphi}
\newcommand{\phis}{{\varphi^*}}
\renewcommand{\th}{\theta}
\newcommand{\Lda}{\Lambda}
\newcommand{\La}{\Lambda}
\newcommand{\lda}{\lambda}
\newcommand{\ka}{\kappa}
\newcommand{\del}{\delta}
\newcommand{\de}{\delta}
\newcommand{\ze}{\zeta}
%\newcommand{\ }{\ }
\newcommand{\la}{\lambda}
\newcommand{\al}{\alpha}
\newcommand{\be}{\beta}
\newcommand{\ga}{\gamma}
\newcommand{\Ga}{\Gamma}
\newcommand{\ep}{\varepsilon}
\newcommand{\De}{\Delta}
\newcommand{\defi}{\mathrel{\mathop:}=}
\newcommand{\forces}{\Vdash}
%\newcommand{\ap}{\mathbin{\wideparen{\ }}}
\newcommand{\Tree}{{\mathrm{Tr}_\N}}
\newcommand{\PTree}{{\mathrm{PTr}_\N}}
\newcommand{\NWO}{\mathit{NWO}}
\newcommand{\Suc}{{\N^{<\N}}}%
\newcommand{\init}{\mathsf{i}}
\newcommand{\ap}{\mathord{^\smallfrown}}
\newcommand{\Cantor}{\mathcal{C}}
%\newcommand{\C}{\Cantor}
\newcommand{\Baire}{\mathcal{N}}
\newcommand{\sig}{\ensuremath{\sigma}}
\newcommand{\fsig}{\ensuremath{F_\sigma}}
\newcommand{\gdel}{\ensuremath{G_\delta}}
\newcommand{\Sig}{\ensuremath{\boldsymbol{\Sigma}}}
\newcommand{\bPi}{\ensuremath{\boldsymbol{\Pi}}}
\newcommand{\Del}{\ensuremath{\boldsymbol\Delta}}
%\renewcommand{\F}{\operador{F}}
\newcommand{\ths}{{\theta^*}}
\newcommand{\om}{\ensuremath{\omega}}
%\renewcommand{\c}{\complement}
\newcommand{\comp}{\mathsf{c}}
\newcommand{\co}[1]{\left(#1\right)^\comp}
\newcommand{\len}[1]{\left|#1\right|}
\DeclareMathOperator{\tlim}{\overline{\mathrm{TLim}}}
\newcommand{\card}[1]{{\left|#1\right|}}
\newcommand{\bigcard}[1]{{\bigl|#1\bigr|}}
%
% Cardinality
%
\newcommand{\lec}{\leqslant_c}
\newcommand{\gec}{\geqslant_c}
\newcommand{\lc}{<_c}
\newcommand{\gc}{>_c}
\newcommand{\eqc}{=_c}
\newcommand{\biy}{\approx}
\newcommand*{\ale}[1]{\aleph_{#1}}
%
\newcommand{\Zerm}{\axiomas{Z}}
\newcommand{\ZC}{\axiomas{ZC}}
\newcommand{\AC}{\axiomas{AC}}
\newcommand{\DC}{\axiomas{DC}}
\newcommand{\MA}{\axiomas{MA}}
\newcommand{\CH}{\axiomas{CH}}
\newcommand{\ZFC}{\axiomas{ZFC}}
\newcommand{\ZF}{\axiomas{ZF}}
\newcommand{\Inf}{\axiomas{Inf}}
%
% Cardinal characteristics
%
\newcommand{\cont}{\mathfrak{c}}
\newcommand{\spl}{\mathfrak{s}}
\newcommand{\bound}{\mathfrak{b}}
\newcommand{\mad}{\mathfrak{a}}
\newcommand{\tower}{\mathfrak{t}}
%
\renewcommand{\hom}[2]{{}^{#1}\hskip-0.116ex{#2}}
\newcommand{\pred}[1][{}]{\mathop{\mathrm{pred}_{#1}}}
%% Postfix operator with supressable space:
%% \newcommand*{\iseg}{\relax\ifnum\lastnodetype>0 \mskip\medmuskip\fi{\downarrow}} %
\newcommand*{\iseg}{{\downarrow}}
\newcommand{\rr}{\mathrel{R}}
\newcommand{\restr}{\upharpoonright}
%\newcommand{\type}{\mathtt{}}
\newcommand{\app}{\mathop{\mathrm{Aprox}}}
\newcommand{\hess}{\triangleleft}
\newcommand{\bx}{\bar{x}}
\newcommand{\by}{\bar{y}}
\newcommand{\bz}{\bar{z}}
\newcommand{\union}{\mathop{\textstyle\bigcup}}
\newcommand{\sm}{\setminus}
\newcommand{\sbq}{\subseteq}
\newcommand{\nsbq}{\subseteq}
\newcommand{\mty}{\emptyset}
\newcommand{\dimg}{\text{\textup{``}}} % direct image
\newcommand{\quine}[1]{\ulcorner{\!#1\!}\urcorner}
%\newcommand{\ntrm}[1]{\textsl{\textbf{#1}}}
\newcommand{\Null}{\calN\!\mathit{ull}}
\DeclareMathOperator{\club}{Club}
\DeclareMathOperator{\otp}{otp}

%%%%%%%%%%%%%%%%%%%%%%%%%
% Variant aleph, beth, etc
% From http://tex.stackexchange.com/q/170476/69595
\makeatletter
\@ifpackageloaded{txfonts}\@tempswafalse\@tempswatrue
\if@tempswa
  \DeclareFontFamily{U}{txsymbols}{}
  \DeclareFontFamily{U}{txAMSb}{}
  \DeclareSymbolFont{txsymbols}{OMS}{txsy}{m}{n}
  \SetSymbolFont{txsymbols}{bold}{OMS}{txsy}{bx}{n}
  \DeclareFontSubstitution{OMS}{txsy}{m}{n}
  \DeclareSymbolFont{txAMSb}{U}{txsyb}{m}{n}
  \SetSymbolFont{txAMSb}{bold}{U}{txsyb}{bx}{n}
  \DeclareFontSubstitution{U}{txsyb}{m}{n}
  \DeclareMathSymbol{\aleph}{\mathord}{txsymbols}{64}
  \DeclareMathSymbol{\beth}{\mathord}{txAMSb}{105}
  \DeclareMathSymbol{\gimel}{\mathord}{txAMSb}{106}
  \DeclareMathSymbol{\daleth}{\mathord}{txAMSb}{107}
\fi
\makeatother

%%%%%%%%%%%%%%%%%%%%%%%%%%%%%%%%%%%%%%%%%%%%%%%%%%%%%%%%%%%%
%%
%% Theorem Environments
%%
%% \newtheorem{theorem}{Theorem}
%% \newtheorem{lemma}[theorem]{Lemma}
%% \newtheorem{prop}[theorem]{Proposition}
%% \newtheorem{corollary}[theorem]{Corollary}
%% \newtheorem{claim}{Claim}
%% \newtheorem*{claim*}{Claim}
%% \theoremstyle{definition}
%% \newtheorem{definition}[theorem]{Definition}
%% \newtheorem{remark}[theorem]{Remark}
%% \newtheorem{example}[theorem]{Example}
%% \theoremstyle{remark}
%% \newtheorem*{remark*}{Remark}
%%
%%%%%%%%%%%%%%%%%%%%%%%%%%%%%%%%%%%%%%%%%%%%%%%%%%%%%%%%%%%%%%%%%%%%%%

%% \newenvironment{inducc}{\begin{list}{}{\itemindent=2.5em \labelwidth=4em}}{\end{list}}
%% \newcommand{\caso}[1]{\item[\fbox{#1}]}
\newenvironment{proofofclaim}{\begin{proof}[Proof of Claim]}{\end{proof}}


%%% Local Variables: 
%%% mode: latex
%%% TeX-master: "first_steps_into_forcing"
%%% End: 

\renewcommand{\ttdefault}{cmtt}
\usepackage{csquotes}
\usepackage{enumitem}

\newlist{inlinelist}{enumerate*}{1}
\setlist*[inlinelist,1]{%
  label=(\roman*),
}
% *** MISC UTILITY PACKAGES ***
%
%\usepackage{ifpdf}
% Heiko Oberdiek's ifpdf.sty is very useful if you need conditional
% compilation based on whether the output is pdf or dvi.
% usage:
% \ifpdf
%   % pdf code
% \else
%   % dvi code
% \fi
% The latest version of ifpdf.sty can be obtained from:
% http://www.ctan.org/pkg/ifpdf
% Also, note that IEEEtran.cls V1.7 and later provides a builtin
% \ifCLASSINFOpdf conditional that works the same way.
% When switching from latex to pdflatex and vice-versa, the compiler may
% have to be run twice to clear warning/error messages.






% *** CITATION PACKAGES ***
%
%\usepackage{cite}
% cite.sty was written by Donald Arseneau
% V1.6 and later of IEEEtran pre-defines the format of the cite.sty package
% \cite{} output to follow that of the IEEE. Loading the cite package will
% result in citation numbers being automatically sorted and properly
% "compressed/ranged". e.g., [1], [9], [2], [7], [5], [6] without using
% cite.sty will become [1], [2], [5]--[7], [9] using cite.sty. cite.sty's
% \cite will automatically add leading space, if needed. Use cite.sty's
% noadjust option (cite.sty V3.8 and later) if you want to turn this off
% such as if a citation ever needs to be enclosed in parenthesis.
% cite.sty is already installed on most LaTeX systems. Be sure and use
% version 5.0 (2009-03-20) and later if using hyperref.sty.
% The latest version can be obtained at:
% http://www.ctan.org/pkg/cite
% The documentation is contained in the cite.sty file itself.






% *** GRAPHICS RELATED PACKAGES ***
%
\ifCLASSINFOpdf
  % \usepackage[pdftex]{graphicx}
  % declare the path(s) where your graphic files are
  % \graphicspath{{../pdf/}{../jpeg/}}
  % and their extensions so you won't have to specify these with
  % every instance of \includegraphics
  % \DeclareGraphicsExtensions{.pdf,.jpeg,.png}
\else
  % or other class option (dvipsone, dvipdf, if not using dvips). graphicx
  % will default to the driver specified in the system graphics.cfg if no
  % driver is specified.
  % \usepackage[dvips]{graphicx}
  % declare the path(s) where your graphic files are
  % \graphicspath{{../eps/}}
  % and their extensions so you won't have to specify these with
  % every instance of \includegraphics
  % \DeclareGraphicsExtensions{.eps}
\fi
% graphicx was written by David Carlisle and Sebastian Rahtz. It is
% required if you want graphics, photos, etc. graphicx.sty is already
% installed on most LaTeX systems. The latest version and documentation
% can be obtained at: 
% http://www.ctan.org/pkg/graphicx
% Another good source of documentation is "Using Imported Graphics in
% LaTeX2e" by Keith Reckdahl which can be found at:
% http://www.ctan.org/pkg/epslatex
%
% latex, and pdflatex in dvi mode, support graphics in encapsulated
% postscript (.eps) format. pdflatex in pdf mode supports graphics
% in .pdf, .jpeg, .png and .mps (metapost) formats. Users should ensure
% that all non-photo figures use a vector format (.eps, .pdf, .mps) and
% not a bitmapped formats (.jpeg, .png). The IEEE frowns on bitmapped formats
% which can result in "jaggedy"/blurry rendering of lines and letters as
% well as large increases in file sizes.
%
% You can find documentation about the pdfTeX application at:
% http://www.tug.org/applications/pdftex





% *** MATH PACKAGES ***
%
%\usepackage{amsmath}
% A popular package from the American Mathematical Society that provides
% many useful and powerful commands for dealing with mathematics.
%
% Note that the amsmath package sets \interdisplaylinepenalty to 10000
% thus preventing page breaks from occurring within multiline equations. Use:
%\interdisplaylinepenalty=2500
% after loading amsmath to restore such page breaks as IEEEtran.cls normally
% does. amsmath.sty is already installed on most LaTeX systems. The latest
% version and documentation can be obtained at:
% http://www.ctan.org/pkg/amsmath





% *** SPECIALIZED LIST PACKAGES ***
%
%\usepackage{algorithmic}
% algorithmic.sty was written by Peter Williams and Rogerio Brito.
% This package provides an algorithmic environment fo describing algorithms.
% You can use the algorithmic environment in-text or within a figure
% environment to provide for a floating algorithm. Do NOT use the algorithm
% floating environment provided by algorithm.sty (by the same authors) or
% algorithm2e.sty (by Christophe Fiorio) as the IEEE does not use dedicated
% algorithm float types and packages that provide these will not provide
% correct IEEE style captions. The latest version and documentation of
% algorithmic.sty can be obtained at:
% http://www.ctan.org/pkg/algorithms
% Also of interest may be the (relatively newer and more customizable)
% algorithmicx.sty package by Szasz Janos:
% http://www.ctan.org/pkg/algorithmicx




% *** ALIGNMENT PACKAGES ***
%
%\usepackage{array}
% Frank Mittelbach's and David Carlisle's array.sty patches and improves
% the standard LaTeX2e array and tabular environments to provide better
% appearance and additional user controls. As the default LaTeX2e table
% generation code is lacking to the point of almost being broken with
% respect to the quality of the end results, all users are strongly
% advised to use an enhanced (at the very least that provided by array.sty)
% set of table tools. array.sty is already installed on most systems. The
% latest version and documentation can be obtained at:
% http://www.ctan.org/pkg/array


% IEEEtran contains the IEEEeqnarray family of commands that can be used to
% generate multiline equations as well as matrices, tables, etc., of high
% quality.




% *** SUBFIGURE PACKAGES ***
%\ifCLASSOPTIONcompsoc
%  \usepackage[caption=false,font=normalsize,labelfont=sf,textfont=sf]{subfig}
%\else
%  \usepackage[caption=false,font=footnotesize]{subfig}
%\fi
% subfig.sty, written by Steven Douglas Cochran, is the modern replacement
% for subfigure.sty, the latter of which is no longer maintained and is
% incompatible with some LaTeX packages including fixltx2e. However,
% subfig.sty requires and automatically loads Axel Sommerfeldt's caption.sty
% which will override IEEEtran.cls' handling of captions and this will result
% in non-IEEE style figure/table captions. To prevent this problem, be sure
% and invoke subfig.sty's "caption=false" package option (available since
% subfig.sty version 1.3, 2005/06/28) as this is will preserve IEEEtran.cls
% handling of captions.
% Note that the Computer Society format requires a larger sans serif font
% than the serif footnote size font used in traditional IEEE formatting
% and thus the need to invoke different subfig.sty package options depending
% on whether compsoc mode has been enabled.
%
% The latest version and documentation of subfig.sty can be obtained at:
% http://www.ctan.org/pkg/subfig




% *** FLOAT PACKAGES ***
%
%\usepackage{fixltx2e}
% fixltx2e, the successor to the earlier fix2col.sty, was written by
% Frank Mittelbach and David Carlisle. This package corrects a few problems
% in the LaTeX2e kernel, the most notable of which is that in current
% LaTeX2e releases, the ordering of single and double column floats is not
% guaranteed to be preserved. Thus, an unpatched LaTeX2e can allow a
% single column figure to be placed prior to an earlier double column
% figure.
% Be aware that LaTeX2e kernels dated 2015 and later have fixltx2e.sty's
% corrections already built into the system in which case a warning will
% be issued if an attempt is made to load fixltx2e.sty as it is no longer
% needed.
% The latest version and documentation can be found at:
% http://www.ctan.org/pkg/fixltx2e


%\usepackage{stfloats}
% stfloats.sty was written by Sigitas Tolusis. This package gives LaTeX2e
% the ability to do double column floats at the bottom of the page as well
% as the top. (e.g., "\begin{figure*}[!b]" is not normally possible in
% LaTeX2e). It also provides a command:
%\fnbelowfloat
% to enable the placement of footnotes below bottom floats (the standard
% LaTeX2e kernel puts them above bottom floats). This is an invasive package
% which rewrites many portions of the LaTeX2e float routines. It may not work
% with other packages that modify the LaTeX2e float routines. The latest
% version and documentation can be obtained at:
% http://www.ctan.org/pkg/stfloats
% Do not use the stfloats baselinefloat ability as the IEEE does not allow
% \baselineskip to stretch. Authors submitting work to the IEEE should note
% that the IEEE rarely uses double column equations and that authors should try
% to avoid such use. Do not be tempted to use the cuted.sty or midfloat.sty
% packages (also by Sigitas Tolusis) as the IEEE does not format its papers in
% such ways.
% Do not attempt to use stfloats with fixltx2e as they are incompatible.
% Instead, use Morten Hogholm'a dblfloatfix which combines the features
% of both fixltx2e and stfloats:
%
% \usepackage{dblfloatfix}
% The latest version can be found at:
% http://www.ctan.org/pkg/dblfloatfix




% *** PDF, URL AND HYPERLINK PACKAGES ***
%
%\usepackage{url}
% url.sty was written by Donald Arseneau. It provides better support for
% handling and breaking URLs. url.sty is already installed on most LaTeX
% systems. The latest version and documentation can be obtained at:
% http://www.ctan.org/pkg/url
% Basically, \url{my_url_here}.




% *** Do not adjust lengths that control margins, column widths, etc. ***
% *** Do not use packages that alter fonts (such as pslatex).         ***
% There should be no need to do such things with IEEEtran.cls V1.6 and later.
% (Unless specifically asked to do so by the journal or conference you plan
% to submit to, of course. )


% correct bad hyphenation here
\hyphenation{op-tical net-works semi-conduc-tor}


\begin{document}
%
% paper title
% Titles are generally capitalized except for words such as a, an, and, as,
% at, but, by, for, in, nor, of, on, or, the, to and up, which are usually
% not capitalized unless they are the first or last word of the title.
% Linebreaks \\ can be used within to get better formatting as desired.
% Do not put math or special symbols in the title.
\title{Separation in Generic Extensions for Isabelle}


% author names and affiliations
% use a multiple column layout for up to three different
% affiliations
\author{\IEEEauthorblockN{Emmanuel Gunther}
\IEEEauthorblockA{FaMAF\\ Universidad Nacional de C\'ordoba\\
    C\'ordoba, Argentina\\
Email: gunther@famaf.unc.edu.ar}
\and
\IEEEauthorblockN{Miguel Pagano}
\IEEEauthorblockA{FaMAF\\ Universidad Nacional de C\'ordoba\\
    C\'ordoba, Argentina\\
Email: pagano@famaf.unc.edu.ar}
\and
\IEEEauthorblockN{Pedro S\'anchez Terraf}
\IEEEauthorblockA{CIEM-FaMAF\\Universidad Nacional de C\'ordoba\\
  C\'ordoba, Argentina\\
Email:  sterraf@famaf.unc.edu.ar}
}

% conference papers do not typically use \thanks and this command
% is locked out in conference mode. If really needed, such as for
% the acknowledgment of grants, issue a \IEEEoverridecommandlockouts
% after \documentclass

% for over three affiliations, or if they all won't fit within the width
% of the page, use this alternative format:
% 
%\author{\IEEEauthorblockN{Michael Shell\IEEEauthorrefmark{1},
%Homer Simpson\IEEEauthorrefmark{2},
%James Kirk\IEEEauthorrefmark{3}, 
%Montgomery Scott\IEEEauthorrefmark{3} and
%Eldon Tyrell\IEEEauthorrefmark{4}}
%\IEEEauthorblockA{\IEEEauthorrefmark{1}School of Electrical and Computer Engineering\\
%Georgia Institute of Technology,
%Atlanta, Georgia 30332--0250\\ Email: see http://www.michaelshell.org/contact.html}
%\IEEEauthorblockA{\IEEEauthorrefmark{2}Twentieth Century Fox, Springfield, USA\\
%Email: homer@thesimpsons.com}
%\IEEEauthorblockA{\IEEEauthorrefmark{3}Starfleet Academy, San Francisco, California 96678-2391\\
%Telephone: (800) 555--1212, Fax: (888) 555--1212}
%\IEEEauthorblockA{\IEEEauthorrefmark{4}Tyrell Inc., 123 Replicant Street, Los Angeles, California 90210--4321}}




% use for special paper notices
%\IEEEspecialpapernotice{(Invited Paper)}




% make the title area
\maketitle

% As a general rule, do not put math, special symbols or citations
% in the abstract
\begin{abstract}
  We mechanize, in the proof assistant
  Isabelle, a proof of the
  axiom-scheme of Separation in 
  generic extensions of models of set theory  
  by using the fundamental theorems of forcing.
  We also formalize the satisfaction of the axioms of
  Extensionality, Foundation, Union, and Infinity.
  We also we extend Paulson's library on constructibility  with
  renaming of variables for internalized formulas, an improvement on
  definitions by recursion on well-founded  relations and sharpening
  of the hypotheses in his development of relativization and
  absolutness.
\end{abstract}

% no keywords


% For peer review papers, you can put extra information on the cover
% page as needed:
% \ifCLASSOPTIONpeerreview
% \begin{center} \bfseries EDICS Category: 3-BBND \end{center}
% \fi
%
% For peerreview papers, this IEEEtran command inserts a page break and
% creates the second title. It will be ignored for other modes.
\IEEEpeerreviewmaketitle

%%%%%%%%%%%%%%%%%%%%%%%%%%%%%%%%%%%%%%%%%%%%%%%%%%%%%%%%%%%%%%%%%%%%%%          
\section{Introduction}
% no \IEEEPARstart
Zermelo-Fraenkel Set Theory ($\ZF$) has a prominent place among formal
theories; in particular, it provides a foundation for mathematics and
most of the formal toolkit used everyday by the computer scientist has
also Set Theory at its base (cf.~\cite{paulson1995set}). In this time
of mechanization of mathematics~\cite{avigad2018mechanization}, it
seems natural to ask for a mechanization of the most salient results
of Set Theory.

After G\"odel's Incompleteness Theorems, we cannot expect to have a
formal proof of the consistency of Set Theory in $\ZF$. Besides its own
consistency, there are other results which are undecided by $\ZF$: the
undecidability of Continuum Hypothesis leads to the development of
techniques for independence proofs. First G\"odel introduced the
theory of \emph{inner models}, which gives rise to his model $L$ of
the \emph{Axiom of Constructibility} \cite{godel-L} and proved the
relative consistency of the Axiom of Choice and the Generalized
Continuum Hypothesis with $\ZF$. Thirty years later Paul
J. Cohen~\cite{Cohen-CH-PNAS} devised the technique of \emph{forcing},
which is the only known way of \emph{extending} models of $\ZF$; in
particular, it can be used to prove the relative consistency of the
negation of the Continuum Hypothesis. 

In this work we address a substantial part of formalizing the proof
that given a model $M$ of $\ZF$, any \emph{generic extension} $M[G]$
given by forcing also satisfies $\ZF$. As remarked by
\citet[][p.250]{kunen2011set} \enquote{[...] in verifying that $M[G]$
  is a model for set theory, the hardest axiom to verify is
  Comprehension.}  The most important achievement of this paper is the
mechanization in Isabelle a considerable part of this result; en route
to this, we also formalized the satisfaction by $M[G]$ of
Extensionality, Foundation, Union, and Infinity.

Our development benefited from the remarkable work done by Lawrence
Paulson \cite{paulson_2003} on the formalization of G\"odel's
constructible universe in the proof assistant \emph{Isabelle}. The
ultimate goal of our project is the formalization of forcing to
complete the mechanization of the independence of the Continuum
Hypothesis. We think that this project constitutes an interesting case
which stresses how feasible is to formally implement mathematics that
involve several levels of reasoning.

The \emph{Formal Abstract} project~\cite{hales-fabstracts} proposes
that the mechanization of mathematics can advance by formalizing some
main result under the explicit assumption of the material
(definitions, propositions, lemmas) on which it is based. In line with
this advice, we profited that the proof that those axioms hold in
generic extension is independent of the \emph{proofs} of the
``fundamental'' theorems of forcing. Let us remark that the definition
of the forces relation is, by itself, quite demanding; the
formalization of it and of the fundamental theorems of forcing roughly
comprises little less than a half of our full project.

It might be a little surprising the lack of formalizations of forcing
and generic extensions. As far as we know, the development of
\citet{JFR6232} in homotopy type theory for constructing generic
extensions in a sheaf-theoretic setting is the unique mechanization of
forcing. This contrast with the fruitful use of forcing techniques to
extend the Curry-Howard isomorphism to classical axioms
\cite{Miquel:2011:FPT:2058525.2059614,lmcs:1070}. Moreover, the
combination of forcing with intuitionistic type theory
\cite{coquand2010note} gives rise both to positive results (an
algorithm to obtain witnesses of the continuity of definable
functionals \cite{coquand2012computational}) and also negative
(the independence of Markov's principle \cite{lmcs:3859}).

% moreover forcing translations \cite{jaber:hal-01319066}.

\fbox{Parece mucho comienzo sólo para introducir a Kunen}
\fbox{¿lo puedo achurar un poco?}

In a gross simplification, there are two aspects to a formalization
project like this one: thematic and programmatic. The first concerns
the handling of all the theoretical concepts and results in the
subject, while the second involves the practical issues of the
implementation and design. In the case of forcing, the main intricacy
lies in the first aspect. In this sense, following a sensible
presentation of the material is key.  The authoritative reference 
on the subject during the last 30 years has been Kunen's classical
\cite{kunen1980}. In our
formalizaton we have followed a recent rewrite \cite{kunen2011set}
of that  textbook, which presents the material in the same sharp 
style but offering a lot of details. In some sense this project
wouldn't exist without this book. As alternative, introductory
resources, the  interested reader can check
\cite{chow-beginner-forcing}; also, the book \cite{weaver2014forcing}
contains a thorough treament minimizing the technicalities.

We briefly describe the contents of each
section. Section~\ref{sec:isabelle} contains the bare minimun
requirements to understand the (meta)logics used in Isabelle. Next, an
overview of the model theory of set theory is presented in
Section~\ref{sec:axioms-models-set-theory}. There is an ``internal''
representation of first-order formulas as sets, implemented by
Paulson; Section~\ref{sec:renaming} discusses syntactical
transformations of the former, mainly permutation of variables. 
In Section~\ref{sec:generic-extensions} the generic extensions are
succintly reviewed and how the treatment of well founded recursion in
Isabelle was enhanced. We take care of the ``easy axioms'' in
Section~\ref{sec:easy-axioms}; these are the ones that
do not depend on the forcing theorems. We describe the latter in
Section~\ref{sec:forcing}. We adapted the  work by Paulson to our
needs, and this is described in
Section~\ref{sec:hack-constructible}. We present the proof
of the Separation Axiom Scheme in Section~\ref{sec:proof-separation},
which follows closely its implementation. A plan for future work and
some immediate conclusions are offered in
Section~\ref{sec:conclusions-future-work}.

%%% Local Variables: 
%%% mode: latex
%%% TeX-master: "Separation_In_MG"
%%% ispell-local-dictionary: "american"
%%% End: 


%%
%% macros for Isabelle generated LaTeX output
%%

%%% Simple document preparation (based on theory token language and symbols)

% isabelle environments

\newcommand{\isabellecontext}{UNKNOWN}
\newcommand{\setisabellecontext}[1]{\def\isabellecontext{#1}}

\newcommand{\isastyle}{\UNDEF}
\newcommand{\isastylett}{\UNDEF}
\newcommand{\isastyleminor}{\UNDEF}
\newcommand{\isastyleminortt}{\UNDEF}
\newcommand{\isastylescript}{\UNDEF}
\newcommand{\isastyletext}{\normalsize\rm}
\newcommand{\isastyletxt}{\rm}
\newcommand{\isastylecmt}{\rm}

\newcommand{\isaspacing}{%
  \sfcode 42 1000 % .
  \sfcode 63 1000 % ?
  \sfcode 33 1000 % !
  \sfcode 58 1000 % :
  \sfcode 59 1000 % ;
  \sfcode 44 1000 % ,
}

%symbol markup -- \emph achieves decent spacing via italic corrections
\newcommand{\isamath}[1]{\emph{$#1$}}
\newcommand{\isatext}[1]{\emph{#1}}
\DeclareRobustCommand{\isascriptstyle}{\def\isamath##1{##1}\def\isatext##1{\mbox{\isaspacing\isastylescript##1}}}
\newcommand{\isactrlsub}[1]{\emph{\isascriptstyle${}\sb{#1}$}}
\newcommand{\isactrlsup}[1]{\emph{\isascriptstyle${}\sp{#1}$}}
\DeclareRobustCommand{\isactrlbsub}{\emph\bgroup\math{}\sb\bgroup\mbox\bgroup\isaspacing\isastylescript}
\DeclareRobustCommand{\isactrlesub}{\egroup\egroup\endmath\egroup}
\DeclareRobustCommand{\isactrlbsup}{\emph\bgroup\math{}\sp\bgroup\mbox\bgroup\isaspacing\isastylescript}
\DeclareRobustCommand{\isactrlesup}{\egroup\egroup\endmath\egroup}
\newcommand{\isactrlbold}[1]{{\bfseries\upshape\boldmath#1}}

\newcommand{\isaantiqcontrol}[1]{\isatt{{\char`\\}{\char`\<}{\char`\^}#1{\char`\>}}}
\newenvironment{isaantiq}{{\isacharat\isacharbraceleft}}{{\isacharbraceright}}

%\newdimen\isa@parindent\newdimen\isa@parskip

\newenvironment{isabellebody}{%
\isamarkuptrue\par%
\parindent\parindent0pt%
\parskip\parskip0pt%
\isaspacing\isastyle}{\par}

\newenvironment{isabellebodytt}{%
\isamarkuptrue\par%
\parindent\parindent0pt%
\parskip\parskip\parskip0pt%
\isaspacing\isastylett}{\par}

\newenvironment{isabelle}
{\begin{trivlist}\begin{isabellebody}\item\relax}
{\end{isabellebody}\end{trivlist}}

\newenvironment{isabellett}
{\begin{trivlist}\begin{isabellebodytt}\item\relax}
{\end{isabellebodytt}\end{trivlist}}

\newcommand{\isa}[1]{\emph{\isaspacing\isastyleminor #1}}
\newcommand{\isatt}[1]{\emph{\isaspacing\isastyleminortt #1}}

\newcommand{\isaindent}[1]{\hphantom{#1}}
\newcommand{\isanewline}{\mbox{}\par\mbox{}}
\newcommand{\isasep}{}
\newcommand{\isadigit}[1]{#1}

\newcommand{\isachardefaults}{%
\def\isacharbell{\isamath{\bigbox}}%requires stmaryrd
\chardef\isacharbang=`\!%
\chardef\isachardoublequote=`\"%
\chardef\isachardoublequoteopen=`\"%
\chardef\isachardoublequoteclose=`\"%
\chardef\isacharhash=`\#%
\chardef\isachardollar=`\$%
\chardef\isacharpercent=`\%%
\chardef\isacharampersand=`\&%
\chardef\isacharprime=`\'%
\chardef\isacharparenleft=`\(%
\chardef\isacharparenright=`\)%
\chardef\isacharasterisk=`\*%
\chardef\isacharplus=`\+%
\chardef\isacharcomma=`\,%
\chardef\isacharminus=`\-%
\chardef\isachardot=`\.%
\chardef\isacharslash=`\/%
\chardef\isacharcolon=`\:%
\chardef\isacharsemicolon=`\;%
\chardef\isacharless=`\<%
\chardef\isacharequal=`\=%
\chardef\isachargreater=`\>%
\chardef\isacharquery=`\?%
\chardef\isacharat=`\@%
\chardef\isacharbrackleft=`\[%
\chardef\isacharbackslash=`\\%
\chardef\isacharbrackright=`\]%
\chardef\isacharcircum=`\^%
\chardef\isacharunderscore=`\_%
\def\isacharunderscorekeyword{\_}%
\chardef\isacharbackquote=`\`%
\chardef\isacharbackquoteopen=`\`%
\chardef\isacharbackquoteclose=`\`%
\chardef\isacharbraceleft=`\{%
\chardef\isacharbar=`\|%
\chardef\isacharbraceright=`\}%
\chardef\isachartilde=`\~%
\def\isacharverbatimopen{\isacharbraceleft\isacharasterisk}%
\def\isacharverbatimclose{\isacharasterisk\isacharbraceright}%
\def\isacartoucheopen{\isatext{\raise.3ex\hbox{$\scriptscriptstyle\langle$}}}%
\def\isacartoucheclose{\isatext{\raise.3ex\hbox{$\scriptscriptstyle\rangle$}}}%
}


% keyword and section markup

\newcommand{\isakeyword}[1]
{\emph{\bf\def\isachardot{.}\def\isacharunderscore{\isacharunderscorekeyword}%
\def\isacharbraceleft{\{}\def\isacharbraceright{\}}#1}}
\newcommand{\isacommand}[1]{\isakeyword{#1}}

\newcommand{\isamarkupheader}[1]{\section{#1}}
\newcommand{\isamarkupchapter}[1]{\chapter{#1}}
\newcommand{\isamarkupsection}[1]{\section{#1}}
\newcommand{\isamarkupsubsection}[1]{\subsection{#1}}
\newcommand{\isamarkupsubsubsection}[1]{\subsubsection{#1}}
\newcommand{\isamarkupparagraph}[1]{\paragraph{#1}}
\newcommand{\isamarkupsubparagraph}[1]{\subparagraph{#1}}

\newif\ifisamarkup
\newcommand{\isabeginpar}{\par\ifisamarkup\relax\else\medskip\fi}
\newcommand{\isaendpar}{\par\medskip}
\newenvironment{isapar}{\parindent\isa@parindent\parskip\isa@parskip\isabeginpar}{\isaendpar}
\newenvironment{isamarkuptext}{\par\isastyletext\begin{isapar}}{\end{isapar}}
\newenvironment{isamarkuptxt}{\par\isastyletxt\begin{isapar}}{\end{isapar}}
\newcommand{\isamarkupcmt}[1]{{\isastylecmt--- #1}}


% styles

\def\isabellestyle#1{\csname isabellestyle#1\endcsname}

\newcommand{\isabellestyledefault}{%
\def\isastyle{\tt\slshape}%
\def\isastylett{\tt}%
\def\isastyleminor{\tt\slshape}%
\def\isastyleminortt{\tt}%
\def\isastylescript{\footnotesize\tt\slshape}%
\isachardefaults%
}
\isabellestyledefault

\newcommand{\isabellestylett}{%
\def\isastyle{\small\tt}%
\def\isastylett{\small\tt}%
\def\isastyleminor{\small\tt}%
\def\isastyleminortt{\small\tt}%
\def\isastylescript{\footnotesize\tt}%
\isachardefaults%
}

\newcommand{\isabellestyleit}{%
\def\isastyle{\small\it}%
\def\isastylett{\small\tt}%
\def\isastyleminor{\it}%
\def\isastyleminortt{\tt}%
\def\isastylescript{\footnotesize\it}%
\isachardefaults%
\def\isacharunderscorekeyword{\mbox{\_}}%
\def\isacharbang{\isamath{!}}%
\def\isachardoublequote{}%
\def\isachardoublequoteopen{}%
\def\isachardoublequoteclose{}%
\def\isacharhash{\isamath{\#}}%
\def\isachardollar{\isamath{\$}}%
\def\isacharpercent{\isamath{\%}}%
\def\isacharampersand{\isamath{\&}}%
\def\isacharprime{\isamath{\mskip2mu{'}\mskip-2mu}}%
\def\isacharparenleft{\isamath{(}}%
\def\isacharparenright{\isamath{)}}%
\def\isacharasterisk{\isamath{*}}%
\def\isacharplus{\isamath{+}}%
\def\isacharcomma{\isamath{\mathord,}}%
\def\isacharminus{\isamath{-}}%
\def\isachardot{\isamath{\mathord.}}%
\def\isacharslash{\isamath{/}}%
\def\isacharcolon{\isamath{\mathord:}}%
\def\isacharsemicolon{\isamath{\mathord;}}%
\def\isacharless{\isamath{<}}%
\def\isacharequal{\isamath{=}}%
\def\isachargreater{\isamath{>}}%
\def\isacharat{\isamath{@}}%
\def\isacharbrackleft{\isamath{[}}%
\def\isacharbackslash{\isamath{\backslash}}%
\def\isacharbrackright{\isamath{]}}%
\def\isacharunderscore{\mbox{\_}}%
\def\isacharbraceleft{\isamath{\{}}%
\def\isacharbar{\isamath{\mid}}%
\def\isacharbraceright{\isamath{\}}}%
\def\isachartilde{\isamath{{}\sp{\sim}}}%
\def\isacharbackquoteopen{\isatext{\raise.3ex\hbox{$\scriptscriptstyle\langle$}}}%
\def\isacharbackquoteclose{\isatext{\raise.3ex\hbox{$\scriptscriptstyle\rangle$}}}%
\def\isacharverbatimopen{\isamath{\langle\!\langle}}%
\def\isacharverbatimclose{\isamath{\rangle\!\rangle}}%
}

\newcommand{\isabellestyleliteral}{%
\isabellestyleit%
\def\isacharunderscore{\_}%
\def\isacharunderscorekeyword{\_}%
\chardef\isacharbackquoteopen=`\`%
\chardef\isacharbackquoteclose=`\`%
}

\newcommand{\isabellestyleliteralunderscore}{%
\isabellestyleliteral%
\def\isacharunderscore{\textunderscore}%
\def\isacharunderscorekeyword{\textunderscore}%
}

\newcommand{\isabellestylesl}{%
\isabellestyleit%
\def\isastyle{\small\sl}%
\def\isastylett{\small\tt}%
\def\isastyleminor{\sl}%
\def\isastyleminortt{\tt}%
\def\isastylescript{\footnotesize\sl}%
}


% tagged regions

%plain TeX version of comment package -- much faster!
\let\isafmtname\fmtname\def\fmtname{plain}
\usepackage{comment}
\let\fmtname\isafmtname

\newcommand{\isafold}[1]{\emph{$\langle\mathord{\mathit{#1}}\rangle$}}

\newcommand{\isakeeptag}[1]%
{\includecomment{isadelim#1}\includecomment{isatag#1}\csarg\def{isafold#1}{}}
\newcommand{\isadroptag}[1]%
{\excludecomment{isadelim#1}\excludecomment{isatag#1}\csarg\def{isafold#1}{}}
\newcommand{\isafoldtag}[1]%
{\includecomment{isadelim#1}\excludecomment{isatag#1}\csarg\def{isafold#1}{\isafold{#1}}}

\isakeeptag{theory}
\isakeeptag{proof}
\isakeeptag{ML}
\isakeeptag{visible}
\isadroptag{invisible}

\IfFileExists{isabelletags.sty}{\usepackage{isabelletags}}{}


%%%%%%%%%%%%%%%%%%%%%%%%%%%%%%%%%%%%%%%%%%%%%%%%%%%%%%%%%%%%%%%%%%%%%%          
\section{Axioms of set theory}

\medskip
\fbox{\em \dots\ discussion of axioms \dots }
\medskip

An equivalent formulation of the Foundation Axiom states that the
universe of sets can be decomposed in a transfinite hierarchy of
sets. 
\begin{theorem}
  Let $V_{\al}\defi\union\{\P(V_\be) : \be<\al\}$ for each ordinal
  $\al$. Then each $V_\al$ is a set and 
  $\forall x. \exists\al .\ \Ord(\al) \y x\in V_\al$.  
\end{theorem}

\medskip
\fbox{\em \dots\ discussion of axioms \dots }
\medskip

Models of the theory $\ZFC$ consist of a pair $\lb M,E\rb$ where $M$
is a set and $E$ is a binary relation on $M$. Forcing is a technique
to extend very special kind of models, where $M$ is a countable
transitive set (i.e., every element of $M$ is a subset of $M$) and
$E$ is the membership relation $\in$ restricted to $M$. In this case
we simply refer to $M$ as a \emph{countable transitive model} or
\emph{ctm}.

The existence of a ctm of $\ZFC$ can be proved from the existence of a
model  $\lb N,E\rb$ such that the relation $E$ is well founded. The
L\"owenheim-Skolem 
Theorem ensures that there is an countable elementary submodel 
$\lb N',E\restr N'\rb\preccurlyeq  \lb N,E\rb$ which must also be
well founded; then the 
Mostowski collapsing function \cite[Def.~I.9.31]{kunen2011set} sends $\lb
N',E\restr N'\rb$ 
isomorphically to an $\lb M,\in\rb$ where $M$ is transitive.

By G\"odel's Second Incompleteness Theorem we cannot  prove that
there exists a model of $\ZFC$  (assuming that our base theory is not
stronger than $\ZFC$ to start with), but for consistency proofs,
usually a model of only a finite subset of $\ZFC$ suffices. Hence the
following meta-theoretic result by Montague applies:
%
\begin{theorem}[Reflection Principle]\label{th:reflection-principle}
  For every finite $\Phi\sbq\ZFC$, $\ZFC$ proves: ``There exists
    unboundedly many $\al$ such that $V_\al\models \Phi$.''
\end{theorem}
%
Since the sets $V_\al$ are well founded, we can repeat the above
argument to obtain a ctm of $\Phi$.
%
%%% Local Variables: 
%%% mode: latex
%%% TeX-master: "Separation_In_MG"
%%% ispell-local-dictionary: "american"
%%% End: 


%%%%%%%%%%%%%%%%%%%%%%%%%%%%%%%%%%%%%%%%%%%%%%%%%%%%%%%%%%%%%%%%%%%%%%          
\section{Renaming}
\label{sec:renaming}
\newcommand{\renaming}[2]{(#1)[#2]}
\newcommand{\inFm}[2]{#1 \in #2}
\newcommand{\eqFm}[2]{#1 = #2}
\newcommand{\negFm}[1]{\neg #1}
\newcommand{\andFm}[2]{#1 \wedge #2}
\newcommand{\forallFm}[1]{\forall #1}

\newcommand{\inIFm}[2]{\mathsf{Member}(#1,#2)}
\newcommand{\eqIFm}[2]{\mathsf{Equal}(#1,#2)}
\newcommand{\nandIFm}[2]{\mathsf{Nand}(#1,#2)}
\newcommand{\forallIFm}[1]{\mathsf{Forall(#1)}}


In the course of our work we need to reason about renaming of formulas
and its effect on their satisfiability. Internalized formulas are
implemented using de Bruijn indices for variables and the arity of a
formula $\phi$ gives the least natural number containing all the free
variables in $\phi$. Following Fiore et al. \cite{fiore-abssyn}, one can
understand the arity of a formula as the context of the free
variables; notice that the arity of $\forallFm{\phi}$ is the
predecessor of the arity of $\phi$. Renamings are, consequently,
mappings between finite sets; since we can think of $\mathsf{succ}(n)$
as the coproduct $1+n = \{0\} \cup \{1,\dots,n\}$, then given a
renaming $f \colon n \to m$, the 
unique morphism $\mathsf{id}_1+f \colon 1+n \to 1+m$ is used to rename
free variables in a quantified formula. 

\begin{definition}[Renaming]
  Let $\phi$ be a formula of arity $n$ and let $f \colon n \to m$, the
  renaming of $\phi$ by $f$, denoted $\renaming{\phi}{f}$, is defined
  by recursion on $\phi$:
  \begin{gather*}
    \renaming{\inFm{i}{j}}{f} = \inFm{f\,i}{f\,j}\\
    \renaming{\eqFm{i}{j}}{f} = \eqFm{f\,i}{f\,j}\\
    \renaming{\negFm{\phi}}{f} = \negFm{\renaming{\phi}{f}}\\
    \renaming{\andFm{\phi}{\psi}}{f} = \andFm{\renaming{\phi}{f}}{\renaming{\psi}{f}}\\
    \renaming{\forallFm{\phi}}{f} = \forallFm{\renaming{\phi}{\mathsf{id}_1+f}}
  \end{gather*}
\end{definition}

As usual, if $M$ is a set, $a_0,\dots,a_{n-1}$ are elements of $M$, and
$\phi$ is a formula of arity $n$, we write
\[
M,[a_0,\dots,a_{n-1}] \models \phi
\]
to denote that $\phi$ is satisfied by $M$ when $i$ is interpreted
as $a_i$ ($i=0,\dots,n-1$). We call the list $[a_0,\dots,a_{n-1}]$ the
\emph{environment}.

The action of renaming on environments re-indexes the variables. An
easy proof connects satisfaction with renamings.
\begin{lemma}
  \label{lem:renaming}
  Let $\phi$ be a formula of arity $n$, $f \colon n \to m$ be a
  renaming, and let $\rho=[a_1,\ldots,a_n]$ and
  $\rho'=[b_1,\ldots,b_m]$ be environments of length $n$ and $m$,
  respectively. If for all $i \in n$, $a_i = b_{j}$ where $j=f\,i$,
  then $M,\rho\models \phi$ is equivalent to
  $M,\rho' \models \renaming{\phi}{f}$.
\end{lemma}

An important resource in Isabelle/ZF is the facility for defining
inductive sets \cite{paulson2000fixedpoint,paulson1995set} together
with a principle for defining functions by structural recursion.
Internalized formulas are a prime example of this, so we define
a function \isa{ren} that associates to each formula an internalized
function that can be later applied to suitable arguments. Notice that
Paulson used \isa{Nand} because it is more economical.
\begin{isabelle}
\isamarkuptrue%
\isacommand{consts}\isamarkupfalse%
\ ren\ {\isacharcolon}{\isacharcolon}\ {\isachardoublequoteopen}i{\isacharequal}{\isachargreater}i{\isachardoublequoteclose}\isanewline
\isacommand{primrec}\isamarkupfalse%
\isanewline
\ {\isachardoublequoteopen}ren{\isacharparenleft}Member{\isacharparenleft}x{\isacharcomma}y{\isacharparenright}{\isacharparenright}\ {\isacharequal}\isanewline
\ \ {\isacharparenleft}{\isasymlambda}\ n\ {\isasymin}\ nat\ {\isachardot}\ {\isasymlambda}\ m\ {\isasymin}\ nat{\isachardot}\ {\isasymlambda}f\ {\isasymin}\ n\ {\isasymrightarrow}\ m{\isachardot}\ Member\ {\isacharparenleft}f{\isacharbackquote}x{\isacharcomma}\ f{\isacharbackquote}y{\isacharparenright}{\isacharparenright}{\isachardoublequoteclose}\isanewline
\ \isanewline
\ {\isachardoublequoteopen}ren{\isacharparenleft}Equal{\isacharparenleft}x{\isacharcomma}y{\isacharparenright}{\isacharparenright}\ {\isacharequal}\isanewline
\ \ {\isacharparenleft}{\isasymlambda}\ n\ {\isasymin}\ nat\ {\isachardot}\ {\isasymlambda}\ m\ {\isasymin}\ nat{\isachardot}\ {\isasymlambda}f\ {\isasymin}\ n\ {\isasymrightarrow}\ m{\isachardot}\ Equal\ {\isacharparenleft}f{\isacharbackquote}x{\isacharcomma}\ f{\isacharbackquote}y{\isacharparenright}{\isacharparenright}{\isachardoublequoteclose}\isanewline
\ \isanewline
\ {\isachardoublequoteopen}ren{\isacharparenleft}Nand{\isacharparenleft}p{\isacharcomma}q{\isacharparenright}{\isacharparenright}\ {\isacharequal}\isanewline
\ \ {\isacharparenleft}{\isasymlambda}\ n\ {\isasymin}\ nat\ {\isachardot}\ {\isasymlambda}\ m\ {\isasymin}\ nat{\isachardot}\ {\isasymlambda}f\ {\isasymin}\ n\ {\isasymrightarrow}\ m{\isachardot}\ \isanewline
\ \ \ Nand\ {\isacharparenleft}ren{\isacharparenleft}p{\isacharparenright}{\isacharbackquote}n{\isacharbackquote}m{\isacharbackquote}f{\isacharcomma}\ ren{\isacharparenleft}q{\isacharparenright}{\isacharbackquote}n{\isacharbackquote}m{\isacharbackquote}f{\isacharparenright}{\isacharparenright}{\isachardoublequoteclose}\isanewline
\ \isanewline
\ {\isachardoublequoteopen}ren{\isacharparenleft}Forall{\isacharparenleft}p{\isacharparenright}{\isacharparenright}\ {\isacharequal}\isanewline
\ \  {\isacharparenleft}{\isasymlambda}\ n\ {\isasymin}\ nat\ {\isachardot}\ {\isasymlambda}\ m\ {\isasymin}\ nat{\isachardot}\ {\isasymlambda}f\ {\isasymin}\ n\ {\isasymrightarrow}\ m{\isachardot}\ \isanewline
\ \ \ Forall\ {\isacharparenleft}ren{\isacharparenleft}p{\isacharparenright}{\isacharbackquote}succ{\isacharparenleft}n{\isacharparenright}{\isacharbackquote}succ{\isacharparenleft}m{\isacharparenright}{\isacharbackquote}sum{\isacharunderscore}id{\isacharparenleft}n{\isacharcomma}f{\isacharparenright}{\isacharparenright}{\isacharparenright}{\isachardoublequoteclose}
\end{isabelle}

In the last equation, \isa{sum{\isacharunderscore}id} corresponds to
the coproduct morphism $\mathsf{id}_{1}+f \colon 1 + n \to 1 +
n$. Since the schema for recursively defined functions does not allow
parameters, we are forced to return a function of three arguments
(\isa{n,m,f}). This also exposes some inconveniences of working in the
untyped realm of set theory; for example to use \isa{ren} we will need
to prove that the renaming is a function. Besides some auxiliary
results (for example that the application of renaming to suitable
arguments yields a formula), the main result corresponding to
Lemma~\ref{lem:renaming} is:
\begin{isabelle}
\isacommand{lemma}\isamarkupfalse%
\ sats{\isacharunderscore}iff{\isacharunderscore}sats{\isacharunderscore}ren\ {\isacharcolon}\ \isanewline
\ \ \isakeyword{fixes}\ {\isasymphi}\isanewline
\ \ \isakeyword{assumes}\ {\isachardoublequoteopen}{\isasymphi}\ {\isasymin}\ formula{\isachardoublequoteclose}\isanewline
\ \ \isakeyword{shows}\ \ {\isachardoublequoteopen}{\isasymAnd}\ n\ m\ {\isasymrho}\ {\isasymrho}{\isacharprime}\ f\ {\isachardot}\ \isanewline
\ \ {\isasymlbrakk}n{\isasymin}nat\ {\isacharsemicolon}\ m{\isasymin}nat\ {\isacharsemicolon}\ f\ {\isasymin}\ n{\isasymrightarrow}m\ {\isacharsemicolon}\ arity{\isacharparenleft}{\isasymphi}{\isacharparenright}\ {\isasymle}\ n\ {\isacharsemicolon}\isanewline
\ \ \ \ \ {\isasymrho}\ {\isasymin}\ list{\isacharparenleft}M{\isacharparenright}\ {\isacharsemicolon}\ {\isasymrho}{\isacharprime}\ {\isasymin}\ list{\isacharparenleft}M{\isacharparenright}\ {\isacharsemicolon}\ \isanewline
\ \ \ {\isasymAnd}\ i\ {\isachardot}\ i{\isacharless}n\ {\isasymLongrightarrow}\ nth{\isacharparenleft}i{\isacharcomma}{\isasymrho}{\isacharparenright}\ {\isacharequal}\ nth{\isacharparenleft}f{\isacharbackquote}i{\isacharcomma}{\isasymrho}{\isacharprime}{\isacharparenright}\ {\isasymrbrakk}\ {\isasymLongrightarrow}\isanewline
\ \ sats{\isacharparenleft}M{\isacharcomma}{\isasymphi}{\isacharcomma}{\isasymrho}{\isacharparenright}\ {\isasymlongleftrightarrow}\ sats{\isacharparenleft}M{\isacharcomma}ren{\isacharparenleft}{\isasymphi}{\isacharparenright}{\isacharbackquote}n{\isacharbackquote}m{\isacharbackquote}f{\isacharcomma}{\isasymrho}{\isacharprime}{\isacharparenright}{\isachardoublequoteclose}\end{isabelle}

All our uses of this lemma involve concrete renamings on small
numbers, but we also tested it with more abstract ones for arbitrary
numbers. All the renamings of the first kind follow the same pattern
and, more importantly, share equal proofs. We would like to develop
some \texttt{ML} tools in order to automatize this.
%% We think that it should be possible, and clearer,
%% to express all the current renamings in the theory \isa{Formula} using
%% our approach. 


%%% Local Variables: 
%%% mode: latex
%%% TeX-master: "Separation_In_MG"
%%% ispell-local-dictionary: "american"
%%% End: 


%%%%%%%%%%%%%%%%%%%%%%%%%%%%%%%%%%%%%%%%%%%%%%%%%%%%%%%%%%%%%%%%%%%%%%          
\section{Generic extensions}

We will swiftly review some definitions in order to reach the concept
of \emph{generic extension}. Given a ctm $M$ of $\ZFC$, a forcing
notion $\lb\PP,\leq,\1\rb$, and a filter
$G\sbq\PP$, a new set $M[G]$ is defined. Each element $a\in M[G]$ is
determined by an element $\dot a$ of $M$. Actually, the structure of
each $\dot a$ is used to construct $a$. They are related by a
map $\val$ that takes $G$ a parameter:
\[
\val(G,\dot a) = a.
\] 
Then the extension is defined by the image of the map $\val(G,\cdot)$:
\[
M[G] \defi \{\val(G,\tau): \tau\in M\}.
\]
Metatheoretically, it is straightforward to see that $M[G]$ is a
transitive set that satisfies the axioms of Pairing and Union and
includes $M\cup\{G\}$. Nevertheless there are no a priori reasons for
$M[G]$ to satisfy the rest of the $\ZFC$ 
axioms. The original insight by Cohen was to define the notion of
\emph{genericity} for a filter $G\sbq\PP$ and to prove that whenever
$G$ is generic, $M[G]$ will satisfy $\ZFC$.

The Separation Axiom  is the first that requires the notion of
genericity and the use of the forcing machinery, which we review in
the Section~\ref{sec:forcing}.

%%% Local Variables: 
%%% mode: latex
%%% TeX-master: "Separation_In_MG"
%%% ispell-local-dictionary: "american"
%%% End: 


%%%%%%%%%%%%%%%%%%%%%%%%%%%%%%%%%%%%%%%%%%%%%%%%%%%%%%%%%%%%%%%%%%%%%%          
\subsection{Recursion and values of names}

The map $\val$ used in the definition of the generic extension is
characterized by the recursive equation
\begin{equation}
  \label{eq:val}
  \val(G,\tau) = \{val(G,\sigma) :\exists p \in\PP .%
  \lb\sigma,p\rb \in \tau \wedge p \in G \}
\end{equation}

As is well-known, the principle of  recursion on
well-founded relations \cite[p.~48]{kunen2011set} allows us to define
a recursive function $F \colon A\to A$ by choosing a well-founded
relation $R \subseteq A\times A$ and a functional
$H\colon A\times (A \to A) \to A$ satisfying
$F(a)=H(a,F\!\upharpoonright\!(R^{-1}(a)))$. \citet{paulson1995set}
made this principle available in Isabelle/ZF via the the operator
\isa{wfrec}. The formalization of the corresponding functional
$\mathit{Hv}$ for $\val$ is straightforward:
%
\begin{isabelle}
\isacommand{definition}\isamarkupfalse%
\isanewline
\ \ Hv\ {\isacharcolon}{\isacharcolon}\ {\isachardoublequoteopen}i{\isasymRightarrow}i{\isasymRightarrow}i{\isasymRightarrow}i{\isachardoublequoteclose}\ \isakeyword{where}\isanewline
\ \ {\isachardoublequoteopen}Hv{\isacharparenleft}G{\isacharcomma}y{\isacharcomma}f{\isacharparenright}\ {\isacharequal}{\isacharequal}\ {\isacharbraceleft}f{\isacharbackquote}x\ {\isachardot}{\isachardot}\ x{\isasymin}domain{\isacharparenleft}y{\isacharparenright}{\isacharcomma}\ {\isasymexists}p{\isasymin}P{\isachardot}\ {\isacharless}x{\isacharcomma}p{\isachargreater}\ {\isasymin}\ y\ {\isasymand}\ p\ {\isasymin}\ G\ {\isacharbraceright}{\isachardoublequoteclose}
\end{isabelle}
In the references \cite{kunen2011set,weaver2014forcing} $\val$ is
applied only to \emph{names}, that are certain elements of $M$
characterized by a recursively defined predicate. The well-founded
relation used to justify Equation~\eqref{eq:val} is
\[ x \mathrel{\mathit{ed}} y \iff \exists p . \lb x,p\rb\in y. \] In
order to use \isa{wfrec} the relation should be expressed as a set, so
in \cite{2018arXiv180705174G} we originally took the restriction of
$\mathit{ed}$ to the whole universe 
$M$; i.e. $\mathit{ed}\cap M\times M$.  Although this decision was
adequate for that work, we now required more flexibility (for
instance, in order to apply $\val$ to arguments that we can't assume
that are in $M$, see Eq.~(\ref{eq:val-of-m}) below).

The remedy is to restrict $\mathit{ed}$ to the
transitive closure of the actual parameter:
\begin{isabelle}
\isacommand{definition}\isamarkupfalse%
\isanewline
\ val\ {\isacharcolon}{\isacharcolon}\ {\isachardoublequoteopen}i{\isasymRightarrow}i{\isasymRightarrow}i{\isachardoublequoteclose}\ \isakeyword{where}\isanewline
\ {\isachardoublequoteopen}val{\isacharparenleft}G{\isacharcomma}{\isasymtau}{\isacharparenright}{\isacharequal}{\isacharequal}\ wfrec{\isacharparenleft}edrel{\isacharparenleft}eclose{\isacharparenleft}{\isacharbraceleft}{\isasymtau}{\isacharbraceright}{\isacharparenright}{\isacharparenright}{\isacharcomma}{\isasymtau}{\isacharcomma}Hv{\isacharparenleft}G{\isacharparenright}{\isacharparenright}{\isachardoublequoteclose}
\end{isabelle}

In order to show that this definition satisfies~(\ref{eq:val}) we had
to supplement the existing recursion tools with a key, albeit
intuitive, result stating that when computing the value of a recursive 
function on some argument $a$, one can restrict the relation to some
ambient set if it includes $a$ and all of its predecessors.
\begin{isabelle}
\isacommand{lemma}\isamarkupfalse%
\ wfrec{\isacharunderscore}restr\ {\isacharcolon}\isanewline
\ \ \isakeyword{assumes}\ {\isachardoublequoteopen}relation{\isacharparenleft}r{\isacharparenright}{\isachardoublequoteclose}\ {\isachardoublequoteopen}wf{\isacharparenleft}r{\isacharparenright}{\isachardoublequoteclose}\ \isanewline
\ \ \isakeyword{shows}\ \ {\isachardoublequoteopen}a{\isasymin}A\ {\isasymLongrightarrow}\ {\isacharparenleft}r{\isacharcircum}{\isacharplus}{\isacharparenright}{\isacharminus}{\isacharbackquote}{\isacharbackquote}{\isacharbraceleft}a{\isacharbraceright}\ {\isasymsubseteq}\ A\ {\isasymLongrightarrow}\ \isanewline
\ \ \ \ \ \ \ \ \ \ wfrec{\isacharparenleft}r{\isacharcomma}a{\isacharcomma}H{\isacharparenright}\ {\isacharequal}\ wfrec{\isacharparenleft}r{\isasyminter}A{\isasymtimes}A{\isacharcomma}a{\isacharcomma}H{\isacharparenright}{\isachardoublequoteclose}
\end{isabelle}
As a consequence, we are able to formalize Equation~(\ref{eq:val}) as follows:
\begin{isabelle}
  \isacommand{lemma}\isamarkupfalse%
  \ def{\isacharunderscore}val{\isacharcolon}\isanewline
  \ {\isachardoublequoteopen}val{\isacharparenleft}G{\isacharcomma}x{\isacharparenright}\ {\isacharequal}\ {\isacharbraceleft}val{\isacharparenleft}G{\isacharcomma}t{\isacharparenright}\ {\isachardot}{\isachardot}\ t{\isasymin}domain{\isacharparenleft}x{\isacharparenright}\ {\isacharcomma}\isanewline
\ \ \ \ \ \ \ \ \ \ \ \ \ \ \ \ \ \ \ {\isasymexists}p{\isasymin}P{\isachardot}\ {\isacharless}t{\isacharcomma}p{\isachargreater}{\isasymin}x\ {\isasymand}\ p{\isasymin}G\ {\isacharbraceright}{\isachardoublequoteclose}
\end{isabelle}
and the monotonicity of $\val$ follows automatically after a
substitution.
\begin{isabelle}
\isacommand{lemma}\isamarkupfalse%
\ val{\isacharunderscore}mono{\isacharcolon}\ {\isachardoublequoteopen}x{\isasymsubseteq}y\ {\isasymLongrightarrow}\ val{\isacharparenleft}G{\isacharcomma}x{\isacharparenright}\ {\isasymsubseteq}\ val{\isacharparenleft}G{\isacharcomma}y{\isacharparenright}{\isachardoublequoteclose}\isanewline
%
\ \ \isacommand{by}\isamarkupfalse%
\ {\isacharparenleft}subst\ {\isacharparenleft}{\isadigit{1}}\ {\isadigit{2}}{\isacharparenright}\ def{\isacharunderscore}val{\isacharcomma}\ force{\isacharparenright}%
\end{isabelle}
More interestingly we can give a neat equation for values of
names defined by Separation, say $B = \{x\in A\times \PP.\ Q(x)\}$,
then
\begin{equation}
\val(G,B) = \{\val(G,t) : t\in A , \exists p\in \PP \cap G.\ Q(\lb t,p\rb) \} \label{eq:val-name-sep}
\end{equation}

We close our discussion of names and their values by making explicit
the names for elements in $M$; once more, we refer to
\cite{2018arXiv180705174G} for our formalization. The definition of
$\chk(x)$ is a straightforward $\in$-recursion:
\begin{equation*}
  \label{eq:def-check}
  \chk(x)\defi\{\lb\chk(y),\1\rb : y\in x\}
\end{equation*}
An easy $\in$-induction shows $\val(G,\chk(x))=x$.
But to conclude $M\subseteq M[G]$ one also needs to have
$\chk(x) \in M$; this result requires the internalization of
recursively defined functions. This is also needed to prove
$G \in M[G]$; let us define
$\dot G= \{\lb \chk(p),p\rb : p \in \PP \}$, it is easy to prove
$\val(G,\dot G) = G$. Proving $\dot G\in M$ involves knowing
$\chk(x) \in M$ and using one instance of Replacement.

Paulson proved absoluteness results for definitions by recursion and
one of our next goals is to instantiate at $\#\#M$ the appropriate
locale 
\isa{M{\isacharunderscore}eclose} which is the last layer of a pile of
locales. It will take us some time to prove that any ctm of $\ZF$ 
satisfies the
assumptions involved in those locales; as we mentioned, Paulson's work
is mostly done \emph{externally}, i.e. the assumptions are instances
of Separation and Replacement where the predicates and functions are
Isabelle functions of type \isa{i{\isasymRightarrow}i} and
\isa{i{\isasymRightarrow}o}, respectively. In contrast, we assume that
$M$ is a model of $\ZF$, therefore to deduce that $M$ satisfies a
Separation instance, we have to define an internalized formula whose
satisfaction is equivalent to the external predicate (cf. the
interface described in Section~\ref{sec:axioms-models-set-theory} and
also the concrete example given in the proof of Union below).

In the meantime, we declare a locale
\isa{M{\isacharunderscore}extra{\isacharunderscore}assms} assembling
both assumptions ($M$ being closed under $\chk$ and the instance of
Replacement); in this paper we explicitly mention where we use them.

%%% Local Variables: 
%%% mode: latex
%%% TeX-master: "Separation_In_MG"
%%% ispell-local-dictionary: "american"
%%% End: 


% %%%%%%%%%%%%%%%%%%%%%%%%%%%%%%%%%%%%%%%%%%%%%%%%%%%%%%%%%%%%%%%%%%%%%%          
\section{Prior results}

Paulson's work provides an appropriate starting point for our
developments.

There are other formalizations of set theory (Isabelle/HOL, Mizar, Automath).

To the best of our knowledge, only preliminary approaches to the
formalization of forcing exist; v.g. \cite{Quirin}, which has already been
discussed in our previous work \cite{2018arXiv180705174G}. 

%%% Local Variables: 
%%% mode: latex
%%% TeX-master: "Separation_In_MG"
%%% ispell-local-dictionary: "american"
%%% End: 


%%%%%%%%%%%%%%%%%%%%%%%%%%%%%%%%%%%%%%%%%%%%%%%%%%%%%%%%%%%%%%%%%%%%%%          
\section{Hacking of \isatt{ZF-Constructible}}
\label{sec:hack-constructible}
In \cite{paulson_2003}, Paulson presented his formalization of the
relative consistency of the Axiom of Choice. This development is
included inside the Isabelle distribution with session name
\isatt{ZF-Constructible}. The main technical devices, invented by
G\"odel for this purpose, are \emph{relativization} and
\emph{absoluteness}. In a nutshell, to relativize a formula $\phi$ to
a class $C$, it is enough to restrict its quantifiers to $C$. The
example of \isatt{upair\_ax} in
Section~\ref{sec:axioms-models-set-theory}, the relativized version of
the Pairing Axiom, is extracted from \texttt{Relative}, one of the
core theories of \isatt{ZF-Constructible}. On the other hand, $\phi$
is \emph{absolute} for $C$ if it is equivalent to its relativization,
meaning that the statement made $\phi$ coincides with what $C$
``believes'' $\phi$ is saying. Paulson shows that under certain
hypotheses  on a class $M$ (condensed in the locale \isatt{M\_trivial}), a plethora of
absoluteness and closure results can be proved about $M$.

The development of forcing, and the study of ctms in general, takes
absoluteness as a starting point. We were not able to work with
\isatt{ZF-Constructible} right out-of-the-box. The main reason is that
we can't expect state the ``class version'' of Replacement for a
\emph{set} $M$ by
using first-order formulas, since predicates \isatt{P::"i=>o"} can't
be proved to be only the definable ones. Therefore, we had to make
some modifications to the various sets of hypotheses in several
locales to make the results available as tools for the present and
future developments.

%% There are several lemmas that were declared, in later developments, as
%% introduction/simplification rules (notably, the rule
%% \isatt{equalityI}). They raised a warning, and we have 
%% eliminated some of them, but in some cases we had to keep them because
%% the proof works because it is insisted that the rule is \emph{safe}.

The most notable changes, located in the theory \texttt{Relative}, are
the following:
\begin{enumerate}
\item\label{item:1} We eliminated the requirement that the relative Axiom of Replacement
  is satisfied by a class $M$ to be in the locale \isatt{M\_trivial}. 
\item\label{item:2} We moved the requirement of the Powerset Axiom to \isatt{M\_basic}. 
\item\label{item:3} We replaced the need that the set of natural numbers is in $M$ by the
  milder hypothesis that $M(0)$. Actually, most results should follow
  by only assuming that $M$ is nonempty.
\end{enumerate}

As a consequence of Item~\ref{item:1}, the lemma
\isatt{strong\_replacementI} is no longer valid and was commented
out.

We moved the requirement $M(\mathtt{nat})$ to the locale
\isatt{M\_trancl} (inside the theory \isatt{WF\_absolute}), where it is needed for the first time. Some results,
for instance \isatt{rtran\_closure\_mem\_iff} and 
\isatt{iterates\_imp\_wfrec\_replacement} had to be moved inside that
locale.

The proof, for instance, that the constructible universe $L$ satisfies
the modified locale \isatt{M\_trivial} holds with minor
modifications. Nevertheless, in order to have a neater presentation,
we have stripped off several sections concerning $L$ from the theories
\isatt{L\_axioms} and \isatt{Internalize}, and we merged them to form
the new file  \isatt{Internalizations}. 

%%% Local Variables: 
%%% mode: latex
%%% TeX-master: "Separation_In_MG"
%%% ispell-local-dictionary: "american"
%%% End: 


%%%%%%%%%%%%%%%%%%%%%%%%%%%%%%%%%%%%%%%%%%%%%%%%%%%%%%%%%%%%%%%%%%%%%%          
\section{Extensionality, Foundation, Union, Infinity}
\label{sec:easy-axioms}
\newcommand{\quantRel}[3]{#1 #2\kern -1pt[#3]}
\newcommand{\forallRel}[2]{\quantRel{\forall}{#1}{#2}}
\newcommand{\existsRel}[2]{\quantRel{\exists}{#1}{#2}}

It is straightforward to show that the generic extension $M[G]$
satisfies extensionality and foundation. Showing that it is closed
under Union depends on $G$ being a filter. On the other hand, infinity
takes some more effort.

% To say that $A$ satisfies some axiom, say extensionality, means
% that the relativized version of extensionality is true in $A$:
% \[
% \forallRel{x}{A}. \forallRel{y}{A}. (\forallRel{w}{A} . w \in x \leftrightarrow w\in y) \rightarrow x = y
% \]

For Extensionality in $M[G]$, the assumption 
$\forall w[M[G]]. w\in x \leftrightarrow w\in y$ yields 
$\forall w. w\in x \leftrightarrow w\in y$ by transitivity of $M[G]$. %\footnote{We generalized slightly some results by Paulson:
%  in fact, several basic lemmas are valid for any transitive class
%  (for instance, absoluteness of Union). We plan to define a new
%  hierarchy of locales for ZF where transitive classes will be at the
%  base.}
Therefore, by (ambient) Extensionality we conclude $x=y$. 

Foundation for $M[G]$ does not depend on $M[G]$ being transitive: in
this case we take $x\in M[G]$ and prove, relativized to $M[G]$,  that there is an
$\in$\kern -1pt-minimal element in $x$. Instantiating the global Foundation
Axiom for $x\cap M[G]$ we get a minimal $y$, so it is still minimal
when considered relative to $M[G]$. 

It is noteworthy that the proofs in the Isar dialect of Isabelle
strictly follow the argumentation of the two previous paragraphs.

The Union Axiom asserts that if $x$ is a set, then there exists
another set (the union of $x$) containing all the elements in each
element of $x$. The relativized version of Union asks to give a name
$\pi_a$ for each $a\in M[G]$ and proving $\val(G,\pi_a)=\union a$.
Let $\tau$ be the name for $a$, i.e.\ $a=\val(G,\tau)$; 
\citet{kunen2011set} gives $\pi_a$ in terms of $\tau$:
\begin{align*}
  \pi_a = \{\langle\theta,p \rangle :  %&\in \dom(\union(\dom(\tau))) \times \PP : \\
\exists \langle\sigma,q\rangle  \in \tau .
 \exists r . \langle \theta,r\rangle \in \sigma \wedge
    p\leqslant r \wedge p \leqslant q \}
\end{align*}
Our formal definition is slightly different in order to ease the proof of
$\pi_a \in M$.  Since $\pi_a$ it is defined using Separation, one
needs to internalize $\pi_a$ as a formula
\isa{union{\isacharunderscore}name{\isacharunderscore}fm}. The
equation $\val(G,\pi_a)=\union a$ is proved by showing the mutual
inclusion; in both cases one uses that $G$ is a filter.
\begin{isabelle}
  \isacommand{lemma}\isamarkupfalse%
\ Union{\isacharunderscore}MG{\isacharunderscore}Eq\ {\isacharcolon}\ \isanewline
\ \ \isakeyword{assumes}\ {\isachardoublequoteopen}a\ {\isasymin}\ M{\isacharbrackleft}G{\isacharbrackright}{\isachardoublequoteclose}\ \isakeyword{and}\ {\isachardoublequoteopen}a\ {\isacharequal}\ val{\isacharparenleft}G{\isacharcomma}{\isasymtau}{\isacharparenright}{\isachardoublequoteclose}\ \isakeyword{and}\isanewline
\ \ \ \ \ \ \ \ \ \ {\isachardoublequoteopen}filter{\isacharparenleft}G{\isacharparenright}{\isachardoublequoteclose}\ \isakeyword{and}\ {\isachardoublequoteopen}{\isasymtau}\ {\isasymin}\ M{\isachardoublequoteclose}\isanewline
\ \ \isakeyword{shows}\ {\isachardoublequoteopen}{\isasymUnion}\ a\ {\isacharequal}\ val{\isacharparenleft}G{\isacharcomma}Union{\isacharunderscore}name{\isacharparenleft}{\isasymtau}{\isacharparenright}{\isacharparenright}{\isachardoublequoteclose}
\end{isabelle}
Since Union is absolute for any transitive class we may conclude that
$M[G]$ is closed under Union:
\begin{isabelle}
  \isacommand{lemma}\isamarkupfalse%
  \ union{\isacharunderscore}in{\isacharunderscore}MG\ {\isacharcolon}\ \isanewline
  \ \ \isakeyword{assumes}\ {\isachardoublequoteopen}filter{\isacharparenleft}G{\isacharparenright}{\isachardoublequoteclose}\isanewline
\ \ \isakeyword{shows}\ {\isachardoublequoteopen}Union{\isacharunderscore}ax{\isacharparenleft}{\isacharhash}{\isacharhash}M{\isacharbrackleft}G{\isacharbrackright}{\isacharparenright}{\isachardoublequoteclose}
\end{isabelle}

The proof of Infinity for $M[G]$ takes advantage of some absoluteness
results proved in the locale \isa{M{\isacharunderscore}trivial}. Since
we have already proved that $M[G]$ is transitive, $\emptyset\in M[G]$
assuming $G$ being generic, and also that it satisfies Pairing and
Union, we can instantiate \isa{M{\isacharunderscore}trivial}:
\begin{isabelle}
\isacommand{sublocale}\isamarkupfalse%
\ G{\isacharunderscore}generic\ {\isasymsubseteq}\ M{\isacharunderscore}trivial{\isachardoublequoteopen}{\isacharhash}{\isacharhash}M{\isacharbrackleft}G{\isacharbrackright}{\isachardoublequoteclose}
\end{isabelle}
We assume that $M$ satisfies Infinity,\footnote{This is equivalent to
  say that there is an $I\in M$ having an empty set relativized to $M$
  and closed under successor, again relativized to $M$.} therefore we
obtain $I \in M$ such that $\emptyset\in I$ and, $x \in I$ implies
$\mathit{succ}(x)\in I$; this follows by absoluteness of empty and
succesor for $M$. Using one further assumption, namely that
$\chk(x) \in M$ if $x\in M$, we deduce $\val(G,\chk(I)) = I \in M[G]$.
Now we can use absoluteness of emptyness and successor, this time for
$M[G]$, to conclude that $M[G]$ satisfies Infinity.
\begin{isabelle}
  \isacommand{lemma}\isamarkupfalse%
\ infinty{\isacharunderscore}in{\isacharunderscore}MG\ {\isacharcolon}\ {\isachardoublequoteopen}infinity{\isacharunderscore}ax{\isacharparenleft}{\isacharhash}{\isacharhash}M{\isacharbrackleft}G{\isacharbrackright}{\isacharparenright}{\isachardoublequoteclose}
\end{isabelle}
\noindent\fbox{En Generic Extensions decimos que es fácil ver que $M[G]$}
\fbox{ ... incluye a $M \cup {G}$, por qué no sacar $I\in M[G]$ de ahí?}

%%% Local Variables: 
%%% mode: latex
%%% TeX-master: "Separation_In_MG"
%%% ispell-local-dictionary: "american"
%%% End: 


%%%%%%%%%%%%%%%%%%%%%%%%%%%%%%%%%%%%%%%%%%%%%%%%%%%%%%%%%%%%%%%%%%%%%%          
\section{Set models and forcing}
\label{sec:forcing}

\subsection{The $\ZFC$ axioms as locales}\label{sec:zfc-axioms-as-locales}
The description of set models of fragments of $\ZFC$ was performed
using Isabelle contexts (\emph{locales}) that fix a variable $M::\tyi$
and pack assumptions stating that $\lb M, \in\rb$ satisfy some
axioms. For example, the locale \isatt{M{\uscore}Z{\uscore}basic}
states that Zermelo set theory holds in $M$. The Union Axiom (\isatt{Union{\uscore}ax}), for
instance, is defined as follows:
\[
\forall x[\isatt{\#\#}M].\ \exists z[\isatt{\#\#}M].\ \isatt{big{\uscore}union}(\isatt{\#\#}M,x,z)
\]
%% % Same a above
%% \begin{isabelle}%
%% {\isasymforall}x{\isacharbrackleft}{\kern0pt}{\isacharhash}{\kern0pt}{\isacharhash}{\kern0pt}M{\isacharbrackright}{\kern0pt}{\isachardot}{\kern0pt}\ {\isasymexists}z{\isacharbrackleft}{\kern0pt}{\isacharhash}{\kern0pt}{\isacharhash}{\kern0pt}M{\isacharbrackright}{\kern0pt}{\isachardot}{\kern0pt}\ big{\isacharunderscore}{\kern0pt}union{\isacharparenleft}{\kern0pt}{\isacharhash}{\kern0pt}{\isacharhash}{\kern0pt}M{\isacharcomma}{\kern0pt}\ x{\isacharcomma}{\kern0pt}\ z{\isacharparenright}{\kern0pt}%
%% \end{isabelle}%
using relativized, relational versions of the axioms of Isabelle/ZF
since the interaction with \session{ZF-Constructible} was smoother
that way and, as it was mentioned in Section~\ref{sec:isabellezf},
this third layer of the formalization has more tools at our
disposal. Later, an equivalent statement “in the codes” (our fourth
layer) is obtained,
\[
  \isatt{Union{\uscore}ax}(\isatt{\#\#}M) \lsii A, [\,]\models \isatt{{\isasymcdot}Union\ Ax{\isasymcdot}}
\]
%% % Same a above
%% \begin{isabelle}
%% Union{\isacharunderscore}{\kern0pt}ax{\isacharparenleft}{\kern0pt}{\isacharhash}{\kern0pt}{\isacharhash}{\kern0pt}A{\isacharparenright}{\kern0pt}\ {\isasymlongleftrightarrow}\ A{\isacharcomma}{\kern0pt}\ {\isacharbrackleft}{\kern0pt}{\isacharbrackright}{\kern0pt}\ {\isasymTurnstile}\ {\isasymcdot}Union\ Ax{\isasymcdot}\isasep
%% \end{isabelle}
where \isatt{{\isasymcdot}Union\ Ax{\isasymcdot}} is the $\formula$ code for the
Union axiom. For the axiom schemes, \session{ZF-Constructible} defines
the expressions
\[
  \isatt{separation}(N,Q)
  \text{ and }
  \isatt{strong{\uscore}replacement}(N,R)
\]
that consume a class $N$ and predicates $Q::\tyi\fun\tyo$ and $R::\tyi\fun\tyi\fun\tyo$, and state that $N$
satisfies the $Q$-instance of Separation and the $R$-instance of
Replacement, respectively. In the statement of the Separation Axiom in
\isatt{M{\uscore}Z{\uscore}basic} and
the many replacement instances, predicates $Q$ and $R$ are actually
defined as the satisfaction of formulas of the appropriate arity; we can only show
that this form of the axioms hold in generic extensions. 

Further locales gather the assumption of transitivity of $M$ and
particular replacement instances expressed by means of the
\isatt{replacement{\uscore}assm(M,\isasymphi)} predicate ($@$ denotes
list concatenation):
\begin{multline}\label{eq:replacement_assm_def}
\varphi \in \formula  \limp \mathit{env} \in \isatt{list}(M) \limp \isatt{arity}(\varphi) \leq 2+ \isatt{length}(\mathit{env}) \limp \\
 \isatt{strong{\uscore}replacement}(\isatt{\#\#} M, \lambda x\, y.\ (M, [x,y]
\mathbin{@} \mathit{env}  \models \varphi))
\end{multline}
%% % Same as above
%% \begin{isabelle}
%% {\isasymphi}\;{\isasymin}\;formula\ {\isasymlongrightarrow}\ env\;{\isasymin}\;list{\isacharparenleft}{\kern0pt}M{\isacharparenright}{\kern0pt}\ {\isasymlongrightarrow} arity{\isacharparenleft}{\kern0pt}{\isasymphi}{\isacharparenright}{\kern0pt}\;{\isasymle}\;{\isadigit{2}}\ {\isacharplus}{\kern0pt}\isactrlsub {\isasymomega}\ length{\isacharparenleft}{\kern0pt}env{\isacharparenright}{\kern0pt}\ {\isasymlongrightarrow}\isanewline
%% \ \ \ \ strong{\isacharunderscore}{\kern0pt}replacement{\isacharparenleft}{\kern0pt}{\isacharhash}{\kern0pt}{\isacharhash}{\kern0pt}M{\isacharcomma}{\kern0pt}{\isasymlambda}x\ y{\isachardot}{\kern0pt}\ {\isacharparenleft}{\kern0pt}M\ {\isacharcomma}{\kern0pt}\ {\isacharbrackleft}{\kern0pt}x{\isacharcomma}{\kern0pt}y{\isacharbrackright}{\kern0pt}{\isacharat}{\kern0pt}env\ {\isasymTurnstile}\ {\isasymphi}{\isacharparenright}{\kern0pt}{\isacharparenright}
%% \end{isabelle}
%
%% Some of those instances, which we included for simplicity of design,
%% are actually “fake” instances in that they can be obtained from
%% Powerset, or from other replacement instances already available in the
%% respective context. An example of this is the sole instance of the
%% \isatt{M{\uscore}basic} locale, originally from
%% \session{ZF-Constructible}; we managed to eliminate it but in
%% doing so we had to reprove many lemmas from
%% that session in the weakened context. 
The locales gathering all 32 replacement instances necessary are named
\isatt{M{\uscore}ZF1} through \isatt{M{\uscore}ZF4},
\isatt{M{\uscore}ZF{\uscore}ground},
\isatt{M{\uscore}ZF{\uscore}ground{\uscore}notCH}, and
\isatt{M{\uscore}ZF{\uscore}ground{\uscore}CH} (see
Section~\ref{sec:repl-instances} for details).

%% Missing: Locales have to be interpreted (part of "big picture"?)

\subsection{The fundamental theorems}
Let $\lb \PP, {\preceq} ,\1\rb \in M$ be a forcing notion. In order to
fix the notation, the
$\val$ interpretation function takes $3$ arguments so that if $G\sbq \PP$, we have
$M[G]\defi \{ \val(\PP,G,\punto{a}) : \punto{a}\in M \}$.

The version of the Forcing Theorems that we formalized follows the
considerations on the $\forces^*$ relation as discussed in Kunen's new
\emph{Set Theory}
\cite[p.~257ff]{kunen2011set}. But in contrast to the point made on
p.~260 of this book, we internalized the recursion to define the forcing relation,
in that it involves codes for formulas, and the meta-theoretic formula
transformer $\phi\mapsto\mathit{Forces}_\phi$ is replaced by the
set-theoretic class function $\forceisa:: \tyi \fun \tyi$, which was defined by using
Isabelle/ZF facilities for primitive recursion.

Next we state this version of the fundamental theorems in a compact
way.
\begin{theorem}\label{th:forcing-thms}
  There exists a function  $\forceisa$
  such that for every
  $\phi\in\formula$ and $\punto{a}_0,\dots,\punto{a}_n\in M$,
  \begin{enumerate}
  \item\label{item:definability} (Definability)
    $\forceisa(\phi)\in\formula$, 
  \end{enumerate}
  where the 
  arity of $\forceisa(\phi)$ is at most $\isatt{arity}(\phi) + 4$; and if
  “$p \forces \phi\ [\punto{a}_0,\dots,\punto{a}_n]$”
  denotes
  “$M, [p,\PP,\preceq,\1, \punto{a}_0,\dots,\punto{a}_n]  \models
  \forceisa(\phi)$”, we have:
  \begin{enumerate}
    \setcounter{enumi}{1}
  \item\label{item:truth-lemma} (Truth Lemma) for every $M$-generic $G$,
    \[
      \exists p\in G.\ \; p \forces \phi\ [\punto{a}_0,\dots,\punto{a}_n]
    \]
    is equivalent to 
    \[
      M[G], [\val(\PP,G,\punto{a}_0),\dots,\val(\PP,G,\punto{a}_n)]
      \models \phi.
    \]
  \item \label{item:density-lemma} (Density Lemma) $p \forces \phi\ [\punto{a}_0,\dots,\punto{a}_n]$
    if and only if 
    $\{q\in \PP :  q \forces \phi\ [\punto{a}_0,\dots,\punto{a}_n]\}$
    is dense below $p$.
  \end{enumerate}
\end{theorem}
We actually have to define $\forceisa$ before we state the fundamental
theorems, so the main existential quantifier above does not appear in the
formalization.
Moreover, the items in Theorem~\ref{th:forcing-thms} appear as three separated lemmas in
\theory{Forcing{\uscore}Theorems} of our
\session{Independence\_CH} session \cite{Independence_CH-AFP},
and benefit from the $\isatt{map}$ function that applies a function to
each element of a list. For instance, the Truth Lemma is stated as
follows:
\begin{isabelle}
\isacommand{lemma}\isamarkupfalse%
\ truth{\isacharunderscore}{\kern0pt}lemma{\isacharcolon}{\kern0pt}\isanewline
\ \ \isakeyword{assumes}\isanewline
\ \ \ \ {\isachardoublequoteopen}{\isasymphi}{\isasymin}formula{\isachardoublequoteclose}\ {\isachardoublequoteopen}M{\isacharunderscore}{\kern0pt}generic{\isacharparenleft}{\kern0pt}G{\isacharparenright}{\kern0pt}{\isachardoublequoteclose}\isanewline
\ \ \ \ {\isachardoublequoteopen}env{\isasymin}list{\isacharparenleft}{\kern0pt}M{\isacharparenright}{\kern0pt}{\isachardoublequoteclose}\ {\isachardoublequoteopen}arity{\isacharparenleft}{\kern0pt}{\isasymphi}{\isacharparenright}{\kern0pt}{\isasymle}length{\isacharparenleft}{\kern0pt}env{\isacharparenright}{\kern0pt}{\isachardoublequoteclose}\isanewline
\ \ \isakeyword{shows}\isanewline
\ \ \ \ {\isachardoublequoteopen}{\isacharparenleft}{\kern0pt}{\isasymexists}p{\isasymin}G{\isachardot}{\kern0pt}\ p\ {\isasymtturnstile}\ {\isasymphi}\ env{\isacharparenright}{\kern0pt}\ \ \ {\isasymlongleftrightarrow}\ \ \ M{\isacharbrackleft}{\kern0pt}G{\isacharbrackright}{\kern0pt}{\isacharcomma}{\kern0pt}\ map{\isacharparenleft}{\kern0pt}val{\isacharparenleft}{\kern0pt}P{\isacharcomma}{\kern0pt}G{\isacharparenright}{\kern0pt}{\isacharcomma}{\kern0pt}env{\isacharparenright}{\kern0pt}\ {\isasymTurnstile}\ {\isasymphi}{\isachardoublequoteclose}
\end{isabelle}
where the $\forces$ notation (and its precedence) had already been set up in the
\theory{Forces{\uscore}Definition} theory as follows:
\begin{isabelle}
\isacommand{abbreviation}\isamarkupfalse%
\ Forces\ {\isacharcolon}{\kern0pt}{\isacharcolon}{\kern0pt}\ {\isachardoublequoteopen}{\isacharbrackleft}{\kern0pt}i{\isacharcomma}{\kern0pt}\ i{\isacharcomma}{\kern0pt}\ i{\isacharbrackright}{\kern0pt}\ {\isasymRightarrow}\ o{\isachardoublequoteclose}\ \ {\isacharparenleft}{\kern0pt}{\isachardoublequoteopen}{\isacharunderscore}{\kern0pt}\ {\isasymtturnstile}\ {\isacharunderscore}{\kern0pt}\ {\isacharunderscore}{\kern0pt}{\isachardoublequoteclose}\ {\isacharbrackleft}{\kern0pt}{\isadigit{3}}{\isadigit{6}}{\isacharcomma}{\kern0pt}{\isadigit{3}}{\isadigit{6}}{\isacharcomma}{\kern0pt}{\isadigit{3}}{\isadigit{6}}{\isacharbrackright}{\kern0pt}\ {\isadigit{6}}{\isadigit{0}}{\isacharparenright}{\kern0pt}\ \isakeyword{where}\isanewline
\ \ {\isachardoublequoteopen}p\ {\isasymtturnstile}\ {\isasymphi}\ env\ \ \ {\isasymequiv}\ \ \ M{\isacharcomma}{\kern0pt}\ {\isacharparenleft}{\kern0pt}{\isacharbrackleft}{\kern0pt}p{\isacharcomma}{\kern0pt}P{\isacharcomma}{\kern0pt}leq{\isacharcomma}{\kern0pt}{\isasymone}{\isacharbrackright}{\kern0pt}\ {\isacharat}{\kern0pt}\ env{\isacharparenright}{\kern0pt}\ {\isasymTurnstile}\ forces{\isacharparenleft}{\kern0pt}{\isasymphi}{\isacharparenright}{\kern0pt}{\isachardoublequoteclose}\isanewline
\end{isabelle}

Kunen first describes forcing for atomic formulas using a mutual
recursion
%% \begin{multline*}
%%   \forceseq (p,t_1,t_2) \defi 
%%   \forall s\in\dom(t_1)\cup\dom(t_2).\ \forall q\pleq p .\\
%%   \forcesmem(q,s,t_1)\lsii \forcesmem(q,s,t_2)
%% \end{multline*}
%% \begin{multline*}
%%   \forcesmem(p,t_1,t_2) \defi  \forall v\pleq p. \ \exists q\pleq v.\\  
%%   \exists s.\ \exists r\in \PP .\ \lb s,r\rb \in  t_2 \land q
%%   \pleq r \land \forceseq(q,t_1,s)
%% \end{multline*}
but then \cite[p.~257]{kunen2011set} it is cast as a single
recursively defined function $F$ over a wellfounded  relation $R$.
In our formalization, these are called $\frcat$ and 
$\isatt{frecR}$, respectively, and are defined on tuples $\lb \mathit{ft},t_1,t_2,p\rb$ (where
$\mathit{ft}\in\{0,1\}$ indicates whether the atomic formula being
forced is an equality or a membership, respectively).
Forcing for general formulas is then defined by recursion on the
datatype $\formula$ as indicated above. Technical details on the
implementation and proofs of the
Forcing Theorems have been spelled out in our
\cite{2020arXiv200109715G}.

%%% Local Variables: 
%%% mode: latex
%%% TeX-master: "independence_ch_isabelle"
%%% ispell-local-dictionary: "american"
%%% End: 


%%%%%%%%%%%%%%%%%%%%%%%%%%%%%%%%%%%%%%%%%%%%%%%%%%%%%%%%%%%%%%%%%%%%%%          
\section{Proof of Separation}

This proof can be found in the file \verb|Separation_Axiom.thy| of the
development, which we proceed to discuss.

The key technical result is the following:
\begin{isabelle}
  \isacommand{lemma}\isamarkupfalse%
  \ Collect{\isacharunderscore}sats{\isacharunderscore}in{\isacharunderscore}MG\ {\isacharcolon}\isanewline
  \ \ \isakeyword{assumes}\isanewline
  \ \ \ \ {\isachardoublequoteopen}{\isasympi}\ {\isasymin}\ M{\isachardoublequoteclose}\ {\isachardoublequoteopen}{\isasymsigma}\ {\isasymin}\ M{\isachardoublequoteclose}\ {\isachardoublequoteopen}val{\isacharparenleft}G{\isacharcomma}\ {\isasympi}{\isacharparenright}\ {\isacharequal}\ c{\isachardoublequoteclose}\ {\isachardoublequoteopen}val{\isacharparenleft}G{\isacharcomma}\ {\isasymsigma}{\isacharparenright}\ {\isacharequal}\ w{\isachardoublequoteclose}\isanewline
  \ \ \ \ {\isachardoublequoteopen}{\isasymphi}\ {\isasymin}\ formula{\isachardoublequoteclose}\ {\isachardoublequoteopen}arity{\isacharparenleft}{\isasymphi}{\isacharparenright}\ {\isasymle}\ {\isadigit{2}}{\isachardoublequoteclose}\isanewline
  \ \ \isakeyword{shows}\ \ \ \ \isanewline
  \ \ \ \ {\isachardoublequoteopen}{\isacharbraceleft}x{\isasymin}c{\isachardot}\ sats{\isacharparenleft}M{\isacharbrackleft}G{\isacharbrackright}{\isacharcomma}\ {\isasymphi}{\isacharcomma}\ {\isacharbrackleft}x{\isacharcomma}\ w{\isacharbrackright}{\isacharparenright}{\isacharbraceright}{\isasymin}\ M{\isacharbrackleft}G{\isacharbrackright}{\isachardoublequoteclose}
\end{isabelle}
%
From this, using absoluteness, we will be able to derive the
$\phi$-instance of Separation. 

To show that   
\[
S\defi\{x\in c : M[G],[x,w]\models \phi(x_0,x_1)\} \in M[G],
\]
it is enough to provide a name $n\in M$ for this set.
 
The candidate name is
\[
n \defi \{u \in\dom(\pi)\times\PP :M,[u,\PP,\leq,\1,\sig,\pi]\models \psi\}
\]
where
\[
\psi \defi \exists \th\, p.\ x_0=\lb\th,p\rb \y 
   \forceisa(\th\in x_5\y\phi(\th,x_4)).
\]
The fact that $n\in M$ follows by an application of a six-variable
instance of Separation in $M$ (lemma \isatt{six{\isacharunderscore}sep{\isacharunderscore}aux}).

Almost a third part of the proof involves the syntactic handling of
internalized formulas and permutation of variables. The more
substantive portion concerns proving that actually $\val(G,n)=S$.

Let's first focus into the predicate 
\begin{equation}\label{eq:1}
M,[u,\PP,\leq,\1,\sig,\pi]\models \psi
\end{equation}
defining $n$ by separation. By definition of the satisfaction
relation and %% permuting variables
absoluteness, we have that it is equivalent to the fact
that there exist $\th,p\in M$ with   $u=\lb\th,p\rb$  and 
\[
M,[\PP,\leq,\1,p,\th,\sig,\pi]\models \forceisa(x_4\in
x_6\y\phi(x_4,x_5)). 
\]
% (Note that the variable $x_7$ is not used.)
This, in turn, is equivalent by the Definition of Forcing to: \emph{For all $M$-generic
filters $F$ such that $p\in F$,} 
\begin{equation}\label{eq:2}
M[F],[\val(F,\th),\val(F,\sig),\val(F,\pi)]\models x_0\in
x_2\y\phi(x_0,x_1). 
\end{equation}
We can instantiate this statement with $G$ and obtain
\[
p\in G \impl M[G],[\val(G,\th),w,c]\models x_0\in
x_2\y\phi(x_0,x_1). 
\] 
Let $Q(\th,p)$ be the last statement. We have just seen that
(\ref{eq:1}) implies 
\[
\exists \th,p\in M.\ u=\lb\th,p\rb \y Q(\th,p).
\]
Hence $n$ is included in 
\[
m\defi \{u \in\dom(\pi)\times\PP : \exists \th,p\in M.\ u=\lb\th,p\rb
\y Q(\th,p)\}. 
\]

It can be seen that
\begin{lemma}
  $\val(G,m) = S$.
\end{lemma}
\noindent And by monotonicity of $\val$ we obtain
\begin{lemma}
  $\val(G,n)\sbq \val(G,m)$.
\end{lemma}
To complete the proof, it is therefore enough to show that
$S\sbq \val(G,n)$. For this, let $x\in S$. Hence there exists
$\lb\th,q\rb\in\pi$ such that 
$q\in G$ and $x=\val(G,\th)$. 

On the other hand, since 
\[
M[G],[\val(G,\th),\val(G,\sig),\val(G,\pi)]\models
 x_0\in x_2\y\phi( x_0, x_1),
\]
by the  Truth Lemma there must exist $r\in G$ such that
\[
M,[\PP,\leq,\1,r,\th,\sig,\pi]\models
\forceisa(x_4\in x_6\y\phi( x_4, x_5)).
\]
Since $G$ is a filter, there is $p\in G$ such that $p\leq q, r$.
By Strengthening, we have
\[
M,[\PP,\leq,\1,p,\th,\sig,\pi]\models
\forceisa(x_4\in x_6\y \phi( x_4, x_5)),
\]
which by the Truth Lemma gives us: \emph{for all $M$-generic $F$,
  $p\in F$ implies} 
\[
M[F],[\val(F,\th),\val(F,\sig),\val(F,\pi)]\models
 x_0\in  x_2 \y\phi( x_0, x_1).
\]
Note this is the same as (\ref{eq:2}). In fact, we have that
\begin{lemma}
  $\val(G,n)$ equals
  \[
  \{\val(G,\th): \th\in\dom(\pi) \y  p \in G 
  \y \text{(\ref{eq:2}) holds}\}.
  \]
\end{lemma}
This shows that $S\sbq \val(G,n)$, since $\th\in\dom(\pi)$ by construction. 
   
%%% Local Variables: 
%%% mode: latex
%%% TeX-master: "Separation_In_MG"
%%% ispell-local-dictionary: "american"
%%% End: 


% An example of a floating figure using the graphicx package.
% Note that \label must occur AFTER (or within) \caption.
% For figures, \caption should occur after the \includegraphics.
% Note that IEEEtran v1.7 and later has special internal code that
% is designed to preserve the operation of \label within \caption
% even when the captionsoff option is in effect. However, because
% of issues like this, it may be the safest practice to put all your
% \label just after \caption rather than within \caption{}.
%
% Reminder: the "draftcls" or "draftclsnofoot", not "draft", class
% option should be used if it is desired that the figures are to be
% displayed while in draft mode.
%
%\begin{figure}[!t]
%\centering
%\includegraphics[width=2.5in]{myfigure}
% where an .eps filename suffix will be assumed under latex, 
% and a .pdf suffix will be assumed for pdflatex; or what has been declared
% via \DeclareGraphicsExtensions.
%\caption{Simulation results for the network.}
%\label{fig_sim}
%\end{figure}

% Note that the IEEE typically puts floats only at the top, even when this
% results in a large percentage of a column being occupied by floats.


% An example of a double column floating figure using two subfigures.
% (The subfig.sty package must be loaded for this to work.)
% The subfigure \label commands are set within each subfloat command,
% and the \label for the overall figure must come after \caption.
% \hfil is used as a separator to get equal spacing.
% Watch out that the combined width of all the subfigures on a 
% line do not exceed the text width or a line break will occur.
%
%\begin{figure*}[!t]
%\centering
%\subfloat[Case I]{\includegraphics[width=2.5in]{box}%
%\label{fig_first_case}}
%\hfil
%\subfloat[Case II]{\includegraphics[width=2.5in]{box}%
%\label{fig_second_case}}
%\caption{Simulation results for the network.}
%\label{fig_sim}
%\end{figure*}
%
% Note that often IEEE papers with subfigures do not employ subfigure
% captions (using the optional argument to \subfloat[]), but instead will
% reference/describe all of them (a), (b), etc., within the main caption.
% Be aware that for subfig.sty to generate the (a), (b), etc., subfigure
% labels, the optional argument to \subfloat must be present. If a
% subcaption is not desired, just leave its contents blank,
% e.g., \subfloat[].


% An example of a floating table. Note that, for IEEE style tables, the
% \caption command should come BEFORE the table and, given that table
% captions serve much like titles, are usually capitalized except for words
% such as a, an, and, as, at, but, by, for, in, nor, of, on, or, the, to
% and up, which are usually not capitalized unless they are the first or
% last word of the caption. Table text will default to \footnotesize as
% the IEEE normally uses this smaller font for tables.
% The \label must come after \caption as always.
%
%\begin{table}[!t]
%% increase table row spacing, adjust to taste
%\renewcommand{\arraystretch}{1.3}
% if using array.sty, it might be a good idea to tweak the value of
% \extrarowheight as needed to properly center the text within the cells
%\caption{An Example of a Table}
%\label{table_example}
%\centering
%% Some packages, such as MDW tools, offer better commands for making tables
%% than the plain LaTeX2e tabular which is used here.
%\begin{tabular}{|c||c|}
%\hline
%One & Two\\
%\hline
%Three & Four\\
%\hline
%\end{tabular}
%\end{table}


% Note that the IEEE does not put floats in the very first column
% - or typically anywhere on the first page for that matter. Also,
% in-text middle ("here") positioning is typically not used, but it
% is allowed and encouraged for Computer Society conferences (but
% not Computer Society journals). Most IEEE journals/conferences use
% top floats exclusively. 
% Note that, LaTeX2e, unlike IEEE journals/conferences, places
% footnotes above bottom floats. This can be corrected via the
% \fnbelowfloat command of the stfloats package.



\section{Conclusions and future work}
There are several technical milestones that have to be reached in the
course of a formalization of the theory of forcing. The first one, and most
obvious, is the bulk of set- and meta-theoretical concepts needed to work
with. This pushed us, in a sense,  into building on top of Isabelle/ZF,
since we know of no other development in set theory of such
depth (and breadth). In this paper we worked on setting the stage for the work with
generic extensions; in particular, this involves some purely mathematical
results, as the Rasiowa-Sikorski lemma. 

Other milestones in this formalization project
involve 
\begin{enumerate}
\item the definition
  of the forcing relation, 
\item proving the Fundamental Theorem of forcing
  (that relates truth in $M$ to that in $M[G]$), and 
\item using it to show
  that $M[G]\models \ZFC$. 
\end{enumerate}
The theory is very modular and this is
witnessed by the fact 
that the last goal does not depend on the proof of the Fundamental
Theorem nor on the definition of the forcing relation. Our next task
will be to obtain the last goal in that enumeration. 

To this end, we will develop an interface between Paulson's
relativization results and countable models of $\ZFC$. This will show
that every ctm $M$ is closed under well-founded recursion and, in
particular, that contains names for each of its
elements. Consequently, the proof of  $M\sbq M[G]$ will be
complete. A landmark will be to prove the Axiom Scheme
of Separation (the first that needs to use the machinery of forcing
nontrivially). As a part of the new formalization, we will provide
Isar versions of the longer applicative proofs presented in this work.

\ack{We'd like to thank the anonymous referees for reading the paper
  carefully and for their detailed and constructive criticism.}
%%% Local Variables:
%%% mode: latex
%%% ispell-local-dictionary: "american"
%%% TeX-master: "first_steps_into_forcing"
%%% End:




% conference papers do not normally have an appendix


% use section* for acknowledgment



% trigger a \newpage just before the given reference
% number - used to balance the columns on the last page
% adjust value as needed - may need to be readjusted if
% the document is modified later
%\IEEEtriggeratref{8}
% The "triggered" command can be changed if desired:
%\IEEEtriggercmd{\enlargethispage{-5in}}

% references section

% can use a bibliography generated by BibTeX as a .bbl file
% BibTeX documentation can be easily obtained at:
% http://mirror.ctan.org/biblio/bibtex/contrib/doc/
% The IEEEtran BibTeX style support page is at:
% http://www.michaelshell.org/tex/ieeetran/bibtex/
\bibliographystyle{IEEEtranSN}
% argument is your BibTeX string definitions and bibliography database(s)
\bibliography{../LSFA/citados}
%
% <OR> manually copy in the resultant .bbl file
% set second argument of \begin to the number of references
% (used to reserve space for the reference number labels box)

% that's all folks
\end{document}


%%% Local Variables:
%%% mode: latex
%%% ispell-local-dictionary: "american"
%%% End:
