Elaboration on comments by Timothy Y. Chow
===========================================

On the relative consistency of $\neg\CH$
----------------------------------------

It is perhaps important to make clear that we have not formalized the
relative consistency of $\neg\CH$, and we are
actually not aiming for that in the short term. In our context, going
for the plain consistency result seems off the mark, since we
can not weaken our base theory (which is essentially equivalent to $\ZF$)%% ,
%% and we catch our tail.
Even if we intended to do so, the standard route to use ctms for a
relative consistency proof (through the Reflection Theorem) is
prohibitive for us, since our metatheory does not have the required
induction principles (on Isabelle/ZF formulas---in contrast to
formulas coded as sets, as in our presentation).


Work by Mathias
---------------

A summary is offered in \cite[Sect.~6]{kanamori-mathias}.

%%%%%%%%%%%%%%%%%%%%%%%%%%%%%%%%%%%%%%%%%%%%%%%%%%%%%%%%%

In \cite[Sect.~1]{mathias:hal-01188043}, models $M$ of Zermelo Set
Theory are offered for which each of the inclusions $M\subseteq M[G]$ and
$M\supseteq M[G]$ fail, where the poset $\pP$ is the trivial
$\{\1 \}$. In one of them, $K$, we have $\om\in K \sm K[G]$, hence the
ordinals do not coincide.

%%%%%%%%%%%%%%%%%%%%%%%%%%%%%%%%%%%%%%%%%%%%%%%%%%%%%%%%%

In \cite{mathias-provident}, a reasonably minimal fragment
$\mathsc{Prov}$ of $\ZF$ that allows to do set forcing is identified,
and transitive sets satisfying it are called
\emph{provident}. Existence of rank and of transitive closure are
among them; hence their appearance in our list seems
justified. Nevertheless, $\mathsc{Prov}$ is even a restriction of
Kripke-Platek Set Theory, and thus it does not include neither
Powerset nor full Separation. The detailed theory of provident sets is
developed in \cite{mathias-bowler-gentle}. % Its Section~0 is very
% readable.


%%%%%%%%%%%%%%%%%%%%%%%%%%%%%%%%%%%%%%%%%%%%%%%%%%%%%%%%%

