\subsection{The Rasiowa-Sikorski lemma}
In order to formalize of the proof of the Rasiowa-Sikorski lemma in
general, a version of the Axiom of Choice ($\AC$) must be proved. This
version is known as the \emph{Principle of Dependent choices}:
\begin{quote}
  ($\DC$) Let $\rr$ be a binary relation on $A$, and $a\in A$. If
  $\forall x\in A\,  \exists y\in A\, x\rr y$, then there exists
  $f:\om\to A$ such that $f(0)=a$ and $f(n)\rr f(n+1)$ for all
  $n\in\om$.
\end{quote}

A different version of $\DC$ (without the constraint $f(0)=a$, but
allowing a parameter that allows to prove an equivalence with $\AC$) was
formalized as part of the Isabelle/ZF library, along with many
statements that follow from or are equivalent to $\AC$. Since the
proofs there required a greater generality, it would have been very
contrived to derived this ``pointed'' version of $\DC$ from that
development. Instead, we preferred to give a direct proof based in the
one that appears in Moschovakis \cite{moschovakis1994notes}. Another
advantage is that this proof depends on the following version of $\AC$
(existence of choice functions on $\P(X)\sm \{\mty\}$):
\[
\exists f:\P(X)\sm \{\mty\}\to X \, \forall A\sbq X.\ A\neq \mty \limp
f(A) \in A.
\]
This statement is more standard than the one deemed ``$\AC$'' in the
Isabelle/ZF development, that involves a silent use of Replacement
since it is written as an axiom scheme.


%%% Local Variables:
%%% mode: latex
%%% ispell-local-dictionary: "american"
%%% TeX-master: "first_steps_into_forcing"
%%% End:
