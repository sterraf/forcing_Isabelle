\subsection{Related work}\label{sec:related-work}
Formalization of mathematics serves many purposes. The most obvious
one is to increase reliability in a result and/or its proof. This has
been the original motivation that lead Voevodsky to gather many
researchers around \emph{homotopy type theory} and its formalization in Coq
\cite{hottbook}; the same applies to the four colour theorem (checked by
Gonthier \cite{MR2463991}) and the formidable \emph{Flyspeck} project
\cite{MR3659768} by the team conducted by Hales.  

In our particular case, forcing and the set theoretic techniques that
are being formalized can be regarded as a mature technology and thus
the main goal is not to increase confidence. Nevertheless, the level
of detail in a formalization of this sort always provide additional
information about the inner workings of the theory: It is expected, for
instance, to have a detailed account of which axioms are necessary to
define and use forcing. Finally, we support the vision that a growing
corpus of formalized mathematics can be a useful library for the
future generations. The question of how to systematize this corpus is
an ongoing project by Paulson \cite{ALEXANDRIA}.

We will now discuss very succinctly recent formalizations of
set theory and forcing. The closest formalizations are those based on
Isabelle. Let us remark that Isabelle allows for different logical
foundations; in particular, Paulson carried out his formalizations
on top of Isabelle/FOL which is based on first order logic.

There is another major framework in Isabelle based on higher order
logic, Isabelle/HOL. This framework is very active, and as a 
consequence more automated tools are available. Isabelle/HOL has 
basic chapters on set theory. One of those, by Steven Obua, proceeds up to
well founded relations and provides translations between types in HOL
(for instance \isatt{nat}) to  sets  (elements of type
\isatt{ZF}). Another one, by A.~Popescu and D.~Traytel, reaches
cardinal artihmetic. This is fairly limited for our purposes.

Concerning automation, B.~Zhan has developed a new tool called
\texttt{auto2} and applied it to untyped set theory
\cite{10.1007/978-3-319-66107-0_32}. He has redeveloped essentially
the basic results in Isabelle/ZF, but goes in a different
direction. Nevertheless, a majority of results in
Isabelle/ZF are not yet implemented using this tool, and  another
downside is that proofs using it do not follow the standard Isar
language (see Section~\ref{sec:isabellezf}).

As far as we know, there is little progress on formalizations of
forcing in type theory. Most relevant is the work by K.~Quirin
\cite{Quirin}, where a sheaf-theoretic initial approach to forcing is
implemented in Coq. This language is extremely different to the
standard approach of constructing models of $\ZFC$, and it might be
difficult (once the forcing machinery is set) to translate results in
the literature using ctms to this one. In any case, the translation to
set theory of what Quirin accomplishes is to define a generic
extension (where $\CH$ should fail) and to construct a set $K$ (a
candidate counterexample) and injections $\N\hookrightarrow K$ and
$K\hookrightarrow\R$. But the most important part, that is, that there
are no surjections $\N\twoheadrightarrow K$ and
$K\twoheadrightarrow\R$, is left for a future
work. % Finally, Jaber et al.\ \cite{jaber:hal-01319066}
% port forcing to type theory; notice that this is not a formalization
% of forcing. Instead, they construct new logics from type theory
% by means of forcing conditions.

%%% Local Variables:
%%% mode: latex
%%% ispell-local-dictionary: "american"
%%% TeX-master: "first_steps_into_forcing"
%%% End:
