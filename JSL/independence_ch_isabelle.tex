% This is samplepaper.tex, a sample chapter demonstrating the
% LLNCS macro package for Springer Computer Science proceedings;
% Version 2.20 of 2017/10/04
%
\documentclass[runningheads]{llncs}
%
\usepackage[utf8]{inputenc}
\usepackage{isabelle_indepCH,isabellesym_indepCH}
\usepackage{amsmath,amsfonts,amssymb}
\usepackage{bbm}  % Para el \bb{1}
\usepackage{tikz}
\usepackage[english]{babel}
\usepackage{multidef}
\usepackage{verbatim}
\usepackage{stmaryrd} %% para \llbracket
\usepackage{hyperref}
\usepackage{xcolor}
\usepackage{framed}
\usepackage[numbers]{natbib}

%%
%% \usepackage[bottom=2cm, top=2cm, left=2cm, right=2cm]{geometry}
%% \usepackage{titling}
%% \setlength{\droptitle}{-10ex} 
%%
\renewcommand{\o}{\vee}
\renewcommand{\O}{\bigvee}
\newcommand{\y}{\wedge}
\newcommand{\Y}{\bigwedge}
\newcommand{\limp}{\longrightarrow}
\newcommand{\lsii}{\longleftrightarrow}
%%
%\newcommand{\DeclareMathOperator}[2]{\newcommand{#1}{\mathop{\mathrm{#2}}}}

\DeclareMathOperator{\cf}{cf}
\DeclareMathOperator{\dom}{domain}
\DeclareMathOperator{\im}{img}
\DeclareMathOperator{\Fn}{Fn}
\DeclareMathOperator{\rk}{rk}
\DeclareMathOperator{\mos}{mos}
\DeclareMathOperator{\trcl}{trcl}
\DeclareMathOperator{\Con}{Con}
\DeclareMathOperator{\Club}{Club}


\newcommand{\modelo}[1]{\mathbf{#1}}
\newcommand{\axiomas}[1]{\mathit{#1}}
\newcommand{\clase}[1]{\mathsf{#1}}
\newcommand{\poset}[1]{\mathbb{#1}}
\newcommand{\operador}[1]{\mathbf{#1}}

%% \newcommand{\Lim}{\clase{Lim}}
%% \newcommand{\Reg}{\clase{Reg}}
%% \newcommand{\Card}{\clase{Card}}
%% \newcommand{\On}{\clase{On}}
%% \newcommand{\WF}{\clase{WF}}
%% \newcommand{\HF}{\clase{HF}}
%% \newcommand{\HC}{\clase{HC}}
%%
%% El siguiente comando reemplaza todos los anteriores:
%%
\multidef{\clase{#1}}{Card,HC,HF,Lim,On->Ord,Reg,WF,Ord}
\newcommand{\ON}{\On}

%% En lugar de usar todo el paquete bbm:
\DeclareMathAlphabet{\mathbbm}{U}{bbm}{m}{n} 
\newcommand{\1}{\mathbbm{1}}
\newcommand{\PP}{\mathbbm{P}}

%%
%% \newcommand{\calD}{\mathcal{D}}
%% \newcommand{\calS}{\mathcal{S}}
%% \newcommand{\calU}{\mathcal{U}}
%% \newcommand{\calB}{\mathcal{B}}
%% \newcommand{\calL}{\mathcal{L}}
%% \newcommand{\calF}{\mathcal{F}}
%% \newcommand{\calT}{\mathcal{T}}
%% \newcommand{\calW}{\mathcal{W}}
%% \newcommand{\calA}{\mathcal{A}}
%%
%% El siguiente comando reemplaza todos los anteriores:
%%
\multidef[prefix=cal]{\mathcal{#1}}{A-Z}
%%
%% \newcommand{\A}{\modelo{A}}
%% \newcommand{\BB}{\modelo{B}}
%% \newcommand{\ZZ}{\modelo{Z}}
%% \newcommand{\PP}{\modelo{P}}
%% \newcommand{\QQ}{\modelo{Q}}
%% \newcommand{\RR}{\modelo{R}}
%%
%% El siguiente comando reemplaza todos los anteriores:
%%
\multidef{\modelo{#1}}{A,BB->B,CC->C,NN->N,QQ->Q,RR->R,ZZ->Z}

\multidef[prefix=p]{\mathbb{#1}}{A-Z}
%% \newcommand{\B}{\modelo{B}}
%% \newcommand{\C}{\modelo{C}}
%% \newcommand{\F}{\modelo{F}}
%% \newcommand{\D}{\modelo{D}}

\newcommand{\Th}{\mb{Th}}
\newcommand{\Mod}{\mb{Mod}}

\newcommand{\Se}{\operador{S^\prec}}
\newcommand{\Pu}{\operador{P_u}}
\renewcommand{\Pr}{\operador{P_R}}
\renewcommand{\H}{\operador{H}}
\renewcommand{\S}{\operador{S}}
\newcommand{\I}{\operador{I}}
\newcommand{\E}{\operador{E}}

\newcommand{\se}{\preccurlyeq}
\newcommand{\ee}{\succ}
\newcommand{\id}{\approx}
\newcommand{\subm}{\subseteq}
\newcommand{\ext}{\supseteq}
\newcommand{\iso}{\cong}
%%
\renewcommand{\emptyset}{\varnothing}
\newcommand{\rel}{\mathcal{R}}
\newcommand{\Pow}{\mathop{\mathcal{P}}}
\renewcommand{\P}{\Pow}
\newcommand{\BP}{\mathrm{BP}}
\newcommand{\func}{\rightarrow}
\newcommand{\ord}{\mathrm{Ord}}
\newcommand{\R}{\mathbb{R}}
\newcommand{\N}{\mathbb{N}}
\newcommand{\Z}{\mathbb{Z}}
\renewcommand{\I}{\mathbb{I}}
\newcommand{\Q}{\mathbb{Q}}
\newcommand{\B}{\mathbf{B}}
\newcommand{\lb}{\langle}
\newcommand{\rb}{\rangle}
\newcommand{\impl}{\rightarrow}
\newcommand{\ent}{\Rightarrow}
\newcommand{\tne}{\Leftarrow}
\newcommand{\sii}{\Leftrightarrow}
\renewcommand{\phi}{\varphi}
\newcommand{\phis}{{\varphi^*}}
\renewcommand{\th}{\theta}
\newcommand{\Lda}{\Lambda}
\newcommand{\La}{\Lambda}
\newcommand{\lda}{\lambda}
\newcommand{\ka}{\kappa}
\newcommand{\del}{\delta}
\newcommand{\de}{\delta}
\newcommand{\ze}{\zeta}
%\newcommand{\ }{\ }
\newcommand{\la}{\lambda}
\newcommand{\al}{\alpha}
\newcommand{\be}{\beta}
\newcommand{\ga}{\gamma}
\newcommand{\Ga}{\Gamma}
\newcommand{\ep}{\varepsilon}
\newcommand{\De}{\Delta}
\newcommand{\defi}{\mathrel{\mathop:}=}
\newcommand{\forces}{\Vdash}
%\newcommand{\ap}{\mathbin{\wideparen{\ }}}
\newcommand{\Tree}{{\mathrm{Tr}_\N}}
\newcommand{\PTree}{{\mathrm{PTr}_\N}}
\newcommand{\NWO}{\mathit{NWO}}
\newcommand{\Suc}{{\N^{<\N}}}%
\newcommand{\init}{\mathsf{i}}
\newcommand{\ap}{\mathord{^\smallfrown}}
\newcommand{\Cantor}{\mathcal{C}}
%\newcommand{\C}{\Cantor}
\newcommand{\Baire}{\mathcal{N}}
\newcommand{\sig}{\ensuremath{\sigma}}
\newcommand{\fsig}{\ensuremath{F_\sigma}}
\newcommand{\gdel}{\ensuremath{G_\delta}}
\newcommand{\Sig}{\ensuremath{\boldsymbol{\Sigma}}}
\newcommand{\bPi}{\ensuremath{\boldsymbol{\Pi}}}
\newcommand{\Del}{\ensuremath{\boldsymbol\Delta}}
%\renewcommand{\F}{\operador{F}}
\newcommand{\ths}{{\theta^*}}
\newcommand{\om}{\ensuremath{\omega}}
%\renewcommand{\c}{\complement}
\newcommand{\comp}{\mathsf{c}}
\newcommand{\co}[1]{\left(#1\right)^\comp}
\newcommand{\len}[1]{\left|#1\right|}
\DeclareMathOperator{\tlim}{\overline{\mathrm{TLim}}}
\newcommand{\card}[1]{{\left|#1\right|}}
\newcommand{\bigcard}[1]{{\bigl|#1\bigr|}}
%
% Cardinality
%
\newcommand{\lec}{\leqslant_c}
\newcommand{\gec}{\geqslant_c}
\newcommand{\lc}{<_c}
\newcommand{\gc}{>_c}
\newcommand{\eqc}{=_c}
\newcommand{\biy}{\approx}
\newcommand*{\ale}[1]{\aleph_{#1}}
%
\newcommand{\Zerm}{\axiomas{Z}}
\newcommand{\ZC}{\axiomas{ZC}}
\newcommand{\AC}{\axiomas{AC}}
\newcommand{\DC}{\axiomas{DC}}
\newcommand{\MA}{\axiomas{MA}}
\newcommand{\CH}{\axiomas{CH}}
\newcommand{\ZFC}{\axiomas{ZFC}}
\newcommand{\ZF}{\axiomas{ZF}}
\newcommand{\Inf}{\axiomas{Inf}}
%
% Cardinal characteristics
%
\newcommand{\cont}{\mathfrak{c}}
\newcommand{\spl}{\mathfrak{s}}
\newcommand{\bound}{\mathfrak{b}}
\newcommand{\mad}{\mathfrak{a}}
\newcommand{\tower}{\mathfrak{t}}
%
\renewcommand{\hom}[2]{{}^{#1}\hskip-0.116ex{#2}}
\newcommand{\pred}[1][{}]{\mathop{\mathrm{pred}_{#1}}}
%% Postfix operator with supressable space:
%% \newcommand*{\iseg}{\relax\ifnum\lastnodetype>0 \mskip\medmuskip\fi{\downarrow}} %
\newcommand*{\iseg}{{\downarrow}}
\newcommand{\rr}{\mathrel{R}}
\newcommand{\restr}{\upharpoonright}
%\newcommand{\type}{\mathtt{}}
\newcommand{\app}{\mathop{\mathrm{Aprox}}}
\newcommand{\hess}{\triangleleft}
\newcommand{\bx}{\bar{x}}
\newcommand{\by}{\bar{y}}
\newcommand{\bz}{\bar{z}}
\newcommand{\union}{\mathop{\textstyle\bigcup}}
\newcommand{\sm}{\setminus}
\newcommand{\sbq}{\subseteq}
\newcommand{\nsbq}{\subseteq}
\newcommand{\mty}{\emptyset}
\newcommand{\dimg}{\text{\textup{``}}} % direct image
\newcommand{\quine}[1]{\ulcorner{\!#1\!}\urcorner}
%\newcommand{\ntrm}[1]{\textsl{\textbf{#1}}}
\newcommand{\Null}{\calN\!\mathit{ull}}
\DeclareMathOperator{\club}{Club}
\DeclareMathOperator{\otp}{otp}
\DeclareMathOperator{\val}{\mathit{val}}
\DeclareMathOperator{\chk}{\mathit{check}}
\DeclareMathOperator{\edrel}{\mathit{edrel}}
\DeclareMathOperator{\eclose}{\mathit{eclose}}
\DeclareMathOperator{\Memrel}{\mathit{Memrel}}
\renewcommand{\PP}{\mathbb{P}}
\renewcommand{\app}{\mathrm{App}}
\newcommand{\formula}{\isatt{formula}}
\newcommand{\tyi}{\isatt{i}}
\newcommand{\tyo}{\isatt{o}}
\newcommand{\forceisa}{\mathop{\mathtt{forces}}}
\newcommand{\equ}{\mathbf{e}}
\newcommand{\bel}{\mathbf{b}}
\newcommand{\atr}{\mathit{atr}}
\newcommand{\concat}{\mathbin{@}}
\newcommand{\dB}[1]{\mathbf{#1}}
\newcommand{\ed}{\mathrel{\isatt{ed}}}
\newcommand{\frecR}{\mathrel{\isatt{frecR}}}
\newcommand{\forceseq}{\mathop{\isatt{forces{\isacharunderscore}eq}}}
\newcommand{\forcesmem}{\mathop{\isatt{forces{\isacharunderscore}mem}}}
\newcommand{\forcesat}{\mathop{\isatt{forces{\isacharunderscore}at}}}
\newcommand{\pleq}{\preceq}
%\renewcommand{\isacharunderscorekeyword}{\mbox{\_}}
%\renewcommand{\isacharunderscore}{\mbox{\_}}
\renewcommand{\isasymtturnstile}{\isamath{\Vdash}}
\renewcommand{\isacharminus}{-}
\newcommand{\uscore}{\isacharunderscore}
\newcommand{\session}[1]{\textit{#1}}
\newcommand{\theory}[1]{\texttt{#1}}
\newcommand{\punto}[1]{\overset{\tikz\draw[fill=black] (0,0) circle (0.6pt);}{#1}}

%%%%%%%%%%%%%%%%%%%%%%%%%
% Variant aleph, beth, etc
% From http://tex.stackexchange.com/q/170476/69595
\makeatletter
\@ifpackageloaded{txfonts}\@tempswafalse\@tempswatrue
\if@tempswa
  \DeclareFontFamily{U}{txsymbols}{}
  \DeclareFontFamily{U}{txAMSb}{}
  \DeclareSymbolFont{txsymbols}{OMS}{txsy}{m}{n}
  \SetSymbolFont{txsymbols}{bold}{OMS}{txsy}{bx}{n}
  \DeclareFontSubstitution{OMS}{txsy}{m}{n}
  \DeclareSymbolFont{txAMSb}{U}{txsyb}{m}{n}
  \SetSymbolFont{txAMSb}{bold}{U}{txsyb}{bx}{n}
  \DeclareFontSubstitution{U}{txsyb}{m}{n}
  \DeclareMathSymbol{\aleph}{\mathord}{txsymbols}{64}
  \DeclareMathSymbol{\beth}{\mathord}{txAMSb}{105}
  \DeclareMathSymbol{\gimel}{\mathord}{txAMSb}{106}
  \DeclareMathSymbol{\daleth}{\mathord}{txAMSb}{107}
\fi
\makeatother

%%%%%%%%%%%%%%%%%%%%%%%%%%%%%%%%%%%%%%%%%%%%%%%%%%%%%%%%%%%%
%%
%% Theorem Environments
%%
% \newtheorem{theorem}{Theorem}
% \newtheorem{lemma}[theorem]{Lemma}
% \newtheorem{prop}[theorem]{Proposition}
% \newtheorem{corollary}[theorem]{Corollary}
% \newtheorem{claim}{Claim}
% \newtheorem*{claim*}{Claim}
% \theoremstyle{definition}
% \newtheorem{definition}[theorem]{Definition}
% \newtheorem{remark}[theorem]{Remark}
% \newtheorem{example}[theorem]{Example}
% \theoremstyle{remark}
% \newtheorem*{remark*}{Remark}

%%%%%%%%%%%%%%%%%%%%%%%%%%%%%%%%%%%%%%%%%%%%%%%%%%%%%%%%%%%%%%%%%%%%%%

%% \newenvironment{inducc}{\begin{list}{}{\itemindent=2.5em \labelwidth=4em}}{\end{list}}
%% \newcommand{\caso}[1]{\item[\fbox{#1}]}
\newenvironment{proofofclaim}{\begin{proof}[Proof of Claim]}{\end{proof}}

\newcommand{\quantRel}[3]{#1 #2\kern -1pt[#3]}
\newcommand{\forallRel}[2]{\quantRel{\forall}{#1}{#2}}
\newcommand{\existsRel}[2]{\quantRel{\exists}{#1}{#2}}

\newif\ifarXiv
\newif\ifIEEE

%%% Local Variables: 
%%% mode: latex
%%% TeX-master: "independence_ch_isabelle"
%%% End: 

\usepackage{graphicx}
% Used for displaying a sample figure. If possible, figure files should
% be included in EPS format.
%
\hypersetup{
  colorlinks,
  urlcolor={blue},
  linkcolor={blue!50!black},
  citecolor={blue!50!black},
}
% If you use the hyperref package, please uncomment the following line
% to display URLs in blue roman font according to Springer's eBook style:
% \renewcommand\UrlFont{\color{blue}\rmfamily}

\begin{document}
%
\title{Some lessons after the formalization of the ctm approach to forcing%
  \thanks{Supported by Secyt-UNC project 33620180100465CB and Conicet.}%
}
%
\titlerunning{Lessons after formalizing ctm forcing}
% If the paper title is too long for the running head, you can set
% an abbreviated paper title here
%
\author{Emmanuel Gunther\inst{1} \and
Miguel Pagano\inst{1} \and \\
Pedro Sánchez Terraf\inst{1,2}%\orcidID{0000-0003-3928-6942}
\and
Matías Steinberg\inst{1}
}
%
\authorrunning{E.~Gunther, M.~Pagano, P.~Sánchez Terraf, M.~Steinberg}
% First names are abbreviated in the running head.
% If there are more than two authors, 'et al.' is used.
%
\institute{Universidad Nacional de C\'ordoba. 
  \\  Facultad de Matem\'atica, Astronom\'{\i}a,  F\'{\i}sica y
  Computaci\'on.
  \and
    Centro de Investigaci\'on y Estudios de Matem\'atica (CIEM-FaMAF),
    Conicet. C\'ordoba. Argentina. \\
    \email{\{gunther,pagano,sterraf\}@famaf.unc.edu.ar\\
        matias.steinberg@mi.unc.edu.ar}
}
%
\maketitle              % typeset the header of the contribution
%
\begin{abstract}
  We'll discuss some highlights of our computer-verified
  proof of the construction, given a countable transitive set model $M$
  of $\ZFC$, of generic extensions  satisfying $\ZFC+\neg\CH$ and $\ZFC+\CH$.
  In particular,
  we isolated a set $\Delta$ of 39 instances
  of Axiom of Replacement and a function $F$
  such that such that for any finite fragment $\Phi\sbq\ZFC$,
  $F(\Phi)\sbq\ZFC$ is also finite and if
  $M\models F(\Phi) + \Delta$ then $M[G]\models \Phi + \neg \CH$.
  We also obtained the formulas yielded by the Forcing Definability Theorem
  explicitly.

  To achieve this, we worked in the proof assistant \emph{Isabelle},
  basing our development on the theory Isabelle/ZF by L.~Paulson and
  others.

  %% The vantage point of the talk will be that of a mathematician but
  %% elements from the computer science perspective will be
  %% present. Perhaps some myths regarding what can effectively be done
  %% using  proof assistants/checkers will
  %% be dispelled.

  We'll also compare our formalization with the recent one by Jesse
  M.~Han and Floris van Doorn in the proof assistant \emph{Lean}.

  \keywords{forcing \and Isabelle/ZF \and countable transitive models
    \and absoluteness \and generic extension \and constructibility.}
\end{abstract}
%
%
%
\section{Introduction}
\label{sec:introduction}

This paper is the culmination of our project on the computerized
formalization of the undecidability of the Continuum Hypothesis
($\CH$) from Zermelo-Fraenkel set theory with Choice ($\ZFC$), under the
assumption of the existence of a countable transitive model (ctm) of
$\ZFC$. In contrast to our reports of the previous steps towards this
goal
\cite{2018arXiv180705174G,2019arXiv190103313G,2020arXiv200109715G}, we
intend here to present our development to the mathematical logic
community. For this reason, we start with a general discussion around
the formalization of mathematics.

\subsection{Formalized mathematics}
The use of computers to assist the creation and verification of
mathematics has seen a steady grow. But the general awareness on the
matter still seems to be a bit scant (even among mathematicians
involved in foundations), and the venues devoted to the communication
of formalized mathematics are, mainly, computer science journals and
conferences: JAR, ITP, IJCAR, CPP, CICM, and others.

Nevertheless, the discussion about the subject in central mathematical
circles is increasing; there were some hints on the ICM2018 panel on
“machine-assisted” proofs
\cite{https://doi.org/10.48550/arxiv.1809.08062} and a lively
promotion by Kevin Buzzard, during his ICM2022 special plenary lecture
\cite{2021arXiv211211598B}.

%% These assistants provide several dialects, among which we single out:
%% \begin{enumerate}
%% \item Procedural: Useful for exploration/research.
%% \item Declarative: Only one that can be read by humans!
%% \end{enumerate}

Before we start an in-depth discussion, a point should be made clear:
A formalized proof is not the same as an \emph{automatic proof}. The
reader surely understands that, aside from results of a very specific sort, no current
technology allows us to write a reasonably complex (and correct)
theorem statement in a computer and expect to obtain a proof after hitting “Enter”, at
least not after a humanly feasible wait. On the other hand, it is
quite possible that the same reader has some mental image that
formalizing a proof requires making each application of Modus Ponens
explicit.

The fact is that \emph{proof assistants} are designed for the human prover to
be able to decompose a statement to be proved into smaller subgoals
which can actually be fed into some automatic tool. The balance between
what these tools are able to handle is not  easily appreciated by
intuition: Sometimes, ``trivial'' steps are not solved by them, which
can result in obvious frustration; but they would quickly solve some
goals that do not look like a ``mere computation.''

To appreciate the extent of mathematics formalizable, it is convenient to recall
some major successful projects, such as the Four Color Theorem
\cite{MR2463991}, the Odd Order Theorem
\cite{10.1007/978-3-642-39634-2_14}, and the proof the Kepler's
Conjecture \cite{MR3659768}. There is a vast mathematical corpus at
the Archive of Formal Proofs (AFP) based on Isabelle; and formalizations of
brand new mathematics like the Liquid Tensor Experiment
\cite{LTE2020,LTE2021} and the definition of perfectoid spaces \cite{10.1145/3372885.3373830}
have been achieved using Lean.

We will continue our description of proof assistants in
Section~\ref{sec:proof-assist-isabelle}. We kindly invite the reader
to enrich the previous exposition by reading the apt summary by
A.~Koutsoukou-Argyraki \cite{angeliki} and the interviews
therein; some of the experts consulted have also discussed
in \cite{2022arXiv220704779B} the status of formalized versus standard
proof in mathematics.

\subsection{Our achievements}
We formalized a model-theoretic rendition of forcing (Sect.~\ref{sec:forcing}), showing how to
construct proper extensions of ctms of $\ZF$ (respectively, with
$\AC$), and we formalized the basic forcing notions required to obtain
ctms of $\ZFC + \neg\CH$ and of $\ZFC + \CH$ (Sect.~\ref{sec:models-ch-negation}). No metatheoretic issues
(consistency, FOL calculi, etc) were formalized, so we were mainly
concerned with the mathematics of forcing. Nevertheless, by inspecting
the foundations underlying our proof assistant Isabelle
(Section~\ref{sec:isabelle-metalogic-meta}) it can be stated that our
formalization is a bona fide proof in $\ZF$ of the previous
constructions.

In order to reach our goals, we provided basic results that were
missing from Isabelle's $\ZF$ library, starting from ones
involving cardinal successors, countable sets, etc.
(Section~\ref{sec:extension-isabellezf}). We also extended the treatment of relativization of
set-theoretical concepts (Section~\ref{sec:tools-relativization}).
%% We redesigned Isabelle/ZF results on non-absolute concepts to work
%% relative to a class.

One added value that is obtained from the present formalization is
that we identified a handful of instances of Replacement which are
sufficient to set the forcing machinery up
(Section~\ref{sec:repl-instances}), on the basis of Zermelo set theory.
The eagerness to obtain this level of detail might be a consequence of
“an unnatural tendency to investigate, for the most part, trivial
minutiae of the formalism” on our part, as it was put by Cohen
\cite{zbMATH02012060}, but we would rather say that we were driven by
curiosity.

The code of our formalization can be accessed at the
AFP site, via the following link:
\begin{center}
  \url{https://www.isa-afp.org/entries/Independence_CH.html}
\end{center}

%%% Local Variables: 
%%% mode: latex
%%% TeX-master: "independence_ch_isabelle"
%%% ispell-local-dictionary: "american"
%%% End: 

%% 
%% \section{Isabelle and (meta)theories}

\begin{enumerate}
\item \emph{Pure} (meta-theory).
\item Isabelle/ZF + $\exists M$ ctm of $\ZF(C)$ (theory). 
\item $\formula$ (inner theory).
\item \textbf{Theorem}: $\exists N$ ctm of $\ZF(C)+\neg\CH$.
\end{enumerate}

in comparison to the ``proof theory approach'':

\begin{enumerate}
\item Primitive recursive arithmetic (meta-theory).
\item \textbf{Theorem}: Con($\ZF(C)$) $\implies$ Con($\ZF(C)+\neg\CH$).
\end{enumerate}

and then to the ``type-theoretic approach'':

\begin{enumerate}
\item CiC (meta-theory).
\item CiC (theory, $\geq \ZFC + \exists \kappa$ inaccessible).
\item \textbf{Theorem}: Con($\ZF(C)+\neg\CH$).
\end{enumerate}

%%% Local Variables: 
%%% mode: latex
%%% TeX-master: "forcing_in_isabelle_zf"
%%% ispell-local-dictionary: "american"
%%% End: 

%% 
%% \section{Relativization,  absoluteness, and the axioms}
\label{sec:relat-absol}

The concepts of relativization and absoluteness (due to Gödel, in his
proof of the relative consistency of $\AC$ \cite{godel-L}) 
are both
prerrequisites and powerful tools in working with transitive
models. A \emph{class} is simply a predicate $C(x)$ with at least one
free variable $x$.
The \emph{relativization} $\phi^C(\bx)$ of a set-theoretic
definition
$\phi$ (of a relation such
as ``$x$ is a subset of $y$'' or of a function like $y=\P(x)$) to
a class $C$ is obtained by restricting all of its quantifiers to $C$.

\[
x \sbq^C y \equiv \forall z.\ C(z) \longrightarrow (z\in x
\longrightarrow z\in y)
\]

The new formula $\phi^C(\bx)$ corresponds to what is obtained by defining
the concept ``inside'' $C$. In fact, for a class corresponding to a
set $c$ (i.e.\ $C(x) \defi x \in c$), the relativization $\phi^C$ of a 
sentence $\phi$ is equivalent to the satisfaction of $\phi$ in the
first-order model $\lb c, \in\rb$.

It turns out that many concepts mean the
same after relativization to a nonempty transitive class $C$; formally
\[
\forall\bx.\ C(\bx) \longrightarrow (\phi^C(\bx) \longleftrightarrow
\phi(\bx))
\]
When this is the case, we say that the relation defined by $\phi$ is
\emph{absolute for transitive models}. (The absoluteness of functions
moreover requires that
the relativized definition also behaves functionally over $C$.) As
examples, the relation of inclusion $\subseteq$ ---and actually, any
relation defined by a formula (equivalent to one) using only bounded
quantifiers 
$(\forall x\in y)$ and $(\exists x\in y)$--- is absolute for
transitive models. On the contrary, this is not the case with the powerset
operation.

A benefit of the work with transitive models is that many 
concepts concepts (pairs, unions, and fundamentally ordinals) are
uniform across the universe \isatt{i}, a ctm (of an adequate fragment of
$\ZF$) $M$ and any of its extensions $M[G]$

A part of this project is to refactorize Paulson's formalization
\cite{paulson_2003} of Gödel's \cite{godel-L}. The main objective is
to maximize applicability of the relativization machinery by adjusting
the hypothesis of a greater part of it early development. Paulson's
architecture had only in mind the consistency of $\ZFC$, but, for
instance, in order to apply it in the development of forcing, too much
is assumed at the beginning; more seriously, some assumptions can't be
regarded as ``first-order'' (v.g. the Replacement Scheme).

The version of \isatt{ZF-Constructible} we present weakens the
assumptions of many absoluteness theorems to that
of a nonempty transitive class; also there are some stronger results
such as the relativization of powersets.

Apart from the axiom schemes, the $\ZFC$ 
axioms are stated as predicates on classes (that is,
of type \isatt{(i{\isasymRightarrow}o){\isasymRightarrow}o}); some
of them were already defined by Paulson, and this formulation allows a
better interaction with \isatt{ZF-Constructible}. 
The axioms of Pairing, Union, Foundation,
Extensionality, and Infinity are relativizations of the respective
traditional 
first-order sentences to the class argument. For the Axiom of Choice
we selected a version best suited for the work with transitive
models: the relativization of a sentence stating that for every $x$
there is surjection from an ordinal onto $x$. Finally, Separation and
Replacement were treated separately to effectively obtain first-order
versions. It is to be noted that predicates in Isabelle/ZF are akin to
second order variables and thus do not correspond to first-order
formulas; in particular, there is no induction principle for functions
of type \isatt{i{\isasymRightarrow}o}. For that reason, Separation and
Replacement predicate \emph{on the satisfaction} of a formula $\phi$.
With respect to our previous \cite{2019arXiv190103313G}, we lifted the
arity restriction for the parameter $\phi$ on these schemes,
streamlining various proofs as an immediate benefit and avoiding the
need for tupling.

Once this class versions of the axioms are set up, it is not hard to
apply our synthesis method to obtain their internal, first-order
counterparts.
\begin{framed}
  Some code for axiom synthesis.
\end{framed}

%%% Local Variables: 
%%% mode: latex
%%% TeX-master: "forcing_in_isabelle_zf"
%%% ispell-local-dictionary: "american"
%%% End: 

%% 
%% \section{The definition of $\forceisa$}
\label{sec:definition-forces}

The core of the development is showing the definability of the
relation of forcing. As we explained in our previous
report~\cite[Sect.~8]{2019arXiv190103313G}, this comprises the
definition of a function $\forceisa$ that maps the set of internal
formulas into itself. It is the very reason of applicability of
forcing that the satisfaction of a first-order formula $\phi$ in all
of the generic extensions of a ctm $M$ can be ``controlled'' in a
definable way from $M$ (viz., by satisfaction of the formula
$\forceisa(\phi)$).

In fact, given a forcing notion $\PP$ (i.e. a preorder with a top element)
in a ctm $M$,
Kunen defines the \emph{forcing relation} model-theoretically 
by considering all extensions $M[G]$ with $G$ generic for $\PP$.
Then two fundamental results are proved, the Truth Lemma and the
Definability Lemma; but the proof of the first one is based on the
formula that witnesses Definability. To make sense of this in our 
formalization, we started with the internalized relation and then
proved that it is equivalent to the semantic version 
(``\isatt{definition{\isacharunderscore}of{\isacharunderscore}forces},'' in
the next section).
For that reason, the usual notation of the forcing relation 
$p \Vdash \phi\ \mathit{env}$ (for $\mathit{env}$ a list of
``names''), abbreviates in our code the
satisfaction by $M$ of $\forceisa(\phi)$:
\begin{isabelle}
\ \ {\isachardoublequoteopen}p\ {\isasymtturnstile}\ {\isasymphi}\ env\ \ \ {\isasymequiv}\ \ \ M{\isacharcomma}\ {\isacharparenleft}{\isacharbrackleft}p{\isacharcomma}P{\isacharcomma}leq{\isacharcomma}one{\isacharbrackright}\ {\isacharat}\ env{\isacharparenright}
    {\isasymTurnstile}\ forces{\isacharparenleft}{\isasymphi}{\isacharparenright}{\isachardoublequoteclose}
\end{isabelle}

The definition of $\forceisa$ proceeds by recursion
over the set $\formula$ and its base case, that is, for
atomic formulas, is (in)famously the most complicated one. Actually,
newcomers can be puzzled by the fact that forcing for atomic
formulas is also defined by (mutual) recursion: to know if $\tau_1\in\tau_2$ is
forced by $p$ (notation: $\forcesmem(p,\tau_1,\tau_2)$), one must check if $\tau_1=\sigma$ is forced for $\sigma$
moving in the transitive closure of $\tau_2$. To disentangle this, one
must realize that this last recursion must be described syntactically:
the definition of $\forceisa(\phi)$ for atomic $\phi$ is then an
internal definition of the alleged recursion on names. 

Our aim was to follow the definition proposed by Kunen
in~\cite[p.~257]{kunen2011set}, where the following mutual recursion
is given:
\begin{multline}\label{eq:def-forcing-equality}
  \forceseq (p,t_1,t_2) \defi 
  \forall s\in\dom(t_1)\cup\dom(t_2).\ \forall q\pleq p .\\
  \forcesmem(q,s,t_1)\lsii 
  \forcesmem(q,s,t_2),
\end{multline}
\begin{multline}\label{eq:def-forcing-membership}
  \forcesmem(p,t_1,t_2) \defi  \forall v\pleq p. \ \exists q\pleq v. \\
  \exists s.\ \exists r\in \PP .\ \lb s,r\rb \in
      t_2 \land q \pleq r \land \forceseq(q,t_1,s)
\end{multline}
Note that the definition of $\forcesmem$ is equivalent to require 
 the set 
\[
\{q\pleq p : \exists \lb s,r\rb\in t_2 . \ q\pleq r \land \forceseq(q,t_1,s)\}
\]
to be dense below $p$.

It was not straightforward to use the recursion machinery of
Isabelle/ZF to define $\forceseq$ and $\forcesmem$. For this, we
defined a relation $\frecR$ on 4-tuples of elements of $M$, proved
that it is well-founded and, more important, we also proved an
induction principle for this relation:
%
\begin{isabelle}
\isacommand{lemma}\isamarkupfalse%
\ forces{\isacharunderscore}induction{\isacharcolon}\isanewline
\ \ \isakeyword{assumes}\isanewline
\ \ \ \ {\isachardoublequoteopen}{\isasymAnd}{\isasymtau}\ {\isasymtheta}{\isachardot}\ {\isasymlbrakk}{\isasymAnd}{\isasymsigma}{\isachardot}\ {\isasymsigma}{\isasymin}domain{\isacharparenleft}{\isasymtheta}{\isacharparenright}\ {\isasymLongrightarrow}\ Q{\isacharparenleft}{\isasymtau}{\isacharcomma}{\isasymsigma}{\isacharparenright}{\isasymrbrakk}\ {\isasymLongrightarrow}\ R{\isacharparenleft}{\isasymtau}{\isacharcomma}{\isasymtheta}{\isacharparenright}{\isachardoublequoteclose}\footnotemark\isanewline
\ \ \ \ {\isachardoublequoteopen}{\isasymAnd}{\isasymtau}\ {\isasymtheta}{\isachardot}\ {\isasymlbrakk}{\isasymAnd}{\isasymsigma}{\isachardot}\ {\isasymsigma}{\isasymin}domain{\isacharparenleft}{\isasymtau}{\isacharparenright}\ {\isasymunion}\ domain{\isacharparenleft}{\isasymtheta}{\isacharparenright}\ {\isasymLongrightarrow}\ R{\isacharparenleft}{\isasymsigma}{\isacharcomma}{\isasymtau}{\isacharparenright}\ {\isasymand}\ R{\isacharparenleft}{\isasymsigma}{\isacharcomma}{\isasymtheta}{\isacharparenright}{\isasymrbrakk}\isanewline
\ \ \ \ \ \  {\isasymLongrightarrow}\ Q{\isacharparenleft}{\isasymtau}{\isacharcomma}{\isasymtheta}{\isacharparenright}{\isachardoublequoteclose}\isanewline
\ \ \isakeyword{shows}\isanewline
\ \ \ \ {\isachardoublequoteopen}Q{\isacharparenleft}{\isasymtau}{\isacharcomma}{\isasymtheta}{\isacharparenright}\ {\isasymand}\ R{\isacharparenleft}{\isasymtau}{\isacharcomma}{\isasymtheta}{\isacharparenright}{\isachardoublequoteclose}
\end{isabelle}
\footnotetext{The logical primitives of \emph{Pure} are
\isatt{\isasymLongrightarrow}, \isatt{\&\&\&}, and \isatt{\isasymAnd}
(implication, conjunction, and universal
quantification, resp.), which operate on the meta-Booleans
\isatt{prop}.}
%
and 
obtained both functions as cases of a another one, 
$\forcesat$, using a single recursion on $\frecR$. Then we obtained 
(\ref{eq:def-forcing-equality}) and (\ref{eq:def-forcing-membership})
as our corollaries \isatt{def{\isacharunderscore}forces{\isacharunderscore}eq} and
\isatt{def{\isacharunderscore}forces{\isacharunderscore}mem}.

Other approaches, like the one in Neeman~\cite{neeman-course} (and
Kunen's older book \cite{kunen1980}), prefer
to have a single, more complicated, definition by simple recursion for
$\forceseq$ and then define $\forcesmem$ explicitly. On hindsight,
this might have been a little simpler to do, but we preferred to be as
faithful to the text as possible concerning this point.

Once $\forcesat$ and its relativized version
$\isatt{is{\isacharunderscore}forces{\isacharunderscore}at}$ were
defined, we proceeded to show absoluteness and provided internal
definitions for the recursion on names using results in
\isatt{ZF-Constructible}. This finished the atomic case of the
formula-transformer $\forceisa$. The characterization of $\forceisa$
for negated and universal quantified formulas is given by the
following lemmas, respectively:
%
\begin{isabelle}
\isacommand{lemma}\isamarkupfalse%
\ sats{\isacharunderscore}forces{\isacharunderscore}Neg{\isacharcolon}\isanewline
\ \ \isakeyword{assumes}\isanewline
\ \ \ \ {\isachardoublequoteopen}p{\isasymin}P{\isachardoublequoteclose}\ {\isachardoublequoteopen}env\ {\isasymin}\ list{\isacharparenleft}M{\isacharparenright}{\isachardoublequoteclose}\ {\isachardoublequoteopen}{\isasymphi}{\isasymin}formula{\isachardoublequoteclose}\isanewline
\ \ \isakeyword{shows}\isanewline
\ \ \ \ {\isachardoublequoteopen}M{\isacharcomma}\ {\isacharbrackleft}p{\isacharcomma}P{\isacharcomma}leq{\isacharcomma}one{\isacharbrackright}\ {\isacharat}\ env\ {\isasymTurnstile}\ forces{\isacharparenleft}Neg{\isacharparenleft}{\isasymphi}{\isacharparenright}{\isacharparenright}\ \ \ {\isasymlongleftrightarrow}\ \isanewline
\ \ \ \ \ {\isasymnot}{\isacharparenleft}{\isasymexists}q{\isasymin}M{\isachardot}\ q{\isasymin}P\ {\isasymand}\ is{\isacharunderscore}leq{\isacharparenleft}{\isacharhash}{\isacharhash}M{\isacharcomma}leq{\isacharcomma}q{\isacharcomma}p{\isacharparenright}\ {\isasymand}\ \isanewline
\ \ \ \ \ \ \ \ \ \ M{\isacharcomma}\ {\isacharbrackleft}q{\isacharcomma}P{\isacharcomma}leq{\isacharcomma}one{\isacharbrackright}{\isacharat}env\ {\isasymTurnstile}\ forces{\isacharparenleft}{\isasymphi}{\isacharparenright}{\isacharparenright}{\isachardoublequoteclose}\isanewline

\isacommand{lemma}\isamarkupfalse%
\ sats{\isacharunderscore}forces{\isacharunderscore}Forall{\isacharcolon}\isanewline
\ \ \isakeyword{assumes}\isanewline
\ \ \ \ {\isachardoublequoteopen}p{\isasymin}P{\isachardoublequoteclose}\ {\isachardoublequoteopen}env\ {\isasymin}\ list{\isacharparenleft}M{\isacharparenright}{\isachardoublequoteclose}\ {\isachardoublequoteopen}{\isasymphi}{\isasymin}formula{\isachardoublequoteclose}\isanewline
\ \ \isakeyword{shows}\isanewline
\ \ \ \ {\isachardoublequoteopen}M{\isacharcomma}{\isacharbrackleft}p{\isacharcomma}P{\isacharcomma}leq{\isacharcomma}one{\isacharbrackright}\ {\isacharat}\ env\ {\isasymTurnstile}\ forces{\isacharparenleft}Forall{\isacharparenleft}{\isasymphi}{\isacharparenright}{\isacharparenright}\ {\isasymlongleftrightarrow}\ \isanewline
\ \ \ \ \ {\isacharparenleft}{\isasymforall}x{\isasymin}M{\isachardot}\ \ \ M{\isacharcomma}\ {\isacharbrackleft}p{\isacharcomma}P{\isacharcomma}leq{\isacharcomma}one{\isacharcomma}x{\isacharbrackright}\ {\isacharat}\ env\ {\isasymTurnstile}\ forces{\isacharparenleft}{\isasymphi}{\isacharparenright}{\isacharparenright}{\isachardoublequoteclose}
\end{isabelle}

Let us note in passing another improvement over our previous report:
we made a couple of new technical results concerning recursive
definitions. Paulson proved absoluteness of functions defined by
well-founded recursion over a transitive relation. Some of our
definitions by recursion (\emph{check} and \emph{forces}) do not fit
in that scheme.  One can replace the relation $R$ for its transitive
closure $R^+$ in the recursive definition because one can prove, in
general, that
$F\!\upharpoonright\!(R^{-1}(x))(y) =
F\!\upharpoonright\!({R^+}^{-1}(x))(y)$ whenever $(x,y) \in R$.


%%% Local Variables: 
%%% mode: latex
%%% TeX-master: "forcing_in_isabelle_zf"
%%% ispell-local-dictionary: "american"
%%% End: 

%% 
%% \section{The forcing theorems}
\label{sec:forcing-theorems}

After the definition of $\forceisa$ is complete, the proof of the
Fundamental Theorems of Forcing is comparatively straightforward, and
we were able to follow Kunen very closely. The more involved points of
this part of the development were those where we needed to proved that
various (dense) subsets of $\PP$ were in $M$; for this, we had to
recourse to several absoluteness ad-hoc lemmas.

The first results concern characterizations of the forcing
relation. Two of them are \isatt{Forces{\isacharunderscore}Member}:
\begin{center}
  \isatt{{\isacharparenleft}p\ {\isasymtturnstile}\ Member{\isacharparenleft}n{\isacharcomma}m{\isacharparenright}\ env{\isacharparenright}\ {\isasymlongleftrightarrow}\ forces{\isacharunderscore}mem{\isacharparenleft}p{\isacharcomma}t{\isadigit{1}}{\isacharcomma}t{\isadigit{2}}{\isacharparenright}},
\end{center}
where \isatt{t{\isadigit{1}}} and \isatt{t{\isadigit{1}}} are the
\isatt{n}th resp.\ \isatt{m}th elements of \isatt{env}, and  \isatt{Forces{\isacharunderscore}Forall}:
\begin{center}
  \isatt{{\isacharparenleft}p\ {\isasymtturnstile}\ Forall{\isacharparenleft}{\isasymphi}{\isacharparenright}\ env{\isacharparenright}\ {\isasymlongleftrightarrow}\ {\isacharparenleft}{\isasymforall}x{\isasymin}M{\isachardot}\ {\isacharparenleft}p\ {\isasymtturnstile}\ {\isasymphi}\ {\isacharparenleft}{\isacharbrackleft}x{\isacharbrackright}\ {\isacharat}\ env{\isacharparenright}{\isacharparenright}{\isacharparenright}}.
\end{center}
These two, along with  \isatt{Forces{\isacharunderscore}Equal} and
\isatt{Forces{\isacharunderscore}Nand}, appear in Kunen as the
inductive definition of the forcing relation \cite[Def.~IV.2.42]{kunen2011set}.

As with the previous section, the proofs of the forcing theorems have two different
flavours: The ones for the atomic formulas proceed by using the
principle of 
\isatt{forces{\isacharunderscore}induction}, and then an induction on
$\formula$ wraps the former with the remaining cases (\isatt{Nand} and \isatt{Forall}). 

As an example of the first class, we can take a look at our
formalization of \cite[Lem.~IV.2.40(a)]{kunen2011set}:

\begin{isabelle}
  \isacommand{lemma}\isamarkupfalse%
  \ IV{\isadigit{2}}{\isadigit{4}}{\isadigit{0}}a{\isacharcolon}\isanewline
  \ \ \isakeyword{assumes}\isanewline
  \ \ \ \ {\isachardoublequoteopen}M{\isacharunderscore}generic{\isacharparenleft}G{\isacharparenright}{\isachardoublequoteclose}\isanewline
  \ \ \isakeyword{shows}\ \isanewline
  \ \ \ \ {\isachardoublequoteopen}{\isacharparenleft}{\isasymtau}{\isasymin}M{\isasymlongrightarrow}{\isasymtheta}{\isasymin}M{\isasymlongrightarrow}{\isacharparenleft}{\isasymforall}p{\isasymin}G{\isachardot}forces{\isacharunderscore}eq{\isacharparenleft}p{\isacharcomma}{\isasymtau}{\isacharcomma}{\isasymtheta}{\isacharparenright}{\isasymlongrightarrow}val{\isacharparenleft}G{\isacharcomma}{\isasymtau}{\isacharparenright}{\isacharequal}val{\isacharparenleft}G{\isacharcomma}{\isasymtheta}{\isacharparenright}{\isacharparenright}{\isacharparenright}{\isasymand}\isanewline
  \ \ \ \ \ {\isacharparenleft}{\isasymtau}{\isasymin}M{\isasymlongrightarrow}{\isasymtheta}{\isasymin}M{\isasymlongrightarrow}{\isacharparenleft}{\isasymforall}p{\isasymin}G{\isachardot}forces{\isacharunderscore}mem{\isacharparenleft}p{\isacharcomma}{\isasymtau}{\isacharcomma}{\isasymtheta}{\isacharparenright}{\isasymlongrightarrow}val{\isacharparenleft}G{\isacharcomma}{\isasymtau}{\isacharparenright}{\isasymin}val{\isacharparenleft}G{\isacharcomma}{\isasymtheta}{\isacharparenright}{\isacharparenright}{\isacharparenright}{\isachardoublequoteclose}
\end{isabelle}
%
Its proof starts by an introduction of \isatt{forces{\isacharunderscore}induction};
the  inductive cases for each atomic type were handled before as
separate lemmas (\isatt{IV240a{\isacharunderscore}mem} and \isatt{IV240a{\isacharunderscore}eq}). We
illustrate with the statement of the latter.
%
\begin{isabelle}
\isacommand{lemma}\isamarkupfalse%
\ IV{\isadigit{2}}{\isadigit{4}}{\isadigit{0}}a{\isacharunderscore}eq{\isacharcolon}\isanewline
\ \ \isakeyword{assumes}\isanewline
\ \ \ \ {\isachardoublequoteopen}M{\isacharunderscore}generic{\isacharparenleft}G{\isacharparenright}{\isachardoublequoteclose}\ {\isachardoublequoteopen}p{\isasymin}G{\isachardoublequoteclose}\ {\isachardoublequoteopen}forces{\isacharunderscore}eq{\isacharparenleft}p{\isacharcomma}{\isasymtau}{\isacharcomma}{\isasymtheta}{\isacharparenright}{\isachardoublequoteclose}\isanewline
\ \ \ \ \isakeyword{and}\isanewline
\ \ \ \ IH{\isacharcolon}{\isachardoublequoteopen}{\isasymAnd}q\ {\isasymsigma}{\isachardot}\ q{\isasymin}P\ {\isasymLongrightarrow}\ q{\isasymin}G\ {\isasymLongrightarrow}\ {\isasymsigma}{\isasymin}domain{\isacharparenleft}{\isasymtau}{\isacharparenright}\ {\isasymunion}\ domain{\isacharparenleft}{\isasymtheta}{\isacharparenright}\ {\isasymLongrightarrow}\ \isanewline
\ \ \ \ \ \ \ \ {\isacharparenleft}forces{\isacharunderscore}mem{\isacharparenleft}q{\isacharcomma}{\isasymsigma}{\isacharcomma}{\isasymtau}{\isacharparenright}\ {\isasymlongrightarrow}\ val{\isacharparenleft}G{\isacharcomma}{\isasymsigma}{\isacharparenright}\ {\isasymin}\ val{\isacharparenleft}G{\isacharcomma}{\isasymtau}{\isacharparenright}{\isacharparenright}\ {\isasymand}\isanewline
\ \ \ \ \ \ \ \ {\isacharparenleft}forces{\isacharunderscore}mem{\isacharparenleft}q{\isacharcomma}{\isasymsigma}{\isacharcomma}{\isasymtheta}{\isacharparenright}\ {\isasymlongrightarrow}\ val{\isacharparenleft}G{\isacharcomma}{\isasymsigma}{\isacharparenright}\ {\isasymin}\ val{\isacharparenleft}G{\isacharcomma}{\isasymtheta}{\isacharparenright}{\isacharparenright}{\isachardoublequoteclose}\isanewline
\ \ \isakeyword{shows}\isanewline
\ \ \ \ {\isachardoublequoteopen}val{\isacharparenleft}G{\isacharcomma}{\isasymtau}{\isacharparenright}\ {\isacharequal}\ val{\isacharparenleft}G{\isacharcomma}{\isasymtheta}{\isacharparenright}{\isachardoublequoteclose}
\end{isabelle}

As an example of the second kind of induction (on formulas), we choose the
following relatively simple result:

\begin{isabelle}
\isacommand{lemma}\isamarkupfalse%
\ strengthening{\isacharunderscore}lemma{\isacharcolon}\isanewline
\ \ \isakeyword{assumes}\ \isanewline
\ \ \ \ {\isachardoublequoteopen}p{\isasymin}P{\isachardoublequoteclose}\ {\isachardoublequoteopen}{\isasymphi}{\isasymin}formula{\isachardoublequoteclose}\ {\isachardoublequoteopen}r{\isasymin}P{\isachardoublequoteclose}\ {\isachardoublequoteopen}r{\isasympreceq}p{\isachardoublequoteclose}\isanewline
\ \ \isakeyword{shows}\isanewline
\ \ \ \ {\isachardoublequoteopen}{\isasymAnd}env{\isachardot}\ env{\isasymin}list{\isacharparenleft}M{\isacharparenright}\ {\isasymLongrightarrow}\ arity{\isacharparenleft}{\isasymphi}{\isacharparenright}{\isasymle}length{\isacharparenleft}env{\isacharparenright}\ {\isasymLongrightarrow}\ p\ {\isasymtturnstile}\ {\isasymphi}\ env\isanewline 
\ \ \ \ \ {\isasymLongrightarrow}\ r\ {\isasymtturnstile}\ {\isasymphi}\ env{\isachardoublequoteclose}\isanewline
%
%
\isacommand{using}\isamarkupfalse%
\ assms{\isacharparenleft}{\isadigit{2}}{\isacharparenright}
\end{isabelle}
%
The proof is divided in the 4 cases of definition of an element of $\formula$,
%
\begin{isabelle}
\isacommand{proof}\isamarkupfalse%
\ {\isacharparenleft}induct{\isacharparenright}\isanewline
\ \ \isacommand{case}\isamarkupfalse%
\ {\isacharparenleft}Member\ n\ m{\isacharparenright}\isanewline
\ \ \isacommand{then}\isamarkupfalse%
\isanewline
\ \ \dots
\isanewline
\ \ \isacommand{show}\isamarkupfalse%
\ {\isacharquery}case\ \isanewline
\ \ \ \ \isacommand{using}\isamarkupfalse%
\ Forces{\isacharunderscore}Member{\isacharbrackleft}of\ {\isacharunderscore}\ {\isachardoublequoteopen}nth{\isacharparenleft}n{\isacharcomma}env{\isacharparenright}{\isachardoublequoteclose}\ {\isachardoublequoteopen}nth{\isacharparenleft}m{\isacharcomma}env{\isacharparenright}{\isachardoublequoteclose}\ env\ n\ m{\isacharbrackright}\isanewline
\ \ \ \ \ \ strengthening{\isacharunderscore}mem{\isacharbrackleft}of\ p\ r\ {\isachardoublequoteopen}nth{\isacharparenleft}n{\isacharcomma}env{\isacharparenright}{\isachardoublequoteclose}\ {\isachardoublequoteopen}nth{\isacharparenleft}m{\isacharcomma}env{\isacharparenright}{\isachardoublequoteclose}{\isacharbrackright}\ \isacommand{by}\isamarkupfalse%
\ simp
\end{isabelle}
%
where the final step depends on previously proved
\isatt{strengthening{\isacharunderscore}mem} and the characterization of
$\forceisa$ for membership 


The case of equality is entirely analogous, and the \isatt{Nand} and
\isatt{Forall} cases are handled very simply.
%
\begin{isabelle}
\isacommand{next}\isamarkupfalse%
\isanewline
\ \ \isacommand{case}\isamarkupfalse%
\ {\isacharparenleft}Equal\ n\ m{\isacharparenright}\isanewline
\ \ \dots\isanewline
\isacommand{next}\isamarkupfalse%
\isanewline
\ \ \isacommand{case}\isamarkupfalse%
\ {\isacharparenleft}Nand\ {\isasymphi}\ {\isasympsi}{\isacharparenright}\isanewline
\ \ \isacommand{with}\isamarkupfalse%
\ assms\isanewline
\ \ \isacommand{show}\isamarkupfalse%
\ {\isacharquery}case\ \isanewline
\ \ \ \ \isacommand{using}\isamarkupfalse%
\ Forces{\isacharunderscore}Nand\ Transset{\isacharunderscore}intf{\isacharbrackleft}OF\ trans{\isacharunderscore}M\ {\isacharunderscore}\ P{\isacharunderscore}in{\isacharunderscore}M{\isacharbrackright}\ pair{\isacharunderscore}in{\isacharunderscore}M{\isacharunderscore}iff\isanewline
\ \ \ \ \ \ Transset{\isacharunderscore}intf{\isacharbrackleft}OF\ trans{\isacharunderscore}M\ {\isacharunderscore}\ leq{\isacharunderscore}in{\isacharunderscore}M{\isacharbrackright}\ leq{\isacharunderscore}transD\ \isacommand{by}\isamarkupfalse%
\ auto\isanewline
\isacommand{next}\isamarkupfalse%
\isanewline
\ \ \isacommand{case}\isamarkupfalse%
\ {\isacharparenleft}Forall\ {\isasymphi}{\isacharparenright}\isanewline
\ \ \isacommand{with}\isamarkupfalse%
\ assms\isanewline
\ \ \isacommand{have}\isamarkupfalse%
\ {\isachardoublequoteopen}p\ {\isasymtturnstile}\ {\isasymphi}\ {\isacharparenleft}{\isacharbrackleft}x{\isacharbrackright}\ {\isacharat}\ env{\isacharparenright}{\isachardoublequoteclose}\ \isakeyword{if}\ {\isachardoublequoteopen}x{\isasymin}M{\isachardoublequoteclose}\ \isakeyword{for}\ x\isanewline
\ \ \ \ \isacommand{using}\isamarkupfalse%
\ that\ Forces{\isacharunderscore}Forall\ \isacommand{by}\isamarkupfalse%
\ simp\isanewline
\ \ \isacommand{with}\isamarkupfalse%
\ \underline{Forall}\ \isanewline
\ \ \isacommand{have}\isamarkupfalse%
\ {\isachardoublequoteopen}r\ {\isasymtturnstile}\ {\isasymphi}\ {\isacharparenleft}{\isacharbrackleft}x{\isacharbrackright}\ {\isacharat}\ env{\isacharparenright}{\isachardoublequoteclose}\ \isakeyword{if}\ {\isachardoublequoteopen}x{\isasymin}M{\isachardoublequoteclose}\ \isakeyword{for}\ x\isanewline
\ \ \ \ \isacommand{using}\isamarkupfalse%
\ that\ pred{\isacharunderscore}le{\isadigit{2}}\ \isacommand{by}\isamarkupfalse%
\ {\isacharparenleft}simp{\isacharparenright}\isanewline
\ \ \isacommand{with}\isamarkupfalse%
\ assms\ \underline{Forall}\isanewline
\ \ \isacommand{show}\isamarkupfalse%
\ {\isacharquery}case\ \isanewline
\ \ \ \ \isacommand{using}\isamarkupfalse%
\ Forces{\isacharunderscore}Forall\ \isacommand{by}\isamarkupfalse%
\ simp\isanewline
\isacommand{qed}\isamarkupfalse
\end{isabelle}
%
It can be noted that the inductive hypothesis
gets used in the last case (underlined here as
\isatt{\underline{Forall}}), but not in the case for \isatt{Nand}.



%%% Local Variables: 
%%% mode: latex
%%% TeX-master: "forcing_in_isabelle_zf"
%%% ispell-local-dictionary: "american"
%%% End: 

%% 
%% \section{Example of proper extension}
\label{sec:example-proper-extension}

Even when the axioms of $\ZFC$ are proved in the generic extension,
one cannot claim that the magic of forcing has taken place unless one
is able to provide some \emph{proper} extension with the \emph{same
ordinals}. After all, one is assuming from starters a model $M$ of $\ZFC$,
and in some trivial cases $M[G]$ might end up to be exactly $M$; this
is where \emph{proper} enters the stage. But, for instance, in the
presence of large cardinals, a model $M'\supsetneq M$ might be an
end-extension of $M$ ---this is were we ask the two models to have the
same ordinals, the same \emph{height}. 

Three theory files contain the relevant
results. \verb|Ordinals_In_MG.thy| shows, using the closure of $M$
under ranks, that $M$ and $M[G]$ share the same ordinals (actually,
ranks of elements of $M[G]$ are bounded by the ranks of their names in
$M$):
\begin{isabelle}
\isacommand{lemma}\isamarkupfalse%
\ rank{\isacharunderscore}val{\isacharcolon}\ {\isachardoublequoteopen}rank{\isacharparenleft}val{\isacharparenleft}G{\isacharcomma}x{\isacharparenright}{\isacharparenright}\ {\isasymle}\ rank{\isacharparenleft}x{\isacharparenright}{\isachardoublequoteclose}\isanewline
\isacommand{lemma}\isamarkupfalse%
\ Ord{\isacharunderscore}MG{\isacharunderscore}iff{\isacharcolon}\isanewline
\ \ \isakeyword{assumes}\ {\isachardoublequoteopen}Ord{\isacharparenleft}{\isasymalpha}{\isacharparenright}{\isachardoublequoteclose}\ \isanewline
\ \ \isakeyword{shows}\ {\isachardoublequoteopen}{\isasymalpha}\ {\isasymin}\ M\ {\isasymlongleftrightarrow}\ {\isasymalpha}\ {\isasymin}\ M{\isacharbrackleft}G{\isacharbrackright}{\isachardoublequoteclose}
\end{isabelle}

To prove these results, we found it useful to formalize induction over
the relation \isatt{ed}$(x,y) \defi x\in\dom(y)$, which is key
to arguments involving names.
\begin{isabelle}
\isacommand{lemma}\isamarkupfalse%
\ ed{\isacharunderscore}induction{\isacharcolon}\isanewline
\ \ \isakeyword{assumes}\ {\isachardoublequoteopen}{\isasymAnd}x{\isachardot}\ {\isasymlbrakk}{\isasymAnd}y{\isachardot}\ \ ed{\isacharparenleft}y{\isacharcomma}x{\isacharparenright}\ {\isasymLongrightarrow}\ Q{\isacharparenleft}y{\isacharparenright}\ {\isasymrbrakk}\ {\isasymLongrightarrow}\ Q{\isacharparenleft}x{\isacharparenright}{\isachardoublequoteclose}\isanewline
\ \ \isakeyword{shows}\ {\isachardoublequoteopen}Q{\isacharparenleft}a{\isacharparenright}{\isachardoublequoteclose}
\end{isabelle}

\verb|Succession_Poset.thy| contains our first example of a poset
that interprets the locale
\isatt{forcing{\isacharunderscore}notion}, essentially the notion for
adding one Cohen real. It is the set $2^{<\om}$ of all finite binary
sequences partially  ordered by reverse inclusion.
The sufficient condition for a proper extension is that
the forcing poset is \emph{separative}: every element has two
incompatible (\isatt{{\isasymbottom}s}) extensions. Here,
\isatt{seq{\isacharunderscore}upd{\isacharparenleft}f{\isacharcomma}x{\isacharparenright}}
adds \isatt{x} to the end of the sequence \isatt{f}.

\begin{isabelle}
\isacommand{lemma}\isamarkupfalse%
\ seqspace{\isacharunderscore}separative{\isacharcolon}\isanewline
\ \ \isakeyword{assumes}\ {\isachardoublequoteopen}f{\isasymin}{\isadigit{2}}{\isacharcircum}{\isacharless}{\isasymomega}{\isachardoublequoteclose}\isanewline
\ \ \isakeyword{shows}\ {\isachardoublequoteopen}seq{\isacharunderscore}upd{\isacharparenleft}f{\isacharcomma}{\isadigit{0}}{\isacharparenright}\ {\isasymbottom}s\ seq{\isacharunderscore}upd{\isacharparenleft}f{\isacharcomma}{\isadigit{1}}{\isacharparenright}{\isachardoublequoteclose}
\end{isabelle}
 
We prove in the theory file \verb|Proper_Extension.thy| that, in
general, every separative forcing notion gives rise to a proper
extension.

%%% Local Variables: 
%%% mode: latex
%%% TeX-master: "forcing_in_isabelle_zf"
%%% ispell-local-dictionary: "american"
%%% End: 

%% 
%% \section{The axioms of replacement and choice}
\label{sec:axioms-replacement-choice}

In \cite{2019arXiv190103313G} we proved that any generic extension
preserves the satisfaction of almost all the axioms, including the separation scheme
(we adapted those proofs to the current statement of the axiom
schemes). Our proofs that Replacement and choice hold in every generic
extension depend on further relativized concepts and closure properties.

\subsection{Replacement}

The proof of the Replacement Axiom scheme in $M[G]$ in Kunen uses the
Reflection Principle relativized to $M$. We took an alternative
pathway, following Neeman \cite{neeman-course}. In his course notes,
he uses the relativization of the cumulative hierarchy of sets. 

The
family of all sets of rank less than $\alpha$ is called
\isatt{Vset}$(\alpha)$ in Isabelle/ZF. We showed, in the theory file
\verb|Relative_Univ.thy|
 the following
relativization and closure results concerning this function, for a
class $M$ satisfying the locale \isatt{M{\isacharunderscore}eclose}
plus the Powerset Axiom and four instances of replacement.
%
\begin{isabelle}
\isacommand{lemma}\isamarkupfalse%
\ Vset{\isacharunderscore}abs{\isacharcolon}\ {\isachardoublequoteopen}{\isasymlbrakk}\ M{\isacharparenleft}i{\isacharparenright}{\isacharsemicolon}\ M{\isacharparenleft}V{\isacharparenright}{\isacharsemicolon}\ Ord{\isacharparenleft}i{\isacharparenright}\ {\isasymrbrakk}\ {\isasymLongrightarrow}\ \isanewline
\ \ \ \ \ \ \ \  \ \  \ \ \ \ \ \ \ \ \ \ is{\isacharunderscore}Vset{\isacharparenleft}M{\isacharcomma}i{\isacharcomma}V{\isacharparenright}\ {\isasymlongleftrightarrow}\ V\ {\isacharequal}\ {\isacharbraceleft}x{\isasymin}Vset{\isacharparenleft}i{\isacharparenright}{\isachardot}\ M{\isacharparenleft}x{\isacharparenright}{\isacharbraceright}{\isachardoublequoteclose}
\end{isabelle}
\begin{isabelle}
\isacommand{lemma}\isamarkupfalse%
\ Vset{\isacharunderscore}closed{\isacharcolon}\ {\isachardoublequoteopen}{\isasymlbrakk}\ M{\isacharparenleft}i{\isacharparenright}{\isacharsemicolon}\ Ord{\isacharparenleft}i{\isacharparenright}\ {\isasymrbrakk}\ {\isasymLongrightarrow}\ M{\isacharparenleft}{\isacharbraceleft}x{\isasymin}Vset{\isacharparenleft}i{\isacharparenright}{\isachardot}\ M{\isacharparenleft}x{\isacharparenright}{\isacharbraceright}{\isacharparenright}{\isachardoublequoteclose}
\end{isabelle}
We also have the basic result
\begin{isabelle}
\isacommand{lemma}\isamarkupfalse%
\ M{\isacharunderscore}into{\isacharunderscore}Vset{\isacharcolon}\isanewline
\ \ \isakeyword{assumes}\ {\isachardoublequoteopen}M{\isacharparenleft}a{\isacharparenright}{\isachardoublequoteclose}\isanewline
\ \ \isakeyword{shows}\ {\isachardoublequoteopen}{\isasymexists}i{\isacharbrackleft}M{\isacharbrackright}{\isachardot}\ {\isasymexists}V{\isacharbrackleft}M{\isacharbrackright}{\isachardot}\ ordinal{\isacharparenleft}M{\isacharcomma}i{\isacharparenright}\ {\isasymand}\ is{\isacharunderscore}Vfrom{\isacharparenleft}M{\isacharcomma}{\isadigit{0}}{\isacharcomma}i{\isacharcomma}V{\isacharparenright}\ {\isasymand}\ a{\isasymin}V{\isachardoublequoteclose}
\end{isabelle}
stating that $M$ is included in 
$\union\{\isatt{Vset}(\alpha) : \alpha\in M\}$ (it's actually equal).

For the proof of the Replacement Axiom, we assume that $\phi$ is
functional in its first two variables when interpreted in $M[G]$ and
the first ranges over the domain \isatt{c}${}\in M[G]$. Then we show
that the collection of
all values of the second variable, when the first ranges over
\isatt{c}, belongs to $M[G]$:
%
\begin{isabelle}
\isacommand{lemma}\isamarkupfalse%
\ Replace{\isacharunderscore}sats{\isacharunderscore}in{\isacharunderscore}MG{\isacharcolon}\isanewline
\ \ \isakeyword{assumes}\isanewline
\ \ \ \ {\isachardoublequoteopen}c{\isasymin}M{\isacharbrackleft}G{\isacharbrackright}{\isachardoublequoteclose}\ {\isachardoublequoteopen}env\ {\isasymin}\ list{\isacharparenleft}M{\isacharbrackleft}G{\isacharbrackright}{\isacharparenright}{\isachardoublequoteclose}\isanewline
\ \ \ \ {\isachardoublequoteopen}{\isasymphi}\ {\isasymin}\ formula{\isachardoublequoteclose}\ {\isachardoublequoteopen}arity{\isacharparenleft}{\isasymphi}{\isacharparenright}\ {\isasymle}\ {\isadigit{2}}\ {\isacharhash}{\isacharplus}\ length{\isacharparenleft}env{\isacharparenright}{\isachardoublequoteclose}\isanewline
\ \ \ \ {\isachardoublequoteopen}univalent{\isacharparenleft}{\isacharhash}{\isacharhash}M{\isacharbrackleft}G{\isacharbrackright}{\isacharcomma}\ c{\isacharcomma}\ {\isasymlambda}x\ v{\isachardot}\ {\isacharparenleft}M{\isacharbrackleft}G{\isacharbrackright}{\isacharcomma}\ {\isacharbrackleft}x{\isacharcomma}v{\isacharbrackright}{\isacharat}env\ {\isasymTurnstile}\ {\isasymphi}{\isacharparenright}{\isacharparenright}{\isachardoublequoteclose}\isanewline
\ \ \isakeyword{shows}\isanewline
\ \ \ \ {\isachardoublequoteopen}{\isacharbraceleft}v{\isachardot}\ x{\isasymin}c{\isacharcomma}\ v{\isasymin}M{\isacharbrackleft}G{\isacharbrackright}\ {\isasymand}\ {\isacharparenleft}M{\isacharbrackleft}G{\isacharbrackright}{\isacharcomma}\ {\isacharbrackleft}x{\isacharcomma}v{\isacharbrackright}{\isacharat}env\ {\isasymTurnstile}\ {\isasymphi}{\isacharparenright}{\isacharbraceright}\ {\isasymin}\ M{\isacharbrackleft}G{\isacharbrackright}{\isachardoublequoteclose}
\end{isabelle}
%
From this, the satisfaction of the Replacement Axiom in $M[G]$ follows
very easily.

The proof of the previous lemma, following Neeman, proceeds as usual
by turning an argument concerning elements of $M[G]$ to one involving
names lying in $M$, and connecting both worlds by using the forcing
theorems. In the case at hand, by functionality of $\phi$ we know that
for every $x\in c\cap M[G]$ there exists exactly one $v\in M[G]$ such
that
$M[G], [x,v]\mathbin{@} \mathit{env} \models \phi$. Now,
given a name $\pi'\in M$ for $c$, every name of an element of $c$
belongs to $\pi\defi \dom(\pi')\times \PP$, which is easily seen to be
in $M$. We will use $\pi$ to be the domain in an application of the
Replacement Axiom in $M$. But now, obviously, we have lost
functionality since there are many names $\dot v\in M$ for a fixed $v$
in $M[G]$. To solve this issue, for each $\rho p \defi\lb\rho,p\rb\in
\pi$ we calculate the
minimum rank of some $\tau\in M$ such that 
$p\forces \phi(\rho,\tau,\dots)$ if there is one, or $0$ otherwise. By
Replacement in $M$, we can show that the supremum \isatt{?sup} of these ordinals
belongs to $M$ and we can construct a \isatt{?bigname} $\defi$ 
\isatt{{\isacharbraceleft}x{\isasymin}Vset{\isacharparenleft}{\isacharquery}sup{\isacharparenright}{\isachardot}\ x\ {\isasymin}\
}$M$\isatt{{\isacharbraceright}\ {\isasymtimes}\ {\isacharbraceleft}one{\isacharbraceright}}
whose interpretation by (any generic) $G$ will include all possible elements
as $v$ above.

The previous calculation required some absoluteness and closure
results regarding the minimum ordinal binder, \isatt{Least}$(Q)$, also
denoted $\mu x. Q(x)$, that can be found in the theory file
\verb|Least.thy|.

\subsection{Choice}
A first important observation is that the proof of $\AC$ in $M[G]$
only requires the assumption that $M$ satisfies (a finite fragment of)
$\ZFC$. There is no need to invoke Choice in the metatheory.

Although our previous version of the development used $\AC$, that was
only needed to show the Rasiowa-Sikorski Lemma (RSL) for
arbitrary posets. We have modularized the proof of the latter
and now the version for countable posets that we use to show the
existence of generic filters
does not require Choice (as it is well known). We also bundled the
full RSL along with our implementation of the principle of dependent
choices in an independent branch of the dependency graph, which is the
only place where the theory \texttt{ZF.AC} is invoked.

Our statement of the Axiom of Choice is the one preferred for
arguments involving transitive classes satisfying $\ZF$:
%
\begin{center}
\isatt{{\isasymforall}x{\isacharbrackleft}M{\isacharbrackright}{\isachardot}\ {\isasymexists}a{\isacharbrackleft}M{\isacharbrackright}{\isachardot}\ {\isasymexists}f{\isacharbrackleft}M{\isacharbrackright}{\isachardot}\ ordinal{\isacharparenleft}M{\isacharcomma}a{\isacharparenright}\ {\isasymand}\ surjection{\isacharparenleft}M{\isacharcomma}a{\isacharcomma}x{\isacharcomma}f{\isacharparenright}}
\end{center}
%
The Simplifier is able to show automatically that this
statement is equivalent to the next one, in which the real notions of
ordinal and surjection appear:
%
\begin{center}
\isatt{{\isasymforall}x{\isacharbrackleft}M{\isacharbrackright}{\isachardot}\ {\isasymexists}a{\isacharbrackleft}M{\isacharbrackright}{\isachardot}\ {\isasymexists}f{\isacharbrackleft}M{\isacharbrackright}{\isachardot}\ Ord{\isacharparenleft}a{\isacharparenright}\ {\isasymand}\ f\ {\isasymin}\ surj{\isacharparenleft}a{\isacharcomma}x{\isacharparenright}}
\end{center}

As with the forcing axioms, the proof of $\AC$ in $M[G]$ follows the pattern of Kunen
\cite[IV.2.27]{kunen2011set} and is rather
straightforward; the only complicated technical point being to show
that the relevant name belongs to $M$. We assume that \isatt{a}${}\neq\emptyset$
belongs to $M[G]$ and has a name $\tau\in M$. By $\AC$ in $M$, there
is a surjection \isatt{s} from an ordinal $\alpha\in M$ ($\subseteq M[G]$) onto
$\dom(\tau)$. Now
%
\begin{center}
\isatt{{\isacharbraceleft}opair{\isacharunderscore}name{\isacharparenleft}check{\isacharparenleft}{\isasymbeta}{\isacharparenright}{\isacharcomma}s{\isacharbackquote}{\isasymbeta}{\isacharparenright}{\isachardot}\ {\isasymbeta}{\isasymin}{\isasymalpha}{\isacharbraceright}\ {\isasymtimes}\ {\isacharbraceleft}one{\isacharbraceright}}
\end{center}
%
is a name for a function \isatt{f} with domain $\alpha$ such that \isatt{a}
is included in its range, and where
\isatt{opair{\isacharunderscore}name}$(\sig,\rho)$ is a name for the
ordered pair $\lb\val(G,\sig),\val(G,\rho)\rb$. From this, $\AC$ in
$M[G]$ follows easily.

\subsection{The main theorem}
With all these elements in place, we are able to transcript the main
theorem of our formalization:
\begin{isabelle}
\isacommand{theorem}\isamarkupfalse%
\ extensions{\isacharunderscore}of{\isacharunderscore}ctms{\isacharcolon}\isanewline
\ \ \isakeyword{assumes}\ \isanewline
\ \ \ \ {\isachardoublequoteopen}M\ {\isasymapprox}\ nat{\isachardoublequoteclose}\ {\isachardoublequoteopen}Transset{\isacharparenleft}M{\isacharparenright}{\isachardoublequoteclose}\ {\isachardoublequoteopen}M\ {\isasymTurnstile}\ ZF{\isachardoublequoteclose}\isanewline
\ \ \isakeyword{shows}\ \isanewline
\ \ \ \ {\isachardoublequoteopen}{\isasymexists}N{\isachardot}\ \isanewline
\ \ \ \ \ \ M\ {\isasymsubseteq}\ N\ {\isasymand}\ N\ {\isasymapprox}\ nat\ {\isasymand}\ Transset{\isacharparenleft}N{\isacharparenright}\ {\isasymand}\ N\ {\isasymTurnstile}\ ZF\ {\isasymand}\ M{\isasymnoteq}N\ {\isasymand}\isanewline
\ \ \ \ \ \ {\isacharparenleft}{\isasymforall}{\isasymalpha}{\isachardot}\ Ord{\isacharparenleft}{\isasymalpha}{\isacharparenright}\ {\isasymlongrightarrow}\ {\isacharparenleft}{\isasymalpha}\ {\isasymin}\ M\ {\isasymlongleftrightarrow}\ {\isasymalpha}\ {\isasymin}\ N{\isacharparenright}{\isacharparenright}\ {\isasymand}\isanewline
\ \ \ \ \ \ {\isacharparenleft}M{\isacharcomma}\ {\isacharbrackleft}{\isacharbrackright}{\isasymTurnstile}\ AC\ {\isasymlongrightarrow}\ N\ {\isasymTurnstile}\ ZFC{\isacharparenright}{\isachardoublequoteclose}
\end{isabelle}
Here, \isatt{\isasymapprox} stands for equipotency, \isatt{nat} is the
set of natural numbers, and the predicate 
\isatt{Transset} indicates transitivity; and as usual, \isatt{AC}
denotes the Axiom of Choice, and \isatt{ZF} and \isatt{ZFC} the
corresponding subsets of \isatt{formula}.

%%% Local Variables: 
%%% mode: latex
%%% TeX-master: "forcing_in_isabelle_zf"
%%% ispell-local-dictionary: "american"
%%% End: 

%% 
\section{Random comments}

\begin{enumerate}
\item Satisfaction of having one's material fully formalized
  (v.g. cofinality).
\item The concept of absoluteness (\verb|_abs|) splitted into two components
  during the formalization: \verb|is_foo_iff| and \verb|foo_rel_abs| (this last one
  being the concept of absoluteness from  “mathematical practice”) and
  the first one involves a purely relational presentation, introduced
  by Paulson.
\item
  In order to extract information concerning the proof, the need for a
  proof assistant with a more computational flavour (Agda, Coq)
  arises! As a second thought, obviously.
\item  \textbf{Pros and cons of an untyped environment}: Closer to the
  set-theoretic point of view/prone to the most stupid errors.
\item \textbf{“Pros” and cons of lack of automation}: we had to be extremely
  detailed, which provides explicit information / the formalization
  work was pain in the neck and much lower.
\item In order to extract information concerning the proof, the need for a
  proof assistant with a more computational flavour (Agda, Coq)
  arises! As a second thought, obviously.
\end{enumerate}

%%% Local Variables: 
%%% mode: latex
%%% TeX-master: "independence_ch_isabelle"
%%% ispell-local-dictionary: "american"
%%% End: 


\section{Related work}
\label{sec:related-work}

\textbf{Reviewer's comments}
{\it
  \begin{itemize}
  \item There, it would be appropriate to contrast what was done in
    Paulson's work on constructibility with the current work on forcing.
  \item More to the point, the recent work by Han and van Doorn on
    forcing in Lean deserves more discussion.  They have gone further
    than the current authors, having proved the independence of the
    continuum hypothesis.  They prefer Boolean-valued models as being
    more direct in use than the authors' countable transitive models.
    \begin{itemize}
    \item Readers will want to know whether the type-theoretic approach
      is better/worse/just different than using Isabelle/ZF, and
    \item are there any benefits to the ctm approach?
    \item Is the type-theory encoding of ZF really accurate?
    \item How about comparing proofs of equivalent statements in the two
      approaches for length and readability?
    \end{itemize}
  \end{itemize}
}

To the best of our knowledge, all of the previous works in
formalization of the method 
of forcing have been done in different variants of type theory, and
none of them uses the ctm approach. The
most important is the recent one by 
Han and van Doorn
\cite{han_et_al:LIPIcs:2019:11074,DBLP:conf/cpp/HanD20}, which includes
a formalization of the independence of $\CH$ by means
the Boolean-valued approach to forcing, using the Lean
proof assistant \cite{DBLP:conf/cade/MouraKADR15}.


\begin{itemize}
\item The consistency strength of Lean requires infinitely
  many inaccessibles. More precisely, let Lean$_n$ be the theory of
  CiC foundations of Lean restricted to $n$ type universes.  Carneiro
  \cite{carneiro-ms-thesis}, proved the consistency of Lean$_n$ from
  $\ZFC$ plus the existence of $n$ inaccessible
  cardinals. It is also reported in \cite{carneiro-ms-thesis} that
  Werner's results in \cite{10.5555/645869.668660} can be adapted to
  show that Lean$_{n+2}$ proves the consistency of the latter theory. 

  On the other hand, Isabelle's \emph{Pure} is based on
  ``intuitionistic higher order logic.'' In Paulson
  \cite{Paulson1989} it is proved that \emph{Pure} is sound for
  intuitionistic first order logic, thus it does not add any strength
  to it. On top of this, the axiomatization of Isabelle/ZF results in
  a system equiconsistent with $\ZFC$. Our running assumption, that of
  the existence of a countable transitive model, is considerably
  weaker (directly and consistency-wise) than the existence of a
  single inaccesible cardinal. In that sense, directly obtain
  unprovability results in first order logic, the meta theoretic
  machinery used to obtain them is far heavier than the one we use to
  operate model-theoretically.
  %
\item We may discuss in finer detail the shape of the axioms of
  Isabelle/ZF. It is perhaps more correct to say it is an
  notational variant of NBG set theory, because the schemes of
  Replacement and Separation feature higher order (free) variables
  playing the role of formula variables. It can't be proved that the
  axioms thus written correspond to first order sentences. This is the
  reason that our relativized versions only apply to set models, where
  we can restrict the formula variables to predicates that actually
  come from first order variables. In that sense, the axioms of the
  locale \isatt{M{\isacharunderscore}ZF} correspond faithfully to the
  $ZF$ axioms.
\item \fbox{\bf take care of repetitions} In our opinion, one of the
  main benefits of using transitive models is that many fundamental
  notions are absolute and thus the many statements can be interpreted
  transparently. It also provides a very concrete way to understand
  generic objects: as sets that (in the non trivial case) are provably
  not in the original model; this dispells any mystical feel around
  this concept (contrary to the case when the ground model is the
  universe of all sets). In addition, two-valued semantics is
  closer to our intuition ($\leftarrow$ revise).
\end{itemize}
%%% Local Variables: 
%%% mode: latex
%%% TeX-master: "forcing_in_isabelle_zf"
%%% ispell-local-dictionary: "american"
%%% End: 


\section{Some lessons}\label{sec:lessons}

We want to finish this report by gathering some of the conclusions we
reached after the experience of formalizing the basics of forcing in a
proof assistant.

\subsection{Aims of a formalization and planning}

We believe that in every project of formalization of mathematics,
there is a tension between the haste to verify the target results and
the need to obtain a readable, albeit extremely detailed, corpus of
statements and proofs. This tension is mirrored in two differents
purposes of formalization: Developing new mathematics from scratch and
producing verified results en route to this, versus verifying and
documenting material that has already been produced on paper.

Our present project clearly belongs to the second category, so we
prioritized trying to obtain formal proofs that mimicked standard
prose (the highlight being the sample proof in
Section~\ref{sec:sample-formal-proof}). We feel that the Isar language
provided with Isabelle has the right balance between elegance and
efficacy. Another crucial aspect to achieve this goal is the level of
detail of the blueprint for the formalization. We must however confess
that we learned many of the subtleties of Isabelle in the making, and
many engineering decisions were also taken before it was clear the
precise way things would develop in the future.

A similar experience, but on an opposite side of the formalization
spectrum happened to the Liquid Tensor Project as described by Scholze
in \cite{LTE2021}. People involved in the formalization simply pushed
their way to reach the summit, formalizing lemma after lemma. They
actually wrote the blueprint for that formalization \emph{afterwards}
it was complete! From time to time, we were also frenziedly trying to
get the results formalized, going beyond what we had planned.

As a result from this, some design choices that seemed reasonable at
first were proved to be inconvenient. For instance, we should had
better used predicates (of type $\tyi\fun\tyi\fun\tyo$) for the
forcing posets' order relations; this is the way they
are presented in the \session{Delta\_System\_Lemma} session. A similar
problem is that we require the forcing poset to be an element of $M$,
so the present infrastructure does not allow class forcing out of the
box. (The latter change seems to be rather straightforward, but the
former does not.)

Nearly the final stage of the project, we decided to go for the minimal
set of definitions and versions of lemmas that were needed to obtain
our target results. For example, we only proved the Delta System Lemma
for $\aleph_1$-sized families (thus limiting us to the plain ccc) and
showed preservation of sequences by considering countably closed
forcings (in fact, we formalized the bare minimum requirement of being
$(<\omega+1)$-closed). In doing this we went against the tried and
true advice that one should formalize the most general version of the
results available.

\subsection{Bureaucracy and scale factors}

\begin{enumerate}
\item Bureaucracy vs ML programming.
\item The “math” was already formalized on 22 November 2020.
  We finished the last goal on 22 August 2021.
  (Update: 20 November 2021 \& 28 November 2021, for CH)
\item Missing: automation of closure of models under operations.
\item Missing: basic arithmetic for dealing with arities.
\end{enumerate}

\begin{enumerate}
\item It is extremely misleading when automatic tools (\isatt{simp}, \isatt{auto}, etc)
  stop working just because of the sheer size of the goal. Oftentimes,
  in math, we disregard scale issues but they must always be taken
  into account in CS.
\item Example: $\forceisa(0\in 1)$ is expandable,
  $\forceisa(\neg\neg  0\in 1)$ is not.
\item Example: Synthesis of $\forceisa$; could have been fully synthesized,
  but that was dirty “strategy”.
\item The know-how of computer scientists on this kind of engineering is
  very important
\end{enumerate}

\subsection{You might have formalized it, and still be wrong}
\begin{enumerate}
\item Example: restriction of relations.
\item Pollack, “Pollack consistency” by
  Wiedijk. Cf. \theory{Definitions\_Main} (thanks to discussions with
  Vidnyánszky). Opacity of automated proofs. 
\item Plot twist: You can be right without knowing. Intuition may drive proofs
  even if we are not working on what we believe we are.
\end{enumerate}

\subsection{Beware of the “Code fever”}\label{sec:beware-code-fever}
\begin{enumerate}
\item “We know that doing math is fun---formalization is like DRUGS”
\item Feeling of accomplishment after seeing your writings
  validated beyond reasonable doubt (v.g. cofinality).

\item One easily forgets about the “Power of the Board.”
\end{enumerate}

\subsection{The Devil's on the shortcuts}
\begin{enumerate}
\item
  Our proofs of the “definition of forces” (and many
  consequences) and of the lemma for “forcing a value” of function
  depend on the countability of the ground model. 
\item
  Density arguments (look for “TODO”, “general versions”).
\end{enumerate}

%%% Local Variables: 
%%% mode: latex
%%% TeX-master: "independence_ch_isabelle"
%%% ispell-local-dictionary: "american"
%%% End: 

%%
%% \section{Conclusions and future work}
There are several technical milestones that have to be reached in the
course of a formalization of the theory of forcing. The first one, and most
obvious, is the bulk of set- and meta-theoretical concepts needed to work
with. This pushed us, in a sense,  into building on top of Isabelle/ZF,
since we know of no other development in set theory of such
depth (and breadth). In this paper we worked on setting the stage for the work with
generic extensions; in particular, this involves some purely mathematical
results, as the Rasiowa-Sikorski lemma. 

Other milestones in this formalization project
involve 
\begin{enumerate}
\item the definition
  of the forcing relation, 
\item proving the Fundamental Theorem of forcing
  (that relates truth in $M$ to that in $M[G]$), and 
\item using it to show
  that $M[G]\models \ZFC$. 
\end{enumerate}
The theory is very modular and this is
witnessed by the fact 
that the last goal does not depend on the proof of the Fundamental
Theorem nor on the definition of the forcing relation. Our next task
will be to obtain the last goal in that enumeration. 

To this end, we will develop an interface between Paulson's
relativization results and countable models of $\ZFC$. This will show
that every ctm $M$ is closed under well-founded recursion and, in
particular, that contains names for each of its
elements. Consequently, the proof of  $M\sbq M[G]$ will be
complete. A landmark will be to prove the Axiom Scheme
of Separation (the first that needs to use the machinery of forcing
nontrivially). As a part of the new formalization, we will provide
Isar versions of the longer applicative proofs presented in this work.

\ack{We'd like to thank the anonymous referees for reading the paper
  carefully and for their detailed and constructive criticism.}
%%% Local Variables:
%%% mode: latex
%%% ispell-local-dictionary: "american"
%%% TeX-master: "first_steps_into_forcing"
%%% End:

%
% ---- Bibliography ----
%
% BibTeX users should specify bibliography style 'splncs04'.
% References will then be sorted and formatted in the correct style.
%
%\bibliographystyle{splncs04}
\bibliographystyle{mi-estilo-else}
\bibliography{independence_ch_isabelle,clam2021}

\end{document}

%%% Local Variables: 
%%% mode: latex
%%% ispell-local-dictionary: "american"
%%% End: 
