\section{Related work}
\label{sec:related-work}

\textbf{Reviewer's comments}
{\it
  \begin{itemize}
  \item There, it would be appropriate to contrast what was done in
    Paulson's work on constructibility with the current work on forcing.
  \item More to the point, the recent work by Han and van Doorn on
    forcing in Lean deserves more discussion.  They have gone further
    than the current authors, having proved the independence of the
    continuum hypothesis.  They prefer Boolean-valued models as being
    more direct in use than the authors' countable transitive models.
    \begin{itemize}
    \item Readers will want to know whether the type-theoretic approach
      is better/worse/just different than using Isabelle/ZF, and
    \item are there any benefits to the ctm approach?
    \item Is the type-theory encoding of ZF really accurate?
    \item How about comparing proofs of equivalent statements in the two
      approaches for length and readability?
    \end{itemize}
  \end{itemize}
}

To the best of our knowledge, all of the previous works in
formalization of the method 
of forcing have been done in different variants of type theory, and
none of them uses the ctm approach. The
most important is the recent one by 
Han and van Doorn
\cite{han_et_al:LIPIcs:2019:11074,DBLP:conf/cpp/HanD20}, which includes
a formalization of the independence of $\CH$ by means
the Boolean-valued approach to forcing, using the Lean
proof assistant \cite{DBLP:conf/cade/MouraKADR15}.


\begin{itemize}
\item The consistency strength of Lean requires infinitely
  many inaccessibles. More precisely, let Lean$_n$ be the theory of
  CiC foundations of Lean restricted to $n$ type universes.  Carneiro
  \cite{carneiro-ms-thesis}, proved the consistency of Lean$_n$ from
  $\ZFC$ plus the existence of $n$ inaccessible
  cardinals. It is also reported in \cite{carneiro-ms-thesis} that
  Werner's results in \cite{10.5555/645869.668660} can be adapted to
  show that Lean$_{n+2}$ proves the consistency of the latter theory. 

  On the other hand, Isabelle's \emph{Pure} is based on
  ``intuitionistic higher order logic.'' In Paulson
  \cite{Paulson1989} it is proved that \emph{Pure} is sound for
  intuitionistic first order logic, thus it does not add any strength
  to it. On top of this, the axiomatization of Isabelle/ZF results in
  a system equiconsistent with $\ZFC$. Our running assumption, that of
  the existence of a countable transitive model, is considerably
  weaker (directly and consistency-wise) than the existence of a
  single inaccesible cardinal. In that sense, directly obtain
  unprovability results in first order logic, the meta theoretic
  machinery used to obtain them is far heavier than the one we use to
  operate model-theoretically.
  %
\item We may discuss in finer detail the shape of the axioms of
  Isabelle/ZF. It is perhaps more correct to say it is an
  notational variant of NBG set theory, because the schemes of
  Replacement and Separation feature higher order (free) variables
  playing the role of formula variables. It can't be proved that the
  axioms thus written correspond to first order sentences. This is the
  reason that our relativized versions only apply to set models, where
  we can restrict the formula variables to predicates that actually
  come from first order variables. In that sense, the axioms of the
  locale \isatt{M{\isacharunderscore}ZF} correspond faithfully to the
  $ZF$ axioms.
\item \fbox{\bf take care of repetitions} In our opinion, one of the
  main benefits of using transitive models is that many fundamental
  notions are absolute and thus the many statements can be interpreted
  transparently. It also provides a very concrete way to understand
  generic objects: as sets that (in the non trivial case) are provably
  not in the original model; this dispells any mystical feel around
  this concept (contrary to the case when the ground model is the
  universe of all sets). In addition, two-valued semantics is
  closer to our intuition ($\leftarrow$ revise).
\end{itemize}
%%% Local Variables: 
%%% mode: latex
%%% TeX-master: "forcing_in_isabelle_zf"
%%% ispell-local-dictionary: "american"
%%% End: 
