\section{22 replacement instances to rule them all}
\label{sec:repl-instances-appendix}

In \isa{instances1{\uscore}fms}:

\begin{itemize}
\item Transitive closure:
  \begin{itemize}
  \item
    \sout{\isa{eclose{\uscore}repl1{\uscore}intf{\uscore}fm}.} \isa{eclose{\uscore}closed{\uscore}fm}.

    To prove closure under iteration of $X\mapsto\union X$.
  \item
    \sout{\isa{eclose{\uscore}repl2{\uscore}intf{\uscore}fm}} \isa{eclose{\uscore}abs{\uscore}fm}.

    Auxiliary instance used to show absoluteness.
  \end{itemize}
%% The instances so far
%% are needed to interpret locale
%% \locale{M{\uscore}eclose}.
\item \isa{wfrec{\uscore}rank{\uscore}fm}.
  
  For $\in$-rank.
  %
\item \sout{\isa{trans{\uscore}repl{\uscore}HVFrom{\uscore}fm}.} \isa{transrec{\uscore}VFrom{\uscore}fm}.

  For the cumulative hierarchy (rank initial segments).
\end{itemize}

%% The last two and next pair have the same syntactic structure, because
%% they are definitions by well-founded recursion.
In \isa{instances2{\uscore}fms} (for cardinal arithmetic):
\begin{itemize}
\item Ordertypes:
  \begin{itemize}
  \item
    \sout{\isa{wfrec{\uscore}replacement{\uscore}order{\uscore}pred{\uscore}fm}.}
    \isa{wfrec{\uscore}ordertype{\uscore}fm}.

    Well-founded recursion for the construction of ordertypes.
    %
  \item
    \sout{\isa{replacement{\uscore}is{\uscore}order{\uscore}eq{\uscore}map{\uscore}fm}.}
    \isa{omap{\uscore}replacement{\uscore}fm}.
    
    Auxiliary instance for the definition of ordertypes.
  \end{itemize}
\item Aleph:
  \begin{itemize}
    %
  \item
    \sout{\isa{replacement{\uscore}is{\uscore}order{\uscore}body{\uscore}fm}.}
    \isa{ordtype{\uscore}replacement{\uscore}fm}.

    Replacement through the ordertype function, for Hartogs' Theorem.
    %
  \item
    \sout{\isa{replacement{\uscore}HAleph{\uscore}wfrec{\uscore}repl{\uscore}body{\uscore}fm}.}
    \isa{wfrec{\uscore}Aleph{\uscore}fm}.

    The well-founded recursion to define Aleph.
  \end{itemize}
\end{itemize}

In \isa{instances{\uscore}ground{\uscore}fms}:

\begin{itemize}
\item \isa{wfrec{\uscore}Hcheck{\uscore}fm}.
  
  Well-founded recursion to define check-names.
  %
\item \isa{wfrec{\uscore}Hfrc{\uscore}at{\uscore}fm}.

  Well-founded recursion for the definition of forcing for atomic formulas.
  %
\item
  \sout{\isa{Lambda{\uscore}in{\uscore}M{\uscore}fm(check{\uscore}fm(2,0,1),1)}.}
  \isa{lam{\uscore}replacement{\uscore}check{\uscore}fm}.

  Replacement through $x\mapsto \lb x,\check{x}\rb$ (for the
  definition of $\punto{G}$).
  %
\end{itemize}

In 
\isa{instances{\uscore}ground{\uscore}notCH{\uscore}fms}:
\begin{itemize}
\item
  \sout{\isa{replacement{\uscore}is{\uscore}trans{\uscore}apply{\uscore}image{\uscore}fm}.}
  \isa{recursive{\uscore}construction{\uscore}fm}.

  Recursive construction of sets using a choice function.
  %
\item
  \sout{\isa{replacement{\uscore}transrec{\uscore}apply{\uscore}image{\uscore}body{\uscore}fm}.}
  \isa{recursive{\uscore}construction{\uscore}abs{\uscore}fm}.
  
  Absoluteness of the previous construction.
\end{itemize}
%
\begin{itemize}
\item
  \sout{\isa{replacement{\uscore}dcwit{\uscore}repl{\uscore}body{\uscore}fm}.}
  \isa{dc{\uscore}abs{\uscore}fm}.
  
  Absoluteness of the recursive construction in the proof of the
  Dependent Choices from $\AC$.
\end{itemize}

%%% Local Variables:
%%% mode: latex
%%% TeX-master: "independence_ch_isabelle"
%%% ispell-local-dictionary: "american"
%%% End: 
