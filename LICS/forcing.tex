%%%%%%%%%%%%%%%%%%%%%%%%%%%%%%%%%%%%%%%%%%%%%%%%%%%%%%%%%%%%%%%%%%%%%%          
\section{Forcing}
\label{sec:forcing}

Given a ctm $M$, and an $M$-generic filter $G\sbq\PP$, the Forcing
Theorems relate satisfaction of a formula 
$\phi$ in the generic extension $M[G]$ to the satisfaction of another
formula $\forceisa(\phi)$ in $M$. The map $\forceisa$ is defined by
recursion on 
the structure of $\phi$. It is to be noted that the base case (viz.,
for atomic $\phi$) contains all the complexity; the case for
connectives and quantifiers is then straightforward.
%% and indeed it can be
%% said that writing down this definition is half
%% a way the path to set up the forcing machinery.
In order to state the properties of this map
in sufficient generality to prove that  $M[G]$ satisfies $\ZF$, we work with
internalized formulas, because it is not possible to carry inductive
arguments over \tyo

We will now make more precise the properties of the map
$\forceisa$ and how it relates satisfaction in $M$ to that in
$M[G]$. Actually, if the formula $\phi$ has $n$ free variables,
$\forceisa(\phi)$ will have $n+4$ free variables, where the first four account
for the forcing notion and a particular element of it. 

In referring to formulas with free variables and their interpretation,
we use the following conventions. If  $M$ is a set, $a_0,\dots,a_n$
are elements of $M$, and $\phi$ has its free variables
among $x_0,\dots,x_n$, we write
\[
M,[a_0,\dots,a_n] \models \phi(x_0,\dots,x_n)
\]
to denote that $\phi$ is satisfied by $M$ when $x_i$ is interpreted
to be $a_i$ ($i=0,\dots,n$). We call the list $[a_0,\dots,a_n]$ the
\emph{environment}. In the case of a formula of the form
$\forceisa(\phi)$, we'll make an abuse of notation and indicate the
variables inside the argument of $\forceisa$. As an example, take the
formula $\phi\defi x_1\in x_0$. Then
\[
M,[a,b] \models x_1\in x_0
\]
will hold whenever $b\in a$; and instead of writing $\forceisa(\phi)$
we will write $\forceisa(x_5\in x_4)$, as in
\[
M,[\PP,\leq,\1,p,\tau,\rho] \models \forceisa(x_5\in x_4).
\]

If
$\phi=\phi(x_0,\dots,x_n)$, the notation used by Kunen
\cite{kunen2011set,kunen1980} for $\forceisa(\phi)$ is 
\[
p\forces_{\PP,\leq,\1}^* \phi(x_0,\dots,x_n).
\]
Here, the extra parameters are $\PP,\leq,\1,$ and $p\in\PP$, and the
first three are usually omitted. %(since the forcing notion is fixed
% throughout the discussion).  
Afterwards, the \emph{forcing relation}
$\forces$ can be obtained by 
interpreting $\forces^*$ in a ctm $M$, for fixed
$\lb\PP,\leq,\1\rb\in M$: $p\forces \phi(\tau_0,\dots,\tau_n)$ holds
if and only if
\begin{equation}\label{eq:3}
M,[\PP,\leq,\1,p,\tau_0,\dots,\tau_n]\models \forceisa(x_4,\dots,x_{n+4}).
\end{equation}

%-%-%-%-%-%-%-%-%-%-%-%-%-%-%-%-%-%-%-%-%-%-%-%-%-%-%-%-%-%-%-%-%-%-%-
\subsection{The fundamental theorems}
\label{sec:fundamental-theorems}
Modern treatments of the theory of forcing start
by defining the 
forcing relation semantically and later it is proved  that the
characterization given by (\ref{eq:3}) indeed holds, and hence the
forcing relation is \emph{definable}.

Then the definition of the forcing relation is stated as a
\begin{lemma}[Definition of Forcing]\label{lem:definition-of-forcing}
  Let $M$ be a ctm of $\ZF$, $\lb\PP,\leq,\1\rb$ a forcing notion
  in $M$, $p\in\PP$, and $\phi(x_0,\dots,x_n)$ a formula in the
  language of set 
  theory with all free variables displayed. Then the
  following are equivalent, for all $\tau_0,\dots,\tau_n\in M$:
  \begin{enumerate}
  \item $M,[\PP,\leq,\1,p,\tau_0,\dots,\tau_n] 
  \models\forceisa(\phi(x_4,\dots,x_{n+4}))$.
  \item For all $M$-generic filters $G$ such that $p\in G$,
    $M[G],[\val(G,\tau_0),\dots,\val(G,\tau_n)] \models\phi(x_0,\dots,x_n)$.
  \end{enumerate}
\end{lemma}

The \emph{Truth Lemma} states that the forcing
relation indeed relates 
satisfaction in $M[G]$ to that in $M$. 
\begin{lemma}[Truth Lemma]\label{lem:truth-lemma}
  Assume the same hypothesis of
  Lemma~\ref{lem:definition-of-forcing}. Then the
  following are equivalent, for all $\tau_0,\dots,\tau_n\in M$, and
  $M$-generic $G$: 
  \begin{enumerate}
  \item $M[G],[\val(G,\tau_0),\dots,\val(G,\tau_n)]
  \models\phi(x_0,\dots,x_n)$.
  \item  There exists $p\in G$ such that $M,[\PP,\leq,\1,p,\tau_0,\dots,\tau_n] 
  \models\forceisa(\phi(x_4,\dots,x_{n+4}))$.
  \end{enumerate}
\end{lemma}
The previous two results combined are the ones usually called the
\emph{fundamental theorems}. 

The following auxiliary results (adapted from
\cite[IV.2.43]{kunen2011set}) are also handy in forcing arguments.
\begin{lemma}[Strengthening]\label{lem:strengthen} 
  Assume the same hypothesis of Lemma~\ref{lem:definition-of-forcing}.
  $M, [\PP,\leq,\1,p,\dots]\models\forceisa(\phi)$ and $p_1\leq p$
  implies $M, [\PP,\leq,\1,p_1,\dots]\models\forceisa(\phi)$.
\end{lemma}
\begin{lemma}[Density]\label{lem:density}
  Assume the same hypothesis of
  Lemma~\ref{lem:definition-of-forcing}. $M,[\dots]\models\forceisa(\phi)$ 
  if and only if 
  $M,[\dots]\models$``$\{p_1\leq p : \forceisa(\phi,\PP,\leq,\1,p_1)\}$ is
  dense below $p$.''
\end{lemma}
%
All these results are proved by recursion in
$\formula$.

The locale \isatt{forcing\_thms} includes all these results as assumptions on the
mapping $\forceisa$, plus a typing condition  and its
effect on arities:
\begin{isabelle}
\isacommand{locale}\isamarkupfalse%
\ forcing{\isacharunderscore}thms\ {\isacharequal}\ forcing{\isacharunderscore}data\ {\isacharplus}\isanewline
\ \isakeyword{fixes}\ forces\ {\isacharcolon}{\isacharcolon}\ {\isachardoublequoteopen}i\ {\isasymRightarrow}\ i{\isachardoublequoteclose}\isanewline
\ \isakeyword{assumes}\
definition{\isacharunderscore}of{\isacharunderscore}forces{\isacharcolon}\isanewline 
\ \ {\isachardoublequoteopen}p{\isasymin}P\ {\isasymLongrightarrow}\ {\isasymphi}{\isasymin}formula\ {\isasymLongrightarrow}\ env{\isasymin}list{\isacharparenleft}M{\isacharparenright}\ {\isasymLongrightarrow}\isanewline
\ \ \ \  sats{\isacharparenleft}M{\isacharcomma}forces{\isacharparenleft}{\isasymphi}{\isacharparenright}{\isacharcomma}\ {\isacharbrackleft}P{\isacharcomma}leq{\isacharcomma}one{\isacharcomma}p{\isacharbrackright}\ {\isacharat}\ env{\isacharparenright}\ {\isasymlongleftrightarrow}\isanewline
\ \ \ \ \ \
{\isacharparenleft}{\isasymforall}G{\isachardot}{\isacharparenleft}M{\isacharunderscore}generic{\isacharparenleft}G{\isacharparenright}{\isasymand}\
p{\isasymin}G{\isacharparenright}\ {\isasymlongrightarrow}\isanewline 
\ \ \ \ \ \ sats{\isacharparenleft}M{\isacharbrackleft}G{\isacharbrackright}{\isacharcomma}{\isasymphi}{\isacharcomma}map{\isacharparenleft}val{\isacharparenleft}G{\isacharparenright}{\isacharcomma}env{\isacharparenright}{\isacharparenright}{\isacharparenright}{\isachardoublequoteclose}\isanewline
\ \ \isakeyword{and}\ \ definability{\isacharbrackleft}TC{\isacharbrackright}{\isacharcolon}\ {\isachardoublequoteopen}{\isasymphi}{\isasymin}formula\ {\isasymLongrightarrow}\isanewline
\ \ \ \ \ \ \ \ \  forces{\isacharparenleft}{\isasymphi}{\isacharparenright}\ {\isasymin}\ formula{\isachardoublequoteclose}\isanewline
\ \ \isakeyword{and}\ \ \ arity{\isacharunderscore}forces{\isacharcolon}\ \ {\isachardoublequoteopen}{\isasymphi}{\isasymin}formula\ {\isasymLongrightarrow}\isanewline
\ \ \ \ \ \ \
arity{\isacharparenleft}forces{\isacharparenleft}{\isasymphi}{\isacharparenright}{\isacharparenright}\
{\isacharequal}\ arity{\isacharparenleft}{\isasymphi}{\isacharparenright}\ {\isacharhash}{\isacharplus}\ {\isadigit{4}}{\isachardoublequoteclose}\isanewline
\ \ \isakeyword{and}\ \dots
\end{isabelle}

The presentation of the Fundamental Theorems of Forcing in a locale
can be regarded as a \emph{formal abstract} as in the project
envisioned by Hales \cite{hales-fabstracts}, where  statements of
mathematical theorems proven in the literature are posed in a language
that is both human- and computer-readable. The point is to take
particular care so that, v.g., there are no missing hypotheses, so it
is possible to take this statement as firm ground on which start a
formalization of a proof. 

%%% Local Variables: 
%%% mode: latex
%%% TeX-master: "Separation_In_MG"
%%% ispell-local-dictionary: "american"
%%% End: 
