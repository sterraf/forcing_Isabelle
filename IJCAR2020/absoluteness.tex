\section{Relativization,  absoluteness, and the axioms}
\label{sec:relat-absol}

The concepts of relativization and absoluteness (due to Gödel, in his
proof of the relative consistency of $\AC$ \cite{godel-L}) 
are both
prerequisites and powerful tools in working with transitive
models. A \emph{class} is simply a predicate $C(x)$ with at least one
free variable $x$.
The \emph{relativization} $\phi^C(\bx)$ of a set-theoretic
definition
$\phi$ (of a relation such
as ``$x$ is a subset of $y$'' or of a function like $y=\P(x)$) to
a class $C$ is obtained by restricting all of its quantifiers to $C$.

\[
x \sbq^C y \equiv \forall z.\ C(z) \longrightarrow (z\in x
\longrightarrow z\in y)
\]

The new formula $\phi^C(\bx)$ corresponds to what is obtained by defining
the concept ``inside'' $C$. In fact, for a class corresponding to a
set $c$ (i.e.\ $C(x) \defi x \in c$), the relativization $\phi^C$ of a 
sentence $\phi$ is equivalent to the satisfaction of $\phi$ in the
first-order model $\lb c, \in\rb$.

It turns out that many concepts mean the
same after relativization to a nonempty transitive class $C$; formally
\[
\forall\bx.\ C(\bx) \longrightarrow (\phi^C(\bx) \longleftrightarrow
\phi(\bx))
\]
When this is the case, we say that the relation defined by $\phi$ is
\emph{absolute for transitive models}.\footnote{Absoluteness of
  functions also requires the relativized definition to behave
  functionally over $C$.} As examples, the relation of inclusion
$\subseteq$ ---and actually, any relation defined by a formula
(equivalent to one) using only bounded quantifiers $(\forall x\in y)$
and $(\exists x\in y)$--- is absolute for transitive models. On the
contrary, this is not the case with the powerset operation.

A benefit of working with transitive models
is that many concepts (pairs, unions, and fundamentally ordinals) are
uniform across the universe \isatt{i}, a ctm (of an adequate fragment
of $\ZF$) $M$ and any of its extensions $M[G]$.
For that reason, then one can reason
``externally'' about absolute concepts, instead
of ``inside'' the model; thus, one has at hand any already proved
lemma about the real concept.

Paulson's formalization \cite{paulson_2003} of Gödel's \cite{godel-L}
already contains absoluteness results which were crucial to our
project; we realized however that they could be refactorized to be
more useful. The main objective is to maximize applicability of
the relativization machinery by adjusting the hypothesis of a greater
part of it early development. Paulson's architecture had only in mind
the consistency of $\ZFC$, but, for instance, in order to apply it in
the development of forcing, too much is assumed at the beginning; more
seriously, some assumptions cannot be regarded as ``first-order''
(v.g. the Replacement Scheme) because of the occurrence of predicate
variables.
The version we present of the constructibility library,
\isatt{ZF-Constructible-Trans}, weakens the
assumptions of many absoluteness theorems to that
of a nonempty transitive class; we also include some stronger results
such as the relativization of powersets.

Apart from the axiom schemes, the $\ZFC$ axioms are initially stated
as predicates on classes (that is, of type
\isatt{(i{\isasymRightarrow}o){\isasymRightarrow}o}); this formulation
allows a better interaction with \isatt{ZF-Constructible}.  The axioms
of Pairing, Union, Foundation, Extensionality, and Infinity are
relativizations of the respective traditional first-order sentences to
the class argument. For the Axiom of Choice we selected a version best
suited for the work with transitive models: the relativization of a
sentence stating that for every $x$ there is surjection from an
ordinal onto $x$. Finally, Separation and Replacement were treated
separately to effectively obtain first-order versions afterwards. It
is to be noted that predicates in Isabelle/ZF are akin to second order
variables and thus do not correspond to first-order formulas.
%% in particular, there is no induction principle for functions
%% of type \isatt{i{\isasymRightarrow}o}. 
For that reason, Separation and Replacement predicate \emph{on the
  satisfaction} of a formula $\phi$.  We improved their specification,
with respect to our previous \cite{2019arXiv190103313G}, by lifting
the arity restriction for the parameter $\phi$; consequently we
get rid of tupling and thus various proofs are now slicker.

A benefit of having class versions of the axioms is that we can
apply our synthesis method to obtain their internal, first-order
counterparts.% \footnote{It should be noted that our method for
  % synthetic definition requires that the schematic goal had a
  % particular format where the only unbound schematic variable occurs
  % as an (open) formula whose satisfaction is equivalent to some
  % relativized predicate.}
For the case of the Pairing Axiom, the statement for classes is the
following
\begin{isabelle}
upair{\isacharunderscore}ax{\isacharparenleft}C{\isacharparenright}{\isacharequal}{\isacharequal}{\isasymforall}x{\isacharbrackleft}C{\isacharbrackright}{\isachardot}{\isasymforall}y{\isacharbrackleft}C{\isacharbrackright}{\isachardot}{\isasymexists}z{\isacharbrackleft}C{\isacharbrackright}{\isachardot}\ upair{\isacharparenleft}C{\isacharcomma}x{\isacharcomma}y{\isacharcomma}z{\isacharparenright}
\end{isabelle}
where \isatt{upair} says that \isatt{z} is the unordered pair of
\isatt{x} and \isatt{y}, relative to \isatt{C}.

The following schematic lemma synthesizes its internal version,%
\footnote{The use of such schematic goals and the original definition
  of the collection of lemmas \isatt{sep{\isacharunderscore}rules} are due to Paulson.}
\begin{isabelle}
\isacommand{schematic{\isacharunderscore}goal}\isamarkupfalse%
\ ZF{\isacharunderscore}pairing{\isacharunderscore}auto{\isacharcolon}\isanewline
\ \ \ \ {\isachardoublequoteopen}upair{\isacharunderscore}ax{\isacharparenleft}{\isacharhash}{\isacharhash}A{\isacharparenright}\ {\isasymlongleftrightarrow}\ {\isacharparenleft}A{\isacharcomma}\ {\isacharbrackleft}{\isacharbrackright}\ {\isasymTurnstile}\ {\isacharquery}zfpair{\isacharparenright}{\isachardoublequoteclose}\isanewline
\isacommand{unfolding}\isamarkupfalse%
\ upair{\isacharunderscore}ax{\isacharunderscore}def\ \isanewline
\ \ \isacommand{by}\isamarkupfalse%
\ {\isacharparenleft}{\isacharparenleft}rule\ sep{\isacharunderscore}rules\ {\isacharbar}\ simp{\isacharparenright}{\isacharplus}{\isacharparenright}
\end{isabelle}
(the actual formula obtained is
\isatt{Forall(Forall(Exists(upair{\isacharunderscore}fm(2,1,0))))}) and
our \isatt{synthesize} method introduces a new term
\isatt{ZF{\isacharunderscore}pairing{\isacharunderscore}fm} for it.
\begin{isabelle}
\isacommand{synthesize}\isamarkupfalse%
\ {\isachardoublequoteopen}ZF{\isacharunderscore}pairing{\isacharunderscore}fm{\isachardoublequoteclose}\ \isakeyword{from{\isacharunderscore}schematic}\ {\isachardoublequoteopen}ZF{\isacharunderscore}pairing{\isacharunderscore}auto{\isachardoublequoteclose}%
\end{isabelle}


%%% Local Variables: 
%%% mode: latex
%%% TeX-master: "forcing_in_isabelle_zf"
%%% ispell-local-dictionary: "american"
%%% End: 
