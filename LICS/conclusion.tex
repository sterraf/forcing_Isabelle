%%%%%%%%%%%%%%%%%%%%%%%%%%%%%%%%%%%%%%%%%%%%%%%%%%%%%%%%%%%%%%%%%%%%%%          
\section{Conclusions and future work}
\label{sec:conclusions-future-work}

The proof of the Powerset Axiom does not present difficulties.

By G\"odel's Second Incompleteness Theorem we cannot  prove that
there exists a model of $\ZF$  (assuming that our base theory is not
stronger than $\ZF$ to start with), but for consistency proofs,
usually a model of only a finite subset of $\ZF$ suffices. Recall that
an equivalent formulation of the Foundation Axiom states that the 
universe of sets can be decomposed in a transfinite hierarchy of
sets. 
\begin{theorem}
  Let $V_{\al}\defi\union\{\P(V_\be) : \be<\al\}$ for each ordinal
  $\al$. Then each $V_\al$ is a set and 
  $\forall x. \exists\al .\ \Ord(\al) \y x\in V_\al$.  
\end{theorem}
%
Then the following meta-theoretic result by Montague applies:
%
\begin{theorem}[Reflection Principle]\label{th:reflection-principle}
  For every finite $\Phi\sbq\ZF$, $\ZF$ proves: ``There exists
    unboundedly many $\al$ such that $V_\al\models \Phi$.''
\end{theorem}
%
Since the sets $V_\al$ are well founded, we can repeat the
proof of Lemma~\ref{lem:wf-model-implies-ctm} to obtain a ctm of
$\Phi$. 

The proof of the Replacement Axiom, both in Kunen and in
other presentations \cite{neeman-course} require a relativized proof
of the Reflection Principle. 

\medskip
\fbox{Decir que Separaci\'on con 6 variables en $M$}

\fbox{implica Separaci\'on con a lo sumo 2 en $M[G]$}
\medskip


We believe that the
\isatt{ZF-Constructible} session  would benefit from some changes in
the way some results are organized accross theory files; for
intance, by cataloging all or most of the internalized formulas in one
file. Another, most basic example would be to start out with an even
more locale that only assumes $M$ to be a nonempty transitive class,
as many absoluteness results follow from these hypotheses.  
It is our desire to advocate a future work to a thorough
revision of the development of constructibility to maximize modularity.

\section*{Acknowledgment}


%%% Local Variables: 
%%% mode: latex
%%% TeX-master: "Separation_In_MG"
%%% End: 
