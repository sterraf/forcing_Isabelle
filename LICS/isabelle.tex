%%%%%%%%%%%%%%%%%%%%%%%%%%%%%%%%%%%%%%%%%%%%%%%%%%%%%%%%%%%%%%%%%%%%%%          
\section{Isabelle}

%-%-%-%-%-%-%-%-%-%-%-%-%-%-%-%-%-%-%-%-%-%-%-%-%-%-%-%-%-%-%-%-%-%-%-
\subsection{Logics}

Isabelle provides a meta-language called \emph{Pure} that consists of
a fragment of higher order logic. The meta-Boolean type is called
\isatt{prop}. Meta-connectives
\isatt{\isasymLongrightarrow} and \isatt{\&\&\&} fulfill the role of
implication and conjunction, and the meta-binder \isatt{\isasymAnd}
corresponds to universal quantification. 

On top of \emph{Pure}, theories/object logics can be defined, with
their own types, connectives and rules. Rules can be written  using
meta-implication: ``$P$, $Q$, and $R$ yield $S$'' can be written
\[
P \ \isatt{\isasymLongrightarrow}\ Q\ \isatt{\isasymLongrightarrow}\ R\ \isatt{\isasymLongrightarrow}\ S
\]
(as usual,  \isatt{\isasymLongrightarrow} associates to the right), and
syntactic sugar is provided to curry the previous rule as follows:
\[
\isasymlbrakk P; Q; R \isasymrbrakk \ \isatt{\isasymLongrightarrow}\ S.
\]
One further example is given by induction on the natural numbers
\isatt{nat},
\[
\isasymlbrakk P(0);\ (\textstyle\isasymAnd
x.\ P(x)\ \isasymLongrightarrow\ P(\isatt{succ}(x))) \isasymrbrakk
\ \isasymLongrightarrow\ P(n), 
\]
where we are omitting the ``typing'' assumtions on $n$ and $x$.

We work in the object theory \emph{Isabelle/ZF}. Two types are defined
in this theory: \tyo, the object-Booleans, and \tyi,
sets. It must be insisted that the types are defined axiomatically, not
recursively. That is, although there are constants and functions that
generate elements of both types, neither of them are 
\emph{initial}, in the sense that they are not the least types
obtained by combining the given constants and operations. This will
have concrete consequences in our strategy to approach the development.

%-%-%-%-%-%-%-%-%-%-%-%-%-%-%-%-%-%-%-%-%-%-%-%-%-%-%-%-%-%-%-%-%-%-%-
\subsection{Locales}

Locales \cite{ballarin2010tutorial} provide a neat facility to
encapsulate a context (fixed objects and assumptions on them) that is
to be used in proving several theorems, as in usual mathematical
practice. 

In this paper, locales have a further use. The \emph{Fundamental
  Theorems of Forcing} we use talk about a specific map $\forceisa$
from formulas to formulas. The definition of $\forceisa$ is involved
and we will not dwell on this now; but applications of those theorems
do not require to know how it is defined. Therefore, we black-box it
and pack everything in a locale that assumes that there is such a
map that satisfies the Fundamental Theorems.

\begin{description}
\item[\texttt{forcing\_notion}] preorden con top
\item[\texttt{countable\_generic}] lo anterior con una familia contable de densos.
\item[\texttt{M\_ZF}] axiomas.
\item[\texttt{forcing\_data}]: lo anterior contable transitivo y una notion.
\item[\texttt{forcing\_thms}]: eso.
\item[\texttt{G\_generic}]: lo anterior y G es genérico.
\item[\texttt{M\_extra\_assms}]: check in M e instancia de reemplazo para G.
\item[\texttt{G\_generic\_extra}]: los dos anteriores (no sé si sigue estando)
\end{description}


%%% Local Variables: 
%%% mode: latex
%%% TeX-master: "Separation_In_MG"
%%% ispell-local-dictionary: "american"
%%% End: 
