\section{Isabelle and (meta)theories}
\label{sec:isabelle-metatheories}

Isabelle \cite{Isabelle,DBLP:books/sp/Paulson94} is a general proof
assistant based on fragment of higher-order logic called
\emph{Pure}. The logical primitives of this meta-level are
\isatt{\isasymLongrightarrow}, \isatt{\&\&\&}, and \isatt{\isasymAnd}
(implication, conjunction, and universal
quantification, resp.), which operate on the meta-booleans
\isatt{prop}. The results presented in this work are theorems of a
version of $\ZF$ set theory (without the Axiom of Choice, $\AC$) 
called \emph{Isabelle/ZF}, which is one of the
``object logics'' that can be defined on top of Pure (which is then
used as a language to define rules). Isabelle/ZF defines types
\isatt{i} and \isatt{o} for sets and booleans, resp., and the $\ZF$
axioms are written down as terms of type \isatt{o}.

More specifically, our results work under the hypothesis of
the existence of a ctm of $\ZFC$.% 
%%
\footnote{The countability is only used to prove the existence of
  generic filters and can be thus replaced in favor of this
  hypothesis.} 
%%
This hypothesis follows, for instance, from the existence of an
inaccessible cardinal. As such, our framework is weaker than those
found usually in type theories with universes, but allows us to work
``platonistically''--- assuming we are in a universe of sets (namely,
\isatt{i}) an performing constructions there. We may state the main
theorem of our formalization:
\begin{isabelle}
\isacommand{theorem}\isamarkupfalse%
\ extensions{\isacharunderscore}of{\isacharunderscore}ctms{\isacharcolon}\isanewline
\ \ \isakeyword{assumes}\ \isanewline
\ \ \ \ {\isachardoublequoteopen}M\ {\isasymapprox}\ nat{\isachardoublequoteclose}\ {\isachardoublequoteopen}Transset{\isacharparenleft}M{\isacharparenright}{\isachardoublequoteclose}\ {\isachardoublequoteopen}M\ {\isasymTurnstile}\ ZF{\isachardoublequoteclose}\isanewline
\ \ \isakeyword{shows}\ \isanewline
\ \ \ \ {\isachardoublequoteopen}{\isasymexists}N{\isachardot}\ \isanewline
\ \ \ \ \ \ M\ {\isasymsubseteq}\ N\ {\isasymand}\ N\ {\isasymapprox}\ nat\ {\isasymand}\ Transset{\isacharparenleft}N{\isacharparenright}\ {\isasymand}\ N\ {\isasymTurnstile}\ ZF\ {\isasymand}\ M{\isasymnoteq}N\ {\isasymand}\isanewline
\ \ \ \ \ \ {\isacharparenleft}{\isasymforall}{\isasymalpha}{\isachardot}\ Ord{\isacharparenleft}{\isasymalpha}{\isacharparenright}\ {\isasymlongrightarrow}\ {\isacharparenleft}{\isasymalpha}\ {\isasymin}\ M\ {\isasymlongleftrightarrow}\ {\isasymalpha}\ {\isasymin}\ N{\isacharparenright}{\isacharparenright}\ {\isasymand}\isanewline
\ \ \ \ \ \ {\isacharparenleft}M{\isacharcomma}\ {\isacharbrackleft}{\isacharbrackright}{\isasymTurnstile}\ AC\ {\isasymlongrightarrow}\ N\ {\isasymTurnstile}\ ZFC{\isacharparenright}{\isachardoublequoteclose}
\end{isabelle}
Here, \isatt{\isasymapprox} stands for equipotence, \isatt{nat} is the
set of natural numbers, and the predicate 
\isatt{Transset} indicates transitivity.

On the downside, our approach is not able to provide us with finitary
consistency proofs. It is well known that, for example, the
implication $\Con(\ZF) \implies \Con(\ZFC+\neg\CH)$ can be proved in
\emph{primitive recursive arithmetic (PRA)}. To achieve this, however,
it would have implied to work focusing on the proof mechanisms
and distracting us from our main goal, that is, formalize the ctm
approach currently used by many mathematicians.

\begin{framed}
  Internalized formulas, MIGHT BE HERE OR IN THE NEXT SECTION.
\end{framed}

%%% Local Variables: 
%%% mode: latex
%%% TeX-master: "forcing_in_isabelle_zf"
%%% ispell-local-dictionary: "american"
%%% End: 
