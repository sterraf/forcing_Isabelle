\section{Relativization and absoluteness}

The concepts of relativization and absoluteness (due to Gödel, in his
proof of the relative consistency of $\AC$ \cite{godel-L}) 
are both
prerrequisites and powerful tools in working with transitive
models. The \emph{relativization} $\phi^C(\bx)$ of a set-theoretic
definition
$\phi$ (of relation such
as ``$x$ is a subset of $y$'' or of a function like $y=\P(x)$) to
a class $C$ is obtained by taking its (first-order) definition and
restricting all of its quantifiers to $C$.

\[
x \sbq^C y \equiv \forall z.\ C(z) \longrightarrow (z\in x
\longrightarrow z\in y)
\]

The new formula $\phi^C(\bx)$ corresponds to what is obtained by defining
the concept ``inside'' $C$. It turns out that many concepts mean the
same after relativization to a nonempty transitive class $C$; formally
\[
\forall\bx.\ C(\bx) \longrightarrow (\phi^C(\bx) \longleftrightarrow
\phi^C(\bx))
\]
When this is the case, we say that the relation defined by $\phi$ is
\emph{absolute for transitive models}. (The absoluteness of functions
moreover requires that
the relativized definition also behaves functionally over $C$.) As
examples, the relation of inclusion $\subseteq$ ---and actually, any
relation defined by a formula (equivalent to one) using only bounded
quantifiers 
$(\forall x\in y)$ and $(\exists x\in y)$--- is absolute for
transitive models. On the contrary, this is not the case with the powerset
operation.

A benefit of the work with transitive models is that [one concept of
  ordinal accross the universe / $M$ / $M[G]$]

A part of this project is to refactorize Paulson's formalization
\cite{paulson_2003} of Gödel's \cite{godel-L}. The main objective is
to maximize applicability of the relativization machinery by adjusting
the hypothesis of a greater part of it early development. Paulson's
architecture had only in mind the consistency of $\ZFC$, but, for
instance, in order to apply it in the development of forcing, too much
is assumed at the beginning; more seriously, some assumptions can't be
regarded as ``first-order'' (v.g. the Replacement Scheme).

The version of \isatt{ZF-Constructible} we present weakens the
assumptions of [many / a dozen / several] absoluteness theorems to that
of a nonempty transitive class.

%%% Local Variables: 
%%% mode: latex
%%% TeX-master: "forcing_in_isabelle_zf"
%%% ispell-local-dictionary: "american"
%%% End: 
